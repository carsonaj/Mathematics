\documentclass[12pt]{amsart}
\usepackage[margin=1in]{geometry} 
\usepackage{amsmath,amsthm,amssymb,setspace, mathtools}
\usepackage{tikz-cd} 

\usepackage{color}   %May be necessary if you want to color links
\usepackage{hyperref}
\hypersetup{
	colorlinks=true, %set true if you want colored links
	linktoc=all,     %set to all if you want both sections and subsections linked
	linkcolor=black,  %choose some color if you want links to stand out
	urlcolor=cyan
}


%
%
%
\newif\ifhideproofs
%\hideproofstrue %uncomment to hide proofs
%
%
%
%
\ifhideproofs
\usepackage{environ}
\NewEnviron{hide}{}
\let\proof\hide
\let\endproof\endhide
\fi

\theoremstyle{definition}
\newtheorem{definition}{Definition}[subsection]
\newtheorem{defn}[definition]{Definition}
\newtheorem{ax}[definition]{Axiom}
\newtheorem{note}[definition]{Note}
\newtheorem{thm}[definition]{Theorem}
\newtheorem{lem}[definition]{Lemma}
\newtheorem{prop}[definition]{Proposition}
\newtheorem{cor}[definition]{Corollary}
\newtheorem{conj}[definition]{Conjecture}
\newtheorem{ex}[definition]{Exercise}


\newcommand{\al}{\alpha}
\newcommand{\gam}{\gamma}
\newcommand{\Gam}{\Gamma}
\newcommand{\be}{\beta} 
\newcommand{\ze}{\zeta} 
\newcommand{\del}{\delta} 
\newcommand{\Del}{\Delta}
\newcommand{\lam}{\lambda}  
\newcommand{\Lam}{\Lambda} 
\newcommand{\ep}{\epsilon}
\newcommand{\sig}{\sigma} 
\newcommand{\om}{\omega}
\newcommand{\Om}{\Omega}
\newcommand{\C}{\mathbb{C}}
\newcommand{\N}{\mathbb{N}}
\newcommand{\E}{\mathbb{E}}
\newcommand{\Z}{\mathbb{Z}}
\newcommand{\R}{\mathbb{R}}
\newcommand{\T}{\mathbb{T}}
\newcommand{\Q}{\mathbb{Q}}
\renewcommand{\P}{\mathbb{P}}
\newcommand{\MA}{\mathcal{A}}
\newcommand{\MC}{\mathcal{C}}
\newcommand{\MD}{\mathcal{D}}
\newcommand{\MB}{\mathcal{B}}
\newcommand{\MF}{\mathcal{F}}
\newcommand{\MG}{\mathcal{G}}
\newcommand{\ML}{\mathcal{L}}
\newcommand{\MN}{\mathcal{N}}
\newcommand{\MS}{\mathcal{S}}
\newcommand{\MP}{\mathcal{P}}
\newcommand{\ME}{\mathcal{E}}
\newcommand{\MT}{\mathcal{T}}
\newcommand{\MM}{\mathcal{M}}
\newcommand{\MI}{\mathcal{I}}
\newcommand{\MU}{\mathcal{U}}
\newcommand{\MO}{\mathcal{O}}

\newcommand{\tbf}[1]{\textbf{#1}}

\newcommand{\ui}{[0,1]}
\newcommand{\p}{\partial}

\newcommand{\io}{\text{ i.o.}}
%\newcommand{\ev}{\text{ ev.}}
\renewcommand{\r}{\rangle}
\renewcommand{\l}{\langle}

\newcommand{\RG}{[0,\infty]}
\newcommand{\Rg}{[0,\infty)}
\newcommand{\Ru}{(\infty, \infty]}
\newcommand{\Rd}{[\infty, \infty)}
\newcommand{\Ll}{L^1_{\text{loc}}(\R^n)}

\newcommand{\limfn}{\liminf \limits_{n \rightarrow \infty}}
\newcommand{\limpn}{\limsup \limits_{n \rightarrow \infty}}
\newcommand{\limn}{\lim \limits_{n \rightarrow \infty}}
\newcommand{\convt}[1]{\xrightarrow{\text{#1}}}
\newcommand{\conv}[1]{\xrightarrow{#1}} 
\newcommand{\seq}[2]{(#1_{#2})_{#2 \in \N}}
\newcommand{\op}[1]{\mathcal{#1}^{\text{op}}}


\newcommand{\lsc}{lower semicontinuous}

\newcommand{\as}[1]{\overset{#1}{\sim}}
\newcommand{\astx}[1]{\overset{\text{#1}}{\sim}}

\DeclareMathOperator{\supp}{supp}
\DeclareMathOperator{\sgn}{sgn}
\DeclareMathOperator{\spn}{span}
\DeclareMathOperator{\iso}{Iso}
\DeclareMathOperator{\id}{id}
\DeclareMathOperator{\Aut}{Aut}
\DeclareMathOperator{\Homeo}{Homeo}
\DeclareMathOperator{\Sym}{Sym}
\DeclareMathOperator{\cl}{cl}
\DeclareMathOperator{\Int}{Int}
\DeclareMathOperator{\bal}{bal}
\DeclareMathOperator{\cnv}{conv}
\DeclareMathOperator{\epi}{epi}
\DeclareMathOperator{\dom}{dom}
\DeclareMathOperator{\cod}{cod}
\DeclareMathOperator{\Obj}{Obj}
\DeclareMathOperator{\Hom}{Hom}

\DeclareMathOperator*{\argmax}{arg\,max}
\DeclareMathOperator*{\argmin}{arg\,min}


\newcommand{\lex}[1]{\label{ex:#1}}
\newcommand{\ld}[1]{\label{defn:#1}}
\newcommand{\lax}[1]{\label{ax:#1}}
\newcommand{\rex}[1]{Exercise \ref{ex:#1}}
\newcommand{\rd}[1]{Definition \ref{defn:#1}}
\newcommand{\rax}[1]{Axiom \ref{ax:#1}}


\begin{document}
	
	\title{Introduction to Category Theory}
	\author{Carson James}
	\maketitle
	
	\tableofcontents
	
	\section*{Preface}
	\begin{flushleft}
		\href{https://creativecommons.org/licenses/by-nc-sa/4.0/legalcode.txt}{cc-by-nc-sa}
	\end{flushleft}
	
	
	
	\newpage
	
	
	\newpage
	
	\section{Categories, Functors and Natural Transformations}
	
	\subsection{von Neumann–Bernays–Gödel Set Theory}
	
	\begin{defn} \ld{11001} 
		Let $x$ be a class. Then $x$ is said to be a set iff there exists a class $A$ such that $x \in A$. 
	\end{defn}

	\begin{note}
		We can define cartesion products, relations, and functions for classes just like for sets. 
	\end{note}
	
	\begin{ax} \lax{11001.1} \textbf{Axiom of Replacement:}  \\
		Let $A$, $B$ be classes and $f:A \rightarrow B$. If $A$ is a set, then $f(A)$ is a set. 
	\end{ax}

	\begin{ax} \lax{11002} \textbf{Schema of Specification:}  \lax{11002}\\
		Let $\phi$ a propositional function on sets. Then there exists a class $A$ such that for each set $x$, $x \in A$ iff $\phi(x)$. 
	\end{ax}
	
	\begin{ex} \lex{11003}
		There exists a class $A$ such that for each class $x$, $x \in A$ iff $x$ is a set.
	\end{ex}
	
	\begin{proof}
		Define $\phi$ by $$\phi(x) : x = x$$ 
		\rax{11002} implies that there exists a class $A$ such that for each set $x$, $x \in A$ iff $x = x$. Let $x$ be a class. If $x \in A$, then by definition, $x$ is a set. \\
		Conversely, if $x$ is a set, then by construction, $x \in A$.
	\end{proof}
	
	\begin{ex} \lex{11004}
		There exists a class $A$ such that for each class $G$ and $*: G \times G \rightarrow G$, $(G, *) \in A$ iff $(G, *)$ is a group.
	\end{ex}
	
	\begin{proof}
		Define $\phi_1$, $\phi_2$ and $\phi_3$ by 
		\begin{itemize}
			\item $\phi_1(G, *): *:G \times G \rightarrow G$ is associative
			\item $\phi_2(G, *):$ there exists $e \in G$ such that for each $g \in G$, $e * g = g * e = g$
			\item $\phi_3(G, *):$ for each $g \in G$ there exists $h \in G$ such that $g * h = h * g = e$
		\end{itemize}
		Define $\phi$ by 
		$$\phi(G, *) : \phi_1(G, *) \text{ and } \phi_2(G, *) \text{ and } \phi_3(G, *)$$
		Then there exists a class $A$ such that for each set $G$ and  $*: G \times G \rightarrow G$, $(G, *) \in A$ iff $\phi(G, *)$ $(G, *) \text{ ``is a group"}$. Therefore, for each group $(G, *)$, $(G, *) \in A$.
		\textbf{FINISH!!!}
	\end{proof}

	
	
	
	
	
	
	
	
	
	
	
	
	
	
	
	
	
	
	
	
	
	
	
	
	
	
	
	
	
	
	
	
	
	
	
	\newpage
	\subsection{Categories}
	
	
	\begin{defn}  \ld{12001}
		Let $\MC_0$, $\MC_1$ be classes and $\dom, \cod : \MC_1 \rightarrow \MC_0$ class functions. Set $\MC = (C_0, C_1, \dom, \cod)$. Then $\MC$ is said to be a \textbf{category} if 
		\begin{enumerate}
			\item (composition): for each $f,g \in C_1$, if $\cod(f) = \dom(g)$, then there exists $g \circ f \in C_1$ such that $\dom(g \circ f) = \dom(f)$ and $\cod(g \circ f) = \cod(g)$
			\item (associativity): for each $f,g,h \in C_1$, if $\cod(f) = \dom(g)$ and $\cod(g) = \dom(h)$, then $$(h \circ g) \circ f = h \circ (g \circ f)$$  
			\item (identity): for each $X \in \MC_0$, there exists $1_{X} \in C_1$ such that $\dom(1_X) = \cod(1_X) = X$ and for each $f, g \in C_1$, if $\dom(f) = X$ and $\cod(g) = X$, then $$f \circ 1_X = f \text{ and } 1_X \circ g = g$$ 
		\end{enumerate}
		We define the
		\begin{itemize}
			\item \textbf{objects of $\MC$}, denoted $\Obj(\MC)$, by $\Obj(\MC) = C_0$
			\item \textbf{morphisms of $\MC$}, denoted $\Hom_{\MC}$, by $\Hom_{\MC} = C_1$
		\end{itemize}
		For $X, Y \in \Obj(\MC)$, we define the \textbf{morphisms from $X$ to $Y$}, denoted $\Hom_{\MC}(X, Y)$, by $\Hom_{\MC}(X, Y) = \{f \in \Hom(\MC): \dom(f) = X \text{ and } \cod(f) = Y\}$.
	\end{defn}

	\begin{note} 
		We typically define a category $\MC$ by specifying 
		\begin{itemize}
			\item $\Obj(\MC)$
			\item for $X,Y \in \Obj(\MC)$, the class $\Hom_{\MC}(X, Y)$
			\item for $X,Y,Z \in \Obj(\MC)$, $f \in \Hom_{\MC}(X, Y)$ and $g \in \Hom_{\MC}(Y, Z)$, the composite morphism $g \circ f \in \Hom_{\MC}(X,Z)$.
		\end{itemize}
		and then show 
		\begin{itemize}
			\item well-definedness of composition
			\item associativity of composition 
			\item existence of identities 
		\end{itemize}
	\end{note}
	
	\begin{defn} \ld{12002}
		Let $\MC$ be a category. Then $\MC$ is said to be 
		\begin{itemize}
			\item \textbf{small} if $\Obj(\MC)$ and $\Hom_{\MC}$ are sets
			\item \textbf{locally small} if for each $A,B \in \Obj(\MC)$, $\Hom_{\MC}(A,B)$ is a set 
		\end{itemize}
	\end{defn}

	\begin{ex} \lex{12003}
		Let $\MC$ be a category. If $\MC$ is small, then $\MC$ is a set. 
	\end{ex}

	\begin{proof}
		Suppose that $\MC$ is small. Then $\Obj(\MC)$ and $\Hom_{\MC}$ are sets. Then $\MP(\Obj(\MC))$, $\MP(\Hom_{\MC})$ and $\Obj(\MC)^{\Hom_{\MC}}$ are sets. Hence $\Obj(\MC) \times \Hom_{\MC} \times \Obj(\MC)^{\Hom_{\MC}} \times \Obj(\MC)^{\Hom_{\MC}}$ is a set. By definition, $\MC = (\Obj(\MC), \Hom_{\MC}, \dom, \cod) \in \Obj(\MC) \times \Hom_{\MC} \times \Obj(\MC)^{\Hom_{\MC}} \times \Obj(\MC)^{\Hom_{\MC}}$. By definition, $\MC$ is a set. 
	\end{proof}
	
	\begin{ex} \lex{12004}
		There exists a class $A$ such that $\MC \in A$ iff $\MC$ is a small category. 
	\end{ex}

	\begin{proof}
		\rex{12003} implies that for each category $\MC$, $\MC$ is small implies that $\MC$ is a set. Define $\phi$ by $$\phi(\MC): \MC \text{ is a small category} $$ 
		Then \rax{11002} implies that there exists a class $A$ such that $\MC \in A$ iff $\MC$ is a small category.  
	\end{proof}

	\begin{defn}  \ld{12004}
		Let $\MC$ be a category, we define the \text{dual of $\MC$} or the \textbf{opposite of $\MC$}, denoted $\op{C}$, by 
		\begin{itemize}
			\item $\Obj(\op{C}) = \Obj(\MC)$
			\item for $X,Y \in \Obj(\op{C})$, $\Hom_{\op{C}}(X,Y) = \Hom_{\MC}(Y,X)$
			\item for $f \in \Hom_{\op{C}}(X, Y), g \in \Hom_{\op{C}}(Y,Z)$, $g \circ_{\op{C}} f = f \circ_{\MC} g$
		\end{itemize}
	\end{defn}

	\begin{ex}  \lex{12005}
		Let $\MC$ be a category. Then $\op{C}$ is a category.
	\end{ex}
	
	\begin{proof}\
		\begin{itemize}
			\item for $W,X,Y,Z \in \Obj(\MC)$, $f \in \Hom_{\op{C}}(W, X)$ and $g \in \Hom_{\op{C}}(X, Y)$ and $h \in \Hom_{\op{C}}(Y, Z)$. Then 
			\begin{align*}
				(h \circ_{\op{C}} g) \circ_{\op{C}} f 
				&= f \circ_{\MC}( h \circ_{\op{C}} g) \\
				&= f \circ_{\MC} (g \circ_{\MC} h) \\
				&= (f \circ_{\MC} g) \circ_{\MC} h \\
				&= h \circ_{\op{C}} (f \circ_{\MC} g) \\
				&= h \circ_{\op{C}} (g \circ_{\op{C}} f)
			\end{align*}
			So composition is associative.
			\item Let $X \in \Obj(\MC)$ and $f,g \in \Hom_{\op{C}}$. Suppose that $\dom(f) = X$ and $\cod(g) = X$ 
			Then 
			\begin{align*}
				f  \circ_{\op{C}} 1_X
				&= 1_X  \circ_{\MC} f \\
				&= f
			\end{align*}
			and 
			\begin{align*}
				1_X  \circ_{\op{C}} g 
				&= g \circ_{\MC} 1_X \\
				&= g
			\end{align*}
		So $(1_X)_{\op{C}} =  (1_X)_{\MC}$.
		\end{itemize}
	\end{proof}
	
	\begin{defn} \ld{12006}
		Let $\MC$ be a category and $X \in \Obj(\MC)$. We define the \textbf{slice category of $\MC$ over $X$}, denoted $\MC / X$, by
		\begin{itemize}
			\item $\Obj(\MC / X) = \{f \in \Hom_{\MC}: \cod(f) = X\}$
			\item for $f,g \in \Obj(\MC / X)$, 
			$$\Hom_{\MC / X}(f, g) = \{\al \in \Hom_{\MC} : \dom(\al) = \dom(f) \text{, } \cod(\al) = \dom(g) \text{ and } f = g \circ \al \}$$
			i.e. for $f \in \Hom_{\MC}(A, X)$ and $g \in \Hom_{\MC}(B, X)$, $\al \in \Hom_{\MC / X}(f, g)$ iff the following diagram commutes: 
			\[ \begin{tikzcd}
				A \arrow[rr, "\al"] \arrow[dr, "f"'] 	
				& & B  \arrow[dl, "g"] \\
				& X 
			\end{tikzcd}
			\]
			\item for $f,g, h \in \Obj(\MC / X)$, $\al \in \Hom_{\MC / X}(f, g)$ and $\be \in \Hom_{\MC / X}(g, h)$, 
			$$\be \circ_{\MC / X} \al = \be \circ_{\MC} \al$$
		\end{itemize}
	\end{defn}
	
	\begin{ex}  \lex{12007}
		Let $\MC$ be a category and $X \in \Obj(\MC)$. Then $\MC / X$ is a category.
	\end{ex}

	\begin{proof}\
		\begin{itemize}
			\item $f,g, h \in \Obj(\MC / X)$, $\al \in \Hom_{\MC / X}(f, g)$ and $\be \in \Hom_{\MC / X}(g, h)$. Then $f = g \circ_{\MC} \al$ and $g = h \circ_{\MC} \be$, i.e. the following diagrams commute:
			\[ \begin{tikzcd}
				\dom(f) \arrow[rr, "\al"] \arrow[dr, "f"'] 	
				&& \dom(g)  \arrow[dl, "g"]  
				&& \dom(g) \arrow[rr, "\be"] \arrow[dr, "g"'] 	
				&& \dom(h)  \arrow[dl, "h"]\\
				& X 
				&& &&X
			\end{tikzcd}
			\]
			Therefore, we have that 
			\begin{align*}
				f 
				& = g \circ_{\MC} \al \\
				& = (h \circ_{\MC} \be) \circ_{\MC} \al \\
				&= h \circ_{\MC} (\be \circ_{\MC} \al) 
			\end{align*}
			i.e. the following diagram commutes:
			\[ \begin{tikzcd}
				\dom(f) \arrow[rr, "\be \circ_{\MC} \al"] \arrow[dr, "f"'] 	
				& & \dom(h)  \arrow[dl, "g"] \\
				& X 
			\end{tikzcd}
			\]
			which implies that 
			\begin{align*}
				\be \circ_{\MC / X} \al 
				& = \be \circ_{\MC} \al \\
				& \in \Hom_{\MC / X}(f, h)
			\end{align*}
			and composition is well defined. 
			\item Associativity of $\circ_{\MC / X}$ follows from associativity of $\circ_\MC$.
			\item Let $f \in \Obj(\MC / X)$ and $\al, \be \in \Hom_{\MC / X}$. Since $f \circ 1_{\dom_{\MC}(f)} = f$, i.e. the following diagram commutes: 
			\[ \begin{tikzcd}
				\dom(f) \arrow[rr, "1_{\dom(f)}"] \arrow[dr, "f"'] 	
				& & \dom(f)  \arrow[dl, "f"] \\
				& X 
			\end{tikzcd}
			\]
			we have that $1_{\dom(f)} \in \Hom_{\MC / X}(f, f)$.
			Suppose that $\dom_{\MC / X}(\al) = f$ and $\cod_{\MC /X}(\be) = f$. Then 
			\begin{align*}
				\al \circ_{\MC /X} 1_{\dom(f)}
				&= \al \circ_{C} 1_{\dom(f)} \\
				&= \al
			\end{align*}
			and 
			\begin{align*}
				1_{\dom(f)} \circ_{\MC /X} \be
				&= 1_{\dom(f)} \circ_{\MC} \be \\
				&= \be
			\end{align*}
			So $(1_f)_{\MC / X} = (1_{\dom(f)})_{\MC}$.
		\end{itemize}
	\end{proof}
	
	
	
	
	
	
	
	
	
	
	
	
	
	
	
	
	
	
	
	
	
	
	
	
	
	
	
	
	
	
	
	\newpage
	\subsection{Functors}
	
	\begin{defn} \ld{13001}
		Let $\MC$ and $\MD$ be categories and $F_0 : \Obj(\MC) \rightarrow \Obj(\MD)$, $F_1: \Hom_{\MC} \rightarrow \Hom_{\MD}$ class functions. Set $F = (F_0, F_1)$. Then $F$ is said to be a \text{functor from $\MC$ to $\MD$}, denoted  $F:\MC \rightarrow \MD$, if 
		\begin{enumerate}
			\item for each $A,B \in \Obj(\MC)$ and $f \in \Hom_{\MC}(A, B)$, $F_1(f) \in \Hom_{\MD}(F_0(A), F_0(B))$
			\item for each $A,B, C \in \Obj(\MC)$, $f \in \Hom_{\MC}(A,B)$ and $g \in \Hom_{\MC}(B, C)$, $F_1(g \circ f) = F_1(g) \circ F_1(f)$
			\item for each $A \in \Obj(\MC)$,  $F_1(\id_A) = \id_{F_0(A)}$
		\end{enumerate}
	\end{defn}

	\begin{note}
		For $A \in \Obj(C)$ and $f \in \Hom_{\MC}$, we typically write $F(A)$ and $F(f)$ instead of $F_0(A)$ and $F_1(f)$ respectively.
	\end{note}
	
	\begin{defn} \ld{13006}
		Let $\MC$, $\MD$ and $\ME$ be categories and $F:\MC \rightarrow \MD$, $G: \MD \rightarrow \ME$ functors. We define the \textbf{composition of $G$ with $F$}, denoted $G \circ F: \MC \rightarrow \ME$, by 
		\begin{itemize}
			\item $G \circ F (A) = G(F(A))$
			\item $G \circ F (f) = G(F(f))$
		\end{itemize} 
	\end{defn}
	
	\begin{ex}  \lex{13007}
		Let $\MC$, $\MD$ and $\ME$ be categories and $F:\MC \rightarrow \MD$, $G: \MD \rightarrow \ME$ functors. Then  $G \circ F: \MC \rightarrow \ME$ is a functor.
	\end{ex}

	\begin{proof}\
		\begin{enumerate}
				\item Let $A, B \in \Obj(\MC)$ and $f \in \Hom_{\MC}(A,B)$. Since $F(f) \in \Hom_{\MD}(F(A), F(B))$, we have that $G(F(f)) \in \Hom_{\ME}(G(F(A)), G(F(B)))$. Then 
			\begin{align*}
				G \circ F (f) 
				& = G(F(f)) \\
				& \in \Hom_{\ME}(G(F(A)), G(F(B))) \\
				& =  \Hom_{\ME}(G \circ F(A), G \circ F (B)) \\
			\end{align*}
			\item Let $A,B, C \in \Obj(\MC)$, $f \in \Hom_{\MC}(A,B)$ and $g \in \Hom_{\MC}(B, C)$. Then 
			\begin{align*}
				G \circ F(g \circ f) 
				& = G (F (g \circ f)) \\
				& = G(F(g) \circ F(f)) \\
				& = G(F(g)) \circ G(F(f)) \\
				& = G \circ F (g) \circ G \circ F (f) \\
			\end{align*}
			\item Let $A \in \Obj(\MC)$. Then 
			\begin{align*}
				G \circ F(\id_{A})
				& = G(F(\id_A) )\\
				& = G(\id_{F(A)}) \\
				& = \id_{G(F(A))} \\
				& = \id_{G \circ F(A)} 
			\end{align*}
		\end{enumerate}
		So $G \circ F: \MC \rightarrow \ME$ is a functor. 
	\end{proof}

	\begin{ex}  \lex{13008}
		Let $\MC$, $\MD$, $\ME$, $\MF$ be categories and $F:\MC \rightarrow \MD$, $G : \MD \rightarrow \ME$, $H: \ME \rightarrow \MF$ functors. Then $(H \circ G) \circ F = H \circ (G \circ F)$.
	\end{ex}

	\begin{proof}
		Let $A, B \in \Obj(\MC)$ and $f \in \Hom_{\MC}(A,B)$. Then 
		\begin{itemize}
				\item 
				\begin{align*}
					(H \circ G) \circ F(A) 
					& = H \circ G (F(A)) \\
					& = H (G(F(A))) \\
					& = H( G \circ F (A)) \\
					& = H \circ (G \circ F) (A)
				\end{align*}
				\item  
				\begin{align*}
					(H \circ G) \circ F(f) 
					& = H \circ G (F(f)) \\
					& = H (G(F(f))) \\
					& = H( G \circ F (f)) \\
					& = H \circ (G \circ F) (f)
				\end{align*}
		\end{itemize}
		Hence $(H \circ G) \circ F = H \circ (G \circ F)$.
	\end{proof}

	\begin{defn} \ld{13004}
		Let $\MC$ be a category. We define the \textbf{identity functor from $\MC$ to $\MC$}, denoted $\id_{\MC}: \MC \rightarrow \MC$, by 
		\begin{itemize}
			\item $\id_{\MC}(A) = A$, $(A \in \Obj(\MC))$
			\item $\id_{\MC}(f) = f$, $(f \in \Hom_{\MC})$
		\end{itemize}
	\end{defn}
	
	\begin{ex}  \lex{13005}
		Let $\MC$ be a category. Then $\id_{\MC}: \MC \rightarrow \MC$ is a functor.
	\end{ex}
	
	\begin{proof}\
		\begin{enumerate}
			\item Let $A, B \in \Obj(\MC)$ and $f \in \Hom_{\MC}(A,B)$. Then 
			\begin{align*}
				\id_{\MC}(f) 
				& = f \\
				& \in \Hom_{\MC}(A, B) \\
				& = \Hom_{\MC}(\id_{\MC}(A), \id_{\MC}(B)) \\
			\end{align*}
			\item Let $A,B, C \in \Obj(\MC)$, $f \in \Hom_{\MC}(A,B)$ and $g \in \Hom_{\MC}(B, C)$. Then 
			\begin{align*}
				\id_{\MC}(g \circ f) 
				& = g \circ f \\
				& = \id_{\MC}(g) \circ \id_{\MC} (f)
			\end{align*}
			\item Let $A \in \Obj(\MC)$. Then 
			\begin{align*}
				\id_{\MC}(\id_{A})
				& = \id_A \\
				& = \id_{\id_{\MC}(A)}
			\end{align*}
		\end{enumerate}
	\end{proof}

	\begin{ex} \lex{13006}
		Let $\MC$ and $\MD$ be categories and $F:\MC \rightarrow \MD$. Then 
		\begin{enumerate}
			\item $\id_{\MD} \circ F = F$
			\item $F \circ \id_{\MC} = F$
		\end{enumerate}
	\end{ex}

	\begin{proof}\
		\begin{enumerate}
			\item Let $A, B \in \Obj(\MC)$ and $f \in \Hom_{\MC}(A, B)$. Then 
			\begin{align*}
				\id_{\MD} \circ F(A) 
				& = \id_{\MD}(F(A)) \\
				& = F(A)
			\end{align*}
			and 
			\begin{align*}
				\id_{\MD} \circ F(f) 
				& = \id_{\MD}(F(f)) \\
				& = F(f)
			\end{align*} 
			Since $A, B \in \Obj(\MC)$ and $f \in \Hom_{\MC}(A, B)$ are arbitrary, $\id_{\MD} \circ F = F$.
			\item Let $A, B \in \Obj(\MC)$ and $f \in \Hom_{\MC}(A, B)$. Then
			\begin{align*}
				F \circ \id_{\MC}(A) 
				& = F(\id_{\MC}(A)) \\
				& = F(A) 
			\end{align*}
			and 
			\begin{align*}
				F \circ \id_{\MC}(f) 
				& = F (\id_{\MC}(f)) \\
				& = F(f)
			\end{align*} 
			Since $A, B \in \Obj(\MC)$ and $f \in \Hom_{\MC}(A, B)$ are arbitrary, $ F \circ \id_{\MC} = F$.
		\end{enumerate}
	\end{proof}
	
	\begin{ex} \lex{13008.1}
		Let $\MC$ and $\MD$ be categories and $F:\MC \rightarrow \MD$. If $\MC$ is small, then $F$ is a set.
	\end{ex}
	
	\begin{proof}
		Suppose that $\MC$ is small. Then $\Obj(\MC)$ and $\Hom_{\MC}$ are sets. By definition, there exist $F_0: \Obj(\MC) \rightarrow \Obj(\MD)$ and $F_1: \Hom_{\MC} \rightarrow \Hom_{\MD}$ such that $F = (F_0, F_1)$. \rax{11001.1} implies that $F_0(\Obj(\MC))$ and $F_1(\Hom_{\MC})$ are sets. Therefore, $\Obj(\MC) \times F_0(\Obj(C))$ and $\Hom_{\MC} \times F_1(\Hom_{\MC})$ are sets. Hence $\MP(\Obj(\MC) \times F_0(\Obj(C)))$ and $\MP(\Hom_{\MC} \times F_1(\Hom_{\MC}))$ are sets. Since $F_0 \subset \Obj(\MC) \times F_0(\Obj(C))$ and $F_1 \subset \Hom_{\MC} \times F_1(\Hom_{\MC})$, we have that $F_0 \in \MP(\Obj(\MC) \times F_0(\Obj(C)))$ and $F_1 \in \MP(\Hom_{\MC} \times F_1(\Hom_{\MC}))$. Hence $F_0$ and $F_1$ are sets. Thus $F = (F_0, F_1)$ is a set. 
	\end{proof}

	\begin{ex} \lex{13008.2}
		Let $\MC$ and $\MD$ be categories. Suppose that $\MC$ is small. Then there exists a class $A$ such that for each class $F$, $F \in A$ iff $F: \MC \rightarrow \MD$.   
	\end{ex}
	
	\begin{proof} 
		Let $\MC$ and $\MD$ be categories. Suppose that $\MC$ is small. Define $\phi$ by 
		$$\phi(F) : F : \MC \rightarrow \MD$$ 
		Then there exists a class $A$ such that for each set $F$, $F \in A$ iff $\phi(F)$. Let $F$ be a class. Suppose that $F \in A$. By \rd{11001}, $F$ is a set. Since $F$ is a set and $F \in A$, we have that $\phi(F)$. Hence $F: \MC \rightarrow \MD$. \\
		Conversely, suppose that $F: \MC \rightarrow \MD$. \rex{13008.1} implies that $F$ is a set. Since $F$ is a set and $\phi(F)$ is true, we have that $F \in A$. 
	\end{proof}
	
	\begin{defn} \ld{13009}
		We define $\tbf{Cat}$ by 
		\begin{itemize}
			\item $\Obj(\tbf{Cat}) = \{\MC: \MC \text{ is a small category}\}$.
			\item for $\MC,\MD \in \Obj(\tbf{Cat})$, 
			$$\Hom_{\tbf{Cat}}(\MC,\MD) = \{F : F: \MC \rightarrow \MD \}$$
			\item for $\MC,\MD, \ME \in \Obj(\tbf{Cat})$, $F \in \Hom_{\tbf{Cat}}(\MC,\MD)$ and $G \in \Hom_{\tbf{Cat}}(\MD, \ME)$, $$G \circ_{\tbf{Cat}} F = G \circ F$$
		\end{itemize}
	\end{defn}


	\begin{ex}  \lex{13010}
		We have that \tbf{Cat} is 
		\begin{enumerate}
			\item a category
			\item locally small
		\end{enumerate} 
	\end{ex}

	\begin{proof}\
		\begin{enumerate}
			\item The previous exercises imply the associativity of composition and the existence of identities.
			\item Let $\MC, \MD \in \Obj(\tbf{Cat})$ and $F \in \Hom_{\tbf{Cat}}(\MC, \MD)$. \rd{12002} implies that $\Obj(\MC)$, $\Obj(\MD)$, $\Hom_{\MC}$ and $\Hom_{\MD}$ are sets. Then $\Obj(\MD)^{\Obj(\MC)}$ and $\Hom_{\MD}^{ \Hom_{\MC}}$ are sets. Hence $ \Obj(\MD)^{\Obj(\MC)} \times  \Hom_{\MD}^{ \Hom_{\MC}}$ is a set. Let $F \in \Hom_{\tbf{Cat}}(\MC, \MD)$. Then there exist $F_0 \in \Obj(\MD)^{\Obj(\MC)} $ and $F_1 \in \Hom_{\MD}^{ \Hom_{\MC}}$ such that $F = (F_0, F_1)$. Therefore $F \in  \Obj(\MD)^{\Obj(\MC)} \times  \Hom_{\MD}^{ \Hom_{\MC}}$. Since $F \in \Hom_{\tbf{Cat}}(\MC, \MD)$ is arbitrary, 
			\begin{align*}
				\Hom_{\tbf{Cat}}(\MC, \MD) 
				& \subset \Obj(\MD)^{\Obj(\MC)} \times  \Hom_{\MD}^{ \Hom_{\MC}}
			\end{align*}
			which implies that $\Hom_{\tbf{Cat}}(\MC, \MD)$ is a set. Therefore, $\tbf{Cat}$ is locally small.
		\end{enumerate} 
	\end{proof}

	
	
	
	
	
	
	
	
	
	
	
	
	
	
	
	
	
	
	
	
	
	
	
	
	
	
	
	
	
	
	\newpage
	\subsection{Natural Transformations}
	
	\begin{defn} \ld{14001}
		Let $\MC$ and $\MD$ be categories, $F, G: \MC \rightarrow \MD$ and $ \al : \Obj(\MC) \rightarrow  \Hom_{\MD}$. Then $\al$ is said to be a \textbf{natural transformation from $F$ to $G$}, denoted $\al: F \Rightarrow G$, if
		\begin{enumerate}
			\item for each $A \in \Obj(\MC)$, $\al_A \in \Hom_{\MD}(F(A), G(A))$
			\item for each $A, B \in \Obj(\MC)$ and $f \in \Hom_{\MC}(A,B)$, $G(f) \circ \al_A = \al_B \circ F(f)$, i.e. the following diagram commutes: 
			\[ 
			\begin{tikzcd}
				F(A)  \arrow[r, "\al_A"]  \arrow[d, "F(f)"']  & G(A)   \arrow[d, "G(f)"]\\
				F(B) \arrow[r, "\al_B"] &  G(B) \\
			\end{tikzcd}
			\]
		\end{enumerate}
	\end{defn}


	\begin{defn} \ld{14002}
		Let $\MC$, $\MD$ be categories, $F, G, H:\MC \rightarrow \MD$ functors and $\al: F \Rightarrow G$, $\be : G \Rightarrow H$ natural transformations. We define the \textbf{composition of $\be$ with $\al$}, denoted $\be \circ \al: F \Rightarrow H$, by 
		$$(\be \circ \al)_{A} = \be_A \circ \al_A$$ 
	\end{defn}
	
	\begin{ex}  \lex{14003}
		Let $\MC$, $\MD$ be categories, $F, G, H:\MC \rightarrow \MD$ functors and $\al: F \Rightarrow G$, $\be : G \Rightarrow H$ natural transformations. Then $ \be \circ \al: F \Rightarrow H$ is a natural transformation.
	\end{ex}
	
	\begin{proof}\
		\begin{enumerate}
			\item Let $A \in \Obj(\MC)$. Since $\al_A \in \Hom_{\MD}(F(A), G(A))$ and $\be_A \in \Hom_{\MD}(G(A), H(A))$, we have that 
			\begin{align*}
				(\be \circ \al)_A 
				& = \be_A \circ \al_A \\
				& \in \Hom_{\MD}(F(A), H(A))
			\end{align*}
			\item Let $A,B \in \Obj(\MC)$ and $f \in \Hom_{\MC}(A,B)$. Since $\al:F \Rightarrow G$ and $\be: G \Rightarrow H$, $G(f) \circ \al_A = \al_B \circ F(f)$ and $H(f) \circ \be_A = \be_B \circ G(f)$. Therefore 
			\begin{align*}
				H(f) \circ (\be \circ \al)_A
				& = H(f) \circ (\be_A \circ \al_A) \\
				& = (H(f) \circ \be_A) \circ \al_A \\
				& = (\be_B \circ G(f)) \circ \al_A \\
				& = \be_B \circ (G(f)\circ \al_A) \\
				& = \be_B \circ (\al_B \circ F(f)) \\
				& = (\be_B \circ \al_B) \circ F(f) \\
				& = (\be \circ \al)_B \circ F(f)
			\end{align*}
		\end{enumerate}
		So $\be \circ \al: F \Rightarrow H$ is a natural transformation. 
	\end{proof}

	\begin{ex} \lex{14003.1}
		Let $\MC$, $\MD$ be categories, $F, G, H, I:\MC \rightarrow \MD$ functors and $\al: F \Rightarrow G$, $\be : G \Rightarrow H$ and $\gam: H \Rightarrow I$ natural transformations. Then $$(\gam \circ \be) \circ \al = \gam \circ (\be \circ \al)$$
	\end{ex}

	\begin{proof}
		Let $A \in \Obj(\MC)$. By definition,  
		\begin{align*}
			[(\gam \circ \be) \circ \al]_{A} 
			& = (\gam \circ \be)_A \circ \al_{A} \\
			& = (\gam_A \circ \be_A) \circ \al_A \\
			& = \gam_A \circ (\be_A \circ \al_A) \\
			& = \gam_A \circ (\be \circ \al)_A \\
			& = [\gam \circ (\be \circ \al)]_A \\
		\end{align*}
		Since $A \in \Obj(\MC)$ is arbitrary, 
		$$(\gam \circ \be) \circ \al = \gam \circ (\be \circ \al)$$
	\end{proof}

	\begin{defn} \ld{14003.2}
		Let $\MC$, $\MD$ be categories and $F: \MC \rightarrow \MD$. We define the \textbf{identity natural transformation from $F$ to $F$}, denoted $\id_{F}: F \Rightarrow F$, by 
		$$(\id_F)_A = \id_{F(A)}$$
	\end{defn}
	
	\begin{ex}  \lex{14003.3}
		Let $\MC$, $\MD$ be categories and $F: \MC \rightarrow \MD$. Then $\id_F: F \Rightarrow F$ is a natural transformation from $F$ to $F$.
	\end{ex}
	
	\begin{proof}\
		\begin{enumerate}
			\item Let $A \in \Obj(\MC)$. Then 
			\begin{align*}
				(\id_F)_A 
				& = \id_{F(A)} \\
				& \in \Hom_{\MD}(F(A), F(A)) \\
				& = \Hom_{\MC}(\id_{\MC}(A), \id_{\MC}(B)) \\
			\end{align*}
			\item Let $A,B \in \Obj(\MC)$, $f \in \Hom_{\MC}(A,B)$. Then 
			\begin{align*}
				F(f) \circ (\id_{F})_{A} 
				& = F(f) \circ \id_{F(A)} \\
				& = F(f) \\
				& = \id_{F(B)} \circ F(f) \\
				& = (\id_{F})_{B} \circ F(f)
			\end{align*}
		\end{enumerate}
	\end{proof}

	\begin{ex} \lex{14003.4}
		Let $\MC$, $\MD$ be categories, $F,G: \MC \rightarrow \MD$ and $\al : F \Rightarrow G$. Then 
		\begin{enumerate}
			\item $\id_{G} \circ \al = \al$
			\item $\al \circ \id_F = \al$
		\end{enumerate} 
	\end{ex}

	\begin{proof}\
		\begin{enumerate}
			\item Let $A \in \Obj(\MC)$. Then 
			\begin{align*}
				(\id_{G} \circ \al)_{A}
				& = (\id_G)_A \circ \al_A \\
				& = \id_{G(A)} \circ \al_A \\
				& = \al_A 
			\end{align*}
			Since $A \in \Obj(C)$ is arbitrary, $\id_{G} \circ \al = \al$
			\item Let $A \in \Obj(\MC)$. Then
			\begin{align*}
				(\al \circ \id_F)_{A}
				& = \al_A \circ (\id_F)_A \\
				& = \al_A \circ \id_{F(A)} \\
				& = \al_A 
			\end{align*}
			Since $A \in \Obj(C)$ is arbitrary, $\al \circ \id_F = \al$.
		\end{enumerate}
	\end{proof}

	\begin{ex} \lex{14004}
		Let $\MC$ and $\MD$ be categories, $F,G: \MC \rightarrow \MD$ and $\al: F \Rightarrow G$. If $\MC$ is small, then $\al$ is a set.
	\end{ex}

	\begin{proof}
		Suppose that $\MC$ is small. Then $\Obj(\MC)$ is a set. Since $\al: \Obj(\MC) \rightarrow \Hom_{\MD}$, \rax{11001.1} implies that $\al(\Obj(\MC))$ is a set. Then $ \Obj(\MC) \times \al(\Obj(\MC))$ is a set. Therefore $\MP(\Obj(\MC) \times \al(\Obj(\MC)))$ is a set. Since $\al \subset \Obj(\MC) \times \al(\Obj(\MC))$, we have that $\al \in \MP(\Obj(\MC) \times \al(\Obj(\MC)))$ which implies that $\al$ is a set. 
	\end{proof}

	\begin{ex} \lex{14005}
			Let $\MC$ and $\MD$ be categories and $F,G: \MC \rightarrow \MD$. If $\MC$ is small, then there exists a class $A$ such that for each class $\al$, $\al \in A$ iff $\al: F \Rightarrow G$. 
	\end{ex}

	\begin{proof}
		Suppose that $\MC$ is small. Define $\phi$ by 
		$$\phi(\al): \al: F \Rightarrow G$$
		\rax{11002} implies that there exists a class $A$ such that for each set $\al$, $\al \in A$ iff $\phi(\al)$. 
		Let $\al$ be a class. Suppose that $\al \in A$. By \rd{11001}, $\al$ is a set. Since $\al$ is a set and $\al \in A$, we have that $\phi(\al)$. Hence $\al:F \Rightarrow G$. \\
		Conversely, suppose that $\al: F \Rightarrow G$.  Since $\MC$ is small, \rex{14004} implies that $\al$ is a set. Since $\phi(\al)$, we have that $\al \in A$. 
	\end{proof}
	
	\begin{defn} \ld{14006}
		Let $\MC$ and $\MD$ be categories. Suppose that $\MC$ is small. We define the \textbf{functor category from $\MC$ to $\MD$}, denoted $\MD^{\MC}$, by 
		\begin{itemize}
			\item $\Obj(\MD^{\MC}) = \{F: F: \MC \rightarrow \MD\}$
			\item For $F, G \in \Obj(\MD^{\MC})$, $\Hom_{\MD^{\MC}}(F, G) = \{\al: \al: F \Rightarrow G \}$
			\item For $F, G, H \in \Obj(\MD^{\MC})$, $\al \in \Hom_{\MD^{\MC}}(F, G)$ and $\be \in \Hom_{\MD^{\MC}}(G, H)$, 
			$\be \circ_{\MD^{\MC}} \al = \be \circ \al $
		\end{itemize} 
	\end{defn}

	\begin{ex} \lex{14007}
		Let $\MC$ and $\MD$ be categories. Suppose that $\MC$ is small. Then $\MD^{\MC}$ is a category.
	\end{ex}

	\begin{proof}\
		The previous exercises imply the associativity of composition and existence of identities.
	\end{proof}















































	\newpage
	\subsection{Product Categories}
	
	\begin{defn}
		Let $\MC$ and $\MD$ be categories. We define the \textbf{product category of $\MC$ and $\MD$}, denoted $\MC \times \MD$ by 
		\begin{itemize}
			\item $\Obj(\MC \times \MD) = \{(A, B): A \in \Obj(\MC) \text{ and } B \in \Obj(\MD)\}$
			\item for each $(A, A'), (B, B') \in \Obj(\MC \times \MD)$, $\Hom_{\MC \times \MD}((A, A'), (B, B')) = \{(f,g): f \in \Hom_{\MC}(A, B) \text{ and }  g \in \Hom_{\MC}(A', B') \}$
			\item for each  $(A, A'), (B, B'), (C, C') \in \Obj(\MC \times \MD)$, $(f, f') \in \Hom_{\MC \times \MD}((A, A'), (B, B'))$ and $(g, g') \in \Hom_{\MC \times \MD}((B, B'), (C, C'))$, 
			$$(g,g') \circ_{\MC \times \MD} (f, f') = (g \circ_{\MC} f, g' \circ_{\MD} f')$$
		\end{itemize}
	\end{defn}

	\begin{ex}
		Let $\MC$ and $\MD$ be categories. Then $\MC \times \MD$ is a category. 
	\end{ex}

	\begin{proof}\
		\begin{itemize}
			\item Let $(A, A'), (B, B'), (C, C') \in \Obj(\MC \times \MD)$, $(f, f') \in \Hom_{\MC \times \MD}((A, A'), (B, B'))$ and $(g, g') \in \Hom_{\MC \times \MD}((B, B'), (C, C'))$. Then $f \in \Hom_{\MC}(A, B)$, $g \in \Hom_{\MC}(B, C)$, $f'  \in \Hom_{\MD}(A', B')$, and  $g' \in \Hom_{\MD}(B', C')$. Hence $g \circ_{\MC} f \in \Hom_{\MC}(A, C)$ and $g' \circ_{\MD} f' \in \Hom_{\MD}(A', C')$. Thus 
			\begin{align*}
				(g,g') \circ_{\MC \times \MD} (f, f') 
				&= (g \circ_{\MC} f, g' \circ_{\MD} f') \\
				& \in \Hom_{\MC \times \MD}((A, A'), (C, C'))
			\end{align*}
			Thus, composition is well defined. \\
			\item Let $(A, A'), (B, B'), (C, C'), (D, D') \in \Obj(\MC \times \MD)$, $(f, f') \in \Hom_{\MC \times \MD}((A, A'), (B, B'))$, $(g, g') \in \Hom_{\MC \times \MD}((B, B'), (C, C'))$ and $(h, h') \in \Hom_{\MC \times \MD}((C, C'), (D, D'))$. Then 
			\begin{align*}
				[(h, h') \circ_{\MC \times \MD} (g , g') ] \circ_{\MC \times \MD} (f, f')
				& = (h \circ_{\MC} g, h' \circ_{\MD} g') \circ_{\MC \times \MD} (f, f') \\
				& = ((h \circ_{\MC} g) \circ_{\MC} f, (h' \circ_{\MD} g') \circ_{\MD} f') \\
				& = (h \circ_{\MC} ( g \circ_{\MC} f), h' \circ_{\MD} (g' \circ_{\MD} f')) \\
				& = (h, h') \circ_{\MC \times \MD} (g \circ_{\MC} f, g' \circ_{\MD} f') \\
				& = (h, h') \circ_{\MC \times \MD} [(g, g') \circ_{\MC \times \MD} (f, f')]
			\end{align*} 
			Thus composition is associative. \\
			\item Let $(A, B) \in \Obj(\MC \times \MD)$, $(f, f'), (g,g') \in \Hom_{\MC \times \MD}$. Suppose that $\dom_{\MC \times \MD}(f, f') = (A, B)$ and $\cod_{\MC \times \MD}(g, g') = (A, B)$. Then $\dom_{\MC}(f) = A$, $\dom_{\MD}(f') = B$, $\cod_{\MC}(g) = A$ and  $\cod_{\MD}(g') = B$. Hence
			\begin{align*}
				(f, f') \circ_{\MC \times \MD} (1_A, 1_{B}) 
				& = (f \circ_{\MC} 1_A, f' \circ_{\MD} 1_{B}) \\
				& = (f, f)
			\end{align*}
			and
			\begin{align*}
				(1_A, 1_{B})  \circ_{\MC \times \MD} (g, g')
				& = (1_A \circ_{\MC} g , 1_{B} \circ g') \\
				& = (g, g')
			\end{align*}
			Therefore $(1_{(A, B)})_{\MC \times \MD} = (1_A, 1_{B})$.
		\end{itemize}
	\end{proof}
	
\end{document}