%% filename: amsbook-template.tex
%% version: 1.1
%% date: 2014/07/24
%%
%% American Mathematical Society
%% Technical Support
%% Publications Technical Group
%% 201 Charles Street
%% Providence, RI 02904
%% USA
%% tel: (401) 455-4080
%%      (800) 321-4267 (USA and Canada only)
%% fax: (401) 331-3842
%% email: tech-support@ams.org
%% 
%% Copyright 2006, 2008-2010, 2014 American Mathematical Society.
%% 
%% This work may be distributed and/or modified under the
%% conditions of the LaTeX Project Public License, either version 1.3c
%% of this license or (at your option) any later version.
%% The latest version of this license is in
%%   http://www.latex-project.org/lppl.txt
%% and version 1.3c or later is part of all distributions of LaTeX
%% version 2005/12/01 or later.
%% 
%% This work has the LPPL maintenance status `maintained'.
%% 
%% The Current Maintainer of this work is the American Mathematical
%% Society.
%%
%% ====================================================================

%    AMS-LaTeX v.2 driver file template for use with amsbook
%
%    Remove any commented or uncommented macros you do not use.

\documentclass{book}

%    For use when working on individual chapters
%\includeonly{}

%    For use when working on individual chapters
%\includeonly{}

%    Include referenced packages here.
\usepackage[left =.5in, right = .5in, top = 1in, bottom = 1in]{geometry} 
\usepackage{amsmath}
\usepackage{amsthm}
\usepackage{amssymb}
\usepackage{setspace}
\usepackage{mathtools}
\usepackage{tikz}  
\usepackage{tikz-cd}
\usepackage{tkz-fct}
\usepackage{pgfplots}
\usepackage{environ}
\usepackage{tikz-cd} 
\usepackage{enumitem}
\usepackage{color}   %May be necessary if you want to color links
%\usepackage{xr}

\usepackage{hyperref}
\hypersetup{
	colorlinks=true, %set true if you want colored links
	linktoc=all,     %set to all if you want both sections and subsections linked
	linkcolor=black,  %choose some color if you want links to stand out
	urlcolor=cyan
}
\usepackage[symbols,nogroupskip,sort=none]{glossaries-extra}

\pgfplotsset{every axis/.append style={
		axis x line=middle,    % put the x axis in the middle
		axis y line=middle,    % put the y axis in the middle
		axis line style={<->,color=black}, % arrows on the axis
		xlabel={$x$},          % default put x on x-axis
		ylabel={$y$},          % default put y on y-axis
}}


\theoremstyle{definition}
\newtheorem{definition}{Definition}[subsection]
\newtheorem{defn}[definition]{Definition}
\newtheorem{note}[definition]{Note}
\newtheorem{ax}[definition]{Axiom}
\newtheorem{thm}[definition]{Theorem}
\newtheorem{lem}[definition]{Lemma}
\newtheorem{prop}[definition]{Proposition}
\newtheorem{cor}[definition]{Corollary}
\newtheorem{conj}[definition]{Conjecture}
\newtheorem{ex}[definition]{Exercise}
\newtheorem{exmp}[definition]{Example}
\newtheorem{soln}[definition]{Solution}

\setcounter{tocdepth}{3}

% hide proofs
\newif\ifhideproofs
%\hideproofstrue %uncomment to hide proofs
\ifhideproofs
\NewEnviron{hide}{}
\let\proof\hide
\let\endproof\endhide
\fi

% lower-case greek
\newcommand{\al}{\alpha}
\newcommand{\be}{\beta}
\newcommand{\gam}{\gamma}
\newcommand{\del}{\delta}
\newcommand{\ep}{\epsilon}
\newcommand{\ze}{\zeta} 
\newcommand{\kap}{\kappa} 
\newcommand{\lam}{\lambda}  
\newcommand{\sig}{\sigma} 
\newcommand{\omi}{\omicron}
\newcommand{\up}{\upsilon}
\newcommand{\om}{\omega}

% upper-case greek
\newcommand{\Gam}{\Gamma}
\newcommand{\Del}{\Delta}
\newcommand{\Lam}{\Lambda} 
\newcommand{\Sig}{\Sigma} 
\newcommand{\Om}{\Omega}

% blackboard bold
\newcommand{\C}{\mathbb{C}}
\newcommand{\E}{\mathbb{E}}
\newcommand{\F}{\mathbb{F}}
\renewcommand{\H}{\mathbb{H}}
\newcommand{\K}{\mathbb{K}}
\newcommand{\N}{\mathbb{N}}
\renewcommand{\O}{\mathbb{O}}
\newcommand{\Q}{\mathbb{Q}}
\newcommand{\R}{\mathbb{R}}
\renewcommand{\S}{\mathbb{S}}
\newcommand{\T}{\mathbb{T}}
\newcommand{\V}{\mathbb{V}}
\newcommand{\Z}{\mathbb{Z}}

% math caligraphic
\newcommand{\MA}{\mathcal{A}}
\newcommand{\MB}{\mathcal{B}}
\newcommand{\MC}{\mathcal{C}}
\newcommand{\MD}{\mathcal{D}}
\newcommand{\ME}{\mathcal{E}}
\newcommand{\MF}{\mathcal{F}}
\newcommand{\MG}{\mathcal{G}}
\newcommand{\MH}{\mathcal{H}}
\newcommand{\MI}{\mathcal{I}}
\newcommand{\MJ}{\mathcal{J}}
\newcommand{\MK}{\mathcal{K}}
\newcommand{\ML}{\mathcal{L}}
\newcommand{\MM}{\mathcal{M}}
\newcommand{\MN}{\mathcal{N}}
\newcommand{\MO}{\mathcal{O}}
\newcommand{\MP}{\mathcal{P}}
\newcommand{\MQ}{\mathcal{Q}}
\newcommand{\MR}{\mathcal{R}}
\newcommand{\MS}{\mathcal{S}}
\newcommand{\MT}{\mathcal{T}}
\newcommand{\MU}{\mathcal{U}}
\newcommand{\MV}{\mathcal{V}}
\newcommand{\MW}{\mathcal{W}}
\newcommand{\MX}{\mathcal{X}}
\newcommand{\MY}{\mathcal{Y}}
\newcommand{\MZ}{\mathcal{Z}}

% mathfrak
\newcommand{\MFX}{\mathfrak{X}}
\newcommand{\MFg}{\mathfrak{g}}
\newcommand{\MFh}{\mathfrak{h}}

% tilde 
\newcommand{\tMA}{\tilde{\MA}}
\newcommand{\tMB}{\tilde{\MB}}
\newcommand{\tU}{\tilde{U}}
\newcommand{\tV}{\tilde{V}}
\newcommand{\tphi}{\tilde{\phi}}
\newcommand{\tpsi}{\tilde{\psi}}
\newcommand{\tF}{\tilde{F}}

\newcommand{\iid}{\stackrel{iid}{\sim}}





% label/reference
% internal label/reference
\newcommand{\lex}[1]{\label{ex:#1}}
\newcommand{\rex}[1]{Exercise \ref{ex:#1}}

\newcommand{\ld}[1]{\label{defn:#1}}
\newcommand{\rd}[1]{Definition \ref{defn:#1}}

\newcommand{\lax}[1]{\label{ax:#1}}
\newcommand{\rax}[1]{Axiom \ref{ax:#1}}

\newcommand{\lfig}[1]{\label{fig:#1}}
\newcommand{\rfig}[1]{Figure \ref{fig:#1}}

% external reference
\newcommand{\extrex}[2]{Exercise \ref{#1-ex:#2}}

\newcommand{\extrd}[2]{Definition \ref{#1-defn:#2}}

\newcommand{\extrax}[2]{Axiom \ref{#1-ax:#2}}

\newcommand{\extrfig}[2]{Figure \ref{#1-fig:#2}}

% external documents (EDIT HERE)
%\externaldocument[analysis-]{"/home/carson/Desktop/Github/Mathematics/Introduction to Analysis/Introduction to Analysis.tex"}




% math operators
\DeclareMathOperator{\supp}{supp}
\DeclareMathOperator{\sgn}{sgn}
\DeclareMathOperator{\spn}{span}
\DeclareMathOperator{\Iso}{Iso}
\DeclareMathOperator{\Eq}{Eq}
\DeclareMathOperator{\id}{id}
\DeclareMathOperator{\Aut}{Aut}
\DeclareMathOperator{\Endo}{End}
\DeclareMathOperator{\Homeo}{Homeo}
\DeclareMathOperator{\Sym}{Sym}
\DeclareMathOperator{\Alt}{Alt}
\DeclareMathOperator{\cl}{cl}
\DeclareMathOperator{\Int}{Int}
\DeclareMathOperator{\bal}{bal}
\DeclareMathOperator{\cyc}{cyc}
\DeclareMathOperator{\cnv}{conv}
\DeclareMathOperator{\epi}{epi}
\DeclareMathOperator{\dom}{dom}
\DeclareMathOperator{\cod}{cod}
\DeclareMathOperator{\codim}{codim}
\DeclareMathOperator{\Obj}{Obj}
\DeclareMathOperator{\Derivinf}{Deriv^{\infty}}
\DeclareMathOperator{\Hom}{Hom}
\DeclareMathOperator*{\argmax}{arg\,max}
\DeclareMathOperator*{\argmin}{arg\,min}
\DeclareMathOperator{\diam}{\text{diam}}
\DeclareMathOperator{\rnk}{\text{rank}}
\DeclareMathOperator{\tr}{\text{tr}}
\DeclareMathOperator{\prj}{\text{proj}}
\DeclareMathOperator{\nab}{\nabla}
\DeclareMathOperator{\diag}{\text{diag}}
\DeclareMathOperator*{\ind}{\text{ind}}
\DeclareMathOperator*{\ar}{\text{arity}}
\DeclareMathOperator*{\cur}{\text{cur}}
\DeclareMathOperator*{\Part}{\text{Part}}
\DeclareMathOperator{\Var}{\text{Var}}
\DeclareMathOperator*{\FIP}{\text{FIP}} 
\DeclareMathOperator*{\Fun}{\text{Fun}} 
\DeclareMathOperator*{\Rel}{\text{Rel}} 
\DeclareMathOperator*{\Cons}{\text{Cons}} 
\DeclareMathOperator*{\Sg}{\text{Sg}} 
\DeclareMathOperator*{\ot}{\otimes}
\DeclareMathOperator{\uni}{Uni}

% Algebra
\DeclareMathOperator{\inv}{\text{inv}}
\DeclareMathOperator{\mult}{\text{mult}}
\DeclareMathOperator{\smult}{\text{smult}}

% category theory
\DeclareMathOperator*{\Set}{\text{\tbf{Set}}}
\DeclareMathOperator*{\BanAlg}{\text{\tbf{BanAlg}}}
\DeclareMathOperator*{\Meas}{\text{\tbf{Meas}}}
\DeclareMathOperator*{\TopMeas}{\text{\tbf{TopMeas}}}
\DeclareMathOperator*{\Msrpos}{\text{\tbf{Msr}}_{+}}
\DeclareMathOperator*{\TopMsrpos}{\text{\tbf{TopMsr}}_{+}}
\DeclareMathOperator*{\TopRadMsrpos}{\text{\tbf{TopRadMsr}}_{+}}
\DeclareMathOperator*{\TopRadMsrone}{\text{\tbf{TopRadMsr}}_{1}}
\DeclareMathOperator*{\MsrC}{\text{\tbf{Msr}}_{\C}} 
\DeclareMathOperator*{\TopMsrC}{\text{\tbf{TopMsr}}_{\C}} 
\DeclareMathOperator*{\TopRadMsrC}{\text{\tbf{TopRadMsr}}_{\C}} 
\DeclareMathOperator*{\Maninf}{\text{\tbf{Man}}^{\infty}} 
\DeclareMathOperator*{\ManBndinf}{\text{\tbf{ManBnd}}^{\infty}} 
\DeclareMathOperator*{\Man0}{\text{\tbf{Man}}^{0}}
\DeclareMathOperator*{\Buninf}{\text{\tbf{Bun}}^{\infty}} 
\DeclareMathOperator*{\VecBuninf}{\text{\tbf{VecBun}}^{\infty}} 
\DeclareMathOperator*{\Field}{\text{\tbf{Field}}} 
\DeclareMathOperator*{\Mon}{\text{\tbf{Mon}}} 
\DeclareMathOperator*{\Grp}{\text{\tbf{Grp}}}
\DeclareMathOperator*{\Semgrp}{\text{\tbf{Semgrp}}}
\DeclareMathOperator*{\LieGrp}{\text{\tbf{LieGrp}}} 
\DeclareMathOperator*{\Alg}{\text{\tbf{Alg}}} 
\DeclareMathOperator*{\Vect}{\text{\tbf{Vect}}} 
\DeclareMathOperator*{\Mod}{\text{\tbf{Mod}}}
\DeclareMathOperator*{\Rep}{\text{\tbf{Rep}}} 
\DeclareMathOperator*{\URep}{\text{\tbf{URep}}}
\DeclareMathOperator*{\Ban}{\text{\tbf{Ban}}} 
\DeclareMathOperator*{\Hilb}{\text{\tbf{Hilb}}} 
\DeclareMathOperator*{\Prob}{\text{\tbf{Prob}}} 
\DeclareMathOperator*{\PrinBuninf}{\text{\tbf{PrinBun}}^{\infty}}

\DeclareMathOperator*{\Top}{\text{\tbf{Top}}}
\DeclareMathOperator*{\TopField}{\text{\tbf{TopField}}} 
\DeclareMathOperator*{\TopMon}{\text{\tbf{TopMon}}} 
\DeclareMathOperator*{\TopGrp}{\text{\tbf{TopGrp}}}
\DeclareMathOperator*{\TopVect}{\text{\tbf{TopVect}}} 
\DeclareMathOperator*{\TopEq}{\text{\tbf{TopEq}}}

\DeclareMathOperator*{\VectR}{\text{\tbf{Vect}}_{\R}}
\DeclareMathOperator*{\VectC}{\text{\tbf{Vect}}_{\C}} 
\DeclareMathOperator*{\VectK}{\text{\tbf{Vect}}_{\K}}
\DeclareMathOperator*{\Cat}{\text{\tbf{Cat}}}
\DeclareMathOperator*{\0}{\mbf{0}}
\DeclareMathOperator*{\1}{\mbf{1}}


\DeclareMathOperator*{\Cone}{\text{\tbf{Cone}}}

\DeclareMathOperator*{\Cocone}{\text{\tbf{Cocone}}}


% Algebra
\DeclareMathOperator{\End}{\text{End}} 
\DeclareMathOperator{\rep}{\text{Rep}} 




% notation
\renewcommand{\r}{\rangle}
\renewcommand{\l}{\langle}
\renewcommand{\div}{\text{div}}
\renewcommand{\Re}{\text{Re} \,}
\renewcommand{\Im}{\text{Im} \,}
\newcommand{\Img}{\text{Img} \,}
\newcommand{\grad}{\text{grad}}
\newcommand{\tbf}[1]{\textbf{#1}}
\newcommand{\tcb}[1]{\textcolor{blue}{#1}}
\newcommand{\tcr}[1]{\textcolor{red}{#1}}
\newcommand{\mbf}[1]{\mathbf{#1}}
\newcommand{\ol}[1]{\overline{#1}}
\newcommand{\ub}[1]{\underbar{#1}}
\newcommand{\tl}[1]{\tilde{#1}}
\newcommand{\p}{\partial}
\newcommand{\Tn}[1]{T^{r_{#1}}_{s_{#1}}(V)}
\newcommand{\Tnp}{T^{r_1 + r_2}_{s_1 + s_2}(V)}
\newcommand{\Perm}{\text{Perm}}
\newcommand{\wh}[1]{\widehat{#1}}
\newcommand{\wt}[1]{\widetilde{#1}}
\newcommand{\defeq}{\vcentcolon=}
\newcommand{\Con}{\text{Con}}
\newcommand{\ConKos}{\text{Con}_{\text{Kos}}}
\newcommand{\trl}{\triangleleft}
\newcommand{\trr}{\triangleright}
\newcommand{\alg}{\text{alg}}
\newcommand{\Triv}{\text{Triv}}
\newcommand{\Der}{\text{Der}}
\newcommand{\cnj}{\text{conj}}

\newcommand{\lcm}{\text{lcm}}
\newcommand{\Imax}{\MI_{\text{max}}}


\DeclareMathOperator*{\Rl}{\text{Re}}
\DeclareMathOperator*{\Imn}{\text{Imn}}



% limits
\newcommand{\limfn}{\liminf \limits_{n \rightarrow \infty}}
\newcommand{\limpn}{\limsup \limits_{n \rightarrow \infty}}
\newcommand{\limn}{\lim \limits_{n \rightarrow \infty}}
\newcommand{\convt}[1]{\xrightarrow{\text{#1}}}
\newcommand{\conv}[1]{\xrightarrow{#1}} 
\newcommand{\seq}[2]{(#1_{#2})_{#2 \in \N}}

% intervals
\newcommand{\RG}{[0,\infty]}
\newcommand{\Rg}{[0,\infty)}
\newcommand{\Rgp}{(0,\infty)}
\newcommand{\Ru}{(\infty, \infty]}
\newcommand{\Rd}{[\infty, \infty)}
\newcommand{\ui}{[0,1]}

% integration \newcommand{\dm}{\, d m}
\newcommand{\dmu}{\, d \mu}
\newcommand{\dnu}{\, d \nu}
\newcommand{\dlam}{\, d \lambda}
\newcommand{\dP}{\, d P}
\newcommand{\dQ}{\, d Q}
\newcommand{\dm}{\, d m}
\newcommand{\dsh}{\, d \#}

% abreviations 
\newcommand{\lsc}{lower semicontinuous}

% misc
\newcommand{\as}[1]{\overset{#1}{\sim}}
\newcommand{\astx}[1]{\overset{\text{#1}}{\sim}}
\newcommand{\io}{\text{ i.o.}}
%\newcommand{\ev}{\text{ ev.}}
\newcommand{\Ll}{L^1_{\text{loc}}(\R^n)}

\newcommand{\loc}{\text{loc}}
\newcommand{\BV}{\text{BV}}
\newcommand{\NBV}{\text{NBV}}
\newcommand{\TV}{\text{TV}}

\newcommand{\op}[1]{\mathcal{#1}^{\text{op}}}


% Glossary - Notation
\glsxtrnewsymbol[description={finite measures on $(X, \MA)$}]{n000001}{$\MM_+(X, \MA)$}
\glsxtrnewsymbol[description={velocity}]{v}{\ensuremath{v}}


\makeindex

\begin{document}
	
	\frontmatter
	
	\title{Introduction to Category Theory}
	
	%    Remove any unused author tags.
	
	%    author one information
	\author{Carson James}
	\thanks{}
	
	\date{}
	
	\maketitle
	
	%    Dedication.  If the dedication is longer than a line or two,
	%    remove the centering instructions and the line break.
	%\cleardoublepage
	%\thispagestyle{empty}
	%\vspace*{13.5pc}
	%\begin{center}
	%  Dedication text (use \\[2pt] for line break if necessary)
	%\end{center}
	%\cleardoublepage
	
	%    Change page number to 6 if a dedication is present.
	\setcounter{page}{4}
	
	\tableofcontents
	\printunsrtglossary[type=symbols,style=long,title={Notation}]
	
	%    Include unnumbered chapters (preface, acknowledgments, etc.) here.
	%\include{}
	
	\mainmatter
	%    Include main chapters here.
	%\include{}
	
	\chapter*{Preface}
	\addcontentsline{toc}{chapter}{Preface}
	
	\begin{flushleft}
		\href{https://creativecommons.org/licenses/by-nc-sa/4.0/legalcode.txt}{cc-by-nc-sa}
	\end{flushleft}

	\newpage
	
	
	
	
	
	
	
	
	
	
	
	
	
	
	
	
	\chapter{Basic Concepts}
	
	\section{von Neumann–Bernays–Gödel Set Theory}
	
	\begin{defn} \ld{11001} 
		Let $x$ be a class. Then $x$ is said to be a set iff there exists a class $A$ such that $x \in A$. 
	\end{defn}

	\begin{defn}
		product of two classes
	\end{defn}
	
	\begin{defn}
		Let $A$, $B$ be classes and $R \subset A \times B$. elation from $A$ to $B$.
	\end{defn}

	\begin{note}
		We can define cartesion products, relations, and functions for classes just like for sets. 
	\end{note}

	\begin{ax} \lax{11001.1a} \textbf{Axiom of Extensionality:}  \\
		Let $x$ and $y$ be classes. If for each set $a$, $a \in x$ iff $a \in y$, then $x = y$.
	\end{ax}
	
	\begin{ax} \lax{11001.1b} \textbf{Axiom of Pairing:}  \\
		Let $a$, $b$ be sets. Then there exists a set $p$ such that for each for each set $x$, $x \in p$ iff $x = a$ or $x = b$. 
	\end{ax}

	\begin{ex}
		Let $a$, $b$ be sets. Then there exists a unique set $p$ such that for each for each set $x$, $x \in p$ iff $x = a$ or $x = b$. 
	\end{ex}
	
	\begin{proof}
		By \rax{11001.1b} implies that there exists a set $p$ such that for each for each set $x$, $x \in p$ iff $x = a$ or $x = b$. Let $q$ be a set. Suppose that for each for each set $x$, $x \in q$ iff $x = a$ or $x = b$. Then  
	\end{proof}

	\begin{defn}
		Let $x$ and $y$ be sets. We define $(x,y) = \{\}$, denoted 
	\end{defn}
	
	\begin{ax} \lax{11001.1} \textbf{Axiom of Replacement:}  \\
		Let $A$, $B$ be classes and $f:A \rightarrow B$. If $A$ is a set, then $f(A)$ is a set. 
	\end{ax}

	\begin{ax} \lax{11002} \textbf{Schema of Specification:}  \lax{11002}\\
		Let $\phi$ a propositional function on sets. Then there exists a class $A$ such that for each set $x$, $x \in A$ iff $\phi(x)$. 
	\end{ax}
	
	\begin{ex} \lex{11003}
		There exists a class $A$ such that for each class $x$, $x \in A$ iff $x$ is a set.
	\end{ex}
	
	\begin{proof}
		Define $\phi$ by $$\phi(x) : x = x$$ 
		\rax{11002} implies that there exists a class $A$ such that for each set $x$, $x \in A$ iff $x = x$. Let $x$ be a class. If $x \in A$, then by definition, $x$ is a set. \\
		Conversely, if $x$ is a set, then by construction, $x \in A$.
	\end{proof}
	
	\begin{ex} \lex{11004}
		There exists a class $A$ such that for each class $G$ and $*: G \times G \rightarrow G$, $(G, *) \in A$ iff $(G, *)$ is a group.
	\end{ex}
	
	\begin{proof}
		Define $\phi_1$, $\phi_2$ and $\phi_3$ by 
		\begin{itemize}
			\item $\phi_1(G, *): *:G \times G \rightarrow G$ is associative
			\item $\phi_2(G, *):$ there exists $e \in G$ such that for each $g \in G$, $e * g = g * e = g$
			\item $\phi_3(G, *):$ for each $g \in G$ there exists $h \in G$ such that $g * h = h * g = e$
		\end{itemize}
		Define $\phi$ by 
		$$\phi(G, *) : \phi_1(G, *) \text{ and } \phi_2(G, *) \text{ and } \phi_3(G, *)$$
		Then there exists a class $A$ such that for each set $G$ and  $*: G \times G \rightarrow G$, $(G, *) \in A$ iff $\phi(G, *)$ $(G, *) \text{ ``is a group"}$. Therefore, for each group $(G, *)$, $(G, *) \in A$.
		\textbf{FINISH!!!}
	\end{proof}
	
	
	\subsection{TO DO}
	\begin{enumerate} 
		\item \tcb{ cover existence of subclasses, products of classes to be able to define class relations and subsequently class functions}
		\item 
	\end{enumerate}
	
	
	
	
	
	
	
	
	
	
	
	
	
	
	
	
	
	
	
	
	
	
	
	
	
	
	
	
	
	
	
	
	
	
	\newpage
	\section{Categories}
	
	\subsection{Introduction}
	
	\begin{defn}  \ld{12001}
		Let $\MC_0$, $\MC_1$ be classes and $\dom, \cod : \MC_1 \rightarrow \MC_0$ class functions. Set $\MC = (C_0, C_1, \dom, \cod)$. Then $\MC$ is said to be a \textbf{category} if 
		\begin{enumerate}
			\item (composition): for each $f,g \in C_1$, if $\cod(f) = \dom(g)$, then there exists $g \circ f \in C_1$ such that $\dom(g \circ f) = \dom(f)$ and $\cod(g \circ f) = \cod(g)$
			\item (associativity): for each $f,g,h \in C_1$, if $\cod(f) = \dom(g)$ and $\cod(g) = \dom(h)$, then $$(h \circ_{\MC} g) \circ_{\MC} f = h \circ_{\MC} (g \circ_{\MC} f)$$  
			\item (identity): for each $X \in \MC_0$, there exists $\id^{\MC}_{X} \in C_1$ such that $\dom(\id^{\MC}_X) = \cod(\id^{\MC}_X) = X$ and for each $f, g \in C_1$, if $\dom(f) = X$ and $\cod(g) = X$, then $$f \circ_{\MC} \id^{\MC}_X = f \text{ and } \id^{\MC}_X \circ_{\MC} g = g$$ 
		\end{enumerate}
		We define the
		\begin{itemize}
			\item \textbf{objects of $\MC$}, denoted $\Obj(\MC)$, by $\Obj(\MC) = C_0$
			\item \textbf{morphisms of $\MC$}, denoted $\Hom_{\MC}$, by $\Hom_{\MC} = C_1$
		\end{itemize}
		For $X, Y \in \Obj(\MC)$, we define the \textbf{morphisms of $\MC$ from $X$ to $Y$}, denoted $\Hom_{\MC}(X, Y)$, by $\Hom_{\MC}(X, Y) = \{f \in \Hom(\MC): \dom(f) = X \text{ and } \cod(f) = Y\}$.
	\end{defn}

	\begin{note}
		When the context is clear, we write $g \circ f$ and $\id_X$ in place of $g \circ_{\MC} f $ and $\id^{\MC}_{X}$ respectively.
	\end{note}

	\begin{defn}
		Let $\MC$ be a category. We define $\Hom_{\MC}^{(2)} = \{(g,f) \in \Hom_{\MC} \times \Hom_{\MC} : \cod(f) = \dom(g) \}$.
	\end{defn}

	\begin{ex}
		Let $\MC$ be a category. Then 
		\begin{enumerate}
			\item $\circ \in \MR$
			\item $\circ: \Hom_\MC^{(2)} \rightarrow \Hom_{\MC}$
		\end{enumerate}
	\end{ex}

	\begin{proof}
		\begin{itemize}
			Let $(g, f ) \in \Hom_{\MC}^{(2)}$. Since $\MC$ is a category, there exists $g$
		\end{itemize}
	\end{proof}

	\begin{note} 
		We typically define a category $\MC$ by specifying 
		\begin{itemize}
			\item $\Obj(\MC)$
			\item for $X,Y \in \Obj(\MC)$, the class $\Hom_{\MC}(X, Y)$
			\item for $X,Y,Z \in \Obj(\MC)$, $f \in \Hom_{\MC}(X, Y)$ and $g \in \Hom_{\MC}(Y, Z)$, the composite morphism $g \circ f \in \Hom_{\MC}(X,Z)$.
		\end{itemize}
		and then show 
		\begin{itemize}
			\item well-definedness of composition
			\item associativity of composition 
			\item existence of identities 
		\end{itemize}
	\end{note}

	\begin{defn} \ld{12001.1}
		We define the \textbf{empty category}, denoted $\0$, by 
		\begin{itemize}
			\item $\Obj(\0) = \varnothing$
			\item $\Hom_{\0} = \varnothing$
		\end{itemize}
	\end{defn}

	\begin{ex} \lex{12001.2}
		We have that $\0$ is a category.
	\end{ex}
	
	\begin{proof}
		Vacuously true.
	\end{proof}

	\begin{defn} \ld{12001.}
		We define the \textbf{trivial category}, denoted $\1$, by 
		\begin{itemize}
			\item $\Obj(\1) = \{*\}$
			\item $\Hom_{\1} = \{\id_*\}$
		\end{itemize}
	\end{defn}

	\begin{ex} \lex{12001.}
		We have that $\1$ is a category.
	\end{ex}

	\begin{proof}
		Clear.
	\end{proof}
	
	\begin{defn} \ld{12001.}
		We define $\Set$ by  
		\begin{itemize}
			\item $\Obj(\Set) = \{A: A \text{ is a set}\}$ 
			\item for each $A,B \in \Obj(\Set)$, $\Hom_{\Set}(A,B) = \{f: f:A \rightarrow B\}$
			\item for $A, B, C \in \Set$, $f \in \Hom_{\Set}(A, B)$ and $g \in \Hom_{\Set}(B, C)$, $g \circ_{\Set} f = g \circ f$. 
		\end{itemize}
	\end{defn}

	\begin{ex} \lex{12001.}
		We have that $\Set$ is a category.
	\end{ex}

	\begin{proof}\
		\begin{itemize}
			\item \tbf{well-definedness of composition:}
			\item \tbf{associativity of composition:} 
			\item \tbf{existence of identities:} 
		\end{itemize}
		\tbf{FINISH!!!}
	\end{proof}
	
	\begin{defn} \ld{12002}
		Let $\MC$ be a category. Then $\MC$ is said to be 
		\begin{itemize}
			\item \textbf{small} if $\Obj(\MC)$ and $\Hom_{\MC}$ are sets
			\item \textbf{locally small} if for each $A,B \in \Obj(\MC)$, $\Hom_{\MC}(A,B)$ is a set 
		\end{itemize}
	\end{defn}

	\begin{ex} \lex{12003}
		Let $\MC$ be a category. If $\MC$ is small, then $\MC$ is a set. 
	\end{ex}

	\begin{proof}
		Suppose that $\MC$ is small. Then $\Obj(\MC)$ and $\Hom_{\MC}$ are sets. Then $\MP(\Obj(\MC))$, $\MP(\Hom_{\MC})$ and $\Obj(\MC)^{\Hom_{\MC}}$ are sets. Hence $\Obj(\MC) \times \Hom_{\MC} \times \Obj(\MC)^{\Hom_{\MC}} \times \Obj(\MC)^{\Hom_{\MC}}$ is a set. By definition, $\MC = (\Obj(\MC), \Hom_{\MC}, \dom, \cod) \in \Obj(\MC) \times \Hom_{\MC} \times \Obj(\MC)^{\Hom_{\MC}} \times \Obj(\MC)^{\Hom_{\MC}}$. By definition, $\MC$ is a set. 
	\end{proof}
	
	\begin{ex} \lex{12004}
		There exists a class $A$ such that $\MC \in A$ iff $\MC$ is a small category. 
	\end{ex}

	\begin{proof}
		\rex{12003} implies that for each category $\MC$, $\MC$ is small implies that $\MC$ is a set. Define $\phi$ by $$\phi(\MC): \MC \text{ is a small category} $$ 
		Then \rax{11002} implies that there exists a class $A$ such that $\MC \in A$ iff $\MC$ is a small category.  
	\end{proof}





















	
	
	\subsection{Opposite Category}

	\begin{defn}  \ld{12004}
		Let $\MC$ be a category, we define the \text{dual of $\MC$} or the \textbf{opposite of $\MC$}, denoted $\op{C}$, by 
		\begin{itemize}
			\item $\Obj(\op{C}) = \Obj(\MC)$
			\item for $X,Y \in \Obj(\op{C})$, $\Hom_{\op{C}}(X,Y) = \Hom_{\MC}(Y,X)$
			\item for $X, Y, Z \in \Obj(\op{C})$ and $f \in \Hom_{\op{C}}(X, Y), g \in \Hom_{\op{C}}(Y,Z)$, $g \circ_{\op{C}} f = f \circ_{\MC} g$
		\end{itemize}
	\end{defn}

	\begin{ex}  \lex{12005}
		Let $\MC$ be a category. Then $\op{C}$ is a category.
	\end{ex}
	
	\begin{proof}\
		\begin{itemize}
			\item for $W,X,Y,Z \in \Obj(\MC)$, $f \in \Hom_{\op{C}}(W, X)$ and $g \in \Hom_{\op{C}}(X, Y)$ and $h \in \Hom_{\op{C}}(Y, Z)$. Then 
			\begin{align*}
				(h \circ_{\op{C}} g) \circ_{\op{C}} f 
				&= f \circ_{\MC}( h \circ_{\op{C}} g) \\
				&= f \circ_{\MC} (g \circ_{\MC} h) \\
				&= (f \circ_{\MC} g) \circ_{\MC} h \\
				&= h \circ_{\op{C}} (f \circ_{\MC} g) \\
				&= h \circ_{\op{C}} (g \circ_{\op{C}} f)
			\end{align*}
			So composition is associative.
			\item Let $X \in \Obj(\MC)$ and $f,g \in \Hom_{\op{C}}$. Suppose that $\dom(f) = X$ and $\cod(g) = X$ 
			Then 
			\begin{align*}
				f  \circ_{\op{C}} \id_X
				&= \id_X  \circ_{\MC} f \\
				&= f
			\end{align*}
			and 
			\begin{align*}
				\id_X  \circ_{\op{C}} g 
				&= g \circ_{\MC} \id_X \\
				&= g
			\end{align*}
		So $(\id_X)_{\op{C}} =  (\id_X)_{\MC}$.
		\end{itemize}
	\end{proof}














	
	
	
	\subsection{Slice Category}
	
	\begin{defn} \ld{12006}
		Let $\MC$ be a category and $X \in \Obj(\MC)$. We define the \textbf{slice category of $\MC$ over $X$}, denoted $\MC / X$, by
		\begin{itemize}
			\item $\Obj(\MC / X) = \{f \in \Hom_{\MC}: \cod(f) = X\}$
			\item for $f,g \in \Obj(\MC / X)$, 
			$$\Hom_{\MC / X}(f, g) = \{\al \in \Hom_{\MC} : \dom(\al) = \dom(f) \text{, } \cod(\al) = \dom(g) \text{ and } f = g \circ \al \}$$
			i.e. for $f \in \Hom_{\MC}(A, X)$ and $g \in \Hom_{\MC}(B, X)$, $\al \in \Hom_{\MC / X}(f, g)$ iff the following diagram commutes: 
			\[ 
			\begin{tikzcd}
				A \arrow[rr, "\al"] \arrow[dr, "f"'] 	
				& & B  \arrow[dl, "g"] \\
				& X 
			\end{tikzcd}
			\]
			\item for $f,g, h \in \Obj(\MC / X)$, $\al \in \Hom_{\MC / X}(f, g)$ and $\be \in \Hom_{\MC / X}(g, h)$, 
			$$\be \circ_{\MC / X} \al = \be \circ_{\MC} \al$$
		\end{itemize}
	\end{defn}
	
	\begin{ex}  \lex{12007}
		Let $\MC$ be a category and $X \in \Obj(\MC)$. Then $\MC / X$ is a category.
	\end{ex}

	\begin{proof}\
		\begin{itemize}
			\item $f,g, h \in \Obj(\MC / X)$, $\al \in \Hom_{\MC / X}(f, g)$ and $\be \in \Hom_{\MC / X}(g, h)$. Then $f = g \circ_{\MC} \al$ and $g = h \circ_{\MC} \be$, i.e. the following diagrams commute:
			\[ \begin{tikzcd}
				\dom(f) \arrow[rr, "\al"] \arrow[dr, "f"'] 	
				&& \dom(g)  \arrow[dl, "g"]  
				&& \dom(g) \arrow[rr, "\be"] \arrow[dr, "g"'] 	
				&& \dom(h)  \arrow[dl, "h"]\\
				& X 
				&& &&X
			\end{tikzcd}
			\]
			Therefore, we have that 
			\begin{align*}
				f 
				& = g \circ_{\MC} \al \\
				& = (h \circ_{\MC} \be) \circ_{\MC} \al \\
				&= h \circ_{\MC} (\be \circ_{\MC} \al) 
			\end{align*}
			i.e. the following diagram commutes:
			\[ 
			\begin{tikzcd}
				\dom(f) \arrow[rr, "\be \circ_{\MC} \al"] \arrow[dr, "f"'] 	
				& & \dom(h)  \arrow[dl, "g"] \\
				& X 
			\end{tikzcd}
			\]
			which implies that 
			\begin{align*}
				\be \circ_{\MC / X} \al 
				& = \be \circ_{\MC} \al \\
				& \in \Hom_{\MC / X}(f, h)
			\end{align*}
			and composition is well defined. 
			\item Associativity of $\circ_{\MC / X}$ follows from associativity of $\circ_\MC$.
			\item Let $f \in \Obj(\MC / X)$ and $\al, \be \in \Hom_{\MC / X}$. Since $f \circ \id_{\dom_{\MC}(f)} = f$, i.e. the following diagram commutes: 
			\[ \begin{tikzcd}
				\dom_{\MC}(f) \arrow[rr, "\id_{\dom_{\MC}(f)}"] \arrow[dr, "f"'] 	
				& & \dom_{\MC}(f)  \arrow[dl, "f"] \\
				& X 
			\end{tikzcd}
			\]
			we have that $\id_{\dom_{\MC}(f)} \in \Hom_{\MC / X}(f, f)$.
			Suppose that $\dom_{\MC / X}(\al) = f$ and $\cod_{\MC /X}(\be) = f$. Then 
			\begin{align*}
				\al \circ_{\MC /X} \id_{\dom_{\MC}(f)}
				&= \al \circ_{C} \id_{\dom_{\MC}(f)} \\
				&= \al
			\end{align*}
			and 
			\begin{align*}
				\id_{\dom_{\MC}(f)} \circ_{\MC /X} \be
				&= \id_{\dom_{\MC}(f)} \circ_{\MC} \be \\
				&= \be
			\end{align*}
			So $\id_f = \id_{\dom_{\MC}(f)}$.
		\end{itemize}
	\end{proof}









































\subsection{Subcategories}

\begin{defn}
	Let $\MC$ and $\MD$ be categories. Then $\MD$ is said to be a \tbf{subcategory of $\MC$}, denoted $\MD \subset \MC$, if 
	\begin{enumerate}
		\item $\Obj(\MD) \subset \Obj(\MC)$
		\item for each $A, B \in \Obj(\MD)$, $\Hom_{\MD}(A,B) \subset \Hom_{\MC}(A, B)$
		\item for each $A,B,C \in \Obj(\MD)$, $d \in \Hom_{\MD}(A, B)$ and $g \in \Hom_{\MD}(B, C)$, $g \circ_{\MD} f = g \circ_{\MC} f$
		\item for each $A \in \Obj(\MD)$, $\id_A $
	\end{enumerate}
\end{defn}




















	
	\subsection{Product Categories}
	
	\begin{defn}
		Let $\MC$ and $\MD$ be categories. We define the \textbf{product category of $\MC$ and $\MD$}, denoted $\MC \times \MD$ by 
		\begin{itemize}
			\item $\Obj(\MC \times \MD) = \{(A, B): A \in \Obj(\MC) \text{ and } B \in \Obj(\MD)\}$
			\item for each $(A, A'), (B, B') \in \Obj(\MC \times \MD)$, $\Hom_{\MC \times \MD}((A, A'), (B, B')) = \{(f,g): f \in \Hom_{\MC}(A, B) \text{ and }  g \in \Hom_{\MC}(A', B') \}$
			\item for each  $(A, A'), (B, B'), (C, C') \in \Obj(\MC \times \MD)$, $(f, f') \in \Hom_{\MC \times \MD}((A, A'), (B, B'))$ and $(g, g') \in \Hom_{\MC \times \MD}((B, B'), (C, C'))$, 
			$$(g,g') \circ_{\MC \times \MD} (f, f') = (g \circ_{\MC} f, g' \circ_{\MD} f')$$
		\end{itemize}
	\end{defn}
	
	\begin{ex}
		Let $\MC$ and $\MD$ be categories. Then $\MC \times \MD$ is a category. 
	\end{ex}
	
	\begin{proof}\
		\begin{itemize}
			\item \textbf{well-definedness of composition: } \\ 
			Let $(A, A'), (B, B'), (C, C') \in \Obj(\MC \times \MD)$, $(f, f') \in \Hom_{\MC \times \MD}((A, A'), (B, B'))$ and $(g, g') \in \Hom_{\MC \times \MD}((B, B'), (C, C'))$. Then $f \in \Hom_{\MC}(A, B)$, $g \in \Hom_{\MC}(B, C)$, $f'  \in \Hom_{\MD}(A', B')$, and  $g' \in \Hom_{\MD}(B', C')$. Hence $g \circ_{\MC} f \in \Hom_{\MC}(A, C)$ and $g' \circ_{\MD} f' \in \Hom_{\MD}(A', C')$. Thus 
			\begin{align*}
				(g,g') \circ_{\MC \times \MD} (f, f') 
				&= (g \circ_{\MC} f, g' \circ_{\MD} f') \\
				& \in \Hom_{\MC \times \MD}((A, A'), (C, C'))
			\end{align*}
			Thus, composition is well defined. \\
			\item \textbf{associativity of composition:} \\
			Let $(A, A'), (B, B'), (C, C'), (D, D') \in \Obj(\MC \times \MD)$, $(f, f') \in \Hom_{\MC \times \MD}((A, A'), (B, B'))$, $(g, g') \in \Hom_{\MC \times \MD}((B, B'), (C, C'))$ and $(h, h') \in \Hom_{\MC \times \MD}((C, C'), (D, D'))$. Then 
			\begin{align*}
				[(h, h') \circ_{\MC \times \MD} (g , g') ] \circ_{\MC \times \MD} (f, f')
				& = (h \circ_{\MC} g, h' \circ_{\MD} g') \circ_{\MC \times \MD} (f, f') \\
				& = ((h \circ_{\MC} g) \circ_{\MC} f, (h' \circ_{\MD} g') \circ_{\MD} f') \\
				& = (h \circ_{\MC} ( g \circ_{\MC} f), h' \circ_{\MD} (g' \circ_{\MD} f')) \\
				& = (h, h') \circ_{\MC \times \MD} (g \circ_{\MC} f, g' \circ_{\MD} f') \\
				& = (h, h') \circ_{\MC \times \MD} [(g, g') \circ_{\MC \times \MD} (f, f')]
			\end{align*} 
			Thus composition is associative. \\
			\item \textbf{existence of identities: } \\
			Let $(A, B) \in \Obj(\MC \times \MD)$, $(f, f'), (g,g') \in \Hom_{\MC \times \MD}$. Suppose that $\dom_{\MC \times \MD}(f, f') = (A, B)$ and $\cod_{\MC \times \MD}(g, g') = (A, B)$. Then $\dom_{\MC}(f) = A$, $\dom_{\MD}(f') = B$, $\cod_{\MC}(g) = A$ and  $\cod_{\MD}(g') = B$. Hence
			\begin{align*}
				(f, f') \circ_{\MC \times \MD} (\id_A, \id_{B}) 
				& = (f \circ_{\MC} \id_A, f' \circ_{\MD} \id_{B}) \\
				& = (f, f)
			\end{align*}
			and
			\begin{align*}
				(\id_A, \id_{B})  \circ_{\MC \times \MD} (g, g')
				& = (\id_A \circ_{\MC} g , \id_{B} \circ g') \\
				& = (g, g')
			\end{align*}
			Therefore $(\id_{(A, B)})_{\MC \times \MD} = (\id_A, \id_{B})$.
		\end{itemize}
	\end{proof}
	














	
	
	

	
	
	
	
	
	
	
	
	
	
	
	
	
	
	
	
	
	
	
	
	
	
	
	
	
	
	\newpage
	\section{Functors}
	
	
	\subsection{Introduction}
	
	\begin{defn} \ld{13001}
		Let $\MC$ and $\MD$ be categories and $F_0 : \Obj(\MC) \rightarrow \Obj(\MD)$, $F_1: \Hom_{\MC} \rightarrow \Hom_{\MD}$ class functions. Set $F = (F_0, F_1)$. Then $F$ is said to be a \text{functor from $\MC$ to $\MD$}, denoted  $F:\MC \rightarrow \MD$, if 
		\begin{enumerate}
			\item for each $A,B \in \Obj(\MC)$ and $f \in \Hom_{\MC}(A, B)$, $F_1(f) \in \Hom_{\MD}(F_0(A), F_0(B))$
			\item for each $A,B, C \in \Obj(\MC)$, $f \in \Hom_{\MC}(A,B)$ and $g \in \Hom_{\MC}(B, C)$, $F_1(g \circ f) = F_1(g) \circ F_1(f)$
			\item for each $A \in \Obj(\MC)$,  $F_1(\id_A) = \id_{F_0(A)}$
		\end{enumerate}
	\end{defn}

	\begin{note}
		For $A \in \Obj(C)$ and $f \in \Hom_{\MC}$, we typically write $F(A)$ and $F(f)$ instead of $F_0(A)$ and $F_1(f)$ respectively.
	\end{note}

	\begin{defn}
		Let $\MC$ be a category. We define the \textbf{empty functor} from $\0$ to $\MC$, denoted $E_{\MC}: \0 \rightarrow \MC$ by $(E_{\MC})_0 = (E_{\MC})_1 = \varnothing$. 
	\end{defn}

	\begin{ex}
		Let $\MC$ be a category. Then $E_{\MC}: \0 \rightarrow \MC$ is a functor.
	\end{ex}

	\begin{proof}
		Since $\Obj(\0) = \varnothing$ and $\Hom_{\0} = \varnothing$, this is vacuously true. 
	\end{proof}

	\begin{defn}
		Let $\MC, \MD$ be categories and $X \in \Obj(\MD)$. We define the \textbf{constant functor} from $\MC$ onto $X$, denoted $\Del^{\MC}_X: \MC \rightarrow \MD$ by 
		\begin{itemize}
			\item $\Del^{\MC}_X (A) = X$
			\item $\Del^{\MC}_X (f) = \id_X$
		\end{itemize}
	\end{defn}
	
	\begin{ex}
		Let $\MC, \MD$ be categories and $X \in \Obj(\MD)$. Then $\Del^{\MC}_X : \MC \rightarrow \MD$ is a functor.
	\end{ex}

	\begin{proof}\
		\begin{enumerate}
			\item Let $A, B \in \Obj(\MC)$ and $f \in \Hom_{\MC}(A, B)$. Then 
			\begin{align*}
				\Del^{\MC}_X (f)
				& = \id_X \\
				& \in \Hom_{\MD}(X, X) \\
				& = \Hom_{\MD}(\Del^{\MC}_X (A), \Del^{\MC}_X (B))
			\end{align*}  
			\item Let $A,B,C \in \Obj(\MC)$, $f \in \Hom_{\MC}(A, B)$ and $g \in \Hom_{\MC}(B, C)$. Then 
			\begin{align*}
				\Del^{\MC}_X (g \circ f) 
				& = \id_X \\
				& = \id_X \circ \id_X \\
				& = \Del^{\MC}_X (g) \circ \Del^{\MC}_X (f)
			\end{align*} 
			\item Let $A \in \Obj(\MC)$. Then 
			\begin{align*}
				\Del^{\MC}_X (\id_A)
				& = \id_X \\
				& = \id_{\Del^{\MC}_X (A)}
			\end{align*}
		\end{enumerate}
		So $\Del^{\MC}_X : \MC \rightarrow \MD$ is a functor.
	\end{proof}


















	\subsection{Category of Small Categories}

	\begin{defn} \ld{13006}
		Let $\MC$, $\MD$ and $\ME$ be categories and $F:\MC \rightarrow \MD$, $G: \MD \rightarrow \ME$ functors. We define the \textbf{composition of $G$ with $F$}, denoted $G \circ F: \MC \rightarrow \ME$, by 
		\begin{itemize}
			\item $G \circ F (A) = G(F(A))$
			\item $G \circ F (f) = G(F(f))$
		\end{itemize} 
	\end{defn}
	
	\begin{ex}  \lex{13007}
		Let $\MC$, $\MD$ and $\ME$ be categories and $F:\MC \rightarrow \MD$, $G: \MD \rightarrow \ME$ functors. Then  $G \circ F: \MC \rightarrow \ME$ is a functor.
	\end{ex}

	\begin{proof}\
		\begin{enumerate}
				\item Let $A, B \in \Obj(\MC)$ and $f \in \Hom_{\MC}(A,B)$. Since $F(f) \in \Hom_{\MD}(F(A), F(B))$, we have that $G(F(f)) \in \Hom_{\ME}(G(F(A)), G(F(B)))$. Then 
			\begin{align*}
				G \circ F (f) 
				& = G(F(f)) \\
				& \in \Hom_{\ME}(G(F(A)), G(F(B))) \\
				& =  \Hom_{\ME}(G \circ F(A), G \circ F (B)) \\
			\end{align*}
			\item Let $A,B, C \in \Obj(\MC)$, $f \in \Hom_{\MC}(A,B)$ and $g \in \Hom_{\MC}(B, C)$. Then 
			\begin{align*}
				G \circ F(g \circ f) 
				& = G (F (g \circ f)) \\
				& = G(F(g) \circ F(f)) \\
				& = G(F(g)) \circ G(F(f)) \\
				& = G \circ F (g) \circ G \circ F (f) \\
			\end{align*}
			\item Let $A \in \Obj(\MC)$. Then 
			\begin{align*}
				G \circ F(\id_{A})
				& = G(F(\id_A) )\\
				& = G(\id_{F(A)}) \\
				& = \id_{G(F(A))} \\
				& = \id_{G \circ F(A)} 
			\end{align*}
		\end{enumerate}
		So $G \circ F: \MC \rightarrow \ME$ is a functor. 
	\end{proof}

	\begin{ex}  \lex{13008}
		Let $\MC$, $\MD$, $\ME$, $\MF$ be categories and $F:\MC \rightarrow \MD$, $G : \MD \rightarrow \ME$, $H: \ME \rightarrow \MF$ functors. Then $(H \circ G) \circ F = H \circ (G \circ F)$.
	\end{ex}

	\begin{proof}
		Let $A, B \in \Obj(\MC)$ and $f \in \Hom_{\MC}(A,B)$. Then 
		\begin{itemize}
				\item 
				\begin{align*}
					(H \circ G) \circ F(A) 
					& = H \circ G (F(A)) \\
					& = H (G(F(A))) \\
					& = H( G \circ F (A)) \\
					& = H \circ (G \circ F) (A)
				\end{align*}
				\item  
				\begin{align*}
					(H \circ G) \circ F(f) 
					& = H \circ G (F(f)) \\
					& = H (G(F(f))) \\
					& = H( G \circ F (f)) \\
					& = H \circ (G \circ F) (f)
				\end{align*}
		\end{itemize}
		Hence $(H \circ G) \circ F = H \circ (G \circ F)$.
	\end{proof}

	\begin{defn} \ld{13004}
		Let $\MC$ be a category. We define the \textbf{identity functor from $\MC$ to $\MC$}, denoted $\id_{\MC}: \MC \rightarrow \MC$, by 
		\begin{itemize}
			\item $\id_{\MC}(A) = A$, $(A \in \Obj(\MC))$
			\item $\id_{\MC}(f) = f$, $(f \in \Hom_{\MC})$
		\end{itemize}
	\end{defn}
	
	\begin{ex}  \lex{13005}
		Let $\MC$ be a category. Then $\id_{\MC}: \MC \rightarrow \MC$ is a functor.
	\end{ex}
	
	\begin{proof}\
		\begin{enumerate}
			\item Let $A, B \in \Obj(\MC)$ and $f \in \Hom_{\MC}(A,B)$. Then 
			\begin{align*}
				\id_{\MC}(f) 
				& = f \\
				& \in \Hom_{\MC}(A, B) \\
				& = \Hom_{\MC}(\id_{\MC}(A), \id_{\MC}(B)) \\
			\end{align*}
			\item Let $A,B, C \in \Obj(\MC)$, $f \in \Hom_{\MC}(A,B)$ and $g \in \Hom_{\MC}(B, C)$. Then 
			\begin{align*}
				\id_{\MC}(g \circ f) 
				& = g \circ f \\
				& = \id_{\MC}(g) \circ \id_{\MC} (f)
			\end{align*}
			\item Let $A \in \Obj(\MC)$. Then 
			\begin{align*}
				\id_{\MC}(\id_{A})
				& = \id_A \\
				& = \id_{\id_{\MC}(A)}
			\end{align*}
		\end{enumerate}
	\end{proof}

	\begin{ex} \lex{13006}
		Let $\MC$ and $\MD$ be categories and $F:\MC \rightarrow \MD$. Then 
		\begin{enumerate}
			\item $\id_{\MD} \circ F = F$
			\item $F \circ \id_{\MC} = F$
		\end{enumerate}
	\end{ex}

	\begin{proof}\
		\begin{enumerate}
			\item Let $A, B \in \Obj(\MC)$ and $f \in \Hom_{\MC}(A, B)$. Then 
			\begin{align*}
				\id_{\MD} \circ F(A) 
				& = \id_{\MD}(F(A)) \\
				& = F(A)
			\end{align*}
			and 
			\begin{align*}
				\id_{\MD} \circ F(f) 
				& = \id_{\MD}(F(f)) \\
				& = F(f)
			\end{align*} 
			Since $A, B \in \Obj(\MC)$ and $f \in \Hom_{\MC}(A, B)$ are arbitrary, $\id_{\MD} \circ F = F$.
			\item Let $A, B \in \Obj(\MC)$ and $f \in \Hom_{\MC}(A, B)$. Then
			\begin{align*}
				F \circ \id_{\MC}(A) 
				& = F(\id_{\MC}(A)) \\
				& = F(A) 
			\end{align*}
			and 
			\begin{align*}
				F \circ \id_{\MC}(f) 
				& = F (\id_{\MC}(f)) \\
				& = F(f)
			\end{align*} 
			Since $A, B \in \Obj(\MC)$ and $f \in \Hom_{\MC}(A, B)$ are arbitrary, $ F \circ \id_{\MC} = F$.
		\end{enumerate}
	\end{proof}
	
	\begin{ex} \lex{13008.1}
		Let $\MC$ and $\MD$ be categories and $F:\MC \rightarrow \MD$. If $\MC$ is small, then $F$ is a set.
	\end{ex}
	
	\begin{proof}
		Suppose that $\MC$ is small. Then $\Obj(\MC)$ and $\Hom_{\MC}$ are sets. By definition, there exist $F_0: \Obj(\MC) \rightarrow \Obj(\MD)$ and $F_1: \Hom_{\MC} \rightarrow \Hom_{\MD}$ such that $F = (F_0, F_1)$. \rax{11001.1} implies that $F_0(\Obj(\MC))$ and $F_1(\Hom_{\MC})$ are sets. Therefore, $\Obj(\MC) \times F_0(\Obj(C))$ and $\Hom_{\MC} \times F_1(\Hom_{\MC})$ are sets. Hence $\MP(\Obj(\MC) \times F_0(\Obj(C)))$ and $\MP(\Hom_{\MC} \times F_1(\Hom_{\MC}))$ are sets. Since $F_0 \subset \Obj(\MC) \times F_0(\Obj(C))$ and $F_1 \subset \Hom_{\MC} \times F_1(\Hom_{\MC})$, we have that $F_0 \in \MP(\Obj(\MC) \times F_0(\Obj(C)))$ and $F_1 \in \MP(\Hom_{\MC} \times F_1(\Hom_{\MC}))$. Hence $F_0$ and $F_1$ are sets. Thus $F = (F_0, F_1)$ is a set. 
	\end{proof}

	\begin{ex} \lex{13008.2}
		Let $\MC$ and $\MD$ be categories. Suppose that $\MC$ is small. Then there exists a class $A$ such that for each class $F$, $F \in A$ iff $F: \MC \rightarrow \MD$.   
	\end{ex}
	
	\begin{proof} 
		Let $\MC$ and $\MD$ be categories. Suppose that $\MC$ is small. Define $\phi$ by 
		$$\phi(F) : F : \MC \rightarrow \MD$$ 
		Then there exists a class $A$ such that for each set $F$, $F \in A$ iff $\phi(F)$. Let $F$ be a class. Suppose that $F \in A$. By \rd{11001}, $F$ is a set. Since $F$ is a set and $F \in A$, we have that $\phi(F)$. Hence $F: \MC \rightarrow \MD$. \\
		Conversely, suppose that $F: \MC \rightarrow \MD$. \rex{13008.1} implies that $F$ is a set. Since $F$ is a set and $\phi(F)$ is true, we have that $F \in A$. 
	\end{proof}
	
	\begin{defn} \ld{13009}
		We define $\tbf{Cat}$ by 
		\begin{itemize}
			\item $\Obj(\tbf{Cat}) = \{\MC: \MC \text{ is a small category}\}$.
			\item for $\MC,\MD \in \Obj(\tbf{Cat})$, 
			$$\Hom_{\tbf{Cat}}(\MC,\MD) = \{F : F: \MC \rightarrow \MD \}$$
			\item for $\MC,\MD, \ME \in \Obj(\tbf{Cat})$, $F \in \Hom_{\tbf{Cat}}(\MC,\MD)$ and $G \in \Hom_{\tbf{Cat}}(\MD, \ME)$, $$G \circ_{\tbf{Cat}} F = G \circ F$$
		\end{itemize}
	\end{defn}


	\begin{ex}  \lex{13010}
		We have that \tbf{Cat} is 
		\begin{enumerate}
			\item a category
			\item locally small
		\end{enumerate} 
	\end{ex}

	\begin{proof}\
		\begin{enumerate}
			\item \rex{13007} implies that composition is well defined. \rex{13008} implies that composition is associative. \rex{13005} and \rex{13006} imply the existence of identities.
			\item Let $\MC, \MD \in \Obj(\tbf{Cat})$ and $F \in \Hom_{\tbf{Cat}}(\MC, \MD)$. \rd{12002} implies that $\Obj(\MC)$, $\Obj(\MD)$, $\Hom_{\MC}$ and $\Hom_{\MD}$ are sets. Then $\Obj(\MD)^{\Obj(\MC)}$ and $\Hom_{\MD}^{ \Hom_{\MC}}$ are sets. Hence $ \Obj(\MD)^{\Obj(\MC)} \times  \Hom_{\MD}^{ \Hom_{\MC}}$ is a set. Let $F \in \Hom_{\tbf{Cat}}(\MC, \MD)$. Then there exist $F_0 \in \Obj(\MD)^{\Obj(\MC)} $ and $F_1 \in \Hom_{\MD}^{ \Hom_{\MC}}$ such that $F = (F_0, F_1)$. Therefore $F \in  \Obj(\MD)^{\Obj(\MC)} \times  \Hom_{\MD}^{ \Hom_{\MC}}$. Since $F \in \Hom_{\tbf{Cat}}(\MC, \MD)$ is arbitrary, 
			\begin{align*}
				\Hom_{\tbf{Cat}}(\MC, \MD) 
				& \subset \Obj(\MD)^{\Obj(\MC)} \times  \Hom_{\MD}^{ \Hom_{\MC}}
			\end{align*}
			which implies that $\Hom_{\tbf{Cat}}(\MC, \MD)$ is a set. Therefore, $\tbf{Cat}$ is locally small.
		\end{enumerate} 
	\end{proof}















	\subsection{Comma Categories}

	
	\begin{defn} \ld{13011}
		Let $\MA$, $\MB$, $\MC$ be a categories and $S: \MA \rightarrow \MC$, $T: \MB \rightarrow \MC$ functors. We define the \textbf{comma category of $S$ to $T$}, denoted $(S \downarrow T)$, by 
		\begin{itemize}
			\item $\Obj(S \downarrow T) = \{(A, B, h): A \in \Obj(\MA), B \in \Obj(\MB) \text{, and } h \in \Hom_{\MC}(S(A), T(B))\}$
			\item For $(A_1, B_1, h_1), (A_2, B_2, h_2) \in \Obj(S \downarrow T)$, 
			\begin{align*}
				& \Hom_{(S \downarrow T)}((A_1, B_1, h_1), (A_2, B_2, h_2)) = \\
				& \quad \quad \{(\al, \be): \al \in \Hom_{\MA}(A_1, A_2) \text{, } \be \in \Hom_{\MB}(B_1, B_2) \text{ and } T(\be) \circ_{\MC} h_1 = h_2 \circ_{\MC} S(\al)\}
			\end{align*}
			i.e. for $(A_1, B_1, h_1), (A_2, B_2, h_2) \in \Obj(S \downarrow T)$, $\al \in \Hom_{\MA}(A_1, A_2)$ and $\be \in \Hom_{\MB}(B_1, B_2)$, $(\al, \be) \in \Hom_{(S \downarrow T)}((A_1, B_1, h_1), (A_2, B_2, h_2))$ iff the following diagram commutes:
			\[ 
			\begin{tikzcd}
				S(A_1) \arrow[r, "S(\al)"] \arrow[d, "h_1"'] & S(A_2)  \arrow[d, "h_2"] \\
				T(B_1) \arrow[r, "T(\be)"']                 & T(B_2)
			\end{tikzcd}
			\]
			\item For 
			\begin{itemize}
				\item $(A_1, B_1, h_1), (A_2, B_2, h_2), (A_3, B_3, h_3) \in \Obj(S \downarrow T)$
				\item $(\al_{12}, \be_{12}) \in \Hom_{(S \downarrow T)}((A_1, B_1, h_1), (A_2, B_2, h_2))$
				\item $(\al_{23}, \be_{23}) \in \Hom_{(S \downarrow T)}((A_2, B_2, h_2), (A_3, B_3, h_3))$
			\end{itemize}
			we define 
			$$(\al_{23}, \be_{23}) \circ_{(S \downarrow T)} (\al_{12}, \be_{12}) = (\al_{23} \circ_{\MA} \al_{12}, \be_{23} \circ_{\MB} \be_{12})$$
		\end{itemize}
	\end{defn}
	
	\begin{ex} \lex{13013}
		Let $\MA$, $\MB$, $\MC$ be a categories and $S: \MA \rightarrow \MC$, $T: \MB \rightarrow \MC$ functors. Then $(S \downarrow T)$ is a category.
	\end{ex}
	
	\begin{proof}\
		\begin{itemize}
			\item \textbf{well-definedness of composition:} \\
			Let 
			\begin{itemize}
				\item $(A_1, B_1, h_1), (A_2, B_2, h_2), (A_3, B_3, h_3) \in \Obj(S \downarrow T)$
				\item $(\al_{12}, \be_{12}) \in \Hom_{(S \downarrow T)}((A_1, B_1, h_1), (A_2, B_2, h_2))$ 
				\item $(\al_{23}, \be_{23}) \in \Hom_{(S \downarrow T)}((A_2, B_2, h_2), (A_3, B_3, h_3))$
			\end{itemize}
			By definition, $\al_{12} \in \Hom_{\MA}(A_1, A_2)$, $\al_{23} \in \Hom_{\MA}(A_2, A_3)$, $\be_{12} \in \Hom_{\MB}(B_1, B_2)$, $\be_{23} \in \Hom_{\MB}(B_2, B_3)$, $T(\be_{12}) \circ_{\MC} h_1 = h_2 \circ S(\al_{12})$ and $T(\be_{23}) \circ_{\MC} h_2 = h_3 \circ_{\MC} S(\al_{23})$, \\
			i.e. the following diagram commutes:
			\[ 
			\begin{tikzcd}
				S(A_1) \arrow[r, "S(\al_{12})"] \arrow[d, "h_1"'] & S(A_2)  \arrow[d, "h_2"] \arrow[r, "S(\al_{23})"] 
				& S(A_3) \arrow[d, "h_3"]\\
				T(B_1) \arrow[r, "T(\be_{12})"']                  & T(B_2)                   \arrow[r, "T(\be_{23})"'] 
				& T(B_3) 
			\end{tikzcd}
			\]
			Then $\al_{23} \circ_{\MA} \al_{12} \in \Hom_{\MA}(A_1, A_3)$, $\be_{23} \circ_{\MB} \be_{12} \in \Hom_{\MB}(B_1, B_3)$ and 
			\begin{align*}
				T(\be_{23} \circ_{\MB} \be_{12}) \circ_{\MC} h_1
				& = (T(\be_{23}) \circ_{\MC} T(\be_{12})) \circ_{\MC} h_1 \\
				& = T(\be_{23}) \circ_{\MC} (T(\be_{12}) \circ_{\MC} h_1) \\
				& = T(\be_{23}) \circ_{\MC} (h_2 \circ_{\MC} S(\al_{12})) \\
				& = (T(\be_{23}) \circ_{\MC} h_2) \circ_{\MC} S(\al_{12}) \\
				& = (h_3 \circ_{\MC} S(\al_{23})) \circ_{\MC} S(\al_{12}) \\
				& = h_3 \circ_{\MC} (S(\al_{23}) \circ_{\MC} S(\al_{12})) \\
				& = h_3 \circ_{\MC} S(\al_{23} \circ_{\MA} \al_{12})
			\end{align*}
			i.e. the following diagram commutes:
			\[ 
			\begin{tikzcd}
				S(A_1) \arrow[rr, "S(\al_{23} \circ_{\MA} \al_{12})"] \arrow[d, "h_1"']&  & S(A_3)  \arrow[d, "h_3"] \\
				T(B_1) \arrow[rr, "T(\be_{23} \circ_{\MB} \be_{12})"']                       &  & T(B_3)
			\end{tikzcd}
			\]
			Hence $(\al_{23} \circ_{\MA} \al_{12}, \be_{23} \circ_{\MB} \be_{12}) \in \Hom_{(S \downarrow T)}((A_1, B_1, h_1), (A_3, B_3, h_3))$ and composition is well defined. \\
			\item \textbf{associativity of composition:} \\
			Let 
			\begin{itemize}
				\item $(A_1, B_1, h_1), (A_2, B_2, h_2), (A_3, B_3, h_3), (A_4, B_4, h_4) \in \Obj(S \downarrow T)$
				\item $(\al_{12}, \be_{12}) \in \Hom_{(S \downarrow T)}((A_1, B_1, h_1), (A_2, B_2, h_2))$ 
				\item $(\al_{23}, \be_{23}) \in \Hom_{(S \downarrow T)}((A_2, B_2, h_2), (A_3, B_3, h_3))$
				\item $(\al_{34}, \be_{34}) \in \Hom_{(S \downarrow T)}((A_3, B_3, h_3), (A_4, B_4, h_4))$
			\end{itemize}
			Then 
			\begin{align*}
				[ (\al_{34}, \be_{34}) \circ_{(S \downarrow T)} (\al_{23}, \be_{23}) ] \circ_{(S \downarrow T)} (\al_{12}, \be_{12})
				& = (\al_{34} \circ_{\MA} \al_{23}, \be_{34} \circ_{\MB} \be_{23}) \circ_{(S \downarrow T)} (\al_{12}, \be_{12}) \\
				& = ([\al_{34} \circ_{\MA} \al_{23}] \circ_{\MA} \al_{12}, [\be_{34} \circ_{\MB} \be_{23}] \circ_{\MB} \be_{12}) \\
				& =  (\al_{34} \circ_{\MA} [\al_{23} \circ_{\MA} \al_{12}], \be_{34} \circ_{\MB} [\be_{23} \circ_{\MB} \be_{12}]) \\
				& = (\al_{34}, \be_{34}) \circ_{(S \downarrow T)} (\al_{23} \circ_{\MA} \al_{12}, \be_{23} \circ_{\MB} \be_{12}) \\
				& = (\al_{34}, \be_{34}) \circ_{(S \downarrow T)} [(\al_{23}, \be_{23}) \circ_{(S \downarrow T)} (\al_{12}, \be_{12})]
			\end{align*}
			So composition is associative. \\
			\item \textbf{existence of identities:} \\
			Let
			\begin{itemize}
				\item $(A_1, B_1, h_1), (A_2, B_2, h_2), \in \Obj(S \downarrow T)$
				\item $(\al, \be) \in \Hom_{(S \downarrow T)}((A_1, B_1, h_1), (A_2, B_2, h_2))$ 
			\end{itemize}
			By definition, 
			\begin{itemize}
				\item $\al \in \Hom_{\MA}(A_1, A_2)$, $\be \in \Hom_{\MB}(B_1, B_2)$
				\item $h_1 \in \Hom_{\MC}(S(A_1), T(B_1))$, $h_2 \in \Hom_{\MC}(S(A_2), T(B_2))$ 
				\item $T(\be) \circ h_1 = h_2 \circ S(\al)$
			\end{itemize}
			Since $\id_{A_1} \in \Hom_{\MA}(A_1, A_1)$, $\id_{B_1} \in \Hom_{\MB}(B_1, B_1)$, and
			\begin{align*}
				T(\id_{B_1}) \circ_{\MC}  h_1 
				& = \id_{T(B_1)} \circ_{\MC} h_1 \\
				& = h_1 \\
				& = h_1 \circ_{\MC} \id_{S(A_1)} \\
				& = h_1 \circ_{\MC} S(\id_{A_1}) 
			\end{align*}
			i.e. the following diagram commutes:
			\[
			\begin{tikzcd}
				S(A_1) \arrow[r, "S(\id_{A_1})"] \arrow[d, "h_1"']  & S(A_1)  \arrow[d, "h_1"] \\
				T(B_1) \arrow[r, "T(\id_{B_1})"']                   & T(B_1)
			\end{tikzcd}
			\]
			we have that $(\id_{A_1}, \id_{B_1}) \in \Hom_{(S \downarrow T)}((A_1, B_1, h_1), (A_1, B_1, h_1))$. Similarly $(\id_{A_2}, \id_{B_2}) \in \Hom_{(S \downarrow T)}((A_2, B_2, h_2), (A_2, B_2, h_2))$. Therefore
			\begin{align*}
				(\al, \be) \circ_{(S \downarrow T)} (\id_{A_1}, \id_{B_1}) 
				& = (\al \circ_{\MA} \id_{A_1}, \be \circ_{\MB} \id_{B_1}) \\
				& = (\al, \be)
			\end{align*}
			and 
			\begin{align*}
				(\id_{A_2}, \id_{B_2}) \circ_{(S\downarrow T)} (\al, \be) 
				& = (\id_{A_2} \circ_{\MA} \al ,  \id_{B_2} \circ_{\MB} \be ) \\
				& = (\al, \be)
			\end{align*}
			Since $(A_1, B_1, h_1), (A_2, B_2, h_2), \in \Obj(S \downarrow T)$ and\\ $(\al, \be) \in \Hom_{(S \downarrow T)}((A_1, B_1, h_1), (A_2, B_2, h_2))$ are arbitrary, we have that for each $(A, B, h) \in \Obj(S \downarrow T)$, $\id_{(A, B, h)} = (\id_A, \id_B)$.
		\end{itemize}
	\end{proof}

	\begin{defn}
		Let $\MC, \MD$ be a categories, $X \in \Obj(\MD)$ and $F: \MC \rightarrow \MD$. We define the \textbf{comma category from $X$ to $F$}, denoted $(X \downarrow F)$, by $(X \downarrow F) = (\Del^{\1}_X \downarrow F)$. \\
		We may make the following identification:
		\begin{itemize}
			\item $\Obj(X \downarrow F) = \{(A, f): A \in \Obj(\MC) \text{ and } f \in \Hom_{\MD}(X, F(A))\}$ 
			\item For $(A_1, f_1), (A_2, f_2) \in \Obj(X \downarrow F)$, 
			$$\Hom_{(X \downarrow F)}((A_1, f_1), (A_2, f_2)) = \{\al \in \Hom_{\MC}(A_1, A_2) \text{ and } F(\al) \circ f_1 = f_2\}$$
			i.e. for $(A_1, f_1), (A_2, f_2) \in \Obj(X \downarrow F)$ and $\al \in \Hom_{A_1, A_2}$, $\al \in \Hom_{(X \downarrow F)}((A_1, f_1), (A_2, f_2))$ iff the following diagram commutes:
			\[ 
			\begin{tikzcd}
				&  X \arrow[dl, "f_1"']  \arrow[dr, "f_2"] & \\
				F(A_1)  \arrow[rr, "F(\al)"'] &&  F(A_2) \\ 
			\end{tikzcd}
			\]
			\item For 
			\begin{itemize}
				\item $(A_1, f_1), (A_2, f_2), (A_3, f_3) \in \Obj(X \downarrow F)$
				\item $\al \in \Hom_{(X \downarrow F)}((A_1, f_1), (A_2, f_2))$
				\item $\be \in \Hom_{(X \downarrow F)}((A_2, f_2), (A_3, f_3))$
			\end{itemize}
			we define 
			$$\be \circ_{(X \downarrow F)} \al = \be \circ_{\MC} \al $$
		\end{itemize}
	\end{defn}

	\begin{defn}
		Let $\MC, \MD$ be a categories, $X \in \Obj(\MD)$ and $F: \MC \rightarrow \MD$. We define the \textbf{comma category from $F$ to $X$}, denoted $(F \downarrow X)$, by $(F \downarrow X) = (F \downarrow \Del^{\1}_X)$.\\
		We may make the following identification:
		\begin{itemize}
			\item $\Obj(F \downarrow X) = \{(A, f): A \in \Obj(\MC) \text{ and } f \in \Hom_{\MD}(F(A), X)\}$ 
			\item For $(A_1, f_1), (A_2, f_2) \in \Obj(F \downarrow X)$, 
			$$\Hom_{(X \downarrow F)}((A_1, f_1), (A_2, f_2)) = \{\al \in \Hom_{\MC}(A_1, A_2) \text{ and } f_2 \circ F(\al)  = f_1\}$$
			i.e. for $(A_1, f_1), (A_2, f_2) \in \Obj(F \downarrow X)$ and $\al \in \Hom_{A_1, A_2}$, $\al \in \Hom_{(F \downarrow X)}((A_1, f_1), (A_2, f_2))$ iff the following diagram commutes:
			\[ 
			\begin{tikzcd}
				F(A_1) \arrow[dr, "f_1"'] \arrow[rr, "F(\al)"] &&  F(A_2) \arrow[dl, "f_2"]\\
				& X  & 
			\end{tikzcd}
			\]
			\item For 
			\begin{itemize}
				\item $(A_1, f_1), (A_2, f_2), (A_3, f_3) \in \Obj(F \downarrow X)$
				\item $\al \in \Hom_{(F \downarrow X)}((A_1, f_1), (A_2, f_2))$
				\item $\be \in \Hom_{(F \downarrow X)}((A_2, f_2), (A_3, f_3))$
			\end{itemize}
			we define 
			$$\be \circ_{(F \downarrow X)} \al = \be \circ_{\MC} \al $$
		\end{itemize}
	\end{defn}
	
	
	
	
	
	
	
	
	
	
	
	
	
	
	
	
	
	
	
	
	
	
	
	
	
	
	
	
	
	
	
	
	
	
	
	
	
	
	
	
	
	
	
	\newpage
	\section{Natural Transformations}
	
	\subsection{Introduction}
	
	\begin{defn} \ld{14001}
		Let $\MC$ and $\MD$ be categories, $F, G: \MC \rightarrow \MD$ and $ \al : \Obj(\MC) \rightarrow  \Hom_{\MD}$. Then $\al$ is said to be a \textbf{natural transformation from $F$ to $G$}, denoted $\al: F \Rightarrow G$, if
		\begin{enumerate}
			\item for each $A \in \Obj(\MC)$, $\al_A \in \Hom_{\MD}(F(A), G(A))$
			\item for each $A, B \in \Obj(\MC)$ and $f \in \Hom_{\MC}(A,B)$, $G(f) \circ \al_A = \al_B \circ F(f)$, i.e. the following diagram commutes: 
			\[ 
			\begin{tikzcd}
				F(A)  \arrow[r, "\al_A"]  \arrow[d, "F(f)"']  & G(A)   \arrow[d, "G(f)"]\\
				F(B) \arrow[r, "\al_B"] &  G(B) \\
			\end{tikzcd}
			\]
		\end{enumerate}
	\end{defn}






	
	
	
	
	
	
	
	
	
	
	\subsection{Category of Functors}
	\begin{defn} \ld{14002}
		Let $\MC$, $\MD$ be categories, $F, G, H:\MC \rightarrow \MD$ functors and $\al: F \Rightarrow G$, $\be : G \Rightarrow H$ natural transformations. We define the \textbf{composition of $\be$ with $\al$}, denoted $\be \circ \al: F \Rightarrow H$, by 
		$$(\be \circ \al)_{A} = \be_A \circ \al_A$$ 
	\end{defn}
	
	\begin{ex}  \lex{14003}
		Let $\MC$, $\MD$ be categories, $F, G, H:\MC \rightarrow \MD$ functors and $\al: F \Rightarrow G$, $\be : G \Rightarrow H$ natural transformations. Then $ \be \circ \al: F \Rightarrow H$ is a natural transformation.
	\end{ex}
	
	\begin{proof}\
		\begin{enumerate}
			\item Let $A \in \Obj(\MC)$. Since $\al_A \in \Hom_{\MD}(F(A), G(A))$ and $\be_A \in \Hom_{\MD}(G(A), H(A))$, we have that 
			\begin{align*}
				(\be \circ \al)_A 
				& = \be_A \circ \al_A \\
				& \in \Hom_{\MD}(F(A), H(A))
			\end{align*}
			\item Let $A,B \in \Obj(\MC)$ and $f \in \Hom_{\MC}(A,B)$. Since $\al:F \Rightarrow G$ and $\be: G \Rightarrow H$, $G(f) \circ \al_A = \al_B \circ F(f)$ and $H(f) \circ \be_A = \be_B \circ G(f)$. Therefore 
			\begin{align*}
				H(f) \circ (\be \circ \al)_A
				& = H(f) \circ (\be_A \circ \al_A) \\
				& = (H(f) \circ \be_A) \circ \al_A \\
				& = (\be_B \circ G(f)) \circ \al_A \\
				& = \be_B \circ (G(f)\circ \al_A) \\
				& = \be_B \circ (\al_B \circ F(f)) \\
				& = (\be_B \circ \al_B) \circ F(f) \\
				& = (\be \circ \al)_B \circ F(f)
			\end{align*}
		\end{enumerate}
		So $\be \circ \al: F \Rightarrow H$ is a natural transformation. 
	\end{proof}

	\begin{ex} \lex{14003.1}
		Let $\MC$, $\MD$ be categories, $F, G, H, I:\MC \rightarrow \MD$ functors and $\al: F \Rightarrow G$, $\be : G \Rightarrow H$ and $\gam: H \Rightarrow I$ natural transformations. Then $$(\gam \circ \be) \circ \al = \gam \circ (\be \circ \al)$$
	\end{ex}

	\begin{proof}
		Let $A \in \Obj(\MC)$. By definition,  
		\begin{align*}
			[(\gam \circ \be) \circ \al]_{A} 
			& = (\gam \circ \be)_A \circ \al_{A} \\
			& = (\gam_A \circ \be_A) \circ \al_A \\
			& = \gam_A \circ (\be_A \circ \al_A) \\
			& = \gam_A \circ (\be \circ \al)_A \\
			& = [\gam \circ (\be \circ \al)]_A \\
		\end{align*}
		Since $A \in \Obj(\MC)$ is arbitrary, 
		$$(\gam \circ \be) \circ \al = \gam \circ (\be \circ \al)$$
	\end{proof}

	\begin{defn} \ld{14003.2}
		Let $\MC$, $\MD$ be categories and $F: \MC \rightarrow \MD$. We define the \textbf{identity natural transformation from $F$ to $F$}, denoted $\id_{F}: F \Rightarrow F$, by 
		$$(\id_F)_A = \id_{F(A)}$$
	\end{defn}
	
	\begin{ex}  \lex{14003.3}
		Let $\MC$, $\MD$ be categories and $F: \MC \rightarrow \MD$. Then $\id_F: F \Rightarrow F$ is a natural transformation from $F$ to $F$.
	\end{ex}
	
	\begin{proof}\
		\begin{enumerate}
			\item Let $A \in \Obj(\MC)$. Then 
			\begin{align*}
				(\id_F)_A 
				& = \id_{F(A)} \\
				& \in \Hom_{\MD}(F(A), F(A))
			\end{align*}
			\item Let $A,B \in \Obj(\MC)$, $f \in \Hom_{\MC}(A,B)$. Then 
			\begin{align*}
				F(f) \circ (\id_{F})_{A} 
				& = F(f) \circ \id_{F(A)} \\
				& = F(f) \\
				& = \id_{F(B)} \circ F(f) \\
				& = (\id_{F})_{B} \circ F(f)
			\end{align*}
		\end{enumerate}
	\end{proof}

	\begin{ex} \lex{14003.4}
		Let $\MC$, $\MD$ be categories, $F,G: \MC \rightarrow \MD$ and $\al : F \Rightarrow G$. Then 
		\begin{enumerate}
			\item $\id_{G} \circ \al = \al$
			\item $\al \circ \id_F = \al$
		\end{enumerate} 
	\end{ex}

	\begin{proof}\
		\begin{enumerate}
			\item Let $A \in \Obj(\MC)$. Then 
			\begin{align*}
				(\id_{G} \circ \al)_{A}
				& = (\id_G)_A \circ \al_A \\
				& = \id_{G(A)} \circ \al_A \\
				& = \al_A 
			\end{align*}
			Since $A \in \Obj(C)$ is arbitrary, $\id_{G} \circ \al = \al$
			\item Let $A \in \Obj(\MC)$. Then
			\begin{align*}
				(\al \circ \id_F)_{A}
				& = \al_A \circ (\id_F)_A \\
				& = \al_A \circ \id_{F(A)} \\
				& = \al_A 
			\end{align*}
			Since $A \in \Obj(C)$ is arbitrary, $\al \circ \id_F = \al$.
		\end{enumerate}
	\end{proof}

	\begin{ex} \lex{14004}
		Let $\MC$ and $\MD$ be categories, $F,G: \MC \rightarrow \MD$ and $\al: F \Rightarrow G$. If $\MC$ is small, then $\al$ is a set.
	\end{ex}

	\begin{proof}
		Suppose that $\MC$ is small. Then $\Obj(\MC)$ is a set. Since $\al: \Obj(\MC) \rightarrow \Hom_{\MD}$, \rax{11001.1} implies that $\al(\Obj(\MC))$ is a set. Then $ \Obj(\MC) \times \al(\Obj(\MC))$ is a set. Therefore $\MP(\Obj(\MC) \times \al(\Obj(\MC)))$ is a set. Since $\al \subset \Obj(\MC) \times \al(\Obj(\MC))$, we have that $\al \in \MP(\Obj(\MC) \times \al(\Obj(\MC)))$ which implies that $\al$ is a set. 
	\end{proof}

	\begin{ex} \lex{14005}
			Let $\MC$ and $\MD$ be categories and $F,G: \MC \rightarrow \MD$. If $\MC$ is small, then there exists a class $A$ such that for each class $\al$, $\al \in A$ iff $\al: F \Rightarrow G$. 
	\end{ex}

	\begin{proof}
		Suppose that $\MC$ is small. Define $\phi$ by 
		$$\phi(\al): \al: F \Rightarrow G$$
		\rax{11002} implies that there exists a class $A$ such that for each set $\al$, $\al \in A$ iff $\phi(\al)$. 
		Let $\al$ be a class. Suppose that $\al \in A$. By \rd{11001}, $\al$ is a set. Since $\al$ is a set and $\al \in A$, we have that $\phi(\al)$. Hence $\al:F \Rightarrow G$. \\
		Conversely, suppose that $\al: F \Rightarrow G$.  Since $\MC$ is small, \rex{14004} implies that $\al$ is a set. Since $\phi(\al)$, we have that $\al \in A$. 
	\end{proof}
	
	\begin{defn} \ld{14006}
		Let $\MC$ and $\MD$ be categories. Suppose that $\MC$ is small. We define the \textbf{functor category from $\MC$ to $\MD$}, denoted $\MD^{\MC}$, by 
		\begin{itemize}
			\item $\Obj(\MD^{\MC}) = \{F: F: \MC \rightarrow \MD\}$
			\item For $F, G \in \Obj(\MD^{\MC})$, $\Hom_{\MD^{\MC}}(F, G) = \{\al: \al: F \Rightarrow G \}$
			\item For $F, G, H \in \Obj(\MD^{\MC})$, $\al \in \Hom_{\MD^{\MC}}(F, G)$ and $\be \in \Hom_{\MD^{\MC}}(G, H)$, 
			$\be \circ_{\MD^{\MC}} \al = \be \circ \al $
		\end{itemize} 
	\end{defn}

	\begin{ex} \lex{14007}
		Let $\MC$ and $\MD$ be categories. Suppose that $\MC$ is small. Then $\MD^{\MC}$ is a category.
	\end{ex}

	\begin{proof}\
		\rex{14003} implies that composition is well-defined. \rex{14003.1} implies that composition is associative. \rex{14003.3} and \rex{14003.4} imply the existence of identities.  
	\end{proof}


	
	
	
	
	
	
	
	\subsection{Diagonal Functor}

	\begin{defn} \ld{14008}
		Let $\MC$, $\MD$ be categories, $X, Y \in \Obj(\MD)$ and $f \in \Hom_{\MD}(X, Y)$. We define the \textbf{constant natural transformation on $\MC$ at $f$}, denoted $\del^{\MC}_f: \Del^{\MC}_X \Rightarrow \Del^{\MC}_Y$, by
		$$(\del^{\MC}_f)_A = f$$
	\end{defn}
	
	\begin{ex} \lex{14009}
		Let $\MC$, $\MD$ be categories, $X, Y \in \Obj(\MD)$ and $f \in \Hom_{\MD}(X, Y)$. Then $\del^{\MC}_f: \Del^{\MC}_X \Rightarrow \Del^{\MC}_Y $ is a natural transformation.
	\end{ex}
	
	\begin{proof} \
		\begin{enumerate}
			\item By definition, for each $A \in \Obj(\MC)$ $(\del^{\MC}_f)_A \in \Hom_{\MD}(\Del^{\MC}_X(A), \Del^{\MC}_Y(A))$.
			\item Let $A, B \in \Obj(\MC)$ and $g \in \Hom_{\MC}(A, B)$. Then 
			\begin{align*}
				\Del^{\MC}_Y(g) \circ (\del^{\MC}_f)_A 
				& = \id_Y \circ f \\
				& = f \\
				& = f \circ \id_X \\
				& = (\del^{\MC}_f)_B \circ \Del^{\MC}_X(g)  
			\end{align*}
			i.e. the following diagram commutes:
			\[ 
			\begin{tikzcd}[baseline= 7]
				\Del^{\MC}_X(A)  \arrow[r, "(\del^{\MC}_f)_A"]  \arrow[d, "\Del^{\MC}_X(g)"']  & \Del^{\MC}_Y(A)   \arrow[d, "\Del^{\MC}_Y(g)"]\\
				\Del^{\MC}_X(B) \arrow[r, "(\del^{\MC}_f)_B"] &  \Del^{\MC}_Y(B) \\
			\end{tikzcd}
			=
			\begin{tikzcd}[baseline= 7]
				X  \arrow[r, "f"]  \arrow[d, "\id_X"']  & Y   \arrow[d, "\id_Y"]\\
				X \arrow[r, "f"] &  Y \\
			\end{tikzcd}
			\]
		\end{enumerate}
		So $\del^{\MC}_f: \Del^{\MC}_X \Rightarrow \Del^{\MC}_Y$ is a natural transformation.
	\end{proof}
	 
	\begin{ex} \lex{14010}
		Let $\MC, \MD$ be categories, $X, Y, Z \in \Obj(\MD)$, $f \in \Hom_{\MD}(X,Y)$ and $g \in \Hom_{\MD}(Y,Z)$. Then $\del^{\MC}_{g \circ f} = \del^{\MC}_g \circ \del^{\MC}_f$.
	\end{ex}

	\begin{proof}
		Let $A \in \Obj(\MC)$. Then 
		\begin{align*}
			(\del^{\MC}_{g \circ f})_A
			& = g \circ f \\
			& = (\del^{\MC}_g)_A \circ (\del^{\MC}_f)_A \\
			& = (\del^{\MC}_g \circ \del^{\MC}_f)_A
		\end{align*}
		Since $A \in \Obj(\MC)$ is arbitrary, $\del^{\MC}_{g \circ f} = \del^{\MC}_g \circ \del^{\MC}_f$.
	\end{proof}

	\begin{ex} \lex{14011}
		Let $\MC, \MD$ be categories and $X \in \Obj(\MD)$. Then $\del^{\MC}_{\id_X} = \id_{\Del^{\MC}_X}$.
	\end{ex}

	\begin{proof}
		Let $A \in \Obj(\MC)$. Then 
		\begin{align*}
			(\del^{\MC}_{\id_X})_A
			& = \id_X \\
			& = \id_{\Del^{\MC}_X(A)} \\
			& = (\id_{\Del^{\MC}_X})_A 
		\end{align*}
		Since $A \in \Obj(\MC)$ is arbitrary, $\del^{\MC}_{\id_X} = \id_{\Del^{\MC}_X}$
	\end{proof}

	\begin{defn} \ld{14011}
		Let $\MC$, $\MD$ be categories. Suppose that $\MC$ is small. We define the \textbf{$\MC$-ary diagonal functor} on $\MD$, denoted by $\Del^{\MC}: \MD \rightarrow \MD^{\MC}$, by
		\begin{itemize}
			\item $\Del^{\MC}(X) = \Del^{\MC}_X$
			\item $\Del^{\MC}(f) = \del^{\MC}_f$
		\end{itemize}
	\end{defn}

	\begin{ex} \lex{14012}
		Let $\MC$, $\MD$ be categories. Suppose that $\MC$ is small. Then $\Del^{\MC}: \MD \rightarrow \MD^{\MC}$ is a functor.
	\end{ex}

	\begin{proof}\
		\begin{enumerate}
			\item \rex{14009} implies that for each $X, Y \in \Obj(\MD)$ and $f \in \Hom_{\MD}(X, Y)$, $\Del^{\MC}(f) \in \Hom_{\MD^{\MC}} (\Del^{\MC}(X), \Del^{\MC}(Y))$
			\item \rex{14010} implies that for each $X, Y, Z \in \Obj(\MD)$, $f \in \Hom_{\MD}(X, Y)$ and $g \in \Hom_{\MD}(Y, Z)$, $\Del^{\MC}(g \circ f) = \Del^{\MC}(g) \circ \Del^{\MC}(f)$
			\item \rex{14011} implies that for each $X \in \Obj(\MD)$, $\Del^{\MC}(\id_X) = \id_{\Del^{\MC}(X)}$
		\end{enumerate}
		 So $\Del^{\MC} : \MD \rightarrow \MD^{\MC}$ is a functor.
	\end{proof}


	
	
	
	
	
	
	
	
	
	
	
	
	
	
	
	
	
	
	
	
	
	
	
	
	
	
	
	
	
	
	
	
	
	
	
	
	
	
	
	
	
	
	
	
	
	
	
	
	
	
	\newpage
	\section{Algebra of Morphisms}
	
	\subsection{Isomorphisms}
	
	\begin{ex} \lex{15001} \textbf{Uniqueness of Identities:} \\
		Let $\MC$ be a category. Then for each $A \in \Obj(\MC)$, there exists a unique $e_A \in \Hom_{\MC}(A, A)$ such that for each $B \in \Obj(\MC)$, $f \in \Hom_{\MC}(A, B)$ and $g \in \Hom_{\MC}(B, A)$, $f \circ e_A = f$ and $e_{A} \circ g = g$.
	\end{ex}
	
	\begin{proof}
		Let $A \in \Obj(\MC)$. 
		\begin{itemize}
			\item \tbf{Existence:} \\
			Since $\MC$ is a category, by definition there exists $\id_A \in \Hom_{\MC}(A, A)$ such that for each $B \in \Obj(\MC)$, $f \in \Hom_{\MC}(A, B)$ and $g \in \Hom_{\MC}(B, A)$, $f \circ \id_A = f$ and $\id_{A} \circ g = g$. 
			\item \tbf{Uniqueness: } \\
			Let $e_A \in \Hom_{\MC}(A, A)$. Suppose that for each $B \in \Obj(\MC)$, $f \in \Hom_{\MC}(A, B)$ and $g \in \Hom_{\MC}(B, A)$, $f \circ e_A = f$ and $e_{A} \circ g = g$. Then 
			\begin{align*}
				e_A
				& = e_A \circ \id_A \\
				& = \id_A 
			\end{align*}
		\end{itemize}
	\end{proof}
	
	\begin{defn} \ld{15002}
		Let $\MC$ be a category, $A,B \in \Obj(\MC)$ and $f \in \Hom_{\MC}(A, B)$. Then $f$ is said to be an \textbf{isomorphism} if there exists $g \in \Hom_{\MC}(B, A)$ such that $g \circ f = \id_{A}$ and $f \circ g = \id_{B}$. 
	\end{defn}
	
	\begin{ex} \lex{15003} \textbf{Uniqueness of Inverses:} \\
		Let $\MC$ be a category, $A,B \in \Obj(\MC)$ and $f \in \Hom_{\MC}(A, B)$. If $f$ is an isomorphism, then there exists a unique $g \in \Hom_{\MC}(B, A)$ such that $g \circ f = \id_A$ and $f \circ g = \id_B$.
	\end{ex}
	
	\begin{proof}
		Suppose that $f$ is an isomorphism. 
		\begin{itemize}
			\item \tbf{Existence:} \\
			By definition, there exists $g \in \Hom_{\MC}(B, A)$ such that $g \circ f = \id_{A}$ and $f \circ g = \id_B$.
			\item \tbf{Uniqueness: } \\
			Let $g' \in \Hom_{\MC}(B, A)$. Suppose that $g' \circ f = \id_A$, $f \circ g' = \id_B$. Then 
			\begin{align*}
				g'
				& = g' \circ \id_{B} \\
				& = g' \circ (f \circ g) \\
				& = (g' \circ f) \circ  g \\
				& = \id_A \circ g \\
				& = g
			\end{align*}
		\end{itemize}
	\end{proof}

	\begin{defn} \ld{15003.1}
		Let $\MC$ be a category, $A,B \in \Obj(\MC)$ and $f \in \Hom_{\MC}(A, B)$. Suppose that $f$ is an isomorphism. We define the \textbf{inverse of $f$}, denoted $f^{-1}$, to be the unique $g \in \Hom_{\MC}(B, A)$ such that $g \circ f = \id_A$ and $f \circ g = \id_B$.
	\end{defn}
	
	\begin{ex} \lex{15004}
		Let $\MC$ be a category and $A \in \Obj(\MC)$. Then $\id_A$ is an isomorphism and $(\id_A)^{-1} = \id_A$. 
	\end{ex}
	
	\begin{proof}
		Since $\id_A \circ \id_A = \id_A$, we have that $\id_A$ is an isomorphism and $(\id_A)^{-1} = \id_A$.
	\end{proof}
	
	\begin{ex} \lex{15005}
		Let $\MC$ be a category and $A, B \in \Obj(\MC)$ and $f \in \Hom_{\MC}(A,B)$. If $f$ is an isomorphism, then $f^{-1}$ is an isomorphism and $(f^{-1})^{-1} = f$.
	\end{ex}
	
	\begin{proof}
		Suppose that $f$ is an isomorphism. By definition, $f \circ f^{-1} = \id_{B}$ and $f^{-1} \circ f = \id_A$. Hence $f^{-1}$ is an isomorphism and $(f^{-1})^{-1} = f$.
	\end{proof}
	
	\begin{ex} \lex{15006}
		Let $\MC$ be a category, $A,B,C \in \Obj(\MC)$, $f \in \Hom_{\MC}(A, B)$ and $g \in \Hom_{\MC}(B, C)$. If $f$ and $g$ are isomorphisms, then $g \circ f$ is an isomorphism and $(g \circ f)^{-1} = f^{-1} \circ g^{-1}$. 
	\end{ex}
	
	\begin{proof}
		Suppose that $f$ and $g$ are isomorphisms. Then 
		\begin{align*}
			(f^{-1} \circ g^{-1}) \circ (g \circ f) 
			& = ((f^{-1} \circ g^{-1}) \circ g) \circ f \\
			& = (f^{-1} \circ (g^{-1} \circ g)) \circ f \\
			& = (f^{-1} \circ \id_B) \circ f \\
			& = f^{-1} \circ f \\
			& = \id_{A}
		\end{align*}
		and 
		\begin{align*}
			(g \circ f) \circ (f^{-1} \circ g^{-1}) 
			& = ((g \circ f) \circ f^{-1}) \circ g^{-1} \\
			& = (g \circ (f \circ f^{-1})) \circ g^{-1} \\
			& = (g \circ \id_{B}) \circ g^{-1} \\
			& = g \circ  g^{-1} \\
			& = \id_{C}
		\end{align*}
		So $g \circ f$ is an isomorphism and $(g \circ f)^{-1} = f^{-1} \circ g^{-1}$.
	\end{proof}
	
	\begin{defn} \ld{15007}
		Let $\MC$ be a category and $A,B \in \Obj(\MC)$. Then $A$ is said to be \textbf{isomorphic} to $B$ if there exists $f \in \Hom_{\MC}(A, B)$ such that $f$ is an isomorphism. 
	\end{defn}
	
	\begin{ex} \lex{15008}
		Let $\MC$ be a category. We define the relation $\cong$ on $\Obj(\MC)$ by $A \cong B$ iff $A$ is isomorphic to $B$. Then $\cong$ is an equivalence relation on $\Obj(\MC)$.
	\end{ex}
	
	\begin{proof} \
		\begin{enumerate}
			\item \textbf{reflexivity: } \\
			Let $A \in \Obj(\MC)$. \rex{15004} implies that $\id_A$ is an isomorphism. So $A \cong A$. Since $A \in \Obj(\MC)$ is arbitrary, we have that for each $A \in \Obj(\MC)$, $A \cong A$ and thus $\cong$ is reflexive. 
			\item \textbf{symmetry: } \\
			Let $A, B \in \Obj(\MC)$. Suppose that $A \cong B$. Then there exists $f \in \Hom_{\MC}(A,B)$ such that $f$ is an isomorphism. \rex{15005} implies that $f^{-1}$ is an isomorphism. Since $f^{-1} \in \Hom_{\MC}(B, A)$, $B \cong A$.  Since $A, B \in \Obj(\MC)$ are arbitrary, we have that for each $A, B \in \Obj(\MC)$, $A \cong B$ implies that $B \cong A$ and thus $\cong$ is reflexive. 
			\item \textbf{transitivity: }
			Let $A, B, C \in \Obj(\MC)$. Suppose that $A \cong B$ and $B \cong C$. Then there exist $f \in \Hom_{\MC}(A,B)$ and $g \in \Hom_{\MC}(B, C)$ such that that $f$ and $g$ are isomorphisms. \rex{15006} implies that $g \circ f$ is an isomorphism. Since $g \circ f \in \Hom_{\MC}(A, C)$, $A \cong C$. Since $A, B, C \in \Obj(\MC)$ are arbitrary, we have that for each $A, B, C \in \Obj(\MC)$, $A \cong B$ and $B \cong C$ implies that $A \cong C$ and thus $\cong$ is transitive. 
		\end{enumerate}
		Since $\cong$ is reflexive, symmetric and transitive, $\cong$ is an equivalence relation on $\Obj(\MC)$.
	\end{proof}
	
	\begin{defn} \ld{15009}
		Let $\MC$ be a category, $A,B \in \Obj(\MC)$ and $f:A \rightarrow B$. Then 
		\begin{itemize}
			\item 
			$f$ is said to be a \textbf{monomorphism} if for each $C \in \Obj(C)$ and $g,h \in \Hom_{\MC}(C, A)$, $f \circ g = f \circ h$ implies that $g = h$, i.e. we have the following implication of commutative diagrams: 
			\[ 
			\begin{tikzcd}[baseline= 7]
				C \arrow[r, "g"] \arrow[d, "h"'] & A \arrow[d, "f"] \\
				A \arrow[r, "f"'] & B\\
			\end{tikzcd}
			\implies
			\begin{tikzcd}[baseline= 7]
				C \arrow[bend right=60, swap]{r}{h} \arrow[bend right=-60]{r}{g} & A  \\
			\end{tikzcd}
			\]
			\item 
			$f$ is said to be an \textbf{epimorphism} if for each $C \in \Obj(C)$ and $g,h \in \Hom_{\MC}(B, C)$, $g \circ f = h \circ f$ implies that $g = h$, i.e. we have the following implication of commutative diagrams: \\
			\[ 
			\begin{tikzcd}[baseline= 7]
				A \arrow[r, "f"] \arrow[d, "f"'] & B \arrow[d, "g"] \\
				B \arrow[r, "h"'] & C\\
			\end{tikzcd}
			\implies
			\begin{tikzcd}[baseline= 7]
				B \arrow[bend right=60, swap]{r}{h} \arrow[bend right=-60]{r}{g} & C  \\
			\end{tikzcd}
			\]
		\end{itemize}
	\end{defn}
	
	\begin{ex} \lex{15010}
		Let $A, B \in \Obj(\Set)$ and $f \in \Hom_{\Set}(A,B)$. Then 
		\begin{enumerate}
			\item $f$ is a monomorphism iff $f$ is injective
			\item $f$ is an epimorphism iff $f$ is surjective 
		\end{enumerate} 
		\tbf{Hint: } consider $C = \{0\}$ and $C = \{0,1\}$.
	\end{ex}
	
	\begin{proof}\
		\begin{enumerate}
			\item Suppose that $f$ is injective. Let $C \in \Obj(\Set)$ and $g,h \in \Hom_{\Set}(C, A)$. Suppose that $f \circ g = f \circ h$. Let $x \in C$. Then $f(g(x)) = f(h(x))$. Injectivity of $f$ implies that $g(x) = h(x)$. Since $x \in C$ is arbitrary, $g = h$. Hence $f$ is a monomorphism. \\
			Conversely, suppose that $f$ is a monomorphism. Let $a, b \in A$. Suppose that $f(a) = f(b)$. Set $C = \{0\}$ and define $g,h: C \rightarrow A$ by $g(0) = a$ and $h(0) = b$. Then  
			\begin{align*}
				f \circ g(0) 
				& = f(g(0)) \\
				& = f(a) \\
				& = f(b) \\
				& = f(h(0)) \\
				& = f \circ h(0) 
			\end{align*}
			Therefore $f \circ g = f \circ h$. Since $f$ is a monomorphism, we have that $g = h$. Hence 
			\begin{align*}
				a 
				& = g(0) \\
				& = h(0) \\
				& = b
			\end{align*}
			\item Suppose that $f$ is surjective. Let $C \in \Obj(\MC)$ and $g,h \in \Hom_{\Set}(B, C)$. Suppose that $g \circ f = h \circ f$. Let $y \in B$. Surjective of $f$ implies that there exists $x \in A$ such that $y = f(x)$. 
			Then 
			\begin{align*}
				g(y)
				& = g(f(x)) \\
				& = g \circ f (x) \\
				& = h \circ f (x) \\
				& = h(f(x)) \\
				& = h(y) 
			\end{align*}
			Since $y \in B$ is arbitrary, $g = h$. Hence $f$ is an epimorphism. \\
			Conversely, suppose that $f$ is an epimorphism. Set $C = \{0,1\}$ and define $g, h: B \rightarrow C$ by $g = \chi_{f(A)}$ and $h = \chi_{B}$. Then $g \circ f = h \circ f$. Since $f$ is an epimorphism, $g = h$ and $f(A) = B$. Hence $f$ is surjective.  
		\end{enumerate}
	\end{proof}
	
	\begin{ex} \lex{15011}
		Let $\MC$ be a category, $A,B \in \Obj(\MC)$ and $f \in \Hom_{\MC}(A, B)$. If $f$ is an isomorphism, then $f$ is a monomorphism and $f$ is an epimorphism. 
	\end{ex}
	
	\begin{proof}
		Suppose that $f$ is an isomorphism. 
		\begin{itemize}
			\item (monomorphism) \\
			Let $C \in \Obj(\MC)$ and $g, h \in \Hom_{\MC}(C, A)$. Suppose that $f \circ g = f \circ h$. Then
			\begin{align*}
				g 
				& = \id_A \circ g \\
				& = (f^{-1} \circ f) \circ g \\
				& = f^{-1} \circ (f \circ g) \\
				& = f^{-1} \circ (f \circ h) \\
				& = (f^{-1} \circ f) \circ h \\
				& = \id_A \circ h \\
				& = h
			\end{align*}
			So $f$ is a monomorphism.
			\item  (epimorphism) \\
			Let $C \in \Obj(\MC)$ and $g, h \in \Hom_{\MC}(B, C)$. Suppose that $ g \circ f = h \circ f$. Then
			\begin{align*}
				g 
				& = g \circ \id_B \\
				& = g \circ (f \circ f^{-1}) \\
				& = (g \circ f) \circ f^{-1} \\
				& = (h \circ f) \circ f^{-1} \\
				& = h \circ (f \circ f^{-1}) \\
				& = h \circ \id_B \\
				& = h
			\end{align*}
			So $f$ is an epimorphism.
		\end{itemize}
	\end{proof}
	
	\begin{defn} \ld{15011.1}
		Let $\MC$ and $\MD$ be categories, $F,G: \MC \rightarrow \MD$ and $\al : F \Rightarrow G$. Then $\al$ is said to be a \textbf{natural isomorphism} if for each $A \in \Obj(\MC)$, $\al_A$ is an isomorphism.
	\end{defn}

	\begin{defn} \ld{15011.2}
		Let $\MC$ and $\MD$ be categories, $F,G: \MC \rightarrow \MD$ and $\al : F \Rightarrow G$. Suppose that $\al$ is a natural isomorphism. We define $\al^{-1}: G \Rightarrow F$ by $(\al^{-1})_A = \al_A^{-1}$. 
	\end{defn}

	\begin{ex} \lex{15011.3}
		Let $\MC$ and $\MD$ be categories, $F,G: \MC \rightarrow \MD$ and $\al : F \Rightarrow G$. Suppose that $\al$ is a natural isomorphism. Then $\al^{-1}: G \Rightarrow F$ is a natural transformation. 
	\end{ex}
	
	\begin{proof}\
		\begin{enumerate}
			\item Let $A \in \Obj(\MC)$. Since $\al_A \in \Hom_{\MD}(F(A), G(A))$, we have that 
			\begin{align*}
				(\al^{-1})_A
				& = \al_A^{-1} \\
				& \in \Hom_{\MD}(G(A), F(A))
			\end{align*}
			\item Let $A, B \in \Obj(\MC)$ and $f \in \Hom_{\MC}(A,B)$. Since $G(f) \circ \al_A = \al_B \circ F(f)$, i.e. the following diagram commutes:
			\[ 
			\begin{tikzcd}
				F(A)  \arrow[r, "\al_A"]  \arrow[d, "F(f)"']  & G(A)   \arrow[d, "G(f)"]\\
				F(B) \arrow[r, "\al_B"] &  G(B) \\
			\end{tikzcd}
			\]
			we have that 
			\begin{align*}
				F(f) \circ (\al^{-1})_A 
				& = F(f) \circ \al_A^{-1} \\
				& = \id_{F(B)} \circ (F(f) \circ \al_A^{-1}) \\
				& = (\al_B^{-1} \circ \al_B) \circ (F(f) \circ \al_A^{-1}) \\
				& = \al_B^{-1} \circ (\al_B \circ (F(f) \circ \al_A^{-1})) \\
				& = \al_B^{-1} \circ ((\al_B \circ F(f)) \circ \al_A^{-1}) \\
				& = \al_B^{-1} \circ ((G(f) \circ \al_A) \circ \al_A^{-1}) \\
				& = \al_B^{-1} \circ (G(f) \circ (\al_A \circ \al_A^{-1})) \\
				& = \al_B^{-1} \circ (G(f) \circ \id_{G(A)}) \\
				& = \al_B^{-1} \circ G(f) \\
				& = (\al^{-1})_B \circ G(f) 
 			\end{align*}
 			i.e. the following diagram commutes:
 			\[ 
 			\begin{tikzcd}
 				G(A)  \arrow[r, "(\al^{-1})_A"]  \arrow[d, "G(f)"']  & F(A)   \arrow[d, "F(f)"]\\
 				G(B) \arrow[r, "(\al^{-1})_B"] &  F(B) \\
 			\end{tikzcd}
 			\]
 			So $\al^{-1}: G \Rightarrow F$.
		\end{enumerate}
	\end{proof}

	\begin{ex} \lex{15011.4}
		Let $\MC$ and $\MD$ be categories, $F,G: \MC \rightarrow \MD$ and $\al : F \Rightarrow G$. Suppose that $\al$ is a natural isomorphism. Then $\al^{-1} \circ \al = \id_{F}$ and $\al \circ \al^{-1} = \id_G$.
	\end{ex}

	\begin{proof}
		Let $A \in \Obj(\MC)$. Then 
		\begin{align*}
			(\al^{-1} \circ \al )_A 
			& = (\al^{-1})_A \circ \al_A \\
			& = \al_A^{-1} \circ \al_A \\
			& = \id_{F(A)} \\
			& = (\id_F)_A
		\end{align*}
		and 
			\begin{align*}
			(\al \circ \al^{-1} )_A 
			& = \al_A \circ (\al^{-1})_A\\
			& = \al_A \circ \al_A^{-1}\\
			& = \id_{G(A)} \\
			& = (\id_G)_A
		\end{align*}
		Since $A \in \Obj(\MC)$ is arbitrary, $\al^{-1} \circ \al = \id_F$ and $\al \circ \al^{-1} = \id_G$.
	\end{proof}

	\begin{ex} \lex{15011.5}
		Let $\MC$ and $\MD$ be categories. Suppose that $\MC$ is small. Let $F,G \in \MD^{\MC}$ and $\al \in \Hom_{\MD^{\MC}}(F, G)$. Then $\al$ is an isomorphism iff $\al$ is a natural isomorphism.  
	\end{ex}

	\begin{proof}
		Suppose that $\al$ is an isomorphism. Then there exists $\be \in \Hom_{\MD^{\MC}}(G, F)$ such that $\be \circ \al = \id_F$ and $\al \circ \be - \id_G$. Let $A \in \Obj(\MC)$. Then
		\begin{align*}
			\be_A \circ \al_A 
			& = (\be \circ \al)_A \\
			& = (\id_F)_A \\
			& = \id_{F(A)}
		\end{align*}
		and 
		\begin{align*}
			\al_A \circ \be_A
			& = (\al \circ \be)_A \\
			& = (\id_G)_A \\
			& = \id_{G(A)}
		\end{align*}
		Hence $\al_A$ is an isomorphism. Since $A \in \Obj(\MC)$ is arbitrary, $\al$ is a natural isomorphism. \\
		Conversely, suppose that $\al$ is a natural isomorphism. \rex{15011.3} and \rex{15011.4} imply that $\al$ is an isomorphism.
	\end{proof}






















	\subsection{Initial and Final Objects}
	
	\begin{defn} \ld{15012}
		Let $\MC$ be a category and $0 \in \Obj(\MC)$. Then $0$ is said to be \textbf{initial} if for each $A \in \Obj(\MC)$, there exists $f \in \Hom_{\MC}(0, A)$ such that $\Hom_{\MC}(0, A) = \{f\}$. 
	\end{defn}

	\begin{defn} \ld{15012.1}
		Let $\MC$ be a category and $1 \in \Obj(\MC)$. Then $1$ is said to be \textbf{final} if for each $A \in \Obj(\MC)$, there exists $f \in \Hom_{\MC}(A, 1)$ such that $\Hom_{\MC}(A, 1) = \{f\}$. 
	\end{defn}
	
	\begin{ex} \lex{15013}
		Let $\MC$ be a category and $0 \in \Obj(\MC)$. If $0$ is initial, then $\Hom_{\MC}(0, 0) = \{\id_0\}$.
	\end{ex}
	
	\begin{proof}
		Suppose that $0$ is initial. Then there exists a $f \in \Hom_{\MC}(0,0)$ such that $\Hom_{\MC}(0, 0) = \{f\}$. Since $\id_0 \in \Hom_{\MC}(0,0)$, $f = \id_0$ and therefore $\Hom_{\MC}(0, 0) = \{\id_0\}$.
	\end{proof}

		\begin{ex} \lex{15013.1}
		Let $\MC$ be a category and $1 \in \Obj(\MC)$. If $1$ is final, then $\Hom_{\MC}(1, 1) = \{\id_1\}$.
	\end{ex}
	
	\begin{proof}
		Similar to \rex{15013}
	\end{proof}
	
	\begin{ex} \lex{15014}
		Let $\MC$ be a category and $0, 0' \in \Obj(\MC)$. If $0$ and $0'$ are initial, then $0$ and $0'$ are isomorphic.
	\end{ex}
	
	\begin{proof}
		Suppose that $0$ and $0'$ are initial. By definition, there exist $f \in \Hom_{\MC}(0, 0')$ and $f' \in \Hom_{\MC}(0', 0)$ such that $\Hom_{\MC}(0, 0') = \{f\}$ and $\Hom_{\MC}(0', 0) = \{f'\}$, i.e. we have the following commutative diagram:
		\[ 
		\begin{tikzcd}[baseline= 7]
			\arrow[loop left, "f' \circ f"] 0 \arrow[r, bend right= 60, "f", swap] & 0' \arrow[l, bend left=-60, "f'", swap] \arrow[loop right, "f \circ f'"] \\
		\end{tikzcd}
		\]
		\rex{15013} implies that $f' \circ f = \id_{0}$ and $f \circ f' = \id_{0'}$. Hence $f$ is an isomorphism. Since $f \in \Hom_{\MC}(0, 0')$, we have that $0 \cong 0'$.
	\end{proof}

	\begin{ex} \lex{15014.1}
		Let $\MC$ be a category and $1, 1' \in \Obj(\MC)$. If $1$ and $1'$ are final, then $1$ and $1'$ are isomorphic.
	\end{ex}
	
	\begin{proof}
		Similar to \rex{15014}
	\end{proof}
	
	\begin{ex} \lex{15015}
		We have that $\varnothing$ is initial in $\Set$. 
	\end{ex}
	
	\begin{proof}
		Let $A \in \Obj(\Set)$. Define $f \in \Hom_{\Set}(\varnothing, A)$ by $f = \varnothing$. Let $g \in \Hom_{\Set}(\varnothing, A)$. Then $g = f$. Since $g \in \Hom_{\Set}(\varnothing, A)$ is arbitrary, $\Hom_{\Set}(\varnothing, A) = \{f\}$. Hence $\varnothing$ is initial.
	\end{proof}

	\begin{ex} \lex{15015.1}
		We have that $ \{\varnothing\}$  is terminal in $\Set$. 
	\end{ex}
	
	\begin{proof}
		Let $A \in \Obj(\Set)$. Define $f \in \Hom_{\Set}(A, \{\varnothing\})$ by $f(x) = \varnothing$. Let $g \in \Hom_{\Set}(A, \{\varnothing\})$. Then $g = f$. Since $g \in \Hom_{\Set}(A, \{\varnothing\})$ is arbitrary, $\Hom_{\Set}(A, \{\varnothing\}) = \{f\}$. Hence $\{\varnothing\}$ is final.
	\end{proof}
	
	\begin{ex} \lex{15016}
		We have that $\0$ is initial in $\Cat$. 
	\end{ex}

	\begin{proof}
		Let $\MC \in \Obj(\Cat)$. It is clear that $\Hom_{\Cat}(\0, \MC) = \{E_\MC\}$. Hence $\0$ is initial in $\Cat$.
	\end{proof}

	\begin{ex} \lex{15016.1}
		We have that $\1$ is final in $\Cat$. 
	\end{ex}
	
	\begin{proof}
		Let $\MC \in \Obj(\Cat)$. It is clear that $\Hom_{\Cat}(\MC, \1) = \{ \Del^{\MC}_* \}$. Hence $\1$ is final in $\Cat$.
	\end{proof}

	\begin{defn} \ld{15017}
		Let $\MC$, $\MD$ be categories and $0 \in \Obj(\MD)$ and $F: \MC \rightarrow \MD$. Suppose that $0$ is initial in $\MD$. Then for each $A \in \Obj(\MC)$, there exists $f_A \in \Hom_{\MD}(0, F(A))$ such that $\Hom_{\MD}(0, F(A)) = \{f_A\}$. We define the \textbf{initial natural transformation induced by $0$} from $\Del^{\MC}_0$ to $F$, denoted $\zeta_{0}: \Del^{\MC}_0 \Rightarrow F$, by 
		$(\eta_0)_A = f_A$.
	\end{defn}

	\begin{defn} \ld{15017.1}
		Let $\MC$, $\MD$ be categories and $1 \in \Obj(\MD)$ and $F: \MC \rightarrow \MD$. Suppose that $1$ is final in $\MD$. Then for each $A \in \Obj(\MC)$, there exists $f_A \in \Hom_{\MD}(F(A), 1)$ such that $\Hom_{\MD}(F(A), 1) = \{f_A\}$. We define the \textbf{final natural transformation induced by $1$} from $F$ to $\Del^{\MC}_1$, denoted $\phi_{1}: F \Rightarrow \Del^{\MC}_1$, by 
		$(\phi_{1})_A = f_A$.
	\end{defn}

	\begin{ex} \lex{15018}
		Let $\MC$, $\MD$ be categories and $0 \in \Obj(\MD)$ and $F: \MC \rightarrow \MD$. Suppose that $0$ is initial in $\MD$. Then $\eta_0: \Del^{\MC}_0 \Rightarrow F$ is a natural transformation.
	\end{ex}

	\begin{proof} \
		\begin{enumerate}
			\item By definition, for each $A \in \Obj(\MC)$, $(\eta_0)_A \in \Hom_{\MD}(\Del^{\MC}_0(A), F(A))$
			\item Let $A, B \in \Obj(\MC)$ and $f \in \Hom_{\MC}(A, B)$. Since 
			\begin{align*}
				F(f) \circ (\eta_0)_A 
				& \in \Hom_{\MD}(0, F(B)) \\
				& = \{(\eta_0)_B\}
			\end{align*}
			we have that 
			\begin{align*}
				F(f) \circ (\eta_0)_A 
				& = (\eta_0)_B \\
				& = (\eta_0)_B \circ \id_0 
			\end{align*}
			i.e. the following diagram commutes:
			\[ 
			\begin{tikzcd}[baseline= 7]
				\Del^{\MC}_0(A)  \arrow[r, "(\eta_0)_A"]  \arrow[d, "\Del^{\MC}_0(f)"']  & F(A)   \arrow[d, "F(f)"]\\
				\Del^{\MC}_0(B) \arrow[r, "(\eta_0)_B"] &  F(B) \\
			\end{tikzcd}
			=
			\begin{tikzcd}[baseline= 7]
				0  \arrow[r, "(\eta_0)_A"]  \arrow[d, "\id_0"']  & F(A)   \arrow[d, "F(f)"]\\
				0 \arrow[r, "(\eta_0)_B"] &  F(B) \\
			\end{tikzcd}
			\]
		\end{enumerate}
		So $\eta_0: \Del^{\MC}_0 \Rightarrow F$ is a natural transformation.
	\end{proof}
	
	\begin{ex} \lex{15019}
		Let $\MC$, $\MD$ be categories and $1 \in \Obj(\MD)$ and $F: \MC \rightarrow \MD$. Suppose that $1$ is final in $\MD$. Then $\phi_{1}: F \Rightarrow \Del^{\MC}_0$ is a natural transformation.
	\end{ex}

	\begin{proof}
		Similar to \rex{15018}
	\end{proof}
	
	\begin{ex} \lex{15020}
		Let $\MC$, $\MD$ be categories and $0 \in \Obj(\MD)$. Suppose that $\MC$ is small. If $0$ is initial in $\MD$, then $\Del^{\MC}_0$ is initial in $\MD^{\MC}$. 
	\end{ex}

	\begin{proof}
		Suppose that $0$ is initial in $\MD$. Let $F \in \Obj(\MD^{\MC})$, $\al \in \Hom_{\MD^{\MC}}(\Del^{\MC}_0, F)$ and $A \in \Obj(\MC)$. Then
		\begin{align*}
			\al_A 
			& \in \Hom_{\MD}(\Del^{\MC}_0(A), F(A)) \\
			& =  \Hom_{\MD}(0, F(A)) \\
			& = \{(\eta_0)_A\}
		\end{align*}
		Hence $\al_A = (\eta_0)_A$. Since $A \in \Obj(\MC)$ is arbitrary, $\al = \eta_0$. Since $\al \in  \Hom_{\MD^{\MC}}(\Del^{\MC}_0, F)$ is arbitrary, $ \Hom_{\MD^{\MC}}(\Del^{\MC}_0, F) = \{\eta_0\}$. Therefore $\Del^{\MC}_0$ is initial in $\MD^{\MC}$.
	\end{proof}


	\begin{ex} \lex{15021}
		Let $\MC$, $\MD$ be categories and $1 \in \Obj(\MD)$. Suppose that $\MC$ is small. If $1$ is final in $\MD$, then $\Del^{\MC}_1$ is final in $\MD^{\MC}$. 
	\end{ex}

	\begin{proof}
		Similar to \rex{15020}.
	\end{proof}
	
















































	
	
	
	
	
	
	
	
	
	
	
	
	
	
	
	
	
	
	
	
	
	
	
	
	
	
	
	
	
	
	
	
	
	
	
	
	
	
	
	
	
	
	
	
	
	
	
	
	
	
	\newpage
	\chapter{Universal Morphisms and Limits}
	
	\subsection{Universal Morphisms}

	\begin{defn}
		Let $\MC, \MD$ be a categories, $X \in \Obj(\MD)$, $F: \MC \rightarrow \MD$, $A \in \Obj(\MC)$ and $f \in \Hom_{\MD}(X, F(A))$. Then $(A, f)$ is said to be a \textbf{universal morphism} from $X$ to $F$ if for each $A' \in \Obj(\MC)$ $f' \in \Hom_{\MD}(X, F(A'))$, there exists a unique $\al \in \Hom_{\MC}(A, A')$ such that $f' = F(\al) \circ f$, i.e. the following diagram commutes:
		\[
		\begin{tikzcd}
			X \arrow[r, "f"]  \arrow[dr, "f'"'] & F(A) \arrow[d, "F(\al)", dashed] \\
			                                    & F(A')  
		\end{tikzcd}
		\hspace{.3cm}
		\begin{tikzcd}
			A  \arrow[d, "\al", dashed] \\
			A'  
		\end{tikzcd}
		\]
	\end{defn}

	\begin{defn}
		Let $\MC, \MD$ be a categories, $X \in \Obj(\MD)$, $F: \MC \rightarrow \MD$, $A \in \Obj(\MC)$ and $f \in \Hom_{\MD}(F(A), X)$. Then $(A, f)$ is said to be a \textbf{universal morphism} from $F$ to $X$ if for each $A' \in \Obj(\MC)$ $f' \in \Hom_{\MD}(F(A'), X)$, there exists a unique $\al \in \Hom_{\MC}(A', A)$ such that $f' = f \circ F(\al)$, i.e. the following diagram commutes:
		\[
		\begin{tikzcd}
			X   & F(A) \arrow[l, "f"'] \\
			& F(A')  \arrow[ul, "f'"] \arrow[u, "F(\al)"', dashed]  
		\end{tikzcd}
		\hspace{.3cm}
		\begin{tikzcd}
			A \\
			A'  \arrow[u, "\al"', dashed]   
		\end{tikzcd}
		\]
	\end{defn}

	\begin{ex}
		Let $\MC, \MD$ be a categories, $X \in \Obj(\MD)$, $F: \MC \rightarrow \MD$, $A \in \Obj(\MC)$ and $f \in \Hom_{\MD}(X, F(A))$. Then $(A, f)$ is a universal morphism from $X$ to $F$ iff $(A, f)$ is initial in $(X \downarrow F)$.  
	\end{ex}

	\begin{proof}
		
	\end{proof}

	\begin{ex}
		Let $\MC, \MD$ be a categories, $X \in \Obj(\MD)$, $F: \MC \rightarrow \MD$ $A \in \Obj(\MC)$ and $f \in \Hom_{\MD}(F(A), X)$. Then $(A, f)$ is a universal morphism from $F$ to $X$ iff $(A, f)$ is terminal in $(F \downarrow X)$.  
	\end{ex}

	\begin{proof}
		
	\end{proof}


















	
	
	\newpage
	\section{Limits}
	
	\begin{defn}
		Let $\MJ$, $\MC$ be categories and $D: \MJ \rightarrow \MC$. Then $D$ is said to be a \textbf{diagram of type $\MJ$ in $\MC$}.
	\end{defn}

	\begin{note}
		We are usually interested in the case that $\MJ$ is small. We will identify a diagram $D$ with its image. 
	\end{note}

	\begin{exmp}
		Define $\MJ$ by
		\begin{itemize}
			\item  $\Obj(\MJ) = \{1, 2, 3\}$ and for $i,j \in \Obj(\MJ)$, $\Hom_{\MJ}(i, j) = \{a_{i,j}\}$,
			\item for $i,j \in \Obj(\MJ)$, $\Hom_{\MJ}(i,j) = \{a_{ij}\}$.
		\end{itemize}
		Let $\MC$ be a category and $D: \MJ \rightarrow \MC$. Without including the identity morphisms or compositions, we can visualize $D$ as follows:
		\[
		\begin{tikzcd}
			&     1  \arrow[dl, "a"'] \arrow[r, "b"]        &  2 \arrow[dl, "c"] \\
			3 \arrow[r, "d"'] &        4   & \\
		\end{tikzcd}
		\overset{D}{\longrightarrow}
		\begin{tikzcd}
			&     D_1  \arrow[dl] \arrow[r]        &  D2 \arrow[dl] \\
			D_3 \arrow[r] &        D_4   & \\
		\end{tikzcd}
		\]
	\end{exmp}

	\begin{defn}
		Let $\MJ$, $\MC$ be categories. Suppose that $\MJ$ is small. Let $D \in \Obj(\MC^{\MJ})$. We define the \textbf{category of cones to $D$}, denoted $\Cone(D)$, by $\Cone(D) = (\Del^{\MJ} \downarrow D)$.
	\end{defn}


	\begin{exmp}
		Let $\MJ$
		\[
		\begin{tikzcd}
			&  &  X \arrow[ddd, "\phi_1"'] \arrow[ddddll, "\phi_3"']  \arrow[dddrrr, "\phi_2"], \arrow[ddddr, "\phi_4"] &       &    \\
			&       &     &       &    \\
			&       &     &       &    \\
			&       &  D_1  \arrow[dll] \arrow[rrr]    &       & &  D2 \arrow[dll] \\
			D_3 \arrow[rrr] &       &     &  D_4  &    \\
		\end{tikzcd}
		\]
	\end{exmp}
	

	\begin{defn}
		Let $\MJ$, $\MC$ be categories. Suppose that $\MJ$ is small. Let $D \in \Obj(\MC^{\MJ})$. We define the \textbf{category of cocones from $D$}, denoted $\Cocone(D)$, by $\Cocone(D) = (D \downarrow \Del^{\MJ})$.
	\end{defn}

	\begin{defn}
		Let $\MJ$, $\MC$ be categories. Suppose that $\MJ$ is small. Let $D \in \Obj(\MC^{\MJ})$ and $(X, \phi) \in \Cone(D)$. Then $(X, \phi)$ is said to be a \textbf{limit of $D$} if $(X, \phi)$ is a universal morphism from $\Del^{\MJ}$ to $D$.  
	\end{defn}



	\begin{note}
		Let $\MJ$, $\MC$ be categories. Suppose that $\MJ$ is small. Let $D \in \Obj(\MC^{\MJ})$ and $(X, \phi) \in \Cone(D)$. Then 
		\begin{align*}
			\text{$(X, \phi)$  is a limit of $D$} 
			& \iff \text{$(X, \phi)$ is terminal in $\Cone(D)$} \\
			& \iff \text{for each $(Y, \psi) \in \Cone(D)$, there exists a unique} \\
			& \hspace{1.3cm }\text{$f \in \Hom_{\MC}(Y, X)$ such that for each $j \in \MJ$, $\psi_j = \phi_j \circ f$}
		\end{align*} 
	\end{note} 


	\begin{defn}
		Let $\MJ$, $\MC$ be categories. Suppose that $\MJ$ is small. Let $D \in \Obj(\MC^{\MJ})$ and $(X, \phi) \in \Cocone(D)$. Then $(X, \phi)$ is said to be a \textbf{colimit of $D$} if $(X, \phi)$ is a universal morphism from $D$ to $\Del^{\MJ}$.  
	\end{defn}

	\begin{note}
		Let $\MJ$, $\MC$ be categories. Suppose that $\MJ$ is small. Let $D \in \Obj(\MC^{\MJ})$ and $(X, \phi) \in \Cone(D)$. Then 
		\begin{align*}
			\text{$(X, \phi)$  is a colimit of $D$} 
			& \iff \text{$(X, \phi)$ is initial in $\Cocone(D)$} \\
			& \iff \text{for each $(Y, \psi) \in \Cocone(D)$, there exists a unique} \\
			& \hspace{1.3cm }\text{$f \in \Hom_{\MC}(X, Y)$ such that for each $j \in \MJ$, $\psi_j = f \circ \phi_j $}
		\end{align*} 
	\end{note} 


	
	
	
	
	
	
	
	
	
	
	
	
	
	
	
	
	
	
	\subsection{Products and Coproducts}
	
	
	
	
	
	
	
	
	
	
	
	
	
	
	
	\subsection{Equalizers and Coequalizers}
	
	

	\section{TO DO}
	\begin{itemize}
		\item Define subcategories and full subcategories and show that if $\Obj(D) \subset \Obj(C)$ and for each $X, Y \in \Obj(D)$, $\Hom_{D}(X, Y) = \Hom_{C}(X,Y)$, then $D$ is a full subcategory of $C$. I used this in differential 
		\item 
	\end{itemize}
	
	
	
	
	
	
	
	
	
	
	
	
	

	
	
	
	
	
	
	
	
	
	
	
	
	
	
	
	
	\chapter{Monoidal Categories}
	
	\begin{defn}
		
	\end{defn}
	
	
	
	
	
	
	
	
	
	
	
	
	
	
	
	
	
	
	
	
	
	
	
	
	
	
	
	
	
	
	
	
	\appendix
	\chapter{App}
	
	\section{Reading Diagrams and associated digraphs of diagrams}
	
	\begin{defn}
		Let 
		\[ 
		\begin{tikzcd}[baseline= 7]
			C \arrow[r, "g"] \arrow[d, "h"'] & A \arrow[d, "f"] \\
			A \arrow[r, "f"'] & B\\
		\end{tikzcd}
		\implies
		\begin{tikzcd}[baseline= 7]
			C \arrow[bend right=60, swap]{r}{h} \arrow[bend right=-60]{r}{g} & A  \\
		\end{tikzcd}
		\]
		
		see an intro to the language of category theory by roman for description
	\end{defn}
	
	
	\begin{defn}
		A diagram is said to be \tbf{commutative} if for each path of length $\geq 2$, in the associated digraph gives the same morphism.  
	\end{defn}
	
	
	
	\backmatter
	
	
	
	
	
	
	
	
	
	
	
	
	
	
\end{document}