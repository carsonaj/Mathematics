\documentclass[12pt]{amsart}
\usepackage[margin=1in]{geometry} 
\usepackage{amsmath,amsthm,amssymb,setspace, mathtools}

\usepackage{color}   %May be necessary if you want to color links
\usepackage{hyperref}
\hypersetup{
	colorlinks=true, %set true if you want colored links
	linktoc=all,     %set to all if you want both sections and subsections linked
	linkcolor=black,  %choose some color if you want links to stand out
	urlcolor=cyan
}


%
%
%
\newif\ifhideproofs
%\hideproofstrue %uncomment to hide proofs
%
%
%
%
\ifhideproofs
\usepackage{environ}
\NewEnviron{hide}{}
\let\proof\hide
\let\endproof\endhide
\fi

\theoremstyle{definition}
\newtheorem{definition}{Definition}[subsection]
\newtheorem{defn}[definition]{Definition}
\newtheorem{note}[definition]{Note}
\newtheorem{thm}[definition]{Theorem}
\newtheorem{lem}[definition]{Lemma}
\newtheorem{prop}[definition]{Proposition}
\newtheorem{cor}[definition]{Corollary}
\newtheorem{conj}[definition]{Conjecture}
\newtheorem{ex}[definition]{Exercise}



\DeclareMathOperator{\supp}{supp}

\newcommand{\p}{\partial}

\newcommand{\al}{\alpha}
\newcommand{\Gam}{\Gamma}
\newcommand{\bet}{\beta} 
\newcommand{\del}{\delta} 
\newcommand{\Del}{\Delta}
\newcommand{\lam}{\lambda}  
\newcommand{\Lam}{\Lambda} 
\newcommand{\ep}{\epsilon}
\newcommand{\sig}{\sigma} 
\newcommand{\om}{\omega}
\newcommand{\Om}{\Omega}
\newcommand{\C}{\mathbb{C}}
\newcommand{\N}{\mathbb{N}}
\newcommand{\E}{\mathbb{E}}
\newcommand{\Z}{\mathbb{Z}}
\newcommand{\R}{\mathbb{R}}
\newcommand{\T}{\mathbb{T}}
\newcommand{\Q}{\mathbb{Q}}
\renewcommand{\P}{\mathbb{P}}
\newcommand{\MA}{\mathcal{A}}
\newcommand{\MC}{\mathcal{C}}
\newcommand{\MB}{\mathcal{B}}
\newcommand{\MF}{\mathcal{F}}
\newcommand{\MG}{\mathcal{G}}
\newcommand{\ML}{\mathcal{L}}
\newcommand{\MN}{\mathcal{N}}
\newcommand{\MS}{\mathcal{S}}
\newcommand{\MP}{\mathcal{P}}
\newcommand{\ME}{\mathcal{E}}
\newcommand{\MT}{\mathcal{T}}
\newcommand{\MM}{\mathcal{M}}
\newcommand{\MI}{\mathcal{I}}

\newcommand{\io}{\text{ i.o.}}
\newcommand{\ev}{\text{ ev.}}
\renewcommand{\r}{\rangle}
\renewcommand{\l}{\langle}

\newcommand{\RG}{[0,\infty]}
\newcommand{\Rg}{[0,\infty)}
\newcommand{\Ll}{L^1_{\text{loc}}(\R^n)}

\newcommand{\limfn}{\liminf \limits_{n \rightarrow \infty}}
\newcommand{\limpn}{\limsup \limits_{n \rightarrow \infty}}
\newcommand{\limn}{\lim \limits_{n \rightarrow \infty}}
\newcommand{\convt}[1]{\xrightarrow{\text{#1}}}
\newcommand{\conv}[1]{\xrightarrow{#1}} 
\newcommand{\seq}[2]{(#1_{#2})_{#2 \in \N}}

\newcommand{\loc}{\text{loc}}

\DeclareMathOperator{\sgn}{sgn}
\DeclareMathOperator{\spn}{span}
\DeclareMathOperator{\Img}{Im}

\newcommand{\Hom}{\text{Hom}}



\newcommand{\lex}[1]{\label{ex:#1}}
\newcommand{\ld}[1]{\label{defn:#1}}
\newcommand{\rex}[1]{Exercise \ref{ex:#1}}
\newcommand{\rd}[1]{Definition \ref{defn:#1}}



\begin{document}
	
	\title{Introduction to Algebra}
	\author{Carson James}
	\maketitle
	
	\tableofcontents
	
	\newpage
	
	
	\section{Groups}
	
	\subsection{Direct Products}
	
	\begin{defn}
	Let $G,H$ be groups. Define a product $*:(G \times H) \times (G \times H) \rightarrow G \times H$ by 
	$$(x_1,y_1) * (x_2, y_2) = (x_1x_2, y_1y_2)$$
	Then $(G \times H, *)$ is called the \textbf{direct product of $G$ and $H$}.
	\end{defn}	
	
	\begin{ex}
	\lex{1} Let $G,H$ be groups. Then the direct product $G \times H$ is a group.
	\end{ex}
	\begin{proof}
	Clear.
	\end{proof}
	
	\begin{defn} 
	Let $G,H$ be groups. Define $\pi_G :G \times H \rightarrow G$ and $\pi_H :G \times H \rightarrow H$ by $\pi_G(x,y) = x$ and $\pi_H(x,y) = y$.  Then $\pi_G$ and $\pi_H$ are respectively called the \textbf{projection maps onto $G$ and $H$}.
	\end{defn}	
	
	\begin{ex}
	\lex{2} Let $G,H$ be groups. Then 
	\begin{enumerate}
	\item $\pi_G: G \times H \rightarrow G$ and $\pi_H : G \times H \rightarrow H$ are homomorphisms
	\item $\ker \pi_G \cong H$ and $\ker \pi_H \cong G$
\end{enumerate}	 
	\end{ex}
	
	\begin{proof}\
	\begin{enumerate}
	\item Clear
	\item Define $\iota_G:G \rightarrow \ker \pi_H$ by $$\iota_G(x) = (x, e_H)$$ Then $\iota_G$ is an isomorphism. Similarly, we can define $\iota_H:H \rightarrow \ker \pi_G$ and show that it is an isomorphism.
	\end{enumerate}
	\end{proof}
	
	\begin{defn}
	Let $G,H, K$ be groups, $\phi \in \Hom(G,K)$ and $\psi \in \Hom(H, K)$. We define $\phi \times \psi: G \times H \rightarrow K$ by $\phi \times \psi(x,y) = \phi(x) \psi(y)$ 
	\end{defn}	
	
	\begin{ex}
	\lex{3} Let $G,H, K$ be groups, $\phi \in \Hom(G,K)$ and $\psi \in \Hom(H, K)$. If $K$ is abelian, then $\phi \times \psi \in Hom(G \times H,K)$.
	\end{ex}
	
	\begin{proof}
	Let $x_1, x_2 \in G$ and $y_1, y_2 \in H$. Then 
	\begin{align*}
	\phi \times \psi[(x_1, y_1)(x_2, y_2)] 
	&= \phi \times \psi (x_1x_2, y_1y_2) \\
	&= \phi(x_1x_2) \psi(y_1y_2) \\
	&= \phi(x_1)\phi(x_2)\psi(y_1)\psi(y_2) \\
	&= \phi(x_1)\psi(y_1)\phi(x_2)\psi(y_2) \\
	&= [\phi \times \psi(x_1, y_1)] [\phi \times \psi(x_2, y_2) ]
	\end{align*}
	\end{proof}
	
	\begin{ex}
	\lex{4} Let $G,H, K$ be groups and $\phi \in \Hom(G \times H, K)$. Then there exist $\phi_G \in \Hom(G,K)$, $\phi_H \in \Hom(H, K)$ such that $\phi_G \times \phi_H = \phi$.
	\end{ex}
	
	\begin{proof}
	Suppose that $K$ is abelian. Define $\iota_G \in \Hom(G, \ker \pi_H)$ and $\iota_H \in \Hom(H, \ker \pi_G)$ as in part $(2)$ of \rex{2} Define $\phi_G \in \Hom(G, K)$ and $\phi_H \in \Hom(H,K)$ by  $\phi_G = \phi \circ \iota_G$ and $\phi_H = \phi \circ \iota_H $. Let $(x,y) \in G \times H$. Then 
	\begin{align*}
	\phi_G \times \phi_H(x,y) 
	&= \phi_G(x) \phi_H(y) \\
	&= \phi \circ \iota_G(x) \phi \circ \iota_H(y) \\
	&= \phi(x, e_H)\phi(e_G, y) \\
	&= \phi(x,y) \\
	\end{align*}
	So $\phi = \phi_G \times \phi_H$
	\end{proof}
	
	
	
	
	
	
	
	
	
	
	
	
	
	
	
	
	\newpage
	\section{Rings}
	
	\begin{defn}
	Let $R$ be a set and $+, *: R \times R 				
	\rightarrow R$ (we write $a+b$ and 
	$ab$ in place of $+(a,b)$ and $*(a,b)$ respectively).
	Then $R$ is said to be a \textbf{ring} if for each 
	$a,b,c \in R$,
	\begin{enumerate}
	\item $R$ is an abelian group with respect to $+$.
	 The identity element with respect to $+$ is denoted
	 by $0$.
	\item $R$ is a monoid with respect to $*$. The  
	identity element of $R$ with respect to $*$ is denoted $1$. 
	\item $R$ is commutative with respect to $*$.
	\item $*$ distributes over $+$.
	\end{enumerate}
	\end{defn}
	
	\begin{defn}
	Let $R$ be a ring and $I \subset R$. Then $I$ is said 
	to be an \textbf{ideal} of $R$ if for each $a \in R$ and $x,y \in I$,
	\begin{enumerate}
	\item  $x + y \in I$
	\item  $ax \in I$
	\end{enumerate}
	\end{defn}
	
	\begin{defn}
	Let $R$ be a ring and $A,B \subset R$. We define the \textbf{product} of $A$ and $B$, denoted $AB$, to be $$AB = \bigg \{\sum_{i=1}^n a_ib_i: a_i \in A, b_i \in B, n \in \N \bigg \}$$
	\end{defn}	
	
	\begin{ex}
	Let $R$ be a ring and $I \subset R$. Then $I$ is an ideal of $R$ iff $RI \subset I$. 
	\end{ex}
	
	\begin{proof}
	Suppose that $RI \subset I$. Let $a \in R$ and $x,y \in I$. Then by assumption $x + y = 1x + 1y \in I$ and $ax \in I$. So $I$ is an ideal of $R$\\
	Conversely, suppose that $I$ is an ideal of $R$. Let $a_1, \cdots, a_n \in R$ and $x_1, \cdots, x_n \in I$. Then by assumption, for each $i = 1, \cdots, n$, $a_ix_i \in I$ and therefore $\sum\limits_{i=1}^n a_ib_i \in I$. Hence $RI \subset I$.
	\end{proof}
	
	
	
	
	
	
	
	
	
	
	
	
	
	
	
	
	
	
	
	
	
	
	
	\newpage	
	\section{Modules}
	
	\subsection{Introduction}
	
	\begin{defn}
	Let $R$ be a ring, $M$ a set, $+: M\times M \rightarrow M$ and $*: R 
	\times M \rightarrow M$ (we write $rx$ in place of 
	$*(r,x)$). Then $M$ is said to be an 
	\textbf{$R$-module}
	if 
	\begin{enumerate}
	\item $M$ is an abelian group with respect to $+$. The identity element of $M$ with respect to $+$ is denoted by 0.
	\item for each $r \in R$, $*(r, \cdot)$ is a group endomorphism of $M$
	\item for each $x \in M$, $*(\cdot, x)$ is a group homomorphism from $R$ to $M$
	\item $*$ is a monoid action of $R$ on $M$
	\end{enumerate}
	\end{defn}
	
	\begin{note}
	For the remainder of this section, we assume that $R$ is a commutative ring. 
	\end{note}
	
	\begin{ex}
	Let $M$ be an $R$-module. Then for each $r \in R$ and $x \in M$, 
	\begin{enumerate}
	\item $r0 = 0$
	\item $0x = 0$
	\item $(-1)x = -x$
	\end{enumerate}
	\end{ex}
	
	\begin{proof} Let $r \in R$ and $x \in M$. Then 
	\begin{enumerate}
	\item 
	\begin{align*}
	r0 
	&= r(0+0) \\
	&= r0 + r0
	\end{align*} 
	which implies that $r0 = 0$.
	\item 
	\begin{align*}
	0x 
	&= (0+0)x \\
	&= 0x + 0x
	\end{align*} 
	which implies that $0x = 0$.
	\item 
	\begin{align*}
	(-1)x + x 
	&= (-1)x + 1x \\ 
	&= (-1 + 1)x \\
	&= 0x \\
	&= 0
	\end{align*}
	which implies that $(-1)x = -x$.
	\end{enumerate}
	\end{proof}
	
	\begin{defn}
	Let $M$ an $R$-module and $N \subset M$. Then $N$ is said to be a \textbf{submodule} of $M$ if for each $r \in R$ and $x,y \in N$, we have that $rx \in N$ and $x+y \in N$.
	\end{defn}
	
	\begin{defn}
	Let $M$ be an $R$-module. We define $\MS(M) = \{N \subset M: N \text{ is a submodule of }M\}$.
	\end{defn}	
	
	\begin{ex}
	Let $M$ be an $R$-module and $N \in \MS(M)$. Then $N$ is a subgroup of $M$.
	\end{ex}
	
	\begin{proof}
	Let $x,y \in M$. Then $x-y = 1x + (-1)y \in N$. So $N$ is a subgroup of $M$.
	\end{proof}
	
	\begin{defn}
	Let $M$ be an $R$-module and $N \in \MS(M)$. We define  
	\begin{enumerate}
	\item $M/N = \{x + N: x \in M\}$ 
	\item $+: M/N \times M/N \rightarrow M/N$ by $$(x+N) + (y+N) = (x+y) + N$$
	\item $*: R \times M/N \rightarrow M/N$ by $$r(x+N) = (rx) + N$$
	\end{enumerate}
	Under these operations (see next exercise), $M/N$ is an $R$-module known as the \textbf{quotient module} of $M$ by $N$.
	\end{defn}	
	
	\begin{ex} 
	Let $M$ be an $R$-module and $N \in \MS(M)$. Then
	\begin{enumerate}
	\item the monoid action defined above is well defined
	\item the quotient $M/N$ is an $R$-module
	\end{enumerate}
	\end{ex}
	
	\begin{proof}\
	\begin{enumerate}
	\item Let $r \in R$ and $x +N, y +N \in M/N $. Recall from group theory that $x + N = y + N$ iff $x-y \in N$. Suppose that $x + N = y + N$. Then $x - y \in N$ and there exists $n \in N$ such that $x-y = n$. Therefore
	\begin{align*}
	rx - ry 
	&= r(x-y) \\
	&= rn \\
	&\in N
	\end{align*}
	So $rx + N = ry + N$.
	\item Properties $(1)$ - $(4)$ in the definition of a module are easily shown to be satisfied for $M/N$ since they are true for $M$.
	\end{enumerate}
	\end{proof}
	
	\begin{defn}
	Let $M$ and $N$ be $R$-modules and $\phi:M \rightarrow N$. Then $\phi$ is said to be a \textbf{module homomorphism} if for each $r \in R$ and $x,y \in M$
	\begin{enumerate}
	\item $\phi(rx) = r\phi(x)$
	\item $\phi(x+y) = \phi(x) + \phi(y)$
	\end{enumerate}
	\end{defn}	
	
	\begin{ex}
	Let $M$ and $N$ be $R$-modules and $\phi:M \rightarrow N$. Then $\phi$ is a  iff for each $r \in R$ and $x,y \in M$, $\phi(x+ry) = \phi(x) + r \phi(y)$.
	\end{ex}
	
	\begin{proof}
	Clear.
	\end{proof}
	
	\begin{ex}
	Let $M$ and $N$ be $R$-modules and $\phi:M \rightarrow N$ a homomorphism. Then 
	\begin{enumerate}
	\item $\ker \phi$ is a submodule of $M$
	\item $ \Img \phi$ is a submodule of $N$ 
	\end{enumerate}
	\end{ex}
	
	\begin{proof}
	Let $r \in R$, $x,y \in \ker \phi$ and $w,z \in \Img \phi$. Then 
	\begin{enumerate}
	\item 
	\begin{align*}
	\phi(rx) 
	&= r\phi(x) \\
	&=r 0 \\
	&= 0
\end{align*}	
	So $rx \in \ker \phi$. Group theory tells us that $\ker \phi$ is a subgroup of $M$, so $x+y \in \ker \phi$. Hence $\ker \phi$ is a submodule of $M$. 
	\item Similar.
	\end{enumerate}
	\end{proof}
	
	\begin{defn}
	Let $M$ be an $R$-module and $A \subset M$. We define the \textbf{submodule of $M$ generated by $A$}, denoted $\spn(A)$, to be $$\spn(A) = \bigcap_{N \in \MS(M)} N$$ 
	\end{defn}
	
	\begin{ex}
	Let $M$ be an $R$-module and $A \subset M$. Then 
	$\spn(A) \in \MS(M)$
	\end{ex}
	
	\begin{proof}
	Let $r \in R$ and $x,y \in \spn(A)$. Basic group theory tells us that $\spn(A)$ is a subgroup of $M$. So $x+y \in \spn(A)$. For $N \in \MS(M)$, by definition we have $x \in N$ and therefore $rx \in N$. So $rx \in \spn(A)$. Hence $\spn(A)$ is a submodule of $M$.
	\end{proof}
	
	\begin{ex}
	Let $M$ be an $R$-module and $A \subset M$. If $A \neq \varnothing$, then $$\spn(A) = \bigg \{\sum\limits_{i=1}^n r_ia_i: r_i \in R, a_i \in A, n \in \N \bigg \}$$
	\end{ex}
	
	\begin{proof}
	Clearly 
	\end{proof}
	
	\begin{defn}
	Let $M$
	\end{defn}
	
	
	
	
	
	
	
	
	
	
	
	
	
	
	
	
	
	
	
	
	\newpage
	\section{Fields}
	
	
	
	
	
	
	
	
	
	
	
	
	
	
	
	
	
	
	
	
	
	
	
	
	
	
	
	
	\newpage
	\section{Vector Spaces}
	
	\section{Appendix}
	\subsection{Monoids}
	
	\begin{defn}
	Let $G$ be a set and $*: G \times G \rightarrow G$ (we write $ab$ in place of $*(a,b)$). Then 
	\begin{enumerate}
	\item $*$ is called a \textbf{binary operation} on $G$	
	\item $*$ is said to be \textbf{associative}	if for each $x,y,z \in G$, $(xy)z = x(yz)$
	\item $*$ is said to be \textbf{commutative} if for each $x,y \in G$, $xy = yx$ 
	\end{enumerate}
	\end{defn}
	
	\begin{defn}
	Let $G$ be a set, $*: G \times G \rightarrow G$, $e,x,y \in G$. Then $e$ is said to be an \textbf{identity element} if for each $x \in G$, $ex = xe = x$.
	\end{defn}
	
	\begin{defn}
	Let $G$ be a set and $*: G \times G \rightarrow G$. Then $G$ is said to be a \textbf{monoid} if 
	\begin{enumerate}
	\item $*$ is associative
	\item there exits $e \in G$ such that $e$ is an identity element.
	\end{enumerate}
	\end{defn}	
	
	\begin{ex}
	Let $G$ be a monoid. Then the identity element is unique.
	\end{ex}
	
	\begin{proof}
	Let $e, f \in G$. Suppose that $e$ and $f$ are identity elements. Then $e = ef = f$.
	\end{proof}
	
	\begin{note}
	Unless otherwise specified, we will denote the identity element of a monoid by $e$.
	\end{note}
	
	\begin{defn}
	Let $G$ be a monoid, $X$ a set and $*: G \times X \rightarrow X$ (we write $gx$ in place of $*(g,x)$). Then $*$ is said to be a \textbf{monoid action} of $G$ on $X$ if for each $g,h \in G$ and $x \in X$,
	\begin{enumerate}
	\item $(gh)x = g(hx)$
	\item $ex = x$
	\end{enumerate}
	\end{defn}
	
	
	
	
	
	
	
	
	
	
	
	
	
	
	
	
	
	
	
	
	
	
	
\end{document}