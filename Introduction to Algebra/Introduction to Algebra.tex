%% filename: amsbook-template.tex
%% version: 1.1
%% date: 2014/07/24
%%
%% American Mathematical Society
%% Technical Support
%% Publications Technical Group
%% 201 Charles Street
%% Providence, RI 02904
%% USA
%% tel: (401) 455-4080
%%      (800) 321-4267 (USA and Canada only)
%% fax: (401) 331-3842
%% email: tech-support@ams.org
%% 
%% Copyright 2006, 2008-2010, 2014 American Mathematical Society.
%% 
%% This work may be distributed and/or modified under the
%% conditions of the LaTeX Project Public License, either version 1.3c
%% of this license or (at your option) any later version.
%% The latest version of this license is in
%%   http://www.latex-project.org/lppl.txt
%% and version 1.3c or later is part of all distributions of LaTeX
%% version 2005/12/01 or later.
%% 
%% This work has the LPPL maintenance status `maintained'.
%% 
%% The Current Maintainer of this work is the American Mathematical
%% Society.
%%
%% ====================================================================

%    AMS-LaTeX v.2 driver file template for use with amsbook
%
%    Remove any commented or uncommented macros you do not use.

\documentclass{book}

%    For use when working on individual chapters
%\includeonly{}

%    For use when working on individual chapters
%\includeonly{}

%    Include referenced packages here.
\usepackage[margin=1in]{geometry} 
\usepackage{amsmath}
\usepackage{amsthm}
\usepackage{amssymb}
\usepackage{setspace}
\usepackage{mathtools}
\usepackage{tikz}  
\usepackage{tikz-cd}
\usepackage{tkz-fct}
\usepackage{pgfplots}
\usepackage{environ}
\usepackage{tikz-cd} 
\usepackage{enumitem}
\usepackage{color}   %May be necessary if you want to color links
\usepackage{hyperref}
\hypersetup{
	colorlinks=true, %set true if you want colored links
	linktoc=all,     %set to all if you want both sections and subsections linked
	linkcolor=black,  %choose some color if you want links to stand out
	urlcolor=cyan
}
\usepackage[symbols,nogroupskip,sort=none]{glossaries-extra}

\pgfplotsset{every axis/.append style={
		axis x line=middle,    % put the x axis in the middle
		axis y line=middle,    % put the y axis in the middle
		axis line style={<->,color=black}, % arrows on the axis
		xlabel={$x$},          % default put x on x-axis
		ylabel={$y$},          % default put y on y-axis
}}


\theoremstyle{definition}
\newtheorem{definition}{Definition}[subsection]
\newtheorem{defn}[definition]{Definition}
\newtheorem{note}[definition]{Note}
\newtheorem{ax}[definition]{Axiom}
\newtheorem{thm}[definition]{Theorem}
\newtheorem{lem}[definition]{Lemma}
\newtheorem{prop}[definition]{Proposition}
\newtheorem{cor}[definition]{Corollary}
\newtheorem{conj}[definition]{Conjecture}
\newtheorem{ex}[definition]{Exercise}
\newtheorem{exmp}[definition]{Example}

\setcounter{tocdepth}{3}

% hide proofs
\newif\ifhideproofs
%\hideproofstrue %uncomment to hide proofs
\ifhideproofs
\NewEnviron{hide}{}
\let\proof\hide
\let\endproof\endhide
\fi

% lower-case greek
\newcommand{\al}{\alpha}
\newcommand{\be}{\beta}
\newcommand{\gam}{\gamma}
\newcommand{\del}{\delta}
\newcommand{\ep}{\epsilon}
\newcommand{\ze}{\zeta} 
\newcommand{\kap}{\kappa} 
\newcommand{\lam}{\lambda}  
\newcommand{\sig}{\sigma} 
\newcommand{\omi}{\omicron}
\newcommand{\up}{\upsilon}
\newcommand{\om}{\omega}

% upper-case greek
\newcommand{\Gam}{\Gamma}
\newcommand{\Del}{\Delta}
\newcommand{\Lam}{\Lambda} 
\newcommand{\Sig}{\Sigma} 
\newcommand{\Om}{\Omega}

% blackboard bold
\newcommand{\C}{\mathbb{C}}
\newcommand{\E}{\mathbb{E}}
\newcommand{\F}{\mathbb{F}}
\renewcommand{\H}{\mathbb{H}}
\newcommand{\K}{\mathbb{K}}
\newcommand{\N}{\mathbb{N}}
\renewcommand{\O}{\mathbb{O}}
\newcommand{\Q}{\mathbb{Q}}
\newcommand{\R}{\mathbb{R}}
\newcommand{\T}{\mathbb{T}}
\newcommand{\V}{\mathbb{V}}
\newcommand{\Z}{\mathbb{Z}}

% math caligraphic
\newcommand{\MA}{\mathcal{A}}
\newcommand{\MB}{\mathcal{B}}
\newcommand{\MC}{\mathcal{C}}
\newcommand{\MD}{\mathcal{D}}
\newcommand{\ME}{\mathcal{E}}
\newcommand{\MF}{\mathcal{F}}
\newcommand{\MG}{\mathcal{G}}
\newcommand{\MH}{\mathcal{H}}
\newcommand{\MI}{\mathcal{I}}
\newcommand{\MJ}{\mathcal{J}}
\newcommand{\MK}{\mathcal{K}}
\newcommand{\ML}{\mathcal{L}}
\newcommand{\MM}{\mathcal{M}}
\newcommand{\MN}{\mathcal{N}}
\newcommand{\MO}{\mathcal{O}}
\newcommand{\MP}{\mathcal{P}}
\newcommand{\MQ}{\mathcal{Q}}
\newcommand{\MR}{\mathcal{R}}
\newcommand{\MS}{\mathcal{S}}
\newcommand{\MT}{\mathcal{T}}
\newcommand{\MU}{\mathcal{U}}
\newcommand{\MV}{\mathcal{V}}
\newcommand{\MW}{\mathcal{W}}
\newcommand{\MX}{\mathcal{X}}
\newcommand{\MY}{\mathcal{Y}}
\newcommand{\MZ}{\mathcal{Z}}

% mathfrak
\newcommand{\MFX}{\mathfrak{X}}
\newcommand{\MFg}{\mathfrak{g}}
\newcommand{\MFh}{\mathfrak{h}}

% tilde 
\newcommand{\tMA}{\tilde{\MA}}
\newcommand{\tMB}{\tilde{\MB}}
\newcommand{\tU}{\tilde{U}}
\newcommand{\tV}{\tilde{V}}
\newcommand{\tphi}{\tilde{\phi}}
\newcommand{\tpsi}{\tilde{\psi}}
\newcommand{\tF}{\tilde{F}}

% label/reference
\newcommand{\lex}[1]{\label{ex:#1}}
\newcommand{\rex}[1]{Exercise \ref{ex:#1}}

\newcommand{\ld}[1]{\label{defn:#1}}
\newcommand{\rd}[1]{Definition \ref{defn:#1}}

\newcommand{\lax}[1]{\label{ax:#1}}
\newcommand{\rax}[1]{Axiom \ref{ax:#1}}

\newcommand{\lfig}[1]{\label{fig:#1}}
\newcommand{\rfig}[1]{Figure \ref{fig:#1}}

% math operators
\DeclareMathOperator{\supp}{supp}
\DeclareMathOperator{\sgn}{sgn}
\DeclareMathOperator{\spn}{span}
\DeclareMathOperator{\iso}{Iso}
\DeclareMathOperator{\id}{id}
\DeclareMathOperator{\Aut}{Aut}
\DeclareMathOperator{\Homeo}{Homeo}
\DeclareMathOperator{\Sym}{Sym}
\DeclareMathOperator{\Alt}{Alt}
\DeclareMathOperator{\cl}{cl}
\DeclareMathOperator{\Int}{Int}
\DeclareMathOperator{\bal}{bal}
\DeclareMathOperator{\cnv}{conv}
\DeclareMathOperator{\epi}{epi}
\DeclareMathOperator{\dom}{dom}
\DeclareMathOperator{\cod}{cod}
\DeclareMathOperator{\Obj}{Obj}
\DeclareMathOperator{\Hom}{Hom}
\DeclareMathOperator*{\argmax}{arg\,max}
\DeclareMathOperator*{\argmin}{arg\,min}
\DeclareMathOperator{\diam}{diam}
\DeclareMathOperator{\rnk}{rank}
\DeclareMathOperator{\prj}{\text{proj}}
\DeclareMathOperator{\nab}{\nabla}
\DeclareMathOperator{\diag}{\text{diag}}
\DeclareMathOperator*{\ind}{\text{\text{ind}}}
\DeclareMathOperator*{\ar}{\text{\text{arity}}}
\DeclareMathOperator*{\cur}{\text{\text{cur}}}


% category theory
\DeclareMathOperator*{\Set}{\text{\tbf{Set}}}
\DeclareMathOperator*{\Meas}{\text{\tbf{Meas}}}
\DeclareMathOperator*{\Maninf}{\text{\tbf{Man}}^{\infty}} 
\DeclareMathOperator*{\Man0}{\text{\tbf{Man}}^{0}}
\DeclareMathOperator*{\Buninf}{\text{\tbf{Bun}}^{\infty}} 
\DeclareMathOperator*{\VecBuninf}{\text{\tbf{VecBun}}^{\infty}} 
\DeclareMathOperator*{\VectR}{\text{\tbf{Vect}}_{\R}} 
\DeclareMathOperator*{\Cat}{\text{\tbf{Cat}}}
\DeclareMathOperator*{\0}{\mbf{0}}
\DeclareMathOperator*{\1}{\mbf{1}}
\DeclareMathOperator*{\TopEq}{\text{\tbf{TopEq}}}
\DeclareMathOperator*{\Top}{\text{\tbf{Top}}}
\DeclareMathOperator*{\Cone}{\text{\tbf{Cone}}}
\DeclareMathOperator*{\Cocone}{\text{\tbf{Cocone}}}

% notation
\renewcommand{\r}{\rangle}
\renewcommand{\l}{\langle}
\renewcommand{\div}{\text{div}}
\renewcommand{\Im}{\text{Im} \,}
\newcommand{\grad}{\text{grad}}
\newcommand{\tbf}[1]{\textbf{#1}}
\newcommand{\tcb}[1]{\textcolor{blue}{#1}}
\newcommand{\mbf}[1]{\mathbf{#1}}
\newcommand{\ol}[1]{\overline{#1}}
\newcommand{\p}{\partial}
\newcommand{\Tn}[1]{T^{r_{#1}}_{s_{#1}}(V)}
\newcommand{\Tnp}{T^{r_1 + r_2}_{s_1 + s_2}(V)}
\newcommand{\Perm}{\text{Perm}}



% limits
\newcommand{\limfn}{\liminf \limits_{n \rightarrow \infty}}
\newcommand{\limpn}{\limsup \limits_{n \rightarrow \infty}}
\newcommand{\limn}{\lim \limits_{n \rightarrow \infty}}
\newcommand{\convt}[1]{\xrightarrow{\text{#1}}}
\newcommand{\conv}[1]{\xrightarrow{#1}} 
\newcommand{\seq}[2]{(#1_{#2})_{#2 \in \N}}

% intervals
\newcommand{\RG}{[0,\infty]}
\newcommand{\Rg}{[0,\infty)}
\newcommand{\Ru}{(\infty, \infty]}
\newcommand{\Rd}{[\infty, \infty)}
\newcommand{\ui}{[0,1]}

% integration \newcommand{\dm}{\, d m}
\newcommand{\dmu}{\, d \mu}
\newcommand{\dnu}{\, d \nu}
\newcommand{\dlam}{\, d \lambda}
\newcommand{\dP}{\, d P}
\newcommand{\dQ}{\, d Q}
\newcommand{\dm}{\, d m}
\newcommand{\dsh}{\, d \#}

% abreviations 
\newcommand{\lsc}{lower semicontinuous}

% misc
\newcommand{\as}[1]{\overset{#1}{\sim}}
\newcommand{\astx}[1]{\overset{\text{#1}}{\sim}}
\newcommand{\io}{\text{ i.o.}}
%\newcommand{\ev}{\text{ ev.}}
\newcommand{\Ll}{L^1_{\text{loc}}(\R^n)}

\newcommand{\loc}{\text{loc}}
\newcommand{\BV}{\text{BV}}
\newcommand{\NBV}{\text{NBV}}
\newcommand{\TV}{\text{TV}}

\newcommand{\op}[1]{\mathcal{#1}^{\text{op}}}


% Glossary - Notation
\glsxtrnewsymbol[description={finite measures on $(X, \MA)$}]{n000001}{$\MM_+(X, \MA)$}
\glsxtrnewsymbol[description={velocity}]{v}{\ensuremath{v}}


\makeindex

\begin{document}
	
	\frontmatter
	
	\title{Introduction to Algebra}
	
	%    Remove any unused author tags.
	
	%    author one information
	\author{Carson James}
	\thanks{}
	
	\date{}
	
	\maketitle
	
	%    Dedication.  If the dedication is longer than a line or two,
	%    remove the centering instructions and the line break.
	%\cleardoublepage
	%\thispagestyle{empty}
	%\vspace*{13.5pc}
	%\begin{center}
	%  Dedication text (use \\[2pt] for line break if necessary)
	%\end{center}
	%\cleardoublepage
	
	%    Change page number to 6 if a dedication is present.
	\setcounter{page}{4}
	
	\tableofcontents
	\printunsrtglossary[type=symbols,style=long,title={Notation}]
	
	%    Include unnumbered chapters (preface, acknowledgments, etc.) here.
	%\include{}
	
	\mainmatter
	%    Include main chapters here.
	%\include{}
	
	\chapter*{Preface}
	\addcontentsline{toc}{chapter}{Preface}
	
	\begin{flushleft}
		\href{https://creativecommons.org/licenses/by-nc-sa/4.0/legalcode.txt}{cc-by-nc-sa}
	\end{flushleft}
	
	\newpage
	
	
	
	\chapter{Set Theory}

\section{Operations and Relations}

\begin{defn}\
	\begin{itemize}
		\item We define $[0] \defeq \varnothing$ and for $k \in \N$, we define $[k] \defeq \{1, \ldots, k\}$. 
		\item Let $S$ be a set and $k \in \N_0$. We define the \tbf{set of $k$-tupels with entries in $S$}, denoted $S^k$, by 
		$$S^k \defeq \{u: [k] \rightarrow S\}$$
		\item Let $S$ be a set. We define the \tbf{set of all tuples with entries in $S$}, denoted $S^*$, by 
		$$S^* \defeq \bigcup_{k \in \N_0} S^k$$
		\item Let $S$ be a set and $k \in \N_0$. We define the \tbf{set of $k$-ary operations on $S$}, denoted $\MF^k(S)$, by $\MF^k(S) \defeq S^{(S^k)}$. We define the \tbf{set of finitary operations on $S$}, denoted $\MF^*(S)$, by
		$$\MF^*(S) \defeq \bigcup_{k \in \N_0} \MF^k(S)$$
		\item Let $S$ be a set. We define the \tbf{operation arity map}, denoted $\ar: \MF^*(S) \rightarrow \N_0$, by 
		$$\ar f \defeq k, \quad f \in \MF^k(S)$$
		\item Let $S$ be a set, $\MF \subset \MF^*(S)$ and $k \in \N_0$. We define the \tbf{$k$-ary members of $\MF$}, denoted $\MF_k$, by 
		$$\MF_k \defeq \MF \cap \MF^k(S)$$
		\item Let $S$ be a set and $k \in \N_0$. We define the \tbf{set of $k$-ary relations on $S$}, denoted $\MR^k(S)$, by $\MR^k(S) \defeq \MP(S^k)$. We define the \tbf{set of finitary relations on $S$}, denoted $\MR^*(S)$, by
		$$\MR^*(S) \defeq \bigcup_{k \in \N_0} \MR^k(S)$$
		\item Let $S$ be a set. We define the \tbf{arity map}, denoted $\ar: \MR^*(S) \rightarrow \N_0$, by 
		$$\ar R \defeq k, \quad f \in \MR^k(S)$$
		\item Let $S$ be a set, $\MR \subset \MR^*(S)$ and $k \in \N_0$. We define the \tbf{$k$-ary members of $\MR$}, denoted $\MR_k$, by 
		$$\MR_k \defeq \MR \cap \MR^k(S)$$
	\end{itemize}
\end{defn}

\begin{defn}
	Let $S$ be a set, $k \geq 2$ and $f \in \MF^k(S)$. Then $f$ is said to be
	\begin{itemize}
		\item \tbf{associative} if for each $x_1, \ldots, x_k, x_{k+1}, \ldots, x_{k + (k-1)} \in S$, 
		\begin{align*}
			f(f(x_1, \ldots, x_k) x_{k+1}, \ldots, x_{k + (k-1)}) 
			& = f(x_1, f(x_2, \ldots, x_{k+1}), x_{k+2}, \ldots, x_{k + (k-1)}) \\
			& \vdots \\
			& = f(x_1, \ldots, x_{k-1}, f(x_k, \ldots, x_{k + (k-1)}) )
		\end{align*}
		\item \tbf{symmetric} if for each $x_1, \ldots, x_k \in S$, $\sig \in S_k$, $f(x_1, \ldots, x_k) = f(x_{\sig(1)}, \ldots, x_{\sig(k)})$.
		\item \tbf{idempotent} if for each $x \in S$,
		$f(x, \ldots, x) = x$
	\end{itemize}
\end{defn}

\begin{defn}
	Let $S$ be a set, $\MF \subset \MF^*(S)$ and $C \subset S$. Then $C$ is said to be  \tbf{$\MF$-closed} if for each $k \in \N_0$, $f \in \MF_k$ and $a \in C^k$, $f(a) \in C$.
\end{defn}

\begin{ex}
	Let $S$ be a set, $\MF \subset \MF^*(S)$ and $\MC \subset \MP(S)$. If for each $C \in \MC$, $C$ is  $\MF$-closed, then $\bigcap\limits_{C \in \MC} C$ is $\MF$-closed
	\tcr{need special case where $k=0$? maybe trivially true?}
\end{ex}

\begin{proof}
	Suppose that for each $C \in \MC$, $C$ is  $\MF$-closed. Let $k \in \N_0$, $f \in \MF_k$, $a_1, \ldots, a_k \in \bigcap\limits_{C \in \MC} C$ and $C_0 \in \MC$. Since $C_0 \in \MC$, we have that 
	\begin{align*}
		a_1, \ldots, a_k 
		& \in \bigcap_{C \in \MC} C \\
		& \subset C_0
	\end{align*}
	Since $C_0$ is $\MF$-closed, we have that $f(a_1, \ldots, a_k) \in C_0$. Since $C_0 \in \MC$ is arbitrary, we have that for each $C \in \MC$, $f(a_1, \ldots, a_k) \in C$. Hence $f(a_1, \ldots, a_k) \in \bigcap\limits_{C \in \MC} C$. Since $k \in \N_0$ and $a_1, \ldots, a_k \in \bigcap\limits_{C \in \MC} C$ are arbitrary, we have that $\bigcap\limits_{C \in \MC} C$ is $\MF$-closed.
\end{proof}


	
	
	
	
	\newpage
	\chapter{Model Theory}
	
	\section{Introduction}
	
	
	
	
	
	
	
	
	
	
	
	
	
	
	
	
	
	
	
	
	
	
	
	
	
	
	
	
	
	
	
	
	
	
	
	
	
	
	
	
	
	
	
	
	\newpage
	\chapter{Lattices}
	
	\begin{defn}
		Let $L$ be a set and $\wedge, \vee: L^2 \rightarrow L$. Then $(L, \wedge, \vee)$ is said to be a \tbf{lattice} if for each $x,y,z \in L$,
		\begin{enumerate}
			\item $(x \vee y) \vee z = x \vee (y \vee z)$ and $(x \wedge y) \wedge z = x \wedge (y \wedge z)$,
			\item $x \vee y = y \vee x$ and $x \wedge y = y \wedge x$,
			\item $x \vee x = x$ and $x \wedge x = x$,
			\item $x \vee (x \wedge y) = x$ and $x \wedge (x \vee y) = x$
		\end{enumerate}
	\end{defn}
	
	
	
	
	
	\section{Closure Operators}
	
	\begin{defn}
		Let $A$ be a set and $C: \MP(A) \rightarrow \MP(A)$. Then $C$ is said to be a \tbf{closure operator on $A$} if for each $X, Y \in \MP(A)$,
		\begin{enumerate}
			\item $X \subset C(X)$,
			\item $C^2(X) = C(X)$,
			\item $X \subset Y$ implies that $C(X) \subset C(Y)$.
		\end{enumerate}
	\end{defn}

	\begin{ex}
		Let $A$ be a set and $C: \MP(A) \rightarrow \MP(A)$. Suppose that $C$ is a closure operator on $A$. Then for each $(E_j)_{j \in J} \subset \MP(A)$,
		\begin{enumerate}
			\item $C \bigg( \bigcap\limits_{j \in J} E_j \bigg) \subset \bigcap\limits_{k \in J} C(E_k)$,
			\item $\bigcup\limits_{k \in J} C(E_k) \subset C \bigg( \bigcup\limits_{j \in J} E_j \bigg)$.
		\end{enumerate}
	\end{ex}

	\begin{proof}
		Let $(E_j)_{j \in J} \subset \MP(A)$. 
		\begin{enumerate}
			\item Let $k \in J$. Then $\bigcap\limits_{j \in J} E_j \subset E_k$.
			So $C \bigg( \bigcap\limits_{j \in J} E_j \bigg) \subset C(E_k)$. Since $k \in J$ is arbitrary, we have that 
			$$C \bigg( \bigcap\limits_{j \in J} E_j \bigg) \subset \bigcap\limits_{k \in J} C(E_k).$$ 
			\item Let $k \in J$. Then $E_k \subset \bigcup_{j \in J} E_j$. Hence $C(E_k) \subset C \bigg( \bigcup\limits_{j \in J} E_j \bigg)$. Since $k \in J$ is arbitrary, we have that 
			\begin{align*}
				\bigcup\limits_{k \in J} C(E_k) 
				& \subset C \bigg( \bigcup\limits_{j \in J} E_j \bigg)
			\end{align*}
			
		\end{enumerate}
	\end{proof}

	\begin{defn}
		Let $A$ be a set, $C: \MP(A) \rightarrow \MP(A)$ and $X \subset A$. Suppose that $C$ is a closure operator on $A$. Then $X$ is said to be $C$-closed if $C(X) = X$.  
	\end{defn}

	\begin{defn}
		Let $A$ be a set and $C: \MP(A) \rightarrow \MP(A)$. Suppose that $C$ is a closure operator on $A$. We define the \tbf{lattice of $C$-closed subsets of $A$}, denoted $L_C(A) \subset \MP(A)$, by 
		$$L_C(A) \defeq \{X \subset A: \text{$X$ is $C$-closed}\}$$. 
	\end{defn}


	\begin{ex}
		Let $A$ be a set and $C: \MP(A) \rightarrow \MP(A)$. Suppose that $C$ is a closure operator on $A$. Then
		\begin{enumerate}
			\item for each $(E_j)_{j \in J} \subset L_C(A)$, $\bigcap\limits_{j \in J} E_j \in L_C(A)$ and $\bigcup\limits_{j \in J} E_j \in L_C(A)$.
			\item $(L_C(A), \subset)$ is a complete lattice
			\tcr{define complete lattice}
			$$ C \bigg( \bigcap\limits_{j \in J} E_j \bigg) = \bigcap\limits_{j \in J} E_j $$ 
			and  
			$$ C \bigg(  \bigcup\limits_{j \in J} E_j \bigg) = \bigcup\limits_{j \in J} E_j.$$
		\end{enumerate}
	\end{ex}

	\begin{proof}\
		\begin{enumerate}
			\item Let $(E_j)_{j \in J} \subset L_C(A)$. 
			\begin{itemize}
				\item \tcr{A previous exercise} \rex{} implies that
				\begin{align*}
					C \bigg( \bigcap\limits_{j \in J} E_j \bigg) 
					& \subset \bigcap\limits_{k \in J} C(E_k) \\
					& = \bigcap\limits_{k \in J} E_k \\
					& \subset C \bigg( \bigcap\limits_{k \in J} E_k \bigg). 
				\end{align*}
				Hence $C \bigg( \bigcap\limits_{j \in J} E_j \bigg) = \bigcap\limits_{k \in J} E_k$.
				\item \tcr{A previous exercise} \rex{} implies that
				\begin{align*}
					\bigcup\limits_{k \in J} E_k 
					& = \bigcup\limits_{k \in J} C(E_k) \\
					& \subset C \bigg( \bigcup\limits_{j \in J} E_j \bigg) \\
					& \subset \bigcap\limits_{k \in J} C(E_k) \\
					& = \bigcap\limits_{k \in J} E_k \\
					& \subset C \bigg( \bigcap\limits_{k \in J} E_k \bigg). 
				\end{align*}
				Hence $C \bigg( \bigcap\limits_{j \in J} E_j \bigg) = \bigcap\limits_{k \in J} E_k$.
			\end{itemize}
			\item 
		\end{enumerate}
		\tcr{FINISH!!!, don't need to show second part, }
	\end{proof}

	\begin{defn}
		then is said to be an \tbf{algebraic closure operator on $A$} if 
	\end{defn}

	
	
	
	
	
	
	
	
	
	
	
	
	
	
	
	
	
	
	
	
	
	
	
	
	
	
	
	
	
	
	
	
	
	
	
	
	
	
	
	
	
	\newpage
	\chapter{Universal Algebra}
	
	\section{Introduction}
	
	\begin{defn}
		Let $A \in \Obj(\Set)$ be a set and $J$ an index set. Suppose that $A \neq \varnothing$. Let $f \in \MF^*(A)^J$. Then $(A, f)$ is said to be an \tbf{algebra with universe $A$ and basic operations $f$}.
	\end{defn}

	\begin{defn}
		Let $(A, f)$ be an algebra. Set $J \defeq \dom f$. We define the \tbf{similarity type of $f$}, denoted $\rho^f: J \rightarrow \N_0$, by $\rho^f(j) \defeq \ar f_j$.
	\end{defn}

	\begin{defn}
		Let $(A, f)$, $(B, g)$ be algebras. Then $(A, f)$ and $(B, g)$ are said to be \tbf{type similar} if $\rho^f = \rho^g$.
	\end{defn}

	\begin{note}
		Set $J_f \defeq \dom f$ and $J_g \defeq \dom g$. Then $(A, f)$ and $(B, g)$ are type similar iff $J_f = J_g$ and for each $j \in J_f$, $\ar f_j = \ar g_j$.
	\end{note}

	\tcr{maybe define similarity type $\rho$ first and then stipulate algebras belonging to the set of algebras of that type, this way we dont need $\rho^f$, only $\rho$}.











	\newpage
	\section{Subalgebras}
	
	\begin{defn}
		Let $(A, f)$ be an algebra and $B \subset A$. Then $B$ is said to be an $f$-subuniverse of $A$ if $B$ is $f$-closed.
	\end{defn}

	\begin{defn}
		Let $(A, f)$ be an algebra and $B \subset A$. Set $\MS \defeq \{S \subset A: \text{$S$ is an $f$-subuniverse of $A$ and $B \subset S$}\}$. We define the \tbf{$f$-subuniverse of $A$ generated by $B$}, denoted $\Sg(B, f)$, by
		$$\Sg(B, f) \defeq \bigcap\limits_{S \in \MS} S $$
	\end{defn}

	\begin{ex}
		Let $(A, f)$ be an algebra and $B \subset A$. Then 
		\begin{enumerate}
			\item $\Sg(B, f)$ is an $f$-subuniverse of $A$
			\item $B \subset \Sg(B, f)$.
		\end{enumerate}.
	\end{ex}

	\begin{proof}\
		\begin{enumerate}
			\item Set $\MS \defeq \{S \subset A: \text{$S$ is an $f$-subuniverse of $A$}\}$. By construction, for each $S \in S$, $S$ is $f$-closed. Since $\Sg(B, f) = \bigcap\limits_{S \in \MS} S$, \rex{} \tcr{A previous exercise in the set theory section} implies that $\Sg(B, f)$ is $f$-closed. Hence $\Sg(B, f)$ is an $f$-subuniverse of $A$. 
			\item By construction, for each $S \in S$, $B \subset S$. Thus
			\begin{align*}
				B
				& \subset \bigcap\limits_{S \in \MS} S \\
				& = \Sg(B, f).
			\end{align*}
		\end{enumerate}
	\end{proof}

	\begin{ex}
		Let $(A, f)$ be an algebra. Then $\Sg(\cdot, f)$ is an algebraic closure operator on $A$.
	\end{ex}

	\begin{proof}
		
	\end{proof}

	\begin{defn}
		Let $(A, f)$, $(B, g)$ be algebras. Suppose that $(A, f)$ and $(B, g)$ are type similar. Set $J \defeq \dom f$ and $\rho \defeq \rho^f$. Then $(B, g)$ is said to be a \tbf{subalgebra} of $(A, f)$ if 
		\begin{enumerate}
			\item $A \subset B$
			\item for each $j \in J$, $f_j|_{B^{\rho(j)}} = g_j$.
		\end{enumerate}
	\end{defn}

	\begin{ex}
		Let $(A, f)$, $(B, g)$ be algebras. Suppose that $(A, f)$ and $(B, g)$ are type similar. If $(B, g)$ is a sub algebra of $(A, f)$, then $B$ is a subuniverse of $A$. 
	\end{ex}

	\begin{proof}
		Set $J \defeq \dom f$. Suppose that $(B, g)$ is a sub algebra of $(A, f)$. Let $j \in J$. Then for each $a_1, \ldots, a_{\rho^f(j)} \in B$, 
		\begin{align*}
			f_j(a_1, \ldots, a_{\rho^f(j)})
			& = f_j|_{B^{\rho^f(j)}}(a_1, \ldots, a_{\rho^f(j)}) \\
			& = g_j((a_1, \ldots, a_{\rho^f(j)})) \\
			& \in B.
		\end{align*}
		Since $j \in J$ is arbitrary, we have that $B$ is $f$-closed. Thus $B$ is a subuniverse of $A$.
	\end{proof}

	



	
	
	
	
	
	
	
	
	
	
	
	
	
	
	
	
	
	
	
	
	
	
	
	
	
	
	
	
	
	
	
	
	
	
	
	
	
	\newpage
	
	\section{Homomorphisms}
	
	\begin{defn}
		Let $(A, f)$, $(B, g)$ be algebras and $h:A \rightarrow B$. Suppose that $(A, f)$ and $(B, g)$ are type similar and set $J \defeq \dom f$, $\rho \defeq \rho^f$. Then $h$ is said to be a \tbf{homomorphism} if for each $j \in J$, and $a_1, \ldots, a_{\rho(j)}$, 
		$$h(f_j(a_1, \ldots, a_{\rho(j)})) = g_j(h(a_1), \ldots, h(a_{\rho(j)})).$$
	\end{defn}
	
	
	
	
	
	
	
	
	
	
	
	
	
	
	
	
	
	
	
	
	
	
	
	
	
	
	
	
	
	
	
	
	
	
	
	
	
	
	
	
	
	
	
	
	
	
	
	
	
	
	
	
	
	
	
	
	
	
	
	
	
	
	
	
	
	
	
	
	
	
	
	
	
	
	
	
	\newpage

	
	
	\chapter{Groups}
	
	\subsection{Direct Products}
	
	\begin{defn}
	Let $G,H$ be groups. Define a product $*:(G \times H) \times (G \times H) \rightarrow G \times H$ by 
	$$(x_1,y_1) * (x_2, y_2) = (x_1x_2, y_1y_2)$$
	Then $(G \times H, *)$ is called the \textbf{direct product of $G$ and $H$}.
	\end{defn}	
	
	\begin{ex}
	\lex{1} Let $G,H$ be groups. Then the direct product $G \times H$ is a group.
	\end{ex}
	\begin{proof}
	Clear.
	\end{proof}
	
	\begin{defn} 
	Let $G,H$ be groups. Define $\pi_G :G \times H \rightarrow G$ and $\pi_H :G \times H \rightarrow H$ by $\pi_G(x,y) = x$ and $\pi_H(x,y) = y$.  Then $\pi_G$ and $\pi_H$ are respectively called the \textbf{projection maps onto $G$ and $H$}.
	\end{defn}	
	
	\begin{ex}
	\lex{2} Let $G,H$ be groups. Then 
	\begin{enumerate}
	\item $\pi_G: G \times H \rightarrow G$ and $\pi_H : G \times H \rightarrow H$ are homomorphisms
	\item $\ker \pi_G \cong H$ and $\ker \pi_H \cong G$
\end{enumerate}	 
	\end{ex}
	
	\begin{proof}\
	\begin{enumerate}
	\item Clear
	\item Define $\iota_G:G \rightarrow \ker \pi_H$ by $$\iota_G(x) = (x, e_H)$$ Then $\iota_G$ is an isomorphism. Similarly, we can define $\iota_H:H \rightarrow \ker \pi_G$ and show that it is an isomorphism.
	\end{enumerate}
	\end{proof}
	
	\begin{defn}
	Let $G,H, K$ be groups, $\phi \in \Hom(G,K)$ and $\psi \in \Hom(H, K)$. We define $\phi \times \psi: G \times H \rightarrow K$ by $\phi \times \psi(x,y) = \phi(x) \psi(y)$ 
	\end{defn}	
	
	\begin{ex}
	\lex{3} Let $G,H, K$ be groups, $\phi \in \Hom(G,K)$ and $\psi \in \Hom(H, K)$. If $K$ is abelian, then $\phi \times \psi \in Hom(G \times H,K)$.
	\end{ex}
	
	\begin{proof}
	Let $x_1, x_2 \in G$ and $y_1, y_2 \in H$. Then 
	\begin{align*}
	\phi \times \psi[(x_1, y_1)(x_2, y_2)] 
	&= \phi \times \psi (x_1x_2, y_1y_2) \\
	&= \phi(x_1x_2) \psi(y_1y_2) \\
	&= \phi(x_1)\phi(x_2)\psi(y_1)\psi(y_2) \\
	&= \phi(x_1)\psi(y_1)\phi(x_2)\psi(y_2) \\
	&= [\phi \times \psi(x_1, y_1)] [\phi \times \psi(x_2, y_2) ]
	\end{align*}
	\end{proof}
	
	\begin{ex}
	\lex{4} Let $G,H, K$ be groups and $\phi \in \Hom(G \times H, K)$. Then there exist $\phi_G \in \Hom(G,K)$, $\phi_H \in \Hom(H, K)$ such that $\phi_G \times \phi_H = \phi$.
	\end{ex}
	
	\begin{proof}
	Suppose that $K$ is abelian. Define $\iota_G \in \Hom(G, \ker \pi_H)$ and $\iota_H \in \Hom(H, \ker \pi_G)$ as in part $(2)$ of \rex{2} Define $\phi_G \in \Hom(G, K)$ and $\phi_H \in \Hom(H,K)$ by  $\phi_G = \phi \circ \iota_G$ and $\phi_H = \phi \circ \iota_H $. Let $(x,y) \in G \times H$. Then 
	\begin{align*}
	\phi_G \times \phi_H(x,y) 
	&= \phi_G(x) \phi_H(y) \\
	&= \phi \circ \iota_G(x) \phi \circ \iota_H(y) \\
	&= \phi(x, e_H)\phi(e_G, y) \\
	&= \phi(x,y) \\
	\end{align*}
	So $\phi = \phi_G \times \phi_H$
	\end{proof}
	
	
	
	
	
	
	
	
	
	
	
	
	
	
	
	
	\newpage
	\section{Rings}
	
	\begin{defn}
	Let $R$ be a set and $+, *: R \times R 				
	\rightarrow R$ (we write $a+b$ and 
	$ab$ in place of $+(a,b)$ and $*(a,b)$ respectively).
	Then $R$ is said to be a \textbf{ring} if for each 
	$a,b,c \in R$,
	\begin{enumerate}
	\item $R$ is an abelian group with respect to $+$.
	 The identity element with respect to $+$ is denoted
	 by $0$.
	\item $R$ is a monoid with respect to $*$. The  
	identity element of $R$ with respect to $*$ is denoted $1$. 
	\item $R$ is commutative with respect to $*$.
	\item $*$ distributes over $+$.
	\end{enumerate}
	\end{defn}
	
	\begin{defn}
	Let $R$ be a ring and $I \subset R$. Then $I$ is said 
	to be an \textbf{ideal} of $R$ if for each $a \in R$ and $x,y \in I$,
	\begin{enumerate}
	\item  $x + y \in I$
	\item  $ax \in I$
	\end{enumerate}
	\end{defn}
	
	\begin{defn}
	Let $R$ be a ring and $A,B \subset R$. We define the \textbf{product} of $A$ and $B$, denoted $AB$, to be $$AB = \bigg \{\sum_{i=1}^n a_ib_i: a_i \in A, b_i \in B, n \in \N \bigg \}$$
	\end{defn}	
	
	\begin{ex}
	Let $R$ be a ring and $I \subset R$. Then $I$ is an ideal of $R$ iff $RI \subset I$. 
	\end{ex}
	
	\begin{proof}
	Suppose that $RI \subset I$. Let $a \in R$ and $x,y \in I$. Then by assumption $x + y = 1x + 1y \in I$ and $ax \in I$. So $I$ is an ideal of $R$\\
	Conversely, suppose that $I$ is an ideal of $R$. Let $a_1, \cdots, a_n \in R$ and $x_1, \cdots, x_n \in I$. Then by assumption, for each $i = 1, \cdots, n$, $a_ix_i \in I$ and therefore $\sum\limits_{i=1}^n a_ib_i \in I$. Hence $RI \subset I$.
	\end{proof}
	
	
	
	
	
	
	
	
	
	
	
	
	
	
	
	
	
	
	
	
	
	
	
	\newpage	
	\section{Modules}
	
	\subsection{Introduction}
	
	\begin{defn}
	Let $R$ be a ring, $M$ a set, $+: M\times M \rightarrow M$ and $*: R 
	\times M \rightarrow M$ (we write $rx$ in place of 
	$*(r,x)$). Then $M$ is said to be an 
	\textbf{$R$-module}
	if 
	\begin{enumerate}
	\item $M$ is an abelian group with respect to $+$. The identity element of $M$ with respect to $+$ is denoted by 0.
	\item for each $r \in R$, $*(r, \cdot)$ is a group endomorphism of $M$
	\item for each $x \in M$, $*(\cdot, x)$ is a group homomorphism from $R$ to $M$
	\item $*$ is a monoid action of $R$ on $M$
	\end{enumerate}
	\end{defn}
	
	\begin{note}
	For the remainder of this section, we assume that $R$ is a commutative ring. 
	\end{note}
	
	\begin{ex}
	Let $M$ be an $R$-module. Then for each $r \in R$ and $x \in M$, 
	\begin{enumerate}
	\item $r0 = 0$
	\item $0x = 0$
	\item $(-1)x = -x$
	\end{enumerate}
	\end{ex}
	
	\begin{proof} Let $r \in R$ and $x \in M$. Then 
	\begin{enumerate}
	\item 
	\begin{align*}
	r0 
	&= r(0+0) \\
	&= r0 + r0
	\end{align*} 
	which implies that $r0 = 0$.
	\item 
	\begin{align*}
	0x 
	&= (0+0)x \\
	&= 0x + 0x
	\end{align*} 
	which implies that $0x = 0$.
	\item 
	\begin{align*}
	(-1)x + x 
	&= (-1)x + 1x \\ 
	&= (-1 + 1)x \\
	&= 0x \\
	&= 0
	\end{align*}
	which implies that $(-1)x = -x$.
	\end{enumerate}
	\end{proof}
	
	\begin{defn}
	Let $M$ an $R$-module and $N \subset M$. Then $N$ is said to be a \textbf{submodule} of $M$ if for each $r \in R$ and $x,y \in N$, we have that $rx \in N$ and $x+y \in N$.
	\end{defn}
	
	\begin{defn}
	Let $M$ be an $R$-module. We define $\MS(M) = \{N \subset M: N \text{ is a submodule of }M\}$.
	\end{defn}	
	
	\begin{ex}
	Let $M$ be an $R$-module and $N \in \MS(M)$. Then $N$ is a subgroup of $M$.
	\end{ex}
	
	\begin{proof}
	Let $x,y \in M$. Then $x-y = 1x + (-1)y \in N$. So $N$ is a subgroup of $M$.
	\end{proof}
	
	\begin{defn}
	Let $M$ be an $R$-module and $N \in \MS(M)$. We define  
	\begin{enumerate}
	\item $M/N = \{x + N: x \in M\}$ 
	\item $+: M/N \times M/N \rightarrow M/N$ by $$(x+N) + (y+N) = (x+y) + N$$
	\item $*: R \times M/N \rightarrow M/N$ by $$r(x+N) = (rx) + N$$
	\end{enumerate}
	Under these operations (see next exercise), $M/N$ is an $R$-module known as the \textbf{quotient module} of $M$ by $N$.
	\end{defn}	
	
	\begin{ex} 
	Let $M$ be an $R$-module and $N \in \MS(M)$. Then
	\begin{enumerate}
	\item the monoid action defined above is well defined
	\item the quotient $M/N$ is an $R$-module
	\end{enumerate}
	\end{ex}
	
	\begin{proof}\
	\begin{enumerate}
	\item Let $r \in R$ and $x +N, y +N \in M/N $. Recall from group theory that $x + N = y + N$ iff $x-y \in N$. Suppose that $x + N = y + N$. Then $x - y \in N$ and there exists $n \in N$ such that $x-y = n$. Therefore
	\begin{align*}
	rx - ry 
	&= r(x-y) \\
	&= rn \\
	&\in N
	\end{align*}
	So $rx + N = ry + N$.
	\item Properties $(1)$ - $(4)$ in the definition of a module are easily shown to be satisfied for $M/N$ since they are true for $M$.
	\end{enumerate}
	\end{proof}
	
	\begin{defn}
	Let $M$ and $N$ be $R$-modules and $\phi:M \rightarrow N$. Then $\phi$ is said to be a \textbf{module homomorphism} if for each $r \in R$ and $x,y \in M$
	\begin{enumerate}
	\item $\phi(rx) = r\phi(x)$
	\item $\phi(x+y) = \phi(x) + \phi(y)$
	\end{enumerate}
	\end{defn}	
	
	\begin{ex}
	Let $M$ and $N$ be $R$-modules and $\phi:M \rightarrow N$. Then $\phi$ is a  iff for each $r \in R$ and $x,y \in M$, $\phi(x+ry) = \phi(x) + r \phi(y)$.
	\end{ex}
	
	\begin{proof}
	Clear.
	\end{proof}
	
	\begin{ex}
	Let $M$ and $N$ be $R$-modules and $\phi:M \rightarrow N$ a homomorphism. Then 
	\begin{enumerate}
	\item $\ker \phi$ is a submodule of $M$
	\item $ \Im \phi$ is a submodule of $N$ 
	\end{enumerate}
	\end{ex}
	
	\begin{proof}
	Let $r \in R$, $x,y \in \ker \phi$ and $w,z \in \Im \phi$. Then 
	\begin{enumerate}
	\item 
	\begin{align*}
	\phi(rx) 
	&= r\phi(x) \\
	&=r 0 \\
	&= 0
\end{align*}	
	So $rx \in \ker \phi$. Group theory tells us that $\ker \phi$ is a subgroup of $M$, so $x+y \in \ker \phi$. Hence $\ker \phi$ is a submodule of $M$. 
	\item Similar.
	\end{enumerate}
	\end{proof}
	
	\begin{defn}
	Let $M$ be an $R$-module and $A \subset M$. We define the \textbf{submodule of $M$ generated by $A$}, denoted $\spn(A)$, to be $$\spn(A) = \bigcap_{N \in \MS(M)} N$$ 
	\end{defn}
	
	\begin{ex}
	Let $M$ be an $R$-module and $A \subset M$. Then 
	$\spn(A) \in \MS(M)$
	\end{ex}
	
	\begin{proof}
	Let $r \in R$ and $x,y \in \spn(A)$. Basic group theory tells us that $\spn(A)$ is a subgroup of $M$. So $x+y \in \spn(A)$. For $N \in \MS(M)$, by definition we have $x \in N$ and therefore $rx \in N$. So $rx \in \spn(A)$. Hence $\spn(A)$ is a submodule of $M$.
	\end{proof}
	
	\begin{ex}
	Let $M$ be an $R$-module and $A \subset M$. If $A \neq \varnothing$, then $$\spn(A) = \bigg \{\sum\limits_{i=1}^n r_ia_i: r_i \in R, a_i \in A, n \in \N \bigg \}$$
	\end{ex}
	
	\begin{proof}
	Clearly 
	\end{proof}
	
	\begin{defn}
	Let $M$
	\end{defn}
	
	
	
	
	
	
	
	
	
	
	
	
	
	
	
	
	
	
	
	
	\newpage
	\section{Fields}
	
	
	
	
	
	
	
	
	
	
	
	
	
	
	
	
	
	
	
	
	
	
	
	
	
	
	
	
	\newpage
	\section{Vector Spaces}
	
	\section{Appendix}
	\subsection{Monoids}
	
	\begin{defn}
	Let $G$ be a set and $*: G \times G \rightarrow G$ (we write $ab$ in place of $*(a,b)$). Then 
	\begin{enumerate}
	\item $*$ is called a \textbf{binary operation} on $G$	
	\item $*$ is said to be \textbf{associative}	if for each $x,y,z \in G$, $(xy)z = x(yz)$
	\item $*$ is said to be \textbf{commutative} if for each $x,y \in G$, $xy = yx$ 
	\end{enumerate}
	\end{defn}
	
	\begin{defn}
	Let $G$ be a set, $*: G \times G \rightarrow G$, $e,x,y \in G$. Then $e$ is said to be an \textbf{identity element} if for each $x \in G$, $ex = xe = x$.
	\end{defn}
	
	\begin{defn}
	Let $G$ be a set and $*: G \times G \rightarrow G$. Then $G$ is said to be a \textbf{monoid} if 
	\begin{enumerate}
	\item $*$ is associative
	\item there exits $e \in G$ such that $e$ is an identity element.
	\end{enumerate}
	\end{defn}	
	
	\begin{ex}
	Let $G$ be a monoid. Then the identity element is unique.
	\end{ex}
	
	\begin{proof}
	Let $e, f \in G$. Suppose that $e$ and $f$ are identity elements. Then $e = ef = f$.
	\end{proof}
	
	\begin{note}
	Unless otherwise specified, we will denote the identity element of a monoid by $e$.
	\end{note}
	
	\begin{defn}
	Let $G$ be a monoid, $X$ a set and $*: G \times X \rightarrow X$ (we write $gx$ in place of $*(g,x)$). Then $*$ is said to be a \textbf{monoid action} of $G$ on $X$ if for each $g,h \in G$ and $x \in X$,
	\begin{enumerate}
	\item $(gh)x = g(hx)$
	\item $ex = x$
	\end{enumerate}
	\end{defn}










































 
 
 
 
 
 
 
 
 
 
 
 
 
 
 
 
 
 
 
 
 
 
 
 
 
 
 
 
 
 
 
 
 
 
 
 
 
 
 
 
 
 
 
 
 
 
 
 
 
 
 
 
 
 
 
 
 
 \appendix
 
 \chapter{Summation}
 
 \begin{defn} \ld{}
 	Let $f:X \rightarrow \Rg$, Then we define $$\sum_{x \in X} f(x) := \sup_{\substack{F \subset X \\ F \text{ finite}}} \sum_{x \in F} f(x)$$ This definition coincides with the usual notion of summation when $X$ is countable. For $f:X \rightarrow \C$, we can write $f = g +ih$ where $g,h:X \rightarrow \R$. If $$\sum_{x \in X}|f(x)| < \infty,$$ then the same is true for $g^+,g^-,h^+,h^-$. In this case, we may define $$\sum_{x \in X} f(x)$$ in the obvious way.
 \end{defn} 
 
 The following note justifies the notation $\sum_{x \in X}f(x)$ where $f:X \rightarrow \C$.
 
 \begin{note}
 	Let $f:X \rightarrow \C$ and $\al:X \rightarrow X$ a bijection. If $\sum_{x \in X}|f(x)|< \infty$, then $\sum_{x \in X}f( \al (x)) = \sum_{x \in X}f(x) $.
 \end{note}
 
 \newpage	
 
 \chapter{Asymptotic Notation}
 
 \begin{defn} \ld{}
 	Let $X$ be a topological space, $Y, Z$ be normed vector spaces, $f:X \rightarrow Y$, $g: X \rightarrow Z$ and $x_0 \in X \cup \{\infty\}$. Then we write $$f = o(g) \hspace{.5cm} \text{ as } x \rightarrow x_0$$ if for each $\ep >0$, there exists $U \in \MN(x_0)$ such that for each $x \in U$, $$\|f(x)\| \leq \ep\|g(x)\|$$
 \end{defn}
 
 \begin{ex} \lex{}
 	Let $X$ be a topological space, $Y, Z$ be normed vector spaces, $f:X \rightarrow Y$, $g: X \rightarrow Z$ and $x_0 \in X \cup \{\infty\}$. If there exists $U \in \MN(x_0)$ such that for each $x \in U \setminus \{x_0\}$, $g(x) > 0$, then $$f = o(g) \text{ as } x \rightarrow x_0 \hspace{.25cm} \text{ iff } \hspace{.25cm}  \lim_{x \rightarrow x_0} \frac{\| f(x) \|}{\| g(x) \|} = 0$$
 \end{ex}	
 
 \begin{ex} \lex{}
 	Let $X$ and $Y$ a be normed vector spaces, $A \subset X$ open and $f:A \rightarrow Y$. Suppose that $0 \in A$. If $f(h) = o(\|h\|)$ as $h \rightarrow 0$, then for each $h \in X$,  $f(th) = o(|t|)$ as $t \rightarrow 0$.
 \end{ex}	
 
 \begin{proof}
 	Suppose that $f(h) = o(\|h\|)$ as $h \rightarrow 0$.  Let $h \in X$ and $\ep >0$. Choose $\del' >0 $ such that for each $h' \in B(0, \del')$, $h' \in A$ and 
 	$$\|f(h')\| \leq \frac{\ep}{\|h\|+1} \|h'\|$$ 
 	Choose $\del >0$ such that for each $t \in B(0,\del)$, $th \in B(0,\del')$. Let $t \in B(0,\del)$. Then 
 	\begin{align*}
 		\|f(th)\| 
 		&\leq \frac{\ep}{\|h\|+1} |t|\|h\| \\
 		&< \ep |t|
 	\end{align*}
 	So $f(th) = o(|t|)$ as $t \rightarrow 0$.
 \end{proof}		
 
 
 
 
 \begin{defn} \ld{}
 	Let $X$ be a topological space, $Y, Z$ be normed vector spaces, $f:X \rightarrow Y$, $g: X \rightarrow Z$ and $x_0 \in X \cup \{\infty\}$. Then we write $$f = O(g) \hspace{.5cm} \text{ as } x \rightarrow x_0$$ if there exists $U \in \MN(x_0)$ and $M \geq 0$ such that for each $x \in U$, $$\|f(x)\| \leq M\|g(x)\|$$
 \end{defn}
 
 
 
 
 
 
 
 
 
 
 
 
 
 
 
 
 
 
 
 
 
 
 
 
 
 
 
 \newpage
 \chapter{Categories}
 
 \tcr{move to notation?}
 
 \begin{defn}
 	We define the category of topological measure spaces, denoted $\TopMsrpos$, by 
 	\begin{itemize}
 		\item $\Obj(\TopMsrpos) \defeq \{(X, \mu): X \in \Obj(\Top) \text{ and } \mu \in M(X)\}$			
 		\item $\Hom_{\TopMsrpos}((X, \mu), (Y, \nu)) \defeq \Hom_{\Top}(X, Y) \cap \Hom_{\Msrpos}((X, \MB(X), \mu), (Y, \MB(Y), \nu))$
 	\end{itemize}
 \end{defn}
 
 
 
 
 
 
 
 
 
 
 
 
 
 
 
 
 
 
 
 
 
 
 
 
 
 
 
 
 
 
 
 
 \newpage
 \chapter{Vector Spaces}
 \tcr{it might be better to cover some category theory and write everything in terms of $\Hom_{\VectK}$ and $\Obj(\VectK)$}
 
 \section{Introduction}
 
 \begin{defn}
 	Let $X$ be a set, $\K$ a field, $+:X \times X \rightarrow X$ and $\cdot:\K \times X \rightarrow X$. Then $(X, +, \cdot)$ is said to be a \tbf{$\K$-vector space} if 
 	\begin{enumerate}
 		\item $(X, +)$ is an abelian group
 		\item 
 	\end{enumerate} 
 \end{defn}
 
 
 \begin{defn}
 	Let $(X, +_X, \cdot_X)$ and $(E, +_E, \cdot_E)$ be vector spaces. Suppose that $E \subset X$. Then $(E, +_E, \cdot_E)$ is said to be a subspace of $X$ if 
 	\begin{enumerate}
 		\item $+_E = +_X|_{E \times E}$
 		\item $\cdot_E = \cdot_X|_{\K \times E}$
 	\end{enumerate}
 \end{defn}
 
 \begin{ex}
 	Let $(X, +_X, \cdot_X)$ and $(E, +_E, \cdot_E)$ be vector spaces. Suppose that $E \subset X$. 
 \end{ex}
 
 \begin{ex}
 	Let $(X, +, \cdot)$ be a vector space and $E \subset X$. Then $E$ is a subspace of $X$
 \end{ex}
 
 
 \begin{defn}
 	Let $X$ be a vector space and $(E_j)_{j \in J}$ a collection of subspaces of $X$. Then $\bigcap\limits_{j \in J}E_j$ is a subspace of $X$. 
 \end{defn}
 
 \begin{proof}
 	Set $E \defeq \bigcap\limits_{j \in J}E_j$. Let $x,y \in E$ and $\lam \in \K$. Then for each $j \in J$, $x,y \in E_j$. Since for each $j \in J$, $E_j$ is a subspace of $X$, we have that for each $j \in J$, $x+ \lam y \in E_j$. Thus $x+\lam y \in E$. Since $x,y \in E$ and $\lam \in \K$ are arbitrary, \tcr{(cite exercise here)} we have that $E$ is a subspace of $X$. 
 \end{proof}
 
 
 
 
 
 
 
 
 
 
 
 
 
 
 
 
 
 
 
 
 
 
 
 
 
 
 \begin{defn}
 	Let $X, Y$ be vector spaces and $T:X \rightarrow Y$. Then $T$ is said to be \tbf{linear} if for each $x_1, x_2 \in X$ and $\lam \in \Lam$, 
 	\begin{enumerate}
 		\item $T(x_1 + x_2) = T(x_1) + T(x_2)$,
 		\item $T(\lam x_1) = \lam T(x_1)$.
 	\end{enumerate}
 	We define $L(X;Y) \defeq \{T:X \rightarrow Y: \text{ $T$ is linear}\}$. 
 \end{defn}
 
 \begin{ex}
 	Let $X,Y$ be vector spaces and $T : X \rightarrow Y$. Then $T$ is linear iff for each $x_1, x_2 \in X$ and $\lam \in \Lam$, 
 	$$T(x_1 + \lam x_2) = T(x_1) + \lam T(x_2)$$
 \end{ex}
 
 \begin{proof}
 	Clear. \tcr{(add details)}
 \end{proof}
 
 \begin{defn}
 	\tcr{define addition/scalar multiplication of linear maps}
 \end{defn}
 
 \begin{ex}
 	Let $X,Y$ be vector spaces. Then $L(X;Y)$ is a $\K$-vector space. 
 \end{ex}
 
 \begin{proof}
 	Clear
 \end{proof}
 
 \begin{defn} \ld{55001}\
 	Let $X$ be a vector space over $\K$ and $T :X \rightarrow \K$. Then $T$ is said to be a \tbf{linear functional on} $X$ if $T$ is linear. We define the \tbf{dual space of $X$}, denoted $X^*$, by $X^* \defeq \{ T:X \rightarrow \K: T \text{ is linear}\}$. 
 \end{defn}
 
 
 \begin{ex}
 	Let $X$ be a vector space. Then $X^*$ is a vector space. 
 \end{ex}
 
 \begin{proof}
 	Clear.
 \end{proof}
 
 
 
 
 
 
 
 
 
 
 
 
 
 
 
 
 
 
 
 
 
 
 
 
 
 
 
 
 
 
 
 \section{Bases}
 
 \begin{defn}
 	Let $X$ be a vector space and $(e_{\al})_{\al \in A} \subset X$. Then $(e_{\al})_{\al \in A}$ is said to be
 	\begin{itemize}
 		\item \tbf{linearly independent} if for each $(\al_j)_{j=1}^n \subset A$, $(\lam_j)_{j=1}^n \subset \K$, $\sum\limits_{j=1}^n \lam_j e_{\al_j} = 0$ implies that for each $j \in [n]$, $\lam_j = 0$.  
 		\item a \tbf{Hamel basis for $X$} if $(e_{\al})_{\al \in A}$ is linearly independent and $\spn (e_{\al})_{\al \in A} = X$. 
 	\end{itemize}
 \end{defn}
 
 \begin{ex}
 	\tcr{every vector space has a Hamel basis}
 \end{ex}
 
 \begin{proof}
 	
 \end{proof}
 
 \begin{ex}
 	
 \end{ex}
 
 
 \begin{ex}
 	Let $X$ be a $\K$-vector space and $x \in X$. Then $x = 0$ iff for each $\phi \in X^*$, $\phi(x) = 0$. 
 \end{ex}
 
 \begin{proof}\
 	\begin{itemize}
 		\item $(\implies):$ \\
 		Suppose that $x = 0$. Linearity implies that for each $\phi \in X^*$ $\phi(x) = 0$. 
 		\item $(\impliedby):$ \\
 		Conversely, suppose that $x \neq 0$. Define $\ep_x: \spn(x) \rightarrow \K$ by $\ep_x(\lam x) \defeq \lam$. Let $u,v \in \spn(x)$. Then there exists $\lam_u, \lam_v \in \K$ such that $u = \lam_u x$ and $v = \lam_v x$. Suppose that $u = v$. Then 
 		\begin{align*}
 			(\lam_u - \lam_v)x
 			& = \lam_u x - \lam_v x \\
 			& = u - v \\
 			& = 0
 		\end{align*}
 		Since $x \neq 0$, we have that $\lam_u - \lam_v = 0$ and therefore $\lam_u = \lam_v$. Hence  
 		\begin{align*}
 			\lam_u 
 			& = \ep_x(u) \\
 			& = \ep_x(v) \\
 			& = \lam_v.
 		\end{align*}
 		Thus $\ep_x$ is well defined. 
 	\end{itemize}
 \end{proof}
 
 
 
 
 
 
 
 
 
 
 
 
 
 
 
 
 
 
 
 
 
 
 
 
 
 
 
 
 
 
 
 
 
 
 
 \newpage
 \section{Multilinear Maps}
 
 \begin{defn}
 	Let $X_1, \cdots, X_n, Y$ be vector spaces and $T: \prod\limits_{j=1}^n X_j \rightarrow \K$. Then $T$ is said to be \tbf{multilinear} if for each $j_0 \in [n]$ and $(x_j)_{j=1}^n \in \prod\limits_{j=1}^n X_j$, $T(x_1, \ldots, x_{j_0 - 1}, \cdot, x_{j_0 + 1})$ is linear. $$L^n(X_1, \dots, X_n; Y) = \bigg\{T : \prod\limits_{j=1}^n X_j \rightarrow Y: T \text{ is multilinear}\bigg \}$$ 
 	If $X_1 = \cdots = X_n = X$, we write $L^n(X;Y)$ in place of $L^n (X, \dots, X; Y) $. 
 \end{defn}
 
 \begin{defn}
 	\tcr{define addition and scalar mult of multilinear maps}
 \end{defn}
 
 \begin{ex}
 	Let $X_1, \cdots, X_n, Y$ be vector spaces. Then $L^n(X_1, \ldots, X_n;Y)$ is a $\K$-vector space.
 \end{ex}
 
 \begin{proof}
 	content...
 \end{proof}
 
 \begin{ex}
 	Let $X_1, \cdots, X_n, Y, Z$ be $\K$-vector spaces, $\al \in L^n(X_1, \ldots, X_n;Y)$ and $\phi \in L^1(Y;Z)$. Then $\phi \circ \al \in L^n(X_1, \ldots, X_n; Z)$. 
 \end{ex}
 
 \begin{proof}
 	Let $(x_j)_{j=1}^n \in \prod\limits_{j=1}^n X_j$ and $j_0 \in [n]$. Define $f:X_{j_0} \rightarrow Y$ by 
 	$$f(a) \defeq \al(x_1, \ldots, x_{j_0-1}, a , x_{j_0+1}, \ldots, x_n) $$
 	Since $\al \in L^n(X_1, \ldots, X_n;Y)$, $f$ is linear. Since $\phi$ is linear, and $\phi \circ f$ is linear. Since $(x_j)_{j=1}^n \in \prod\limits_{j=1}^n X_j$ and $j_0 \in [n]$ are arbitrary, we have that $\phi \circ \al \in L^n(X_1, \ldots, X_n;Y)$. 
 \end{proof}
 
 
 
 
 
 
 
 
 
 
 
 
 
 
 
 
 
 
 
 
 
 
 
 
 
 
 
 
 
 
 
 
 
 
 
 
 
 
 
 
 
 
 \newpage
 \section{Tensor Products}
 
 \begin{defn}
 	Let $X, Y$ and $T$ be vector spaces over $\K$ and $\al \in L^2(X, Y; T)$. Then $(T, \al)$ is said to be a \tbf{tensor product of $X$ and $Y$} if for each vector space $Z$ and $\be \in L^2(X, Y; Z)$, there exists a unique $\phi \in L^1(T;Z)$ such that $\phi \circ \al = \be$, i.e. the following diagram commutes:
 	\[ 
 	\begin{tikzcd}
 		X \times Y \arrow[r, "\al"] \arrow[dr, "\be"'] 	
 		& T  \arrow[d, dashed, "\phi"] \\
 		& Z 
 	\end{tikzcd}
 	\] 
 \end{defn}
 
 \begin{ex}
 	Let $X, Y, S, T$ be vector spaces, $\al \in L^2(X, Y; S)$ and $\be \in L^2(X, Y; T)$. Suppose that $(S, \al)$ and $(T, \be)$ are tensor products of $X$ and $Y$. Then $S$ and $T$ are isomorphic. 
 \end{ex}
 
 \begin{proof}
 	Since $(T, \be)$ is a tensor product of $X$ and $Y$, $\be \in L^2(X,Y; T)$ there exists a unique $f \in L^1(T;T)$ such that $f\ circ \be = \be$, i.e. the following diagram commutes: 
 	\[ 
 	\begin{tikzcd}
 		& T \arrow[dd, dashed, "f"] \\
 		X \times Y \arrow[ur, "\be"] \arrow[dr, "\be"'] 
 		&   \\
 		& T 
 	\end{tikzcd}
 	\] 
 	Since $\id_T \in L^1(T;T)$ and $\id_T \circ \be = \be$, we have that $f = \id_T$. Since $(S, \al)$ is a tensor product of $X$ and $Y$, there exists a unique $\phi: S \rightarrow T$ such that $\phi \circ \al = \be$, i.e. the following diagram commutes: 
 	\[ 
 	\begin{tikzcd}
 		X \times Y \arrow[r, "\al"] \arrow[dr, "\be"'] 	
 		& S  \arrow[d, dashed, "\phi"] \\
 		& T 
 	\end{tikzcd}
 	\] 
 	Similarly, since $(T, \be)$ is a tensor product of $X$ and $Y$, there exists a unique $\psi: T \rightarrow S$ such that $\psi \circ \be = \al$, i.e. the following diagram commutes: 
 	\[ 
 	\begin{tikzcd}
 		X \times Y \arrow[r, "\be"] \arrow[dr, "\al"'] 	
 		& T  \arrow[d, dashed, "\psi"] \\
 		& S 
 	\end{tikzcd}
 	\] 
 	Therefore 
 	\begin{align*}
 		(\phi \circ \psi) \circ \be 
 		& = \phi \circ (\psi \circ \be) \\
 		& = \phi \circ \al \\
 		& = \be, 
 	\end{align*}
 	i.e. the following diagram commutes:
 	\[ 
 	\begin{tikzcd}
 		& T \arrow[d, dashed, "\psi"] \\
 		X \times Y \arrow[ur, "\be"] \arrow[dr, "\be"'] \arrow[r, "\al"]	
 		& S  \arrow[d, dashed, "\phi"] \\
 		& T 
 	\end{tikzcd}
 	\implies
 	\begin{tikzcd}
 		& T \arrow[dd, dashed, "\phi \circ \psi"] \\
 		X \times Y \arrow[ur, "\be"] \arrow[dr, "\be"'] 
 		&  \\
 		& T 
 	\end{tikzcd}
 	\] 
 	By uniqueness of $f \in L^1(T;T)$, we have that 
 	\begin{align*}
 		\id_T
 		& = f \\
 		& = \phi \circ \psi 
 	\end{align*}
 	A similar argument implies that $\psi \circ \phi = \id_S$. Hence $\phi$ and $\psi$ are isomorphisms with $\phi^{-1} = \psi$. Hence $S$ and $T$ are isomorphic.
 \end{proof}
 
 \begin{defn}
 	Let $X, Y$ be vector spaces, $x \in X$ and $y \in Y$. We define $x \otimes y: X^* \times Y^* \rightarrow  \K$ by $x \otimes y(\phi, \psi) \defeq \phi(x) \psi(y)$.  
 \end{defn}
 
 \begin{ex}
 	Let $X, Y$ be vector spaces, $x \in X$ and $y \in Y$. Then $x \otimes y \in L^2(X^*, Y^*; \K)$. 
 \end{ex}
 
 \begin{proof}
 	Let $\phi_1, \phi_2 \in X^*$, $\psi \in Y^*$ and $\lam \in \K$. Then 
 	\begin{align*}
 		x \otimes y(\phi_1 + \lam \phi_2, \psi) 
 		& = [\phi_1 + \lam \phi_2 (x)] \psi(y) \\
 		& = \phi_1(x)\psi(y) + \lam \phi_2(x)\psi(y) \\
 		& = x \otimes y(\phi_1, \psi) + \lam x \otimes y(\phi_2, \psi)
 	\end{align*}
 	Since $\phi_1, \phi_2 \in X^*$, $\psi \in Y^*$ and $\lam \in \K$ are arbitrary, we have that for each $\psi \in Y^*$, $x \otimes y(\cdot, \psi)$ is linear. Similarly for each $\phi \in X^*$, $x \otimes y(\phi, \cdot)$ is linear. Hence $x \otimes y$ is bilinear and $x \otimes y \in L^2(X^*, Y^*; \K)$. 
 \end{proof}
 
 \begin{defn}
 	Let $X, Y$ be vector spaces. We define  
 	\begin{itemize}
 		\item the \tbf{tensor product of $X$ and $Y$}, denoted $X \otimes Y \subset L^2(X^*, Y^*; \K)$, by 
 		$$X \otimes Y \defeq \spn(\text{$x \otimes y: x \in X$ and $y \in Y$}),$$
 		\item the \tbf{tensor map}, denoted $\otimes: X \times Y \rightarrow X \otimes Y$, by $\otimes(x, y) \defeq x \otimes y$.
 	\end{itemize}
 \end{defn}
 
 \begin{ex}
 	Let $X,Y$ be vector spaces, $(x_j)_{j=1}^n \subset X$ and $(y_j)_{j=1}^n \subset Y$. The following are equivalent:
 	\begin{enumerate}
 		\item $\sum\limits_{j=1}^n  x_j \otimes y_j = 0$
 		\item for each $\phi \in X^*$ and $\psi \in Y^*$, $\sum\limits_{j=1}^n  \phi(x_j) \psi(y_j) = 0$
 		\item for each $\phi \in X^*$, $\sum\limits_{j=1}^n  \phi(x_j)  y_j = 0$
 		\item for each $\psi \in Y^*$, $\sum\limits_{j=1}^n  \psi(y_j) x_j = 0$
 	\end{enumerate}
 \end{ex}
 
 \begin{proof}\
 	\begin{enumerate}
 		\item $(1) \implies (2):$ \\
 		Suppose that $\sum\limits_{j=1}^n x_j \otimes y_j = 0$. Let $\phi \in X^*$ and $\psi \in Y^*$. Then 
 		\begin{align*}
 			\sum\limits_{j=1}^n  \phi(x_j) \psi(y_j)
 			& = \phi \bigg( \sum\limits_{j=1}^n \psi(y_j) x_j \bigg) \\
 			& = 
 		\end{align*}
 		\item 
 		\item 
 	\end{enumerate}
 \end{proof}
 
 \begin{ex}
 	Let $X, Y$ be vector spaces. Then $(X \otimes Y, \otimes)$ is a tensor product of $X$ and $Y$. 
 \end{ex}
 
 \begin{proof}
 	Let $Z$ be a vector space and $\al \in L^2(X, Y; Z)$. Define $\phi: X \otimes Y \rightarrow Z$ by $\phi \bigg( \sum\limits_{j=1}^n \lam_j x_j \otimes y_j) \defeq \sum\limits_{j =1}^n \lam_j \al(x_j, y_j)$.  
 	\begin{itemize}
 		\item \tbf{(well defined):} \\
 		Let $u \in X \otimes Y$. Then there exist $(\lam_j)_{j=1}^n \subset \K$, $(x_j)_{j=1}^n \subset X$, $(y_j)_{j=1}^n \subset Y$ such that $u = \sum\limits_{j=1}^n \lam_j x_j \otimes y_j$. Suppose that $u = 0$. Let $\phi \in Z^*$. Then $\phi \circ \al \in L^2(X,Y; Z)$.  
 	\end{itemize}
 \end{proof}
 
 
 
 
 
 
 
 
 
 
 
 
 
 
 
 
 
 
 
 
 
 
 
 
 
 
 
 
 
 
 
 
 
 
 
 
 
 
 
 
 
 
 
 
 
 
 
 
 
 
 
 
 
 
 
 
 
 
 
 
 
 
 
 
 
 
 
 
 
 
 
 
 
 
 
 
 
 
 \backmatter
 \begin{thebibliography}{4}
 	\bibitem{algebra} \href{https://github.com/carsonaj/Mathematics/blob/master/Introduction\%20to\%20Algebra/Introduction\%20to\%20Algebra.pdf}{Introduction to Algebra}
 	
 	\bibitem{analysis}  \href{https://github.com/carsonaj/Mathematics/blob/master/Introduction\%20to\%20Analysis/Introduction\%20to\%20Analysis.pdf}{Introduction to Analysis}	
 	
 	\bibitem{foranal}  \href{https://github.com/carsonaj/Mathematics/blob/master/Introduction\%20to\%20Fourier\%20Analysis/Introduction\%20to\%20Fourier\%20Analysis.pdf}{Introduction to Fourier Analysis}
 	
 	\bibitem{measure}  \href{https://github.com/carsonaj/Mathematics/blob/master/Introduction\%20to\%20Measure\%20and\%20Integration/Introduction\%20to\%20Measure\%20and\%20Integration.pdf}{Introduction to Measure and Integration}
 	
 	
 	
 \end{thebibliography}
 
 
 
 
 
 
 
	
	
	
	
	
	
	
	
	
	
	
	
	
	
	
	
	
	
	
	
	
	
	
\end{document}