%% filename: amsbook-template.tex
%% version: 1.1
%% date: 2014/07/24
%%
%% American Mathematical Society
%% Technical Support
%% Publications Technical Group
%% 201 Charles Street
%% Providence, RI 02904
%% USA
%% tel: (401) 455-4080
%%      (800) 321-4267 (USA and Canada only)
%% fax: (401) 331-3842
%% email: tech-support@ams.org
%% 
%% Copyright 2006, 2008-2010, 2014 American Mathematical Society.
%% 
%% This work may be distributed and/or modified under the
%% conditions of the LaTeX Project Public License, either version 1.3c
%% of this license or (at your option) any later version.
%% The latest version of this license is in
%%   http://www.latex-project.org/lppl.txt
%% and version 1.3c or later is part of all distributions of LaTeX
%% version 2005/12/01 or later.
%% 
%% This work has the LPPL maintenance status `maintained'.
%% 
%% The Current Maintainer of this work is the American Mathematical
%% Society.
%%
%% ====================================================================

%    AMS-LaTeX v.2 driver file template for use with amsbook
%
%    Remove any commented or uncommented macros you do not use.

\documentclass{book}

%    For use when working on individual chapters
%\includeonly{}

%    For use when working on individual chapters
%\includeonly{}

%    Include referenced packages here.
\usepackage[left =.5in, right = .5in, top = 1in, bottom = 1in]{geometry} 
\usepackage{amsmath}
\usepackage{amsthm}
\usepackage{amssymb}
\usepackage{setspace}
\usepackage{mathtools}
\usepackage{tikz}  
\usepackage{tikz-cd}
\usepackage{tkz-fct}
\usepackage{pgfplots}
\usepackage{environ}
\usepackage{tikz-cd} 
\usepackage{enumitem}
\usepackage{color}   %May be necessary if you want to color links
%\usepackage{xr}

\usepackage{hyperref}
\hypersetup{
	colorlinks=true, %set true if you want colored links
	linktoc=all,     %set to all if you want both sections and subsections linked
	linkcolor=black,  %choose some color if you want links to stand out
	urlcolor=cyan
}
\usepackage[symbols,nogroupskip,sort=none]{glossaries-extra}

\pgfplotsset{every axis/.append style={
		axis x line=middle,    % put the x axis in the middle
		axis y line=middle,    % put the y axis in the middle
		axis line style={<->,color=black}, % arrows on the axis
		xlabel={$x$},          % default put x on x-axis
		ylabel={$y$},          % default put y on y-axis
}}


\theoremstyle{definition}
\newtheorem{definition}{Definition}[subsection]
\newtheorem{defn}[definition]{Definition}
\newtheorem{note}[definition]{Note}
\newtheorem{ax}[definition]{Axiom}
\newtheorem{thm}[definition]{Theorem}
\newtheorem{lem}[definition]{Lemma}
\newtheorem{prop}[definition]{Proposition}
\newtheorem{cor}[definition]{Corollary}
\newtheorem{conj}[definition]{Conjecture}
\newtheorem{ex}[definition]{Exercise}
\newtheorem{exmp}[definition]{Example}
\newtheorem{soln}[definition]{Solution}

\setcounter{tocdepth}{3}

% hide proofs
\newif\ifhideproofs
%\hideproofstrue %uncomment to hide proofs
\ifhideproofs
\NewEnviron{hide}{}
\let\proof\hide
\let\endproof\endhide
\fi

% lower-case greek
\newcommand{\al}{\alpha}
\newcommand{\be}{\beta}
\newcommand{\gam}{\gamma}
\newcommand{\del}{\delta}
\newcommand{\ep}{\epsilon}
\newcommand{\ze}{\zeta} 
\newcommand{\kap}{\kappa} 
\newcommand{\lam}{\lambda}  
\newcommand{\sig}{\sigma} 
\newcommand{\omi}{\omicron}
\newcommand{\up}{\upsilon}
\newcommand{\om}{\omega}

% upper-case greek
\newcommand{\Gam}{\Gamma}
\newcommand{\Del}{\Delta}
\newcommand{\Lam}{\Lambda} 
\newcommand{\Sig}{\Sigma} 
\newcommand{\Om}{\Omega}

% blackboard bold
\newcommand{\C}{\mathbb{C}}
\newcommand{\E}{\mathbb{E}}
\newcommand{\F}{\mathbb{F}}
\renewcommand{\H}{\mathbb{H}}
\newcommand{\K}{\mathbb{K}}
\newcommand{\N}{\mathbb{N}}
\renewcommand{\O}{\mathbb{O}}
\newcommand{\Q}{\mathbb{Q}}
\newcommand{\R}{\mathbb{R}}
\renewcommand{\S}{\mathbb{S}}
\newcommand{\T}{\mathbb{T}}
\newcommand{\V}{\mathbb{V}}
\newcommand{\Z}{\mathbb{Z}}

% math caligraphic
\newcommand{\MA}{\mathcal{A}}
\newcommand{\MB}{\mathcal{B}}
\newcommand{\MC}{\mathcal{C}}
\newcommand{\MD}{\mathcal{D}}
\newcommand{\ME}{\mathcal{E}}
\newcommand{\MF}{\mathcal{F}}
\newcommand{\MG}{\mathcal{G}}
\newcommand{\MH}{\mathcal{H}}
\newcommand{\MI}{\mathcal{I}}
\newcommand{\MJ}{\mathcal{J}}
\newcommand{\MK}{\mathcal{K}}
\newcommand{\ML}{\mathcal{L}}
\newcommand{\MM}{\mathcal{M}}
\newcommand{\MN}{\mathcal{N}}
\newcommand{\MO}{\mathcal{O}}
\newcommand{\MP}{\mathcal{P}}
\newcommand{\MQ}{\mathcal{Q}}
\newcommand{\MR}{\mathcal{R}}
\newcommand{\MS}{\mathcal{S}}
\newcommand{\MT}{\mathcal{T}}
\newcommand{\MU}{\mathcal{U}}
\newcommand{\MV}{\mathcal{V}}
\newcommand{\MW}{\mathcal{W}}
\newcommand{\MX}{\mathcal{X}}
\newcommand{\MY}{\mathcal{Y}}
\newcommand{\MZ}{\mathcal{Z}}

% mathfrak
\newcommand{\MFX}{\mathfrak{X}}
\newcommand{\MFg}{\mathfrak{g}}
\newcommand{\MFh}{\mathfrak{h}}

% tilde 
\newcommand{\tMA}{\tilde{\MA}}
\newcommand{\tMB}{\tilde{\MB}}
\newcommand{\tU}{\tilde{U}}
\newcommand{\tV}{\tilde{V}}
\newcommand{\tphi}{\tilde{\phi}}
\newcommand{\tpsi}{\tilde{\psi}}
\newcommand{\tF}{\tilde{F}}

\newcommand{\iid}{\stackrel{iid}{\sim}}





% label/reference
% internal label/reference
\newcommand{\lex}[1]{\label{ex:#1}}
\newcommand{\rex}[1]{Exercise \ref{ex:#1}}

\newcommand{\ld}[1]{\label{defn:#1}}
\newcommand{\rd}[1]{Definition \ref{defn:#1}}

\newcommand{\lax}[1]{\label{ax:#1}}
\newcommand{\rax}[1]{Axiom \ref{ax:#1}}

\newcommand{\lfig}[1]{\label{fig:#1}}
\newcommand{\rfig}[1]{Figure \ref{fig:#1}}

% external reference
\newcommand{\extrex}[2]{Exercise \ref{#1-ex:#2}}

\newcommand{\extrd}[2]{Definition \ref{#1-defn:#2}}

\newcommand{\extrax}[2]{Axiom \ref{#1-ax:#2}}

\newcommand{\extrfig}[2]{Figure \ref{#1-fig:#2}}

% external documents (EDIT HERE)
%\externaldocument[analysis-]{"/home/carson/Desktop/Github/Mathematics/Introduction to Analysis/Introduction to Analysis.tex"}




% math operators
\DeclareMathOperator{\supp}{supp}
\DeclareMathOperator{\sgn}{sgn}
\DeclareMathOperator{\spn}{span}
\DeclareMathOperator{\Iso}{Iso}
\DeclareMathOperator{\Eq}{Eq}
\DeclareMathOperator{\id}{id}
\DeclareMathOperator{\Aut}{Aut}
\DeclareMathOperator{\Endo}{End}
\DeclareMathOperator{\Homeo}{Homeo}
\DeclareMathOperator{\Sym}{Sym}
\DeclareMathOperator{\Alt}{Alt}
\DeclareMathOperator{\cl}{cl}
\DeclareMathOperator{\Int}{Int}
\DeclareMathOperator{\bal}{bal}
\DeclareMathOperator{\cyc}{cyc}
\DeclareMathOperator{\cnv}{conv}
\DeclareMathOperator{\epi}{epi}
\DeclareMathOperator{\dom}{dom}
\DeclareMathOperator{\cod}{cod}
\DeclareMathOperator{\codim}{codim}
\DeclareMathOperator{\Obj}{Obj}
\DeclareMathOperator{\Derivinf}{Deriv^{\infty}}
\DeclareMathOperator{\Hom}{Hom}
\DeclareMathOperator*{\argmax}{arg\,max}
\DeclareMathOperator*{\argmin}{arg\,min}
\DeclareMathOperator{\diam}{\text{diam}}
\DeclareMathOperator{\rnk}{\text{rank}}
\DeclareMathOperator{\tr}{\text{tr}}
\DeclareMathOperator{\prj}{\text{proj}}
\DeclareMathOperator{\nab}{\nabla}
\DeclareMathOperator{\diag}{\text{diag}}
\DeclareMathOperator*{\ind}{\text{ind}}
\DeclareMathOperator*{\ar}{\text{arity}}
\DeclareMathOperator*{\cur}{\text{cur}}
\DeclareMathOperator*{\Part}{\text{Part}}
\DeclareMathOperator{\Var}{\text{Var}}
\DeclareMathOperator*{\FIP}{\text{FIP}} 
\DeclareMathOperator*{\Fun}{\text{Fun}} 
\DeclareMathOperator*{\Rel}{\text{Rel}} 
\DeclareMathOperator*{\Cons}{\text{Cons}} 
\DeclareMathOperator*{\Sg}{\text{Sg}} 
\DeclareMathOperator*{\ot}{\otimes}
\DeclareMathOperator{\uni}{Uni}

% Algebra
\DeclareMathOperator{\inv}{\text{inv}}
\DeclareMathOperator{\mult}{\text{mult}}
\DeclareMathOperator{\smult}{\text{smult}}

% category theory
\DeclareMathOperator*{\Set}{\text{\tbf{Set}}}
\DeclareMathOperator*{\BanAlg}{\text{\tbf{BanAlg}}}
\DeclareMathOperator*{\Meas}{\text{\tbf{Meas}}}
\DeclareMathOperator*{\TopMeas}{\text{\tbf{TopMeas}}}
\DeclareMathOperator*{\Msrpos}{\text{\tbf{Msr}}_{+}}
\DeclareMathOperator*{\TopMsrpos}{\text{\tbf{TopMsr}}_{+}}
\DeclareMathOperator*{\TopRadMsrpos}{\text{\tbf{TopRadMsr}}_{+}}
\DeclareMathOperator*{\TopRadMsrone}{\text{\tbf{TopRadMsr}}_{1}}
\DeclareMathOperator*{\MsrC}{\text{\tbf{Msr}}_{\C}} 
\DeclareMathOperator*{\TopMsrC}{\text{\tbf{TopMsr}}_{\C}} 
\DeclareMathOperator*{\TopRadMsrC}{\text{\tbf{TopRadMsr}}_{\C}} 
\DeclareMathOperator*{\Maninf}{\text{\tbf{Man}}^{\infty}} 
\DeclareMathOperator*{\ManBndinf}{\text{\tbf{ManBnd}}^{\infty}} 
\DeclareMathOperator*{\Man0}{\text{\tbf{Man}}^{0}}
\DeclareMathOperator*{\Buninf}{\text{\tbf{Bun}}^{\infty}} 
\DeclareMathOperator*{\VecBuninf}{\text{\tbf{VecBun}}^{\infty}} 
\DeclareMathOperator*{\Field}{\text{\tbf{Field}}} 
\DeclareMathOperator*{\Mon}{\text{\tbf{Mon}}} 
\DeclareMathOperator*{\Grp}{\text{\tbf{Grp}}}
\DeclareMathOperator*{\Semgrp}{\text{\tbf{Semgrp}}}
\DeclareMathOperator*{\LieGrp}{\text{\tbf{LieGrp}}} 
\DeclareMathOperator*{\Alg}{\text{\tbf{Alg}}} 
\DeclareMathOperator*{\Vect}{\text{\tbf{Vect}}} 
\DeclareMathOperator*{\Mod}{\text{\tbf{Mod}}}
\DeclareMathOperator*{\Rep}{\text{\tbf{Rep}}} 
\DeclareMathOperator*{\URep}{\text{\tbf{URep}}}
\DeclareMathOperator*{\Ban}{\text{\tbf{Ban}}} 
\DeclareMathOperator*{\Hilb}{\text{\tbf{Hilb}}} 
\DeclareMathOperator*{\Prob}{\text{\tbf{Prob}}} 
\DeclareMathOperator*{\PrinBuninf}{\text{\tbf{PrinBun}}^{\infty}}

\DeclareMathOperator*{\Top}{\text{\tbf{Top}}}
\DeclareMathOperator*{\TopField}{\text{\tbf{TopField}}} 
\DeclareMathOperator*{\TopMon}{\text{\tbf{TopMon}}} 
\DeclareMathOperator*{\TopGrp}{\text{\tbf{TopGrp}}}
\DeclareMathOperator*{\TopVect}{\text{\tbf{TopVect}}} 
\DeclareMathOperator*{\TopEq}{\text{\tbf{TopEq}}}

\DeclareMathOperator*{\VectR}{\text{\tbf{Vect}}_{\R}}
\DeclareMathOperator*{\VectC}{\text{\tbf{Vect}}_{\C}} 
\DeclareMathOperator*{\VectK}{\text{\tbf{Vect}}_{\K}}
\DeclareMathOperator*{\Cat}{\text{\tbf{Cat}}}
\DeclareMathOperator*{\0}{\mbf{0}}
\DeclareMathOperator*{\1}{\mbf{1}}


\DeclareMathOperator*{\Cone}{\text{\tbf{Cone}}}

\DeclareMathOperator*{\Cocone}{\text{\tbf{Cocone}}}


% Algebra
\DeclareMathOperator{\End}{\text{End}} 
\DeclareMathOperator{\rep}{\text{Rep}} 




% notation
\renewcommand{\r}{\rangle}
\renewcommand{\l}{\langle}
\renewcommand{\div}{\text{div}}
\renewcommand{\Re}{\text{Re} \,}
\renewcommand{\Im}{\text{Im} \,}
\newcommand{\Img}{\text{Img} \,}
\newcommand{\grad}{\text{grad}}
\newcommand{\tbf}[1]{\textbf{#1}}
\newcommand{\tcb}[1]{\textcolor{blue}{#1}}
\newcommand{\tcr}[1]{\textcolor{red}{#1}}
\newcommand{\mbf}[1]{\mathbf{#1}}
\newcommand{\ol}[1]{\overline{#1}}
\newcommand{\ub}[1]{\underbar{#1}}
\newcommand{\tl}[1]{\tilde{#1}}
\newcommand{\p}{\partial}
\newcommand{\Tn}[1]{T^{r_{#1}}_{s_{#1}}(V)}
\newcommand{\Tnp}{T^{r_1 + r_2}_{s_1 + s_2}(V)}
\newcommand{\Perm}{\text{Perm}}
\newcommand{\wh}[1]{\widehat{#1}}
\newcommand{\wt}[1]{\widetilde{#1}}
\newcommand{\defeq}{\vcentcolon=}
\newcommand{\Con}{\text{Con}}
\newcommand{\ConKos}{\text{Con}_{\text{Kos}}}
\newcommand{\trl}{\triangleleft}
\newcommand{\trr}{\triangleright}
\newcommand{\alg}{\text{alg}}
\newcommand{\Triv}{\text{Triv}}
\newcommand{\Der}{\text{Der}}
\newcommand{\cnj}{\text{conj}}

\newcommand{\lcm}{\text{lcm}}
\newcommand{\Imax}{\MI_{\text{max}}}


\DeclareMathOperator*{\Rl}{\text{Re}}
\DeclareMathOperator*{\Imn}{\text{Imn}}



% limits
\newcommand{\limfn}{\liminf \limits_{n \rightarrow \infty}}
\newcommand{\limpn}{\limsup \limits_{n \rightarrow \infty}}
\newcommand{\limn}{\lim \limits_{n \rightarrow \infty}}
\newcommand{\convt}[1]{\xrightarrow{\text{#1}}}
\newcommand{\conv}[1]{\xrightarrow{#1}} 
\newcommand{\seq}[2]{(#1_{#2})_{#2 \in \N}}

% intervals
\newcommand{\RG}{[0,\infty]}
\newcommand{\Rg}{[0,\infty)}
\newcommand{\Rgp}{(0,\infty)}
\newcommand{\Ru}{(\infty, \infty]}
\newcommand{\Rd}{[\infty, \infty)}
\newcommand{\ui}{[0,1]}

% integration \newcommand{\dm}{\, d m}
\newcommand{\dmu}{\, d \mu}
\newcommand{\dnu}{\, d \nu}
\newcommand{\dlam}{\, d \lambda}
\newcommand{\dP}{\, d P}
\newcommand{\dQ}{\, d Q}
\newcommand{\dm}{\, d m}
\newcommand{\dsh}{\, d \#}

% abreviations 
\newcommand{\lsc}{lower semicontinuous}

% misc
\newcommand{\as}[1]{\overset{#1}{\sim}}
\newcommand{\astx}[1]{\overset{\text{#1}}{\sim}}
\newcommand{\io}{\text{ i.o.}}
%\newcommand{\ev}{\text{ ev.}}
\newcommand{\Ll}{L^1_{\text{loc}}(\R^n)}

\newcommand{\loc}{\text{loc}}
\newcommand{\BV}{\text{BV}}
\newcommand{\NBV}{\text{NBV}}
\newcommand{\TV}{\text{TV}}

\newcommand{\op}[1]{\mathcal{#1}^{\text{op}}}


% Glossary - Notation
\glsxtrnewsymbol[description={finite measures on $(X, \MA)$}]{n000001}{$\MM_+(X, \MA)$}
\glsxtrnewsymbol[description={velocity}]{v}{\ensuremath{v}}


\makeindex

\begin{document}
	
	\frontmatter
	
	\title{Introduction to Algebra}
	
	%    Remove any unused author tags.
	
	%    author one information
	\author{Carson James}
	\thanks{}
	
	\date{}
	
	\maketitle
	
	%    Dedication.  If the dedication is longer than a line or two,
	%    remove the centering instructions and the line break.
	%\cleardoublepage
	%\thispagestyle{empty}
	%\vspace*{13.5pc}
	%\begin{center}
	%  Dedication text (use \\[2pt] for line break if necessary)
	%\end{center}
	%\cleardoublepage
	
	%    Change page number to 6 if a dedication is present.
	\setcounter{page}{4}
	
	\tableofcontents
	\printunsrtglossary[type=symbols,style=long,title={Notation}]
	
	%    Include unnumbered chapters (preface, acknowledgments, etc.) here.
	%\include{}
	
	\mainmatter
	%    Include main chapters here.
	%\include{}
	
	\chapter*{Preface}
	\addcontentsline{toc}{chapter}{Preface}
	
	\begin{flushleft}
		\href{https://creativecommons.org/licenses/by-nc-sa/4.0/legalcode.txt}{cc-by-nc-sa}
	\end{flushleft}
	
	\newpage
	
	
	
	
	
	
	
	
	
	
	
	
	
	
	
	
	
	
	
	
	
	
	
	
	
	
	
	
	
	
	
	
	
	
	
	
	
	
	
	
	
	
	\part{Sets and Order}
	
	\chapter{Set Theory}

\section{Operations and Relations}

\begin{defn}\
	\begin{itemize}
		\item We define $[0] \defeq \varnothing$ and for $k \in \N$, we define $[k] \defeq \{1, \ldots, k\}$. 
		\item Let $S$ be a set and $k \in \N_0$. We define the \tbf{set of $k$-tupels with entries in $S$}, denoted $S^k$, by 
		$$S^k \defeq \{u: [k] \rightarrow S\}$$
		\item Let $S$ be a set. We define the \tbf{set of all tuples with entries in $S$}, denoted $S^*$, by 
		$$S^* \defeq \bigcup_{k \in \N_0} S^k$$
		\item Let $S$ be a set and $k \in \N_0$. We define the \tbf{set of $k$-ary operations on $S$}, denoted $\MF^k(S)$, by $\MF^k(S) \defeq S^{(S^k)}$. We define the \tbf{set of finitary operations on $S$}, denoted $\MF^*(S)$, by
		$$\MF^*(S) \defeq \bigcup_{k \in \N_0} \MF^k(S)$$
		\item Let $S$ be a set. We define the \tbf{operation arity map}, denoted $\ar: \MF^*(S) \rightarrow \N_0$, by 
		$$\ar f \defeq k, \quad f \in \MF^k(S)$$
		\item Let $S$ be a set, $\MF \subset \MF^*(S)$ and $k \in \N_0$. We define the \tbf{$k$-ary members of $\MF$}, denoted $\MF_k$, by 
		$$\MF_k \defeq \MF \cap \MF^k(S)$$
		\item Let $S$ be a set and $k \in \N_0$. We define the \tbf{set of $k$-ary relations on $S$}, denoted $\MR^k(S)$, by $\MR^k(S) \defeq \MP(S^k)$. We define the \tbf{set of finitary relations on $S$}, denoted $\MR^*(S)$, by
		$$\MR^*(S) \defeq \bigcup_{k \in \N_0} \MR^k(S)$$
		\item Let $S$ be a set. We define the \tbf{arity map}, denoted $\ar: \MR^*(S) \rightarrow \N_0$, by 
		$$\ar R \defeq k, \quad f \in \MR^k(S)$$
		\item Let $S$ be a set, $\MR \subset \MR^*(S)$ and $k \in \N_0$. We define the \tbf{$k$-ary members of $\MR$}, denoted $\MR_k$, by 
		$$\MR_k \defeq \MR \cap \MR^k(S)$$
	\end{itemize}
\end{defn}

\begin{defn}
	Let $S$ be a set, $k \geq 2$ and $f \in \MF^k(S)$. Then $f$ is said to be
	\begin{itemize}
		\item \tbf{associative} if for each $x_1, \ldots, x_k, x_{k+1}, \ldots, x_{k + (k-1)} \in S$, 
		\begin{align*}
			f(f(x_1, \ldots, x_k) x_{k+1}, \ldots, x_{k + (k-1)}) 
			& = f(x_1, f(x_2, \ldots, x_{k+1}), x_{k+2}, \ldots, x_{k + (k-1)}) \\
			& \vdots \\
			& = f(x_1, \ldots, x_{k-1}, f(x_k, \ldots, x_{k + (k-1)}) )
		\end{align*}
		\item \tbf{symmetric} if for each $x_1, \ldots, x_k \in S$, $\sig \in S_k$, $f(x_1, \ldots, x_k) = f(x_{\sig(1)}, \ldots, x_{\sig(k)})$.
		\item \tbf{idempotent} if for each $x \in S$,
		$f(x, \ldots, x) = x$
	\end{itemize}
\end{defn}

\begin{defn}
	Let $S$ be a set, $\MF \subset \MF^*(S)$ and $C \subset S$. Then $C$ is said to be  \tbf{$\MF$-closed} if for each $k \in \N_0$, $f \in \MF_k$ and $a \in C^k$, $f(a) \in C$.
\end{defn}

\begin{ex}
	Let $S$ be a set, $\MF \subset \MF^*(S)$ and $\MC \subset \MP(S)$. If for each $C \in \MC$, $C$ is  $\MF$-closed, then $\bigcap\limits_{C \in \MC} C$ is $\MF$-closed
	\tcr{need special case where $k=0$? maybe trivially true?}
\end{ex}

\begin{proof}
	Suppose that for each $C \in \MC$, $C$ is  $\MF$-closed. Let $k \in \N_0$, $f \in \MF_k$, $a_1, \ldots, a_k \in \bigcap\limits_{C \in \MC} C$ and $C_0 \in \MC$. Since $C_0 \in \MC$, we have that 
	\begin{align*}
		a_1, \ldots, a_k 
		& \in \bigcap_{C \in \MC} C \\
		& \subset C_0
	\end{align*}
	Since $C_0$ is $\MF$-closed, we have that $f(a_1, \ldots, a_k) \in C_0$. Since $C_0 \in \MC$ is arbitrary, we have that for each $C \in \MC$, $f(a_1, \ldots, a_k) \in C$. Hence $f(a_1, \ldots, a_k) \in \bigcap\limits_{C \in \MC} C$. Since $k \in \N_0$ and $a_1, \ldots, a_k \in \bigcap\limits_{C \in \MC} C$ are arbitrary, we have that $\bigcap\limits_{C \in \MC} C$ is $\MF$-closed.
\end{proof}


	
	
	
	
	
	
	
	
	
	
	
	
	
	
	
	
	
	
	
	
	
	
	
	
	
	
	
	
	
	
	
	
	
	
	
	
	
	
	
	
	
	
	
	
	
	
	
	\newpage
	\chapter{Ordered Sets}
	
	\section{To Do}
	\tcr{at this point, there should be a simple structure capturing prosets and $\doar S$, measurable spaces and $\sig(\MA)$, topological spaces and $\tau(\ME)$, groups/rings/$\cdots$/algebras and $\l S\r$/$(E)$/$\cdots$/$(F)$}. It seems like we need a category $\MC$ with some structure, another category $\MC_F$ which is like $\MC$, but less structured and some forgetful-like functor $F: \MC\rightarrow \MC_F$. eg, $F:\tbf{Pro} \rightarrow \Set$ or $F:\Top \rightarrow \Set$ and we need an associated poset $\MP$ containing the structure forgotten by $F$, e.g. the lower sets of $X$ or the set of topologies on $X$ and we need a ``generating" or ``contained" object $\ME$ in $\MC_F$ in the sense that there is at least one object $A$ in $\MC$ and monomirphism $\iota_A: \ME \rightarrow F(A)$ in $\MC_F$. We then need a minimal element in the structure poset $\MP$ in some universal sense relating to these monomorphisms. \tcr{Ask people who know category theory}
	
	\section{Prosets}
	
	\subsection{Introduction}
	
	\begin{defn} \ld{def:orderings:prosets:0001} \tbf{Preordered Set:} \\
		Let $X$ be a set and $\leq \, \subset X \times X$ a binary relation on $X$. Then 
		\begin{itemize}
			\item $\leq$ is said to be a \tbf{preorder on $X$} if
			\begin{enumerate}
				\item for each $a \in X$, $a \leq a$
				\item for each $a, b, c \in X$, $a \leq b$ and $b \leq c$ implies that $a \leq c$
			\end{enumerate}
			\item $(X, \leq)$ is said to be a \tbf{preordered set} or \tbf{proset} if $\leq$ is a preorder on $X$.
		\end{itemize}
	\end{defn}
	
	\begin{defn} \ld{def:orderings:prosets:0002}
		Let $(X, \leq)$ be a proset. We define the \tbf{dual order of $\leq$ on $X$}, denoted $\lop$, by $a \lop b$ iff $b \leq a$.
	\end{defn}
	
	\begin{ex} \lex{ex:orderings:prosets:0003}
		Let $(X, \leq)$ be a proset. Then $\lop$ is a preorder on $X$.
	\end{ex}
	
	\begin{proof}\
		\begin{enumerate}
			\item Let $a \in X$. Since $a \leq a$, we have that $a \lop a$. 
			\item Let $a,b,c \in X$. Suppose that $a \lop b$ and $b \lop c$. Then $b \leq a$ and $c \leq b$. Hence $c \leq a$. Thus $a \lop c$. 
		\end{enumerate}
		Therefore $\lop$ is a preorder on $X$. 
	\end{proof}
	
	
	
	
	
	
	
	
	
	
	
	
	
	
	
	
	
	
	
	
	
	
	
	
	
	
	
	
	
	
	
	
	
	
	
	
	
	
	
	
	
	
	
	
	
	
	
	
	
	
	
	\subsection{Products}
	
	\begin{defn} \ld{def:orderings:prosets:0004}
		Let $(A, \leq_A)$ and $(B, \leq_B)$ be prosets. We define the 
		\begin{itemize}
			\item \tbf{product preorder of $\leq_A$ and $\leq_B$ on $A \times B$}, denoted $\leq_A \otimes \leq_B$ by $(a_1,b_1) \leq_A \otimes \leq_B (a_2, b_2)$ iff $a_1 \leq_A a_2$ and $b_1 \leq_B b_2$.
			\item \tbf{product proset of $(A, \leq_A)$ and $(B, \leq_B)$}, denoted $(A, \leq_A) \otimes (B, \leq_B)$ by $(A, \leq_A) \otimes (B, \leq_B) \defeq (A \times B, \leq_A \otimes \leq_B)$
		\end{itemize}
	\end{defn}
	
	\begin{ex} \lex{ex:orderings:prosets:0005}
		\tcr{probably need to change notation since $\otimes$ might be reserved for something else.}
		Let $(A, \leq_A)$ and $(B, \leq_B)$ be prosets. Then 
		\begin{enumerate}
			\item $\leq_A \otimes \leq_B$ is a preorder on $A \times B$,
			\item $(A, \leq_A) \otimes (B, \leq_B)$ is a proset.
		\end{enumerate}
	\end{ex}
	
	\begin{proof}\
		\begin{enumerate}
			\item 
			\begin{enumerate}
				\item Let $(a, b) \in A \times B$. Then $a \leq_A a$ and $b \leq_B b$. Therefore $(a, b) \leq_A \otimes \leq_B (a ,b)$.
				\item Let $(a_1, b_1), (a_2, b_2), (a_3, b_3) \in A \times B$. Suppose that $(a_1, b_1) \leq_A \otimes \leq_B (a_2, b_2)$ and $(a_2, b_2) \leq_A \otimes \leq_B (a_3, b_3)$. Then $a_1 \leq_A a_2$, $a_2 \leq_A a_3$, $b_1 \leq_B b_2$ and $b_2 \leq_B b_3$. Therefore $a_1 \leq_A a_3$ and $b_1 \leq_B b_3$. Hence $(a_1, b_1) \leq_A \otimes \leq_B (a_3, b_3)$.
			\end{enumerate}
			Hence $\leq_A \otimes \leq_B$ is a preorder on $A \times B$.
			\item Since $\leq_A \otimes \leq_B$ is a preorder on $A \times B$, $(X, \leq)$ is a proset.
		\end{enumerate}
	\end{proof}
	
	\begin{defn} \ld{def:orderings:prosets:0006}
		Let $(X, \leq)$ be a proset. We define $\sim_{\leq} \subset X \times X$ by $a \sim_{\leq} b$ iff $a \leq b$ and $b \leq a$.
	\end{defn}
	
	\begin{ex} \lex{ex:orderings:prosets:0007}
		Let $(X, \leq)$ be a proset. Then $\sim_{\leq}$ is an equivalence relation on $X$.
	\end{ex}
	
	\begin{proof}
		Let $x,y, z \in X$. 
		\begin{enumerate}
			\item Since $x \leq x$, $x \sim_{\leq} x$.
			\item Suppose that $x \sim_{\leq} y$. Then $x \leq y$ and $y \leq x$. Thus $y \sim_{\leq} x$. 
			\item Suppose that $x \sim_{\leq} y$ and $y \sim_{\leq} z$. Then $x \leq y$, $y \leq x$, $y \leq z$ and $z \leq y$. Therefore $x \leq z$ and $z \leq x$. Hence $x \sim_{\leq} z$.  
		\end{enumerate}
	\end{proof}
	
	
	
	
	
	
	
	
	
	
	
	
	
	
	
	
	
	
	
	
	
	
	
	
	
	
	
	
	
	
	
	
	
	
	
	
	
	
	
	
	
	
	
	
	
	
	
	
	
	
	
	
	
	
	
	\subsection{Upper and Lower Sets}
	
	\begin{defn} \ld{def:orderings:prosets:0008}
		Let $(X, \leq)$ be a proset and $A \subset X$. Then 
		\begin{itemize}
			\item $A$ is said to be a \tbf{$\leq$-upper set} if for each $a \in A$ and $x \in X$, $a \leq x$ implies that $x \in A$.
			\item $A$ is said to be an \tbf{$\leq$-lower set} if for each $a \in A$ and $x \in X$, $x \leq a$ implies that $x \in A$.
		\end{itemize}
	\end{defn}
	
	\begin{note}
		When the context is clear, we say $A$ is a
		\begin{itemize}
			\item ``upper set" instead of ``$\leq$-upper set"
			\item ``lower set" instead of ``$\leq$-lower set"
		\end{itemize}
	\end{note}
	
	\begin{ex} \lex{ex:orderings:prosets:0009}
		Let $(X, \leq)$ be a proset. Then 
		\begin{enumerate}
			\item $X$ is a $\leq$-upper set 
			\item $X$ is a $\leq$-lower set
		\end{enumerate}
	\end{ex}
	
	\begin{proof}\
		\begin{enumerate}
			\item Let $a, x \in X$. Suppose that $a \leq x$. By assumption, $x \in X$. Since $a,x \in X$ with $a \leq x$ are arbitrary, we have that for each $a,x \in X$, $a \leq x$ implies that $x \in A$. Hence $X$ is a $\leq$-upper set.
			\item Similar to $(1)$.
		\end{enumerate}
	\end{proof}
	
	\begin{ex} \lex{ex:orderings:prosets:0010}
		Let $(X, \leq)$ be a proset and $A \subset X$. Then 
		\begin{enumerate}
			\item $A$ is a $\leq$-upper set iff $A$ is a $\lop$-lower set
			\item $A$ is a $\leq$-lower set iff $A$ is a $\lop$-upper set
		\end{enumerate}
	\end{ex}
	
	\begin{proof}\
		\begin{enumerate}
			\item 
			\begin{itemize}
				\item $(\implies)$: \\
				Suppose that $A$ is a $\leq$-upper set. Let $a \in A$ and $x \in X$. Suppose that $x \lop a$. Then $a \leq x$. Since $A$ is a $\leq$-upper set, $x \in A$. Since $a \in A$ and $x \in X$ with $x \lop a$ is arbitrary, we have that for each $a \in A$ and $x \in X$, $x \lop a$ implies that $x \in A$. Hence $A$ is a $\lop$-lower set. 
				\item $(\impliedby)$: \\
				Suppose that $A$ is a $\lop$-lower set. Let $a \in A$ and $x \in X$. Suppose that $a \leq x$. Then $x \lop a$. Since $A$ is a $\lop$-lower set, $x \in A$. Since $a \in A$ and $x \in X$ with $a \leq x$ is arbitrary, we have that for each $a \in A$ and $x \in X$, $a \leq x$ implies that $x \in A$. Hence $A$ is a $\leq$-upper set. 
			\end{itemize} 
			\item Similar to $(1)$.
		\end{enumerate}
	\end{proof}
	
	\begin{ex} \lex{ex:orderings:prosets:0011}
		Let $(X, \leq)$ be a proset and $A \subset X$. Then 
		\begin{enumerate}
			\item $A$ is an upper set iff $A^c$ is a lower set
			\item $A$ is a lower set iff $A^c$ is an upper set
		\end{enumerate}
	\end{ex}
	
	\begin{proof}\
		\begin{enumerate}
			\item 
			\begin{itemize}
				\item $(\implies)$: \\
				Suppose that $A$ is an upper set. Let $b \in A^c$ and $x \in X$. Suppose that $x \leq b$. For the sake of contradiction, suppose that $x \in A$. Since $x \leq b$ and $A$ is an upper set, $b \in A$. This is a contradiction since $b \in A^c$. Hence $x \in A^c$. Since $b \in A^c$ and $x \in X$ with $x \leq b$ are arbitrary, we have that for each $b \in A^c$ and $x \in X$, $x \leq b$ implies that $x \in A^c$. Thus $A^c$ is a lower set.
				\item $(\impliedby)$: \\
				Suppose that $A^c$ is a $\leq$-lower set. \rex{ex:orderings:prosets:0010} implies that $A^c$ is a $\lop$-upper set. The previous part implies that $A$ is a $\lop$-lower set. Another application of \rex{ex:orderings:prosets:0010} implies that $A$ is a $\leq$-upper set. 
			\end{itemize}
			\item Similar to $(1)$.
		\end{enumerate}
	\end{proof}
	
	\begin{ex} \lex{ex:orderings:prosets:0012}
		Let $(X, \leq)$ be a proset, $(E_{\al})_{\al \in A} \subset \MP(X)$. 
		\begin{enumerate}
			\item If for each $\al \in A$, $E_{\al}$ is an upper set, then 
			\begin{enumerate}
				\item $\bigcup\limits_{\al \in A} E_{\al}$ is an upper set
				\item $\bigcap\limits_{\al \in A} E_{\al}$ is an upper set
			\end{enumerate}
			\item If for each $\al \in A$, $E_{\al}$ is a lower set, then 
			\begin{enumerate}
				\item $\bigcup\limits_{\al \in A} E_{\al}$ is a lower set
				\item $\bigcap\limits_{\al \in A} E_{\al}$ is an lower set
			\end{enumerate}
		\end{enumerate}
	\end{ex}
	
	\begin{proof}\
		\begin{enumerate}
			\item Suppose that for each $\al \in A$, $E_{\al}$ is an upper set. Set $E \defeq \bigcup\limits_{\al \in A} E_{\al}$. 
			\begin{enumerate}
				\item Let $e \in E$ and $x \in X$. Suppose that $e \leq x$. Since $e \in E$, there exists $\al \in A$ such that $e \in E_{\al}$. Since $E_{\al}$ is an upper set and $e \leq x$, we have that 
				\begin{align*}
					x
					& \in E_{\al} \\
					& \subset \bigcup\limits_{\al \in A} E_{\al} \\
					& = E.
				\end{align*}  
				Since $e \in E$ and $x \in X$ with $e \leq x$ are arbitrary, we have that for each $e \in E$ and $x \in X$, $e \leq x$ implies that $x \in E$. Hence $E$ is an upper set. 
				\item Let $\al \in A$. Since $E_{\al}$ is a $\leq$-upper set, \rex{ex:orderings:prosets:0011} implies that $E_{\al}^c$ is a $\leq$-lower set. \rex{ex:orderings:prosets:0010} then implies that $E_{\al}^c$ is a $\lop$-upper set. Since $\al \in A$ is arbitrary, we have that for each $\al \in A$, $E_{\al}^c$ is a $\lop$-upper set. The previous part implies that $\bigcup\limits_{\al \in A} E_{\al}^c$ is a $\lop$-upper set. Since $\bigcap\limits_{\al \in A} E_{\al} = \bigg( \bigcup\limits_{\al \in A} E_{\al}^c \bigg)^c$, \rex{ex:orderings:prosets:0011} implies that $\bigcap\limits_{\al \in A} E_{\al}$ is a $\lop$-lower set. \rex{ex:orderings:prosets:0010} then implies that $\bigcap\limits_{\al \in A} E_{\al}$ is a $\leq$-upper set. 
			\end{enumerate}
			\item Similar to $(1)$. 
		\end{enumerate}
	\end{proof}
	
	\begin{defn} \ld{def:orderings:prosets:0013}
		Let $(X, \leq)$ be a proset and $S \subset X$. We define the
		\begin{itemize}
			\item \tbf{$\leq$-upper sets containing $S$}, denoted $\MU(S, \leq) \subset \MP(X)$, by $\MU(S, \leq) \defeq \{U \subset X: \text{$U$ is a $\leq$-upper set and $S \subset U$}\}$ 
			\item \tbf{$\leq$-lower sets containing $S$}, denoted $\ML(S, \leq) \subset \MP(X)$, by $\ML(S, \leq) \defeq \{L \subset X: \text{$L$ is a $\leq$-lower set and $S \subset L$}\}$.
			\item \tbf{$\leq$-upper set generated by $S$}, denoted $\upar (S, \leq)$, by $\upar (S, \leq) \defeq \bigcap\limits_{U \in \MU(S, \leq)} U$
			\item \tbf{$\leq$-lower set generated by $S$}, denoted $\doar (S, \leq)$, by $\doar (S, \leq) \defeq \bigcap\limits_{L \in \ML(S, \leq)} L$
		\end{itemize}
	\end{defn}
	
	\begin{note}\
		\begin{itemize}
			\item When the context is clear, we write $\MU(S), \ML(S)$, $\upar S$ and $\doar S$ in place of $\MU(S, \leq), \ML(S, \leq)$, $\upar (S, \leq)$ and $\doar (S, \leq)$ respectively. 
			\item If $S = \{s\}$, we write $\upar s$ and $\doar s$ in place of $\upar S$ and $\doar S$ respectively.
			\item \rex{ex:orderings:prosets:0009} implies that $X \in \MU(S)$ and $X \in \ML(S)$.
			\item \rex{ex:orderings:prosets:0012} implies that $\upar S \in \MU(S)$ and $\doar S \in \ML(S)$.
		\end{itemize}
	\end{note}
	
	\begin{ex} \lex{ex:orderings:prosets:0014}
		Let $(X, \leq)$ be a proset and $S \subset X$. Then 
		\begin{enumerate}
			\item $\MU(S, \lop) = \ML(S, \leq)$,
			\item $\ML(S, \lop) = \MU(S, \leq)$,
			\item $\upar (S, \lop) = \doar(S, \leq)$,
			\item $\doar (S, \lop) = \upar(S, \leq)$.
		\end{enumerate}
	\end{ex}
	
	\begin{proof}\
		\begin{enumerate}
			\item 
			\begin{itemize}
				\item Let $L \in \MU(S, \lop)$. Then $S \subset L$ and $L$ is a $\lop$-upper set. \rex{ex:orderings:prosets:0010} then implies that $L$ is a $\leq$-lower set. Hence $L \in \ML(S, \leq)$. Since $L \in \MU(S, \lop)$ is arbitrary, we have that for each $L \in \MU(S, \lop)$, $L \in \ML(S, \leq)$. Thus $\MU(S, \lop) \subset \ML(S, \leq)$.
				\item Let $L \in \ML(S, \leq)$. Then $S \subset L$ and $L$ is a $\leq$-lower set. Another application of \rex{ex:orderings:prosets:0010} implies that $L$ is a $\lop$-upper set. Hence $L \in \MU(S, \lop)$. Since $L \in \ML(S, \leq)$ is arbitrary, we have that for each $L \in \ML(S, \leq)$, $L \in \MU(S, \lop)$. Thus $\ML(S, \leq) \subset \MU(S, \lop)$.
			\end{itemize}
			Since $\MU(S, \lop) \subset \ML(S, \leq)$ and $\ML(S, \leq) \subset \MU(S, \lop)$, we have that $\MU(S, \lop) = \ML(S, \leq)$.
			\item By $(1)$, we have that 
			\begin{align*}
				\ML(S, \lop)
				& = \MU(S, {(\lop)}^{\text{op}}) \\ 
				& = \MU(S, \leq).
			\end{align*}
			\item Part $(1)$ implies that
			\begin{align*}
				\upar (S, \lop)
				& = \bigcap\limits_{U \in \MU(S, \lop)} U \\
				& = \bigcap\limits_{U \in \ML(S, \leq)} U \\
				& = \doar (S, \lop).
			\end{align*}
			\item Similar to $(3)$.
		\end{enumerate}
	\end{proof}
	
	\begin{ex} \lex{ex:orderings:prosets:0014.1}
		Let $(X, \leq)$ be a proset and $S \subset X$. Then
		\begin{enumerate}
			\item $S \in \MU(S, \leq)$ iff $\upar S = S$ 
			\item $S \in \ML(S, \leq)$ iff $\doar S = S$
		\end{enumerate}
	\end{ex}
	
	\begin{proof}\
		\begin{enumerate}
			\item
			\begin{itemize}
				\item $(\implies)$: \\
				Suppose that $S \in \MU(S, \leq)$. Then 
				\begin{align*}
					\upar S
					& = \bigcap\limits_{U \in \MU(S, \leq)} U \\
					& \subset S \\
					& \subset \upar S.
				\end{align*}
				Hence $\upar S = S$.
				\item $(\impliedby)$: \\
				Suppose that $\upar S = S$. Then 
				\begin{align*}
					S
					& = \upar S \\
					& \in \MU(S, \leq).
				\end{align*}
			\end{itemize}
			\item Similar to $(1)$.
		\end{enumerate}
	\end{proof}
	
	\begin{ex} \lex{ex:orderings:prosets:0015}
		Let $(X, \leq)$ be a proset and $a \in X$. Then 
		\begin{enumerate}
			\item $\upar a = \{x \in X: a \leq x\}$
			\item $\doar a = \{x \in X: x \leq a\}$
		\end{enumerate}
	\end{ex}
	
	\begin{proof}\
		\begin{enumerate}
			\item Define $U \subset X$ by $U \defeq \{x \in X: a \leq x\}$. 
			\begin{itemize}
				\item Let $u \in U$ and $x \in X$. Suppose that $u \leq x$. Then  
				\begin{align*}
					a
					& \leq u \\
					& \leq x.
				\end{align*}
				Hence $x \in U$. Since $u \in U$ and $x \in X$ with $u \leq x$ are arbitrary, we have that for each $u \in U$ and $x \in X$, $u \leq x$ implies that $x \in U$. Hence $U$ is an upper set. Since $a \leq a$, $a \in U$. By definition, $U \in \MU(a)$. Therefore 
				\begin{align*}
					\upar a
					& = \bigcap\limits_{U' \in \MU(a)} U' \\
					& \subset U. 
				\end{align*}
				\item Let $x \in U$ and $U' \in \MU(a)$. Since $x \in U$, $a \leq x$. Since $U' \in \MU(a)$, $U'$ is an upper set and $a \in U'$. Thus $x \in U'$. Since $U' \in \MU(a)$ is arbitrary, we have that for each $U' \in \MU(a)$, $x \in U'$. Thus 
				\begin{align*}
					x 
					& \in \bigcap\limits_{U' \in \MU(a)} U' \\
					& = \upar a.
				\end{align*} 
				Since $x \in U$ is arbitrary, we have that for each $x \in U$, $x \in \upar a$. Hence $U \subset \upar a$.
			\end{itemize}
			Since $\upar a \subset U$ and $U \subset \upar a$, we have that $\upar a = U$.
			\item \rex{ex:orderings:prosets:0010} and $(1)$ imply that 
			\begin{align*}
				\doar (a, \leq)
				& = \upar(a, \lop) \\
				& =  \{x \in X: a \lop x\} \\
				& = \{x \in X: x \leq a\}.
			\end{align*}
		\end{enumerate}
	\end{proof}
	
	\begin{ex} \lex{ex:orderings:prosets:0016}
		Let $(X, \leq)$ be a proset and $S_1, S_2 \subset X$. If $S_1 \subset S_2$, then
		\begin{enumerate}
			\item $\upar S_1 \subset \upar S_2$ 
			\item $\doar S_1 \subset \doar S_2$ 
		\end{enumerate}
	\end{ex}
	
	\begin{proof} 
		Suppose that $S_1 \subset S_2$.
		\begin{enumerate}
			\item
			Since $\upar S_2 \in \MU(S_2)$, $S_2 \subset \upar S_2$ and $\upar S_2$ is an upper set. Since
			\begin{align*}
				S_1 
				& \subset S_2 \\
				& \subset \upar S_2,
			\end{align*} 
			we have that $\upar S_2 \in \MU(S_1)$. Therefore
			\begin{align*}
				\upar S_1
				& = \bigcap\limits_{U \in \MU(S_1)} U \\
				& \subset \upar S_2.
			\end{align*}
			\item Part $(1)$ and \rex{ex:orderings:prosets:0014} implies that 
			\begin{align*}
				\doar (S_1, \leq)
				& = \upar (S_1, \lop) \\
				& \subset \upar (S_2, \lop) \\
				& = \doar (S_2, \leq).
			\end{align*}
		\end{enumerate}
	\end{proof}
	
	\begin{ex} \lex{ex:orderings:prosets:0017}
		Let $(X, \leq)$ be a proset and $a,b \in X$. Then the following are equivalent:
		\begin{enumerate}
			\item $a \leq b$,
			\item $\upar b \subset \upar a$,
			\item $\doar a \subset \doar b$.
		\end{enumerate}
	\end{ex}
	
	\begin{proof}\
		\begin{enumerate}
			\item $(1) \implies (2)$: \\
			Suppose that $a \leq b$. Since $\upar a \in \MU(a)$, we have that $a \in \upar a$ and $\upar a$ is an upper set. Since $a \leq b$, we have that $b \in \upar a$. Hence $\upar a \in \MU(b)$. Therefore
			\begin{align*}
				\upar b
				& = \bigcap\limits_{U \in \MU(b)} U \\
				& \subset \upar a.
			\end{align*}
			\item $(2) \implies (3)$: \\
			Suppose that $\upar (b, \leq) \subset \upar (a, \leq)$. \rex{ex:orderings:prosets:0015} then implies that
			\begin{align*}
				b
				& \in \upar (b, \leq) \\
				& \subset \upar (a, \leq) \\
				& = \{x \in X: a \leq x\}.
			\end{align*}
			Hence $a \leq b$. Thus $b \lop a$. \rex{ex:orderings:prosets:0014} and part $(1) \implies (2)$ then imply that 
			\begin{align*}
				\doar(a, \leq)
				& = \upar(a, \lop) \\
				& \subset \upar(b, \lop) \\
				& = \doar(b, \leq).
			\end{align*} 
			\item $(3) \implies (1)$: \\
			Suppose that $\doar a \subset \doar b$. \rex{ex:orderings:prosets:0015} then implies that 
			\begin{align*}
				a
				& \in \doar a \\
				& \subset \doar b \\
				& = \{x \in X: x \leq b\}. 
			\end{align*} 
			Hence $a \leq b$.
		\end{enumerate}
	\end{proof}
	
	\begin{ex} \lex{ex:orderings:prosets:0018}
		Let $(X, \leq)$ be a proset and $S \subset X$. Then 
		\begin{enumerate}
			\item $\upar S = \bigcup\limits_{s \in S} \upar s$
			\item $\doar S = \bigcup\limits_{s \in S} \doar s$
		\end{enumerate}
	\end{ex}
	
	\begin{proof}\
		\begin{enumerate}
			\item Define $U \subset X$ by $U \defeq \bigcup\limits_{s \in S} \upar s$.
			\begin{itemize}
				\item Since for each $s \in S$, $\upar s$ is an upper set, \rex{ex:orderings:prosets:0012} implies that $U$ is an upper set. Let $s \in S$. Then 
				\begin{align*}
					s 
					& \in \upar s \\
					& \subset \bigcup\limits_{s \in S} \upar s \\
					& = U.
				\end{align*}
				Since $s \in S$ is arbitrary, we have that for each $s \in S$, $s \in U$. Hence $S \subset U$. Therefore $U \in \MU(S)$ and 
				\begin{align*}
					\upar S
					& = \bigcap\limits_{U' \in \MU(S)} U' \\
					& \subset U.
				\end{align*}
				\item Let $x \in U$ and $U' \in \MU(S)$. Since $x \in U$, there exists $s \in S$ such that $x \in \upar s$. \rex{ex:orderings:prosets:0015} then implies that $s \leq x$. Since $U' \in \MU(S)$, $U'$ is an upper set and $S \subset U'$. Then
				\begin{align*}
					s
					& \in S \\
					& \subset U'.
				\end{align*}
				Since $U'$ is an upper set and $s \leq x$, $x \in U'$. Since $U' \in \MU(S)$ is arbitrary, we have that for each $U' \in \MU(S)$, $x \in U'$. Thus 
				\begin{align*}
					x
					& \in \bigcap\limits_{U' \in \MU(S)} U' \\
					& = \upar S.
				\end{align*}
				Since $x \in U$ is arbitrary, we have that for each $x \in U$, $x \in \upar S$. Hence $U \subset \upar S$.
			\end{itemize}
			Since $\upar S \subset U$ and $U \subset \upar S$, we have that $\upar S = U$.
			\item Similar to $(1)$.
		\end{enumerate}
	\end{proof}
	
	\begin{ex} \lex{ex:orderings:prosets:0019}
		Let $(X, \leq)$ be a proset and $(E_{\al})_{\al \in A} \subset \MP(X)$. Then 
		\begin{enumerate}
			\item $\upar \bigcup\limits_{\al \in A} E_{\al} = \bigcup\limits_{\al \in A} \upar E_{\al}$
			\item $\doar \bigcup\limits_{\al \in A} E_{\al} = \bigcup\limits_{\al \in A} \doar E_{\al}$
		\end{enumerate}
	\end{ex}
	
	\begin{proof}\
		\begin{enumerate}
			\item 
			\begin{itemize}
				\item Let $x \in \upar \bigcup\limits_{\al \in A} E_{\al}$. \rex{ex:orderings:prosets:0018} implies that there exists $y \in \bigcup\limits_{\al \in A} E_{\al}$ such that $x \in \upar y$. Then there exists $\al_0 \in A$ such that $y \in E_{\al_0}$. Since $\{y\} \subset E_{\al_0}$, \rex{ex:orderings:prosets:0016} implies that $\upar y \subset \upar E_{\al_0}$. Therefore
				\begin{align*}
					x
					& \in \upar y \\
					& \subset \upar E_{\al_0} \\
					& \subset \bigcup\limits_{\al \in A} \upar E_{\al}.
				\end{align*}
				Since $x \in \upar \bigcup\limits_{\al \in A} E_{\al}$ is arbitrary, we have that for each $x \in \upar \bigcup\limits_{\al \in A} E_{\al}$, $x \in \bigcup\limits_{\al \in A} \upar E_{\al}$. Hence $\upar \bigcup\limits_{\al \in A} E_{\al} \subset \bigcup\limits_{\al \in A} \upar E_{\al}$.
				\item Let $x \in \bigcup\limits_{\al \in A} \upar E_{\al}$. Then there exists $\al_0 \in A$ such that $x \in \upar E_{\al_0}$. \rex{ex:orderings:prosets:0018} implies that there exists $y \in E_{\al_0}$ such that $x \in \upar y$. Since 
				\begin{align*}
					\{y\} 
					& \subset E_{\al_0} \\
					& \subset \bigcup\limits_{\al \in A} E_{\al},
				\end{align*}
				\rex{ex:orderings:prosets:0016} implies that
				\begin{align*}
					x
					& \in \upar y \\
					& \subset \upar \bigcup\limits_{\al \in A} E_{\al}. 
				\end{align*}
				Since $x \in \bigcup\limits_{\al \in A} \upar E_{\al}$ is arbitrary, we have that for each $x \in \bigcup\limits_{\al \in A} \upar E_{\al}$, $x \in \upar \bigcup\limits_{\al \in A} E_{\al}$. Hence $\bigcup\limits_{\al \in A} \upar E_{\al} \subset \upar \bigcup\limits_{\al \in A} E_{\al}$. 
			\end{itemize}
			Since $\upar \bigcup\limits_{\al \in A} E_{\al} \subset \bigcup\limits_{\al \in A} \upar E_{\al}$ and $\bigcup\limits_{\al \in A} \upar E_{\al} \subset \upar \bigcup\limits_{\al \in A} E_{\al}$, we have that $\upar \bigcup\limits_{\al \in A} E_{\al} = \bigcup\limits_{\al \in A} \upar E_{\al}$.
			\item Similar to $(1)$.
		\end{enumerate}
	\end{proof}
	
	\begin{defn} \ld{def:orderings:prosets:0020}
		Let $(X, \leq)$ be a proset and $a \in X$. Then $a$ is said to be
		\begin{itemize}
			\item \tbf{$\leq$-maximal} if for each $x \in X$, $a \leq x$ implies that $x \sim_{\leq} a$.  
			\item \tbf{$\leq$-minimal} if for each $x \in X$, $x \leq a$ implies that $x \sim_{\leq} a$. 
		\end{itemize} 
	\end{defn}
	
	\begin{note}
		When the context is clear, we write ``maximal" and ``minimal" instead of ``$\leq$-maximal" and ``$\leq$-minimal" respectively. 
	\end{note}
	
	\begin{ex} \lex{ex:orderings:prosets:0021}
		Let $(X, \leq)$ be a proset and $a \in X$. Then 
		\begin{enumerate}
			\item $a$ is maximal iff $\upar a = \pi_{X/\sim_{\leq}}(a)$
			\item $a$ is minimal iff $\doar a = \pi_{X/\sim_{\leq}}(a)$
		\end{enumerate}
	\end{ex}
	
	\begin{proof}\
		\begin{enumerate}
			\item 
			\begin{itemize}
				\item $(\implies)$: \\
				Suppose that $a$ is maximal. 
				\begin{itemize}
					\item Let $x \in \upar a$. \rex{ex:orderings:prosets:0015} implies $a \leq x$. Since $a$ is maximal, $x \sim_{\leq} a$. Thus $x \in \pi_{X/\sim_{\leq}}(a)$. Since $x \in \upar a$ is arbitrary, we have that for each $x \in \upar a$, $x \in \pi_{X/\sim_{\leq}}(a)$. Hence $\upar a \subset \pi_{X/\sim_{\leq}}(a)$.
					\item Let $x \in \pi_{X/\sim_{\leq}}(a)$. Then $a \sim_{\leq} x$. Hence $a \leq x$ and $x \leq a$. Since $a \leq x$, we have that $x \in \upar a$. Since $x \in \pi_{X/\sim_{\leq}}(a)$ is arbitrary, we have that for each $x \in \pi_{X/\sim_{\leq}}(a)$, $x \in \upar a$. Hence $\pi_{X/\sim_{\leq}}(a) \subset \upar a$.
				\end{itemize}
				Since $\upar a \subset \pi_{X/\sim_{\leq}}(a)$ and $\pi_{X/\sim_{\leq}}(a) \subset \upar a$, we have that $\upar a = \pi_{X/\sim_{\leq}}(a)$.
				\item $(\impliedby)$: \\
				Suppose that $\upar a = \pi_{X/\sim_{\leq}}(a)$. Let $x \in X$. Suppose that $a \leq x$. Then 
				\begin{align*}
					x 
					& \in \upar a \\
					& = \pi_{X/\sim_{\leq}}(a).
				\end{align*}
				Thus $x \sim_{\leq} a$. Since $x \in X$ with $a \leq x$ is arbitrary, we have that for each $x \in X$, $a \leq x$ implies that $x \sim_{\leq} a$. Hence $a$ is maximal.
			\end{itemize}
			\item Similar to $(1)$.
		\end{enumerate}
	\end{proof}
	
	\begin{defn} \ld{def:orderings:prosets:0022}
		Let $(X, \leq)$ be a proset and $A \subset X$. 
		\begin{itemize}
			\item Let $x \in X$. Then $x$ is said to be a
			\begin{itemize}
				\item \tbf{$\leq$-upper bound of $A$} if for each $a \in A$, $a \leq x$
				\item \tbf{$\leq$-lower bound of $A$} if for each $a \in A$, $x \leq a$
			\end{itemize} 
			\item We define 
			\begin{itemize}
				\item $\ubd (A, \leq) \defeq \{x \in X: \text{$x$ is a $\leq$-upper bound of $A$}\}$ 
				\item $\lbd(A, \leq) \defeq \{x \in X: \text{$x$ is a $\leq$-lower bound of $A$}\}$ 
			\end{itemize} 
			\item Then $A$ is said to be 
			\begin{itemize}
				\item \tbf{$\leq$-bounded above} if $\ubd(A, \leq) \neq \varnothing$
				\item \tbf{$\leq$-bounded below} if $\lbd(A, \leq) \neq \varnothing$
			\end{itemize}
		\end{itemize}
	\end{defn}
	
	\begin{note}
		When the context is clear, we write 
		\begin{itemize}
			\item ``upper bound" and ``lower bound" instead of ``$\leq$-upper bound" and ``$\leq$-lower bound" respectively
			\item $\ubd A$ and $\lbd A$ in place of $\ubd (A, \leq)$ and $\lbd (A, \leq)$  respectively
			\item ``bounded above" and ``bounded below" instead of ``$\leq$-bounded above" and ``$\leq$-bounded below" respectively
		\end{itemize}
	\end{note}
	
	\begin{ex} \lex{ex:orderings:prosets:0023}
		Let $(X, \leq)$ be a proset and $A \subset X$. Then 
		\begin{enumerate}
			\item $\ubd A = \bigcap\limits_{x \in A} \upar x$
			\item $\lbd A = \bigcap\limits_{x \in A} \doar x$
		\end{enumerate}
	\end{ex}
	
	\begin{proof}\
		\begin{enumerate}
			\item 
			\begin{itemize}
				\item Let $a \in \ubd A$ and $x \in A$. Then $x \leq a$. Hence $a \in \upar x$. Since $x \in A$ is arbitrary, we have that for each $x \in A$, $a \in \upar x$. Thus $a \in \bigcap\limits_{x \in A} \upar x$. Since $a \in \ubd A$ is arbitrary, we have that for each $a \in \ubd A$, $a \in \bigcap\limits_{x \in A} \upar x$. Hence $\ubd A \subset \bigcap\limits_{x \in A} \upar x$.  
				\item Let $a \in \bigcap\limits_{x \in A} \upar x$ and $x_0 \in A$. Then $a \in \upar x_0$. Hence $x_0 \leq a$. Since $x_0 \in A$ is arbitrary, we have that for each $x_0 \in A$, $x_0 \leq a$. Hence $a \in \ubd A$. Since $a \in \bigcap\limits_{x \in A} \upar x$ is arbitrary, we have that for each $a \in \bigcap\limits_{x \in A} \upar x$, $a \in \ubd A$. Thus $\bigcap\limits_{x \in A} \upar x \subset \ubd A$.
			\end{itemize}
			Since $\ubd A \subset \bigcap\limits_{x \in A} \upar x$ and $\bigcap\limits_{x \in A} \upar x \subset \ubd A$, we have that $\ubd A = \bigcap\limits_{x \in A} \upar x$.
			\item Similar to $(1)$.
		\end{enumerate}
	\end{proof}
	
	\begin{defn} \ld{def:orderings:prosets:0024}
		Let $(X, \leq)$ be a proset and $A \subset X$.
		\begin{itemize}
			\item Let $x \in X$. Then $x$ is said to be a 
			\begin{itemize}
				\item \tbf{supremum of $A$} or \tbf{least upper bound of $A$}, if 
				\begin{enumerate}
					\item $x \in \ubd A$
					\item for each $y \in \ubd A$, $x \leq y$ 
				\end{enumerate}
				\item \tbf{infimum of $A$} or \tbf{greatest lower bound of $A$}, denoted $x \in \inf A$, if 
				\begin{enumerate}
					\item $x \in \lbd A$
					\item for each $y \in \lbd A$, $y \leq x$. 
				\end{enumerate}
			\end{itemize}
			\item 
			\begin{itemize}
				\item We define $\sup A \subset X$ by $\sup A \defeq \{x \in X: \text{$x$ is a supremum of $A$}\}$
				\item We define $\inf A \subset X$ by $\inf A \defeq \{x \in X: \text{$x$ is a infimum of $A$}\}$
			\end{itemize}
		\end{itemize}
	\end{defn}
	
	\begin{ex} \lex{ex:orderings:prosets:0025}
		Let $(X, \leq)$ be a poset and $A \subset X$. Then
		\begin{enumerate}
			\item for each $x, y \in \sup A$, $x \sim_{\leq} y$.
			\item for each $x,y \in \inf A$, $x \sim_{\leq} y$.
		\end{enumerate}   
	\end{ex}
	
	\begin{proof}\
		\begin{enumerate}
			\item Let $x, y \in \sup A$. Then $x,y \in \ubd A$. Since $x \in \sup A$ and $y \in \ubd A$, $x \leq y$. Since $y \in \sup A$ and $x \in \ubd A$, $y \leq x$. Hence $x \sim_{\leq} y$. 
			\item Similar to $(1)$.
		\end{enumerate}
	\end{proof}
	
	\begin{ex} \lex{ex:orderings:prosets:0026}
		Let $(X, \leq)$ be a poset and $A \subset X$. Then 
		\begin{enumerate}
			\item $\sup A = \ubd A \cap \lbd (\ubd A)$
			\item $\inf A = \lbd A \cap \ubd(\lbd A)$
		\end{enumerate}
	\end{ex}
	
	\begin{proof}\
		\begin{enumerate}
			\item We note that for each $x \in X$,
			\begin{align*}
				x \in \sup A
				& \iff \text{$x \in \ubd A$ and for each $y \in \ubd A$, $x \leq y$} \\
				& \iff \text{$x \in \ubd A$ and $x \in \lbd(\ubd A)$} \\
				& \iff x \in \ubd A \cap \lbd (\ubd A). 
			\end{align*}
			Therefore $\sup A = \ubd A \cap \lbd (\ubd A)$.
			\item Similar to $(1)$.
		\end{enumerate}
	\end{proof}
	
	
	
	
	
	
	
	
	
	
	
	
	
	
	
	
	
	
	
	
	
	
	
	
	
	
	
	
	
	
	
	
	
	
	
	
	
	
	
	
	
	
	
	
	
	
	
	
	\subsection{Order Preserving Maps}
	
	\begin{defn} \ld{def:orderings:prosets:0026}
		Let $(X, \leq_X)$ $(Y, \leq_Y)$ be prosets and $f:X \rightarrow Y$. Then $f$ is said to be
		\begin{itemize}
			\item \tbf{$(\leq_X, \leq_Y)$-order preserving} or \tbf{$(\leq_X, \leq_Y)$-monotone} if for each $a,b \in X$, $a \leq_X b$ implies that $f(a) \leq_Y f(b)$. 
			\item \tbf{$(\leq_X, \leq_Y)$-order reversing} or \tbf{$(\leq_X, \leq_Y)$-antitone} if for each $a,b \in X$, $a \leq_X b$ implies that $f(b) \leq_Y f(a)$. 
		\end{itemize}
	\end{defn}
	
	\begin{note}
		When the context is clear we say that $f$ is ``monotone" and ``antitone" instead of ``$(\leq_X, \leq_Y)$-monotone" and ``$(\leq_X, \leq_Y)$-antitone" respectively.
	\end{note}
	
	\begin{ex} \lex{ex:orderings:prosets:0026.1}
		Let $(X, \leq_X)$ $(Y, \leq_Y)$ be prosets and $f:X \rightarrow Y$. Then 
		\begin{enumerate}
			\item 
			the folloing are equivalent: 
			\begin{enumerate}
				\item $f$ is $(\leq_X, \leq_Y)$-antitone 
				\item $f$ is $(\leq_X, \lop_Y)$-monotone
				\item $f$ is $(\lop_X, \leq_Y)$-monotone
				\item $f$ is $(\lop_X, \lop_Y)$-antitone
			\end{enumerate}
			\item the folloing are equivalent: 
			\begin{enumerate}
				\item $f$ is $(\leq_X, \leq_Y)$-monotone 
				\item $f$ is $(\leq_X, \lop_Y)$-antitone
				\item $f$ is $(\lop_X, \leq_Y)$-antitone
				\item $f$ is $(\lop_X, \lop_Y)$-monotone 
			\end{enumerate}
		\end{enumerate}
	\end{ex}
	
	\begin{proof}\
		\begin{enumerate}
			\item 
			\begin{itemize}
				\item $(a) \implies (b)$: \\
				Suppose that $f$ is $(\leq_X, \leq_Y)$-antitone. Let $a,b \in X$. Then 
				\begin{align*}
					a \leq_X b 
					& \implies f(b) \leq_Y f(a) \\
					& \iff f(a) \lop_Y f(b).
				\end{align*}
				Since $a,b \in X$ are arbitrary, we have that for each $a,b \in X$, $a \leq_X b$ implies that $f(a) \lop_Y f(b)$. Hence $f$ is $(\leq_X, \lop_Y)$-monotone. 
				\item $(b) \implies (c)$: \\
				Suppose that $f$ is $(\leq_X, \lop_Y)$-monotone. Let $a,b \in X$. Then 
				\begin{align*}
					a \lop_X b
					& \iff b \lop_X a \\ 
					& \implies f(b) \lop_Y f(a) \\
					& \iff f(a) \leq_Y f(b).
				\end{align*}
				Since $a,b \in X$ are arbitrary, we have that for each $a,b \in X$, $a \lop_X b$ implies that $f(a) \leq_Y f(b)$. Hence $f$ is $(\lop_X, \leq_Y)$-monotone. 
				\item $(c) \implies (d)$: \\
				Suppose that $f$ is $(\lop_X, \leq_Y)$-monotone. Let $a,b \in X$. Then 
				\begin{align*}
					a \lop_X b
					& \implies f(a) \leq_Y f(b) \\
					& \iff f(b) \lop_Y f(a).
				\end{align*}
				Since $a,b \in X$ are arbitrary, we have that for each $a,b \in X$, $a \lop_X b$ implies that $f(b) \lop_Y f(a)$. Hence $f$ is $(\lop_X, \lop_Y)$-antitone. 
				\item $(d) \implies (a)$: \\
				Suppose that $f$ is $(\lop_X, \lop_Y)$-antitone. Let $a,b \in X$. Then 
				\begin{align*}
					a \leq_X b
					& \iff b \lop_X a \\
					& \implies f(a) \lop_Y f(b) \\
					& \iff f(b) \leq_Y f(a).
				\end{align*}
				Since $a,b \in X$ are arbitrary, we have that for each $a,b \in X$, $a \leq_X b$ implies that $f(b) \leq_Y f(a)$. Hence $f$ is $(\leq_X, \leq_Y)$-antitone. 
			\end{itemize}
			\item Set $\leq_Y' \defeq \lop_Y$. Part $(1)$ implies that 
			\begin{align*}
				\text{$f$ is $(\leq_X, (\leq_Y')^{\text{op}})$-monotone}
				& \iff \text{$f$ is $(\leq_X, \leq_Y')$-antitone} \quad \text{(by $1(b) \iff 1(a)$)} \\
				& \iff \text{$f$ is $(\lop_X, (\leq_Y')^{\text{op}})$-antitone} \quad \text{(by $1(a) \iff 1(d)$)} \\
				& \iff \text{$f$ is $(\lop_X, \leq_Y')$-monotone} \quad \text{(by $1(d) \iff 1(c)$)}.
			\end{align*}
			Since $(\leq_Y')^{\text{op}}) = \leq_Y$, we have that
			\begin{align*}
				\text{$f$ is $(\leq_X, \leq_Y)$-monotone}
				& \iff \text{$f$ is $(\leq_X, \lop_Y)$-antitone} \\
				& \iff \text{$f$ is $(\lop_X, \leq_Y)$-antitone} \\
				& \iff \text{$f$ is $(\lop_X, \lop_Y)$-monotone}.
			\end{align*}
		\end{enumerate}
	\end{proof}
	
	\begin{ex} \lex{ex:orderings:prosets:0027}
		Let $(X, \leq_X)$ $(Y, \leq_Y)$ be prosets and $f:X \rightarrow Y$. Then $f$ is $(\leq_X, \leq_Y)$-monotone iff $f$ is $(\lop_X, \lop_Y)$-monotone.
	\end{ex}
	
	\begin{proof}\
		\begin{itemize}
			\item $(\implies)$: \\
			Suppose that $f$ is $(\leq_X, \leq_Y)$-monotone. Let $a,b \in X$. Suppose that $a \lop_X b$. Then $b \leq_X a$. Hence $f(b) \leq_Y f(a)$. Thus $f(a) \lop_Y f(b)$. Since $a,b \in X$ with $a \lop_X b$ are arbitrary, we have that for each $a,b \in X$, $a \lop_X b$ implies that $f(a) \lop_Y f(b)$. Therefore $f$ is $(\lop_X, \lop_Y)$-monotone.
			\item $(\impliedby)$: \\
			Suppose that $f$ is $(\lop_X, \lop_Y)$-monotone. Since ${(\lop_X)}^{\text{op}} = {\leq_X}$, the previous part implies that $f$ is $(\leq_X, \leq_Y)$-monotone.
		\end{itemize}
	\end{proof}
	
	\begin{ex} \lex{ex:orderings:prosets:0028}
		Let $(X, \leq_X)$ $(Y, \leq_Y)$ be prosets and $f:X \rightarrow Y$. Then for each $U \subset Y$, $U$ is a $\leq_Y$-upper set implies that $f^{-1}(U)$ is a $\leq_X$-upper set iff for each $L \subset Y$, $L$ is a $\leq_Y$-lower set implies that $f^{-1}(L)$ is a $\leq_X$-lower set.
	\end{ex}
	
	\begin{proof}\
		\begin{itemize}
			\item $(\implies)$: \\
			Suppose that for each $U \subset Y$, $U$ is a $\leq_Y$-upper set implies that $f^{-1}(U)$ is a $\leq_X$-upper set. Let $L \subset Y$. Suppose that $L$ is a $\leq_Y$-lower set. \rex{ex:orderings:prosets:0011} then implies that $L^c$ is a $\leq_Y$-upper set. By assumption, $f^{-1}(L^c)$ is a $\leq_X$-upper set. Another application of \rex{ex:orderings:prosets:0011} implies that $f^{-1}(L^c)^c$ is a $\leq_X$-lower set. Since
			\begin{align*}
				f^{-1}(L)
				& = [f^{-1}(L)^c]^c \\
				& = f^{-1}(L^c)^c,
			\end{align*}
			we have that $f^{-1}(L)$ is a $\leq_X$-lower set. Since $L \subset Y$ such that $L$ is a $\leq_Y$-lower set is arbitrary, we have that for each $L \subset Y$, $L$ is a $\leq_Y$-lower set implies that $f^{-1}(L)$ is a $\leq_X$-lower set. 
			\item $(\impliedby)$: \\
			Similar to $(\implies)$.
		\end{itemize}
	\end{proof}
	
	\begin{ex} \lex{ex:orderings:prosets:0029}
		Let $(X, \leq_X)$ $(Y, \leq_Y)$ be prosets and $f:X \rightarrow Y$. If $f$ is $(\leq_X, \leq_Y)$-monotone, then for each $U \subset Y$, $U$ is a $\leq_Y$-upper set implies that $f^{-1}(U)$ is a $\leq_X$-upper set.
	\end{ex}
	
	\begin{proof}
		Suppose that $f$ is $(\leq_X, \leq_Y)$-monotone. Let $U \subset Y$. Suppose that $U$ is a $\leq_Y$-upper set. Let $a \in f^{-1}(U)$ and $x \in X$. Suppose that $a \leq x$. Since $a \in f^{-1}(U)$, $f(a) \in U$. Since $f$ is monotone, $f(a) \leq f(x)$. Since $U$ is a $\leq_Y$-upper set and $f(a) \in U$, we have that $f(x) \in U$. Hence $x \in f^{-1}(U)$. Since $a \in f^{-1}(U)$ and $x \in X$ with $a \leq x$ are arbitrary, we have that for each $a \in f^{-1}(U)$ and $x \in X$, $a \leq x$ implies that $x \in f^{-1}(U)$. Thus $f^{-1}(U)$ is a $\leq_X$-upper set. Since $U \subset Y$ such that $U$ is a $\leq_Y$-upper set is arbitrary, we have that for each $U \subset Y$, $U$ is a $\leq_Y$-upper set implies that $f^{-1}(U)$ is a $\leq_X$-upper set.
	\end{proof}
	
	\begin{ex} \lex{ex:orderings:prosets:0030}
		Let $(X, \leq_X)$ $(Y, \leq_Y)$ be prosets, $f:X \rightarrow Y$ and $A \subset X$. If $f$ is $(\leq_X, \leq_Y)$-monotone, then $f(\ubd (A, \leq_X)) \subset \ubd(f(A), \leq_Y)$.
	\end{ex}
	
	\begin{proof}
		Suppose that $f$ is $(\leq_X, \leq_Y)$-monotone. Let $x \in \ubd (A, \leq_X)$ and $y_0 \in f(A)$. Then there exists $x_0 \in A$ such that $f(x_0) = y$. Since $x \in \ubd (A, \leq_X)$, $x_0 \leq x$. Since $f$ is $(\leq_X, \leq_Y)$-monotone,  
		\begin{align*}
			y_0
			& = f(x_0) \\
			& \leq f(x).
		\end{align*}
		Since $y_0 \in f(A)$ is arbitrary, we have that for each $y_0 \in f(A)$, $y_0 \leq f(x)$. Therefore $f(x) \in \ubd(F(A), \leq_Y)$. Since $x \in \ubd (A, \leq_X)$ is arbitrary, ew have that for each $x \in \ubd (A, \leq_X)$, $f(x) \in \ubd(F(A), \leq_Y)$. Hence $f(\ubd (A, \leq_X)) \subset \ubd(f(A), \leq_Y)$.
	\end{proof}
	
	
	
	
	
	
	
	
	
	
	
	
	
	
	
	
	
	
	
	
	
	
	
	
	
	
	
	
	
	
	
	
	
	
	
	
	
	
	
	
	
	
	
	
	
	
	
	
	
	
	
	
	
	\subsection{Galois Connections}
	
	\begin{defn}
		Let $(X, \leq_X)$ and $(Y, \leq_Y)$ be prosets, $f: X \rightarrow Y$ and $g:Y \rightarrow X$. 
		\begin{itemize}
			\item Then $(f, g)$ is said to be a \tbf{$(\leq_X, \leq_Y)$-monotone Galois connection} if for each $x \in X$ and $y \in Y$, 
			$$\text{$f(x) \leq_Y y$ iff $x \leq_X g(y)$.}$$
			\item Then $(f, g)$ is said to be a \tbf{$(\leq_X, \leq_Y)$-antitone Galois connection} if for each $x \in X$ and $y \in Y$, 
			$$\text{$f(x) \leq_Y y$ iff $g(y) \leq_X x$.}$$
		\end{itemize}
	\end{defn}
	
	\begin{note}
		When the context is clear, we say ``monotone (resp. antitone) Galois connection" instead of ``$(\leq_X, \leq_Y)$-monotone (resp. $(\leq_X, \leq_Y)$-antitone) Galois connection".
	\end{note}
	
	\begin{defn}
		Let $(X, \leq_X)$ and $(Y, \leq_Y)$ be prosets, $f: X \rightarrow Y$ and $g:Y \rightarrow X$. If $(f,g)$ is a monotone (resp. antitone) Galois connection, then 
		\begin{itemize}
			\item $f$ is said to be a \tbf{left adjoint} of $g$, 
			\item $g$ is said to be a \tbf{right adjoint} of $f$.
		\end{itemize}
	\end{defn}
	
	\begin{ex}
		Let $(X, \leq_X)$ and $(Y, \leq_Y)$ be prosets, $f: X \rightarrow Y$ and $g:Y \rightarrow X$. Then 
		\begin{enumerate}
			\item $(f,g)$ is a $(\leq_X, \leq_Y)$-monotone Galois connection iff $(g, f)$ is a $(\lop_Y, \lop_X)$-monotone Galois connection
			\item $(f,g)$ is a $(\leq_X, \leq_Y)$-antitone Galois connection iff $(f, g)$ is a $(\lop_X, \leq_Y)$-monotone Galois connection
		\end{enumerate}
	\end{ex}
	
	\begin{proof}\
		\begin{enumerate}
			\item 
			\begin{itemize}
				\item $(\implies)$: \\
				Suppose that $(f,g)$ is a $(\leq_X, \leq_Y)$-monotone Galois connection. Let $y \in Y$ and $x \in X$. By assumption, 
				\begin{align*}
					g(y) \lop_X x
					& \iff x \leq_X g(y) \\
					& \iff f(x) \leq_Y y \\
					& \iff y \lop_Y f(x).
				\end{align*}
				Since $y \in Y$ and $x \in X$ are arbitrary, we have that for each $y \in Y$ and $x \in X$, $g(y) \lop_X x$ iff $y \lop_Y f(x)$. Therefore $(g,f)$ is a $(\lop_Y, \lop_X)$-monotone Galois connection.
				\item $(\impliedby)$: \\
				
			\end{itemize}
			\item 
			\begin{itemize}
				\item $(\implies)$: \\
				\tcr{FINISH!!!}
				\item $(\impliedby)$: \\
				
			\end{itemize}
		\end{enumerate}
	\end{proof}
	
	\begin{ex}
		Let $(X, \leq_X)$ and $(Y, \leq_Y)$ be prosets, $f: X \rightarrow Y$ and $g:Y \rightarrow X$. Then $(f,g)$ us a $(\leq_X, \leq_Y)$-monotone Galois connection iff
		\begin{enumerate}
			\item 
			\begin{enumerate}
				\item for each $x \in X$, $x \leq_X g \circ f (x)$
				\item for each $y \in Y$, $f \circ g (y) \leq_Y y$
			\end{enumerate}
			\item $f$ is $(\leq_X, \leq_Y)$-monotone and $g$ is $(\leq_Y, \leq_X)$-monotone
		\end{enumerate}
	\end{ex}
	
	\begin{proof}\
		\begin{itemize}
			\item $(\implies)$: \\
			\begin{enumerate}
				\item 
				\begin{enumerate}
					\item Let $x \in X$. Define $y \in Y$ by $y \defeq f(x)$. Since $f(x) \leq_Y f(x)$, we have that $f(x) \leq_Y y$. Therefore 
					\begin{align*}
						x 
						& \leq_X g(y) \\
						& = g \circ f(x).
					\end{align*}
					Since $x \in X$ is arbitrary, we have that for each $x \in X$, $x \leq_X g \circ f(x)$.
					\item Let $y \in Y$. \tcr{the previous ex} implies that $(g,f)$ is a $(\lop_Y, \lop_X)$-monotone Galois connection. Part $1(a)$ then implies that for each $y \in Y$, $y \lop_Y f \circ g(y)$. Therefore for each $y \in Y$, $f \circ g(y) \leq_Y y$
				\end{enumerate}
				\item 
				\begin{itemize}
					\item Let $a, b \in X$. Define $x \in X$ and $y \in Y$ by $x \defeq a$ and $y \defeq f(b)$. Suppose that $a \leq_X b$. Part $1(a)$ implies that 
					\begin{align*}
						x
						& = a \\
						& \leq_X b \\
						& \leq_X g(f(x)) \\
						& = g(y).
					\end{align*}
					Since $(f,g)$ is a $(\leq_X, \leq_Y)$-monotone Galois connection, we have that 
					\begin{align*}
						f(a)
						& = f(x) \\
						& \leq y \\
						& = f(b).
					\end{align*}
					Since $a,b \in X$ with $a \leq_X b$ are arbitrary, we have that for each $a,b \in X$, $a \leq_X b$ implies that $f(a) \leq_Y f(b)$. Thus $f$ is $(\leq_X, \leq_Y)$-monotone.
					\item Since $(f,g)$ is a $(\leq_X, \leq_Y)$-monotone Galois connection, \tcr{a previous ex} implies that $(g, f)$ is a $(\lop_Y, \lop_X)$-monotone Galois connection. The previous part implies that $g$ is $(\lop_Y, \lop_X)$-monotone. \rex{ex:orderings:prosets:0026.1} then implies that $g$ is $(\leq_Y, \leq_X)$-monotone.
				\end{itemize}
			\end{enumerate}
			\item $(\impliedby)$: \\
			\tcr{FINISH!!!}
		\end{itemize}
	\end{proof}
	
	\begin{ex}
		Let $(X, \leq_X)$ and $(Y, \leq_Y)$ be prosets and $(f,g)$ a $(\leq_X, \leq_Y)$-monotone Galois connection. Then $f = f \circ g \circ f$ and $g = g \circ f \circ g$.
	\end{ex}
	
	\begin{proof}
		\tcr{FINSIH!!}
	\end{proof}
	
	\begin{ex}
		Let $(X, \leq)$ be a proset. Define $F,G: \MP(X) \rightarrow \MP(X)$ by $F(A) \defeq \ubd A$ and $G(A) \defeq \lbd A$. Then $(F,G)$ is a $(\subset, \subset)$-monotone Galois connection. 
	\end{ex}
	
	\begin{proof}
		\tcr{FINISH!!!}
	\end{proof}
	
	
	
	
	
	
	
	
	
	
	
	
	
	
	
	
	
	
	
	
	
	
	
	
	
	
	
	
	
	
	
	
	
	
	
	
	
	
	
	
	
	
	
	
	
	
	
	\newpage
	\section{Posets}
	
	\subsection{Introduction}
	
	\begin{defn} \ld{def:orderings:posets:0001} \tbf{Poset:} \\
		Let $X$ be a set and ${\leq}  \subset X \times X$ a binary relation on $X$. Then 
		\begin{itemize}
			\item $\leq$ is said to be a \tbf{partial order on $X$} if 
			\begin{enumerate}
				\item $\leq$ is a preorder on $X$
				\item for each $a,b \in X$, $a \leq b$ and $b \leq a$ implies that $a = b$,
			\end{enumerate}
			\item $(X, \leq)$ is $(X, \leq)$ is said to be a \tbf{partially ordered set} or \tbf{poset} if
			$\leq$ is a partial ordering on $X$.
		\end{itemize}
	\end{defn}
	
	\begin{ex} \lex{ex:orderings:posets:0002}
		Let $(X, \leq)$ be a poset and $a,b \in X$. Then $a \sim_{\leq} b$ iff $a = b$. 
	\end{ex}
	
	\begin{proof}\
		\begin{itemize}
			\item $(\implies)$: \\
			Suppose that $a \sim_{\leq} b$. Then $a \leq b$ and $b \leq a$. Since $\leq$ is a partial order, $a = b$. 
			\item $(\impliedby)$: \\
			Since $\sim_{\leq}$ is reflexive, $a = b$ implies that $a \sim_{\leq} b$. 
		\end{itemize}
	\end{proof}
	
	\begin{defn} \ld{def:orderings:posets:0003}
		Let $(X, \leq_X)$ be a proset. Set $Y \defeq X /\sim_{\leq_X}$. We define $\leq_Y \subset Y \times Y$ by $y_1 \leq_Y y_2$ iff there exists $x_1 \in y_1$ and $x_2 \in y_2$ such that $x_1 \leq_X x_2$.
	\end{defn}
	
	\begin{ex} \lex{ex:orderings:posets:0004}
		Let $(X, \leq_X)$ be a proset. Set $Y \defeq X /\sim_{\leq_X}$. Then $(Y, \leq_Y)$ is a poset. 
	\end{ex}
	
	\begin{proof}\
		\begin{enumerate}
			\item 
			\begin{enumerate}
				\item Let $y \in Y$. Then there exists $x \in y$. Since $x \leq_X x$, we have that $y \leq_Y y$.
				\item Let $y_1, y_2, y_3 \in Y$. Suppose that $y_1 \leq_Y y_2$ and $y_2 \leq_Y y_3$. Then there exist $x_1 \in y_1$, $x_2 \in y_2$ and $x_3 \in y_3$ such that $x_1 \leq_X x_2$ and $x_2 \leq_X x_3$. Therefore $x_1 \leq x_3$. Hence $y_1 \leq_Y y_3$.
			\end{enumerate}
			Thus $\leq_Y$ is a preorder on $Y$.
			\item Let $y_1, y_2 \in Y$. Suppose that $y_1 \leq_Y y_2$ and $y_2 \leq_Y y_1$. Then there exist $a_1, b_1 \in y_1$, $a_2, b_2 \in y_2$ such that $a_1 \leq_X a_2$ and $b_2 \leq_X b_1$. Since $a_1, b_1 \in y_1$ and $a_2, b_2 \in y_2$, $a_1 \sim_{\leq_X} b_1$ and $a_2 \sim_{\leq_X} b_2$. Thus 
			\begin{align*}
				a_1
				& \leq_X a_2 \\
				& \leq_X b_2.
			\end{align*}
			and 
			\begin{align*}
				b_2
				& \leq_X b_1 \\
				& \leq_X a_1.
			\end{align*}
			Hence $a_1 \sim_{\leq_X} b_2$ and 
			\begin{align*}
				y_1
				& = \pi(a_1) \\
				& = \pi(b_2) \\
				& = y_2.
			\end{align*} 
		\end{enumerate}
		Therefore $\leq_Y$ is a partial order on $Y$. 
	\end{proof}
	
	\begin{ex} \lex{ex:orderings:posets:0005}
		Let $(X, \leq)$ be a poset and $A \subset X$. Then
		\begin{enumerate}
			\item for each $x, y \in \sup A$, $x = y$.
			\item for each $x,y \in \inf A$, $x = y$.
		\end{enumerate}   
	\end{ex}
	
	\begin{proof}\
		\begin{enumerate}
			\item Let $x, y \in \sup A$. \rex{ex:orderings:prosets:0025} implies that $x \sim_{\leq} y$. \rex{ex:orderings:posets:0002} then implies that $x = y$.
			\item Similar to $(1)$.
		\end{enumerate}
	\end{proof}
	
	\begin{defn} \ld{def:orderings:posets:0006}
		Let $(X, \leq)$ be a poset and $A \subset X$. 
		\begin{itemize}
			\item We say that \tbf{$\sup A$ (resp. $\inf A$) exists} if $\sup A \neq \varnothing$ (resp. $\inf A \neq \varnothing$).
			\item Let $x \in X$. We write $x = \sup A$ (resp. $x = \inf A$) if $x \in \sup A$ (resp. $x \in \inf A$)
		\end{itemize}
	\end{defn}
	
	\begin{ex} \lex{ex:orderings:posets:0006.1}
		Let $(X, \leq)$ be a poset and $A \subset X$. Then 
		\begin{enumerate}
			\item $\sup A = \inf (\ubd A)$ \tcr{in proset subsection, make exercise saying $\sup A = \ubd A \cap \lbd (\ubd A)$, then rework this exercise with this fact to shorten it up, show $A \mapsto \ubd A$ and $A \mapsto \lbd A$ make a galois connection, then $\ubd (\lbd (\ubd A)) = \ubd A$ and this implies that $\sup A = \inf (\ub A)$ and dually, in prosets (i.e. make exercises in proset sections)}
			\item $\inf A = \sup (\lbd A)$
		\end{enumerate}
	\end{ex}
	
	\begin{proof}\
		\begin{enumerate}
			\item \rex{ex:orderings:prosets:0026} implies that $\sup A \in \ubd A \cap \lbd (\ubd A)$.
			\begin{itemize}
				\item By definition, for each $x \in \ubd A$, $\sup A \leq x$. Therefore $\sup A \in \lbd(\ubd A)$. Hence $\sup A \leq \inf (\ubd A)$.
				\item By definition, $\sup A \in \ubd A$ and $\inf (\ubd A) \in \lbd (\ubd A)$. Therefore $\inf (\ubd A) \leq \sup A$.
			\end{itemize}
			Since $\sup A \leq \inf (\ubd A)$ and $\inf (\ubd A) \leq \sup A$, we have that $\sup A = \inf (\ubd A)$.
			\item Similar to $(1)$.
		\end{enumerate}
	\end{proof}
	
	\begin{ex} \lex{ex:orderings:posets:0007} \tbf{Associativity of Supremum:} \\
		Let $(X, \leq)$ be a poset and $(E_{\al})_{\al \in A} \subset \MP(X)$. 
		\begin{enumerate}
			\item Suppose that 
			\begin{itemize}
				\item for each $\al \in A$, $\sup E_{\al}$ exists
				\item $\sup\limits_{\al \in A} \bigg[ \sup E_{\al} \bigg]$ exists.
			\end{itemize}
			Then 
			$$\sup \bigcup\limits_{\al \in A} E_{\al} = \sup\limits_{\al \in A} \bigg[ \sup E_{\al} \bigg]$$
			\item Suppose that 
			\begin{itemize}
				\item for each $\al \in A$, $\inf E_{\al}$ exists
				\item $\inf\limits_{\al \in A} \bigg[ \inf E_{\al} \bigg]$ exists.
			\end{itemize}
			Then 
			$$\inf \bigcup\limits_{\al \in A} E_{\al} = \inf\limits_{\al \in A} \bigg[ \inf E_{\al} \bigg]$$
		\end{enumerate}
	\end{ex}
	
	\begin{proof}\
		\begin{enumerate}
			\item Define $s_1, s_2 \in X$ by $s_1 \defeq \bigcup\limits_{\al \in A} E_{\al}$ and $s_2 \defeq \sup\limits_{\al \in A} \bigg[ \sup E_{\al} \bigg]$. We note by definition, $s_2$ is an upper bound for $\{\sup E_{\al}: \al \in A\}$. Then for each $\al \in A$, $s_2 \geq \sup E_{\al}$. 
			\begin{itemize}
				\item Let $x \in \bigcup\limits_{\al \in A} E_{\al}$. Then there exists $\al_0 \in A$ such that $x \in E_{\al_0}$. Since $\sup E_{\al_0}$ is an upper bound of $E_{\al_0}$, we have that 
				\begin{align*}
					s_2
					& \geq \sup E_{\al_0} \\
					& \geq x.
				\end{align*}
				Since $x \in \bigcup\limits_{\al \in A} E_{\al}$ is arbitrary, we have that $s_2$ is an upper bound for $\bigcup\limits_{\al \in A} E_{\al}$. Therefore
				\begin{align*}
					s_1
					& = \sup \bigcup\limits_{\al \in A} E_{\al} \\
					& \leq s_2.
				\end{align*}   
				\item Let $\al_0 \in A$ and $x \in E_{\al_0}$. Then 
				\begin{align*}
					x
					& \in E_{\al_0} \\
					& \subset \bigcup\limits_{\al \in A} E_{\al}.
				\end{align*}
				Since $s_1$ is an upper bound of $\bigcup\limits_{\al \in A} E_{\al}$, we have that $s_1 \geq x$. Since $x \in E_{\al_0}$ is arbitrary, we have that for each $x \in E_{\al_0}$, $s_1 \geq x$. Hence $s_1$ is an upper bound of $E_{\al_0}$. Therefore $\sup E_{\al_0} \leq s_1$. Since $\al_0 \in A$ is arbitrary, we have that for each $\al \in A$, $s_1 \geq \sup E_{\al}$. Therfore $s_1$ is an upper bound of $\{\sup E_{\al}: \al \in A\}$. Hence 
				\begin{align*}
					s_2
					& = \sup\limits_{\al \in A} \bigg[ \sup E_{\al} \bigg] \\
					& \leq s_1. 
				\end{align*}
			\end{itemize}
			Since $s_1 \leq s_2$ and $s_2 \leq s_1$, we have that $s_1 = s_2$. 
			\item Similar to $(1)$. \tcr{Maybe fill out}
		\end{enumerate}
	\end{proof}
	
	\begin{ex} \lex{ex:orderings:posets:0008}
		Let $(X, \leq)$ be a poset and $a,b \in X$. Then $a \leq b$ iff $b = \sup \{a,b\}$. 
	\end{ex}
	
	\begin{proof}\
		\begin{itemize}
			\item $(\implies)$: \\
			Suppose that $a \leq b$. Since $b \leq b$, we have that for each $c \in \{a,b\}$, $c \leq b$. Hence $b \in \lbd \{a,b\}$. Let $c \in \ubd \{a,b\}$. Then $b \leq c$. Since $c \in \ubd \{a,b\}$ is arbitrary, we have that for each $c \in \ubd \{a, b\}$, $b \leq c$. Hence $b = \sup \{a,b\}$.
			\item $(\impliedby)$: \\
			Suppose that $b = \sup \{a,b\}$. Then $b \in \ubd \{a,b\}$. Therefore $a \leq b$. 
		\end{itemize}
	\end{proof}
	
	\begin{ex} \lex{ex:orderings:posets:0009}
		Let $(X, \leq)$ be a poset. Then 
		\begin{enumerate}
			\item for each $a,b \in \ubd X$, $a = b$, 
			\item for each $a,b \in \lbd X$, $a = b$.
		\end{enumerate}
	\end{ex}
	
	\begin{proof}\
		\begin{enumerate}
			\item Let $a,b \in \ubd X$. Since $a \in X$ and $b \in \ubd X$, $a \leq b$. Similarly, $b \leq a$. Hence $a = b$.
			\item Similar to $(1)$.
		\end{enumerate}
	\end{proof}
	
	
	
	
	
	
	
	
	
	
	
	
	
	
	
	
	
	
	
	
	
	
	
	
	
	
	
	
	
	
	
	
	
	
	
	
	
	
	
	
	
	
	
	
	
	
	
	
	
	
	
	
	
	
	
	
	
	
	\newpage
	\section{Directed Sets}
	
	\begin{defn} \ld{def:orderings:directed_sets:0001} \ld{def:orderings:directed_sets:0001} \tbf{Directed Set:} \\
		Let $A$ be a set and $\leq \, \subset A \times A$ a binary relation on $A$. Then 
		\begin{itemize}
			\item $\leq$ is said to be a \tbf{direction on $A$} if
			\begin{enumerate}
				\item $\leq$ is a preorder on $A$
				\item for each $\al, \be \in A$, $\{\al, \be\}$ is bounded above. 
			\end{enumerate}
			\item $(A, \leq)$ is said to be a \tbf{directed set} if
			\begin{enumerate}
				\item $A \neq \varnothing$
				\item $\leq$ is a direction on $A$
			\end{enumerate}
		\end{itemize}
	\end{defn}
	
	\begin{ex} \lex{ex:orderings:directed_sets:0003}
		Let $(A, \leq_A)$ and $(B, \leq_B)$ be directed sets. Then 
		\begin{enumerate}
			\item $\leq_A \otimes \leq_B$ is a direction on $A \times B$,
			\item $(A, \leq_A) \otimes (B, \leq_B)$ is a directed set.
		\end{enumerate}
	\end{ex}
	
	\begin{proof}\
		\begin{enumerate}
			\item 
			\begin{enumerate}
				\item \rex{ex:orderings:prosets:0002} implies that $\leq_A \otimes \leq_B$ is a preorder of $A \times B$.
				\item Let $(a_1, b_1), (a_2, b_2) \in A \times B$. Then there exist $a \in A$ and $b \in B$ such that $a_1, a_2 \leq_A a$ and $b_1, b_2 \leq_B b$. Hence $(a_1, b_1), (a_2, b_2) \leq_A \otimes \leq_B (a, b)$.
			\end{enumerate}
			Hence $\leq_A \otimes \leq_B$ is a direction on $A \times B$.
			\item 
			\begin{enumerate}
				\item Since $A \neq \varnothing$ and $B \neq \varnothing$, we have that $A \times B \neq \varnothing$.
				\item From above, $\leq_A \otimes \leq_B$ is a direction on $A \times B$. 
			\end{enumerate}
			Hence $(A \times B, \leq_A \otimes \leq_B)$ is a directed set.
		\end{enumerate}
	\end{proof}
	
	
	
	
	
	
	
	
	
	
	
	
	
	
	
	
	
	
	
	
	
	
	
	
	
	
	
	
	
	
	
	
	
	
	
	
	
	
	
	
	
	
	
	
	
	
	
	
	
	
	
	
	
	
	
	
	
	
	
	
	
	
	
	
	
	
	
	
	
	
	
	
	
	
	
	
	
	
	
	
	
	
	
	
	
	
	
	
	
	
	
	
	
	
	
	
	
	
	
	
	
	
	
	
	
	
	
	
	
	
	
	
	
	
	
	
	
	
	
	
	
	
	
	
	
	
	
	
	
	
	\newpage
	
	\part{Algebraic Structures}
	
	\chapter{Lattices}
	
	\section{Introduction}
	
	\subsection{Skew Semilattices}

	\tcr{cite Leech, skew lattices in rings}
	
	\begin{defn} \ld{def:lattices:skew_lattices:0001}
		Let $X$ be a set and $\dmd:X \times X \rightarrow X$ a binary operator on $X$. Then 
		\begin{itemize}
			\item $\dmd$ is said to be a \tbf{skew semilattice operator on $X$} if $\dmd$ is associative and idempotent.
			\item $(X, \dmd)$ is said to be a \tbf{skew semilattice} if $\dmd$ is a skew semilattice operator on $X$.
		\end{itemize}
	\end{defn}
	
	\begin{defn} \ld{def:lattices:skew_lattices:0002}
		Let $(X, \dmd)$ be an skew semilattice. We define the 
		\begin{itemize}
			\item \tbf{join preorder on $X$ induced by $\dmd$}, denoted ${\leq^{\vee}_{\dmd}} \subset X \times X$, by $a \leq^{\vee}_{\dmd} b$ iff $(b \dmd a) \dmd b = b$
			\item \tbf{meet preorder on $X$ induced by $\dmd$}, denoted ${\leq^{\wedge}_{\dmd}} \subset X \times X$, by $a \leq^{\wedge}_{\dmd} b$ iff $(a \dmd b) \dmd a = a$. 
		\end{itemize}
	\end{defn}
	
	\begin{ex} \lex{ex:lattices:skew_lattices:0003}
		Let $(X, \dmd)$ be an skew semilattice. Then 
		\begin{enumerate}
			\item $\leq^{\vee}_{\dmd}$ is a preorder on $X$. \\
			\tbf{Hint:} If $cbc = c$ and $bab = b$, then 
			\begin{enumerate}
				\item $c = (cb)(abc)(abc)$ and $c = cabc$
				\item $cac = (ca)(ca)(bc)$
			\end{enumerate}
			\item $\leq^{\wedge}_{\dmd}$ is a preorder on $X$.
		\end{enumerate}
	\end{ex}
	
	\begin{proof}\
		\begin{enumerate}
			\item Let $a,b,c \in X$.
			\begin{enumerate}
				\item Since $\dmd$ is idempotent, we have that 
				\begin{align*}
					(a \dmd a) \dmd a
					& = a \dmd a \\
					& = a
				\end{align*}
				and therefore $a \leq^{\vee}_{\dmd} a$. 
				\item Suppose that $a \leq^{\vee}_{\dmd} b$ and $b \leq^{\vee}_{\dmd} c$. Then $(b \dmd a) \dmd b = b$ and $(c \dmd b) \dmd c = c$. Since $\dmd$ is associative and idempotent,
				\begin{align*}
					c 
					& = cbc \\
					& = cbabc \\
					& = (cb)(abc) \\
					& = (cb)(abc)(abc) \\
					& = c(bab)cabc \\
					& = cbcabc \\
					& = (cbc)abc \\
					& = cabc.
				\end{align*}
				Therefore 
				\begin{align*}
					cac
					& = (ca)(c) \\
					& = (ca)(cabc) \\
					& = (ca)(ca)(bc) \\
					& = (ca)(bc) \\
					& = cabc \\
					& = c. 
				\end{align*}
				Hence $a \leq^{\vee}_{\dmd} c$.
			\end{enumerate}
			So $\leq^{\vee}_{\dmd}$ is a preorder on $X$.
			\item Similar to $(1)$.
		\end{enumerate}
	\end{proof}
	
	\begin{ex} \lex{ex:lattices:skew_lattices:0004}
		Let $(X, \dmd)$ be an skew semilattice. Then 
		\begin{enumerate}
			\item $(\leq^{\vee}_{\dmd})^{\text{op}} = \leq^{\wedge}_{\dmd}$
			\item $(\leq^{\wedge}_{\dmd})^{\text{op}} = \leq^{\vee}_{\dmd}$
		\end{enumerate}
	\end{ex}
	
	\begin{proof}\
		\begin{enumerate}
			\item Let $a,b \in X$. Then 
			\begin{align*}
				a (\leq^{\vee}_{\dmd})^{\text{op}} b 
				& \iff b \leq^{\vee}_{\dmd} a \\
				& \iff a \dmd b \dmd a = a \\
				& \iff a \leq^{\wedge}_{\dmd} b.
			\end{align*}
			\item Similar to $(1)$.
		\end{enumerate}
	\end{proof}
	
	\begin{ex} \lex{ex:lattices:skew_lattices:0005}
		Let $(X, \dmd)$ be an skew semilattice. Then for each $a,b \in X$, 
		\begin{enumerate}
			\item $a \dmd b \in \ubd( \{a,b\}, \leq_{\dmd}^{\vee})$
			\item $a \dmd b \in \lbd( \{a,b\}, \leq_{\dmd}^{\wedge})$
		\end{enumerate}
	\end{ex}
	
	\begin{proof} 
		Let $a,b \in X$.
		\begin{enumerate}
			\item 
			We note that 
			\begin{align*}
				(a \dmd b) \dmd [a \dmd (a \dmd b) ]
				& = (a \dmd b) \dmd [(a \dmd a) \dmd b] \\
				& = (a \dmd b) \dmd (a \dmd b) \\
				& = a \dmd b. 
			\end{align*}
			Similarly, $[(a \dmd b) \dmd b] \dmd (a \dmd b) = a \dmd b$. Therefore $a \leq_{\dmd}^{\vee} a \dmd b$ and $b \leq_{\dmd}^{\vee} a \dmd b$. Hence $a \dmd b \in \ubd( \{a,b\}, \leq_{\dmd}^{\vee})$.
			\item \tcr{FINISH!!!}
		\end{enumerate}
	\end{proof}
	
	\begin{defn} \ld{def:lattices:skew_lattices:0006}
		Let $X$ be a set and $\vee, \wedge$ skew semilattice operators on $X$. Then $\vee$ and $\wedge$ are said to  \tbf{satisfy the skew lattice absorption identities} if for each $a,b \in X$,
		\begin{enumerate}
			\item $a \wedge (a \vee b) = a$
			\item $a \vee (a \wedge b) = a$
			\item $(b \vee a) \wedge a = a$
			\item $(b \wedge a) \vee a = a$
		\end{enumerate} 
	\end{defn}
	
	\begin{ex} \lex{ex:lattices:skew_lattices:0007}
		Let $X$ be a set and $\vee, \wedge$ skew semilattice operators on $X$. Then $\vee$ and $\wedge$ satisfy the skew lattice absorption identities iff for each $a,b \in X$,
		\begin{enumerate}
			\item $a \vee b = b$ iff $a \wedge b = a$
			\item $a \vee b = a$ iff $a \wedge b = b$
		\end{enumerate}
	\end{ex}
	
	\begin{proof}\
		\begin{itemize}
			\item $(\implies)$: \\
			Suppose that $\vee$ and $\wedge$ satisfy the skew lattice absorption identities. Let $a, b \in X$. 
			\begin{enumerate}
				\item 
				\begin{itemize}
					\item $(\implies)$: \\
					If $a \vee b = b$, then 
					\begin{align*}
						a \wedge b
						& = a \wedge (a \vee b) \\
						& = a.
					\end{align*}
					\item $(\impliedby)$: \\
					If $a \wedge b = a$, then 
					\begin{align*}
						a \vee b
						& = (a \wedge b) \vee b \\
						& = b.
					\end{align*}
				\end{itemize}
				Thus $a \vee b = b$ iff $a \wedge b = a$.
				\item Similar to $(1)$
			\end{enumerate}
			\item $(\impliedby)$: \\
			Suppose that for each $a,b \in X$,
			\begin{enumerate}
				\item $a \vee b = b$ iff $a \wedge b = a$,
				\item $a \vee b = a$ iff $a \wedge b = b$.
			\end{enumerate}
			Let $a, b \in X$.
			\begin{enumerate}
				\item Set $c \defeq a \vee b$. By assumption, 
				\begin{align*}
					a \wedge (a \vee b) = a
					& \iff a \wedge c = a \\
					& \iff a \vee c = c \\
					& \iff a \vee (a \vee b) = c \\
					& \iff (a \vee a) \vee b = c \\
					& \iff a \vee b = c \\
					& \iff c = c.
				\end{align*}
				Since $c =c$, we have that $a \wedge (a \vee b) = a$.
				\item Similar to $(1)$
				\item Similar to $(1)$
				\item Similar to $(1)$
			\end{enumerate}
		\end{itemize}
	\end{proof}
	
	\begin{ex} \lex{ex:lattices:skew_lattices:0008}
		Let $X$ be a set and $\vee, \wedge$ skew semilattice operators on $X$. If $\vee$ and $\wedge$ satisfy the skew lattice absorption identities, then for each $a,b \in X$, $b \vee a \vee b = b$ iff $a \wedge b \wedge a = a$. \\
		\tbf{Hint:} \rex{ex:lattices:skew_lattices:0007} implies that $b \vee (a \vee b) = b$ iff $b \wedge (a \vee b) = a \vee b$. Use, skew lattice absorption identities and associativity.
	\end{ex}
	
	\begin{proof}
		Suppose that $\vee$ and $\wedge$ satisfy the skew lattice absorption identities. Let $a, b \in X$.  
		\begin{itemize}
			\item $(\implies)$: \\
			Suppose that $b \vee a \vee b = b$. Then \rex{ex:lattices:skew_lattices:0007} implies that $b \wedge (a \vee b) = (a \vee b)$.
			Therefore
			\begin{align*}
				a \wedge b
				& = (a \wedge b) \vee [(a \wedge b) \wedge (a \vee b)] \quad \text{(absorption)} \\
				& = (a \wedge b) \vee [a \wedge (b \wedge [a \vee b])] \quad \text{(associativity)} \\
				& = (a \wedge b) \vee [a \wedge (a \vee b)] \quad \text{(from earlier)} \\
				& = (a \wedge b) \vee a \quad \text{(absorption)}.
			\end{align*}
			Another application of \rex{ex:lattices:skew_lattices:0007} implies that $(a \wedge b) \wedge a = a$.
			\item $(\impliedby)$: \\
			Similar to $(\implies)$.
		\end{itemize}
	\end{proof}
	
	
	
	
	
	
	
	
	
	
	
	
	
	
	
	
	
	
	
	
	
	
	
	
	
	
	
	
	
	
	
	
	
	
	
	
	
	
	
	
	
	\subsection{Skew lattices}
	
	\begin{defn} \ld{def:lattices:skew_lattices:0009}
		Let $X$ be a set and $\vee, \wedge$ skew semilattice operators on $X$. Then $(X, \vee, \wedge)$ is said to be a \tbf{skew lattice} if $\vee$ and $\wedge$ satisfy the skew lattice absorption identities.
	\end{defn}
	
	\begin{ex} \lex{ex:lattices:skew_lattices:0010}
		Let $(X, \vee, \wedge)$ be an skew lattice. Then 
		\begin{enumerate}
			\item $\leq_{\vee}^{\vee} = \leq_{\wedge}^{\wedge}$.
			\item $\leq_{\vee}^{\wedge} = \leq_{\wedge}^{\vee} = (\leq_{\vee}^{\vee})^{\text{op}}$.
		\end{enumerate}
	\end{ex}
	
	\begin{proof}\
		\begin{enumerate}
			\item Let $a,b \in X$. Since $(X, \vee, \wedge)$ is an skew lattice, $\vee$ and $\wedge$ satisfy the skew lattice absorption identities. \rex{ex:lattices:skew_lattices:0008} then implies that 
			\begin{align*}
				a \leq_{\vee}^{\vee} b 
				& \iff b \vee a \vee b = b \\
				& \iff a \wedge b \wedge a = a \\
				& \iff a \leq_{\wedge}^{\wedge} b.
			\end{align*}
			Since $a, b \in X$ are arbitrary, we have that for each $a,b \in X$, $a \leq_{\vee}^{\vee} b$ iff $a \leq_{\wedge}^{\wedge} b$. Hence $\leq_{\vee}^{\vee} = \leq_{\wedge}^{\wedge}$. 
			\item \rex{ex:lattices:skew_lattices:0004} and part $(1)$ imply that 
			\begin{align*}
				\leq_{\vee}^{\wedge} 
				& = (\leq_{\vee}^{\vee})^{\text{op}} \\
				& = (\leq_{\wedge}^{\wedge})^{\text{op}} \\
				& = \leq_{\wedge}^{\vee}.
			\end{align*}
		\end{enumerate}
	\end{proof}
	
	\begin{note}
		Let $(X, \vee, \wedge)$ be an skew lattice. When the context is clear, we write $\leq$ in place of $\leq_{\vee}^{\vee}$. 
	\end{note}
	
	
	
	
	
	
	
	
	
	
	
	
	
	
	
	
	
	
	
	
	
	
	
	
	
	
	
	
	
	
	
	
	
	
	
	
	
	
	
	
	
	
	
	
	
	
	
	\newpage
	
	\subsection{Semilattices}
	
	\begin{defn} \ld{def:lattices:lattices:0001}
		Let $X$ be a set and $\dmd:X \times X \rightarrow X$. 
		Then 
		\begin{itemize}
			\item $\dmd$ is said to be a \tbf{semilattice operator on $X$} if $\dmd$ associative, idempotent and commutative.
			\item $(X, \dmd)$ is said to be an \tbf{semilattice} if $\dmd$ is a semilattice operator on $X$.
		\end{itemize}
	\end{defn}
	
	\begin{ex} \lex{ex:lattices:lattices:0002}
		Let $X$ be a set and $\dmd:X \times X \rightarrow X$. 
		Then $\dmd$ is a semilattice operator on $X$ iff 
		\begin{enumerate}
			\item $\dmd$ is a skew semilattice operator on $X$, \item $\dmd$ is commutative.
		\end{enumerate}
	\end{ex}
	
	\begin{proof}
		Clear by definition. 
	\end{proof}
	
	\begin{ex} \lex{ex:lattices:lattices:0003}
		Let $(X, \dmd)$ be a semilattice. Then 
		\begin{enumerate}
			\item for each $a, b \in X$,
			\begin{enumerate}
				\item $a \leq^{\vee}_{\dmd} b$ iff $a \dmd b = b$
				\item $a \leq^{\wedge}_{\dmd} b$ iff $a \dmd b = a$
			\end{enumerate}
			\item 
			\begin{enumerate}
				\item $\leq^{\vee}_{\dmd}$ is a partial order on $X$
				\item $\leq^{\wedge}_{\dmd}$ is a partial order on $X$
			\end{enumerate}
		\end{enumerate}
	\end{ex}
	
	\begin{proof}\
		\begin{enumerate}
			\item Let $a, b \in X$.
			\begin{enumerate}
				\item Since $\dmd$ is commutative, associative and idempotent,
				\begin{align*}
					a \leq^{\vee}_{\dmd} b
					& \iff (b \dmd a) \dmd b = b \\
					& \iff (a \dmd b) \dmd b = b \\
					& \iff a \dmd (b \dmd b) = b \\
					& \iff a \dmd b = b.
				\end{align*}
				\item Similar to $1(a)$.
			\end{enumerate}
			\item 
			\begin{enumerate}
				\item \rex{ex:lattices:skew_lattices:0003} implies that $\leq^{\vee}_{\dmd}$ is a preorder on $X$. Let $a,b \in X$. Suppose that $a \leq^{\vee}_{\dmd} b$ and $b \leq^{\vee}_{\dmd} a$. Part $(1)$ then implies that $a \dmd b = b$ and $b \dmd a = a$. Since $\dmd$ is commutative,
				\begin{align*}
					a
					& = b \dmd a \\
					& = a \dmd b \\
					& = b.
				\end{align*}
				Since $a, b \in X$ are arbitrary, we have that for each $a,b \in X$, if $a \leq^{\vee}_{\dmd} b$ and $b \leq^{\vee}_{\dmd} a$, then $a = b$. Thus $\leq^{\vee}_{\dmd}$ is a a partial order on $X$.
				\item Similar to $2(a)$.
			\end{enumerate}
		\end{enumerate}
	\end{proof}
	
	\begin{ex} \lex{ex:lattices:lattices:0004}
		Let $(X, \dmd)$ be a semilattice. Then for each $a,b \in X$, 
		\begin{enumerate}
			\item $a \dmd b = \sup (\{a, b\}, \leq_{\dmd}^{\vee})$
			\item $a \dmd b = \inf (\{a, b\}, \leq_{\dmd}^{\wedge})$
		\end{enumerate}
	\end{ex}
	
	\begin{proof} 
		Let $a,b \in X$.
		\begin{enumerate}
			\item 
			\rex{ex:lattices:skew_lattices:0005} implies that $a \dmd b \in \ubd( \{a,b\}, \leq_{\dmd}^{\vee})$. Let $c \in \ubd( \{a,b\}, \leq_{\dmd}^{\vee})$. Since $a \leq_{\dmd}^{\vee} c$ and $b \leq_{\dmd}^{\vee} c$, \rex{ex:lattices:lattices:0003} implies that $a \dmd c = c$ and $b \dmd c = c$. Therefore 
			\begin{align*}
				(a \dmd b) \dmd c 
				& = a \dmd (b \dmd c) \\
				& = a \dmd c \\
				& = c.
			\end{align*}
			Another application of \rex{ex:lattices:lattices:0003} implies that $a \dmd b \leq_{\dmd}^{\vee} c$. Since $c \in \ubd( \{a,b\}, \leq_{\dmd}^{\vee})$ is arbitrary, we have that for each $c \in \ubd( \{a,b\}, \leq_{\dmd}^{\vee})$, $a \dmd b \leq_{\dmd}^{\vee} c$. Therefore $a \dmd b = \sup( \{a,b\}, \leq_{\dmd}^{\vee})$.
			\item \tcr{FINISH!!!}
		\end{enumerate}
	\end{proof}
	
	\begin{defn} \ld{def:lattices:lattices:0005}
		Let $(X, \leq)$ be a poset. Then $(X, \leq)$ is said to be a
		\begin{itemize}
			\item \tbf{join-semilattice} if for each $a,b \in X$, $\sup \{a,b\}$ exists. 
			\item \tbf{meet-semilattice} if for each $a,b \in X$, $\inf \{a,b\}$ exists. 
		\end{itemize}
	\end{defn}
	
	\begin{ex} \lex{ex:lattices:lattices:0006}
		Let $(X, \dmd)$ be a semilattice. Then 
		\begin{enumerate}
			\item $(X, \leq_{\dmd}^{\vee})$ is a join-semilattice 
			\item $(X, \leq_{\dmd}^{\wedge})$ is a meet-semilattice
		\end{enumerate}
	\end{ex}
	
	\begin{proof}
		\rex{ex:lattices:lattices:0004} implies that
		\begin{enumerate}
			\item For each $a,b \in X$, $\sup (\{a,b\}, \leq_{\dmd}^{\vee}) = a \dmd b$. Thus $(X, \leq_{\dmd}^{\vee})$ is a join-semilattice.
			\item for each $a,b \in X$, $\inf (\{a,b\}, \leq_{\dmd}^{\wedge}) = a \dmd b$. Thus $(X, \leq_{\dmd}^{\wedge})$ is a meet-semilattice.
		\end{enumerate}
	\end{proof}
	
	\begin{defn} \ld{def:lattices:lattices:0007}
		Let $(X, \leq)$ be a poset.
		\begin{itemize}
			\item If $(X, \leq)$ is a join-semilattice, we define the \tbf{join operator on $X$ induced by $\leq$}, denoted $\vee_{\leq}: X \times X \rightarrow X$, by $a \vee_{\leq} b \defeq \sup \{a,b\}$. 
			\item If $(X, \leq)$ is a meet-semilattice, we define the \tbf{meet operator on $X$ induced by $\leq$}, denoted $\wedge_{\leq}: X \times X \rightarrow X$, by $a \wedge_{\leq} b \defeq \inf \{a,b\}$. 
		\end{itemize}
	\end{defn}
	
	\begin{ex} \lex{ex:lattices:lattices:0008}
		Let $(X, \leq)$ be a poset.
		\begin{enumerate}
			\item If $(X, \leq)$ an join-semilattice, then $\vee_{\leq}$ is a semilattice operator on $X$.
			\item If $(X, \leq)$ an meet-semilattice, then $\wedge_{\leq}$ is a semilattice operator on $X$.
		\end{enumerate}
	\end{ex}
	
	\begin{proof}\
		\begin{enumerate}
			\item Suppose that $(X, \leq)$ an join-semilattice. 
			\begin{enumerate}
				\item \begin{itemize}
					\item Let $a,b,c \in X$. \rex{ex:orderings:posets:0007} implies that
					\begin{align*}
						(a\vee_{\leq}b) \vee_{\leq} c
						& = \sup \{\sup \{a, b\}, c\} \\
						& = \sup \{\sup \{a, b\}, \sup \{c\} \} \\
						& = \sup \{a, b, c\} \\
						& = \sup \{\sup \{a \}, \sup \{b, c\} \} \\
						& = \sup \{a, \sup \{b, c\} \} \\
						& = a \vee_{\leq} (b \vee_{\leq} c).
					\end{align*}
					Hence $\vee_{\leq}$ is associative.
					\item Let $a \in X$. Then 
					\begin{align*}
						a \vee_{\leq} a
						& = \sup \{a, a\} \\
						& = \sup \{a\} \\
						& = a.
					\end{align*}
					Hence $\vee_{\leq}$ is idempotent.
					\item Let $a,b \in X$. Then 
					\begin{align*}
						a \vee_{\leq} b
						& = \sup \{a, b\} \\
						& = \sup \{b, a\} \\
						& = b \vee_{\leq} a.
					\end{align*}
					Hence $\vee_{\leq}$ is commutative.
				\end{itemize}
				Since $\vee_{\leq}$ is associative, idempotent and commutative, $\vee_{\leq}$ is a semilattice operator on $X$.
			\end{enumerate}
			\item Similar to $(1)$.
		\end{enumerate}
	\end{proof}
	
	\begin{ex} \lex{ex:lattices:lattices:0009}
		Let $(X, \leq)$ be a poset. 
		\begin{enumerate}
			\item If $(X, \leq)$ is a join-semilattice, then
			\begin{enumerate}
				\item $\leq_{\vee_{\leq}}^{\vee} = \leq$,
				\item $\leq_{\vee_{\leq}}^{\wedge} = \lop$.
			\end{enumerate} 
			\item If $(X, \leq)$ is a meet-semilattice, then
			\begin{enumerate}
				\item $\leq_{\wedge_{\leq}}^{\wedge} = \leq$,
				\item $\leq_{\wedge_{\leq}}^{\vee} = \lop$.
			\end{enumerate}  
		\end{enumerate}
	\end{ex}
	
	\begin{proof}\
		\begin{enumerate}
			\item Suppose that $(X, \leq)$ is a join-semilattice. 
			\begin{enumerate}
				\item Let $a,b \in X$. Then
				\begin{align*}
					a \leq_{\vee_{\leq}}^{\vee} b 
					& \iff a \vee_{\leq} b = b \quad \text{(\rex{ex:lattices:lattices:0003})} \\
					& \iff \sup (\{a, b\}, \leq) = b \quad \text{(\rd{def:lattices:lattices:0007})} \\
					& \iff a \leq b \quad \text{(\rex{ex:orderings:posets:0008})}.
				\end{align*}
				Since $a,b \in X$ are arbitrary, we have that $\leq_{\vee_{\leq}}^{\vee} = \leq$.
				\item \rex{ex:lattices:skew_lattices:0004} implies that 
				\begin{align*}
					\leq_{\vee_{\leq}}^{\wedge}
					& = (\leq_{\vee_{\leq}}^{\vee})^{\text{op}} \\
					& = \lop.
				\end{align*}
			\end{enumerate}
			\item Similar to $(1)$.
		\end{enumerate}
	\end{proof}
	
	\begin{ex} \lex{ex:lattices:lattices:0010}
		Let $(X, \dmd)$ be a semilattice. Then 
		\begin{enumerate}
			\item $\leq_{\dmd}^{\vee}$ is the unique partial order ${\leq}$ on $X$ such that $(X, \leq)$ is a join-semilattice and for each $a,b \in X$, $a \dmd b = \sup (\{a,b\}, \leq)$.
			\item $\leq_{\dmd}^{\wedge}$ is the unique partial order ${\leq}$ on $X$ such that $(X, \leq)$ is an meet-semilattice and for each $a,b \in X$, $a \dmd b = \inf (\{a,b\}, \leq)$.
		\end{enumerate}
	\end{ex}
	
	\begin{proof}\
		\begin{enumerate}
			\item Let $\leq$ be partial order on $X$. Suppose that $(X, \leq)$ is a join-semilattice and for each $a,b \in X$, $a \dmd b = \sup (\{a,b\}, \leq)$. Let $a,b \in X$. Then 
			\begin{align*}
				a \leq b
				& \iff b = \sup(\{a,b\}, \leq) \quad \text{(\rex{ex:orderings:posets:0008})} \\
				& \iff a \dmd b = b  \quad \text{(by assumption)} \\
				& \iff a \leq_{\dmd}^{\vee} b \quad \text{(\rex{ex:lattices:lattices:0003})}.
			\end{align*}
			Since $a,b \in X$ are arbitrary, we have that for each $a,b \in X$, $a \leq b$ iff $a \leq_{\dmd}^{\vee} b$. Hence $\leq = \leq_{\dmd}^{\vee}$. 
			\item Similar to $(1)$.
		\end{enumerate}
	\end{proof}
	
	\begin{ex} \lex{ex:lattices:lattices:0011}
		Let $(X, \leq)$ be a poset. 
		\begin{enumerate}
			\item If $(X, \leq)$ is a join-semilattice, then $\vee_{\leq}$ is the unique semilattice operator $\dmd$ on $X$ such that for each $a,b \in X$, $a \dmd b = \sup (\{a,b\}, \leq)$.
			\item If $(X, \leq)$ is a meet-semilattice, then $\wedge_{\leq}$ is the unique semilattice operator $\dmd$ on $X$ such that for each $a,b \in X$, $a \dmd b = \inf (\{a,b\}, \leq)$.
		\end{enumerate}
	\end{ex}
	
	\begin{proof}\
		\begin{enumerate}
			\item 
			Let $\dmd: X \times X \rightarrow X$ be a semilattice operator on $X$. Suppose that $a,b \in X$, $a \dmd b = \sup (\{a,b\}, \leq)$. Then for each $a,b \in X$,
			\begin{align*}
				a \dmd b
				& = \sup (\{a,b\}, \leq) \quad \text{(by assumption)} \\
				& = a \vee_{\leq} b \quad \text{(\rd{def:lattices:lattices:0007})}.
			\end{align*}
			Hence $\dmd = \vee_{\leq}$. 
			\item Similar to $(1)$.
		\end{enumerate}
	\end{proof}
	
	
	
	
	
	
	
	
	
	
	
	
	
	
	
	
	
	
	
	
	
	
	
	
	
	
	
	
	
	
	
	
	
	
	
	
	
	
	
	
	
	
	
	
	
	
	
	
	
	
	
	
	
	
	
	\subsection{Lattices}
	
	\begin{defn} \ld{def:lattices:lattices:0012}
		Let $X$ be a set and $\vee, \wedge$ semilattice operators on $X$. Then $\vee$ and $\wedge$ are said to \tbf{satisfy the lattice absorption identities} if for each $a,b \in X$,
		\begin{enumerate}
			\item $a \wedge (a \vee b) = a$
			\item $a \vee (a \wedge b) = a$
		\end{enumerate} 
	\end{defn}
	
	\begin{ex} \lex{ex:lattices:lattices:0013}
		Let $X$ be a set and $\vee, \wedge$ semilattice operators on $X$. Then $\vee$ and $\wedge$ satisfy the lattice absorption identities iff $\vee$ and $\wedge$ satisfy the skew lattice absorption identities.
	\end{ex}
	
	\begin{proof}\
		\begin{itemize}
			\item $(\implies)$: \\
			Clear by commutativity.
			\item $(\impliedby)$: \\
			Immediate.
		\end{itemize}
	\end{proof}
	
	\begin{defn} \ld{def:lattices:lattices:0014}
		Let $X$ be a set and $\vee, \wedge$ semilattice operators on $X$. Then $(X, \vee, \wedge)$ is said to be an \tbf{lattice} if $\vee$ and $\wedge$ satisfy the skew lattice absorption identities.
	\end{defn}
	
	\begin{defn} \ld{def:lattices:lattices:0015}
		Let $(X, \leq)$ be a poset. Then $(X, \leq)$ is said to be an \tbf{ordered lattice} if $(X, \leq)$ is a join-semilattice and $(X, \leq)$ is a meet-semilattice.
	\end{defn}
	
	\begin{ex} \lex{ex:lattices:lattices:0016}
		Let $(X, \vee, \wedge)$ be an lattice. Then $(X, \leq_{\vee}^{\vee})$ is an ordered lattice.
	\end{ex}
	
	\begin{proof}
		\rex{ex:lattices:lattices:0006} implies that $(X, \leq_{\vee}^{\vee})$ is a join-semilattice and $(X, \leq_{\wedge}^{\wedge})$ is a meet-semilattice. \rex{ex:lattices:skew_lattices:0010} implies that $\leq_{\vee}^{\vee} = \leq_{\wedge}^{\wedge}$. Therefore $(X, \leq_{\vee}^{\vee})$ is a join-semilattice and $(X, \leq_{\vee}^{\vee})$ is a meet-semilattice. Hence $(X, \leq_{\vee}^{\vee})$ is an ordered lattice.
	\end{proof}
	
	\begin{ex} \lex{ex:lattices:lattices:0017}
		Let $(X, \leq)$ be an ordered lattice. Then $(X, \vee_{\leq}, \wedge_{\leq})$ is an lattice. 
	\end{ex}
	
	\begin{proof} Let $a,b \in X$. 
		\begin{enumerate}
			\item 
			\begin{itemize}
				\item Since $\sup \{a,b\} \in \ubd \{a,b\}$ is an upper bound of $\{a,b\}$, we have that $a \leq \sup \{a,b\}$. Since $a \leq a$, we have that $a \in \lbd \{a, \sup\{a, b \} \}$. Hence $a \leq \inf \{a, \sup\{a, b \} \}$. 
				\item Since $\inf \{a, \sup\{a,b\}\} \in \lbd \{a, \sup \{a,b\}\}$, we have that $\inf \{a, \sup\{a,b\}\} \leq a$.
			\end{itemize}
			Since $a \leq \inf \{a, \sup\{a, b \} \}$ and $\inf \{a, \sup\{a,b\}\} \leq a$, we have that 
			\begin{align*}
				a \wedge_{\leq} (a \vee_{\leq} b)
				& = \inf \{a, \sup\{a, b \} \} \\
				& = a.
			\end{align*}
			\item Similarly, $a \vee_{\leq} (a \wedge_{\leq} b) = a$.
		\end{enumerate}
		Since $a, b \in X$ are arbitrary, we have that for each $a,b \in X$, 
		\begin{enumerate}
			\item $a \wedge_{\leq} (a \vee_{\leq} b) = a$
			\item $a \vee_{\leq} (a \wedge_{\leq} b) = a$
		\end{enumerate} 
		Hence $\vee_{\leq}$ and $\wedge_{\leq}$ satisfy the lattice absorption identities. Therefore $(X, \vee_{\leq}, \wedge_{\leq})$ is an lattice.
	\end{proof}
	
	\begin{ex} \lex{ex:lattices:lattices:0018}
		Let $(X, \leq)$ be an ordered lattice. Then
		\begin{enumerate}
			\item $\leq_{\vee_{\leq}}^{\vee} = \leq_{\wedge_{\leq}}^{\wedge} = \leq$
			\item $\leq_{\vee_{\leq}}^{\wedge} = \leq_{\wedge_{\leq}}^{\vee} = \lop$
		\end{enumerate}
	\end{ex}
	
	\begin{proof}
		Since $(X, \leq)$ is an ordered lattice, $(X, \leq)$ is a join-semilattice $(X, \leq)$ is a meet-semilattice. \rex{ex:lattices:lattices:0009} then implies that 
		\begin{enumerate}
			\item 
			\begin{align*}
				\leq_{\vee_{\leq}}^{\vee} 
				& = \leq \\
				& = \leq_{\wedge_{\leq}}^{\wedge} 
			\end{align*}
			\item 
			\begin{align*}
				\leq_{\vee_{\leq}}^{\wedge}
				& = \lop \\
				& = \leq_{\wedge_{\leq}}^{\vee}
			\end{align*}
		\end{enumerate}
	\end{proof}
	
	\begin{ex} \lex{ex:lattices:lattices:0019}
		Let $(X, \vee, \wedge)$ be an lattice. Then
		$\leq_{\vee}^{\vee}$ is the unique partial order ${\leq}$ on $X$ such that $(X, \leq)$ is an ordered lattice and for each $a,b \in X$, $a \vee b = \sup (\{a,b\}, \leq)$.
	\end{ex}
	
	\begin{proof}\
		Let $\leq$ be a partial order on $X$. Suppose that $(X, \leq)$ is an ordered lattice and for each $a,b \in X$, $a \vee b = \sup (\{a,b\}, \leq)$. Since $(X, \vee)$ is a semilattice, $(X, \leq_{\vee}^{\vee})$ and $(X, \leq)$ are join-semilattices and for each $a,b \in X$, $a \vee b = \sup(\{a,b\}, \leq_{\vee}^{\vee})$ and $a \vee b = \sup (\{a,b\}, \leq)$, \rex{ex:lattices:lattices:0010} implies that ${\leq} = {\leq_{\vee}^{\vee}}$.
	\end{proof}
	
	\begin{ex} \lex{ex:lattices:lattices:0020} 
		Let $(X, \leq)$ be an ordered lattice. Then $\vee_{\leq}, \wedge_{\leq}$ are the unique semilattice operators $\vee, \wedge$ on $X$ such that $(X, \vee, \wedge)$ is an lattice and for each $a,b \in X$, $a \vee b = \sup(\{a,b\}, \leq)$
	\end{ex}
	
	\begin{proof}\
		Let $\vee, \wedge$ be semilattice operators on $X$. Suppose that $(X, \vee, \wedge)$ is a lattice and for each $a,b \in X$, $a \vee b = \sup(\{a,b\}, \leq)$.
		Then for each $a,b \in X$,
		\begin{align*}
			a \vee_{\leq} b
			& = \sup(\{a,b\}, \leq) \quad \text{(\rd{def:lattices:lattices:0007})} \\
			& = a \vee b \quad \text{(by assumption)}.
		\end{align*}
		Thus $\vee = \vee_{\leq}$. Since $(X, \vee, \wedge)$ is a lattice and for each $a,b \in X$, $a \vee b = \sup(\{a,b\}, \leq)$, \rex{ex:lattices:lattices:0019} implies that $\leq_{\vee}^{\vee} = \leq$.  \rex{ex:lattices:skew_lattices:0010} then implies that 
		\begin{align*}
			\leq_{\wedge}^{\wedge} 
			& = \leq_{\vee}^{\vee} \\
			& = \leq.
		\end{align*}
		Therefore for each $a,b \in X$, 
		\begin{align*}
			a \wedge b
			& = \inf(\{a,b\}, \leq_{\wedge}^{\wedge}) \quad \text{(\rex{ex:lattices:lattices:0004})} \\
			& = \inf(\{a,b\}, \leq) \\
			& = a \wedge_{\leq} b \quad \text{(\rd{def:lattices:lattices:0007})}.
		\end{align*}
		Hence $\wedge = \wedge_{\leq}$.
	\end{proof}
	
	
	
	
	
	
	
	
	
	
	
	
	
	
	
	
	
	
	
	
	
	
	
	
	
	
	
	
	
	
	
	
	
	
	
	
	
	
	
	
	
	
	
	
	
	
	
	
	
	
	\newpage
	\section{Basic Structures}
	
	\subsection{Bounded Lattices}
	
	\begin{defn}
		Let $L$ be a lattice. 
		\begin{itemize}
			\item Let $a \in L$. Then 
			\begin{itemize}
				\item $a$ is said to be a \tbf{one} of $L$ if for each $x \in L$, $x \wedge a = x$
				\item $a$ is said to be a \tbf{zero} of $L$ if for each $x \in L$, $x \vee a = x$.
			\end{itemize}
			\item Then 
			\begin{itemize}
				\item $L$ is said to \tbf{have a zero} if there exists $0 \in L$ such that $0$ is a zero of $L$
				\item $L$ is said to \tbf{have a one} if there exists $1 \in L$ such that $1$ is a one of $L$
				\item $L$ is said to be \tbf{bounded} if $L$ has a zero and $L$ has a one.
			\end{itemize}
		\end{itemize}
	\end{defn}
	
	\begin{ex}
		Let $L$ be a lattice and $a,b \in L$. 
		\begin{enumerate}
			\item If $a,b$ are zeros of $L$, then $a = b$.
			\item If $a,b$ are ones of $L$, then $a = b$.
		\end{enumerate}
	\end{ex}
	
	\begin{proof}\
		\begin{enumerate}
			\item Suppose that $a,b$ are zeros of $L$. Then 
			\begin{align*}
				a
				& = a \wedge b \\
				& = b \wedge a \\
				& = b. 
			\end{align*}
			\item Similar to $(1)$. 
		\end{enumerate}
	\end{proof}
	
	\begin{note}\
		\begin{itemize}
			\item If $L$ has a one, we denote the unique one of $L$ by $1$.
			\item If $L$ has a zero, we denote the unique zero of $L$ by $0$.
		\end{itemize}
	\end{note}
	
	\begin{ex}
		Let $L$ be a lattice. Then 
		\begin{enumerate}
			\item there exists $a \in L$ such that $a$ is a one of $L$ iff $\ubd L = \{a\}$.
			\item there exists $a \in L$ such that $a$ is a zero of $L$ iff $\lbd L = \{a\}$.
		\end{enumerate} 
	\end{ex}
	
	\begin{proof}\
		\begin{enumerate}
			\item 
			\begin{itemize}
				\item $(\implies)$: \\
				Suppose that there exists $a \in L$ such that $x$ is a one of $L$. Let $x \in L$. Since $x \wedge a = x$, we have that $x \leq a$. Since $x \in L$ is arbitrary, we have that for each $x \in L$, $x \leq a$. Hence $a \in \ubd L$. \rex{ex:orderings:posets:0009} then implies that $\ubd L = \{a\}$.
				\item $(\impliedby)$: \\
				Suppose that $\ubd L = \{a\}$. Let $x \in L$. Then Since $a \in \ubd L$, $x \leq a$. Therefore
				\begin{align*}
					x \wedge a
					& = \inf \{x, a\} \\
					& = x.
				\end{align*}
				Since $x \in L$ is arbitrary, we have that for each $x \in L$, $x \wedge a = x$. Hence $a$ is a one of $L$.
			\end{itemize}
			\item Similar to $(1)$.
		\end{enumerate}
	\end{proof}
	
	
	
	
	
	
	
	
	
	
	
	
	
	
	
	
	
	
	
	
	
	
	
	
	
	
	
	
	
	
	
	
	
	
	
	
	
	
	
	
	
	
	
	
	
	
	
	
	
	
	
	
	\subsection{Complete Lattices}
	
	\begin{ex}
		Let $(X, \leq)$ be a poset. Then for each $A \subset X$, $\sup A$ exists iff for each $A \subset X$, $\inf A$ exists. 
	\end{ex}
	
	\begin{proof}\
		\begin{itemize}
			\item $(\implies)$: \\
			Suppose that for each $A \subset X$, $\sup A$ exists. Let $A \subset X$. Define 
			\item $(\impliedby)$: \\
		\end{itemize}
	\end{proof}
	
	\begin{defn}
		Let $(X, \leq)$ be a poset. Then $(X, \leq)$ is said to be a \tbf{complete lattice} if for each $A \subset X$, $\sup A$ exists. 
	\end{defn}
	
	
	
	
	
	
	
	
	
	
	
	
	
	
	
	
	
	
	
	
	
	
	
	
	
	
	
	
	
	
	
	
	
	
	
	
	
	
	
	
	
	
	
	
	
	
	
	
	\subsection{Irreducibility and Primality}
	
	\begin{defn}
		Let $(L, \leq)$ be a poset.
		\begin{itemize}
			\item Suppose that $(L, \leq)$ is a join-semilattice.
			\begin{itemize}
				\item Let $x \in L$. Then $x$ is said to be
				\begin{itemize}
					\item \tbf{join-irreducible} if 
					\begin{enumerate}
						\item $x$ is not a zero of $L$,
						\item for each $a,b \in L$, $x = a \vee b$ implies that $x = a$ or $x = b$ 
					\end{enumerate}
					\item \tbf{join-prime}
					\begin{enumerate}
						\item $x$ is not a zero of $L$,
						\item if for each $a,b \in L$, $x \leq a \vee b$ implies that $x \leq a$ or $x \leq b$
					\end{enumerate}
				\end{itemize}
				\item We define 
				\begin{itemize}
					\item $JI(L) \defeq \{x \in L: \text{$x$ is join-irreducible.}\}$
					\item $JP(L) \defeq \{x \in L: \text{$x$ is join-prime.}\}$
				\end{itemize}
			\end{itemize}
			\item Suppose that $(L, \leq)$ is a meet-semilattice. 
			\begin{itemize}
				\item Let $x \in L$. Then $x$ is said to be 
				\begin{itemize}
					\item \tbf{meet-irreducible} 
					\begin{enumerate}
						\item $x$ is not a one of $L$,
						\item if for each $a,b \in L$, $x = a \wedge b$ implies that $x = a$ or $x = b$ 
					\end{enumerate}
					\item \tbf{meet-prime} if
					\begin{enumerate}
						\item $x$ is not a one of $L$,
						\item for each $a,b \in L$, $a \wedge b \leq x$ implies that $a \leq x$ or $b \leq x$
					\end{enumerate}
				\end{itemize}
				\item We define 
				\begin{itemize}
					\item $MI(L) \defeq \{x \in L: \text{$x$ is meet-irreducible.}\}$
					\item $MP(L) \defeq \{x \in L: \text{$x$ is meet-prime.}\}$
				\end{itemize}
			\end{itemize}
		\end{itemize}
	\end{defn}
	
	\begin{ex}
		Let $(L, \leq)$ be a poset. 
		\begin{enumerate}
			\item Suppose that $(L, \leq)$ is a join-semilattice. Then $JP(L) \subset LI(L)$. 
			\item Suppose that $(L, \leq)$ is a meet-semilattice. Then $MP(L) \subset MI(L)$. 
		\end{enumerate}
	\end{ex}
	
	\begin{proof}\
		\begin{enumerate}
			\item Let $x \in JP(L)$.
			\begin{enumerate}
				\item Since $x$ is join-prime, $x$ is not at zero of $L$.
				\item Let $a,b \in L$. Suppose that $x = a \vee b$. Then $x \leq a \vee b$. Since $x$ is join-prime, $x \leq a$ or $x \leq b$. 
				\begin{itemize}
					\item Suppose that $x \leq a$. Then 
					\begin{align*}
						a \vee b
						& = x \\
						& \leq a \\
						& \leq a \vee b
					\end{align*}
					and therefore $x = a$. Thus $x \leq a$ implies that $x = a$.
					\item Similarly, $x \leq b$ implies that $x = b$. 
				\end{itemize}
				Since $x \leq a$ or $x \leq b$, we have that $x = a$ or $x = b$. Since $a,b \in L$ with $x = a \vee b$ are arbitrary, we have that for each $a,b \in L$, $x = a \vee b$ implies that $x = a$ or $x = b$.
			\end{enumerate} 
			Thus $x \in JI(L)$. Since $x \in JP(L)$ is arbitrary, we have that for each $x \in JP(L)$, $x \in JI(L)$. Hence $JP(L) \subset JI(L)$.
			\item Similar to $(1)$ \tcr{use duality}. 
		\end{enumerate}
	\end{proof}
	
	
	
	
	
	
	
	
	
	
	
	
	
	
	
	
	
	
	
	
	
	
	
	
	
	
	
	
	
	
	
	
	
	
	
	
	
	
	
	
	
	
	
	
	
	
	
	
	
	
	
	
	
	
	
	
	
	
	
	
	
	
	
	
	
	
	
	
	
	
	
	
	
	
	
	
	
	
	
	
	
	
	
	
	
	
	
	
	
	
	
	\subsection{Lattice Homomorphisms}
	
	\begin{defn}
		Let $(X, \leq_X), (Y, \leq_Y)$ be posets and $f:X \rightarrow Y$. 
		\begin{enumerate}
			\item Suppose that $(X, \leq_X), (Y, \leq_Y)$ are join-semilattices. Then $f$ is said to be \tbf{finite $(\leq_X, \leq_Y)$-join preserving} if for each $a,b \in X$, $f(a \vee_{\leq_X} b) = f(a) \vee_{\leq_Y} f(b)$
			\item Suppose that $(X, \leq_X), (Y, \leq_Y)$ are meet-semilattices. Then $f$ is said to be \tbf{finite $(\leq_X, \leq_Y)$-meet preserving} if for each $a,b \in X$, $f(a \wedge_{\leq_X} b) = f(a) \wedge_{\leq_Y} f(b)$
		\end{enumerate}
	\end{defn}
	
	\begin{ex}
		Let $(X, \leq_X), (Y, \leq_Y)$ be posets. 
		\begin{enumerate}
			\item Suppose that $(X, \leq_X)$ and $(Y, \leq_Y)$ are join semilattices. Then $f$ is $(\leq_X, \leq_Y)$-monotone iff for each $a,b \in X$, $f(a) \vee_{\leq_Y} f(b) \leq_Y f(a \vee_{\leq_X} b)$.
			\item Suppose that $(X, \leq_X)$ and $(Y, \leq_Y)$ are meet semilattices. Then $f$ is $(\leq_X, \leq_Y)$-monotone iff for each $a,b \in X$, $f(a) \vee_{\leq_Y} f(b) \leq_Y f(a \vee_{\leq_X} b)$.
		\end{enumerate}
	\end{ex}
	
	\begin{proof}\
		\begin{enumerate}
			\item 
			\begin{itemize}
				\item $(\implies)$: \\
				Suppose that $f$ is $(\leq_X, \leq_Y)$-monotone. Let $a,b \in X$. \rex{ex:orderings:prosets:0030} implies that $f(a \vee_{\leq_X} b) \in \ubd(\{f(a), f(b)\}, \leq_Y)$. Therefore
				\begin{align*}
					f(a) \vee_{\leq_Y} f(b)
					& = \sup (\{f(a), f(b)\}, \leq_Y) \\
					& \leq f(a \vee_{\leq_X} b).
				\end{align*}
				Since $a,b \in X$ are arbitrary, we have that for each $a,b \in X$, $f(a) \vee_{\leq_Y} f(b) \leq_Y f(a \vee_{\leq_X} b)$.
				\item $(\impliedby)$: \\
				Suppose that for each $a,b \in X$, $f(a) \vee_{\leq_Y} f(b) \leq_Y f(a \vee_{\leq_X} b)$. Let $a,b \in X$. Suppose that $a \leq_X b$. \rex{ex:orderings:posets:0008} then implies that $a \vee_{\leq_X} b = b$. By assumption, 
				\begin{align*}
					f(a) \vee_{\leq_Y} f(b) 
					& \leq_Y f(a \vee_{\leq_X} b) \\
					& = f(b) \\
					& \leq f(a) \vee_{\leq_Y} f(b).
				\end{align*}
				Therefore $f(a) \vee_{\leq_Y} f(b) = f(b)$. Another application of \rex{ex:orderings:posets:0008} implies that $f(a) \leq_Y f(b)$. Since $a,b \in X$ with $a \leq b$ are arbitrary, we have that for each $a,b \in X$, $a \leq_X b$ implies that $f(a) \leq_Y f(b)$. Hence $f$ is $(\leq_X, \leq_Y)$-monotone.
			\end{itemize}
			\item \tcr{use duality}
			\tcr{FINISH!!!}
		\end{enumerate}
	\end{proof}
	
	\begin{ex}
		Let $(X, \leq_X), (Y, \leq_Y)$ be posets. 
		\begin{enumerate}
			\item Suppose that $(X, \leq_X)$ and $(Y, \leq_Y)$ are join semilattices. If $f$ preserves finite joins, then $f$ is $(\leq_X, \leq_Y)$-monotone.
			\item Suppose that $(X, \leq_X)$ and $(Y, \leq_Y)$ are meet semilattices. If $f$ preserves finite meets, then $f$ is $(\leq_X, \leq_Y)$-monotone.
		\end{enumerate}
	\end{ex}
	
	\begin{proof}\
		\begin{enumerate}
			\item Suppose that $f$ preserves finite joins. Then \tcr{the previous exercise} implies that
			\begin{align*}
				\text{for each $a,b \in X$, $f(a) \vee_{\leq_Y} f(b) =  f(a \vee_{\leq_X} b)$}
				& \implies \text{for each $a,b \in X$, $f(a) \vee_{\leq_Y} f(b) \leq  f(a \vee_{\leq_X} b)$} \\
				& \implies \text{$f$ is $(\leq_X, \leq_Y)$-monotone}
			\end{align*}
			\item Suppose that $(X, \leq_X)$ and $(Y, \leq_Y)$ are meet semilattices. If $f$ preserves finite meets, then $f$ is $(\leq_X, \leq_Y)$-monotone.
		\end{enumerate}
	\end{proof}
	
	\begin{defn}
		Let $(X, \leq_X), (Y, \leq_Y)$ be complete lattices and $f:X \rightarrow Y$. Then $f$ is said to 
		\begin{enumerate}
			\item \tbf{preserve arbitrary joins} if for each $A \subset X$, $f(\sup (A, \leq_X) ) = \sup(f(A), \leq_Y)$
			\item \tbf{preserve arbitrary meets} if for each $A \subset X$, $f(\inf (A, \leq_X) ) = \inf(f(A), \leq_Y)$
		\end{enumerate}
	\end{defn}
	
	
	
	
	
	
	
	
	
	
	
	
	
	
	
	
	
	
	
	
	
	
	
	
	
	
	
	
	
	
	
	
	
	
	
	
	
	
	
	
	
	
	
	
	
	
	
	
	
	
	
	
	
	
	
	
	
	
	
	\newpage
	\section{Lattice Ideals and Filters}

	\subsection{Introduction}
	
	\begin{defn}
		Let $L$ be a lattice.  
		\begin{itemize}
			\item Let $J \subset L$. Then
			\begin{itemize}
				\item $J$ is said to be an \tbf{ideal of $L$} if 
				\begin{enumerate}
					\item $J \neq \varnothing$,
					\item for each $a,b \in J$, $a \vee b \in J$,
					\item for each $x \in L$ and $x \in J$, $x \wedge a \in J$,
				\end{enumerate}
				\item $J$ is said to be an \tbf{filter of $L$} if 
				\begin{enumerate}
					\item $J \neq \varnothing$,
					\item for each $a,b \in J$, $a \wedge b \in J$,
					\item for each $x \in L$ and $a \in J$, $x \vee a \in J$.
				\end{enumerate}
			\end{itemize}
			\item We define 
			\begin{itemize}
				\item $\MI(L) \defeq \{J \subset L: \text{$J$ is an ideal of $L$}\}$
				\item $\MF(L) \defeq \{J \subset L: \text{$J$ is a filter of $L$}\}$
			\end{itemize}
		\end{itemize}
	\end{defn}
	
	\begin{ex}
		Let $L$ be a lattice. If $L \neq \varnothing$, then 
		\begin{enumerate}
			\item $L \in \MI(L)$,
			\item $L \in \MF(L)$.
		\end{enumerate}
	\end{ex}
	
	\begin{proof}
		Clear. \tcr{(maybe fill out later)}
	\end{proof}
	
	\begin{ex}
		Let $L$ be a lattice and $J \subset L$. Set $\leq_J \defeq \leq|_{J^2}$. Then
		\begin{enumerate}
			\item $J \in \MI(L)$ iff $J$ is a $\leq$-lower set and $(J, \leq_J)$ is upward directed. \tcr{rework after making some exercises about subprosets, subdirected sets and subposets and showing facts about $\sup \/ \inf$ of subprosets and upper/lower sets in sub-prosets}  
			\item $J \in \MF(L)$ iff $J$ is an $\leq$-upper set and $(J, \leq_J)$ is downward directed.
		\end{enumerate}
	\end{ex}
	
	\begin{proof}\
		\begin{enumerate}
			\item 
			\begin{itemize}
				\item $(\implies)$: \\
				Suppose that $J \in \MI(L)$.
				\begin{itemize}
					\item Let $x \in L$ and $a \in J$. Suppose that $x \leq a$. Then $x = \inf(\{a,b\}, \leq)$ and therefore
					\begin{align*}
						x
						& = \inf(\{x, a\}, \leq) \\
						& = x \wedge_{\leq} a \\
						& \in J. 
					\end{align*}  
					Since $a \in J$ and $x \in L$ with $x \leq a$ are arbitrary, we have that $J$ is a down set. 
					\item Since $J \in \MI(L)$, we have that $J \neq \varnothing$. Let $a, b \in J$. Since $J \in \MI(L)$, 
					\begin{align*}
						\sup (\{a,b\}, \leq) \\
						& = a \vee_{\leq} b \\
						& \in J.
					\end{align*}
					Hence \tcr{(ref ex here)}  
					\begin{align*}
						\sup (\{a,b\}, \leq)
						& = \sup (\{a,b\}, \leq_J) \\
						& \in \ubd(\{a,b\}, \leq_J)
					\end{align*}
					and $\ubd(\{a,b\}, \leq_J) \neq \varnothing$.
					Since $a,b \in J$ are arbitrary, we have that for each $a,b \in J$, $\ubd (\{a,b\}, \leq_J) \neq \varnothing$. Since $J \neq \varnothing$ and for each $a,b \in J$, $\ubd(J, \leq_J) \neq \varnothing$, we have that $(J, \leq_J)$ is upward directed.
				\end{itemize}
				\item $(\impliedby)$: \\
				Suppose that $J$ is a $\leq$-lower set and $(J, \leq_J)$ is upward directed. 
				\begin{enumerate}
					\item Since $(J, \leq_J)$ is upward directed, $J \neq \varnothing$.
					\item Let $a,b \in J$. Since $(J, \leq_J)$ is upward directed, $\ubd(\{a,b\}, \leq_J) \neq \varnothing$. Thus there exists \tcr{(ref ex here)} 
					\begin{align}
						c 
						& \in \ubd(\{a,b\}, \leq_J) \\
						& \subset \ubd(\{a,b\}, \leq).
					\end{align}
					Therefore $\sup(\{a,b\}, \leq) \leq c$. Since $J$ is a $\leq$-lower set, $c \in J$ and $\sup(\{a,b\}, \leq) \leq c$, we have that 
					\begin{align*}
						a \vee b
						& = \sup(\{a,b\}, \leq) \\
						& \in J.
					\end{align*}
					Since $a,b \in J$ are arbitrary, we have that for each $a,b \in J$, $a \vee b \in J$.
					\tcr{rework after making some exercises about subprosets, subdirected sets and subposets and showing facts about $\sup \/ \inf$ of subprosets and upper/lower sets in sub-prosets}
					\item Let $x \in L$ and $a \in J$. Then $x \wedge a \leq a$. Since $J$ is a $\leq$-lower set, $x \wedge a \in J$. Since $x \in L$ and $a \in J$ are arbitrary, we have that for each $x \in L$ and $a \in J$, $x \wedge a \in J$.
				\end{enumerate}
				Therefore $J \in \MI(L)$.
			\end{itemize}
			\item 
		\end{enumerate}
	\end{proof}
	
	\begin{ex}
		Let $L$ be a lattice and $J \subset L$.
		\begin{enumerate}
			\item If $L$ has a zero and $J \in \MI(L)$, then $0 \in J$.
			\item If $L$ has a one and $J \in \MF(L)$, then $1 \in J$.
		\end{enumerate} 
	\end{ex}
	
	\begin{proof}\
		\begin{enumerate}
			\item Suppose that $L$ has a zero and $J \in \MI(L)$. Since $J$ is an ideal of $L$, $J \neq \varnothing$ and \tcr{a previous ex} implies $J$ is a lower set. Hence there exists $a \in J$. \tcr{a prev ex} implies that $0 \in \lbd L$. Thus $0 \leq a$. Since $J$ is a lower set, $0 \in J$.
			\item Similar to $(1)$.
		\end{enumerate} 
	\end{proof}
	
	\begin{ex}
		Let $L$ be a lattice. 
		\begin{enumerate}
			\item Let $(J_{\al})_{\al \in A} \subset \MI(L)$. 
			\begin{enumerate}
				\item If $\bigcap\limits_{\al \in A} J_{\al} \neq \varnothing$, then $\bigcap\limits_{\al \in A} J_{\al} \in \MI(L)$. 
				\item If $A = \{1,2\}$, then $\bigcap\limits_{\al \in A} J_{\al} \in \MI(L)$.
				\item If $L$ has a zero, then $\bigcap\limits_{\al \in A} J_{\al} \in \MI(L)$.
			\end{enumerate}
			\item Let $(J_{\al})_{\al \in A} \subset \MF(L)$. 
			\begin{enumerate}
				\item If $\bigcap\limits_{\al \in A} J_{\al} \neq \varnothing$, then $\bigcap\limits_{\al \in A} J_{\al} \in \MF(L)$. 
				\item If $A = \{1,2\}$, then $\bigcap\limits_{\al \in A} J_{\al} \in \MF(L)$.
				\item If $L$ has a one, then $\bigcap\limits_{\al \in A} J_{\al} \in \MF(L)$. 
			\end{enumerate}
		\end{enumerate}
	\end{ex}
	
	\begin{proof} Set $J \defeq \bigcap\limits_{\al \in A} J_{\al}$.
		\begin{enumerate}
			\item 
			\begin{enumerate}
				\item Suppose that $J \neq \varnothing$.
				\begin{enumerate}
					\item By assumption $J \neq \varnothing$.
					\item Let $a, b \in J$ and $\al \in A$. Then $a,b \in J_{\al}$. Since $J_{\al} \in \MI(L)$, $a \vee b \in J_{\al}$. Since $\al \in A$ is arbitrary, we have that for each $\al \in A$, $a \vee b \in J_{\al}$. Hence $a \vee b \in J$.
					\item Let $x \in L$, $a \in J$ and $\al \in A$. Then $a \in J_{\al}$. Since $J_{\al} \in \MI(L)$, $x \in L$ and $a \in J_{\al}$, we have that $x \wedge a \in J_{\al}$. Since $\al \in A$ is arbitrary, we have that for each $\al \in A$, $x \wedge a \in J_{\al}$. Hence $x \wedge a \in J$. Since $x \in L$ and $a \in J$ are arbitrary, we have that for each $x \in L$ and $a \in J$, $x \wedge a \in J$. 
				\end{enumerate}
				Thus $J \in \MI(L)$.
				\item Suppose that $A = \{1,2\}$. 
				\begin{enumerate}
					\item Since $J_1, J_2 \in \MI(L)$, $J_1 \neq \varnothing$ and $J_2 \neq \varnothing$. Thus there exist $x_1 \in J_1$ and $x_2 \in J_2$. Since $J_1 \in \MI(L)$, $x_2 \in L$ and $x_1 \in J_1$, we have that $x_1 \wedge x_2 \in J_1$. Similarly, $x_1 \wedge x_2 \in J_2$. Hence $x_1 \wedge x_2 \in J_1 \cap J_2$ and $J_1 \cap J_2 \neq \varnothing$. Part $1(a)$ implies that $J_1 \cap J_2 \in \MI(L)$.
				\end{enumerate}
				\item Suppose that $L$ has a zero. Let $\al \in A$. Since $J_{\al} \in \MI(L)$, \tcr{a previous exercise} implies that $0 \in J_{\al}$. Since $\al \in A$ is arbitrary, we have that for each $\al \in A$, $0 \in J_{\al}$. Hence $0 \in J$. Thus $J \neq \varnothing$. The 
			\end{enumerate}
			\item Similar to $(1)$.
		\end{enumerate}
	\end{proof}
	
	\begin{ex}
		Let $L$ be a lattice. Then 
		\begin{enumerate}
			\item for each $J_1, J_2 \in \MI(L)$, $\{x \in L: \text{there exist $a \in J_1$ and $b \in J_2$ such that $x \leq a \vee b$} \} \in \MI(L)$.
			\item for each $J_1, J_2 \in \MF(L)$, $\{x \in L: \text{there exist $a \in J_1$ and $b \in J_2$ such that $a \vee b \leq x$} \} \in \MF(L)$ \tcr{(maybe FIX!!!)}.
		\end{enumerate}
	\end{ex}
	
	\begin{proof}\
		\begin{enumerate}
			\item Let $J_1, J_2 \in \MI(L)$. Set $J \defeq \{x \in L: \text{there exist $a \in J_1$ and $b \in J_2$ such that $x \leq a \vee b$} \}$. 
			\begin{enumerate}
				\item Since $J_1, J_2 \in \MI(L)$, we have that $J_1 \neq \varnothing$ and $J_2 \neq \varnothing$. Hence there exist $x_1 \in J_1$ and $x_2 \in J_2$. Since $x_1 \leq x_1 \vee x_2$, we have that $x_1 \in J$. Thus $J \neq \varnothing$. 
				\item Let $a,b \in J$. Then there exist $x_a, x_b \in J_1$ and $y_a, y_b \in J_2$ such that $a \leq x_a \vee y_a$ and $b \leq x_b \vee y_b$. Define $z_1, z_2 \in L$ by $z_1 \defeq x_a \vee x_b$ and $z_2 \defeq y_a \vee y_b$. Since $J_1, J_2 \in \MI(L)$, we have that $z_1 \in J_1$ and $z_2 \in J_2$. We note that
				\begin{align*}
					a 
					& \leq x_a \vee y_a \\
					& \leq (x_a \vee y_a) \vee (x_b \vee y_b) \\
					& = (x_a \vee x_b) \vee (y_a \vee y_b) \\
					& = z_1 \vee z_2.
				\end{align*}
				and similarly $b \leq z_1 \vee z_2$. Therefore $z_1 \vee z_2 \in \ubd(\{a, b\}, \leq)$. Hence
				\begin{align*}
					a \vee b
					& = \sup(\{a, b\}, \leq) \\
					& \leq z_1 \vee z_2.
				\end{align*}
				Thus $a \vee b \in J$. Since $a,b \in J$ are arbitrary, we have that for each $a,b \in J$, $a \vee b \in J$. 
				\item Let $x \in L$ and $y \in J$. Then there exists $a \in J_1$ and $b \in J_2$ such that $y \leq a \vee b$. Then 
				\begin{align*}
					x \wedge y
					& \leq y \\
					& \leq a \vee b. 
				\end{align*}
				Therefore $x \wedge y \in J$. Since $x \in L$ and $y \in J$ are arbitrary, we have that for each $x \in L$ and $y \in J$, $x \wedge y \in J$. 
			\end{enumerate}
			Therefore $J \in \MI(L)$.
			\item Similar to $(1)$. \tcr{FINISH!!!}
		\end{enumerate}
	\end{proof}
	
	\begin{defn}
		Let $L$ be a lattice. Define ${\leq_{\MI(L)}} \defeq {\subset|_{\MI(L)}}$ \tcr{(maybe standardize notation)}.
	\end{defn}
	
	\begin{ex}
		Let $L$ be a lattice. 
		\begin{enumerate}
			\item Let $J_1, J_2 \in \MI(L)$.
			\begin{enumerate}
				\item $\sup(\{J_1, J_2 \}, \leq_{\MI(L)}) = \{x \in L: \text{there exist $a \in J_1$ and $b \in J_2$ such that $x \leq a \vee b$} \}$.
				\item $\inf(\{J_1, J_2 \}, \leq_{\MI(L)}) = J_1 \cap J_2$
			\end{enumerate}
			\item Set ${\leq_{\MF(L)}} \defeq {\subset|_{\MF(L)}}$ \tcr{(maybe standardize notation)}. Let $J_1, J_2 \in \MF(L)$. \tcr{FIX!!!}
			\begin{enumerate}
				\item $\sup(\{J_1, J_2 \}, \leq_{\MI(L)}) = \{x \in L: \text{there exist $a \in J_1$ and $b \in J_2$ such that $x \leq a \vee b$} \}$.
				\item $\inf(\{J_1, J_2 \}, \leq_{\MI(L)}) = J_1 \cap J_2$
			\end{enumerate}
		\end{enumerate}
	\end{ex}
	
	\begin{proof}\
		\begin{enumerate}
			\item 
			\begin{enumerate}
				\item Set $J \defeq \{x \in L: \text{there exist $a \in J_1$ and $b \in J_2$ such that $x \leq a \vee b$} \}$. \tcr{(ref previous ex)} implies that $J \in \MI(L)$.
				\begin{enumerate}
					\item 
					\begin{itemize}
						\item Let $a \in J_1$. Since $J_2 \in \MI(L)$, $J_2 \neq \varnothing$. Hence there exists $b \in J_2$. Then $a \leq a \vee b$. Since $a \in J_1$, $b \in J_2$ and $a \leq a \vee b$, we have that $a \in J$. Since $a \in J_1$ is arbitrary, we have that for each $a \in J_1$, $a \in J$. Thus $J_1 \subset J$. 
						\item Similarly, $J_2 \subset J$.
					\end{itemize}
					Since $J_1,J_2 \subset J$, we have that $J \in \ubd(\{J_1, J_2\}, \leq_{\MI(L)})$. 
					\item Let $K \in \ubd(\{J_1, J_2\}, \leq_{\MI(L)})$ and $x \in J$. Then $J_1, J_2 \subset K$ and there exist $a \in J_1$ and $b \in J_2$ such that $x \leq a \vee b$. Since $a \in J_1$, $b \in J_2$ and $J_1, J_2 \subset K$, we have that $a,b \in K$. Since $K \in \MI(L)$, $a \vee b \in K$. Since $K \in \MI(L)$ \tcr{(ref a prev ex)} implies that $K$ is a lower set. Since $K$ is a lower set, $a \vee b \in K$ and $x \leq a \vee b$, we have that $x \in K$. Since $x \in J$ is arbitrary, we have that for each $x \in J$, $x \in K$. Hence $J \subset K$. Since $K \in \ubd(\{J_1, J_2\}, \leq_{\MI(L)})$ is arbitrary, we have that for each $K \in \ubd(\{J_1, J_2\}, \leq_{\MI(L)})$, $J \leq_{\MI(L)} K$.
				\end{enumerate}
				Since 
				\begin{enumerate}
					\item $J \in \ubd(\{J_1, J_2\}, \leq_{\MI(L)})$, 
					\item for each $K \in \ubd(\{J_1, J_2\}, \leq_{\MI(L)})$, $J \leq_{\MI(L)} K$,
				\end{enumerate}
				we have that $J = \sup(\{J_1, J_2\}, \leq_{\MI(L)})$.
				\item \tcr{(ref previous ex)} implies that $J_1 \cap J_2 \in \MI(L)$. Since $J_1 \cap J_2 \subset J_1$ and $J_1 \cap J_2 \subset J_2$, we have that $J_1 \cap J_2 \in \lbd(\{J_1, J_2\}, \leq_{\MI(L)})$. Since $(\MI(L), \leq_{\MI(L)})$ is a subposet of $(\MP(L), \subset)$, \tcr{(define subproset/subposet and make exercise for the following fact)} implies that
				\begin{align*}
					\inf (\{J_1, J_2\}, \leq_{\MI(L)})
					& \leq \inf (\{J_1, J_2\}, \subset) \\
					& = J_1 \cap J_2 \\
					& \leq \inf (\{J_1, J_2\}, \leq_{\MI(L)}).
				\end{align*} 
				Hence $J_1 \cap J_2 = \inf (\{J_1, J_2\}, \leq_{\MI(L)})$. 
			\end{enumerate}
			\item Similar to $(1)$. \tcr{FINISH!!!} 
		\end{enumerate}
	\end{proof}
	
	\begin{defn}
		Let $L$ be a lattice. Define $\vee_{\MI(L)}, \wedge_{\MI(L)}: \MI(L) \times \MI(L) \rightarrow \MI(L)$ and $\vee_{\MF(L)}, \wedge_{\MF(L)}: \MF(L) \times \MF(L) \rightarrow \MF(L)$ by 
		\begin{enumerate}
			\item 
			\begin{itemize}
				\item $J_1 \vee_{\MI(L)} J_2 \defeq \{x \in L: \text{there exist $a \in J_1$ and $b \in J_2$ such that $x \leq a \vee b$} \}$
				\item $J_1 \wedge_{\MI(L)} J_2 \defeq J_1 \cap J_2$. 
			\end{itemize}
			\item \tcr{FIX!!!}
			\begin{itemize}
				\item $J_1 \vee_{\MF(L)} J_2 \defeq \{x \in L: \text{there exist $a \in J_1$ and $b \in J_2$ such that $x \leq a \vee b$} \}$
				\item $J_1 \wedge_{\MF(L)} J_2 \defeq J_1 \cap J_2$. 
			\end{itemize}
		\end{enumerate}
	\end{defn}
	
	\begin{ex}
		Let $L$ be a lattice. Then 
		\begin{enumerate}
			\item 
			\begin{enumerate}
				\item $(\MI(L), \vee_{\MI(L)}, \wedge_{\MI(L)})$ is a lattice 
				\item $L$ has a zero implies that $(\MI(L), \vee_{\MI(L)}, \wedge_{\MI(L)})$ is a complete lattice.
			\end{enumerate}
			\item 
			\begin{enumerate}
				\item $(\MF(L), \vee_{\MF(L)}, \wedge_{\MF(L)})$ is a lattice 
				\item $L$ has a zero implies that $(\MF(L), \vee_{\MF(L)}, \wedge_{\MF(L)})$ is a complete lattice.
			\end{enumerate}
		\end{enumerate}
	\end{ex}
	
	\begin{proof}\
		\begin{enumerate}
			\item 
			\begin{enumerate}
				\item \tcr{the previous exercise} implies that $(\MI(L), \leq_{\MI(L)})$ is an ordered lattice. \rex{ex:lattices:lattices:0017} then implies that $(\MI(L), \vee_{\MI(L)}, \wedge_{\MI(L)})$ is a lattice. 
				\item 
			\end{enumerate}
			\item 
			\begin{enumerate}
				\item \tcr{FINISH!!!}
				\item 
			\end{enumerate}
		\end{enumerate}
	\end{proof}
	
	
	
	
	
	
	
	
	
	
	
	
	
	
	
	
	
	
	
	
	
	
	
	
	
	
	
	
	
	
	
	
	
	
	
	
	
	
	
	
	
	
	
	
	
	
	
	
	
	\subsection{Properties of Ideals and Filters}
	
	
	\begin{defn}
		Let $L$ be a lattice.
		\begin{itemize}
			\item Let $J \in \MI(L)$. Then $J$ is said to be 
			\begin{itemize}
				\item \tbf{proper} if $J \neq L$. 
				\item \tbf{maximal} if $J$ is maximal in $(\MI(L) \setminus \{L\}, \subset)$. 
				\item \tbf{prime} if 
				\begin{enumerate}
					\item $J$ is proper 
					\item for each $a,b \in L$, $a \wedge b \in J$ implies that $a \in J$ or $b \in J$. 
				\end{enumerate}
				\item \tbf{principal} if there exists $a \in L$ such that $J = \doar a$.
			\end{itemize}
			\item Let $J \in \MF(L)$. Then $J$ is said to be 
			\begin{itemize}
				\item \tbf{proper} if $J \neq L$. 
				\item \tbf{maximal} if $J$ is maximal in $(\MF(L) \setminus \{L\}, \subset)$. 
				\item \tbf{prime} if 
				\begin{enumerate}
					\item $J$ is proper 
					\item for each $a,b \in L$, $a \vee b \in J$ implies that $a \in J$ or $b \in J$.
				\end{enumerate}
				\item \tbf{principal} if there exists $a \in L$ such that $J = \upar a$. 
			\end{itemize}
		\end{itemize}
	\end{defn}
	
	\begin{ex}
		Let $L$ be a lattice and $J \subset L$.
		\begin{enumerate}
			\item Suppose $J \in \MI(L)$. 
			\begin{enumerate}
				\item If $J$ is proper, then for each $x \in J$, $x$ is not a one of $L$.
				\item If $J$ is prinicpal, then $J$ is proper iff for each $x \in J$, $x$ is not a one of $L$.
			\end{enumerate}
			\item Suppose $J \in \MF(L)$. 
			\begin{enumerate}
				\item If $J$ is proper, then for each $x \in J$, $x$ is not a zero of $L$.
				\item If $J$ is prinicpal, then $J$ is proper iff for each $x \in J$, $x$ is not a zero of $L$.
			\end{enumerate}
		\end{enumerate}
	\end{ex}
	
	\begin{proof}\
		\begin{enumerate}
			\item 
			\begin{enumerate}
				\item Suppose that $J$ is proper. Let $x \in J$. For the sake of contradiction, suppose that $x$ is a one of $L$. \tcr{(previous ex)} then implies that $\ub(L, \leq) = \{x\}$. Let $y \in L$. Then $y \leq x$. Since $J$ is a lower set, $x \in J$ and $y \leq x$, we have that $y \in J$. Since $y \in L$ is arbitrary, we have that for each $y \in L$, $y \in J$. Thus 
				\begin{align*}
					L 
					& \subset J \\
					& \subset L.
				\end{align*}
				Thus $J = L$ and $J$ is not proper. Therefore $x$ is not a one of $L$. Since $x \in J$ is arbitrary, we have that for each $x \in J$, $x$ is not a one of $L$. 
				\item Suppose that $J$ is principal. 
				\begin{itemize}
					\item $(\implies)$: \\
					If $J$ is proper, then $1(a)$ implies that for each $x \in J$, $x$ is not a one of $L$.
					\item $(\impliedby)$: \\
					Suppose that $J$ is not proper. Then $J = L$. Since $J$ is principal, there exists $a \in J$ such that $J = \doar a$. Let $x \in L$. Since
					\begin{align*}
						x
						& \in L \\
						& = J \\
						& = \doar a,
					\end{align*} 
					we have that $x \leq a$. Hence 
					\begin{align*}
						x \wedge a
						& = \inf \{x, a\} \\
						& = x.
					\end{align*}
					Since $x \in L$ is arbitrary, we have that for each $x \in L$, $x \wedge a = x$. Hence $a$ is a one of $L$. Therefore $J$ is not proper implies that there exists $a \in J$ such that $a$ is a one of $L$. By contrapositive, we have that if for each $x \in J$, $x$ is not a one of $L$, then $J$ is proper. 
				\end{itemize}
			\end{enumerate}
			\item \tcr{FINISH!!}
		\end{enumerate}
	\end{proof}
	
	\begin{ex}
		Let $L$ be a lattice and $J \subset L$.
		\begin{enumerate}
			\item Suppose that $J \in \MI(L)$ and $J$ is principal. Then $J$ is prime iff there exists $x \in J$ such that $x$ is meet-prime and $J = \doar x$.
			\item Suppose that $J \in \MF(L)$ and $J$ is principal. Then $J$ is prime iff there exists $x \in J$ such that $x$ is join-prime and $J = \upar x$.
		\end{enumerate}
	\end{ex}
	
	\begin{proof}\
		\begin{enumerate}
			\item 
			\begin{itemize}
				\item $(\implies)$: \\
				Suppose that $J$ is prime. Since $J$ is principal, there exists $x \in L$ such that $J = \doar x$. Since $J$ is prime, $J$ is proper and for each $a,b \in L$, $a \wedge b \in J$ implies that $a \in J$ or $b \in J$. 
				\begin{itemize}
					\item Since $J$ is proper, \tcr{previous exercise} implies that $x$ is not a one of $L$. 
					\item Let $a,b \in L$. Suppose that $a \wedge b \leq x$. Since $J$ is a lower set, $x \in J$ and $a \wedge b \leq x$, we have that $a \wedge b \in J$. Therefore $a \in J$ or $b \in J$. Since $J = \doar x$, we have that $a \leq x$ or $b \leq x$. Since $a,b \in L$ with $a \wedge b \leq x$ are arbitrary, we have that for each $a, b \in L$, $a \wedge b \leq x$ implies that $a \leq x$ or $b \leq x$. 
				\end{itemize}
				Thus $x$ is meet-prime. 
				\item $(\impliedby)$: \\
				Suppose that there exists $x \in J$ such that $x$ is meet-prime and $J = \doar x$. Then $J$ is principal.
				\begin{itemize}
					\item For the sake of contradiction, suppose that $J$ is not proper. Then 
					\begin{align*}
						\doar x
						& = J \\
						& = L.
					\end{align*}
					Thus $\ubd L = \{x\}$. \tcr{A previous ex} implies that $x$ is a one of $L$. This is a contracition since $x$ is meet-prime. Thus $J$ is proper. 
					\item Let $a,b \in L$. Suppose that $a \wedge b \in J$. Since $J = \doar x$, we have that $a \wedge b \leq x$. Since $x$ is meet-prime, $a \leq x$ or $b \leq x$. Hence $a \in J$ or $b \in J$. Since $a,b \in L$ with $a \wedge b \in J$ are arbitrary, we have that for each $a,b \in L$, $a \wedge b \in J$ implies that $a \in J$ or $b \in J$.
				\end{itemize}
				Therefore $J$ is prime.
			\end{itemize}
			\item Similar to $(1)$ \tcr{FIX/CHECK/FINISH!!!}
		\end{enumerate}
	\end{proof}
	
	
	
	
	
	
	
	
	
	
	
	
	
	
	
	
	
	
	
	
	
	
	
	
	
	
	
	
	
	
	
	
	
	
	
	
	
	
	
	
	
	
	
	
	
	
	
	
	
	
	
	
	
	
	\subsection{Completely Prime Ideals and Filters}
	
	\begin{defn}
		Let $L$ be a complete lattice. 
		\begin{itemize}
			\item Let $J \in \MI(L)$. Then $J$ is said to be
			\begin{itemize}
				\item \tbf{completely prime} if
				\begin{enumerate}
					\item $J$ is proper
					\item for each $(a_{\al})_{\al \in A} \subset L$, $\bigwedge\limits_{\al \in A} a_{\al} \in J$ implies that there exists $\al \in A$ such that $a_{\al} \in J$.
				\end{enumerate}
			\end{itemize} 
			\item Let $J \in \MF(L)$. Then $J$ is said to be 
			\begin{itemize}
				\item \tbf{completely prime} if 
				\begin{enumerate}
					\item $J$ is proper
					\item for each $(a_{\al})_{\al \in A} \subset L$, $\bigvee\limits_{\al \in A} a_{\al} \in J$ implies that there exists $\al \in A$ such that $a_{\al} \in J$.
				\end{enumerate}
			\end{itemize} 
		\end{itemize}
	\end{defn}
	
	
	
	
	
	
	
	
	
	
	
	
	
	
	
	
	
	
	
	
	
	
	
	
	
	
	
	
	
	
	
	
	
	
	
	
	
	
	
	
	
	
	
	
	
	
	
	
	
	
	
	
	
	
	
	
	
	
	
	
	
	\newpage
	\section{Complete Lattices}
	
	\subsection{Introduction}
	
	\begin{defn} 
		Let $(X, \leq)$ be a poset. Then $(X, \leq)$ is said to satisfy the 
		\begin{itemize}
			\item \tbf{least upper bound (LUB) property} if for each $A \subset X$, if $A \neq \varnothing$ and $A$ is bounded above, then there exists $x \in X$ such that $x = \sup A$
			\item \tbf{greatest lower bound (GLB) property} if for each $A \subset X$, if $A \neq \varnothing$ and $A$ is bounded below, then there exists $x \in X$ such that $x = \inf A$. 
		\end{itemize}
	\end{defn}
	
	\begin{ex}
		LUB iff GLB
	\end{ex}
	
	\begin{proof}
		\tcr{FINISH!!!!}
	\end{proof}
	
	\begin{defn} \tbf{Suplattice:} \\
		Let $(L, \leq)$ be a poset. Then $(L, \leq)$ is said to be a \tbf{suplattice} if for each $A \subset L$, there exists $x \in L$ such that $x = \sup A$. 
	\end{defn}
	
	
	
	
	
	
	
	
	
	
	
	
	
	
	
	
	
	
	
	
	
	\subsection{Complete Lattice Homomorphisms}
	
	
	
	
	
	
	
	
	
	
	
	
	\subsection{Galois Connections}
	
	
	
	
	
	
	
	
	
	
	
	
	
	
	
	
	
	
	
	
	
	
	
	
	
	
	
	
	
	
	
	
	
	
	
	
	
	
	
	
	
	
	
	
	
	
\section{Modular and Distributive Lattices}
	
	\subsection{Introduction}

	\begin{defn}
		Let $L$ be a lattice. Then $L$ is said to be a \tbf{distributive lattice} if for each $a,b,c \in L$, $a \vee (b \wedge c) = (a \vee b) \wedge (a \vee c)$.
	\end{defn}
	
	\begin{ex}
		Let $L$ be a lattice. Then for each $a,b,c \in L$,
		\begin{enumerate}
			\item $(a \wedge b) \vee (a \wedge c) \leq a \wedge (b \vee c)$.
			\item $a \wedge (b \vee c) \leq (a \vee b) \wedge (a \vee c)$
		\end{enumerate}
	\end{ex}
	
	\begin{proof} Let $a,b,c \in L$.
		\begin{enumerate}
			\item 
			\begin{itemize}
				\item Since $a \wedge b \leq a$ and $a \wedge c \leq a$, we have that $a \in \ubd \{a \wedge b, a \wedge c \}$. Therefore 
				\begin{align*}
					(a \wedge b) \vee (a \wedge c) 
					& = \sup \{a \wedge b, a \wedge c \} \\
					& \leq a.
				\end{align*}
				\item Since  
				\begin{align*}
					a \wedge b 
					& \leq b \\
					& \leq b \vee c
				\end{align*}
				and 
				\begin{align*}
					a \wedge c 
					& \leq c \\
					& \leq b \vee c,
				\end{align*}
				we have that $b \vee c \in \ubd \{a \wedge b, a \wedge c\}$. Therefore 
				\begin{align*}
					(a \wedge b) \vee (a \wedge c) 
					& = \sup \{a \wedge b, a \wedge c \} \\
					& \leq b \vee c. 
				\end{align*}
			\end{itemize}
			Then $(a \wedge b) \vee (a \wedge c) \in \lbd\{a, b \vee c\}$. Hence 
			\begin{align*}
				(a \wedge b) \vee (a \wedge c) 
				& \leq \inf \{a, b \vee c\} \\
				& = a \wedge (b \vee c).
			\end{align*}
			\item \tcr{FINISH!!! (use duality)}
		\end{enumerate}
	\end{proof}
	
	\begin{ex}
		Let $L$ be a lattice. Then for each $a,b,c \in L$,
		\begin{enumerate}
			\item $c \leq a$ implies that $(a \wedge b) \vee c \leq a \wedge (b \vee c)$
			\item $a \leq c$ implies that $a \vee (b \wedge c) \leq (a \vee b) \wedge c$
		\end{enumerate}
	\end{ex}
	
	\begin{proof}
		Let $a,b,c \in L$. 
		\begin{enumerate}
			\item Suppose that $c \leq a$. Then $a \wedge c = c$. \tcr{The previous exercise then implies that}
			\begin{align*}
				(a \wedge b) \vee c
				& = (a \wedge b) \vee (a \wedge c) \\
				& \leq a \wedge (b \vee c).
			\end{align*}
			\item \tcr{FINISH!!! use duality}
		\end{enumerate}
	\end{proof}
	
	\begin{ex}
		Let $L$ be a lattice. Then for each $a,b,c \in L$, 
		$$(a \wedge b) \vee (b \wedge c) \vee (c \wedge a) \leq (a \vee b) \wedge (b \vee c) \wedge (c \vee a).$$
	\end{ex}
	
	\begin{proof}
		Let $a,b,c \in L$.
		\begin{itemize}
			\item We first note that
			\begin{align*}
				a \wedge b \leq a \leq a \vee b, \\
				b \wedge c \leq b \leq a \vee b, \\
				c \wedge a \leq a \leq a \vee b.
			\end{align*}
			Therefore $a \vee b \in \ubd \{a \wedge b, b \wedge c, c \wedge a\}$ and
			\begin{align*}
				(a \wedge b) \vee (b \wedge c) \vee (c \wedge a)
				& = \sup \{a \wedge b, b \wedge c, c \wedge a\} \\
				& \leq a \vee b.
			\end{align*}
			\item Similarly, 
			\begin{align*}
				a \wedge b \leq b \leq b \vee c, \\
				b \wedge c \leq c \leq b \vee c, \\
				c \wedge a \leq c \leq b \vee c.
			\end{align*}
			Therefore $b \vee c \in \ubd \{a \wedge b, b \wedge c, c \wedge a\}$ and
			\begin{align*}
				(a \wedge b) \vee (b \wedge c) \vee (c \wedge a)
				& = \sup \{a \wedge b, b \wedge c, c \wedge a\} \\
				& \leq b \vee c.
			\end{align*}
			\item Finally, we have that
			\begin{align*}
				a \wedge b \leq a \leq c \vee a, \\
				b \wedge c \leq c \leq c \vee a, \\
				c \wedge a \leq c \leq c \vee a.
			\end{align*}
			Therefore $c \vee a \in \ubd \{a \wedge b, b \wedge c, c \wedge a\}$ and
			\begin{align*}
				(a \wedge b) \vee (b \wedge c) \vee (c \wedge a)
				& = \sup \{a \wedge b, b \wedge c, c \wedge a\} \\
				& \leq c \vee a.
			\end{align*}
		\end{itemize}
		Hence $(a \wedge b) \vee (b \wedge c) \vee (c \wedge a) \in \lbd \{a \vee b, b \vee c, c \vee a\}$ and 
		\begin{align*}
			(a \wedge b) \vee (b \wedge c) \vee (c \wedge a)
			& \leq \inf \{a \vee b, b \vee c, c \vee a\} \\
			& = (a \vee b) \wedge (b \vee c) \wedge (c \vee a).
		\end{align*}
	\end{proof}
	
	\begin{ex}
		Let $L$ be a lattice. Then the following are equivalent:
		\begin{enumerate}
			\item For each $a,b,c \in L$, $c \leq a$ implies that $a \wedge (b \vee c) = (a \wedge b) \vee c$.
			\item For each $a,b,c \in L$, $c \leq a$ implies that $a \wedge (b \vee c) = (a \wedge b) \vee (a \wedge c)$.
			\item For each $p,q,r \in L$, $p \wedge (q \vee (p \wedge r)) = (p \wedge q) \vee (p \wedge r)$.
		\end{enumerate}
	\end{ex}
	
	\begin{proof}\
		\begin{enumerate}
			\item $(1) \implies (2)$: \\
			Suppose that for each $a,b,c \in L$, $c \leq a$ implies that $a \wedge (b \vee c) = (a \wedge b) \vee c$. Let $a,b,c \in L$. Suppose that $c \leq a$. Then $a \wedge c = c$. By assumption 
			\begin{align*}
				a \wedge (b \vee c) 
				& = (a \wedge b) \vee c \\
				& = (a \wedge b) \vee (a \wedge c)
			\end{align*}
			Since $a,b,c \in L$ with $c \leq a$ are arbitrary, we have that for each $a,b,c \in L$, $c \leq a$ implies that $a \wedge (b \vee c) = (a \wedge b) \vee (a \wedge c)$.
			\item $(2) \implies (3)$: \\
			Suppose that for each $a,b,c \in L$, $c \leq a$ implies that $a \wedge (b \vee c) = (a \wedge b) \vee (a \wedge c)$. Let $p,q,r \in L$. Define $a,b,c \in L$ by $a \defeq p$, $b \defeq q$ and $c \defeq p \wedge r$. Then 
			\begin{align*}
				a \wedge c
				& = p \wedge (p \wedge r) \\
				& = (p \wedge p) \wedge r \\
				& = p \wedge r.
			\end{align*}
			By assumption,
			\begin{align*}
				p \wedge (q \vee (p \wedge r))
				& = a \wedge (b \vee c) \\
				& = (a \wedge b) \vee (a \wedge c) \\
				& = (p \wedge q) \vee (p \wedge r).
			\end{align*}
			Since $p,q,r \in L$ are arbitrary, we have that for each $p,q,r \in L$, $p \wedge (q \vee (p \wedge r)) = (p \wedge q) \vee (p \wedge r)$.
			\item $(3) \implies (1)$: \\
			Suppose that for each $p,q,r \in L$, $p \wedge (q \vee (p \wedge r)) = (p \wedge q) \vee (p \wedge r)$. Let $a,b,c \in L$. Suppose that $c \leq a$. Define $p,q,r \in L$ by $p \defeq a$, $q \defeq b$ and $r \defeq c$. Since $c \leq a$, we have that 
			\begin{align*}
				p \wedge r
				& = a \wedge c \\
				& = c.
			\end{align*}
			By assumption, 
			\begin{align*}
				a \wedge (b \vee c)
				& = p \wedge (q \vee (p \wedge r)) \\
				& = (p \wedge q) \vee (p \wedge r) \\
				& = (a \wedge b) \vee c.
			\end{align*}
			Since $a,b,c \in L$ with $c \leq a$ are arbitrary, we have that for each $a,b,c \in L$, $c \leq a$ implies that $a \wedge (b \vee c) = (a \wedge b) \vee c$.
		\end{enumerate}
	\end{proof}
	
	\begin{defn}
		Let $L$ be a lattice. Then $L$ is said to be \tbf{modular} if for each $a,b,c \in L$, 
		$$\text{$c \leq a$ implies that $a \wedge (b \vee c) = (a \wedge b) \vee c$.}$$
	\end{defn}
	
	\begin{ex}
		Let $L$ be a lattice. Then the following are equivalent:
		\begin{enumerate}
			\item For each $a,b,c \in L$, $a \wedge (b \vee c) = (a \wedge b) \vee (a \wedge c)$.
			\item For each $p,q,r \in L$, $p \vee (q \wedge r) = (p \vee q) \wedge (p \vee r)$.
		\end{enumerate}
	\end{ex}
	
	\begin{proof}\
		\begin{enumerate}
			\item $(1) \implies (2)$: \\
			Suppose that for each $a,b,c \in L$, $a \wedge (b \vee c) = (a \wedge b) \vee (a \wedge c)$. Let $p,q,r \in L$. Define $a,b,c \in L$ by $a \defeq p \vee q$, $b \defeq p$ and $c \defeq r$. By absorption, 
			\begin{align*}
				a \wedge b
				& = (p \vee q) \wedge p \\
				& = p \wedge (p \vee q) \\
				& = p.
			\end{align*}
			By assumption,
			\begin{align*}
				a \wedge c
				& = (p \vee q) \wedge r \\
				& = r \wedge (p \vee q) \\
				& = (r \wedge p) \vee (r \wedge q).
			\end{align*}
			Then by assumption and absorption,
			\begin{align*}
				(p \vee q) \wedge (p \vee r)
				& = a \wedge (b \vee c) \\
				& = (a \wedge b) \vee (a \wedge c) \\
				& = p \vee [(r \wedge p) \vee (r \wedge q)] \\
				& = [p \vee (r \wedge p)] \vee (r \wedge q) \\
				& = [p \vee (p \wedge r)] \vee (q \wedge r) \\
				& = p \vee (q \wedge r).
			\end{align*}
			Since $p,q,r \in L$ are arbitrary, we have that for each $p,q,r \in L$, $p \vee (q \wedge r) = (p \vee q) \wedge (p \vee r)$.
			\item $(2) \implies (1)$: \\
			Suppose that for each $p,q,r \in L$, $p \vee (q \wedge r) = (p \vee q) \wedge (p \vee r)$. Let $a,b,c \in L$. Define $p,q,r \in L$ by $p \defeq a \wedge b$, $q \defeq a$ and $r \defeq c$. 
			By absorption, 
			\begin{align*}
				p \vee q
				& = (a \wedge b) \vee a \\
				& = a \vee (a \wedge b) \\
				& = a.
			\end{align*}
			By assumption,
			\begin{align*}
				p \vee r
				& = (a \wedge b) \vee c \\
				& = c \vee (a \wedge b) \\
				& = (c \vee a) \wedge (c \vee b).
			\end{align*}
			Then by assumption and absorption,
			\begin{align*}
				(a \wedge b) \vee (a \wedge c)
				& = p \vee (q \wedge r) \\
				& = (p \vee q) \wedge (p \vee r) \\
				& = a \wedge [(c \vee a) \wedge (c \vee b)] \\
				& = [a \wedge (c \vee a)] \wedge (c \vee b) \\
				& = [a \wedge (a \vee c)] \wedge (b \vee c) \\
				& = a \wedge (b \vee c).
			\end{align*}
			Since $a,b,c \in L$ are arbitrary, we have that for each $a,b,c \in L$, $a \wedge (b \vee c) = (a \wedge b) \vee (a \wedge c)$.
		\end{enumerate}
	\end{proof}
	
	\begin{defn}
		Let $L$ be a lattice. Then $L$ is said to be \tbf{distributive} if for each $a,b,c \in L$, 
		$$a \wedge (b \vee c) = (a \wedge b) \vee (a \wedge c).$$
	\end{defn}
	
	\begin{ex}
		Let $L$ be a lattice. If $L$ is distributive, then $L$ is modular. 
	\end{ex}
	
	\begin{proof}
		Suppose that $L$ is distributive. \tcr{FINISH!!!}
	\end{proof}
	
	\begin{ex}
		Let $L$ be a lattice. Then for each $a,b,c \in L$, $a \vee (b \wedge c) = (a \vee b) \wedge (a \vee c)$ iff $a \wedge (b \vee c) = (a \wedge b) \vee (a \vee c)$.
	\end{ex}
	
	\begin{proof}
		Let $a,b,c \in L$.
		\begin{itemize}
			\item $(\implies)$: \\
			Suppose that $a \vee (b \wedge c) = (a \vee b) \wedge (a \vee c)$. 
			\item $(\impliedby)$: \\
			
		\end{itemize}
	\end{proof}
	
	
	
	
	
	
	
	
	
	
	
	
	
	
	
	
	
	
	
	
	
	
	
	
	
	
	
	
	
	
	
	
	
	
	
	
	
	
	
	
	
	
	
	
	
	
	
	
	
	
	
	
	
	
	\section{Galois Connections}
	
	\begin{defn}
		Let 
	\end{defn}
	
	
	
	
	
	
	
	
	
	
	
	
	
	
	
	
	
	
	
	
	
	
	
	
	
	
	
	
	
	
	
	
	
	
	
	
	
	
	
	
	
	
	
	
	
	
	
	
	\chapter{Frames and Locales}
	
	\section{Introduction}
	
	
	\subsection{Frames}
	
	\begin{defn}
		Let $L$ be a complete lattice. Then $L$ is said to be a \tbf{frame} if for each $a \in L$ and $(b_{\al})_{\al \in A} \subset L$, $a \wedge \bigg(\bigvee\limits_{\al \in A} b_{\al} \bigg) = \bigvee\limits_{\al \in A} a \wedge b_{\al}$.
	\end{defn}
	
	\begin{defn}
		Let $L, M$ be frames and $f:L \rightarrow M$. Then $f$ is said to be a \tbf{frame homomorphism} if 
		\begin{enumerate}
			\item for each $(x_{\al})_{\al \in A} \subset L$, $f\bigg( \bigvee\limits_{\al \in A} x_{\al} \bigg) = \sup\limits_{\al \in A} f(x_{\al})$.
			\item for each $a,b \in L$, $f(a \wedge b) = f(a) \wedge f(b)$
		\end{enumerate}
		\tcr{maybe reword this with some vocab to make shorter, like "preserves arbitrary joins" and "preserves meets"}
	\end{defn}
	
	\begin{defn}\tcr{(check notation consistent with category theory notes)}
		We define the \tbf{category of frames}, denoted $\Frm$, by 
		\begin{itemize}
			\item $\Obj(\Frm) \defeq \{L: \text{$L$ is a frame}\}$
			\item $\Hom_{\Frm}(L, M) \defeq \{f:L \rightarrow M: \text{$f$ is a frame homomorphism}\}$
		\end{itemize}
	\end{defn}
	
	\begin{ex}
		We have that $\Frm$ is a category
	\end{ex}
	
	\begin{proof}
		\tcr{FINISH!!!}
	\end{proof}
	
	
	
	
	
	
	
	
	
	
	
	
	
	
	
	
	
	
	
	
	
	
	
	
	
	
	
	
	
	
	
	
	
	
	
	
	
	
	
	
	
	
	
	
	
	
	
	
	
	
	
	
	
	
	
	
	
	\subsection{Locales}
	
	\begin{defn}
		We define the \tbf{category of locales}, denoted $\Loc$ by $\Loc \defeq \op{\Frm}$.
	\end{defn}
	
	
	
	
	
	
	
	
	
	
	
	
	
	
	
	
	
	
	
	
	
	
	
	
	
	
	
	
	
	
	
	
	
	
	
	
	
	
	
	
	
	
	
	
	
	
	
	
	
	
	
	
	
	
	
	
	
	
	
	
	\newpage
	\section{More Lattice Stuff to Come}
	\begin{itemize}
		\item talk about join and meet irriducibility
		\item talk about join and meet primality
		\item talk about maximality.
		\item \tcr{the goal is to get all the background for sober topological/measure spaces, locale theory for constructive topology and universal algebra}
	\end{itemize}
	
	
	
	
	
	
	
	
	
	
	
	
	
	
	
	
	
	
	
	
	
	
	
	
	
	
	
	
	
	
	
	
	
	
	
	
	
	
	
	
	\newpage
	\chapter{Model Theory}
	
	\section{Introduction}
	
	
	
	
	
	
	
	
	
	
	
	
	
	
	
	
	
	
	
	
	
	
	
	
	
	
	
	
	
	
	
	
	
	
	
	
	
	
	
	
	
	
	
	
	\newpage
	\chapter{Some Chapter}
	
	\section{Closure Operators}
	
	\begin{defn}
		Let $A$ be a set and $C: \MP(A) \rightarrow \MP(A)$. Then $C$ is said to be a \tbf{closure operator on $A$} if for each $X, Y \in \MP(A)$,
		\begin{enumerate}
			\item $X \subset C(X)$,
			\item $C^2(X) = C(X)$,
			\item $X \subset Y$ implies that $C(X) \subset C(Y)$.
		\end{enumerate}
	\end{defn}

	\begin{ex}
		Let $A$ be a set and $C: \MP(A) \rightarrow \MP(A)$. Suppose that $C$ is a closure operator on $A$. Then for each $(E_j)_{j \in J} \subset \MP(A)$,
		\begin{enumerate}
			\item $C \bigg( \bigcap\limits_{j \in J} E_j \bigg) \subset \bigcap\limits_{k \in J} C(E_k)$,
			\item $\bigcup\limits_{k \in J} C(E_k) \subset C \bigg( \bigcup\limits_{j \in J} E_j \bigg)$.
		\end{enumerate}
	\end{ex}

	\begin{proof}
		Let $(E_j)_{j \in J} \subset \MP(A)$. 
		\begin{enumerate}
			\item Let $k \in J$. Then $\bigcap\limits_{j \in J} E_j \subset E_k$.
			So $C \bigg( \bigcap\limits_{j \in J} E_j \bigg) \subset C(E_k)$. Since $k \in J$ is arbitrary, we have that 
			$$C \bigg( \bigcap\limits_{j \in J} E_j \bigg) \subset \bigcap\limits_{k \in J} C(E_k).$$ 
			\item Let $k \in J$. Then $E_k \subset \bigcup_{j \in J} E_j$. Hence $C(E_k) \subset C \bigg( \bigcup\limits_{j \in J} E_j \bigg)$. Since $k \in J$ is arbitrary, we have that 
			\begin{align*}
				\bigcup\limits_{k \in J} C(E_k) 
				& \subset C \bigg( \bigcup\limits_{j \in J} E_j \bigg)
			\end{align*}
			
		\end{enumerate}
	\end{proof}

	\begin{defn}
		Let $A$ be a set, $C: \MP(A) \rightarrow \MP(A)$ and $X \subset A$. Suppose that $C$ is a closure operator on $A$. Then $X$ is said to be $C$-closed if $C(X) = X$.  
	\end{defn}

	\begin{defn}
		Let $A$ be a set and $C: \MP(A) \rightarrow \MP(A)$. Suppose that $C$ is a closure operator on $A$. We define the \tbf{lattice of $C$-closed subsets of $A$}, denoted $L_C(A) \subset \MP(A)$, by 
		$$L_C(A) \defeq \{X \subset A: \text{$X$ is $C$-closed}\}$$. 
	\end{defn}


	\begin{ex}
		Let $A$ be a set and $C: \MP(A) \rightarrow \MP(A)$. Suppose that $C$ is a closure operator on $A$. Then
		\begin{enumerate}
			\item for each $(E_j)_{j \in J} \subset L_C(A)$, $\bigcap\limits_{j \in J} E_j \in L_C(A)$ and $\bigcup\limits_{j \in J} E_j \in L_C(A)$.
			\item $(L_C(A), \subset)$ is a complete lattice
			\tcr{define complete lattice}
			$$ C \bigg( \bigcap\limits_{j \in J} E_j \bigg) = \bigcap\limits_{j \in J} E_j $$ 
			and  
			$$ C \bigg(  \bigcup\limits_{j \in J} E_j \bigg) = \bigcup\limits_{j \in J} E_j.$$
		\end{enumerate}
	\end{ex}

	\begin{proof}\
		\begin{enumerate}
			\item Let $(E_j)_{j \in J} \subset L_C(A)$. 
			\begin{itemize}
				\item \tcr{A previous exercise} \rex{} implies that
				\begin{align*}
					C \bigg( \bigcap\limits_{j \in J} E_j \bigg) 
					& \subset \bigcap\limits_{k \in J} C(E_k) \\
					& = \bigcap\limits_{k \in J} E_k \\
					& \subset C \bigg( \bigcap\limits_{k \in J} E_k \bigg). 
				\end{align*}
				Hence $C \bigg( \bigcap\limits_{j \in J} E_j \bigg) = \bigcap\limits_{k \in J} E_k$.
				\item \tcr{A previous exercise} \rex{} implies that
				\begin{align*}
					\bigcup\limits_{k \in J} E_k 
					& = \bigcup\limits_{k \in J} C(E_k) \\
					& \subset C \bigg( \bigcup\limits_{j \in J} E_j \bigg) \\
					& \subset \bigcap\limits_{k \in J} C(E_k) \\
					& = \bigcap\limits_{k \in J} E_k \\
					& \subset C \bigg( \bigcap\limits_{k \in J} E_k \bigg). 
				\end{align*}
				Hence $C \bigg( \bigcap\limits_{j \in J} E_j \bigg) = \bigcap\limits_{k \in J} E_k$.
			\end{itemize}
			\item 
		\end{enumerate}
		\tcr{FINISH!!!, don't need to show second part, }
	\end{proof}

	\begin{defn}
		then is said to be an \tbf{algebraic closure operator on $A$} if 
	\end{defn}

	
	
	
	
	
	
	
	
	
	
	
	
	
	
	
	
	
	
	
	
	
	
	
	
	
	
	
	
	
	
	
	
	
	
	
	
	
	
	
	
	
	\newpage
	\chapter{Universal Algebra}
	
	\section{Introduction}
	
	\begin{defn}
		Let $A,J \in \Obj(\Set)$ be a set and $f \in \MF^*(A)^J$. Then $(A, f)$ is said to be an \tbf{algebra} if $A \neq \varnothing$ and $J \neq \varnothing$. 
	\end{defn}
	
	\begin{defn}
		Let $(A, f)$ be an algebra. Set $J \defeq \dom f$. 
		\begin{itemize}
			\item We define the \tbf{universe of $(A, f)$}, denoted $\uni(A, f)$, by $\uni(A, f) \defeq A$.
			\item We define the \tbf{operations of $(A, f)$}, denoted $\oper(A, f)$, by $\oper(A, f) \defeq f$.
			\item We define the \tbf{type of $(A, f)$}, denoted $\typ(A, f): J \rightarrow \N_0$, by $\typ(A, f)(j) \defeq \ar f_j$.
		\end{itemize}
	\end{defn}

	\begin{defn}
		Let $\MA$, $\MB$ be algebras. Then $\MA$ and $\MB$ are said to be \tbf{type-similar}, denoted $\MA \sim_{\typ} \MB$, if $\typ \MA = \typ \MB$.
	\end{defn}
	
	\begin{defn}
		Let $(A, f), (B, g)$ be algebras and $\al:A \rightarrow B$. Suppose that $(A, f) \sim_{\typ} (B, g)$. Set $J \defeq \dom f$ and $\rho \defeq \typ(A, f)$. Write $f = (f_j)_{j \in J}$ and $g = (g_j)_{j \in J}$. Then $\al$ is said to be an \tbf{$((A,f), (B,g))$-homomorphism} if for each $j \in J$ and $a \in A^{\rho(j)}$, 
		$$g_j(\al^{\rho(j)}(a)) = \al(f_j(a)).$$
	\end{defn}
	
	\begin{ex}
		Let $(A, f), (B, g), (C, h)$ be algebras and $\al:A \rightarrow B$, $\be:B \rightarrow C$. Suppose that $(A, f) \sim_{\typ} (B, g), (C, h)$. Set $J \defeq \dom f$ and $\rho \defeq \typ(A, f)$. Write $f = (f_j)_{j \in J}$, $g = (g_j)_{j \in J}$ and $h = (h_j)_{j \in J}$. If $\al$ is a $((A,f), (B,g))$-homomorphism and $\be$ is a $((B,g), (C,h))$-homomorphism, then $\be \circ \al$ is a $((A,f), (C,h))$-homomorphism.
	\end{ex}
	
	\begin{proof}
		Suppose that $\al$ is a $((A,f), (B,g))$-homomorphism and $\be$ is a $((B,g), (C,h))$-homomorphism. Let $j \in J$ and $a \in A^{\rho(j)}$. Since $\al$ is a $((A,f), (B,g))$-homomorphism, $g_j(\al^{\rho(j)}(a)) = \al(f_j(a))$. Define $b \in B^{\rho(j)}$ \tcr{(special case $\rho(j) = 0$?)} by $b \defeq \al^{\rho(j)}(a)$. Since $\be$ is a $((B,g), (C,h))$-homomorphism, $h_j(\be^{\rho(j)}(b)) = \be(g_j(b))$. Therefore
		\begin{align*}
			h_j([\be \circ \al]^{\rho(j)}(a))  
			 & = h_j(\be^{\rho(j)} \circ \al^{\rho(j)}(a)) \\
			 & = h_j(\be^{\rho(j)}(\al^{\rho(j)}(a))) \\
			 & = h_j(\be^{\rho(j)}(b)) \\
			 & = \be(g_j(b)) \\
			 & = \be(g_j(\al^{\rho(j)}(a))) \\
			 & = \be(\al(f_j(a))) \\
			 & = \be \circ \al (f_j(a)).
		\end{align*} 
		Since $j \in J$ and $a \in A^{\rho(j)}$ are arbitrary, we have that for each $j \in J$ and $a \in A^{\rho(j)}$, $h_j([\be \circ \al]^{\rho(j)}(a)) = \be \circ \al (f_j(a))$. Hence $\be \circ \al$ is a $((A,f), (C,h))$-homomorphism.
	\end{proof}
	
	
	\begin{defn}
		Define category of algebras $\Alg(\rho)$ of a given type $\rho$.
		
		\tcr{FINISH!!!!}
	\end{defn}
	
























































	\newpage
	\section{Subalgebras}
	
	\begin{defn}
		Let $(A, f), (B, g) \in \Obj(\Alg)$. Suppose that $(A, f)$ and $(B, g)$ are type-similar. Set $J \defeq \dom f$ and $\rho \defeq \typ (A, f)$. Write $f = (f_j)_{j \in J}$ and $g = (g_j)_{j \in J}$. 
		\begin{itemize}
			\item Then $(B, g)$ is said to be a \tbf{subalgebra of $(A, f)$} if 
			\begin{enumerate}
				\item $B \subset A$ 
				\item for each $j \in J$, $f_j|_{B^{\rho(j)}} = g_j$.
			\end{enumerate}
			\item We define $\suba(\MA) \defeq \{\MB \in \Obj(\Alg): \text{$\MA \sim_{\typ} \MB$ and $\MB$ is a subalgebra of $\MA$}\}$.
		\end{itemize}
	\end{defn}
	
	\begin{defn}
		Let $(A, f) \in \Obj(\Alg)$ and $B \in \Obj(\Set)$. 
		\begin{itemize}
			\item Then $B$ is said to be a \tbf{subuniverse of $(A, f)$} if 
			\begin{enumerate}
				\item $B \subset A$,
				\item $B$ is $\Im f$-closed.
			\end{enumerate}
			\item We define $\subu(\MA) \defeq \{B \in \Obj(\Set): \text{$B$ is a subuniverse of $\MA$}\}$.
		\end{itemize}
	\end{defn}
	
	\begin{ex}
		Let $(A, f)$, $(B, g)$ be algebras. Suppose that $(A, f)$ and $(B, g)$ are type-similar. If $(B, g)$ is a subalgebra of $(A, f)$, then $B$ is a subuniverse of $A$. 
	\end{ex}
	
	\begin{proof}
		Set $J \defeq \dom f$ and $\rho \defeq \typ(A, f)$. Suppose that $(B, g)$ is a subalgebra of $(A, f)$. 
		\begin{enumerate}
			\item Since $(B, g)$ is a subalgebra of $(A, f)$, we have that $B \subset A$.
			\item Let $j \in J$. Then for each $a \in B^\rho(j)$, 
			\begin{align*}
				f_j(a_1, \ldots, a_{\rho(j)})
				& = f_j|_{B^{\rho(j)}}(a_1, \ldots, a_{\rho(j)}) \\
				& = g_j(a_1, \ldots, a_{\rho(j)}) \\
				& \in B.
			\end{align*}
			Since $j \in J$ is arbitrary, we have that $B$ is $\Im f$-closed. Thus $B$ is a subuniverse of $A$.
		\end{enumerate}
	\end{proof}

	\begin{defn}
		Let $\MA$ be an algebra and $B$ a subuniverse of $\MA$. Set $\MS(B, \MA) \defeq \{S \subset A: \text{$S$ is a subuniverse of $\MA$ and $B \subset S$}\}$. We define the \tbf{subuniverse of $\MA$ generated by $B$}, denoted $\Sg(B, \MA)$, by
		$$\Sg(B, \MA) \defeq \bigcap\limits_{S \in \MS(B, \MA)} S $$
	\end{defn}
	\tcr{show $\MS \neq \varnothing$ and intersection of subiniverses is subuniverse}

	\begin{ex}
		Let $(A, f)$ be an algebra and $B \subset A$. Then 
		\begin{enumerate}
			\item $\Sg(B, f)$ is a subuniverse of $A$
			\item $B \subset \Sg(B, f)$.
		\end{enumerate}.
	\end{ex}

	\begin{proof}\
		\begin{enumerate}
			\item Set $\MS \defeq \{S \subset A: \text{$S$ is an $f$-subuniverse of $A$}\}$. By construction, for each $S \in S$, $S$ is $f$-closed. Since $\Sg(B, f) = \bigcap\limits_{S \in \MS} S$, \rex{} \tcr{A previous exercise in the set theory section} implies that $\Sg(B, f)$ is $f$-closed. Hence $\Sg(B, f)$ is an $f$-subuniverse of $A$. 
			\item By construction, for each $S \in S$, $B \subset S$. Thus
			\begin{align*}
				B
				& \subset \bigcap\limits_{S \in \MS} S \\
				& = \Sg(B, f).
			\end{align*}
		\end{enumerate}
	\end{proof}

	\begin{ex}
		Let $(A, f)$ be an algebra. Then $\Sg(\cdot, f)$ is an algebraic closure operator on $A$.
	\end{ex}

	\begin{proof}
		
	\end{proof}

	

	

	



	
	
	
	
	
	
	
	
	
	
	
	
	
	
	
	
	
	
	
	
	
	
	
	
	
	
	
	
	
	
	
	
	
	
	
	
	
	
	
	
	
	
	
	
	
	
	
	
	
	
	
	
	
	
	
	
	
	
	
	
	
	
	
	
	
	
	
	
	
	
	
	
	
	
	
	
	
	
	
	
	
	
	
	
	
	
	
	
	
	
	
	
	
	
	
	
	
	
	
	
	
	
	
	
	
	
	
	
	
	
	
	\newpage

	
	
	\chapter{Groups}
	
	\subsection{Direct Products}
	
	\begin{defn}
	Let $G,H$ be groups. Define a product $*:(G \times H) \times (G \times H) \rightarrow G \times H$ by 
	$$(x_1,y_1) * (x_2, y_2) = (x_1x_2, y_1y_2)$$
	Then $(G \times H, *)$ is called the \textbf{direct product of $G$ and $H$}.
	\end{defn}	
	
	\begin{ex}
	\lex{1} Let $G,H$ be groups. Then the direct product $G \times H$ is a group.
	\end{ex}
	\begin{proof}
	Clear.
	\end{proof}
	
	\begin{defn} 
	Let $G,H$ be groups. Define $\pi_G :G \times H \rightarrow G$ and $\pi_H :G \times H \rightarrow H$ by $\pi_G(x,y) = x$ and $\pi_H(x,y) = y$.  Then $\pi_G$ and $\pi_H$ are respectively called the \textbf{projection maps onto $G$ and $H$}.
	\end{defn}	
	
	\begin{ex}
	\lex{2} Let $G,H$ be groups. Then 
	\begin{enumerate}
	\item $\pi_G: G \times H \rightarrow G$ and $\pi_H : G \times H \rightarrow H$ are homomorphisms
	\item $\ker \pi_G \cong H$ and $\ker \pi_H \cong G$
\end{enumerate}	 
	\end{ex}
	
	\begin{proof}\
	\begin{enumerate}
	\item Clear
	\item Define $\iota_G:G \rightarrow \ker \pi_H$ by $$\iota_G(x) = (x, e_H)$$ Then $\iota_G$ is an isomorphism. Similarly, we can define $\iota_H:H \rightarrow \ker \pi_G$ and show that it is an isomorphism.
	\end{enumerate}
	\end{proof}
	
	\begin{defn}
	Let $G,H, K$ be groups, $\phi \in \Hom(G,K)$ and $\psi \in \Hom(H, K)$. We define $\phi \times \psi: G \times H \rightarrow K$ by $\phi \times \psi(x,y) = \phi(x) \psi(y)$ 
	\end{defn}	
	
	\begin{ex}
	\lex{3} Let $G,H, K$ be groups, $\phi \in \Hom(G,K)$ and $\psi \in \Hom(H, K)$. If $K$ is abelian, then $\phi \times \psi \in Hom(G \times H,K)$.
	\end{ex}
	
	\begin{proof}
	Let $x_1, x_2 \in G$ and $y_1, y_2 \in H$. Then 
	\begin{align*}
	\phi \times \psi[(x_1, y_1)(x_2, y_2)] 
	&= \phi \times \psi (x_1x_2, y_1y_2) \\
	&= \phi(x_1x_2) \psi(y_1y_2) \\
	&= \phi(x_1)\phi(x_2)\psi(y_1)\psi(y_2) \\
	&= \phi(x_1)\psi(y_1)\phi(x_2)\psi(y_2) \\
	&= [\phi \times \psi(x_1, y_1)] [\phi \times \psi(x_2, y_2) ]
	\end{align*}
	\end{proof}
	
	\begin{ex}
	\lex{4} Let $G,H, K$ be groups and $\phi \in \Hom(G \times H, K)$. Then there exist $\phi_G \in \Hom(G,K)$, $\phi_H \in \Hom(H, K)$ such that $\phi_G \times \phi_H = \phi$.
	\end{ex}
	
	\begin{proof}
	Suppose that $K$ is abelian. Define $\iota_G \in \Hom(G, \ker \pi_H)$ and $\iota_H \in \Hom(H, \ker \pi_G)$ as in part $(2)$ of \rex{2} Define $\phi_G \in \Hom(G, K)$ and $\phi_H \in \Hom(H,K)$ by  $\phi_G = \phi \circ \iota_G$ and $\phi_H = \phi \circ \iota_H $. Let $(x,y) \in G \times H$. Then 
	\begin{align*}
	\phi_G \times \phi_H(x,y) 
	&= \phi_G(x) \phi_H(y) \\
	&= \phi \circ \iota_G(x) \phi \circ \iota_H(y) \\
	&= \phi(x, e_H)\phi(e_G, y) \\
	&= \phi(x,y) \\
	\end{align*}
	So $\phi = \phi_G \times \phi_H$
	\end{proof}
	
	
	
	
	
	
	
	
	
	
	
	
	
	
	
	
	\newpage
	\section{Rings}
	
	\begin{defn}
	Let $R$ be a set and $+, *: R \times R 				
	\rightarrow R$ (we write $a+b$ and 
	$ab$ in place of $+(a,b)$ and $*(a,b)$ respectively).
	Then $R$ is said to be a \textbf{ring} if for each 
	$a,b,c \in R$,
	\begin{enumerate}
	\item $R$ is an abelian group with respect to $+$.
	 The identity element with respect to $+$ is denoted
	 by $0$.
	\item $R$ is a monoid with respect to $*$. The  
	identity element of $R$ with respect to $*$ is denoted $1$. 
	\item $R$ is commutative with respect to $*$.
	\item $*$ distributes over $+$.
	\end{enumerate}
	\end{defn}
	
	\begin{defn}
	Let $R$ be a ring and $I \subset R$. Then $I$ is said 
	to be an \textbf{ideal} of $R$ if for each $a \in R$ and $x,y \in I$,
	\begin{enumerate}
	\item  $x + y \in I$
	\item  $ax \in I$
	\end{enumerate}
	\end{defn}
	
	\begin{defn}
	Let $R$ be a ring and $A,B \subset R$. We define the \textbf{product} of $A$ and $B$, denoted $AB$, to be $$AB = \bigg \{\sum_{i=1}^n a_ib_i: a_i \in A, b_i \in B, n \in \N \bigg \}$$
	\end{defn}	
	
	\begin{ex}
	Let $R$ be a ring and $I \subset R$. Then $I$ is an ideal of $R$ iff $RI \subset I$. 
	\end{ex}
	
	\begin{proof}
	Suppose that $RI \subset I$. Let $a \in R$ and $x,y \in I$. Then by assumption $x + y = 1x + 1y \in I$ and $ax \in I$. So $I$ is an ideal of $R$\\
	Conversely, suppose that $I$ is an ideal of $R$. Let $a_1, \cdots, a_n \in R$ and $x_1, \cdots, x_n \in I$. Then by assumption, for each $i = 1, \cdots, n$, $a_ix_i \in I$ and therefore $\sum\limits_{i=1}^n a_ib_i \in I$. Hence $RI \subset I$.
	\end{proof}
	
	
	
	
	
	
	
	
	
	
	
	
	
	
	
	
	
	
	
	
	
	
	
	\newpage	
	\section{Modules}
	
	\subsection{Introduction}
	
	\begin{defn}
	Let $R$ be a ring, $M$ a set, $+: M\times M \rightarrow M$ and $*: R 
	\times M \rightarrow M$ (we write $rx$ in place of 
	$*(r,x)$). Then $M$ is said to be an 
	\textbf{$R$-module}
	if 
	\begin{enumerate}
	\item $M$ is an abelian group with respect to $+$. The identity element of $M$ with respect to $+$ is denoted by 0.
	\item for each $r \in R$, $*(r, \cdot)$ is a group endomorphism of $M$
	\item for each $x \in M$, $*(\cdot, x)$ is a group homomorphism from $R$ to $M$
	\item $*$ is a monoid action of $R$ on $M$
	\end{enumerate}
	\end{defn}
	
	\begin{note}
	For the remainder of this section, we assume that $R$ is a commutative ring. 
	\end{note}
	
	\begin{ex}
	Let $M$ be an $R$-module. Then for each $r \in R$ and $x \in M$, 
	\begin{enumerate}
	\item $r0 = 0$
	\item $0x = 0$
	\item $(-1)x = -x$
	\end{enumerate}
	\end{ex}
	
	\begin{proof} Let $r \in R$ and $x \in M$. Then 
	\begin{enumerate}
	\item 
	\begin{align*}
	r0 
	&= r(0+0) \\
	&= r0 + r0
	\end{align*} 
	which implies that $r0 = 0$.
	\item 
	\begin{align*}
	0x 
	&= (0+0)x \\
	&= 0x + 0x
	\end{align*} 
	which implies that $0x = 0$.
	\item 
	\begin{align*}
	(-1)x + x 
	&= (-1)x + 1x \\ 
	&= (-1 + 1)x \\
	&= 0x \\
	&= 0
	\end{align*}
	which implies that $(-1)x = -x$.
	\end{enumerate}
	\end{proof}
	
	\begin{defn}
	Let $M$ an $R$-module and $N \subset M$. Then $N$ is said to be a \textbf{submodule} of $M$ if for each $r \in R$ and $x,y \in N$, we have that $rx \in N$ and $x+y \in N$.
	\end{defn}
	
	\begin{defn}
	Let $M$ be an $R$-module. We define $\MS(M) = \{N \subset M: N \text{ is a submodule of }M\}$.
	\end{defn}	
	
	\begin{ex}
	Let $M$ be an $R$-module and $N \in \MS(M)$. Then $N$ is a subgroup of $M$.
	\end{ex}
	
	\begin{proof}
	Let $x,y \in M$. Then $x-y = 1x + (-1)y \in N$. So $N$ is a subgroup of $M$.
	\end{proof}
	
	\begin{defn}
	Let $M$ be an $R$-module and $N \in \MS(M)$. We define  
	\begin{enumerate}
	\item $M/N = \{x + N: x \in M\}$ 
	\item $+: M/N \times M/N \rightarrow M/N$ by $$(x+N) + (y+N) = (x+y) + N$$
	\item $*: R \times M/N \rightarrow M/N$ by $$r(x+N) = (rx) + N$$
	\end{enumerate}
	Under these operations (see next exercise), $M/N$ is an $R$-module known as the \textbf{quotient module} of $M$ by $N$.
	\end{defn}	
	
	\begin{ex} 
	Let $M$ be an $R$-module and $N \in \MS(M)$. Then
	\begin{enumerate}
	\item the monoid action defined above is well defined
	\item the quotient $M/N$ is an $R$-module
	\end{enumerate}
	\end{ex}
	
	\begin{proof}\
	\begin{enumerate}
	\item Let $r \in R$ and $x +N, y +N \in M/N $. Recall from group theory that $x + N = y + N$ iff $x-y \in N$. Suppose that $x + N = y + N$. Then $x - y \in N$ and there exists $n \in N$ such that $x-y = n$. Therefore
	\begin{align*}
	rx - ry 
	&= r(x-y) \\
	&= rn \\
	&\in N
	\end{align*}
	So $rx + N = ry + N$.
	\item Properties $(1)$ - $(4)$ in the definition of a module are easily shown to be satisfied for $M/N$ since they are true for $M$.
	\end{enumerate}
	\end{proof}
	
	\begin{defn}
	Let $M$ and $N$ be $R$-modules and $\phi:M \rightarrow N$. Then $\phi$ is said to be a \textbf{module homomorphism} if for each $r \in R$ and $x,y \in M$
	\begin{enumerate}
	\item $\phi(rx) = r\phi(x)$
	\item $\phi(x+y) = \phi(x) + \phi(y)$
	\end{enumerate}
	\end{defn}	
	
	\begin{ex}
	Let $M$ and $N$ be $R$-modules and $\phi:M \rightarrow N$. Then $\phi$ is a  iff for each $r \in R$ and $x,y \in M$, $\phi(x+ry) = \phi(x) + r \phi(y)$.
	\end{ex}
	
	\begin{proof}
	Clear.
	\end{proof}
	
	\begin{ex}
	Let $M$ and $N$ be $R$-modules and $\phi:M \rightarrow N$ a homomorphism. Then 
	\begin{enumerate}
	\item $\ker \phi$ is a submodule of $M$
	\item $ \Im \phi$ is a submodule of $N$ 
	\end{enumerate}
	\end{ex}
	
	\begin{proof}
	Let $r \in R$, $x,y \in \ker \phi$ and $w,z \in \Im \phi$. Then 
	\begin{enumerate}
	\item 
	\begin{align*}
	\phi(rx) 
	&= r\phi(x) \\
	&=r 0 \\
	&= 0
\end{align*}	
	So $rx \in \ker \phi$. Group theory tells us that $\ker \phi$ is a subgroup of $M$, so $x+y \in \ker \phi$. Hence $\ker \phi$ is a submodule of $M$. 
	\item Similar.
	\end{enumerate}
	\end{proof}
	
	\begin{defn}
	Let $M$ be an $R$-module and $A \subset M$. We define the \textbf{submodule of $M$ generated by $A$}, denoted $\spn(A)$, to be $$\spn(A) = \bigcap_{N \in \MS(M)} N$$ 
	\end{defn}
	
	\begin{ex}
	Let $M$ be an $R$-module and $A \subset M$. Then 
	$\spn(A) \in \MS(M)$
	\end{ex}
	
	\begin{proof}
	Let $r \in R$ and $x,y \in \spn(A)$. Basic group theory tells us that $\spn(A)$ is a subgroup of $M$. So $x+y \in \spn(A)$. For $N \in \MS(M)$, by definition we have $x \in N$ and therefore $rx \in N$. So $rx \in \spn(A)$. Hence $\spn(A)$ is a submodule of $M$.
	\end{proof}
	
	\begin{ex}
	Let $M$ be an $R$-module and $A \subset M$. If $A \neq \varnothing$, then $$\spn(A) = \bigg \{\sum\limits_{i=1}^n r_ia_i: r_i \in R, a_i \in A, n \in \N \bigg \}$$
	\end{ex}
	
	\begin{proof}
	Clearly 
	\end{proof}
	
	\begin{defn}
	Let $M$
	\end{defn}
	
	
	
	
	
	
	
	
	
	
	
	
	
	
	
	
	
	
	
	
	\newpage
	\section{Fields}
	
	
	
	
	
	
	
	
	
	
	
	
	
	
	
	
	
	
	
	
	
	
	
	
	
	
	
	
	\newpage
	\section{Vector Spaces}
	
	\section{Appendix}
	\subsection{Monoids}
	
	\begin{defn}
	Let $G$ be a set and $*: G \times G \rightarrow G$ (we write $ab$ in place of $*(a,b)$). Then 
	\begin{enumerate}
	\item $*$ is called a \textbf{binary operation} on $G$	
	\item $*$ is said to be \textbf{associative}	if for each $x,y,z \in G$, $(xy)z = x(yz)$
	\item $*$ is said to be \textbf{commutative} if for each $x,y \in G$, $xy = yx$ 
	\end{enumerate}
	\end{defn}
	
	\begin{defn}
	Let $G$ be a set, $*: G \times G \rightarrow G$, $e,x,y \in G$. Then $e$ is said to be an \textbf{identity element} if for each $x \in G$, $ex = xe = x$.
	\end{defn}
	
	\begin{defn}
	Let $G$ be a set and $*: G \times G \rightarrow G$. Then $G$ is said to be a \textbf{monoid} if 
	\begin{enumerate}
	\item $*$ is associative
	\item there exits $e \in G$ such that $e$ is an identity element.
	\end{enumerate}
	\end{defn}	
	
	\begin{ex}
	Let $G$ be a monoid. Then the identity element is unique.
	\end{ex}
	
	\begin{proof}
	Let $e, f \in G$. Suppose that $e$ and $f$ are identity elements. Then $e = ef = f$.
	\end{proof}
	
	\begin{note}
	Unless otherwise specified, we will denote the identity element of a monoid by $e$.
	\end{note}
	
	\begin{defn}
	Let $G$ be a monoid, $X$ a set and $*: G \times X \rightarrow X$ (we write $gx$ in place of $*(g,x)$). Then $*$ is said to be a \textbf{monoid action} of $G$ on $X$ if for each $g,h \in G$ and $x \in X$,
	\begin{enumerate}
	\item $(gh)x = g(hx)$
	\item $ex = x$
	\end{enumerate}
	\end{defn}










































 
 
 
 
 
 
 
 
 
 
 
 
 
 
 
 
 
 
 
 
 
 
 
 
 
 
 
 
 
 
 
 
 
 
 
 
 
 
 
 
 
 
 
 
 
 
 
 
 
 
 
 
 
 
 
 
 
 \appendix
 
 \chapter{Summation}
 
 \begin{defn} \ld{}
 	Let $f:X \rightarrow \Rg$, Then we define $$\sum_{x \in X} f(x) := \sup_{\substack{F \subset X \\ F \text{ finite}}} \sum_{x \in F} f(x)$$ This definition coincides with the usual notion of summation when $X$ is countable. For $f:X \rightarrow \C$, we can write $f = g +ih$ where $g,h:X \rightarrow \R$. If $$\sum_{x \in X}|f(x)| < \infty,$$ then the same is true for $g^+,g^-,h^+,h^-$. In this case, we may define $$\sum_{x \in X} f(x)$$ in the obvious way.
 \end{defn} 
 
 The following note justifies the notation $\sum_{x \in X}f(x)$ where $f:X \rightarrow \C$.
 
 \begin{note}
 	Let $f:X \rightarrow \C$ and $\al:X \rightarrow X$ a bijection. If $\sum_{x \in X}|f(x)|< \infty$, then $\sum_{x \in X}f( \al (x)) = \sum_{x \in X}f(x) $.
 \end{note}
 
 \newpage	
 
 \chapter{Asymptotic Notation}
 
 \begin{defn} \ld{}
 	Let $X$ be a topological space, $Y, Z$ be normed vector spaces, $f:X \rightarrow Y$, $g: X \rightarrow Z$ and $x_0 \in X \cup \{\infty\}$. Then we write $$f = o(g) \hspace{.5cm} \text{ as } x \rightarrow x_0$$ if for each $\ep >0$, there exists $U \in \MN(x_0)$ such that for each $x \in U$, $$\|f(x)\| \leq \ep\|g(x)\|$$
 \end{defn}
 
 \begin{ex} \lex{}
 	Let $X$ be a topological space, $Y, Z$ be normed vector spaces, $f:X \rightarrow Y$, $g: X \rightarrow Z$ and $x_0 \in X \cup \{\infty\}$. If there exists $U \in \MN(x_0)$ such that for each $x \in U \setminus \{x_0\}$, $g(x) > 0$, then $$f = o(g) \text{ as } x \rightarrow x_0 \hspace{.25cm} \text{ iff } \hspace{.25cm}  \lim_{x \rightarrow x_0} \frac{\| f(x) \|}{\| g(x) \|} = 0$$
 \end{ex}	
 
 \begin{ex} \lex{}
 	Let $X$ and $Y$ a be normed vector spaces, $A \subset X$ open and $f:A \rightarrow Y$. Suppose that $0 \in A$. If $f(h) = o(\|h\|)$ as $h \rightarrow 0$, then for each $h \in X$,  $f(th) = o(|t|)$ as $t \rightarrow 0$.
 \end{ex}	
 
 \begin{proof}
 	Suppose that $f(h) = o(\|h\|)$ as $h \rightarrow 0$.  Let $h \in X$ and $\ep >0$. Choose $\del' >0 $ such that for each $h' \in B(0, \del')$, $h' \in A$ and 
 	$$\|f(h')\| \leq \frac{\ep}{\|h\|+1} \|h'\|$$ 
 	Choose $\del >0$ such that for each $t \in B(0,\del)$, $th \in B(0,\del')$. Let $t \in B(0,\del)$. Then 
 	\begin{align*}
 		\|f(th)\| 
 		&\leq \frac{\ep}{\|h\|+1} |t|\|h\| \\
 		&< \ep |t|
 	\end{align*}
 	So $f(th) = o(|t|)$ as $t \rightarrow 0$.
 \end{proof}		
 
 
 
 
 \begin{defn} \ld{}
 	Let $X$ be a topological space, $Y, Z$ be normed vector spaces, $f:X \rightarrow Y$, $g: X \rightarrow Z$ and $x_0 \in X \cup \{\infty\}$. Then we write $$f = O(g) \hspace{.5cm} \text{ as } x \rightarrow x_0$$ if there exists $U \in \MN(x_0)$ and $M \geq 0$ such that for each $x \in U$, $$\|f(x)\| \leq M\|g(x)\|$$
 \end{defn}
 
 
 
 
 
 
 
 
 
 
 
 
 
 
 
 
 
 
 
 
 
 
 
 
 
 
 
 \newpage
 \chapter{Categories}
 
 \tcr{move to notation?}
 
 \begin{defn}
 	We define the category of topological measure spaces, denoted $\TopMsrpos$, by 
 	\begin{itemize}
 		\item $\Obj(\TopMsrpos) \defeq \{(X, \mu): X \in \Obj(\Top) \text{ and } \mu \in M(X)\}$			
 		\item $\Hom_{\TopMsrpos}((X, \mu), (Y, \nu)) \defeq \Hom_{\Top}(X, Y) \cap \Hom_{\Msrpos}((X, \MB(X), \mu), (Y, \MB(Y), \nu))$
 	\end{itemize}
 \end{defn}
 
 
 
 
 
 
 
 
 
 
 
 
 
 
 
 
 
 
 
 
 
 
 
 
 
 
 
 
 
 
 
 
 \newpage
 \chapter{Vector Spaces}
 \tcr{it might be better to cover some category theory and write everything in terms of $\Hom_{\VectK}$ and $\Obj(\VectK)$}
 
 \section{Introduction}
 
 \begin{defn}
 	Let $X$ be a set, $\K$ a field, $+:X \times X \rightarrow X$ and $\cdot:\K \times X \rightarrow X$. Then $(X, +, \cdot)$ is said to be a \tbf{$\K$-vector space} if 
 	\begin{enumerate}
 		\item $(X, +)$ is an abelian group
 		\item 
 	\end{enumerate} 
 \end{defn}
 
 
 \begin{defn}
 	Let $(X, +_X, \cdot_X)$ and $(E, +_E, \cdot_E)$ be vector spaces. Suppose that $E \subset X$. Then $(E, +_E, \cdot_E)$ is said to be a subspace of $X$ if 
 	\begin{enumerate}
 		\item $+_E = +_X|_{E \times E}$
 		\item $\cdot_E = \cdot_X|_{\K \times E}$
 	\end{enumerate}
 \end{defn}
 
 \begin{ex}
 	Let $(X, +_X, \cdot_X)$ and $(E, +_E, \cdot_E)$ be vector spaces. Suppose that $E \subset X$. 
 \end{ex}
 
 \begin{ex}
 	Let $(X, +, \cdot)$ be a vector space and $E \subset X$. Then $E$ is a subspace of $X$
 \end{ex}
 
 
 \begin{defn}
 	Let $X$ be a vector space and $(E_j)_{j \in J}$ a collection of subspaces of $X$. Then $\bigcap\limits_{j \in J}E_j$ is a subspace of $X$. 
 \end{defn}
 
 \begin{proof}
 	Set $E \defeq \bigcap\limits_{j \in J}E_j$. Let $x,y \in E$ and $\lam \in \K$. Then for each $j \in J$, $x,y \in E_j$. Since for each $j \in J$, $E_j$ is a subspace of $X$, we have that for each $j \in J$, $x+ \lam y \in E_j$. Thus $x+\lam y \in E$. Since $x,y \in E$ and $\lam \in \K$ are arbitrary, \tcr{(cite exercise here)} we have that $E$ is a subspace of $X$. 
 \end{proof}
 
 
 
 
 
 
 
 
 
 
 
 
 
 
 
 
 
 
 
 
 
 
 
 
 
 
 \begin{defn}
 	Let $X, Y$ be vector spaces and $T:X \rightarrow Y$. Then $T$ is said to be \tbf{linear} if for each $x_1, x_2 \in X$ and $\lam \in \Lam$, 
 	\begin{enumerate}
 		\item $T(x_1 + x_2) = T(x_1) + T(x_2)$,
 		\item $T(\lam x_1) = \lam T(x_1)$.
 	\end{enumerate}
 	We define $L(X;Y) \defeq \{T:X \rightarrow Y: \text{ $T$ is linear}\}$. 
 \end{defn}
 
 \begin{ex}
 	Let $X,Y$ be vector spaces and $T : X \rightarrow Y$. Then $T$ is linear iff for each $x_1, x_2 \in X$ and $\lam \in \Lam$, 
 	$$T(x_1 + \lam x_2) = T(x_1) + \lam T(x_2)$$
 \end{ex}
 
 \begin{proof}
 	Clear. \tcr{(add details)}
 \end{proof}
 
 \begin{defn}
 	\tcr{define addition/scalar multiplication of linear maps}
 \end{defn}
 
 \begin{ex}
 	Let $X,Y$ be vector spaces. Then $L(X;Y)$ is a $\K$-vector space. 
 \end{ex}
 
 \begin{proof}
 	Clear
 \end{proof}
 
 \begin{defn} \ld{55001}\
 	Let $X$ be a vector space over $\K$ and $T :X \rightarrow \K$. Then $T$ is said to be a \tbf{linear functional on} $X$ if $T$ is linear. We define the \tbf{dual space of $X$}, denoted $X^*$, by $X^* \defeq \{ T:X \rightarrow \K: T \text{ is linear}\}$. 
 \end{defn}
 
 
 \begin{ex}
 	Let $X$ be a vector space. Then $X^*$ is a vector space. 
 \end{ex}
 
 \begin{proof}
 	Clear.
 \end{proof}
 
 
 
 
 
 
 
 
 
 
 
 
 
 
 
 
 
 
 
 
 
 
 
 
 
 
 
 
 
 
 
 \section{Bases}
 
 \begin{defn}
 	Let $X$ be a vector space and $(e_{\al})_{\al \in A} \subset X$. Then $(e_{\al})_{\al \in A}$ is said to be
 	\begin{itemize}
 		\item \tbf{linearly independent} if for each $(\al_j)_{j=1}^n \subset A$, $(\lam_j)_{j=1}^n \subset \K$, $\sum\limits_{j=1}^n \lam_j e_{\al_j} = 0$ implies that for each $j \in [n]$, $\lam_j = 0$.  
 		\item a \tbf{Hamel basis for $X$} if $(e_{\al})_{\al \in A}$ is linearly independent and $\spn (e_{\al})_{\al \in A} = X$. 
 	\end{itemize}
 \end{defn}
 
 \begin{ex}
 	\tcr{every vector space has a Hamel basis}
 \end{ex}
 
 \begin{proof}
 	
 \end{proof}
 
 \begin{ex}
 	
 \end{ex}
 
 
 \begin{ex}
 	Let $X$ be a $\K$-vector space and $x \in X$. Then $x = 0$ iff for each $\phi \in X^*$, $\phi(x) = 0$. 
 \end{ex}
 
 \begin{proof}\
 	\begin{itemize}
 		\item $(\implies)$: \\
 		Suppose that $x = 0$. Linearity implies that for each $\phi \in X^*$ $\phi(x) = 0$. 
 		\item $(\impliedby)$: \\
 		Conversely, suppose that $x \neq 0$. Define $\ep_x: \spn(x) \rightarrow \K$ by $\ep_x(\lam x) \defeq \lam$. Let $u,v \in \spn(x)$. Then there exists $\lam_u, \lam_v \in \K$ such that $u = \lam_u x$ and $v = \lam_v x$. Suppose that $u = v$. Then 
 		\begin{align*}
 			(\lam_u - \lam_v)x
 			& = \lam_u x - \lam_v x \\
 			& = u - v \\
 			& = 0
 		\end{align*}
 		Since $x \neq 0$, we have that $\lam_u - \lam_v = 0$ and therefore $\lam_u = \lam_v$. Hence  
 		\begin{align*}
 			\lam_u 
 			& = \ep_x(u) \\
 			& = \ep_x(v) \\
 			& = \lam_v.
 		\end{align*}
 		Thus $\ep_x$ is well defined. 
 	\end{itemize}
 \end{proof}
 
 
 
 
 
 
 
 
 
 
 
 
 
 
 
 
 
 
 
 
 
 
 
 
 
 
 
 
 
 
 
 
 
 
 
 \newpage
 \section{Multilinear Maps}
 
 \begin{defn}
 	Let $X_1, \cdots, X_n, Y$ be vector spaces and $T: \prod\limits_{j=1}^n X_j \rightarrow \K$. Then $T$ is said to be \tbf{multilinear} if for each $j_0 \in [n]$ and $(x_j)_{j=1}^n \in \prod\limits_{j=1}^n X_j$, $T(x_1, \ldots, x_{j_0 - 1}, \cdot, x_{j_0 + 1})$ is linear. $$L^n(X_1, \dots, X_n; Y) = \bigg\{T : \prod\limits_{j=1}^n X_j \rightarrow Y: T \text{ is multilinear}\bigg \}$$ 
 	If $X_1 = \cdots = X_n = X$, we write $L^n(X;Y)$ in place of $L^n (X, \dots, X; Y) $. 
 \end{defn}
 
 \begin{defn}
 	\tcr{define addition and scalar mult of multilinear maps}
 \end{defn}
 
 \begin{ex}
 	Let $X_1, \cdots, X_n, Y$ be vector spaces. Then $L^n(X_1, \ldots, X_n;Y)$ is a $\K$-vector space.
 \end{ex}
 
 \begin{proof}
 	content...
 \end{proof}
 
 \begin{ex}
 	Let $X_1, \cdots, X_n, Y, Z$ be $\K$-vector spaces, $\al \in L^n(X_1, \ldots, X_n;Y)$ and $\phi \in L^1(Y;Z)$. Then $\phi \circ \al \in L^n(X_1, \ldots, X_n; Z)$. 
 \end{ex}
 
 \begin{proof}
 	Let $(x_j)_{j=1}^n \in \prod\limits_{j=1}^n X_j$ and $j_0 \in [n]$. Define $f:X_{j_0} \rightarrow Y$ by 
 	$$f(a) \defeq \al(x_1, \ldots, x_{j_0-1}, a , x_{j_0+1}, \ldots, x_n) $$
 	Since $\al \in L^n(X_1, \ldots, X_n;Y)$, $f$ is linear. Since $\phi$ is linear, and $\phi \circ f$ is linear. Since $(x_j)_{j=1}^n \in \prod\limits_{j=1}^n X_j$ and $j_0 \in [n]$ are arbitrary, we have that $\phi \circ \al \in L^n(X_1, \ldots, X_n;Y)$. 
 \end{proof}
 
 
 
 
 
 
 
 
 
 
 
 
 
 
 
 
 
 
 
 
 
 
 
 
 
 
 
 
 
 
 
 
 
 
 
 
 
 
 
 
 
 
 \newpage
 \section{Tensor Products}
 
 \begin{defn}
 	Let $X, Y$ and $T$ be vector spaces over $\K$ and $\al \in L^2(X, Y; T)$. Then $(T, \al)$ is said to be a \tbf{tensor product of $X$ and $Y$} if for each vector space $Z$ and $\be \in L^2(X, Y; Z)$, there exists a unique $\phi \in L^1(T;Z)$ such that $\phi \circ \al = \be$, i.e. the following diagram commutes:
 	\[ 
 	\begin{tikzcd}
 		X \times Y \arrow[r, "\al"] \arrow[dr, "\be"'] 	
 		& T  \arrow[d, dashed, "\phi"] \\
 		& Z 
 	\end{tikzcd}
 	\] 
 \end{defn}
 
 \begin{ex}
 	Let $X, Y, S, T$ be vector spaces, $\al \in L^2(X, Y; S)$ and $\be \in L^2(X, Y; T)$. Suppose that $(S, \al)$ and $(T, \be)$ are tensor products of $X$ and $Y$. Then $S$ and $T$ are isomorphic. 
 \end{ex}
 
 \begin{proof}
 	Since $(T, \be)$ is a tensor product of $X$ and $Y$, $\be \in L^2(X,Y; T)$ there exists a unique $f \in L^1(T;T)$ such that $f\ circ \be = \be$, i.e. the following diagram commutes: 
 	\[ 
 	\begin{tikzcd}
 		& T \arrow[dd, dashed, "f"] \\
 		X \times Y \arrow[ur, "\be"] \arrow[dr, "\be"'] 
 		&   \\
 		& T 
 	\end{tikzcd}
 	\] 
 	Since $\id_T \in L^1(T;T)$ and $\id_T \circ \be = \be$, we have that $f = \id_T$. Since $(S, \al)$ is a tensor product of $X$ and $Y$, there exists a unique $\phi: S \rightarrow T$ such that $\phi \circ \al = \be$, i.e. the following diagram commutes: 
 	\[ 
 	\begin{tikzcd}
 		X \times Y \arrow[r, "\al"] \arrow[dr, "\be"'] 	
 		& S  \arrow[d, dashed, "\phi"] \\
 		& T 
 	\end{tikzcd}
 	\] 
 	Similarly, since $(T, \be)$ is a tensor product of $X$ and $Y$, there exists a unique $\psi: T \rightarrow S$ such that $\psi \circ \be = \al$, i.e. the following diagram commutes: 
 	\[ 
 	\begin{tikzcd}
 		X \times Y \arrow[r, "\be"] \arrow[dr, "\al"'] 	
 		& T  \arrow[d, dashed, "\psi"] \\
 		& S 
 	\end{tikzcd}
 	\] 
 	Therefore 
 	\begin{align*}
 		(\phi \circ \psi) \circ \be 
 		& = \phi \circ (\psi \circ \be) \\
 		& = \phi \circ \al \\
 		& = \be, 
 	\end{align*}
 	i.e. the following diagram commutes:
 	\[ 
 	\begin{tikzcd}
 		& T \arrow[d, dashed, "\psi"] \\
 		X \times Y \arrow[ur, "\be"] \arrow[dr, "\be"'] \arrow[r, "\al"]	
 		& S  \arrow[d, dashed, "\phi"] \\
 		& T 
 	\end{tikzcd}
 	\implies
 	\begin{tikzcd}
 		& T \arrow[dd, dashed, "\phi \circ \psi"] \\
 		X \times Y \arrow[ur, "\be"] \arrow[dr, "\be"'] 
 		&  \\
 		& T 
 	\end{tikzcd}
 	\] 
 	By uniqueness of $f \in L^1(T;T)$, we have that 
 	\begin{align*}
 		\id_T
 		& = f \\
 		& = \phi \circ \psi 
 	\end{align*}
 	A similar argument implies that $\psi \circ \phi = \id_S$. Hence $\phi$ and $\psi$ are isomorphisms with $\phi^{-1} = \psi$. Hence $S$ and $T$ are isomorphic.
 \end{proof}
 
 \begin{defn}
 	Let $X, Y$ be vector spaces, $x \in X$ and $y \in Y$. We define $x \otimes y: X^* \times Y^* \rightarrow  \K$ by $x \otimes y(\phi, \psi) \defeq \phi(x) \psi(y)$.  
 \end{defn}
 
 \begin{ex}
 	Let $X, Y$ be vector spaces, $x \in X$ and $y \in Y$. Then $x \otimes y \in L^2(X^*, Y^*; \K)$. 
 \end{ex}
 
 \begin{proof}
 	Let $\phi_1, \phi_2 \in X^*$, $\psi \in Y^*$ and $\lam \in \K$. Then 
 	\begin{align*}
 		x \otimes y(\phi_1 + \lam \phi_2, \psi) 
 		& = [\phi_1 + \lam \phi_2 (x)] \psi(y) \\
 		& = \phi_1(x)\psi(y) + \lam \phi_2(x)\psi(y) \\
 		& = x \otimes y(\phi_1, \psi) + \lam x \otimes y(\phi_2, \psi)
 	\end{align*}
 	Since $\phi_1, \phi_2 \in X^*$, $\psi \in Y^*$ and $\lam \in \K$ are arbitrary, we have that for each $\psi \in Y^*$, $x \otimes y(\cdot, \psi)$ is linear. Similarly for each $\phi \in X^*$, $x \otimes y(\phi, \cdot)$ is linear. Hence $x \otimes y$ is bilinear and $x \otimes y \in L^2(X^*, Y^*; \K)$. 
 \end{proof}
 
 \begin{defn}
 	Let $X, Y$ be vector spaces. We define  
 	\begin{itemize}
 		\item the \tbf{tensor product of $X$ and $Y$}, denoted $X \otimes Y \subset L^2(X^*, Y^*; \K)$, by 
 		$$X \otimes Y \defeq \spn(\text{$x \otimes y: x \in X$ and $y \in Y$}),$$
 		\item the \tbf{tensor map}, denoted $\otimes: X \times Y \rightarrow X \otimes Y$, by $\otimes(x, y) \defeq x \otimes y$.
 	\end{itemize}
 \end{defn}
 
 \begin{ex}
 	Let $X,Y$ be vector spaces, $(x_j)_{j=1}^n \subset X$ and $(y_j)_{j=1}^n \subset Y$. The following are equivalent:
 	\begin{enumerate}
 		\item $\sum\limits_{j=1}^n  x_j \otimes y_j = 0$
 		\item for each $\phi \in X^*$ and $\psi \in Y^*$, $\sum\limits_{j=1}^n  \phi(x_j) \psi(y_j) = 0$
 		\item for each $\phi \in X^*$, $\sum\limits_{j=1}^n  \phi(x_j)  y_j = 0$
 		\item for each $\psi \in Y^*$, $\sum\limits_{j=1}^n  \psi(y_j) x_j = 0$
 	\end{enumerate}
 \end{ex}
 
 \begin{proof}\
 	\begin{enumerate}
 		\item $(1) \implies (2):$ \\
 		Suppose that $\sum\limits_{j=1}^n x_j \otimes y_j = 0$. Let $\phi \in X^*$ and $\psi \in Y^*$. Then 
 		\begin{align*}
 			\sum\limits_{j=1}^n  \phi(x_j) \psi(y_j)
 			& = \phi \bigg( \sum\limits_{j=1}^n \psi(y_j) x_j \bigg) \\
 			& = 
 		\end{align*}
 		\item 
 		\item 
 	\end{enumerate}
 \end{proof}
 
 \begin{ex}
 	Let $X, Y$ be vector spaces. Then $(X \otimes Y, \otimes)$ is a tensor product of $X$ and $Y$. 
 \end{ex}
 
 \begin{proof}
 	Let $Z$ be a vector space and $\al \in L^2(X, Y; Z)$. Define $\phi: X \otimes Y \rightarrow Z$ by $\phi \bigg( \sum\limits_{j=1}^n \lam_j x_j \otimes y_j) \defeq \sum\limits_{j =1}^n \lam_j \al(x_j, y_j)$.  
 	\begin{itemize}
 		\item \tbf{(well defined):} \\
 		Let $u \in X \otimes Y$. Then there exist $(\lam_j)_{j=1}^n \subset \K$, $(x_j)_{j=1}^n \subset X$, $(y_j)_{j=1}^n \subset Y$ such that $u = \sum\limits_{j=1}^n \lam_j x_j \otimes y_j$. Suppose that $u = 0$. Let $\phi \in Z^*$. Then $\phi \circ \al \in L^2(X,Y; Z)$.  
 	\end{itemize}
 \end{proof}
 
 
 
 
 
 
 
 
 
 
 
 
 
 
 
 
 
 
 
 
 
 
 
 
 
 
 
 
 
 
 
 
 
 
 
 
 
 
 
 
 
 
 
 
 
 
 
 
 
 
 
 
 
 
 
 
 
 
 
 
 
 
 
 
 
 
 
 
 
 
 
 
 
 
 
 
 
 
 \backmatter
 \begin{thebibliography}{4}
 	\bibitem{algebra} \href{https://github.com/carsonaj/Mathematics/blob/master/Introduction\%20to\%20Algebra/Introduction\%20to\%20Algebra.pdf}{Introduction to Algebra}
 	
 	\bibitem{analysis}  \href{https://github.com/carsonaj/Mathematics/blob/master/Introduction\%20to\%20Analysis/Introduction\%20to\%20Analysis.pdf}{Introduction to Analysis}	
 	
 	\bibitem{foranal}  \href{https://github.com/carsonaj/Mathematics/blob/master/Introduction\%20to\%20Fourier\%20Analysis/Introduction\%20to\%20Fourier\%20Analysis.pdf}{Introduction to Fourier Analysis}
 	
 	\bibitem{measure}  \href{https://github.com/carsonaj/Mathematics/blob/master/Introduction\%20to\%20Measure\%20and\%20Integration/Introduction\%20to\%20Measure\%20and\%20Integration.pdf}{Introduction to Measure and Integration}
 	
 	
 	
 \end{thebibliography}
 
 
 
 
 
 
 
	
	
	
	
	
	
	
	
	
	
	
	
	
	
	
	
	
	
	
	
	
	
	
\end{document}