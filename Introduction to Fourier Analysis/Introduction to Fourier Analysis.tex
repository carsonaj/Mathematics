\documentclass[12pt]{amsart}
\usepackage[margin=1in]{geometry} 
\usepackage{amsmath,amsthm,amssymb,setspace, mathtools}

\usepackage{color}   %May be necessary if you want to color links
\usepackage{hyperref}
\hypersetup{
	colorlinks=true, %set true if you want colored links
	linktoc=all,     %set to all if you want both sections and subsections linked
	linkcolor=black,  %choose some color if you want links to stand out
	urlcolor=cyan
}


%
%
%
\newif\ifhideproofs
%\hideproofstrue %uncomment to hide proofs
%
%
%
%
\ifhideproofs
\usepackage{environ}
\NewEnviron{hide}{}
\let\proof\hide
\let\endproof\endhide
\fi

\theoremstyle{definition}
\newtheorem{definition}{Definition}[subsection]
\newtheorem{defn}[definition]{Definition}
\newtheorem{note}[definition]{Note}
\newtheorem{thm}[definition]{Theorem}
\newtheorem{lem}[definition]{Lemma}
\newtheorem{prop}[definition]{Proposition}
\newtheorem{cor}[definition]{Corollary}
\newtheorem{conj}[definition]{Conjecture}
\newtheorem{ex}[definition]{Exercise}



\DeclareMathOperator{\supp}{supp}

\newcommand{\p}{\partial}

\newcommand{\al}{\alpha}
\newcommand{\Gam}{\Gamma}
\newcommand{\bet}{\beta} 
\newcommand{\del}{\delta} 
\newcommand{\Del}{\Delta}
\newcommand{\lam}{\lambda}  
\newcommand{\Lam}{\Lambda} 
\newcommand{\ep}{\epsilon}
\newcommand{\sig}{\sigma} 
\newcommand{\om}{\omega}
\newcommand{\Om}{\Omega}
\newcommand{\kap}{\kappa}
\newcommand{\C}{\mathbb{C}}
\newcommand{\N}{\mathbb{N}}
\newcommand{\E}{\mathbb{E}}
\newcommand{\Z}{\mathbb{Z}}
\newcommand{\R}{\mathbb{R}}
\newcommand{\T}{\mathbb{T}}
\newcommand{\Q}{\mathbb{Q}}
\renewcommand{\P}{\mathbb{P}}
\newcommand{\MA}{\mathcal{A}}
\newcommand{\MC}{\mathcal{C}}
\newcommand{\MB}{\mathcal{B}}
\newcommand{\MF}{\mathcal{F}}
\newcommand{\MG}{\mathcal{G}}
\newcommand{\ML}{\mathcal{L}}
\newcommand{\MN}{\mathcal{N}}
\newcommand{\MS}{\mathcal{S}}
\newcommand{\MP}{\mathcal{P}}
\newcommand{\ME}{\mathcal{E}}
\newcommand{\MT}{\mathcal{T}}
\newcommand{\MM}{\mathcal{M}}
\newcommand{\MI}{\mathcal{I}}

\newcommand{\io}{\text{ i.o.}}
\newcommand{\ev}{\text{ ev.}}
\renewcommand{\r}{\rangle}
\renewcommand{\l}{\langle}

\newcommand{\dm}{\, d m}
\newcommand{\dmu}{\, d \mu}
\newcommand{\dnu}{\, d \nu}
\newcommand{\dlam}{\, d \lambda}

\newcommand{\RG}{[0,\infty]}
\newcommand{\Rg}{[0,\infty)}
\newcommand{\Ll}{L^1_{\text{loc}}(\R^n)}

\newcommand{\limfn}{\liminf \limits_{n \rightarrow \infty}}
\newcommand{\limpn}{\limsup \limits_{n \rightarrow \infty}}
\newcommand{\limn}{\lim \limits_{n \rightarrow \infty}}
\newcommand{\convt}[1]{\xrightarrow{\text{#1}}}
\newcommand{\conv}[1]{\xrightarrow{#1}} 
\newcommand{\seq}[2]{(#1_{#2})_{#2 \in \N}}

\newcommand{\loc}{\text{loc}}

\DeclareMathOperator*{\sgn}{sgn}
\DeclareMathOperator*{\spn}{span}
\DeclareMathOperator*{\Aut}{Aut}
\DeclareMathOperator*{\id}{id}

\newcommand{\lex}[1]{\label{ex:#1}}
\newcommand{\ld}[1]{\label{defn:#1}}
\newcommand{\rex}[1]{Exercise \ref{ex:#1}}
\newcommand{\rd}[1]{Definition \ref{defn:#1}}



\begin{document}
	
	\title{Introduction to Fourier Analysis}
	\author{Carson James}
	\maketitle
	
	\tableofcontents
	
	\newpage
	\section{Fourier Analysis on $\R$}	
	
	\subsection{Schwartz Space and Bump Functions}
	
	\begin{defn}
		\ld{101} Let $f \in C^{\infty}(\R)$, and $\al$, $N \in \N_0$. We define $\| \cdot \|_{\al, N}: C^{\infty}(\R, \C) \rightarrow \RG$ by 
		$$\|f\|_{\al, N} = \sup_{x \in \R} \bigg[  (1 + |x|)^N |\p^{\al}f (x)| \bigg] $$
		We define \textbf{Schwartz space} on $\R$, denoted $\MS$, by $$\MS = \{f \in C^{\infty}(\R): \text{ for each $\al, N \in \N_0$, } \|f\|_{\al, N} < \infty\}$$
	\end{defn}

	\begin{ex}
		\lex{102} Let $f \in \MS$. Then 
		\begin{enumerate}
			\item $f$ is Lipschitz 
			\item $ f \in  L^1(m)$
		\end{enumerate}
	\end{ex}
	
	\begin{proof}\
		\begin{enumerate}
			\item There exists $M > 0$ such that for each $x \in \R$, 
			\begin{align*}
				| \p f(x)| 
				& \leq M(1+|x|)^{-1} \\
				& \leq M
			\end{align*}
			Thus $\p f$ is bounded which implies that $\p^{\al}f$ is Lipschitz.
			\item There exists $C \geq 0$ such that for each $x \in \R$, 
			\begin{align*}
				|f(x)| 
				& \leq C(1+|x|)^{-2} \\
				& \leq C(1+|x|^2)^{-1}
			\end{align*}
			Define $g:\R \rightarrow \Rg$ defined by $g(x) = (1+|x|^2)^{-1}$. Then $g \in L^1(m)$ which implies that $f \in L^1(m)$.
		\end{enumerate}
	\end{proof}

	\begin{ex} We have that $\MS$ is a vector space and for each $\al, N \in \N_0$,  $\| \cdot \|_{\al, N}: \MS \rightarrow \Rg$ is a seminorm on $\MS$.
	\end{ex}

	\begin{proof} Let $f, g \in \MS$ and $\lam \in \C$.
		\begin{enumerate}
			\item 
			\begin{align*}
				\|\lam f\|_{\al, N}
				& = \sup_{x \in \R} \bigg[  (1 + |x|)^N |\p^{\al}[\lam f] (x)| \bigg] \\
				& = \sup_{x \in \R} \bigg[  (1 + |x|)^N |\lam \p^{\al}f (x)| \bigg] \\
				& = \sup_{x \in \R} \bigg[  |\lam| (1 + |x|)^N | \p^{\al}f (x)| \bigg] \\
				& = |\lam| \sup_{x \in \R} \bigg[ (1 + |x|)^N | \p^{\al}f (x)| \bigg] \\
				& = |\lam | \|f\|_{\al, N}
			\end{align*}
			Thus $\lam f \in \MS$ and $\|\lam f\|_{\al, N} = |\lam | \|f\|_{\al, N}$.
		\item \begin{align*}
			\|f  +  g\|_{\al, N} 
			& = \sup_{x \in \R} \bigg[  (1 + |x|)^N |\p^{\al}[f + g] (x)| \bigg] \\
			& = \sup_{x \in \R} \bigg[  (1 + |x|)^N |[\p^{\al} f  + \p g] (x)| \bigg] \\
			& \leq \sup_{x \in \R} \bigg[  (1 + |x|)^N |\p^{\al} f (x)|  +  (1 + |x|)^N |\p g (x)| \bigg] \\
			& \leq \sup_{x \in \R} \bigg[  (1 + |x|)^N |\p^{\al} f (x)| \bigg]   + \sup_{x \in \R} \bigg[ (1 + |x|)^N |\p g (x)| \bigg] \\
			& = \|f\|_{\al, N} + \|g\|_{\al, N} 
		\end{align*}
	 	Hence $f + g \in \MS$ and $\|f + g\|_{\al, N} \leq \|f\|_{\al, N} + \|g\|_{\al, N}$.
		\end{enumerate}
		So $\MS$ is a vector space and $\| \cdot \|_{\al, N}$ is a seminorm on $\MS$.
	\end{proof}

	\begin{ex}
		We have that $\MS$ is a algebra under pointwise multiplication and for each $\al, N \in \N_0$, 
		$$\|fg\|_{\al, N} \leq \sum\limits_{\bet=0}^\al  \|f\|_{\bet, N} \|g\|_{\al - \bet, 0}$$
		\textbf{Hint:} $\p^{\al}[fg] = \sum\limits_{\bet=0}^\al (\p^{\bet}f) (\p^{\al - \bet}g)$
	\end{ex}

	\begin{proof}
		Let $f,g \in \MS$ and $\al, N \in \N_0$. Then 
		\begin{align*}
			\|fg\|_{\al, N}
			& = \sup_{x \in \R} \bigg[ (1 + |x|)^N|\p^{\al}[fg](x)| \bigg] \\
			& = \sup_{x \in \R} \bigg[ (1 + |x|)^N \bigg | \sum\limits_{\bet=0}^\al \p^{\bet}f(x)\p^{\al - \bet}g(x) \bigg|  \bigg] \\
			& \leq \sup_{x \in \R} \bigg[ (1 + |x|)^N \bigg( \sum\limits_{\bet=0}^\al |\p^{\bet}f(x)||\p^{\al - \bet}g(x)| \bigg) \bigg] \\
			& = \sup_{x \in \R} \bigg[   \sum\limits_{\bet=0}^\al (1 + |x|)^N|\p^{\bet}f(x)| |\p^{\al - \bet}g (x)| \bigg] \\
			& \leq \sum\limits_{\bet=0}^\al \sup_{x \in \R} \bigg[ (1 + |x|)^N|\p^{\bet}f(x)| |\p^{\al - \bet}g (x)| \bigg] \\
			& \leq \sum\limits_{\bet=0}^\al \sup_{x \in \R} \bigg[ (1 + |x|)^N|\p^{\bet}f(x)| \bigg]  \sup_{x \in \R} \bigg[|\p^{\al - \bet}g (x) | \bigg] \\
			& = \sum\limits_{\bet=0}^\al  \|f\|_{\bet, N} \|g\|_{\al - \bet, 0} \\
			& < \infty
		\end{align*} 
		So $fg \in \MS$.
	\end{proof}

	\begin{defn}
		Set $\MP = (\|\cdot\|_{\al, N})_{\al, N \in \N_0}$. Then $\MP$ is a countable family of seminorms on $\MS$. We equip $\MS$ with the topology $\MT$ induced by the family of projections $$\pi_{\| \cdot \|_{\al,N}}: \MS \rightarrow \MS / \ker \|\cdot\|_{\al,N} $$ 
		i.e. $\MT = \tau_{\MS}((\pi_{\|\cdot\|_{\al,N}})_{\al, N \in \N_0})$.  \\
		Explicitly, for a net $(f_{\al})_{\al \in A} \subset \MS$ and $f \in \MS$, $f_{\al} \rightarrow f$ iff for each $\al, N \in \N_0$, $\|f_{\al} - f\|_{\al, N} \rightarrow 0$. \\
		Hence $(\MS, \MT)$ is a locally convex space. Since $\MP$ is countable, we may write $\MP = (p_j)_{j \in \N}$ and thus $(\MS, \MT)$ is metrizable with metric
		$$d_{\MS}(f,g) = \sum_{j \in \N} 2^{-j} \frac{p_j(f-g)}{1 + p_j(f-g)}$$
	\end{defn}

	\begin{ex}
		Let $f \in \MS$ and $\al \in \N_0$. Then $\p^{\al}f \in \MS$ and for each $\bet, N \in \N_0$, 
		$$\|\p^{\al} f \|_{\bet, N} \leq \|f \|_{\al + \bet, N}$$ 
	\end{ex}
	
	\begin{proof}
		Let $f \in \MS$, and $\bet$, $N \in \N_0$. By definition, 
		\begin{align*}
			\|\p^{\al} f \|_{\bet, N}
			& = \sup_{x \in \R} \bigg[ (1 + |x|)^N |\p^{\beta} [\p^{\al} f] (x)| \bigg] \\
			&= \sup_{x \in \R} \bigg[ (1 + |x|)^N |\p^{\al + \beta}f (x)| \bigg] \\
			& = \|f \|_{\al + \bet, N} \\
			& < \infty
		\end{align*}
		So $\p^{\al}f \in \MS$.
	\end{proof}

	\begin{ex}
		Let $f \in \MS$. Then for each $\al$, $N \in \N_0$, 
		$$\|f\|_{\al, N} = \|\p^{\al} f\|_{0, N}$$
	\end{ex}

	\begin{proof}
		Clear by preceding exercise.
	\end{proof}

	\begin{ex}
		Let $f \in \MS$ and $\al \in \N$. Define $g: \R \rightarrow \C$ by $g(x) = xf(x)$. Then for each $x \in \R$, $\p^{\al}g(x) = x\p^{\al}f(x) + \al \p^{\al -1} f(x)$.
	\end{ex}
	
	\begin{proof}
		The claim is clear if $\al = 1$. Suppose that $\al > 1$ and that the claim is true for $\al - 1$ so that for each $x \in \R$,  $\p^{\al-1}g(x) = x\p^{\al-1}f(x) + (\al - 1) \p^{\al-2} f(x)$. Then 
		\begin{align*}
			\p^{\al}g(x)
			&= \p [\p^{\al-1}g(x)] \\
			&= \p [x\p^{\al-1}f(x) + (\al - 1) \p^{\al-2} f(x)] \\
			&= \p [x\p^{\al-1}f(x)] + \p[(\al - 1) \p^{\al-2} f(x)] \\
			&= [x \p^{\al}f(x) + \p^{\al-1}f(x)]  + [(\al -1) \p^{\al-1}f(x)] \\
			&= x \p^{\al}f(x) + \al \p^{\al-1}f(x)
		\end{align*}
		So the claim is true for $\al$.
	\end{proof}
	
	\begin{ex}
		Let $f \in \MS$ and $N \in \N_0$. Define $g: \R \rightarrow \C$ by $g(x) = xf(x)$. Then $g \in \MS$ and for each $\al$, $N \in \N_0$, 
		$$ \|g\|_{\al, N} \leq \|f\|_{\al, N+1} + \al \|f\|_{\al-1, N} $$ 
	\end{ex}
	
	\begin{proof}
		Let $\al, N \in \N_0$. The previous exercise implies that  
		\begin{align*}
			\|g\|_{\al, N}
			&= \sup_{x \in \R}\bigg[ (1 + |x|)^N|\p^{\al} xf(x)| \bigg] \\
			& = \sup_{x \in \R}\bigg[ (1 + |x|)^N|x\p^{\al}f(x) + \al \p^{\al -1} f(x)| \bigg] \\
			& \leq \sup_{x \in \R}\bigg[ (1 + |x|)^{N+1}|\p^{\al}f(x)| \bigg] + \al \sup_{x \in \R}\bigg[  (1 + |x|)^N |\p^{\al-1} f(x)| \bigg] \\
			&= \|f\|_{\al, N+1} + \al \|f\|_{\al-1, N}
		\end{align*}
		Since $\al, N \in \N_0$ are arbitrary, $g \in \MS$.
	\end{proof}

	\begin{defn}
		We define the 
		\begin{itemize}
			\item \textbf{position operator}, denoted $X: \MS \rightarrow \MS$, by 
			$$Xf (x) = xf(x)$$
			\item \textbf{momentum operator}, denoted $D: \MS \rightarrow \MS$, by 
			$$Df (x) = -i\p f(x)$$
		\end{itemize} 
	\end{defn}

	\begin{ex}
		We have that 
		\begin{enumerate}
			\item 
			\begin{enumerate}
				\item $X: \MS \rightarrow \MS$ is linear
				\item $D: \MS \rightarrow \MS$ is linear
			\end{enumerate}
			\item 
			\begin{enumerate}
				\item $X: \MS \rightarrow \MS$ is continuous 
				\item $D: \MS \rightarrow \MS$ is continuous
			\end{enumerate}
		\end{enumerate}
	\end{ex}

	\begin{proof}\
		\begin{enumerate}
			\item Clear.
			\item Let $(f_n)_{n \in \N} \subset \MS$. Suppose that $f_n \rightarrow 0$. Then for each $\al, N \in \N_0$, $\|f_n\|_{\al, N} \rightarrow 0$.
			\begin{enumerate}
				\item A previous exercise implies that 
				\begin{align*}
					\|Xf_n\|_{\al, N} 
					& \leq \|f_n\|_{\al, N+1} + \al \|f_n\|_{\al-1, N} \\
					& \rightarrow 0
				\end{align*}
				So $Xf_n \rightarrow 0$ and $X$ is continuous at $0$. Since $X$ is linear, $X$ is continuous.
				\item A previous exercise implies that 
				\begin{align*}
					\|Df_n\|_{\al, N} 
					& = \|\p f_n\|_{\al, N} \\
					& \leq \|f_n\|_{\al+1, N}\\
					& \rightarrow 0
				\end{align*}
				So $Df_n \rightarrow 0$ and $D$ is continuous at $0$. Since $D$ is linear, $D$ is continuous. 
			\end{enumerate}
		\end{enumerate}
	\end{proof}

	\begin{defn}
		Let $f \in \MS$ and $y \in \R$. Then
		\begin{itemize}
			\item for each e$y \in \R$ we define the \textbf{translation of $f$ by $y$}, denoted $\tau_yf: \R \rightarrow \C$, by $\tau_yf(x) = f(x-y)$
			\item for each $\xi \in \R$, we define the \textbf{rotation of $f$} by $\xi$, denoted $\rho_{\xi} f: \R \rightarrow \C$ by $\rho_{\xi}f(x) = e^{-i\xi x}f(x)$
			\item for each $t \neq 0$, we define the \textbf{dilation of $f$} by $t$, denoted $\del_t f: \R \rightarrow \C$ by $\del_tf(x) = f(tx)$
		\end{itemize}
	\end{defn}

	\begin{ex}
		Let $f \in \MS$ and $\al \in \N_0$.  Then   
		\begin{enumerate}
			\item for each $y \in \R$, $\p^{\al}\tau_yf = \tau_y \p^{\al}f$ 
			\item for each $\xi \in \R$ ,
			\begin{align*}
				\p^{\al} \rho_{\xi} f 
				& =  \rho_{\xi} [(-i\xi + \p)^{\al}f] \\ 
				& =   \sum\limits_{k=0}^{\al} {\al \choose k} (-i\xi)^{\al-k}\rho_{\xi}\p^{k}f 
			\end{align*}
			\item for each $t \neq 0$, $\p^{\al} \del_tf = t^{\al} \del_t \p^{\al}f$
		\end{enumerate}
	\end{ex}

	\begin{proof}\
		\begin{enumerate}
			\item Clear by chain rule.
			\item Let $\xi \in \R$. The claim is clear for $\al = 0$ and $\al = 1$. Suppose that $\al >1$ and the claim is true for $\al-1$ so that
			$\p^{\al-1} \rho_{\xi} f =  \rho_{\xi} [(-i\xi + \p)^{\al-1}f]$. Set $g = (-i\xi + \p)^{\al-1}f$. Then 
			\begin{align*}
				\p^{\al} \rho_{\xi}f 
				& = \p [\p^{\al-1} \rho_{\xi} f ] \\
				& = \p \rho_{\xi} [(-i\xi + \p)^{\al-1}f] \\
				& = \p \rho_{\xi} g \\
				& = \rho_{\xi} [(-i\xi + \p)g] \\
				& = \rho_{\xi} [(-i\xi + \p)^{\al}f] 
			\end{align*}
		 	Since $-i \xi \id_{\MS}$ and $\p$ commute, the binomial theorem implies that 
		 	\begin{align*}
		 		\rho_{\xi}[(-i\xi + \p)^{\al}f] 
		 		& = \rho_{\xi}[\sum\limits_{k=0}^{\al} {\al \choose k}] (-i\xi)^{\al-k}\p^{k}f \\
		 		& = \sum\limits_{k=0}^{\al} {\al \choose k}] (-i\xi)^{\al-k}\rho_{\xi}\p^{k}f
		 	\end{align*}
	 		\item Clear by chain rule
		\end{enumerate}
	\end{proof}

	\begin{ex} Let $y \in \R$ and $t \neq 0$. Then
		\begin{enumerate}
			\item for each $x \in \R$,  $(1+|x|) \leq (1 + |y|)(1+ |x-y|)$
			\item there exists $C > 0$ such that for each $x \in \R$, $1+|x| \leq C(1 + |tx|)^2$ 
		\end{enumerate}
	\end{ex}
	
	\begin{proof}\
		\begin{enumerate}
			\item Let $x \in \R$. Then 
			\begin{align*}
				(1 + |y|)(1+ |x-y|) 
				& = 1 + |x-y| + |y| + |y||x-y| \\
				& \geq 1 + |x| + |y||x-y| \\
				& \geq 1 + |x| 
			\end{align*}
			\item Choose $C = \max(1/(2|t|), 1)$. Let $x \in \R$. Then 
			\begin{align*}
				C(1 + |tx|)^2 - (1 + |x|) 
				& = C + 2C|tx| + C(tx)^2 - 1 - |x| \\
				& = C + (2C|t| - 1)|x| + C(tx)^2 - 1 \\
				& = (C-1) + (2C|t| - 1)|x| + C(tx)^2 \\
				& \geq 0
			\end{align*}
			So $1 + |x| \leq C(1 + |tx|)^2$. 
		\end{enumerate}
	\end{proof}

	\begin{ex}
		Let $f \in \MS$. Then 
		\begin{enumerate}
			\item for each $y \in \R$, $\tau_y f \in \MS$ and for each $\al, N \in \N_0$, $\|\tau_yf\|_{\al, N} \leq (1 + |y|)^N \|f\|_{\al, N}$
			\item for each $\xi \in \R$, $\rho_{\xi} f \in \MS$ and for each $\al,N \in \N_0$, $$\|\rho_{\xi} f\|_{\al,N} \leq \sum\limits_{k=0}^{\al} {\al \choose k} |\xi|^{\al-k} \|f\|_{k, N}$$
			\item for each $t \neq 0$, $\del_t f \in \MS$ and $\kap_tf \in \MS$ and there exists $C_t > 0$ such that for each $\al,N \in \N_0$, $\|\del_t f\|_{\al, N} \leq |t|^{\al} C_t^N\|f\|_{\al, 2N}$
		\end{enumerate}
	\end{ex}

	\begin{proof}\
		\begin{enumerate}
			\item  Let $y \in \R$ and $\al,N \in \N_0$. Then 
			\begin{align*}
				\sup_{x \in \R} \bigg[ (1+|x|)^N |\p^{\al} \tau_yf(x)|\bigg] 
				& =  \sup_{x \in \R} \bigg[ (1+|x|)^N | \tau_y\p^{\al} f(x)|\bigg] \\ 
				& =  \sup_{x \in \R} \bigg[ (1+|x|)^N |\p^{\al} f(x - y)|\bigg] \\ 
				& \leq \sup_{x \in \R} \bigg[ (1+|y|)^N(1 + |x -y|)^N |\p^{\al} f(x - y)|\bigg] \\ 
				& = (1+|y|)^N\sup_{x \in \R} \bigg[ (1 + |x -y|)^N |\p^{\al} f(x - y)|\bigg] \\ 
				& = (1+|y|)^N\sup_{x \in \R} \bigg[ (1 + |x|)^N |\p^{\al} f(x)|\bigg] \\ 
				& = (1+|y|)^N \|f\|_{\al, N} \\ 
			\end{align*}
			\item Let $\xi \in \R$ and $\al, N \in \N_0$. Then for each $x \in \R$, we have that
			\begin{align*}
				(1 + |x|)^N |\p^{\al} \rho_{\xi} f(x)| 
				& = (1 + |x|)^N\bigg| \sum\limits_{k=0}^{\al} {\al \choose k} (-i\xi)^{\al-k}\rho_{\xi} \p^{k}f(x) \bigg| \\
				& = (1 + |x|)^N\bigg| \sum\limits_{k=0}^{\al} {\al \choose k} (-i\xi)^{\al-k} e^{-i\xi x}\p^{k}f(x) \bigg| \\
				& \leq (1 + |x|)^N \sum\limits_{k=0}^{\al} {\al \choose k} |\xi|^{\al-k} |\p^{k}f(x)| \\
				& = \sum\limits_{k=0}^{\al} {\al \choose k} |\xi|^{\al-k} (1 + |x|)^N|\p^{k}f(x)| \\
				& \leq \sum\limits_{k=0}^{\al} {\al \choose k} |\xi|^{\al-k} \|f\|_{k, N}
			\end{align*}
			Therefore,
			\begin{align*}
				\|\rho_{\xi}f\|_{\al,N} 
				& = \sup_{x \in \R} \bigg[ (1 + |x|)^N |\p^{\al} \rho_{\xi} f(x)|  \bigg] \\
				& \leq \sum\limits_{k=0}^{\al} {\al \choose k} |\xi|^{\al-k} \|f\|_{k, N}
			\end{align*}
			\item Let $t \neq 0$ and $\al, N \in \N_0$. The previous exercise implies that there exists $C_t > 0$ such that for each $x \in \R$, $1 + |x| \leq C_t(1 + |tx|)^2$. Then for each $x \in \R$, we have that 
			\begin{align*}
				(1 + |x|)^N |\p^{\al} \del_t f(x)| 
				& = (1 + |x|)^N |t|^{\al}|\del_t \p^{\al} f(x)| \\
				& = |t|^{\al} (1 + |x|)^N |\p^{\al} f(tx)| \\
				& \leq |t|^{\al} C_t^N(1 + |tx|)^{2N} | \p^{\al} f(tx)| \\
				& \leq |t|^{\al} C_t^N\|f\|_{\al, 2N}
			\end{align*}
			Therefore 
			\begin{align*}
				\|\del_tf\|_{\al, N} 
				& = \sup_{x \in \R} \bigg[(1 + |x|)^N |\p^{\al} \del_t f(x)|  \bigg] \\
				& \leq |t|^{\al} C_t^N\|f\|_{\al, 2N}
			\end{align*}
		\end{enumerate}
	\end{proof}

	\begin{ex} For each $y, \xi \in \R$, $t \neq 0$, we have that $\tau_y: \MS \rightarrow \MS$, $\rho_{\xi}: \MS \rightarrow \MS$ and $\del_t: \MS \rightarrow \MS$ are 
		\begin{enumerate}
			\item linear
			\item continuous
		\end{enumerate}
	\end{ex}

	\begin{proof} Let $y, \xi \in \R$ and $t \neq 0$.
		\begin{enumerate}
			\item Clear.
			\item Let $(f_n)_{n \in \N} \subset \MS$. Suppose that $f_n \rightarrow 0$. Then for each $\al,N \in \MN_0$, $\|f_n\|_{\al, N} \rightarrow 0$. 
			\begin{itemize}
				\item Let $\al,N \in \MN_0$. Then 
				\begin{align*}
					\|\tau_yf_n\|_{\al, N} 
					& \leq (1 + |y|)^N\|f_n\|_{\al, N} \\
					& \rightarrow 0
				\end{align*}
				Since $\al, N \in \N_0$ are arbitrary, $\tau_yf_n \rightarrow 0$. So $\tau_y$  is continuous at $0$. Since $\tau_y$ is linear, $\tau_y$ is continuous.
				\item Let $\al,N \in \MN_0$. Then 
				\begin{align*}
					\|\rho_{\xi}f_n\|_{\al, N} 
					& \leq \sum\limits_{k=0}^{\al} {\al \choose k} |\xi|^{\al-k} \|f_n\|_{k, N} \\
					& \rightarrow 0
				\end{align*}
				Since $\al, N \in \N_0$ are arbitrary, $\rho_{\xi}f_n \rightarrow 0$. So $\rho_{\xi}$  is continuous at $0$. Since $\rho_{\xi}$ is linear, $\rho_{\xi}$ is continuous.
				\item Let $\al,N \in \MN_0$. Define $C_t$ as in the previous exercise. Then 
				\begin{align*}
					\|\del_t f_n\|_{\al, N} 
					& \leq |t|^{\al} C_t^N \|f_n\|_{\al, 2N} \\
					& \rightarrow 0
				\end{align*}
				Since $\al, N \in \N_0$ are arbitrary, $\del_t f_n \rightarrow 0$. So $\del_t$  is continuous at $0$. Since $\del_t$ is linear, $\del_t$ is continuous.
			\end{itemize} 
		\end{enumerate}
	\end{proof}

	\begin{defn}
		Define  $\tau: \R \rightarrow \MS$ by $\tau f(y) = \tau_y f$
	\end{defn}

	\begin{ex}
		Let $f \in \MS$. Then $\tau f: \R \rightarrow \MS$ is a homomorphism
	\end{ex}

	\begin{note}
		 Let $f,g \in \MS$ and $x \in \R$, Define $h: \R \rightarrow \R$ defined by $h_x(y) = f(x-y)g(y)$. A previous exercise implies that $h_x = (\del_{-1} \tau_xf)g \in \MS \subset L^1(m)$ and for each $\al, N \in \N_0$, $\|h_x\|_{\al, N} \leq \sum_{\bet=0}^{\al} (1+|x|)^N\|f\|_{\bet, N} \|g\|_{\al - \bet, 0} $ \\
		 \textbf{FINISH FIX THIS!!!}
	\end{note}

	\begin{defn}
		Let $f, g \in \MS$. We define the \textbf{convolution of $f$ and $g$}, denoted $f * g: \R \rightarrow \R$ by $$f*g(x) = \int_{\R} f(x-y)g(y) dm(y)$$
	\end{defn}

	\begin{ex}
		Let $f,g \in \MS$. Then for each $\al \in \N_0$, $\p^{\al}(f*g) = (\p^{\al}f)*g$.
	\end{ex}

	\begin{proof}
			The claim is clear if $\al = 0$. Suppose that $\al > 0$ and that the claim is true for $\al - 1$ so that  $\p^{\al-1}(f*g) = (\p^{\al-1}f)*g$. Define $h: \R^2 \rightarrow \R$ by $h(x,y) = \p_x^{\al-1}f(x-y)g(y)$. Then for each $x,y \in \R$, 
			\begin{align*}
				|h(x,y)| 
				& = |\p_x^{\al-1}f(x-y)g(y)| \\
				& \leq \|\tau_yf\|_{\al-1,0}|g(y)| \\
				& \leq \|f\|_{\al-1,0}|g(y)|
			\end{align*} 
			Since $g \in L^1(m)$, we may differentiate under the integral to obtain that
		\begin{align*}
			[\p_x^{\al}(f*g)](x)
			& = \p_x [\p_x^{\al-1} (f*g)](x) \\
			& = \p_x [(\p_x^{\al-1}f)*g](x) \\
			& = \p_x \int_{\R} \p_x^{\al-1}f(x-y)g(y) \dm(y) \\
			& = \int_{\R} \p_x [\p_x^{\al-1}f(x-y)g(y)] \dm(y) \\
			& = \int_{\R} \p_x^{\al}f(x-y)g(y) \dm(y) \\
			& = [(\p_x^{\al}f)*g](x)
		\end{align*}
		So the claim is true for $\al$.
	\end{proof}

	\begin{ex}
		Let $f, g \in \MS$, then $f *g \in \MS$ and there exists $C >0$ such that for each $\al,N \in \N_0$, $\|f*g\|_{\al,N} \leq C\|f\|_{\al, N} \|g\|_{0, N+2}$. 
	\end{ex}

	\begin{proof}
		Set 
		$$C = \int_{\R} \frac{1}{(1+|y|)^2} \dm(y)$$
		Let $\al,N \in \N_0$ and $x \in \R$. Then 
		\begin{align*}
			(1+|x|)^N |\p^{\al} (f*g)(x)|
			& =   (1+|x|)^N |(\p^{\al}f )*g(x)| \\
			& =   (1+|x|)^N \bigg| \int_{\R}\p^{\al}f(x-y) g(y) \dm(y) \bigg|\\
			& \leq   \int_{\R}(1+|x|)^N|\p^{\al}f(x-y) g(y)| \dm(y) \\
			& \leq   \int_{\R}(1+|y|)^N(1+|x-y|)^N|\p^{\al}f(x-y)| |g(y)| \dm(y) \\
			& \leq  \|f\|_{\al, N} \int_{\R}(1+|y|)^N |g(y)| \dm(y) \\
			& =  \|f\|_{\al, N} \int_{\R}(1+|y|)^{N+2} \frac{|g(y)|}{(1+|y|)^2} \dm(y) \\
			& \leq \|f\|_{\al, N} \|g\|_{0, N+2} \int_{\R} \frac{1}{(1+|y|)^2} \dm(y) \\
			& =  C\|f\|_{\al, N} \|g\|_{0, N+2}
		\end{align*}
		Since $x \in \R$ is arbitrary, we have that 
		\begin{align*}
			\|f*g\|_{\al, N}
			& = \sup_{x \in \R} \bigg[ (1+|x|)^N |\p^{\al} (f*g)(x)| \bigg] \\
			& \leq C\|f\|_{\al, N} \|g\|_{0, N+2}
		\end{align*}
	\end{proof}

	\begin{ex}
		The convolution $*: \MS \times \MS \rightarrow \MS$ 
		\begin{enumerate}
			\item is bilinear
			\item is continuous
		\end{enumerate}
	\end{ex}

	\begin{proof}\
		\begin{enumerate}
			\item Clear.
			\item Let $(f_n,g_n)_{n \in \N} \subset \MS \times \MS$ and $(f,g) \in \MS \times \MS$. Suppose that $(f_n, g_n) \rightarrow (f,g)$. Then $f_n \rightarrow f$ and $g_n \rightarrow g$. Hence for each $\al, N \in \N_0$, $\|f_n - f\|_{\al, N} \rightarrow 0$ and $\|g_n - g\|_{\al, N} \rightarrow 0$. In particular 
			\begin{align*}
				\bigg|\|g_n\|_{0,N+2} - \|g\|_{0, N+2} \bigg| 
				& \leq \|g_n - g\|_{0, N+2}  \\
				& \rightarrow 0
			\end{align*}
			So that $(\|g_n\|_{0,N+2})_{n \in \N}$ is bounded.
			Let $\al,N \in \N_0$. Define $C >0$ as in the previous exercise. Then 
			\begin{align*}
				\|f_n *g_n - f*g\|_{\al,N} 
				& = \|f_n *g_n - f*g_n + f*g_n - f*g\|_{\al,N} \\ 
				& \leq \|(f_n -f )*g_n\|_{\al,N} + \|f_ *(g_n -g)\|_{\al,N} \\
				& \leq C\|f_n -f\|_{\al,N}\|g_n\|_{0, N+2} + C\|f\|_{\al,N}\|g_n-g\|_{0, N+2} \\
				& \rightarrow 0
			\end{align*}
			Since $\al,N \in \N_0$ are arbitrary, $f_n *g_n \rightarrow f*g$. Thus $*:\MS \times \MS \rightarrow \MS$ is continuous.
		\end{enumerate}
	\end{proof}

	\begin{ex}
		Let $f,g \in \MS$. Then $\|f*g\|_1 \leq \|f\|_1\|g\|_1$.
	\end{ex}

	\begin{proof}
		Tonelli's theorem implies that 
		\begin{align*}
			\|f*g\|_1
			& = \int_{\R} |f*g(x)| \dm(x) \\
			& = \int_{\R}  \bigg|\int_{\R} f(x-y)g(y) \dm(y) \bigg|  \dm(x) \\
			& \leq  \int_{\R} \bigg[ \int_{\R} |f(x-y)g(y)| \dm(y) \bigg] \dm(x) \\
			& = \int_{\R} \bigg[ \int_{\R} |f(x-y)g(y)| \dm(x) \bigg] \dm(y) \\
			& = \int_{\R} \bigg[ \int_{\R} |f(x-y)| \dm(x) \bigg] |g(y)| \dm(y) \\
			& = \|f\|_1 \int_{\R} |g(y)| \dm(y) \\
			& = \|f\|_1\|g\|_1
		\end{align*}
	\end{proof}

	\begin{ex}
		Let $f, g \in \MS$, then $f*g = g*f$.
	\end{ex}

	\begin{proof}
		Let $x \in R$. Define $a,b:\R \rightarrow \R$ by $a(z) = f(z)g(x-z)$ and $b(y) = x-y$. Then for each $A \in \MB(\R)$, 
		\begin{align*}
			b_*m(A)
			& = m(b^{-1}(A)) \\
			& = m(x - A) \\
			& = m(A)
		\end{align*}
		So $b_*m = m$ and  
		\begin{align*}
			f*g(x)
			& = \int_{\R} f(x-y)g(y) \dm(y) \\
			& = \int_{b^{-1}(\R)} a \circ b \dm \\
			& = \int_{\R} a \, d b_*m \\
			& = \int_{\R} a \dm \\
			& = \int_{\R} g(x-z) f(z) \dm(z) \\
			& = g*f(x)
		\end{align*}
		Since $x \in \R$ is arbitrary, $f*g = g*f$.
	\end{proof}

	\begin{defn}
		We define the \textbf{bump functions} on $\R$, denoted $C_c^{\infty}(\R)$,  by $$C_c^{\infty}(\R) = C_c(\R) \cap C^{\infty}(\R)$$
	\end{defn}

	\begin{ex}
		Let $f \in C_c^{\infty}(\R)$. Then $f \in \MS$. 
	\end{ex}

	\begin{proof}
		Let $\al,N \in \N^0$. Define $g: \R \rightarrow \C$ by 
		$$g(x) = (1+|x|)^N |\p^{\al}f(x)|$$ 
		Then $g$ is continuous. Since $\supp(\p^{\al}f) \subset \supp(f)$, we have that $g \in C_c(\R)$ and
		\begin{align*}
			\sup_{x \in \R} \bigg[ (1+|x|)^N |\p^{\al}f|\bigg] 
			& = \sup_{x \in \R} g(x) \\
			& = \|g\| \\
			& < \infty
		\end{align*}
	\end{proof}

	\begin{ex}
		Define $f:\R \rightarrow \R$ by $f(x) = e^{-x^2}$. Then $f \in \MS$.
	\end{ex}
	
	\begin{proof}
		meh...
	\end{proof}
	

	\begin{ex}
		Define $f:\R \rightarrow \R$ by 
		\[
		f(x) = 
		\begin{cases}
			e^{- \frac{1}{1-x^2}} & x \in (-1, 1) \\
			0 & x \not \in (-1, 1)
		\end{cases}
		\]
		Then $f \in \MS$.
	\end{ex}
	
	\begin{proof}
		meh...
	\end{proof}

	\begin{ex}
		Let $a,b \in \R$. Suppose that $a < b$. Then for each $\ep >0$, there exists $f \in \MS$ such that $\chi_{[a,b]} \leq f \leq \chi_{[a-\ep , b + \ep]}$.
	\end{ex}

	\begin{proof}
		Set $f(x) = $
	\end{proof}

	\begin{ex}
		Let $f \in \MS$. Define
	\end{ex}
	
	
	
	
	
	
	
	
	
	
	
	
	
	
	
	
	
	
	\newpage
	\subsection{The Fourier Transform on $\MS$}
	
	\begin{ex}
		\lex{300} Let $\phi:\R \rightarrow S^1$ be a measurable homomorphism. 
		\begin{enumerate}
			\item Then $\phi \in L^1_{\loc}(\R)$ and there exists $a > 0$ such that $$\int_{(0,a]}\phi dm \neq 0$$
			\item Define $$c = \bigg[ \int_{(0,a]}\phi dm \bigg]^{-1}$$ 
			Then  For each $x \in \R$, $$\phi(x) = c\int_{(x, x+a]}\phi dm$$ 
			\item $\phi \in C^{\infty}(\R)$ and $\phi' = c(\phi(a) - 1)\phi$
			\item Define $b = c(\phi(a) - 1)$ and $g \in C^{\infty}(\R)$ by $g(x) = e^{-bx} \phi(x)$. Then $g$ is constant and there exists $\xi \in \R$ such that for each $x \in \R$, $\phi(x) = e^{2 \pi i \xi x}$.
		\end{enumerate}
	\end{ex}	
	
	\begin{proof}\
		\begin{enumerate}
			\item Let $K \subset \R$ be compact. Then $$\int_K |\phi| dm = m(K) < \infty$$ So $\phi \in L^1_{\loc}(\R)$. For the sake of contradiction, suppose that for each $a >0$, $$\int_{(0,a]}\phi dm = 0$$ 
			Then the FTC implies that $\phi = 0$ a.e. on $\Rg$, which is a contradiction. So there exists $a > 0$ such that $$\int_{(0,a]}\phi dm \neq 0$$
			\item For $x \in \R$, 
			\begin{align*}
				\phi(x) 
				&= c \int_{(0,a]} \phi(x)\phi(t) dm(t) \\
				&= c \int_{(0,a]} \phi(x+t) dm(t) \\
				&= c \int_{(x,x+a]} \phi dm 
			\end{align*}
			\item Part $(2)$ and the FTC imply that $\phi$ is continuous. Let $d \in \R$. Define $f_d \in C((d, \infty))$ by $$f_d(x) = \int_{(d, x]} \phi dm$$ 
			Since $\phi$ is continuous, the FTC implies that $f_d$ is differentiable and for each $x >d$ $f_d'(x) = \phi(x)$. Part $(2)$ implies that for each $x > d$,
			\begin{align*}
				\phi(x) 
				&= c \int_{(x,x+a]} \phi dm \\
				&= c(f_d(x+a) - f_d(x))
			\end{align*}
			So for each $x > d$, $\phi$ is differentiable at $x$ and 
			\begin{align*}
				\phi'(x) 
				&= c(\phi(x+a) - \phi(x)) \\
				&= c(\phi(a) - 1) \phi(x)
			\end{align*}	 
			Since $d \in \R$ is arbitrary, $\phi$ is differentiable and $\phi' = c(\phi(a) - 1) \phi$. This implies that $\phi \in C^{\infty}(\R)$.
			\item Let $x \in \R$. Then 
			\begin{align*}
				g'(x) 
				&= e^{-bx}\phi'(x) - be^{-bx}\phi(x) \\
				&= be^{-bx} \phi(x) - be^{-bx}\phi(x) \\
				&= 0
			\end{align*}
			So $g' = 0$ and $g$ is constant. Hence there exists $k \in \R$ such that for each $x \in \R$, $\phi(x) = ke^{bx}$. Since $\phi(0) = 1$, $k = 1$. Since $|\phi| = 1$, there exists $\xi \in \R$ such that $b = 2 \pi i \xi$. 
		\end{enumerate}
	\end{proof}
	
	\begin{note}
		To summarize, for each measurable homomorphism $\phi:\R \rightarrow S^1$, there exists $\xi \in \R$ such  such that for each $x \in \R$, $\phi(x) = e^{2 \pi i  \xi x}$. 
	\end{note}
	
	\begin{defn}
		Let $f \in \MS$. We define the \textbf{Fourier transform of $f$}, denoted $\hat{f} : \R \rightarrow \C$, by $$ \hat{f}(\xi) = \int_{\R} e^{-i \xi x}f(x) \, dm(x)$$ 
	\end{defn}

	\begin{ex}
		Let $f \in \MS$. Then $\hat{f} \in C_b(\R)$.
	\end{ex}

	\begin{proof}
		Since $f \in \MS$, $f \in L^1(m)$. Then for each $\xi \in \R$,
		\begin{align*}
			|\hat{f}(\xi)| 
			& = \bigg| \int_{\R} e^{-i\xi x}f(x) \dm(x) \bigg| \\
			& \leq \int_{\R} |e^{-i\xi x}f(x)| \dm(x) \\
			& = \int_{\R} |f(x)| \dm(x) \\
			& = \|f\|_1
		\end{align*}
		So $f$ is bounded. Let $(\xi_n)_{n \in \N} \subset \R$ and $\xi \in \R$. Suppose that $\xi_n \rightarrow \xi$. Define $(\phi_n)_{n \in \N} \subset L^1(m)$ and $\phi \in L^1(m)$ by $\phi_n(x) = e^{-i\xi_n x}f(x)$ and $\phi(x) = e^{-i\xi x}f(x)$. Then $\phi_n \convt{p.w.} \phi$ and for each $n \in \N$, 
		\begin{align*}
			|\phi_n|
			& = |f| \\
			& \in L^1(m)
		\end{align*}
		The dominated convergence theorem implies that 
		\begin{align*}
			\hat{f}(\xi_n)
			& = \int_{\R} e^{-i\xi_n x}f(x) \dm(x) \\
			& = \int_{\R} \phi_n \dm \\
			& \rightarrow \int_{\R} \phi \dm \\
			& = \int_{\R} e^{-i\xi x}f(x) \dm(x) \\
			& = \hat{f}(\xi)
		\end{align*}
		So $\hat{f}$ is continuous. Hence $\hat{f} \in C_b(\R)$.
	\end{proof}

	\begin{defn}
		We define the \textbf{Fourier transform on $\MS$}, denoted $\MF: \MS \rightarrow C_b(\R)$, by 
		$$\MF(f) = \hat{f}$$
	\end{defn}
	
	\begin{ex}
		We have that $\MF: \MS \rightarrow C_b(\R)$ is linear. 
	\end{ex}
	
	\begin{proof}
		Let $f,g \in \MS$ and $\lam \in \C$. Then 
		\begin{align*}
			\MF(f + \lam g) 
			& = \int_{\R} e^{-i \xi x} [f(x) + \lam g(x)] \dm(x) \\
			& = \int_{\R} e^{-i \xi x} f(x) + \lam e^{-i \xi x} g(x) \dm(x) \\
			& = \int_{\R} e^{-i \xi x} f(x) \dm(x) + \lam \int_{\R} e^{-i \xi x} g(x) \dm(x) \\
			&= \MF(f) + \lam \MF(g)
		\end{align*}
	\end{proof}

	\begin{ex}
		Let $f \in \MS$ and $\al \in \N^0$. Then 
		\begin{enumerate}
			\item $\MF(X^{\al}f) = (-1)^{\al}D^{\al} \MF(f)$ 
			\item $\MF(D^{\al}f) = X^{\al} \MF(f)$
		\end{enumerate}
	\end{ex}
	
	\begin{proof}\
		\begin{enumerate}
			\item The claim is clear for $\al = 0$. Suppose that $\al > 0$ and that the claim is true for $\al -1$ so that $\MF(X^{\al-1}f) = (-1)^{\al-1}D^{\al-1} \MF(f)$. Define $\phi:\R \times \R \rightarrow \R$ by $\phi(\xi, x) = e^{-i\xi x} x^{\al - 1}f(x)$. Then for each $\xi, x \in \R$, 
			\begin{align*}
				|\p_{\xi} \phi(\xi, x)|
				& = |-ix e^{-i\xi x} x^{\al - 1}f(x)| \\
				& = |x^{\al}f(x)| \\
				& = |(X^{\al}f)(x)|
			\end{align*}
			Since $X^{\al}f \in \MS \subset L^1$, we may switch the order of differentiation and integration to obtain 
			\begin{align*}
				\MF(X^{\al}f) (\xi)
				& = \int_{\R} e^{-i\xi x}x^{\al}f(x) dm(x) \\
				& = \int_{\R} i \p_{\xi}\bigg[ e^{-i\xi x}x^{\al-1}f(x) \bigg] dm(x) \\
				& = i \p_{\xi} \bigg[  \int_{\R} e^{-i\xi x}x^{\al-1}f(x) dm(x) \bigg] \\
				& = i \p_{\xi} \MF(X^{\al-1}f) (\xi) \\
				& = -D \MF(X^{\al-1}f)(\xi) \\
				& = (-1)^{\al}D^{\al} \MF(f) (\xi)
			\end{align*}
			So the claim is true for $\al$.
			\item The claim is clear for $\al = 0$. Suppose that $\al > 0$ and that the claim is true for $\al -1$ so that $\MF(D^{\al-1}f) = X^{\al-1} \MF(f)$. Then integration by parts yields 
			\begin{align*}
				\MF(D^{\al}f)(\xi)
				& = \int_{\R} e^{-i \xi x} [-i \p_x D^{\al-1}f(x)] \dm(x) \\
				&= - \int_{\R} -i \xi e^{-i \xi x} [-iD^{\al-1}f(x)] \dm(x) \\
				&= \xi \int_{\R}   e^{-i \xi x} D^{\al-1}f(x) \dm(x) \\
				&= X \MF(D^{\al-1}f)(\xi) \\
				&= X^{\al} \MF(f)(\xi) \\
			\end{align*}
			So the claim is true for $\al$.  
		\end{enumerate}
	\end{proof}

	\begin{ex}
		Let $P()$
	\end{ex}

	\begin{proof}
		content...
	\end{proof}

	\begin{ex}
		There exists $C >0$ such that for each $f \in \MS$, $\|\hat{f}\|_{0,0} \leq C \|f\|_{0, 2}$.\\
		\textbf{Hint:} Set $$C = \int_{\R} \frac{1}{(1+|x|)^2} \dm(x)$$
	\end{ex}
	
	\begin{proof}
		Set 
		$$C = \int_{\R} \frac{1}{(1+|x|)^2} \dm(x)$$
		Let $f \in \MS$. Let $\xi \in \R$. Then 
		\begin{align*}
			|\hat{f}(\xi)| 
			& = \bigg| \int_{\R} e^{-i\xi x} f(x) \dm(x) \bigg| \\
			& \leq  \int_{\R} | f(x)| \dm(x) \\
			& =  \int_{\R} \frac{(1+|x|)^2|f(x)|}{(1+|x|)^2} \dm(x) \\
			& \leq \|f\|_{0, 2} \int_{\R} \frac{1}{(1+|x|)^2} \dm(x) \\
			& = C\|f\|_{0, 2}
		\end{align*}
		Since $\xi \in \R$ is arbitrary, $\|\hat{f}\|_{0,0} \leq \|f\|_{0, 2}$.
	\end{proof}

	\begin{ex}
		Let $a, b \in \R$ and $N \in \N_0$. Then $(a + b)^N \leq 2^{N-1} (a^N + b^N)$. \\
		\textbf{Hint:} Jensen's inequality
	\end{ex}
	
	\begin{proof}
		Jensen's inequality implies that 
		\begin{align*}
			2^{-N}(a + b)^N 
			& = \bigg(\frac{a}{2} + \frac{b}{2} \bigg)^N \\
			& \leq \bigg(\frac{a^N}{2} + \frac{b^N}{2} \bigg) \\
			& = 2^{-1}(a^N + b^N)
		\end{align*}
	So $(a + b)^N \leq 2^{N-1} (a^N + b^N)$.
	\end{proof}

	\begin{ex}
		We have that $ \MF(\MS) \subset \MS$ and $\MF: \MS \rightarrow \MS$ is continuous. 
	\end{ex}

	\begin{proof}
		Let $f \in \MS$ and $\al, N \in \N_0$. Then the previous exercise implies that for each $\xi \in \R$,
		\begin{align*}
			\xi^{N}\p_{\xi}^{\al} \MF(f)(\xi)
			& =  (-i)^{\al} X^{N} D^{\al} \MF(f)(\xi) \\
			&= i^{\al} X^N \MF(X^{\al} f)(\xi) \\
			&= i^{\al} \MF(D^{N} X^{\al} f)(\xi) \\ 
		\end{align*}
		Set 
		$$C = \int_{\R} \frac{1}{(1+|x|)^2} \dm(x)$$ 
		as in the previous exercise. Since $\MF(X^{\al}f)$, $\MF(D^NX^{\al}f) \in C_b(\R)$, we have that
		\begin{align*}
			\|\MF(f)\|_{\al, N}
			& = \sup_{\xi \in \R} \bigg[ (1 + |\xi|)^N |\p_{\xi}^{\al} \MF(f)(\xi)|\bigg] \\
			& \leq \sup_{\xi \in \R} \bigg[ 2^{N-1}(1 + |\xi|^N) |\p_{\xi}^{\al} \MF(f)(\xi)| \bigg] \\
			& = \sup_{\xi \in \R} \bigg[ |2^{N-1} \p_{\xi}^{\al} \MF(f)(\xi)| + |2^{N-1}\xi^N \p_{\xi}^{\al} \MF(f)(\xi)| \bigg] \\
			& = \sup_{\xi \in \R} \bigg[  |\MF( 2^{N-1} X^{\al} f)(\xi)| + |\MF(2^{N-1} D^N X^{\al} f)(\xi)| \bigg] \\
			& \leq  \| \MF( 2^{N-1} X^{\al} f) \|_{0,0} + \| \MF(2^{N-1} D^N X^{\al} f)\|_{0,0} \\
			& \leq C2^{N-1}\| X^{\al} f \|_{0,2} +  C2^{N-1}\|  D^N X^{\al} f\|_{0,2}
			& < \infty
		\end{align*}
		Since $\al, N \in \N_0$ are arbitrary, $\MF(f) \in \MS$ and since $f \in \MS$ is arbitrary, $\MF(\MS) \subset \MS$. Let $(f_n)_{n \in \N} \subset \MS$. Suppose that $f_n \rightarrow 0$. Since $X,D: \MS \rightarrow \MS$ are continuous, $X^{\al}f_n \rightarrow 0$ and $D^NX^\al f_n \rightarrow 0$. Therefore, $\|X^{\al}f_n\|_{0, 2} \rightarrow 0$ and $\|D^NX^{\al}f_n\|_{0, 2} \rightarrow 0$. From above, we see that 
		\begin{align*}
			\|\MF(f_n)\|_{\al, N} 
			& \leq C2^{N-1}\| X^{\al} f_n \|_{0,2} +  C2^{N-1}\|  D^N X^{\al} f_n \|_{0,2} \\
			& \rightarrow 0
		\end{align*}
		Hence $\MF(f_n) \rightarrow 0$ and $\MF$ is continuous. 
	\end{proof}

	\begin{ex}
		Let $f \in \MS$. Then 
		\begin{enumerate}
			\item for each $y \in \R$, $\MF(\tau_yf) = \rho_{y} \MF(f)$ 
			\item for each $\eta \in \R$, $\MF(\rho_{\eta} f) = \tau_{-\eta} \MF(f)$
			\item $\MF(\del_t f) = t^{-1} \del_{t^{-1}} \MF(f)$
		\end{enumerate}
	\end{ex}

	\begin{proof}\
		\begin{enumerate}
			\item Let $y, \xi \in \R$. Then 
			\begin{align*}
				\MF(\tau_yf)(\xi) 
				& = \int_{\R} e^{-i\xi x} f(x-y) \dm(x) \\
				& = \int_{\R} e^{-i\xi (z+y)} f(z) \dm(z) \\
				& = e^{-i\xi y} \int_{\R} e^{-i\xi z} f(z) \dm(z) \\
				& = e^{-i\xi y} \MF(f)(\xi) \\
				& = \rho_{y} \MF(f)(\xi)
			\end{align*}
			\item Let $\eta, \xi \in \R$. Then 
			\begin{align*}
				\MF(\rho_{\eta}f)(\xi) 
				& = \int_{\R} e^{-i\xi x} e^{-i\eta x}f(x) \dm(x) \\
				& = \int_{\R} e^{-i(\xi + \eta)x} f(x) \dm(x) \\
				& = \MF(f)(\xi + \eta) \\
				&= \tau_{-\eta}\MF(f)(\xi)
			\end{align*}
			\item Let $\xi \in \R$. Then 
			\begin{align*}
				\MF(\del_t f)(\xi) 
				& = \int_{\R} e^{-i\xi x} f(tx) \dm(x) \\
				& = \int_{\R} e^{-i\xi t^{-1} z} f(z) t^{-1}\dm(z) \\
				& = t^{-1}\MF(f)(t^{-1} \xi) \\
				& = t^{-1} \del_{t^{-1}} \MF(f)(\xi)
			\end{align*}
		\end{enumerate}
	\end{proof}

	\begin{ex}
		Let $f,g \in \MS$. Then $\MF(f*g) = \MF(f)\MF(g)$.
	\end{ex}

	\begin{proof}
		Let $\xi \in \R$. Tonelli's theorem implies that  
		\begin{align*}
			\int_{\R} \bigg[ \int_{\R} | e^{-i\xi x} f(x-y)g(y)| \dm(y) \bigg] \dm(x)
			& = \int_{\R}  \bigg[ \int_{\R}  |f(x-y)g(y) |\dm(y) \bigg] \dm(x) \\
			& = \int_{\R}  \bigg[ \int_{\R}  |f(x-y)g(y) |\dm(x) \bigg] \dm(y) \\
			& = \int_{\R}  \bigg[ \int_{\R}  |f(x-y) |\dm(x) \bigg] |g(y)| \dm(y) \\
			& = \|f\|_1 \int_{\R} |g(y)| \dm(y) \\
			& = \|f\|_1\|g\|_1
		\end{align*}
		So we may apply Fubini's theorem and change the order of integration to obtain that
		\begin{align*}
			\MF(f*g)(\xi)
			& = \int_{\R} e^{-i\xi x} (f*g)(x) \dm(x) \\
			& = \int_{\R}  \bigg[ \int_{\R} e^{-i\xi x} f(x-y)g(y) \dm(y) \bigg] \dm(x) \\
			& = \int_{\R}  \bigg[ \int_{\R} e^{-i\xi x} f(x-y)g(y) \dm(x) \bigg] \dm(y) \\
			& = \int_{\R}  \bigg[ \int_{\R} e^{-i\xi x} f(x-y) \dm(x) \bigg] g(y) \dm(y) \\
			& = \int_{\R}  [\MF(\tau_yf)(\xi) ] g(y) \dm(y) \\
			& = \int_{\R}  [e^{-i \xi y}\MF(f)(\xi) ] g(y) \dm(y) \\
			& = \MF(f)(\xi) \int_{\R}  e^{-i \xi y}  g(y) \dm(y) \\
			& = \MF(f)(\xi) \MF(g)(\xi)
		\end{align*}
		Since $\xi \in \R$ is arbitrary, $\MF(f*g) = \MF(f) \MF(g)$
	\end{proof}

	\begin{ex}
		Let $f,g \in \MS$. Then $$\int_{\R}\hat{f} g \dm = \int_{\R}f \hat{g}  \dm$$
	\end{ex}
	
	\begin{proof}
		Tonelli's theorem implies that  
		\begin{align*}
			\int_{\R} \bigg[ \int_{\R}|e^{-i \xi x} f(x) g(\xi)| \dm(x) \bigg] \dm(\xi)
			& = \int_{\R} \bigg[ \int_{\R}|f(x)| \dm(x) \bigg]  |g(\xi)| \dm(\xi) \\
			& = \|f\|_1 \int_{\R}  |g(\xi)| \dm(\xi) \\
			& = \|f\|_1\|g\|_1
		\end{align*}  
		So we may apply Fubini's theorem and switch the order of integration to obtain that
		\begin{align*}
			\int_{\R}\hat{f} g \dm 
			& = \int_{\R} \bigg[ \int_{\R}e^{-i \xi x} f(x) \dm(x) \bigg]  g(\xi) \dm(\xi) \\
			& = \int_{\R} \bigg[ \int_{\R}e^{-i \xi x} f(x) g(\xi) \dm(x) \bigg] \dm(\xi) \\
			& = \int_{\R} \bigg[ \int_{\R}e^{-i \xi x} f(x) g(\xi) \dm(\xi) \bigg] \dm(x) \\
			& = \int_{\R} f(x) \bigg[ \int_{\R}e^{-i \xi x}  g(\xi) \dm(\xi) \bigg] \dm(x) \\
			& = \int_{\R} f(x) \hat{g}(x) \dm(x) \\
			& = \int_{\R} f \hat{g} \dm \\
		\end{align*}
	\end{proof}


	\begin{ex}
		Define $f \in \MS$ by $f(x) = e^{-x^2/2}$. Then $\MF(f) = \sqrt{2 \pi}f$.
	\end{ex}

	\begin{proof}
		Note that for each $\xi \in \R$, 
		\begin{align*}
			\MF(Df)(\xi) 
			& = \int_{\R} e^{-i \xi x}ixe^{-x^2/2} \dm(x) \\
			& = -\int_{\R}  \p_{\xi} \bigg[ e^{-i \xi x} e^{-x^2/2}\bigg] \dm(x) \\
			& = - \p_{\xi} \MF(f)(\xi) 
		\end{align*}
		A previous exercise implies that $\MF(Df) = X \MF(f)$. So for each $\xi \in \R$, $\p_{\xi} \hat{f}(\xi) = - \xi \hat{f}(\xi)$. Define $g \in \C^{\infty}(\R)$ by $g(\xi) = e^{\xi^2/2}$. Then 
		\begin{align*}
			\p_{\xi} (\hat{f} g) 
			& = (\p_{\xi} \hat{f}) g + \hat{f} (\p_{\xi}g) \\
			& = 0
		\end{align*}
		So there exists $C \in \R$ such that $\hat{f}g = C$. Hence for each $\xi \in \R$, 
		\begin{align*}
			\hat{f}(\xi) 
			& = Ce^{-\xi^2/2} \\
			& = Cf(\xi)
		\end{align*}
		Therefore, 
		\begin{align*}
			C
			& = Cf(0) \\
			& = \hat{f}(0) \\
			& = \int_{\R} e^{-x^2/2} \dm(x) \\
			& = \sqrt{2 \pi} 
		\end{align*}
		So $\hat{f} = \sqrt{2 \pi}f$.
	\end{proof}

	\begin{ex}
		Let $f \in \MS$. Define $g: \R \rightarrow L^1$ by $g(x) = \tau_x f$. Then $g$ is continuous. \\
		\textbf{Hint:} approximate by functions in $C_c(\R)$.
	\end{ex}

	\begin{proof}
		Suppose that $f \in C_c(\R)$. Then
	\end{proof}

	\begin{defn}
		Let $f \in \MS$ and $t \neq 0$. We define $f_t \in \MS$ by $f_t = t^{-1} \del_{t^{-1}} f$.   
	\end{defn}

	\begin{ex}
		Let $\phi \in \MS$ and $t \neq 0$. Then 
		$$\int_{\R}\phi_t \dm = \int_{\R} \phi \dm$$ 
	\end{ex}

	\begin{proof}
		We have that 
		\begin{align*}
			\int_{\R}\phi_t \dm 
			& = \int_{\R}t^{-1}\phi(t^{-1}x) \dm(x) \\
			& = \int_{\R} \phi(z) \dm(z) \\
			& = \int_{\R}\phi \dm 
		\end{align*}
	\end{proof}

	\begin{ex}
		Let $\phi \in \MS$. Set 
		$$\al = \int_{\R} \phi \dm$$ 
		Then for each $f \in \MS$,  $f * \phi_{1/n} \conv{L^1} \al f$. \\
		\textbf{Hint:} for each $t \neq 0$ and $x \in \R$, 
		$$f * \phi_t(x) - \al f(x) = \int_{\R} [\tau_{tz}f(x)  -  f(x)] \phi(z) \dm(z) $$
	\end{ex}

	\begin{proof}
		Let $t \neq 0$ and $x \in \R$. The previous exercise implies that 
		\begin{align*}
			f * \phi_t(x) - \al f(x) 
			& = \int_{\R} f(x-y) \phi_t(y) \dm(y) - \int_{\R} \phi(y) \dm(y) f(x) \\
			& = \int_{\R} f(x-y) \phi_t(y) \dm(y) - \int_{\R} \phi_t(y) \dm(y) f(x) \\
			& = \int_{\R} f(x-y) \phi_t(y)  -  f(x) \phi_t(y) \dm(y) \\
			& = \int_{\R} [f(x-y)  -  f(x)] \phi_t(y) \dm(y) \\
			& = \int_{\R} [f(x-y)  -  f(x)] t^{-1}\phi(t^{-1}y) \dm(y) \\
			& = \int_{\R} [f(x-tz)  -  f(x)] \phi(z) \dm(z) \\
			& = \int_{\R} [\tau_{tz}f(x) -  f(x)] \phi(z) \dm(z)   
		\end{align*}
		Tonelli's theorem implies that 
		\begin{align*}
			\|f * \phi_t - \al f \|_1
			& = \int_{\R}|f * \phi_t(x) - \al f(x) | \dm(x) \\
			& \leq \int_{\R} \bigg[ \int_{\R} |\tau_{tz}f(x)  -  f(x)| |\phi(z)| \dm(z) \bigg] \dm(x) \\
			& =  \int_{\R} \bigg[ \int_{\R} |\tau_{tz}f(x)  -  f(x)| |\phi(z)| \dm(x) \bigg] \dm(z) \\
			& = \int_{\R} \|\tau_{tz}f - f\|_1|\phi(z)| \dm(z)
		\end{align*}
		For $n \in \N$, define $g_n \in \MS$ by $g_n(z) = \|\tau_{n^{-1}z}f(x)  -  f(x)\|_1 \phi(z)$. Then $g_n \convt{p.w.} 0$ and 
		\begin{align*}
			|g_n| 
			& \leq 2\|f\|_1|\phi| \\
			& \in L^1(m)
		\end{align*} 
		The dominated convergence theorem implies that 
		
	\end{proof}

	
	


	\begin{defn}
		content...
	\end{defn}
	
	
	
	
	
	
	
	
	
	
	
	
	
	
	
	
	
	
	
	
	\newpage
	\subsection{The Fourier Transform on $\MM(\R)$}
	
	\begin{note}
		Recall that $$\MM(\R) = \{\mu: \MB(\R) \rightarrow \C: \mu \text{ is a complex measure}\}$$
	\end{note}
	
	\begin{defn}
		Let $\mu \in \MM(\R)$. We define the \textbf{Fourier transform of $\mu$}, denoted $\hat{\mu}: \R \rightarrow \C$, by
		$$\hat{\mu}(\xi) = \int_{\R} e^{-i \xi x} \dmu(x)$$ 
	\end{defn}
	
	\begin{ex}
		Let $\mu \in \MM(\R)$. Then Then $\hat{\mu} : \R \rightarrow \C$ is bounded.
	\end{ex}
	
	\begin{proof}
		Let $\xi \in \R$. 
		\begin{align*}
			|\hat{\mu}(\xi)|
			& = \bigg | \int_{\R} e^{-i \xi x} \dmu(x) \bigg| \\
			& \leq \int_{\R} |e^{-i \xi x}| \, d|\mu|(x) \\
			& = |\mu|(\R) \\
		\end{align*}
		So $\hat{\mu}$ is bounded.
	\end{proof}
	
	\begin{ex}
		Let $\mu \in \MM(\R)$. Then $\hat{\mu} \in C_b(\R)$.
	\end{ex}
	
	\begin{proof}
		Let $(\xi_{n})_{n \in \N} \subset \R$ and $\xi \in \R$. Define $(f_n)_{n \in \N} \subset L^1 (\mu)$ and $f \in L^1(\mu)$ by $f_n(x) = e^{-i \xi_n x}$ and $f(x) = e^{-i \xi x}$. Suppose that $\xi_n \rightarrow \xi$. Then $f_n \convt{p.w.} f$ and for each $n \in N$ and $x \in \R$, 
		\begin{align*}
			|f_n(x)|
			&= |e^{-i \xi_n x}| \\
			& = 1 \\
			& \in L^1(|\mu|)
		\end{align*}
		The dominated convergence theorem implies that
		\begin{align*}
			|\hat{\mu}(\xi_n) - \hat{\mu}(\xi)| 
			& = \bigg| \int_{\R} e^{-i \xi_n x} \dmu(x) - \int_{\R} e^{-i \xi x} \dmu(x)\bigg| \\
			& =  \bigg| \int_{\R} e^{-i \xi_n x} - e^{-i \xi x} \dmu(x) \bigg| \\
			& \leq \int_{\R} |e^{-i \xi_n x} - e^{-i \xi x}| \, d|\mu|(x) \\
			& \rightarrow 0
		\end{align*}
		So $\hat{\mu}: \R \rightarrow \C$ is continuous. Hence $\hat{\mu} \in C_b(\R)$.
	\end{proof}
	
	\begin{defn}
		Let $X$ be a real normed vector space. We define $\MF: \MM(\R) \rightarrow C_b(\R)$ by $$\MF(\mu) = \hat{\mu}$$
	\end{defn}
	
	\begin{ex}
		Let $X$ be a real normed vector space. Then $\MF: \MM(\R) \rightarrow C_b(\R)$ is linear.
	\end{ex}
	
	\begin{proof}
		Let $\mu, \nu \in \MM(\R)$ and $\xi \in \R$. Then 
		\begin{align*}
			\MF[\mu + \nu](\xi) 
			& = \int_{\R} e^{-i \xi x} \, d[\mu + \nu](x) \\
			& = \int_{\R} e^{-i \xi x} \dmu(x) + \int_{\R} e^{-i \xi x} \dnu(x) \\
			& = \MF[\mu](\xi) + \MF[\nu](\xi) 
		\end{align*}
		Since $\xi \in \R$ is arbitrary, $\MF(\mu + \nu) = \MF(\mu) + \MF(\nu)$ and $\MF$ is linear.
	\end{proof}
	
	\begin{ex}
		Let $X$ be a real normed vector space. If $X$ is separable, then $\MF$ is injective.  
	\end{ex}
	
	\begin{proof}
		Suppose that $X$ is separable. Let $\mu \in \MM(X)$. Suppose that $\mu \in \ker \MF$. Then $\hat{\mu} =0$ and for each $\phi \in X^*$, 
		\begin{align*}
			0 
			& = \hat{\mu}(\phi) \\
			& = \int_X e^{-i \phi(x)} \dmu(x) \\
			& = \int_{\R} e^{-ix} \, d[\phi_*\mu](x)
		\end{align*}
	\end{proof}
	
	\begin{ex}
		Let $X$ be a real normed vector space. Then $\MF \in L(\MM(X), C_b(X^*))$ and $\|\MF\| \leq 1$.
	\end{ex}
	
	\begin{proof}
		For $\mu \in \MM(X)$ and $\phi \in X^*$, we have that 
		\begin{align*}
			|\MF[\mu](\phi)|
			& =  \bigg| \int_X e^{-i \phi(x)} \dmu(x) \bigg| \\
			& \leq \int_X |e^{-i \phi(x)}| \, d|\mu|(x) \\
			& = |\mu|(X) \\
			& = \|\mu\|
		\end{align*}
		Hence 
		\begin{align*}
			\|\MF(\mu)\| 
			& = \sup_{\phi \in X^*} |\MF[\mu](\phi)| \\
			& \leq \|\mu\|
		\end{align*}
		which implies that $\MF \in L(\MM(X), C_b(X^*))$ and $\|\MF\| \leq 1$.
	\end{proof}
	
	
	
	
	
	
	
	
	
	
	
	
	
	
	
	
	
	
	
	
	
	\newpage
	\section{Fourier Analysis on $\R^n$}	

	\subsection{Schwartz Space}
	\begin{defn}
	\ld{100} Let $\al \in \N_0^n$ and $x, y \in \R^n$. We define 
	\begin{enumerate}
	\item $\l x , y\r  = \sum_{j}x_jy_j$
	\item $|x| = \l x, x\r^{1/2}$
	\item $|\al| = \al_1 + \cdots + \al_n$
	\item $x^\al = x_1^{\al_1}\cdots x_n^{\al_n}$
	\item $\p^{\al} = \p_{x_1}^{\al_1} \cdots \p_{x_n}^{\al_n}$
	\end{enumerate}
	\end{defn}	
	
	\begin{defn}
	\ld{101} Let $f \in C^{\infty}(\R^n)$,$\al \in \N_0^n$ and $N \in \N_0$. We define $$\|f\|_{\al, N} = \sup_{x \in \R^n} (1 + |x|)^N |\p^{\al}f (x) |$$
	We define Schwartz space, denoted $\MS$, by $$\MS = \{f \in C^{\infty}(\R^n): \text{ for each $\al \in \N_0^n$, $N \in \N_0$, } \|f\|_{\al, N} < \infty\}$$
	\end{defn}
	
	\begin{ex}
	\lex{102} For each $f \in \MS$ and $\al \in \N_0^n$, $\p^\al f \in L^1(\R^n)$.
	\end{ex}
	
	\begin{proof}
	Let $f \in \MS$, $\al \in \N_0^n$. Then there exists $C \geq 0$ such that for each $x \in \R^n$, $$| \p^{\al} f(x)| \leq C(1+|x|^{2})^{-1}$$
	Define $g:\R^n \rightarrow \Rg$ defined by $g(x) = (1+|x|^{2})^{-1}$. Then $g \in L^1(\R^n)$ which implies that $\p^{\al} f \in L^1(\R^n)$.
	\end{proof}
	
	\begin{defn}
	
	\end{defn}
	
	
	
	
	
	
	
	
	
	
	
	\newpage
	\subsection{The Convolution}
	\begin{defn}
	\ld{200}Let $f, g \in L^0(\R^n)$. If for a.e. $x \in \R^n$, $$\int_{\R^n} |f(x-y)g(y)| dm(y) < \infty$$  
	we define the \textbf{convolution of $f$ with $g$}, denoted $f * g: \R^n \rightarrow \C$, by $$ f * g(x) = \int_{\R^n} f(x-y)g(y) dm(y)$$
	\end{defn}
	
	\begin{ex}
	\lex{201}Let $f, g \in L^1(\R^n)$. Then $f * g \in L^1(\R^n)$ and $\|f * g\|_1 \leq \|f\|_1 \|g\|_1$. 
	\end{ex}	
	
	\begin{proof}
	Define $h \in L^0(\R^n \times \R^n)$ by $h(x,y) = f(x-y)g(y)$. Tonelli's theorem implies that, 
	\begin{align*}
	\int_{\R^n \times \R^n} |h| dm^2
	&= \int_{\R^n} \bigg[  \int_{\R^n} |f(x-y)g(y)| dm(y) \bigg] dm(x) \\
	&= \int_{\R^n} |g(y)| \bigg[  \int_{\R^n} |f(x-y)| dm(y) \bigg] dm(x) \\
	&=  \|f\|_1 \int_{\R^n} |g(y)| dm(x) \\
	&= \|f\|_1 \|g\|_1\\
	& < \infty
	\end{align*}
	Then $h \in L^1(\R^n \times \R^n)$. Fubini's theorem implies that $f * g \in L^1(\R^n)$. Clearly 
	\begin{align*}
	\|f *g\|_1 
	& \leq \int_{\R^n \times \R^n} |h| dm^2 \\
	& \leq \|f\|_1 \|g\|_1
	\end{align*}
	\end{proof}
	
	
	
	\begin{ex}
	\lex{202} Let $f, g, h \in L^1(\R^n)$. Then $(f * g) * h = f * (g * h)$. \\
	\textbf{Hint:} use the substitution $z \mapsto z-y$
	\end{ex}
	
	\begin{proof}
	Let $x \in \R^n$. Then using the substitution $z \mapsto z-y$ and Fubini's theorem, we obtain
	\begin{align*}
	(f*g)*h(x) 
	&= \int f * g(x - y) h (y) dm(y) \\
	&= \int \bigg[ \int f(x-y-z) g(z) dm(z)  \bigg] h(y) dm(y) \\
	&= \int \bigg[ \int f(x-z) g(z - y) dm(z)  \bigg] h(y) dm(y) \\
	&= \int \bigg[ \int f(x-z) g(z - y)  h(y)dm(z)  \bigg]  dm(y) \\
	&= \int \bigg[ \int f(x-z) g(z - y)  h(y) dm(y)  \bigg] dm(z) \\
	&= \int f(x-z) \bigg[ \int g(z - y)  h(y) dm(y)  \bigg] dm(z) \\
	&= \int f(x-z) g*h(z) dm(z) \\
	&= f*(g*h)(z)
	\end{align*}
	So $(f*g)*h = f*(g*h)$. 
	\end{proof}
	
	\begin{ex}
	\lex{203}Let $f, g \in L^1(\R^n)$. Then $f * g = g* f$. 
	\end{ex}	
	
	\begin{proof}
	Let $x \in \R^n$. Using the transformation $y \mapsto x-y$, we obtain that 
	\begin{align*}
	f*g(x)
	&= \int f(x-y) g(y) dm(y) \\
	&= \int f(y) g(x-y) dm(y) \\
	&= \int g(x-y) f(y) dm(y) \\
	&= g *f(x)
	\end{align*}
	So $f * g = g* f$.
	\end{proof}
	
	\begin{note}
	To summarize, $(L^1(\R^n), *)$ is a commutative Banach algebra.
	\end{note}
	
	
	\begin{ex} \textbf{Young's Inequality:} \\
	\lex{204} Let $p \in [1,\infty]$, $f \in L^1$ and $g \in L^p$. Then $f*g \in L^p$ and $\|f *g\|_p \leq \|f\|_1\|g\|_p$. 
	\end{ex}
	
	\begin{proof}
	Define $K \in L^0(\R^n \times \R^n)$ by $K(x,y) = f(x-y)$. Since for each $x,y \in \R^n$, 
	\begin{align*}
	\int|K(x,y)|dm(x) 
	&= \int|K(x,y)|dm(y) \\
	&= \|f\|_p
	\end{align*} 
	an exercise in section $5.1$ of 
	\cite{measure}
	implies that $f*g \in L^p$ and $\|f *g\|_p \leq \|f\|_1\|g\|_p$.
	\end{proof}
	
	\begin{ex}
	\lex{205} Let $p, q \in [1, \infty]$ be conjugate, $f \in L^p(\R^n)$ and $g \in L^q(\R^n)$. Then 
	\begin{enumerate}
	\item for each $x \in \R^n$, $f * g(x)$ exists. 
	\item $\|f*g\|_u \leq \|f\|_p \|g\|_q $
	\item 
	\end{enumerate}
	\end{ex}
	
	\begin{proof}
	\begin{enumerate}
	\item Let $x \in \R^n$. Holder's inequality implies that 
	\begin{align*}
	\int_{\R^n}|f(x-y)g(y)| dm(y) 
	& \leq \|f\|_p \|g\|_q 
	\end{align*}
	Then $f*g(x)$ exists. 
	\item Let $x \in \R^n$. Then in part $(1)$ we showed that  
	\begin{align*}
	|f*g(x)| 
	&= \bigg| \int_{\R^n} f(x - y)g(y) dm(y)\bigg| \\
	&\leq  \int_{\R^n} |f(x-y)g(y)| dm(y) \\
	& \leq \|f\|_p \|g\|_q
	\end{align*}
	Since $x \in \R^n$ is arbitrary, $\|f*g\|_u \leq \|f\|_p \|g\|_q $.
	\item 
	\end{enumerate}
	\end{proof}
	
	\begin{ex}
	\lex{206} Let $f \in L^1(\R^n)$, $k \in \N$ and $g \in C^k(\R^n)$. Suppose that for each $\al \in \N_0^n$, $|\al| \leq k$ implies that $\p^{\al} g \in L^{\infty}$. Then for each $\al \in \N_0^n$, $|\al| \leq k$ implies that $f *  g \in C^{k}$ and $$\p^{\al}(f*g) = f*\p^{\al}g$$
	\end{ex}
	
	\begin{proof}
	Let $\al \in \N_0^n$. Suppose that $|\al| = 1$. Define $h \in L^0(\R^n \times \R^n)$ by $h(x,y) = g(x-y)f(y)$. Young's inequality imples that for a.e. $ x \in \R^n$, $h(x, \cdot) \in L^1(m)$. For each $y \in \R^n$, $\p^{\al} h (\cdot,y) = \p^{\al}g(\cdot -y)f(y)$ and for each $x,y \in \R^n$, $|\p^{\al} h (x,y)| \leq \|\p^{\al} g\|_{\infty}|f(y)| \in L^1(\R^n)$. An exercise in section $3.3$ of 
	\cite{measure}
	implies that for a.e. $x \in \R^n$, $\p^{\al} (g*f)(x)$ exists and 
	\begin{align*}
	\p^{\al} (f*g)(x) 
	&= \p^{\al} (g * f)(x) \\
	&= \p^{\al}\int_{\R^n} h(x,y) dm(y) \\
	&= \int_{\R^n} \p^{\al} g(x -y)f(y) dm(y) \\
	&= (\p^{\al} g) * f (x) \\
	&=  f * (\p^{\al} g) (x)   
	\end{align*}	 
	Now proceed by induction on $|\al|$.
	\end{proof}
	
	
	
	
	
	
	
	
	
	\newpage
	\subsection{The Fourier Transform}
	
	\begin{defn}
	
	\end{defn}	
	
	\begin{ex}
	\lex{300} Let $\phi:\R \rightarrow S^1$ be a measurable homomorphism. 
	\begin{enumerate}
	\item Then $\phi \in L^1_{\loc}(\R)$ and there exists $a > 0$ such that $$\int_{(0,a]}\phi dm \neq 0$$
	\item Define $$c = \bigg[ \int_{(0,a]}\phi dm \bigg]^{-1}$$ 
	Then  For each $x \in \R$, $$\phi(x) = c\int_{(x, x+a]}\phi dm$$ 
	\item $\phi \in C^{\infty}(\R)$ and $\phi' = c(\phi(a) - 1)\phi$
	\item Define $b = c(\phi(a) - 1)$ and $g \in C^{\infty}(\R)$ by $g(x) = e^{-bx} \phi(x)$. Then $g$ is constant and there exists $\xi \in \R$ such that for each $x \in \R$, $\phi(x) = e^{2 \pi i \xi x}$.
	\end{enumerate}
	\end{ex}	
	
	\begin{proof}\
	\begin{enumerate}
	\item Let $K \subset \R$ be compact. Then $$\int_K |\phi| dm = m(K) < \infty$$ So $\phi \in L^1_{\loc}(\R)$. For the sake of contradiction, suppose that for each $a >0$, $$\int_{(0,a]}\phi dm = 0$$ 
	Then the FTC implies that $\phi = 0$ a.e. on $\Rg$, which is a contradiction. So there exists $a > 0$ such that $$\int_{(0,a]}\phi dm \neq 0$$
	\item For $x \in \R$, 
	\begin{align*}
	\phi(x) 
	&= c \int_{(0,a]} \phi(x)\phi(t) dm(t) \\
	&= c \int_{(0,a]} \phi(x+t) dm(t) \\
	&= c \int_{(x,x+a]} \phi dm 
	\end{align*}
	\item Part $(2)$ and the FTC imply that $\phi$ is continuous. Let $d \in \R$. Define $f_d \in C((d, \infty))$ by $$f_d(x) = \int_{(d, x]} \phi dm$$ 
	Since $\phi$ is continuous, the FTC implies that $f_d$ is differentiable and for each $x >d$ $f_d'(x) = \phi(x)$. Part $(2)$ implies that for each $x > d$,
	\begin{align*}
	\phi(x) 
	&= c \int_{(x,x+a]} \phi dm \\
	&= c(f_d(x+a) - f_d(x))
	\end{align*}
	So for each $x > d$, $\phi$ is differentiable at $x$ and 
	\begin{align*}
	\phi'(x) 
	&= c(\phi(x+a) - \phi(x)) \\
	&= c(\phi(a) - 1) \phi(x)
\end{align*}	 
	Since $d \in \R$ is arbitrary, $\phi$ is differentiable and $\phi' = c(\phi(a) - 1) \phi$. This implies that $\phi \in C^{\infty}(\R)$.
	\item Let $x \in \R$. Then 
	\begin{align*}
	g'(x) 
	&= e^{-bx}\phi'(x) - be^{-bx}\phi(x) \\
	&= be^{-bx} \phi(x) - be^{-bx}\phi(x) \\
	&= 0
	\end{align*}
	So $g' = 0$ and $g$ is constant. Hence there exists $k \in \R$ such that for each $x \in \R$, $\phi(x) = ke^{bx}$. Since $\phi(0) = 1$, $k = 1$. Since $|\phi| = 1$, there exists $\xi \in \R$ such that $b = 2 \pi i \xi$. 
	\end{enumerate}
	\end{proof}
	
	\begin{note}
	To summarize, for each measurable homomorphism $\phi:\R \rightarrow S^1$, there exists $\xi \in \R$ such  such that for each $x \in \R$, $\phi(x) = e^{2 \pi i  \xi x}$. 
	\end{note}
	
	\begin{ex}
	\lex{301} Let $\phi: \R^n \rightarrow S^1$ be a measurable homomorphism. Then there exists $\xi \in \R^n$ such that for each $x \in \R^n$, $\phi(x) = e^{2 \pi i \l \xi, x\r}$. 
	\end{ex}	
	
	\begin{proof}
	When done in the category of measurable groups, an exercise in the section on direct products of groups of \cite{groups}
	implies that there exist measurable homomorphism $(\phi_{j})_{j=1}^n \subset (S^1)^{\R}$ such that $\phi = \bigotimes_{j=1}^n \phi_j$. The previous exercise imples that there exist $\xi \in \R^n$ such that for each $x \in \R^n$, $\phi_j(x_j) = e^{2 \pi i \xi_j x_j}$. Then for each $x \in \R^n$, 
	\begin{align*}
	\phi(x)
	&= \prod_{j=1}^n \phi_j(x_j) \\
	&= \prod e^{2 \pi i \xi_j x_j} \\
	&= e^{2 \pi i \sum\limits_{j=1}^n  \xi_j x_j }\\
	&= e^{2 \pi i \l \xi, x \r}
	\end{align*}
	\end{proof}
	
	\begin{defn}
	\ld{302} Let $f \in L^1(\R^n)$. We define the \textbf{Fourier transform of $f$}, denoted $\hat{f}: \R^n \rightarrow \C$ by 
	$$\hat{f}(\xi) = \frac{1}{(2 \pi)^{n / 2}} \int_{\R^n} f(x) e^{- 2 \pi i \l \xi , x\r} dm(x)$$
	\end{defn}
	
	
	
	
	
	
	
	
	
	
	
	
	
	
	\newpage
	\section{Fourier Analysis on LCA Groups}
	
	
	

	\subsection{The Convolution}	
	\begin{note}
	For the remainder of the section, we fix a locally compact abelian group $G$ and a Haar measure $\mu$ on $G$. 
	\end{note}
	
	\begin{defn} \ld{00000} 
	Let $f, g \in L^1(\mu)$. We define the \textbf{convolution of $f$ with $g$}, denoted $f * g: G \rightarrow \C$, by $$ f * g(x) = \int_X f(x-y)g(y) d\mu(y)$$
	\end{defn}
	
	\begin{ex} \lex{00000} 
	Let $f, g \in L^1(\mu)$. Then $f * g \in L^1(\mu)$. 
	\end{ex}
	
	\begin{proof}
	By Tonelli's theorem, 
	\begin{align*}
	\int_X |f *g| d\mu 
	&\leq \int_X \bigg[  \int_X |f(x-y)g(y)| d\mu(y) \bigg] d\mu(x) \\
	&= \int_X |g(y)| \bigg[  \int_X |f(x-y)| d\mu(y) \bigg] d\mu(x) \\
	&=  \|f\|_1 \int_X |g(y)| d\mu(x) \\
	&= \|f\|_1 \|g\|_1\\
	& < \infty
	\end{align*}
	\end{proof}

























	\newpage
	\section{Fourier Analysis on Banach Spaces}
	
	
	
	
	
	
	
	
	
	
	
	
	
	
	
	
	
	
	
	
	
	
	
	\newpage
	\begin{thebibliography}{4}
	
\bibitem{algebra} \href{https://github.com/carsonaj/Mathematics/blob/master/Introduction\%20to\%20Algebra/Introduction\%20to\%20Algebra.pdf}{Introduction to Algebra}

\bibitem{analysis}  \href{https://github.com/carsonaj/Mathematics/blob/master/Introduction\%20to\%20Analysis/Introduction\%20to\%20Analysis.pdf}{Introduction to Analysis}	

\bibitem{foranal}  \href{https://github.com/carsonaj/Mathematics/blob/master/Introduction\%20to\%20Fourier\%20Analysis/Introduction\%20to\%20Fourier\%20Analysis.pdf}{Introduction to Fourier Analysis}

\bibitem{measure}  \href{https://github.com/carsonaj/Mathematics/blob/master/Introduction\%20to\%20Measure\%20and\%20Integration/Introduction\%20to\%20Measure\%20and\%20Integration.pdf}{Introduction to Measure and Integration}


\end{thebibliography}
	
	
	
\end{document}