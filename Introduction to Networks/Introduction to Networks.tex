\documentclass[12pt]{amsart}
\usepackage[margin=1in]{geometry} 
\usepackage{amsmath,amsthm,amssymb,setspace, mathtools}
\usepackage{physics}

\usepackage{color}   %May be necessary if you want to color links
\usepackage{hyperref}
\hypersetup{
	colorlinks=true, %set true if you want colored links
	linktoc=all,     %set to all if you want both sections and subsections linked
	linkcolor=black,  %choose some color if you want links to stand out
	urlcolor=cyan
}


%
%
%
\newif\ifhideproofs
%\hideproofstrue %uncomment to hide proofs
%
%
%
%
\ifhideproofs
\usepackage{environ}
\NewEnviron{hide}{}
\let\proof\hide
\let\endproof\endhide
\fi

\theoremstyle{definition}
\newtheorem{definition}{Definition}[subsection]
\newtheorem{defn}[definition]{Definition}
\newtheorem{note}[definition]{Note}
\newtheorem{thm}[definition]{Theorem}
\newtheorem{lem}[definition]{Lemma}
\newtheorem{prop}[definition]{Proposition}
\newtheorem{cor}[definition]{Corollary}
\newtheorem{conj}[definition]{Conjecture}
\newtheorem{ex}[definition]{Exercise}
\newtheorem{exm}[definition]{Example}

\newcommand{\al}{\alpha}
\newcommand{\gam}{\gamma}
\newcommand{\Gam}{\Gamma}
\newcommand{\bet}{\beta} 
\newcommand{\del}{\delta} 
\newcommand{\Del}{\Delta}
\newcommand{\lam}{\lambda}  
\newcommand{\Lam}{\Lambda} 
\newcommand{\ep}{\epsilon}
\newcommand{\sig}{\sigma} 
\newcommand{\om}{\omega}
\newcommand{\Om}{\Omega}
\newcommand{\C}{\mathbb{C}}
\newcommand{\N}{\mathbb{N}}
\newcommand{\E}{\mathbb{E}}
\renewcommand{\H}{\mathbb{H}}
\newcommand{\Z}{\mathbb{Z}}
\newcommand{\R}{\mathbb{R}}
\newcommand{\T}{\mathbb{T}}
\newcommand{\Q}{\mathbb{Q}}
\renewcommand{\P}{\mathbb{P}}
\newcommand{\MA}{\mathcal{A}}
\newcommand{\MC}{\mathcal{C}}
\newcommand{\MB}{\mathcal{B}}
\newcommand{\MF}{\mathcal{F}}
\newcommand{\MG}{\mathcal{G}}
\newcommand{\ML}{\mathcal{L}}
\newcommand{\MN}{\mathcal{N}}
\newcommand{\MS}{\mathcal{S}}
\newcommand{\MP}{\mathcal{P}}
\newcommand{\ME}{\mathcal{E}}
\newcommand{\MT}{\mathcal{T}}
\newcommand{\MI}{\mathcal{I}}
\newcommand{\MJ}{\mathcal{J}}
\newcommand{\MM}{\mathcal{M}}
\newcommand{\MX}{\mathcal{X}}

\newcommand{\tMA}{\tilde{\MA}}
\newcommand{\tMB}{\tilde{\MB}}


\newcommand{\tU}{\tilde{U}}
\newcommand{\tV}{\tilde{V}}
\newcommand{\tphi}{\tilde{\phi}}
\newcommand{\tpsi}{\tilde{\psi}}
\newcommand{\tF}{\tilde{F}}

\newcommand{\limn}{\lim \limits_{n \rightarrow \infty}}
\newcommand{\convt}[1]{\xrightarrow{\text{#1}}}
\newcommand{\conv}[1]{\xrightarrow{#1}} 
\newcommand{\seq}[2]{(#1_{#2})_{#2 \in \N}}


\newcommand{\Tn}[1]{T^{r_{#1}}_{s_{#1}}(V)}
\newcommand{\Tnp}{T^{r_1 + r_2}_{s_1 + s_2}(V)}


\newcommand{\p}{\partial}

\renewcommand{\r}{\rangle}
\renewcommand{\l}{\langle}

\newcommand{\RG}{[0,\infty]}
\newcommand{\Rg}{[0,\infty)}
\newcommand{\Ll}{L^1_{\text{loc}}(\R^n)}

\newcommand{\limfn}{\liminf \limits_{n \rightarrow \infty}}
\newcommand{\limpn}{\limsup \limits_{n \rightarrow \infty}}



\DeclareMathOperator{\supp}{supp}
\DeclareMathOperator{\spn}{span}
\DeclareMathOperator{\sgn}{sgn}
\DeclareMathOperator{\iso}{Iso}
\DeclareMathOperator{\id}{id}
\DeclareMathOperator{\Aut}{Aut}

\begin{document}
	
	\title{Introduction to Latent Space Network Statistics}
	\author{Carson James}
	\maketitle
	
	\tableofcontents
	
	\section{General Model}
	\subsection{Introduction}
	\begin{defn}
	Let $(M, d)$ be a metric space, $(G, \tau)$ a topological group, and $\cdot: G \times M \rightarrow M$ a group action.  
	Suppose that for each $g \in G$, the map $x \mapsto g \cdot x$ is an isometry. We define 
	$\bar{d}: M / G \rightarrow \Rg$ by 
	\begin{align*}
	\bar{d}(o_x, o_y) 
	&= \inf_{\substack{a \in o_x \\ b \in o_y}} d(a,b) \\
	&= \inf_{g \in G} d(g \cdot x, y)
	\end{align*}
	
	\end{defn}
	
	
	
	\begin{ex}
	If for each $x \in M$, $o_x$ is closed, then $\bar{d}$ is a metric.
	\end{ex}
	
	\begin{proof}
	Suppose that for each $x \in M$, $o_x$ is closed. We need only show that for each $x,y \in M$, $\bar{d}(o_x , o_y) = 0$ implies that $o_x = o_y$. Suppose that $\bar{d}(o_x , o_y) = 0$. Then $\inf\limits_{ g \in G} d(g \cdot x, y) = 0$. Hence there exists $(\tau_n)_{n \in N} \subset G$ such that $\tau_n \cdot x \rightarrow y$. Since $(\tau_n \cdot x)_{n \in \N} \subset o_x$ and $o_x$ is closed, $y \in o_x$. Thus $o_x = o_y$. 
	\end{proof}
	
	\begin{exm}
	Consider the metric space $(\C, | \cdot |)$, topological group $(S^1, | \cdot |)$ and  the (right) action $x \cdot u = xu$. Then the orbits are concentric cirles, which are closed. 
	\end{exm}
	
	
	
	
	
	
	
	
	
	
	
	
	
	\newpage
	\section{Random Inner Product Graphs}
	\subsection{Introduction}
	
	\begin{exm}
	Consider the metric space $(\C^{n \times d}, \|\cdot\|_F)$, topological group $(U_d, \|\cdot\|_F)$ and  the (right) action $X \cdot U = XU$
	\end{exm}
	
	
	
	
	
	
	
	
	
	
	
	
	\newpage
	\section{Random Kernel Graphs}	
	\subsection{Introduction}
	\begin{defn}
	Let $(X, \MA, \mu)$ be a measure space. Define $\| \cdot \|_*: L^1(X, \MA, \mu) \rightarrow \Rg$ by  
	 $$\| f \|_* = \sup_{A \in \MA} \bigg | \int_A f d\mu \bigg |$$ 
	\end{defn}
	
	\begin{ex}
	Let $(X, \MA, \mu)$ be a measure space. Then $\| \cdot \|_*$ is a norm on $L^1(X, \MA, \mu)$.
	\end{ex}
	
	\begin{proof}
	Clear.
	\end{proof}
	
	\begin{defn}
	Let $(X, d)$ be a compact space. Define $$\Aut(X) = \{\sig:X\rightarrow X: \sig \text{ is a homeomorphism} \}$$ We metrize $\Aut(X)$ with uniform convergence $d_u$. It is known that this topology is equivalent to the compact-open topology.
	\end{defn}
	
	\begin{ex}
	With the setup as above, $(\Aut(X), d_{u} )$ is a topological group.
	\end{ex}
	
	\begin{proof}
	Please see section on topological groups: \href{https://github.com/carsonaj/Mathematics/blob/master/Introduction\%20to\%20Analysis/Introduction\%20to\%20Analysis.pdf}{Analysis Notes}
	\end{proof}
	
	\begin{defn}
	Let $(X,d)$ be a compact metric space and $\mu: \MB(X) \rightarrow \R$ a Borel measure. Define $$\Aut(X, \MB(X), \mu) = \{\sig \in \Aut(X): \sig_* \mu = \mu\}$$ 
	So that $(\Aut(X, \MB(X), \mu), d_{u} )$ is a subspace of $(\Aut(X), d_{u})$.
	\end{defn}
	
	\begin{ex}
	Let $(X,d)$ be a compact metric space and $\mu: \MB(X) \rightarrow \R$ an outer-regular Borel measure. Then $\Aut(X, \MB(X), \mu)$ is a closed subgroup of $\Aut(X)$.
	\end{ex}
	
	\begin{proof}
	Please see section on topological groups: \href{https://github.com/carsonaj/Mathematics/blob/master/Introduction\%20to\%20Analysis/Introduction\%20to\%20Analysis.pdf}{Analysis Notes}
	\end{proof}
	
	\begin{exm}
	With the setup as before, define the (right) group action \\ $\cdot: (L^1(X, \MB(X), \mu), \|\cdot\|_*) \times \Aut(X, \MB(X), \mu) \rightarrow (L^1(X, \MB(X), \mu), \|\cdot\|_*) $ by $f \cdot \sig = f \circ \sig$. Then for each $\sig \in \Aut(X, \MB(X), \mu)$, the map $f \mapsto f \cdot \sig$ is an isometry. 
	\end{exm}
	
	\begin{proof}
	Clear.
	\end{proof}
	
	\begin{ex}
	With the setup from above, the orbits are closed
	\end{ex}
	
	\begin{proof}
	IDK, would like to show. I dont think $\Aut(X, \MB(X), \mu)$ is compact. So still thinking about how to show this.
	\end{proof}
	
	
\end{document}