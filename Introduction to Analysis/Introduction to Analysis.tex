\documentclass[12pt]{amsart}
\usepackage[margin=1in]{geometry} 
\usepackage{amsmath,amsthm,amssymb,setspace, mathtools}

\usepackage{color}   %May be necessary if you want to color links
\usepackage{hyperref}
\hypersetup{
	colorlinks=true, %set true if you want colored links
	linktoc=all,     %set to all if you want both sections and subsections linked
	linkcolor=black,  %choose some color if you want links to stand out
}


%
%
%
\newif\ifhideproofs
%\hideproofstrue %uncomment to hide proofs
%
%
%
%
\ifhideproofs
\usepackage{environ}
\NewEnviron{hide}{}
\let\proof\hide
\let\endproof\endhide
\fi

\newtheorem{thm}{Theorem}[subsection]
\newtheorem{lem}[thm]{Lemma}
\newtheorem{prop}[thm]{Proposition}
\newtheorem{cor}[thm]{Corollary}
\newtheorem{conj}{Conjecture}

\theoremstyle{definition}
\newtheorem{definition}{Definition}[subsection]
\newtheorem{defn}[definition]{Definition}

\theoremstyle{remark}
\newtheorem{remark}{Note}[subsection]
\newtheorem{note}[remark]{Note}

\theoremstyle{definition}
\newtheorem{ex}[definition]{Exercise}



\DeclareMathOperator{\supp}{supp}

\newcommand{\al}{\alpha}
\newcommand{\Gam}{\Gamma}
\newcommand{\be}{\beta} 
\newcommand{\del}{\delta} 
\newcommand{\Del}{\Delta}
\newcommand{\lam}{\lambda}  
\newcommand{\Lam}{\Lambda} 
\newcommand{\ep}{\epsilon}
\newcommand{\sig}{\sigma} 
\newcommand{\om}{\omega}
\newcommand{\Om}{\Omega}
\newcommand{\C}{\mathbb{C}}
\newcommand{\N}{\mathbb{N}}
\newcommand{\E}{\mathbb{E}}
\newcommand{\Z}{\mathbb{Z}}
\newcommand{\R}{\mathbb{R}}
\newcommand{\T}{\mathbb{T}}
\newcommand{\Q}{\mathbb{Q}}
\renewcommand{\P}{\mathbb{P}}
\newcommand{\MA}{\mathcal{A}}
\newcommand{\MC}{\mathcal{C}}
\newcommand{\MB}{\mathcal{B}}
\newcommand{\MF}{\mathcal{F}}
\newcommand{\MG}{\mathcal{G}}
\newcommand{\ML}{\mathcal{L}}
\newcommand{\MN}{\mathcal{N}}
\newcommand{\MS}{\mathcal{S}}
\newcommand{\MP}{\mathcal{P}}
\newcommand{\ME}{\mathcal{E}}
\newcommand{\MT}{\mathcal{T}}
\newcommand{\MM}{\mathcal{M}}
\newcommand{\MI}{\mathcal{I}}

\newcommand{\io}{\text{ i.o.}}
\newcommand{\ev}{\text{ ev.}}
\renewcommand{\r}{\rangle}
\renewcommand{\l}{\langle}

\newcommand{\RG}{[0,\infty]}
\newcommand{\Rg}{[0,\infty)}
\newcommand{\Ll}{L^1_{\text{loc}}(\R^n)}

\newcommand{\limfn}{\liminf \limits_{n \rightarrow \infty}}
\newcommand{\limpn}{\limsup \limits_{n \rightarrow \infty}}
\newcommand{\limn}{\lim \limits_{n \rightarrow \infty}}
\newcommand{\convt}[1]{\xrightarrow{\text{#1}}}
\newcommand{\conv}[1]{\xrightarrow{#1}} 
\newcommand{\seq}[2]{(#1_{#2})_{#2 \in \N}}

\DeclareMathOperator{\sgn}{sgn}



\begin{document}
	
	\title{Introduction to Analysis}
	\author{Carson James}
	\maketitle
	
	\tableofcontents
	
	\section*{Preface}
	\begin{flushleft}
		content...
	\end{flushleft}

	
	
	\newpage
	

	\newpage
	
	\section{Real and Complex Numbers}
	\begin{note}
		As a starting point, we will take as fact the existence of the \textbf{natural numbers} $$\N = \{1, 2, \cdots\}$$ the \textbf{integers} $$\Z = \{\cdots, -2, -2, 0, 1, 2, \cdots\}$$ and the \textbf{rational numbers} $$\Q = \bigg \{\frac{a}{b}: a \in \Z, b \in \N \bigg \}$$
	\end{note}
	\subsection{Real Numbers}
	
	\begin{defn}
		Let $X$ be a set and $\leq$ a relation on $X$. Then $\leq$ is said to be a \textbf{total order} if for each $a,b,c \in X$,
		\begin{enumerate}
			\item $a \leq a$
			\item $a \leq b$ and $b \leq c$ implies that $a \leq  c$ 
			\item $a \leq b$ and $b \leq a$ implies that $a = b$ 
			\item $a \leq b$ or $b \leq a$
		\end{enumerate}
	\end{defn}

	\begin{ex}
		We define the relation $\leq$ on $\Q$ defined by $$\frac{a}{b} \leq \frac{c}{d} \hspace{.2cm} \text{iff} \hspace{.2cm} ad \leq bc$$ Then $\leq$ is a total order of $\Q$.
	\end{ex}

	\begin{proof} Let $\frac{a}{b}, \frac{c}{d}, \frac{e}{f} \in \Q$. Then
		\begin{enumerate}
			\item  $\frac{a}{b} \leq \frac{a}{b}$ since $ab \leq ab$. 
			\item if $\frac{a}{b} \leq \frac{c}{d}$ and $\frac{c}{d} \leq  \frac{e}{f}$, then $ad \leq bc$ and $ cf \leq de$. Multiplying the first inequality by $f$ and the second inequality by $b$, we obtain $adf \leq bcf \leq bde$. Dividing both sides by $d$ yields $af \leq be$. Hence $\frac{a}{b} \leq \frac{e}{f}$. 
			\item if $\frac{a}{b} \leq \frac{c}{d}$ and $\frac{c}{d} \leq \frac{a}{b}$, then $ad \leq bc$ and $bc \leq ab$. This implies that $ad = bc$. Hence $\frac{a}{b} = \frac{c}{d}$.
			\item 
		\end{enumerate}
	\end{proof}
	
	\section{Metric Spaces}
	
	\newpage
	
	
	
	
	
	
	
	
	
	
	
	
	
	
	
	
	
	
	
	
	
	
	
	
	
	
	
	
	
	\section{Topology}
	
	\begin{defn}
		Let $(X,\MA)$ and $(Y,\MB)$ be topological spaces and $f:X \rightarrow Y$. Then 
		\begin{enumerate}
			\item $f$ is said to be \textbf{continuous} if for each $B \in \MB$, $f^{-1}(B) \in \MA$.
			\item $f$ is said to be open if for each $A \in \MA$, $f(A) \in \MB$.
			\item $f$ is said to be \textbf{closed} if for each $A \subset X$, if $A^c \in \MA$, then $f(A)^c \in \MB$.
		\end{enumerate}
	\end{defn}
	
	\begin{ex}
		Let $X, Y$ be topological spaces and $\phi: X \rightarrow Y$ a homeomorphism. Then for each $A \subset X$, 
		\begin{enumerate}
			\item $\overline{\phi(A)} = \phi(\overline{A})$  \item $\phi(A)^{\circ} = \phi(A^{\circ})$  
		\end{enumerate} 
	\end{ex}
	
	\begin{proof}\
		\begin{enumerate}
			\item Let $A \subset X$. Since $A \subset \overline{A}$, we have that $\phi(A) \subset \phi(\overline{A})$. Since $\overline{A}$ is closed, $\phi(\overline{A})$ is closed and thus $\overline{\phi(A)} \subset \phi(\overline{A})$. Conversely, let $x \in \phi(\overline{A})$. Then $\phi^{-1}(x) \in \overline{A}$. Then there exists a net $\l y_{\al}\r \subset A$ such that $y_{\al} \rightarrow \phi^{-1}(x)$. Then $\l \phi(y_{\al}) \r \subset \phi(A)$ and $\phi(y_{\al}) \rightarrow x$. Thus $x \in \overline{\phi(A)}$ and $\phi(\overline{A}) \subset \overline{\phi(A)}$.
			\item Similar
		\end{enumerate} 
	\end{proof}
	
	\newpage
	
	
	
	
	
	
	
	
	
	
	
	
	
	
	
	
	
	
	
	
	
	
	
	
	
	
	
	
		\section{Functional Analysis}
	
	\subsection{Normed Vector Spaces}
	\begin{note}
		In the following, we will consider vector spaces over $\C$. There are analogous results for real vector spaces as well, just replace every $\C$ with $\R$.
	\end{note}
	
	\begin{defn}
		Let $X$ be a normed vector space. Then $X$ is said to be a \textbf{Banach space} if $X$ is complete.  
	\end{defn}
	
	\begin{defn}
		Let $X$ be a normed vector space and $(x_i)_{i=1}^n \subset X$. The series $\sum_{i =1}^{\infty}x_i$ is said to \textbf{converge} if the sequence $s_n := \sum_{i=1}^n x_i$ converges. The series $\sum_{i =1}^{\infty}x_i$ is said to \textbf{converge absolutely} if $\sum_{i\in \N}\|x_i \|< \infty$.
	\end{defn}
	
	\begin{thm}
		Let $X$ be a normed vector space. Then $X$ is complete iff for each $\seq{x}{i} \subset X$, $\sum_{i =1}^{\infty}x_i$ converges absolutely implies that $\sum_{i=1}^{\infty}x_i$ converges. 
	\end{thm}
	
	\begin{proof}
		Suppose that $X$ is complete. Let $\seq{x}{i} \subset X$. Suppose that $\sum_{i=1}^{\infty}x_i$ converges absolutely. Let $\ep >0$. Choose $N \in \N$ such that for each $m,n \in \N$, if $m, n \geq N$ and $m< n$, then $\sum_{m+1}^n \|x_i \|< \ep$. Let $m, n \in \N$. Suppose that $m<n$. Then 
		\begin{align*}
			\|s_n-s_m \|
			&= \bigg \|\sum_{i=1}^n x_i -\sum_{i=1}^m x_i\bigg \|\\
			&= \bigg\|\sum_{i=m+1}^{n} x_i \bigg \| \\
			& \leq \sum_{i=m+1}^n \|x_i \|\\
			& < \ep
		\end{align*}
		
		Thus $(s_n)_{n \in N}$ is cauchy. Since $X$ is complete, $\sum_{i=1}^{\infty}x_i$ converges. \\
		Conversely, Suppose that for each $\seq{x}{i} \subset X$, $\sum_{i =1}^{\infty}x_i$ converges absolutely implies that $\sum_{i=1}^{\infty}x_i$ converges. Let $\seq{x}{i} \subset X$ be cauchy. Proceed inductively to create a strictly increasing sequence $(n_i)_{i \in \N} \subset \N$ such that for each $m, n \in \N$, if $m,n \geq n_i$, then $ \|x_m-x_n \|< 2^{-i}$. Define $(y_i)_{i \in \N} \subset X$ by 
		\[ y_i = \begin{cases}
			x_{n_1} & i=1 \\
			x_{n_i} - x_{n_{i-1}} & i \geq 2\\
		\end{cases}\]
		
		Then $\sum_{i=1}^k y_i = x_{n_k}$ and 
		\begin{align*}
			\sum_{i \in \N} \|y_i \|
			&= \|x_{n_1} \|+ \sum_{i \in \N} \|x_{n_i}-x_{n_{i-1}} \|\\
			& \leq \|x_{n_1} \|+ \sum_{i \in \N}2^{-i}\\
			& = \|x_{n_1} \|+1
		\end{align*}
		Hence $(x_{n_k})_{k \in \N} = (\sum_{i=1}^k y_i)_{i\in \N}$ converges. Since $(x_i)_{i \in \N}$ is cauchy and has a convergent subsequence, it converges. So $X$ is complete.
	\end{proof}
	
	\begin{defn}
		Let $X,Y$ be a normed vector spaces. A linear map $T:X \rightarrow Y$ is said to be \textbf{bounded} if there exists $C \geq 0$ such that for each $x \in X$, $\|Tx \|\leq C \|x \|$.
	\end{defn}
	
	\begin{ex}
		Let $X,Y$ be a normed vector spaces and $T:X \rightarrow Y$ a linear map. Then $T$ is bounded iff there exists $r,s>0$ such that $T(B(0,r)) \subset B(0,s)$
	\end{ex}
	
	\begin{proof}
		Suppose that $T$ is bounded. Then there exists $C \geq 0$ such that for each $x \in X$, $\|Tx \|\leq C \|x \|$. Thus $T(B(0,1)) \subset B(0,C+1)$. Conversely. Suppose that there exists $r,s >0$ such that $T(B(0,r)) \subset B(0,s)$. Define $C = \frac{2s}{r}$. Let $x \in X$. Put $\al = \frac{r}{2\|x \|}$ Then $\al x \in B(0,r)$. So $T(\al x ) = \al T(x) \in B(0,s)$. Hence 
		\begin{align*}
			\|T(\al x) \|
			&= \|\al T(x) \|\\
			&= \vert \al \vert \|T(x) \|\\
			& = \frac{r}{2 \|x \|}  \|T(x) \|\\
			& < s.
		\end{align*}
		Thus $$\|Tx \|< \frac{2 s}{r} \|x \|= C \|x \|$$ So $T$ is bounded. 
	\end{proof}
	
	\begin{thm}
		Let $X,Y$ be normed vector spaces and $T:X \rightarrow Y$ a linear map. Then the following are equivalent:
		\begin{enumerate}
			\item $T$ is continuous
			\item $T$ is continuous at $x=0$
			\item $T$ is bounded
		\end{enumerate}
	\end{thm}
	
	\begin{proof}
		$(1) \implies (2)$:\\
		Trivial\vspace{1cm}\\
		$(2) \implies (3)$:\\
		Suppose that $T$ is continuous at $x=0$. Then there exists $\del>0$ such that for each $x \in X$, if $\|x \|< \del$, then $\|Tx \|< 1$. Choose $C = \frac{2}{\del}$. If $x=0$, then $\|Tx \|\leq C \|x \|$. Suppose that $\|x \|\neq 0$. Define $y = \frac{\del}{2 \|x \|}x$. Then $\|y \|< \del$. So $$ \|Ty \|
		= \frac{\del}{2 \|x \|} \|Tx \|< 1$$
		Thus 
		\begin{align*}
			\|Tx \|
			&< \frac{2}{\del} \|x \| \\
			&=C \|x \|
		\end{align*}
		
		Hence $T$ is bounded.\vspace{1cm}\\
		$(3) \implies (1)$\\
		Suppose that $T$ is bounded. Then there exists $C \geq 0$ such that for each $x \in X$, $\|Tx \|\leq C\|x \|$. Let $\ep >0$. Choose $\del = \frac{\ep}{C+1}$. Let $x,y \in X$ Suppose that $\|x-y \|< \del$. Then 
		\begin{align*}
			\|Tx-Ty \|
			& = \|T(x-y) \| \\
			& \leq C \|x-y \|\\
			&< (C+1) \del\\ 
			&= \ep
		\end{align*}
		
		So $T$ is continuous.
	\end{proof}
	
	\begin{defn}
		Let $X,Y$ be normed vector spaces. Define $L(X,Y) = \{T:X \rightarrow Y: T \text{ is bounded}\}$. Define $\|\cdot\|: L(X,Y)\rightarrow \Rg$ by $$\|T\| = \inf \{C \geq 0: \text{for each }x \in X\text{, } \|Tx \|\leq C\|x\|\}$$ We call $\|\cdot \|$ the \textbf{operator norm on $L(X,Y)$}
	\end{defn}
	
	\begin{ex}
		Let $X,Y$ be normed vector spaces. If $X\neq \{0\}$, then the operater norm on $L(X,Y)$ is given by: 
		\begin{enumerate}
			\item $\|T\| = \sup\limits_{\|x\|=1}\|Tx\|$
			\item $\|T\| = \sup\limits_{x \neq 0}\|x\|^{-1} \|Tx\|$
			\item $\|T\| = \inf \{C \geq 0: \text{for each }x \in X\text{, } \|Tx \|\leq C\|x\|\}$
		\end{enumerate}
	\end{ex}
	
	\begin{proof} Since $X \neq \{0\}$, the supremums in (1) and (2) are well defined. Let $T \in L(X,Y)$. By linearity of $T$, the sets over which the supremums are taken in (1) and (2) are the same. So (1) and (2) are equal.\vspace{1cm}\\
		
		Now, put $M = \sup\limits_{\|x \|=1} \|Tx \|$, $m = \inf \{C \geq 0: \text{ for each }x \in X\text{, } \|Tx \|\leq C \|x \|\}$ and let $x \in X$. If $\|x \|=0$, then $\|Tx \|\leq M \|x \|$. Suppose that $\|x \|\neq 0$. Then 
		\begin{align*}
			\|Tx \|
			&= \bigg(\big\|T(x/\|x\|)\big\|\bigg)\|x \|\\
			& \leq M ||x||
		\end{align*}
		
		Hence $M \in \{C \geq 0: \text{ for each }x \in X\text{, } \|Tx \|\leq C \|x \|\}$. Therefore $m \leq M$
		
		Let $C \in \{C \geq 0: \text{ for each }x \in X\text{, } \|Tx \|\leq C \|x\|\}$. Suppose that $\|x \|=1$. Then $\Vert Tx\Vert \leq C \|x \|= C$. So $M \leq C$. Therefore $M \leq m$. So $M=m$ and the supremum in (1) is the same as the infimum in (3). 
	\end{proof}
	
	\begin{note}
		From here on, unless stated otherwise, we assume $X \neq 0$.
	\end{note}
	
	\begin{ex}
		Let $X,Y$ be normed vector spaces and $T \in L(X,Y)$. Then for each $x \in X$, $\|Tx \| \leq \|T\|\|x \|$
	\end{ex}
	
	\begin{proof}
		This is just part of the previous exercise. Let $x \in X$. If $x = 0$, then $\|Tx \|\leq \|T \|\|x \|$. Suppose that $x \neq 0$. Then $\|Tx \|= T(x/\|x\|)\|x\|\leq \|T \|\|x \|$
	\end{proof}
	
	\begin{ex}
		Let $X, Y$ be normed vector spaces. Then the operator norm is a norm on $L(X,Y)$.
	\end{ex}
	
	\begin{proof}
		Let $S,T \in L(X,Y)$ and $\al \in \C$. For each $x \in X$, we have that 
		\begin{align*}
			\|(S+T)x \|
			&= \|Sx+Tx \|\\
			& \leq \|Sx \|+ \|Tx \|\\
			&\leq \|S \|\|x \|+ \|T \|\|x \|\\
			&= \big(\|S \|+ \|T \|\big) \|x \|
		\end{align*}
		
		So $\|S+T \|\leq \|S \|+ \|T \|$.\vspace{1cm}\\
		
		Using the definition of $\|T \|$, we see that 
		\begin{align*}
			\|\al T \|
			&= \sup_{\|x \|=1} \|(\al T)x \|\\
			&= \sup_{\|x \|=1} \vert \al \vert \|Tx \|\\
			&=\vert \al \vert \sup_{\|x \|=1} \|Tx \|\\
			&=\vert \al \vert \|T \|
		\end{align*} 
		So $\|\al S \|= \vert \al \vert \|S \|$. \vspace{1cm}\\ Suppose that $\|T \|= 0$. Let $x \in X$. Then $\|T x\|\leq \|T \|\|x \|= 0$. So $Tx=0$. Since $x \in X$ is arbitrary, we have that $T=0$. 
	\end{proof}
	
	\begin{ex}
		Let $X$ be a normed vector space. Then addition and scalar multiplication are continuous on $X \times X$ and $\|\cdot \|:X \rightarrow \Rg$ is continuous.
	\end{ex}
	
	\begin{proof}
		Let $\ep > 0$. Choose $\del = \frac{\ep}{2}$. Let $(x_1,y_1), (x_2,y_2) \in X \times X$. Suppose that $\|(x_1,y_1)-(x_2,y_2) \| = \max\{\|x_1-x_2 \|, \|y_1 - y_2 \|\} < \del$. Then 
		\begin{align*}
			\|(x_1 + y_1) - (x_2+y_2) \|
			&= \|(x_1-x_2) + (y_1-y_2) \|\\
			& \leq \| x_1-x_2 \|+ \|y_1-y_2 \|\\
			& < 2\del \\
			&= \ep
		\end{align*} 
		
		Hence addition is uniformly continuous. \vspace{1cm}\\ Let $(\lam_1,x_1) \in \C \times X$ and $\ep >0$. Choose $\del = \min\{\frac{\ep}{2(\vert \lam_1 \vert + \|x_1 \|+ 1)}, \frac{\sqrt{\ep}}{\sqrt{2}}\}$. Let $(\lam_2, x_2) \in \C \times X$. Suppose that $\|(\lam_1, x_1)-(\lam_2,x_2) \| = \max\{\vert \lam_1-\lam_2 \vert , \|x_1 - x_2 \|\} < \del$. Then 
		\begin{align*}
			\|\lam_1x_1 - \lam_2x_2 \|
			&= \|\lam_1x_1 - \lam_1x_2 + \lam_1x_2- \lam_2x_2 \|\\
			&= \|\lam_1(x_1-x_2) + (\lam_1-\lam_2)x_2 \|\\
			& \leq \vert \lam_1 \vert \| x_1-x_2 \|+ \vert \lam_1-\lam_2 \vert \|x_2\|\\
			& \leq \vert \lam_1 \vert  \| x_1-x_2 \|+ \vert \lam_1-\lam_2 \vert (\|x_1 -x_2\|+ \|x_1\|)\\
			& < \vert \lam_1 \vert \del  +  \del( \del + \|x_1 \|)\\
			&= (\vert \lam_1 \vert + \|x_1 \|) \del + \del^2 \\
			&< \frac{\ep}{2}+ \frac{\ep}{2}\\
			&= \ep
		\end{align*}
		Since $(\lam_1, x_1) \in \C \times X$ is arbitrary, scalar multiplication is continuous. \vspace{1cm} \\ Let $\ep > 0$. Choose $\del = \ep$. Let $x,y \in X$. Suppose that $\|x-y \|< \del$. Then 
		\begin{align*}
			\big \vert \|x \|- \|y \|\big  \vert
			& \leq \|x - y \|\\
			&< \del\\
			&=\ep
		\end{align*}  
		So $\|\cdot \|: X \rightarrow \Rg$ is uniformly continuous.
	\end{proof}
	
	\begin{ex}
		Let $X,Y$ be normed vector spaces. If $Y$ is complete, then so is $L(X,Y)$.
	\end{ex}
	
	\begin{proof}
		Suppose that $Y$ is complete. Let $(T_n)_{n \in \N} \subset L(X,Y)$. Suppose that $(T_n)_{n \in \N}$ is Cauchy. Since for each $m,n \in \N$, $\big\vert \|T_m \|- \|T_n \|\big\vert \leq \|T_m -T_n \|$, we have that $(\|T_n \|)_{n \in \N} \subset \Rg$ is Cauchy. Hence $\lim\limits_{n \rightarrow \infty}\|T_n \|$ exists. \vspace{1cm} \\ Let $x \in X$ and $m,n \in \N$. Then 
		\begin{align*}
			\|T_m x - T_n x \|
			&= \|(T_m-T_n) x \|\\
			&\leq \|T_m-T_n \|\|x \|
		\end{align*}
		So $(T_nx)_{n \in \N} \subset Y$ is Cauchy and hence converges. Define $T:X \rightarrow Y$ by $Tx = \lim\limits_{n \rightarrow \infty} T_nx$. \vspace{1cm}\\
		Since addition and scalar multiplication are continuous, $T$ is linear. Let $x \in X$ and $\ep>0$. Choose $N \in \N$ such that for each $n \in N$, if $n \geq N$, then $\|Tx - T_n x\|< \ep$. Then for each $n \in \N$, if $n \geq N$ we have that 
		\begin{align*}
			\|Tx\|
			&\leq \|Tx-T_nx \|+ \|T_nx \|\\
			&< \ep + \|T_nx \|\\
			&\leq \ep + \|T_n \|\|x \|
		\end{align*}  
		Thus $\|Tx \|\leq \ep +(\lim\limits_{n \rightarrow \infty} \|T_n \|) \|x \|$. Since $\ep >0$ is arbitrary, $\|Tx \|\leq (\lim\limits_{n \rightarrow \infty} \|T_n \|) \|x \|$. Thus $T \in L(X,Y)$ and $\|T \|\leq \limn \|T_n \|$. \vspace{1cm} \\
		Note that since addition, scalar multiplication and $\|\cdot \|$ are continuous, we have that for each $n \in \N$ and $x \in X$, $\|(T_n-T_m)x \|$ converges to $\|(T_n-T)x \|$ because 
		\begin{align*}
			\lim_{m \rightarrow \infty} \|(T_n-T_m)x \|
			&= \lim_{m \rightarrow \infty} \|T_nx-T_mx \|\\
			&= \|T_nx-\lim_{m \rightarrow \infty}T_mx \|\\
			&=\|T_nx-Tx \|\\
			&= \|(T_n-T)x \|
		\end{align*} 
		Let $\ep >0 $. Choose $N \in \N$ such that for each $m, n \in \N$ if $n,m \geq N$, then $\|T_n - T_m \|< \ep$. Then for each $n \in \N$ if $n \geq N$, then for each $x \in X$, $$\|(T_n-T_m)x\|\leq \|(T_n-T_m)\|\|x \|< \ep \|x\|$$ Combining this with the previous fact, we see that for each $n \in N$, if $n \geq N$, then for each $x \in X$, $$\|(T_n -T) x\|\leq \ep \|x \|$$ In particular, for each $n \in \N$, if $n \geq N$, then $$ \|T_n -T \|= \sup\limits_{\|x \|= 1}\|(T_n - T)x \|\leq \ep$$ This implies that $T_n$ converges to $T$ in $L(X,Y)$. 
		Since $$\big\vert \|T_n \|- \|T \|\big \vert \leq \|T_n - T \|$$ it is clear that $\limn \|T_n \|= \|T \|$
	\end{proof}
	
	\begin{defn}
		Let $X$ be a normed vector space and $M \subset X$ a closed subspace. Define $\|\cdot\|:X/M \rightarrow \Rg$ by $$\|x+M\| := \inf_{y \in M}\|x+y\|$$
		
		We call $\|\cdot\|$ the \textbf{subspace norm on $X/M$}
	\end{defn}
	
	\begin{ex}
		Let $X$ be a normed vector space and $M \subsetneq X$ a proper, closed subspace of $M$. 
		Then 
		\begin{enumerate}
			\item The previously defined subspace norm on $X/M$ is well defined and is a norm. 
			\item For each $\ep > 0$, there exists $x \in X$ such that $\|x\|=1$ and $\|x+M\| \geq 1-\ep$.
			\item The projection map $\pi:X \rightarrow X/M$ defined by $\pi(x) = x+M$ is continuous and $\|\pi\|=1$. 
			\item If $X$ is complete, then $X/M$ is complete. 
		\end{enumerate} 
	\end{ex}
	
	\begin{proof}
		\begin{enumerate}
			\item  Let $x, y \in X$ and $\al \in \C$. Suppose that $x+M =y+M$. Then there exists $m \in M$ such that $x=y+m$. Since $M$ is a subspace, the map $T:M \rightarrow M$ given by $Tx = x+m$ is a bijection. So $$\inf_{z \in M} \|y+m+z \|= \inf_{z \in M} \|y+z \|$$ which implies that 
			\begin{align*}
				\|x +M \|
				&= \inf_{z \in M} \|x+z \|\\
				&= \inf_{z \in M} \|y+m+z \|\\
				&= \inf_{z \in M} \|y+z \|\\
				&= \|y+M \|
			\end{align*} 
			So $\|\cdot \|: X/M \rightarrow \Rg$ is well defined.\vspace{.5cm}\\
			We observe that for each $z,w \in M$, $$\|x+y+z \|\leq \|x+w \|+ \|y+w+z \|$$
			Taking infimums over $M$ with respect to $z$ in this inequality implies that for each $w \in M$,
			\begin{align*}
				\inf_{z \in M}\|x+y+z \|
				&\leq \inf_{z \in M} \bigg( \|x+w \|+ \|y+w+z \|\bigg) \\
				&= \|x+w \|+\inf_{z \in M}\|y+w+z \|
			\end{align*}
			Again we use the fact that for each $w \in M$, $$\inf_{z \in M}\|y+w+z \|= \inf_{z \in M}\|y+z \|$$
			This implies that for each $w \in M$, $$\inf_{z \in M}\|x+y+z \|\leq \|x+w \|+ \inf_{z \in M}\|y+z \|$$
			
			Therefore, taking infimums over $M$ with respect to $w$ in this inequality yields
			\begin{align*}
				\|x+y+M \|
				&= \inf_{z \in M} \|x+y +z \|\\
				& \leq \inf_{w \in M} \bigg(\|x+w \|+ \inf_{z \in M}\|y+z \|\bigg)\\
				&= \inf_{w \in M} \|x+w \|+ \inf_{z \in M}\|y+z \|\\
				&= \|x+M \|+ \|y+M \|
			\end{align*}
			\vspace{.5cm}\\
			If $\al =0$, then $\al x = 0$. Choosing $z = 0 \in M$ gives $\|\al x+M \|=0 = \vert \al \vert \|x+M \|$. Suppose that $\al \neq 0$. Then the map $T:M \rightarrow M$ given by $Tx = \al ^{-1}x$ is a bijection and thus $\inf\limits_{z \in M} \|x+\al^{-1}z \|= \inf\limits_{z \in M} \|x+z \|$. Hence we have that
			\begin{align*}
				\|\al x+M \|
				&= \inf_{z \in M} \|\al x +z \|\\
				&= \inf_{z \in M} \vert \al \vert \|x +\al^{-1}z \|\\
				&= \vert \al \vert \inf_{z \in M}\|x +\al^{-1}z \|\\
				&= \vert \al \vert \inf_{z \in M}\|x +z \|\\
				&= \vert \al \vert \|x+M \|
			\end{align*} 
			
			Suppose that $\|x \|=0$. Choose a sequence $(z_n)_{n \in N} \subset M$ such that 
			\begin{align*}
				\lim\limits_{n \rightarrow \infty} \|x - z_n \|
				& = \inf_{z \in M} \|x+ z \|\\
				& = 0
			\end{align*} 
			
			Then $\limn z_n =x$. Since $M$ is closed, $x \in M$. Hence $x+M=0+M$. \vspace{1cm}\\
			\item Since $M$ is a proper subspace, there exists $v \in X$ such that $v \not \in M$. Then $\|v +M \|\neq 0$. Let $\ep >0$. Then $(1-\ep)^{-1}\|v+M \|> \|v+M \|$. So there exists $z \in M$ such that $$0< \|v+M\|\leq \|v+z \|< (1-\ep)^{-1} \|v+M \|$$ Choose $x = \|v+z \|^{-1}(v+z)$. Then $\|x \|=1$ and 
			\begin{align*}
				\|x+M \|
				&= \|v+z \|^{-1} \|v+z +M \|\\
				&= \|v+z \|^{-1} \|v +M \|\\
				&> 1-\ep
			\end{align*}\vspace{.5cm}\\
			\item Let $x \in X$. Taking $z=0$, we we see that $\|\pi(x) \|=\|x+M \|\leq \|x+z \|= \|x \|$. So $\pi$ is bounded and in particular, $$\sup_{\|x \|=1} \|\pi(x) \|\leq 1$$ 
			From (2) we see that $$\sup_{\|x \|=1} \|\pi(x) \|\geq 1$$
			Hence $\|\pi\|= 1$. \vspace{.5cm}\\
			\item Suppose that $X$ is complete. Let $(x_i+M)_{i\in \N} \subset X/M$. Suppose that $\sum\limits_{i\in \N} \|x_i+M \|< \infty$. Let $\ep>0$. Then for each $i \in \N$, there exists $z_i \in M$ such that $\|x_i +z_i \|< \|x_i +M \|+ \ep2^{-i}$. Define the sequence $(a_i)_{i\in \N} \subset X$ by $a_i = x_i +z_i$. Then we have 
			\begin{align*}
				\sum_{i\in \N} \|a_i \|
				&= \sum_{i \in N} \|x_i + z_i \|\\
				&\leq \sum_{i \in N} \bigg (\|x_i +M \|+ \ep2^{-i} \bigg)\\
				&= \sum_{i\in \N} \|x_i+M\|+ \ep
			\end{align*}
			Since $\ep>0$ is arbitrary, it follows that $$\sum_{i\in \N} \|a_i \|\leq \sum_{i\in \N} \|x_i+M\|< \infty$$
			Since $X$ is complete, $\sum\limits_{i=1}^{\infty}a_i$ converges in $X$. Define $(s_n)_{n \in \N} \subset X$ and $s \in X$ by $s_n = \sum\limits_{i =1}^n a_i$ and $s = \sum\limits_{i=1}^\infty a_i $. Since $\limn s_n = s$, and $\pi: X \rightarrow X/M$ is continuous, it follows that $\limn \pi(s_n) = \pi(s)$. Since 
			\begin{align*}
				\pi(s_n) 
				&= \sum_{i=1}^n a_i +M\\
				&= \sum_{i=1}^n x_i +M
			\end{align*} 
			We have that $\sum\limits_{i=1}^{\infty}x_i +M$ converges which implies that $X/M$ is complete.
		\end{enumerate}
	\end{proof}
	
	\begin{ex}
		Let $X,Y$ be normed vector spaces and $T \in L(X,Y)$. Then
		\begin{enumerate}
			\item $\ker T$ is closed
			\item there exists a unique map $S :X/ \ker T \rightarrow T(X)$ such that $T = S \circ \pi$. Furthermore $S$ is a bounded linear bijection and $\|S \|= \|T \|$.
		\end{enumerate}
	\end{ex}
	
	\begin{proof}
		\begin{enumerate}
			\item Since $T$ is continuous and $\ker T = T^{-1}(\{0\})$, we have that $\ker T$ is closed.
			\item Suppose that there exists $S_1,S_2 \in L(X/ \ker T, T(X)) $ such that $T = S_1 \circ \pi$ and  $T = S_2 \circ \pi $. Let $x \in X$. Then $$S_1(x + \ker T) = S_1(\pi(x)) = T(x) = S_2(\pi(x)) = S_2(x + \ker T)$$ So $S_1 = S_2$. Therefore such a map is unique.\\
			Define $S: X / \ker T \rightarrow T(X)$ by $S(x+ \ker T) = T(x)$. Then $S$ is clearly a linear bijection that satisfies $T = S \circ \pi$. Let $x \in X$ and $z \in \ker T$. Then 
			\begin{align*}
				\|S(x+ \ker T) \|
				& = \|T(x) \|\\
				& = \|T(x+z) \|\\
				& \leq \|T \|\|x+ z \|
			\end{align*} 
			Thus $$\|S(x+ \ker T) \|\leq \|T \|\inf_{z \in \ker T}  \|x + z \|= \|T \|\|x + \ker T \|$$
			So $S$ is bounded and $\|S \|\leq \|T \|$. This implies that $$\|T \|= \|S \circ \pi \|\leq \|S \|\|\pi \|= \|S \|$$
			Thus $\|S \|= \|T \|$.
		\end{enumerate}
	\end{proof}
	
	\begin{ex}
		Let $X, Y$ be normed vector spaces. Define $\phi: L(X,Y) \times X \rightarrow Y$ by \\$\phi(T,x) = Tx$. Then $\phi$ is continuous.
	\end{ex}
	
	\begin{proof}
		Let $(T_1, x_1) \in L(X,Y) \times X$ and $\ep > 0$. Choose $\del = \min \{\frac{\ep}{2(\|x_1 \|+ \|T_1 \|+1)}, \frac{\sqrt{\ep}}{\sqrt{2}} \}$. Let $(t_2, x_2) \in L(X,Y) \times X$. Suppose that $$\|(T_1, x_1) - (T_2, x_2) \|= \max \{\|T_1 - T_2\|, \|x_1 -x_2 \|\} < \del$$. Then 
		\begin{align*}
			\|\phi(T_1, x_1) - \phi(T_2-x_2) \|
			&= \|T_1 x_ - T_2 x_2 \|\\
			&= \|T_1 x_1 - T_2 x_1 + T_2 x_1 - T_2 x_2 \|\\
			& \leq \|(T_1 - T_2) x_1 \|+ \|T_2(x_1 -x_2) \|\\
			& \leq \|T_1 -T_2 \|\|x_1 \|+ \|T_2 \|\|x_1 -x_2 \|\\
			& \leq \|T_1 -T_2 \|\|x_1 \|+ \big(\|T_1 - T_2 \|+ \|T_1 \|\big)\|x_1 -x_2 \|\\
			& < \del \|x_1 \|+ (\del + \|T_1 \|) \del \\
			&= \del (\|T_1 \|+ \|x_1 \|) + \del^2\\
			& < \frac{\ep}{2} + \frac{\ep}{2}\\
			&= \ep
		\end{align*}
		So $\phi$ is continuous.
	\end{proof}
	
	\begin{ex}
		Let $X$ be a normed vector space and $M \subset X$ a subspace. Then $\overline{M}$ is a subspace.
	\end{ex}
	
	\begin{proof}
		Let $x,y \in \overline{M}$ and $\al \in \C$. Then there exist sequences $(x_n)_{n \in \N} \subset M$ and $(y_n)_{n \in \N} \subset M$ such that $x_n \conv{} x$ and $y_n \conv{} y$. Since $M$ is a subspace, $(x_n +y_n)_{n \in \N} \subset M$ and $(\al x_n)_{n \in \N} \subset M$. Since addition and scalar multiplication are continuous, we have that $x_n + y_n \conv{} x+y$ and $\al x_n \conv{} \al x$. Thus $x+y \in \overline{M}$ and $\al x \in \overline{M}$ and hence $\overline{M}$ is a subspace.
	\end{proof}
	
	\begin{ex}
		Let $X,Y,Z$ be normed vector spaces, $T \in L(X,Y)$ and $S \in L(Y,Z)$. Define $ST:X \rightarrow Z$ by $STx = S(Tx)$. Then $ST \in L(X,Z)$ and $\|ST \|\leq \|S \|\|T \|$. 
	\end{ex}
	
	\begin{proof}
		Clearly $ST$ is linear. Let $x \in X$. Then 
		\begin{align*}
			\|ST x \|
			& = \|S(Tx) \|\\
			& \leq \|S \|\|Tx \|\\
			& \leq \|S \|\|T \|\|x \|
		\end{align*}
		
		So $\|ST \|\leq \|S \|\|T \|$.
	\end{proof}
	
	\begin{defn}
		Let $X,Y$ be a normed vector spaces and $T \in L(X,Y)$. Then $T$ is said to be \textbf{invertible} or an \textbf{isomorphism} if $T$ is a bijection and $T^{-1} \in L(Y,X)$.
	\end{defn}
	
	\begin{defn}
		Let $X$ be a Banach space. Define $GL(X) := \{T \in L(X,X): T \text{ is invertible}\}$.
	\end{defn}
	
	\begin{ex}
		Let $X$ be a Banach space. Then 
		\begin{enumerate}
			\item For each $T \in L(X,X)$, if $\|I- T \|< 1$, then $T$ is invertible and $$T^{-1} = \sum_{n=0}^{\infty}(I-T)^n$$
			\item For each $S,T \in L(X,X)$, if $S$ is invertible and $\|S-T \|< \|S^{-1} \|^{-1}$, then $T$ is invertible. 
			\item $GL(X)$ is open.
		\end{enumerate}
	\end{ex}
	
	\begin{proof}\
		\begin{enumerate}
			\item Let $T \in L(X,X)$. Suppose that $\|I-T \|< 1$. Then $$\sum_{n=0}^{\infty} \|(I -T)^n \| \leq \sum_{n=0}^{\infty} \|I -T \|^{n} < \infty$$ Since $X$ is a complete, so is $L(X,X)$ and thus $\sum\limits_{n=0}^{\infty}(I-T)^n$ converges in $L(X,X)$.\\
			Define $(S_k)_{k=0}^{\infty} \subset L(X,X)$ and $S \in L(X,X)$ by $S_k = \sum\limits_{n=0}^{k} (I-T)^n$ and \\ $S = \sum\limits_{n=0}^{\infty}(I-T)^n$. Then for each $k \in \N$,
			\begin{align*}
				S_k T
				&= S_k - S_k(I-T) \\
				&= (I-T)^0 - (I-T)^{k+1} \\
				&= I - (I-T)^{k+1}
			\end{align*}
			and $\|S_kT - I \|\leq \|I-T \|^{k+1}$. Since multiplication on Banach algebras is continuous, we have that $$ST = (\lim_{k \rightarrow \infty} S_k)T = \lim\limits_{k \rightarrow \infty}S_kT = I$$
			Similarly $TS = I$. Thus $T$ is invertible and $T^{-1} = S \in L(X,X)$. \vspace{.5cm}\\
			\item  Let $S, T \in L(X,X)$. Suppose that $S$ is invertible and $\|S-T \|< \|S^{-1} \|^{-1}$. Then 
			\begin{align*}
				\|I - S^{-1}T \|
				& = \|S^{-1}(S - T) \|\\
				& \leq \|S^{-1} \|\|S -T \|\\
				&< 1
			\end{align*}
			So $S^{-1}T$ is invertible. Thus $T = S (S^{-1}T)$ is invertible. \vspace{.5cm}\\
			\item Let $T \in GL(X)$. Choose $\del = \|T^{-1}\|^{-1}$. By (2), $B(T, \del) \subset GL(X)$.
		\end{enumerate}
	\end{proof}

	\subsection{$l^p$ Spaces}
	
	\begin{defn}
		
		Let $p \in [1, \infty] \cup \{0\}$. We define 
		\[
		l^p(\N) = 
		\begin{cases}
			\C^{\N} & p=0 \\
			\bigg \{f \in l^0(\N): \sum\limits_{n \in \N} |f(n)|^p < \infty \bigg\} & p \in [1, \infty) \\
			\bigg \{f \in l^0(\N): \sup\limits_{n \in \N}|f(n)| < \infty \bigg \} & p = \infty
		\end{cases}\]
		So $l^0(\N)$ consists of the sequences in $\C$ and $l^{\infty}(\N)$ consists of the bounded sequences in $\C$. 
		
		For $p \in [1, \infty]$, we define $\| \cdot \|_p : l^p(\N) \rightarrow [0, \infty)$ by 
		\[
		\|f\|_p = 
		\begin{cases}
			\bigg( \sum\limits_{n \in \N} |f(n)|^p \bigg)^{1/p} & p \in [1, \infty) \\
			\sup\limits_{n \in \N}|f(n)| & p = \infty
		\end{cases}
		\]
	\end{defn}

	

	

	
	\subsection{Linear Functionals}
	
	\begin{defn}
		Let $X$ be a normed vector space and $T :X \rightarrow \C$. Then $T$ is said to be a \textbf{linear functional on} $X$ if $T$ is linear and $T$ is said to be a \textbf{bounded linear functional on} $X$ if $T \in L(X, \C)$. We define the \textbf{dual space of} $X$, denoted $X^*$, by $X^* = L(X, \C)$.
	\end{defn}
	
	\begin{defn}
		Let $X$ be a normed vector space and $p:X \rightarrow \R$. Then $p$ is said to be a \textbf{sublinear functional} if for each $x,y \in X$, $\lam \geq 0$, 
		\begin{enumerate}
			\item $p(x+y) \leq p(x) + p(y)$
			\item $p(\lam x ) = \lam p(x)$
		\end{enumerate}  
	\end{defn}
	
	\begin{note}
		Let $X$ be a vector space and $\|\cdot \|: X \rightarrow \Rg$ be a seminorm, then $\|\cdot \|$ is a sublinear functional.
	\end{note}
	
	\begin{thm}{Hahn-Banach Theorem:}
		Let $X$ be a vector space, $p:X \rightarrow \R$ a sublinear functional, $M \subset X$ a subspace and $f:M \rightarrow C$ a linear functional. If for each $x \in M$, $\vert f(x) \vert \leq p(x)$, then there exists a linear functional $F:X \rightarrow \C$ such that for each $x \in X$, $\vert F(x) \vert \leq p(x)$ and $F|_{M}=f$.
	\end{thm}
	
	\begin{ex}
		Let $X$ be a normed vector space, $M \subset X$ a subspace and $f \in M^*$. Then there exists $F \in X^*$ such that $\|F \|= \|f \|$ and $F|_M = f$.  
	\end{ex}
	
	\begin{proof}
		If $f =0$, Choose $F=0$. Suppose $f \neq 0$. Then $\|f \|\neq 0$ and there exists $x_0 \in M$ such that $x_0  \neq 0$. Thus $\|f \|= \sup \{ \vert f(x) \vert: x \in M \text{ and } \|x \|= 1\}$. Define $p:X \rightarrow \Rg$ by $ p(x) = \|f \|\|x \|$. Then $p$ is a sublinear functional on $X$ and for each $x \in M$, $\vert f(x) \vert \leq p(x)$. So there exists a linear functional $F:X \rightarrow \C$ such that for each $x \in X$, $\vert F(x) \vert \leq p(x) = \|f \|\|x \|$ and $F|_M = f$. Thus $F \in X^*$ with $\|F \|\leq \|f \|$. Also $$\|F \|= \sup_{\substack{ x \in X \\ \|x \|= 1}} \vert F(x) \vert \geq  \sup_{\substack{ x \in M \\ \|x \|= 1}} \vert F(x) \vert = \sup_{\substack{ x \in M \\ \|x \|= 1}} \vert f(x) \vert = \|f \|$$
		
		So $\|F \|= \|f \|$.
	\end{proof}
	
	\begin{ex}
		Let $X$ be a normed vector space, $M \subsetneq X$ a proper closed subspace and $x \in X \setminus M$. Then there exists $F \in X^*$ such that $F|_M = 0$, $\|F \|=1$ and $ F(x) = \|x+M \|\neq 0$. (Hint: Consider $f:M+\C x \rightarrow \C$ defined by $f(m+\lam x) = \lam \|x +M \|$.)
	\end{ex}
	
	\begin{proof}
		Define $f:M+\C x \rightarrow \C$ as above. Clearly $f$ is linear and $f|M = 0$. Let $m \in M$ and $\lam \in \C$. If $\lam = 0$, then $\vert f(m +\lam x) \vert = 0 \leq \|m+ \lam x \|$. Suppose that $\lam \neq 0$. Then 
		\begin{align*}
			\vert f(m+\lam x) \vert 
			& = \vert \lam \vert \|x+M \|\\
			& =  \|\lam x+M \|\\
			& = \inf_{z \in M} \|z+ \lam x \|\\
			& \leq  \|m+ \lam x  \|\\
		\end{align*} 
		So $f \in (M+\C x )^*$ and $\|f \|\leq 1$. Let $\ep >0$. A previous exercise tells us that there exist $m \in M, \lam \in \C$ such that $\|m+ \lam x \|= 1$ and $\|m+ \lam x +M \|> 1- \ep$. Then 
		\begin{align*}
			\vert f(m + \lam x) \vert
			&= \vert \lam \vert \|x+M\|\\
			&=\|\lam x +M \|\\
			&= \|m + \lam x +M \|\\
			&> 1-\ep
		\end{align*}
		
		So $$ \|f \|= \sup_{\substack{z \in M + \C x \\ \|z \|=1}} \vert f(z) \vert \geq 1$$ Hence $\|f \|=1$. 
		The same exercise also tells us that $f(x) = \|x+M\|\neq 0$. Using the previous exercise, there exists $F \in X^*$ such that $\|F \|= \|f \|= 1$ and $F|_{M+\C x} = f$.
	\end{proof}
	
	\begin{ex}
		Let $X$ be a normed vector space and $x \in X$. If $x \neq 0$, then there exists $F \in X^*$ such that $\|F \|= 1$ and $F(x) = \|x \|$.
	\end{ex}
	
	\begin{proof}
		Define $f:\C x \rightarrow \C$ by $f(\lam x) = \lam \|x \|$. Then $f$ is linear and $f(x) = \|x \|$. Clearly $$\sup_{\substack{z \in \C x \\ \|z \|=1}}\vert f(z) \vert = 1$$ 
		So $f \in (\C x)^*$ and $\|f \|= 1$. By a previous exercise, there exists $F \in X^*$ such that $\|F \|= \|f \|=1$ and $F|_{\C x} = f$. 
	\end{proof}
	
	\begin{ex}
		Let $X$ be a normed vector space. Then $X^*$ separates the points of $X$. 
	\end{ex}
	
	\begin{proof}
		Let $x, y \in X$. Suppose that $x \neq y$. Then $x-y \neq 0$. The previous exercies implies that there exists $F \in X^*$ such that $\|F \|= 1$ and $$F(x) - F(y) = F(x-y) = \|x-y \|\neq 0$$ Thus $F(x) \neq F(y)$ and $X^*$ separates the points of $X$.
	\end{proof}
	
	\begin{defn}
		Let $X, Y$ be metric spaces and $T : X \rightarrow Y$. Then $T$ is said to be an \textbf{isometry} if for each $x_1, x_2 \in X$, $d( Tx_1, Tx_2) = d(x_1,x_2) $.
	\end{defn}
	
	\begin{ex}
		Let $X,Y$ be metric spaces and $T:X \rightarrow Y$ and isometry. Then $T$ is injective.
	\end{ex}
	
	\begin{proof}
		Let $x_1, x_2 \in X$. Suppose that $Tx_1=Tx_2$. Then $0= d( Tx_1, Tx_2) = d(x_1,x_2)$. So $x_1 = x_2$. Hence $T$ is injective.
	\end{proof}
	
	\begin{note}
		Let $X,Y$ be metric spaces and $T:X \rightarrow Y$ an isometry. Then $T$ is clearly continuous. If $T$ is surjective, then $T^{-1}$ is an isometry and therefore continuous. Hence $T$ is a homeomorphism.
	\end{note}
	
	\begin{ex}
		Let $X$ be a normed vector space and $x \in X$. Define $\hat{x}:X^* \rightarrow \C$ by $\hat{x}(f) = f(x)$. Then $\hat{x} \in X^{**}$ and $\|\hat{x} \|= \|x \|$.
	\end{ex}
	
	\begin{proof}
		Let $f,g \in X^*$ and $\lam \in \C$. Then $$\hat{x}(f+\lam g) = (f+ \lam g)(x) = f(x) + \lam g(x) = \hat{x}(f) + \lam \hat{x}(g)$$
		So $\hat{x}$ is linear. For each $f \in X^*$, $$\vert \hat{x}(f) \vert = \vert f(x) \vert \leq \|x \|\|f \|$$ Hence $\hat{x} \in X^{**}$ with $\|\hat{x} \|\leq \|x \|$. If $x=0$, then $\hat{x} = 0$ and $\|\hat{x} \|= \|x \|$. Suppose that $x \neq 0$. Then a previous exercise implies that there exists $F \in X^*$ such that $\|F \|=1$ and $F(x) = \|x \|$. Then we have that $$\sup_{\substack{f \in X^* \\ \|f \|= 1 } } \vert \hat{x}(f) \vert  = \sup_{\substack{f \in X^* \\ \|f \|= 1 }}  \vert f(x) \vert \geq \vert F(x) \vert = \|x \|$$
		Hence $\|\hat{x} \|= \|x \|$.
	\end{proof}
	
	
	\begin{ex}
		Let $X$ be a normed vector space. Define $\phi : X \rightarrow X^{**}$ by $\phi(x) = \hat{x}$. Then $\phi$ is a linear isometry. 
	\end{ex}
	
	\begin{proof}
		Let $x,y \in X$ and $\lam \in \C$. Then for each $f \in X^*$, we have that 
		\begin{align*}
			\phi(x+ \lam y)(f) 
			&= \widehat{x+ \lam y}(f) \\
			&= f(x+\lam y) \\
			&= f(x) + \lam f(y) \\
			&= \hat{x}(f) + \lam \hat{y}(f)\\
			&= \phi(x)(f) + \lam \phi(y)(f)
		\end{align*} 
		So $\phi(x+ \lam y) = \phi(x) + \lam \phi(y)$ and $\phi$ is linear. The previous exercise tells us that 
		\begin{align*}
			\|\phi(x) - \phi(y) \|
			&= \|\phi(x-y)\|\\
			&= \|\widehat{x-y} \|= \|x-y \|
		\end{align*}
		So $\phi$ is an isometry.
	\end{proof}
	
	\begin{defn}
		Let $X$ be a normed vector space and define $\phi:X \rightarrow X^{**}$ as above. We define $\widehat{X} = \phi(X) \subset X^{**}$. Since $\widehat{X}$ and $X$ are isomorphic, we may identify $X$ as a subset of $X^{**}$. 
	\end{defn}
	
	\begin{defn}
		Let $X$ be a normed vector space and define $\phi:X \rightarrow X^{**}$ as above. Then $X$ is said to be reflexive if $\phi$ is surjective. In this case $\phi$ is then an isomorphism
	\end{defn}
	
	\begin{ex}
		Let $X$ be a normed vector space and $f:X \rightarrow \C$ a linear functional on $X$. Then $f$ is bounded iff $\ker f$ is closed. 
	\end{ex}
	
	\begin{proof}
		Suppose that $f$ is continuous. Since $\{0\}$ is closed, we have that $\ker f = f^{-1}(\{0\})$ is closed. Conversely, suppose that $\ker f$ is closed. If $\ker f = X$, then $f =0$ and $f$ is continuous. Suppose that $\ker f \neq X$. Then $\ker f$ is a proper, closed subspace of $X$. A previous exercise tells us that there exists $x \in X$ such that $\|x \|= 1$ and $\|x + \ker f \|> \frac{1}{2}$. Let $y \in X$. Suppose that $\|y \|< \frac{1}{2}$. Then for each $z \in \ker f$, 
		\begin{align*}
			\|z -  (x+y)\|
			& = \|(z-x) -y \|\\
			& \geq \|z-x \|- \|y \|\\
			& > \frac{1}{2} - \frac{1}{2} \\
			&=0
		\end{align*}
		
		So $x+y \not \in \ker f$. Therefore $f(B(x,\frac{1}{2})) \cap \{0\} = \varnothing$. If $f(B(x,\frac{1}{2})) $ is unbounded, then $f(B(x,\frac{1}{2})) = \C$ by linearity. This is a contradiction since $0 \not \in f(B(x,\frac{1}{2}))$. So There exists $s > 0$ such that $f(B(x,\frac{1}{2})) \subset B(0,s)$ and thus $f$ is bounded. 
	\end{proof}
	
	\begin{ex}
		Let $X$ be a normed vector space. 
		\begin{enumerate}
			\item Let $M \subsetneq X$ be a proper closed subspace of $X$ and $x \in X \setminus M$. Then $M + \C x$ is closed.
			\item Let $M \subset X$ be a finite dimensional subspace of $X$. Then $M$ is closed.
		\end{enumerate}
	\end{ex}
	
	\begin{proof}
		\begin{enumerate}
			\item Let $y \in X$ and $(y_n)_{n \in \N} \subset M+ \C x$. Suppose that $y_n \conv{} y$. If $y \in M$, then $y \in M+ \C x$. Suppose that $y \not \in M$. For each $n \in \N$, there exists $m_n \in M$ and $\lam_n \in \C$ such that $y_n = m_n + \lam_nx$. A previous exercise tells us that there exists $F \in X^*$ such that $\|F \|= 1$, $F|_M = 0$ and $F(x) = \|x+M \|\neq 0$. Since $F$ is continuous, $F(y_n) \conv{} F(y)$. Since for each $n \in \N$, $$F(y_n) = F(m_n + \lam_n x) = F(m_n)+ \lam_n (F_x) = \lam_n F(x)$$ we have that $\lam_n F(x) \conv{} F(y)$. Since $F(x) \neq 0$, this implies that $\lam_n \conv{} F(x)^{-1} F(y)$. It follows that $\lam_n x \conv{}F(x)^{-1}F(y)x$. Since  for each $n \in \N$, $m_n = y_n - \lam_nx$, we know that $m_n \conv{} y-F(x)^{-1}F(y)x$. Since $(m_n)_{n \in \N} \subset M$ and $M$ is closed, we have that $y-F(x)^{-1}F(y)x \in M$ and therefore $y \in M+\C x$. Hence $M+\C x$ is closed. \vspace{.5cm}\\
			\item If $M = X$, then $M$ is closed. Suppose that $M \neq X$. Let $(x_i)_{i=1}^n$ be a basis for $M$. Define $N_0 = \{0\}$ and for each $i =1,2, \cdots, n$, define $N_i = N_{i-1}+\C x_i$. Since $N_0$ is a proper closed subpace of $X$ and $x_1 \in X \setminus N_0$, (1) implies that $N_1$ is closed. Proceed inductively to obtain that $M = N_n$ is closed.
		\end{enumerate}
	\end{proof}
	
	\begin{ex}
		Let $X$ be an infinite-dimensional normed vector space. 
		\begin{enumerate}
			\item There exists a sequence $(x_n)_{n\in \N} \subset X$ such that for each $m, n \in \N$, $\|x_n \|= 1$ and if $m \neq n$, then $\|x_m - x_n \|> \frac{1}{2}$.
			\item $X$ is not locally compact. 
		\end{enumerate}
	\end{ex}
	
	\begin{proof}
		\begin{enumerate}
			\item Define $N_0 = \{0\}$. Then $N_0$ is a closed proper subspace of $X$. Choose $x_1 \in X$ such that $\|x_1 \|= 1$. Using the results of previous exercises, we proceed inductively. For each $n \geq 2$ we define $N_{n-1} = \text{span}(x_1, x_2, \cdots, x_{n-1})$. Then $N_{n-1}$ is a closed proper subspace of $X$. Thus we may choose $x_n \in X$ such that $\|x_n \|= 1$ and $\|x_n + N_{n-1} \|>  \frac{1}{2}$. Let $m,n \in \N$. Suppose that $m<n$. Then $x_m \in N_{n-1}$. Thus $\|x_n - x_m \|\geq \|x_n + N_{n-1} \|> \frac{1}{2}$\vspace{.5cm}\\
			\item Suppose that $X$ is locally compact. Then $\overline{B(0,1)}$ is compact and therefore sequentially compact. Using $(x_n)_{n \in \N} \subset \overline{B(0,1)}$ defined in (1), we see that there exists a subsequence $(x_{n_k})_{k \in \N}$, $x \in \overline{B(0,1)}$ such that $x_{n_k} \conv{} x$. Then $(x_{n_k})_{k \in \N}$ is Cauchy. So there exists $N \in N$ such that for each $j, k \in \N$, if $j, k \geq N$, then $\|x_{n_j} - x_{n_k} \|< \frac{1}{2}$. Then $\|x_{n_N} - x_{n_{N+1}} \| < \frac{1}{2}$. This is a contradiction since by construction, $\|x_{n_N} - x_{n_{N+1}} \| > \frac{1}{2}$. Thus $X$ is not locally compact.
		\end{enumerate}
	\end{proof}
	
	\begin{ex}
		Let $X,Y$ be normed vector spaces and $T \in L(X,Y)$. 
		\begin{enumerate}
			\item Define the \textbf{adjoint of $T$}, denoted  $T^*:Y^* \rightarrow X^*$ by $T^*(f) = f \circ T$. Then $T^* \in L(Y^*, X^*)$.
			\item Applying the result from (1) twice, we have that $T^{**} \in L(X^{**},Y^{**})$. We have that for each $x \in X$, $T^{**}(\hat{x}) = \widehat{T(x)}$.
			\item $T^*$ is injective iff $T(X)$ is dense in $Y$.
			\item If $T^*(Y^*)$ is dense in $X^*$, then $T$ is injective. The converse is true if $X$ is reflexive.
		\end{enumerate}
	\end{ex}
	
	\begin{proof}
		\begin{enumerate}
			\item Let $f \in Y^*$. Then $\|T^* (f) \|= \|f \circ T \|\leq  \|T \| \|f \|$. So $T^* \in L(Y^*, X^*)$ with $\|T^* \|\leq \|T \|$.\vspace{.5cm}\\
			\item Let $x \in X$. Let $f \in Y^*$. Then 
			\begin{align*}
				T^{**}(\hat{x})(f) 
				&= \hat{x} \circ T^{*}(f) \\
				&= \hat{x}(T^* (f)) \\
				&= \hat{x}(f \circ T) \\
				&= f \circ T (x) \\
				&= f(T(x)) \\
				&= \widehat{T(x)}(f)
			\end{align*} 
			Hence $T^{**}(\hat{x}) = \widehat{T(x)}$.\vspace{.5cm}\\
			\item Suppose that $T(X)$ is not dense in $Y$. Then $\overline{T(X)} \neq Y$. So $T(X)$ is a proper closed subspace of $Y$ and there exists $y \in Y$ such that $y \not \in \overline{T(X)}$. By a previous exercise, there exists $f \in Y^*$ such that $f(y) = \|y+\overline{T(X)} \|\neq 0$, $\|f \|=1$ and $f|_{\overline{T(X)}} = 0$. Let $x \in X$. Then $T^*(f)(x) = f \circ T(x) = 0$. Hence $T^*(f) = 0 = T^*(0)$. Since $f \neq 0$, $T^*$ is not injective.\\ Now suppose that $T(X)$ is dense in $Y$. Let $f,g \in Y^*$. Define $h \in Y^*$ by $h = f-g$ Suppose that $T*(f) = T^*(g)$ Then $T^*(h) = 0$. So for each $x \in X$, $h(T(x)) = 0$. Let $y \in Y$ and $\ep >0$. By continuity, there exists $\del > 0 $ such that for each $y' \in Y$, if $\|y - y' \|< \del$, then $\|h(y) - h(y') \|< \ep$. Since $T(X)$ is dense in $Y$, there exists $x \in X$ such that $\|y - T(x) \|< \del$. Thus 
			\begin{align*}
				\|h (y) \|
				&\leq \|h(y) - h(T(x)) \|+ \|h(T(x)) \|\\
				& = \|h(y) - h(T(x)) \| \\
				& < \ep
			\end{align*} 
			Since $\ep > 0$ is arbitrary, $\|h(y) \|= 0$. This implies that $h(y) = 0$ and therefore $f(y) = g(y) $. Since $y \in Y$ is arbitrary, $f=g$ and $T^*$ is injective. \vspace{.5cm}\\
			\item For the sake of contradiction, suppose that $T^*(Y^*)$ is dense in $X^*$ and $T$ is not injective. Then there exist $x_1, x_2 \in X$ such that $x_1 \neq x_2$ and $T(x_1) = T(x_2)$. Define $x = x_1-x_2$. Then $x \neq 0$ and $T(x) = 0$. A previous exercise implies that there exists $F \in X^*$ such that $F(x) = \|x\|\neq 0$ and $\|F \|= 1$. Let $\ep >0$. Choose $g \in Y^*$ such that $\|F - T^*(g) \|< \ep$. Then 
			\begin{align*}
				\|x \|
				&= \vert F(x) \vert \\
				&\leq \vert F(x) - T^*(g)(x) \vert + \vert T^*(g)(x) \vert \\
				& < \ep \|x \|+ \vert g(T(x)) \vert\\
				&= \ep \|x \|
			\end{align*}
			
			Since $\ep > 0$ is arbitrary, we have that $\|x \|=0$ which is a contradiction. Hence if $T^*(Y^*) $ is dense in $X^*$, then $T$ is injective. \vspace{.5cm}\\ 
			Now, suppose that $X$ is reflexive and $T$ is injective. Let $\phi_1, \phi_2 \in X^{**}$. Suppose that $T^{**}(\phi_1) = T^{**}(\phi_2)$. Then $T^{**}(\phi_1 - \phi_2) = 0$. Since $X$ is reflexive, there exist $x_1, x_2 \in X$ such that $\phi_1 = \hat{x_1}$ and $\phi_2 = \hat{x_2}$. Define $x = x_1 - x_2$. Then $T^{**}(\hat{x}) = 0$. So for each $f \in Y^*$, 
			\begin{align*}
				T^{**}(\hat{x})(f) 
				&= \hat{x} \circ T^*(f)\\
				&= \hat{x}( T^*(f))\\
				&= \hat{x} (f \circ T)\\
				&= f \circ T(x)\\
				&= f(T(x))\\
				&= 0 
			\end{align*}
			Suppose that $T(x) \neq 0$. Then a previous exercise implies that there exists $g \in Y^*$ such that $g(T(x)) = \|T(x) \|\neq 0$ and $\|g \| = 1$. This is a contradiction since $g(T(x)) = 0$. So $T(x) = 0$. Since $T$ is injective, this implies that $x = 0$. Hence $\hat{x}=0$ and thus $\phi_1 = \phi_2$. Thus $T^{**}$ is injective. By (3), we have that $T^*(Y^*)$ is dense in $X^*$.
		\end{enumerate}
	\end{proof}
	
	\begin{ex}
		Let $X$ be a normed vector space. Then $X$ is reflexive iff $X^*$ is reflexive. 
	\end{ex}
	
	\begin{proof}
		Suppose that $X$ is reflexive. Let $\al \in X^{***}$. Define $f :X \rightarrow \C$ by $f(x) = \al(\hat{x})$. Clearly $f$ is linear and a previous exercise tells us that for each $x \in X$, 
		\begin{align*}
			\vert f(x) \vert 
			& \leq \|\al \|\|\hat{x} \|\\
			&= \|\al \|\|x \|
		\end{align*}
		So $f \in X^*$.
		Let $\phi \in X^{**}$. Since $X$ is reflexive, there exists $x \in X$ such that $\phi = \hat{x}$. Then 
		\begin{align*}
			\al(\phi)
			&= \al(\hat{x})\\
			&= f(x)\\
			&= \hat{x}(f)\\
			&= \hat{f}(\hat{x})\\
			&= \hat{f}(\phi)
		\end{align*}
		Hence $\al = \hat{f}$. Thus the map $X^* \rightarrow X^{***}$ given by $f \mapsto \hat{f} $ is surjective and so $X^{*}$ is reflexive.\vspace{.5cm}\\
		Conversely, suppose that $X^*$ is reflexive. Since $\phi:X \rightarrow X^{**}$ given by $\phi(x) = \hat{x}$ is an isometry, $\widehat{X} \subset X^{**}$ is closed. For the sake of contradiction, suppose that $\widehat{X} \neq X^{**}$. Then there exists $\al \in X^{**}$ such that $\al \not \in \widehat{X}$. Thus there exists $F \in X^{***}$ such that $\|F \|= 1$, $F(\al) = \|\al + \widehat{X} \|\neq 0$ and $F|_{\widehat{X}}=0$. Since $X^*$ is reflexive, there exists $f \in X^*$ such that $F = \hat{f}$. A previous exercise tells us that $\|f \|= \|\hat{f} \|= \|F \|= 1$. Since for each $x \in X$, $f(x) = \hat{x}(f) = \hat{f}(\hat{x}) = F(\hat{x}) = 0$, we have that $f = 0$. Thus $\|f \|= 0$, a contradiction. So $\widehat{X} = X^{**}$ and $X$ is reflexive.
		
	\end{proof}
	
	\subsection{The Baire Category and Closed Graph Theorems}
	
	\begin{thm}
		Let $X, Y$ be Banach spaces and $T\in L(X,Y)$. If $T$ is surjective, then $T$ is open.
	\end{thm}
	
	\begin{cor}
		Let $X, Y$ be Banach spaces and $T \in L(X,Y)$. If $T$ is a bijection, then $T^{-1} \in L(X,Y)$.
	\end{cor}
	
	\begin{defn}
		Let $X,Y$ be sets and $f:X \rightarrow Y$. We define the \textbf{graph of f}, $\Gam(f)$, by $\Gam(f) = \{(x,y) \in X \times Y: f(x) = y\}$.
	\end{defn}
	
	\begin{thm}
		Let $X, Y$ be Banach spaces and $T:X \rightarrow Y$ a linear map. If $\Gam(T)$ is closed, then $T \in L(X,Y)$.  
	\end{thm}
	
	\begin{note}
		We recall that $\Gam(T)$ is closed iff for each $(x_n)_{n \in \N} \subset X$, $x \in X$ and $y \in Y$, $x_n \conv{} x$ and $T(x_n) \conv{} y$ implies that $T(x) = y$. 
	\end{note}
	
	\begin{thm}
		
		Let $X, Y$ be Banach spaces and $S \subset L(X,Y)$. If for each $x \in X$, $$\sup_{T \in S} \|Tx \|< \infty$$ then $$\sup_{T \in S} \|T \|< \infty$$
	\end{thm}
	
	\begin{ex}
		Let $\mu$ be counting measure on $(N, \MP(\N))$. Define $h: \N \rightarrow \N$ and $ \nu$ on $(N, \MP(\N))$ by $h(n) = n$ and $d \nu = h d \mu$. Define $X=L^1(\nu)$ and $Y = L^1(\mu)$. Equip both $X$ and $Y$ with the $L^1$ norm with respect to $\mu$. 
		\begin{enumerate}
			\item We have that $X$ is a proper subspace of $Y$ and therefore $X$ is not complete.
			\item Define $T: X \rightarrow Y$ by $Tf(n) = nf(n)$. Then $T$ is linear, $\Gam(T)$ is closed, and $T$ is unbounded.
			\item Define $S:Y \rightarrow X$ by $Sg(n) = \frac{1}{n}g(n)$. Then $S \in L(Y,X)$, $S$ is surjective and $S$ is not open. 
		\end{enumerate}
	\end{ex}
	
	\begin{proof}\
		\begin{enumerate}
			\item Note that for each $f: \N \rightarrow \C$, 
			\begin{align*}
				{\|f \|}_{\mu, 1}
				&= \sum_{n=1}^{\infty} \vert f(n) \vert  \\
				& \leq \sum_{n=1}^{\infty} n \vert f(n) \vert  \\
				& = \|f \|_{\nu,1} 
			\end{align*} 
			Hence $X$ is a subspace of $Y$. Define $f : \N \rightarrow \C$ by $f(n) = \frac{1}{n^2}$. Then $$\|f \|_{\mu, 1} = \sum_{n=1}^{\infty} \frac{1}{n^2} < \infty$$ So  $f \in Y$. However $$\|f \|_{\nu, 1} = \sum_{n=1}^\infty \frac{1}{n} = \infty$$ So $f \not \in X$. Thus $X$ is a proper subspace of $Y$. Let $g \in Y$ and $\ep >0$. Since the simple functions are dense in $L^1(\mu)$, there exists $\phi \in L^1(\mu)$ such that $\phi$ is simple and $\|g - \phi \|_{\mu ,1} < \ep$. Then there exist $(c_i)_{i=1}^k \subset \C$ and $ (E_i)_{i=1}^k \subset \MP(\N)$ such that for each $i,j \in  \{1,2,\cdots, k\}$, $E_i$ is finite, $i \neq j$ implies that $E_i \cap E_j = \varnothing$ and  $$\phi = \sum_{i=1}^kc_i \chi_{E_i}$$ Define $c = \max\{\vert c_i \vert: i=1,2,\cdots k\}$ and $m = \max \bigg[ \bigcup_{i=1}^k E_i \bigg]$. Then 
			\begin{align*}
				\|\phi \|_{\nu,1} 
				&=  \sum_{n=1}^m n \vert \phi(n) \vert \\
				& \leq \sum_{n=1}^m  mc \\
				& = c m^2 \\
				& < \infty
			\end{align*}
			Hence $\phi \in X$ and $X$ is dense in $Y$. Since $X$ is a dense, proper subspace, it is not closed. Since $Y$ is complete and $X \subset Y$ is not closed, we have that $X$ is not complete.
			\item Clearly $T$ is linear. Let $(f_j)_{j \in \N} \subset X$, $f \in X$ and $g \in Y$. Suppose that $f_j \conv{L^1(\mu)} f$ and $Tf_j \conv{L^1(\mu)} g$. 
			
			Note that for each $j \in \N$ and $n \in \N$, $$\vert f_j(n) - f(n) \vert \leq \sum_{n =1}^{\infty}\vert f_j(n) - f(n) \vert = \|f_j-f \|_{\mu, 1}$$ and $$\vert nf_j(n) - g(n) \vert \leq \sum_{n =1}^{\infty}\vert nf_j(n) - g(n) \vert = \|Tf_j - g\|_{\mu, 1}$$  
			Thus for each $n \in \N$, $f_j(n) \conv{j} f(n)$ and $nf_j(n) \conv{j} g(n)$. This implies that for each $n \in \N$, $nf(n) = g(n)$. Thus $Tf = g$ which implies that $\Gam(T)$ is closed. Suppose, for the sake of contradiction, that $T$ is bounded. Then there exists $C \geq 0$ such that for each $f \in X$, $\|Tf \|_{\mu,1} \leq C \|f \|_{\mu, 1}$. Choose $n \in \N$ such that $n > C$. Define $f: \N \rightarrow \C$ by $f = \chi_{\{n\}}$. As established above, $S^+ \subset L^1(\mu)$. Then $\|f \|_{\mu,1} = 1$ and
			\begin{align*}
				\|Tf \|_{\mu,1}
				& = n \\
				&> C\\
				& = C \|f \|_{\mu,1}
			\end{align*}
			which is a contradiction. So $T$ is unbounded.
			\item Clearly $S$ is linear. Let $g \in Y$. Then \begin{align*}
				\|Sg \|_{\mu,1} 
				&= \sum_{n =1}^{\infty} \frac{1}{n} \vert g(n) \vert \\
				& \leq  \sum_{n =1}^{\infty} \vert g(n) \vert \\
				& = \|g \|_{\mu,1}
			\end{align*}
			So $S$ is bounded and $\|S \|\leq 1$. Thus $S \in L(Y,X)$. Let $f \in X$. Define $g: \N \rightarrow \C$ by $g(n) = nf(n)$. By defnition, $g \in Y$ and we have that
			\begin{align*}
				Sg(n) 
				&= \frac{1}{n}g(n) \\
				& = f(n)
			\end{align*}
			Hence $Sg =f$ and thus $S$ is surjective. Let $g \in Y$. Suppsose that $Sg = 0$. Then $$\sum_{n=1}^{\infty} \frac{1}{n}\vert g(n)\vert =\|Sg \| = 0$$ Thus for each $n \in \N$, $g(n) = 0$. Hence $\ker S = \{0\}$ and $S$ is injective. Note that for each $A \subset Y$, $S(A)= T^{-1}(A)$. If $S$ is open, then $T$ is continuous which as shown above is a contradiction. So $g$ is not open. 
		\end{enumerate}
	\end{proof}
	
	\begin{ex}
		Let $X = C^1([0,1])$ and $Y=C([0,1])$. Equip both $X$ and $Y$ with the uniform norm. 
		\begin{enumerate}
			\item Then $X$ is not complete
			\item Define $T: X \rightarrow Y$ by $Tf = f'$. Then $\Gam(T)$ is closed and $T$ is not bounded. 
		\end{enumerate}
	\end{ex}
	
	\begin{proof}
		\begin{enumerate}
			\item Recall that for each $a,b \geq 0$ and $p \in \N$, $$(a^{\frac{1}{p}}+b^{\frac{1}{p}})^p = \sum_{n=0}^p  {p \choose n} a^{\frac{n}{p}}b^{\frac{p-n}{p}} \geq a + b$$ Thus $(a+b)^{\frac{1}{p}} \leq a^{\frac{1}{p}}+b^{\frac{1}{p}}$.\\
			For each $n \in \N$, define $f_n: [0,1] \rightarrow \C$ by $f_n(x) = \sqrt{(x-\frac{1}{2})^2+ \frac{1}{n^2}}$. Then $(f_n)_{n \in \N} \subset X$. Define $f:[0,1] \rightarrow \C$ by $f(x) = \vert x-\frac{1}{2}\vert$. Then $f \in Y \cap X^c$. Note that for each $n \in \N$, $f \leq f_n$. Our observation above implies that for each $x \in X$,
			\begin{align*}
				f_n(x) 
				&= \bigg[ (x-\frac{1}{2})^2 + \frac{1}{n^2} \bigg]^{\frac{1}{2}}\\
				& \leq \vert x-\frac{1}{2} \vert + \frac{1}{n}
			\end{align*}
			Thus $0 \leq f_n - f \leq \frac{1}{n} $. This implies that $f_n \convt{u} f$. Since $f \not \in X$, $X$ is not complete. \vspace{.5cm}\\
			\item Let $(f_n)_{n \in \N} \subset X$, $f \in X$ and $g \in Y$. Suppose that $f_n \convt{u} f$ and $Tf_n \convt{u} g$. Let $x \in [0,1]$. Then $f_n(x) \conv{} f(x)$ and $f_n(0) \conv{} f(0)$ and $f_n' \conv{u} g$. Applying the DCT to this sequence of integrable functions that converges uniformly to an integrable function on a finite measure space (a previous exercise) we have that
			\begin{align*}
				f_n(x) - f_n(0) 
				&= \int_{[0,x]} f_n' dm \\
				& \conv{} \int_{[0,x]} g dm \\ 
			\end{align*} 
			Since $f_n(x) - f_n(0) \conv{} f(x) - f(0)$, we know that $$f(x) - f(0) = \int_{[0,x]} g dm$$. Thus $Tf = g$ and $\Gam(T)$ is closed. \\
			Suppose for the sake of contradiction that $T$ is bounded. Then there exists $C \geq 0$ such that for each $f \in X$, $\|T f \|\leq C \|f \|$. Choose $n \in \N$ such that $n > C$. Define $f \in X$ by $f(x) = x^n$. Then $\|f \|= 1$ and 
			\begin{align*}
				\|Tf \|
				&= \|f' \|\\
				&= n \\
				&> C \\
				&= C \|f \|
			\end{align*}
			which is a contradiction. So $T$ is not bounded.
		\end{enumerate}
	\end{proof}
	
	\begin{ex}
		Let $X, Y$ be Banach spaces and $T \in L(X,Y)$. Then $X/\ker T \cong T(X)$ iff $T(X)$ is closed.
	\end{ex}
	
	\begin{proof}
		Since $X$ is a banach space and $T$ is continuous, we have that $\ker T$ is closed and $X/ \ker T$ is a Banach space. Suppose that $X/ \ker T \cong T(X)$. Then $T(X)$ is complete. Since $Y$ is complete, this implies that $T(X)$ is closed. \\
		Conversely Suppose that $T(X)$ is closed. Then $T(X)$ is complete. Define $S: X/ \ker T \rightarrow T(X)$ by $S(x + \ker T) = T(x)$. A previous exercise tells us that the map $S: X/ \ker T \rightarrow T(X)$ defined by $S(x + \ker T) = T(x)$ is a bounded linear bijection. Since $T(X)$ is complete and $S$ is surjective, $S^{-1}$ is bounded and thus $S$ is an isomorphism.   
	\end{proof}
	
	\begin{ex}
		Let $X$ be a separable Banach space. Define $B_X = \{x \in X: \|x \|< 1\}$. Let $(x_n)_{n \in \N} \subset B_X $ a dense subset of the unit ball and $\mu$ the counting measure on $(\N, \MP(\N))$. Define $T: L^1(\mu) \rightarrow X$ by $$Tf = \sum_{n=1}^{\infty}f(n)x_n$$ Then 
		\begin{enumerate}
			\item $T$ is well defined and $T \in L(L^1(\mu), X)$
			\item $T$ is surjective
			\item There exists a closed subspace $K \subset L^1(\mu)$ such that $L^1(\mu)/K \cong X$ 
		\end{enumerate} 
	\end{ex}
	
	\begin{proof}
		\begin{enumerate}
			\item Let $f \in L^1(\mu)$. Since $X$ is complete and 
			\begin{align*}
				\sum_{n=1}^{\infty}\|f(n)x_n \|
				& = \sum_{n=1}^{\infty} \vert f(n) \vert \|x_n \|\\
				& \leq \sum_{n=1}^{\infty} \vert f(n) \vert \\
				&< \infty 
			\end{align*}
			we have that $\sum_{n=1}^{\infty} f(n)x_n $ converges and thus $Tf \in X$. Hence $T$ is well defined. \vspace{.5cm}\\
			Clearly $T$ is linear. Let $f \in L^1(\mu)$. Then
			\begin{align*}
				\|Tf \|
				&= \| \sum_{n=1}^{\infty} f(n)x_n \|\\
				& \leq \sum_{n=1}^{\infty} \|f(n)x_n \|\\
				& \leq \sum_{n=1}^{\infty} \vert f(n) \vert \\
				&= \|f \|_1
			\end{align*}
			So $T$ is bounded with $\|T \|\leq 1$.\vspace{.5cm}\\
			\item Let $x \in X$. Suppose that $\|x \|< 1$. Then $x \in B_X$. So there exists $n_1 \in \N$ such that $\|x - x_{n_1} \|< \frac{1}{2}$. Then $2(x-x_{n_1}) \in B_X$. Since for each $j \in \N$, $B_X\setminus (x_n)_{n=1}^j$ is dense in $B_X$, there exists $n_2 \in \N$ such that $x_{n_2} \not \in (x_n)_{n=1}^{n_1}$ and $\|2(x- x_{n_1}) - x_{n_2} \|< \frac{1}{2}$ which implies that $\|x- (x_{n_1} - \frac{1}{2}x_{n_2}) \|< \frac{1}{4}$. \vspace{.5cm}\\ 
			Proceed inductively to obtain a subsequence $(x_{n_k})_{k \in \N}$ such that for each $k \geq 2$, $x_{n_k} \not \in (x_n)_{n=1}^{n_{k-1}}$ and $\|x - \sum_{j=1}^k 2^{1-j}x_{n_j} \|< \frac{1}{2^k}$. Then $x = \sum_{k=1}^{\infty}2^{1-k}x_{n_k}$. \vspace{.5cm} \\ 
			Define $f:\|\rightarrow \C$ by $f = \sum_{k=1}^{\infty}2^{1-k}\chi_{\{n_k\}}$. Then $\|f \|_1 = \sum_{k=1}^{\infty}2^{1-k}< \infty$, so $f \in L^1(\mu)$ and $Tf = \sum_{k=1}^{\infty}2^{1-k}x_{n_k} = x$. Now, suppose that $\|x \|\geq 1$, then $\frac{1}{2\|x \|}x \in B_X$. The above argument shows that there exists $f \in L^1(\mu)$ such that $Tf = \frac{1}{2\|x \|}x$. Then $2 \|x \|f \in L^1(\mu)$ and $T(2 \|x \|f) = 2 \|x \|Tf =x$. \\
			So for each $x \in X$, there exists $f \in L^1(\mu)$ such that $Tf = x$ and thus $T$ is surjective. 
			\item Since $X$ is a Banach space and $T$ is surjective, the previous exercise implies that $L^1(\mu)/\ker T \cong X$. 
		\end{enumerate}
	\end{proof}
	
	\begin{ex}
		Let $X, Y$ be Banach spaces and $T:X \rightarrow Y$ a linear map. If for each $f \in Y^*$, $f \circ T \in X^*$, then $T \in L(X,Y)$. 
	\end{ex}
	
	\begin{proof}
		Suppose that for each $f \in Y^*$, $f \circ T \in X^*$. Let $x \in X$, 
	\end{proof}
	
	
	\subsection{Hilbert Spaces}
	
	\begin{defn}
		Let $H$ be a vector space and $\l \cdot, \cdot \r: H \rightarrow \C$. Then $\l \cdot, \cdot \r$ is said to be an \textbf{inner product} on $H$ if for each $x,y,z \in H$and $c \in \C$
		\begin{enumerate}
			\item $\l x , y + cz\r = \l x , y \r + c\l x , z\r $
			\item $\l x , y \r = \l y , x\r^*$
			\item $\l x , x \r \geq 0$
			\item if $\l x ,x \r = 0$, then $x = 0$.  
		\end{enumerate}
	\end{defn}



	\subsection{Banach Algebras}
	
	\begin{defn}
		Let $X$ be a Banach space and an associative algebra. Then $X$ is said to be a \textbf{Banach algebra} if for each $S,T \in X$, $\|ST \|\leq \|S \|\|T \|$. If there exists $I \in X$ such that $I \neq 0$ and for each $T \in X$, $IT = TI = T$, then $X$ is said to be \textbf{unital} with identity $I$. An element $T \in X$ is said to be \textbf{invertible} if there exists $S \in X$ such that $TS=ST = I$.
	\end{defn}
	
	\begin{ex}
		Let $X$ be a unital Banach algebra. Then $\|I \|\leq 1$. 
	\end{ex}
	
	\begin{proof}
		Since $I \neq 0$, $\|I \|\neq 0$. By definition, $$\|I \|= \|I I \|\leq \|I \|\|I \|$$ Hence $1 \leq \|I \|$.
	\end{proof}
	
	\begin{note}
		If $X$ is a Banach space, then a previous exercise implies that $L(X,X)$ equipped with composition is a unital Banach algebra where $I$ is the identity operator. It is easy to see that $\|I \|=1$.
	\end{note}
	
	\begin{note}
		Let $X$ be a  Banach algebra. Then the set of invertible elements in $X$ is a group.  
	\end{note}
	
	\begin{ex}
		Let $X$ be a Banach algebra. Then mulitplication is continuous. 
	\end{ex}
	
	\begin{proof}
		Let $(S_1,T_1) \in X \times X$ and $\ep > 0$. Choose $\del = \min\{\frac{\ep}{2(\|S_1 \|+ \|T_1 \|+1)}, \frac{\sqrt{\ep}}{\sqrt{2}}\}$. Let $(S_2, T_2) \in X \times X$. Suppose that $$\|(S_1, T_1) = (S_2, T_2) \|= \max \{ \|S_2 -S_2 \|, \|T_1 - T_2 \|\} < \del$$. Then 
		\begin{align*}
			\|S_1T_1 - S_2T_2 \|
			&= \|S_1T_1 - S_2T_1 +S_2T_1 - S_2T_2 \|\\
			& \leq \|S_1 -S_2 \|\|T_1 \|+ \|S_2 \|\|T_1 - T_2 \|\\
			& \leq \|S_1 -S_2 \|\|T_1 \|+ \big( \|S_1-S_2 \|+ \|S_1 \|\big) \|T_1 - T_2 \|\\
			& \leq \del \|T_1 \|+(\del + \|S_1 \|) \del \\
			&= \del (\|S_1 \|+ \|T_1 \|) + \del^2 \\
			& < \frac{\ep}{2} + \frac{\ep}{2}\\
			&= \ep
		\end{align*}
	\end{proof}
	\newpage
	
	\newpage
	
	
	
	
	
	
\end{document}