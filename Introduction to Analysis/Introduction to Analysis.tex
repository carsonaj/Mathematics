\documentclass[12pt]{amsart}
\usepackage[margin=1in]{geometry} 
\usepackage{amsmath,amsthm,amssymb,setspace, mathtools}
\usepackage{physics}
\usepackage{tikz-cd} 

\usepackage{color}   %May be necessary if you want to color links
\usepackage{hyperref}
\hypersetup{
	colorlinks=true, %set true if you want colored links
	linktoc=all,     %set to all if you want both sections and subsections linked
	linkcolor=black,  %choose some color if you want links to stand out
	urlcolor=cyan
}


%
%
%
\newif\ifhideproofs
%\hideproofstrue %uncomment to hide proofs
%
%
%
%
\ifhideproofs
\usepackage{environ}
\NewEnviron{hide}{}
\let\proof\hide
\let\endproof\endhide
\fi

\theoremstyle{definition}
\newtheorem{definition}{Definition}[subsection]
\newtheorem{defn}[definition]{Definition}
\newtheorem{note}[definition]{Note}
\newtheorem{thm}[definition]{Theorem}
\newtheorem{lem}[definition]{Lemma}
\newtheorem{prop}[definition]{Proposition}
\newtheorem{cor}[definition]{Corollary}
\newtheorem{conj}[definition]{Conjecture}
\newtheorem{ex}[definition]{Exercise}


\newcommand{\al}{\alpha}
\newcommand{\gam}{\gamma}
\newcommand{\Gam}{\Gamma}
\newcommand{\be}{\beta} 
\newcommand{\ze}{\zeta} 
\newcommand{\del}{\delta} 
\newcommand{\Del}{\Delta}
\newcommand{\lam}{\lambda}  
\newcommand{\Lam}{\Lambda} 
\newcommand{\ep}{\epsilon}
\newcommand{\sig}{\sigma} 
\newcommand{\om}{\omega}
\newcommand{\Om}{\Omega}
\newcommand{\C}{\mathbb{C}}
\newcommand{\N}{\mathbb{N}}
\newcommand{\E}{\mathbb{E}}
\newcommand{\Z}{\mathbb{Z}}
\newcommand{\R}{\mathbb{R}}
\newcommand{\T}{\mathbb{T}}
\newcommand{\Q}{\mathbb{Q}}

\newcommand{\MA}{\mathcal{A}}
\newcommand{\MC}{\mathcal{C}}
\newcommand{\MB}{\mathcal{B}}
\newcommand{\MF}{\mathcal{F}}
\newcommand{\MG}{\mathcal{G}}
\newcommand{\ML}{\mathcal{L}}
\newcommand{\MN}{\mathcal{N}}
\newcommand{\MS}{\mathcal{S}}
\newcommand{\MP}{\mathcal{P}}
\newcommand{\ME}{\mathcal{E}}
\newcommand{\MT}{\mathcal{T}}
\newcommand{\MM}{\mathcal{M}}
\newcommand{\MI}{\mathcal{I}}
\newcommand{\MU}{\mathcal{U}}
\newcommand{\MO}{\mathcal{O}}
\newcommand{\MQ}{\mathcal{Q}}

\newcommand{\tbf}[1]{\textbf{#1}}
\newcommand{\ol}[1]{\overline{#1}}

\newcommand{\ui}{[0,1]}
\newcommand{\p}{\partial}

\newcommand{\io}{\text{ i.o.}}
%\newcommand{\ev}{\text{ ev.}}
\renewcommand{\r}{\rangle}
\renewcommand{\l}{\langle}

\newcommand{\RG}{[0,\infty]}
\newcommand{\Rg}{[0,\infty)}
\newcommand{\Ru}{(\infty, \infty]}
\newcommand{\Rd}{[\infty, \infty)}
\newcommand{\Ll}{L^1_{\text{loc}}(\R^n)}

\newcommand{\limfn}{\liminf \limits_{n \rightarrow \infty}}
\newcommand{\limpn}{\limsup \limits_{n \rightarrow \infty}}
\newcommand{\limn}{\lim \limits_{n \rightarrow \infty}}
\newcommand{\convt}[1]{\xrightarrow{\text{#1}}}
\newcommand{\conv}[1]{\xrightarrow{#1}} 
\newcommand{\seq}[2]{(#1_{#2})_{#2 \in \N}}

\newcommand{\lsc}{lower semicontinuous}

\newcommand{\as}[1]{\overset{#1}{\sim}}
\newcommand{\astx}[1]{\overset{\text{#1}}{\sim}}

\DeclareMathOperator{\supp}{supp}
\DeclareMathOperator{\sgn}{sgn}
\DeclareMathOperator{\spn}{span}
\DeclareMathOperator{\iso}{Iso}
\DeclareMathOperator{\id}{id}
\DeclareMathOperator{\Aut}{Aut}
\DeclareMathOperator{\Homeo}{Homeo}
\DeclareMathOperator{\Sym}{Sym}
\DeclareMathOperator{\cl}{cl}
\DeclareMathOperator{\Int}{Int}
\DeclareMathOperator{\bal}{bal}
\DeclareMathOperator{\cnv}{conv}
\DeclareMathOperator{\epi}{epi}

\DeclareMathOperator*{\argmax}{arg\,max}
\DeclareMathOperator*{\argmin}{arg\,min}


\newcommand{\lex}[1]{\label{ex:#1}}
\newcommand{\ld}[1]{\label{defn:#1}}
\newcommand{\rex}[1]{Exercise \ref{ex:#1}}
\newcommand{\rd}[1]{Definition \ref{defn:#1}}


\begin{document}
	
	\title{Introduction to Analysis}
	\author{Carson James}
	\maketitle
	
	\tableofcontents
	
	\section*{Preface}
	\begin{flushleft}
		\href{https://creativecommons.org/licenses/by-nc-sa/4.0/legalcode.txt}{cc-by-nc-sa}
	\end{flushleft}

	
	
	\newpage
	

	\newpage
	
	\section{Set Theory}
	\subsection{Product Sets}
	
	\begin{defn}
	Let $(X_{\al})_{\al \in A}$ be a collection of sets. We define the \tbf{Cartesian product}, denoted $\prod\limits_{\al \in A}X_{\al}$, by 
	\begin{equation*}
	\prod_{\al \in A}X_{\al} = \{f:A \rightarrow \bigcup_{\al \in A} X_{\al}: \text{ for each $\al \in A$, $f(\al) \in X_{\al}$}\}
	\end{equation*}
	\end{defn}
	
	\begin{defn}
	Let $(X_{\al})_{\al \in A}$ be a collection of sets.  For $\al \in A$, we define the \tbf{projection map onto $X_{\al}$}, denoted $\pi_{\al}:\prod\limits_{\al \in A}X_{\al} \rightarrow X_{\al}$, by 
	\begin{equation*}
	\pi_{\al}(f) = f(\al)
	\end{equation*}
	\end{defn}

	\begin{ex}
		Let $(A_{\lam})_{\lam \in \Lam}$ be a collection of sets and $B$ a set. Then 
		$$\bigg( \bigcup\limits_{\lam \in \Lam} A_{\lam} \bigg) \times B = \bigcup\limits_{\lam \in \Lam} (A_{\lam} \times B)$$
	\end{ex}

	\begin{proof}
		Let $(x,y) \in \bigg( \bigcup\limits_{\lam \in \Lam} A_{\lam} \bigg) \times B$. Then $x \in \bigcup\limits_{\lam \in \Lam} A_{\lam}$ and $y \in B$. Therefore, there exists $\lam \in \Lam$ such that $x \in A_{\lam}$. Hence 
		\begin{align*}
			(x,y) 
			& \in A_{\lam} \times B \\
			& \subset \bigcup\limits_{\lam \in \Lam} (A_{\lam} \times B)
		\end{align*}
		Thus $\bigg( \bigcup\limits_{\lam \in \Lam} A_{\lam} \bigg) \times B \subset  \bigcup\limits_{\lam \in \Lam} (A_{\lam} \times B)$. \\
		Conversely, let $(x,y) \in \bigcup\limits_{\lam \in \Lam} (A_{\lam} \times B)$. Then there exists $\lam \in \Lam$ such that $(x,y) \in A_{\lam} \times B$. Then 
		\begin{align*}
			x 
			& \in A_{\lam} \\
			& \subset \bigcup_{\lam \in \Lam} A_{\lam}
		\end{align*}
		and $y \in B$. Hence $(x,y) \in \bigg( \bigcup\limits_{\lam \in \Lam} A_{\lam} \bigg) \times B$. So $ \bigcup\limits_{\lam \in \Lam} (A_{\lam} \times B) \subset \bigg( \bigcup\limits_{\lam \in \Lam} A_{\lam} \bigg) \times B$.
	\end{proof}

	\begin{defn}
		Let $X, Y$ be sets and $U \subset X \times Y$. For each $(x_0, y_0) \in U$, we define $U_{x_0} = \{y \in Y: (x_0,y) \in U\}$ and $U^{y_0} = \{x \in X: (x,y_0) \in U\}$.
	\end{defn}

	\begin{defn}
		Let $X, Y$ and $Z$ be sets, $U \subset X \times Y$ and $f: U \rightarrow Z$. For each $(x_0, y_0) \in U$, we define $f_{x_0}: U_{x_0} \rightarrow Z$ and $f^{y_0}: U^{y_0} \rightarrow Z$ by $f_{x_0} = f(x_0, \cdot)$ and $f^{y_0} = f(\cdot, y_0)$.
	\end{defn}
	
	\begin{ex}
		Let $X, Y$ and $Z$ be sets, $U \subset X \times Y$, $f: U \rightarrow Z$ and $(x_0, y_0) \in U$. Then for each $V \subset Z$, $(f_{x_0})^{-1}(V) = (f^{-1}(V))_{x_0}$ and $(f^{y_0})^{-1}(V) = (f^{-1}(V))^{y_0}$.
	\end{ex}

	\begin{proof}
		Let $V \subset Z$. Then for each $x \in U^{y_0}$,
		\begin{align*}
			x \in (f^{y_0})^{-1}(V) 
			& \iff f^{y_0}(x) \in V \\
			& \iff f(x, y_0) \in V \\
			& \iff (x, y_0) \in f^{-1}(V) \\
			& \iff x \in (f^{-1}(V))^{y_0}
		\end{align*}
		So $(f^{y_0})^{-1}(V) = (f^{-1}(V))^{y_0}$. Similarly, $(f_{x_0})^{-1}(V) = (f^{-1}(V))_{x_0}$. 
	\end{proof}
	
	
	
	
	
	
	
	
	
	
	
	
	
	
	
	
	
	
	\newpage
	\subsection{Quotient Sets}
	\begin{defn}
	Let $X$ be a set and $\sim$ an equivalence relation on $X$. We define the \tbf{quotient set} of $X$ by $\sim$, denoted $X/ {\sim}$, by 
	\begin{equation*}
	X/ {\sim} = \{\bar{x}: x \in X\}
	\end{equation*}
	\end{defn}
	
	
	\section{Real and Complex Numbers}
	\begin{note}
		As a starting point, we will take as fact the existence of the \tbf{natural numbers} $$\N = \{1, 2, \cdots\}$$ the \tbf{integers} $$\Z = \{\cdots, -2, -2, 0, 1, 2, \cdots\}$$ and the \tbf{rational numbers} $$\Q = \bigg \{\frac{a}{b}: a \in \Z, b \in \N \bigg \}$$
	\end{note}
	\subsection{Real Numbers}
	
	\begin{defn} \ld{}
		Let $X$ be a set and $\leq$ a relation on $X$. Then $\leq$ is said to be a \tbf{total order} if for each $a,b,c \in X$,
		\begin{enumerate}
			\item $a \leq a$
			\item $a \leq b$ and $b \leq c$ implies that $a \leq  c$ 
			\item $a \leq b$ and $b \leq a$ implies that $a = b$ 
			\item $a \leq b$ or $b \leq a$
		\end{enumerate}
	\end{defn}

	\begin{ex} \lex{}
		We define the relation $\leq$ on $\Q$ defined by $$\frac{a}{b} \leq \frac{c}{d} \hspace{.2cm} \text{iff} \hspace{.2cm} ad \leq bc$$ Then $\leq$ is a total order of $\Q$.
	\end{ex}

	\begin{proof} Let $\frac{a}{b}, \frac{c}{d}, \frac{e}{f} \in \Q$. Then
		\begin{enumerate}
			\item  $\frac{a}{b} \leq \frac{a}{b}$ since $ab \leq ab$. 
			\item if $\frac{a}{b} \leq \frac{c}{d}$ and $\frac{c}{d} \leq  \frac{e}{f}$, then $ad \leq bc$ and $ cf \leq de$. Multiplying the first inequality by $f$ and the second inequality by $b$, we obtain $adf \leq bcf \leq bde$. Dividing both sides by $d$ yields $af \leq be$. Hence $\frac{a}{b} \leq \frac{e}{f}$. 
			\item if $\frac{a}{b} \leq \frac{c}{d}$ and $\frac{c}{d} \leq \frac{a}{b}$, then $ad \leq bc$ and $bc \leq ab$. This implies that $ad = bc$. Hence $\frac{a}{b} = \frac{c}{d}$.
			\item 
		\end{enumerate}
	\end{proof}
	
	
	
	
	
	
	
	
	
	
	
	
	
	
	
	
	
	\newpage
	\section{Metric Spaces}
	\subsection{Introduction}
	\begin{defn} \ld{}
	Let $M$ be a set and $d: M \times M \rightarrow \R$. Then $d$ is said to be a \tbf{metric on $M$} if for each $x,y,z \in M$, 
	\begin{enumerate}
	\item $d(x,y) = 0$ iff $x = y$
	\item $d(x, y) \leq d(x, z) + d(z, y)$
\end{enumerate}	 
	\end{defn}	
	
	\begin{ex} \lex{}
	Let $M$ be a set and $d: M \times M \rightarrow \R$ a metric on $M$. Then for each $x,y \in M$, $d(x,y) \geq 0$. 
	\end{ex}
	
	\begin{proof}
	Let $x, y, z \in M$. Then $d(x,z) \leq d(x, y) + d(y,z)$. This implies that $d(x,z) - d(x, y) \leq d(y, z)$. Since $z$ is arbitrary, taking $z=x$, we obtain 
	\begin{align*}
	d(x,x) - d(x, y) \leq d(y, x)
	& \implies - d(x, y) \leq d(x, y) \\
	& \implies 0 \leq 2 d(x,y) \\
	& \implies d(x,y) \geq 0
	\end{align*}
	\end{proof}	

	\begin{defn} \ld{}
		Let $M$ be a set and $d: M \times M \rightarrow \Rg$ a metric. Then $(M, d)$ is called a \tbf{metric space}.
	\end{defn}	
	
	\begin{note}
		We usually suppress the metric and write $M$ in place of $(M, d)$.
	\end{note}	

	\begin{defn}
		Let $M$ be a metric space, $x \in M$ and $r > 0$. We define the 
		\begin{itemize}
			\item \tbf{open ball of radius $r$ at $x$}, denoted $B(x, r)$, by $$B(x, r) = \{y \in M: d(x,y) < r\}$$
			\item \tbf{closed ball of radius $r$ at $x$}, denoted $\bar{B}(x, r)$, by $$\bar{B}(x, r) = \{y \in M: d(x,y) \leq r\}$$
		\end{itemize}
	\end{defn}

	\begin{defn}
		Let $M$ be a metric space and $A \subset M$. Then $A$ is said to be 
		\begin{itemize}
			\item \tbf{open} if for each $x \in A$, there exists $r > 0$ such that $B(x, r) \subset A$
			\item \tbf{closed} if $A^c$ is open
		\end{itemize}  
	\end{defn}

	\begin{defn}
		Let $M$ be a metric space. Then $M$ is said to be \tbf{separable} if there exists $D \subset M$ such that $D$ is countable and for each $x \in M$ and $\ep >0$, there exists $y \in D$ such that $d(x, y) < \ep$.
	\end{defn}
	
	\begin{ex}
		Let $M$ be a metric space. If $M$ is separable, then for each $A \subset M$, if $A$ is open, then 
		\begin{enumerate}
			\item there there exist $(x_n)_{n \in \N} \subset X$ and $(r_n)_{n \in \N} \subset (0, \infty)$ such that $$A = \bigcup\limits_{n \in \N} B(x_n, r_n)$$
			i.e. $A$ is a countable union of open balls
			\item there there exist $(x_n)_{n \in \N} \subset X$ and $(r_n)_{n \in \N} \subset (0, \infty)$ such that $$A = \bigcup\limits_{n \in \N} \bar{B}(x_n, r_n)$$
			i.e. $A$ is a countable union of closed balls.
		\end{enumerate}
	\end{ex}

	\begin{proof}
		Suppose that $M$ is separable. Then there exists $(x_n)_{n \in \N} \subset M$ such that for each $x \in M$ and $\ep >0$, there exists $N \in \N$ such that $d(x, x_N) < \ep$. Let $A \subset X$. Suppose that $A$ is open.
		\begin{enumerate}
			\item  Set 
			$$\MB = \{B(x_n, r): r \in \Q \text{ and } B(x_n, r) \subset A\}$$ 
			Note that $\MB$ is countable. Let $x \in A$. Since $A$ is open, there exists $s \in \R$ such that $B(x, s) \subset A$. Then there exists  $r \in \Q \cap (0, r)$. Choose $N \in \N$ such that $d(x, x_N) < r/2$. Let $y \in B(x_N, r/2)$, then 
			\begin{align*}
				d(x, y) 
				& \leq d(x, x_N) + d(x_N, y) \\
				& < r/2 + r/2 \\
				& = r
			\end{align*}
			Therefore 
			\begin{align*}
				x 
				& \in B(x_N, r/2) \\
				& \subset B(x, r) \\
				& \subset A
			\end{align*}
			Hence $B(x_N, r/2) \in \MB$ and $x \in \bigcup\limits_{B \in \MB}B$. Since $x \in A$ is arbitrary, $A \subset \bigcup\limits_{B \in \MB}B$.
			\item Similar, but take $r/4$ instead of $r/2$.
		\end{enumerate}
	\end{proof}
	
	\begin{defn} \ld{}
	Let $(M,d)$ be a metric space and $A,B \subset M$. We define the \tbf{distance between $A$ and $B$}, denoted $d(A,B)$, by $$d(A,B) = \inf_{\substack{a \in A \\ b \in B}} d(a,b)$$
	\end{defn}
	
	\begin{ex} \lex{}
	Let $(M,d)$ be a metric space. Then for each $A,B \subset M$ and $c \in M$, $$d(A,B) \leq d(A,c) + d(c, B)$$
	\end{ex}
	
	\begin{proof}
	Let $A,B \subset M$, $c \in M$ and $\ep>0$. Choose $a \in A$ and $b \in B$ such that $d(a,c) < d(A,c)+ \ep/2$ and  $d(c,b) < d(c,B)+ \ep/2$. Then 
	\begin{align*}
	d(A,B) 
	&\leq d(a,b) \\
	&\leq d(a,c) + d(c,b) \\
	&< d(A,c) + \frac{\ep}{2} + d(c,B) + \frac{\ep}{2} \\
	&= d(A,c) + d(c,B) + \ep
	\end{align*}
	Since $\ep >0$ is arbitrary, $d(A,B) \leq d(A,c) + d(c,B)$.
	\end{proof}
	
	\begin{defn} \ld{}
	Let $M$ be a set, $d_1, d_2: M \times M \rightarrow \Rg$ metrics on $M$. Then $d_1$ and $d_2$ are said to be 
	\begin{itemize}
	\item \tbf{topologically equivalent} if for each $(x_n)_{n \in \N} \subset M$ and $x \in M$, $x_n \conv{d_1} x$ iff $x_n \conv{d_2} x$ 
	\item\tbf{equivalent} if there exist $A, B > 0$ such that $$A d_1 \leq d_2 \leq B d_1$$
\end{itemize}		
	\end{defn}	
	
	\begin{defn} \ld{}
	Let $(X, d_X)$ and $(Y, d_Y)$ be metric spaces and $f: X \rightarrow Y$. Then $f$ is said to be \tbf{Lipchitz} if there exists $K \geq 0$ such that for each $a, b \in X$, $$d_Y(f(a), f(b)) \leq Kd_X(a,b)$$
	\end{defn}	
	
	\begin{ex} \lex{}
	Let $(X, d_X)$ and $(Y, d_Y)$ be metric spaces and $f: X \rightarrow Y$. If $f$ is Lipchitz, then $f$ is uniformly continuous.	
	\end{ex}
	
	\begin{proof}
	By definition, there exists $K \geq 0$ such that for each $a, b \in X$, $$d_Y(f(a), f(b)) \leq Kd_X(a,b)$$ Let $\ep >0$. Choose $\del = \ep / (K+1)$. Let $a, b \in X$. Suppose that $d_X(a,b) < \del$. Then 
	\begin{align*}
	d_Y(f(a), f(b)) 
	& \leq Kd_X(a,b) \\
	& < K \del \\
	&= K \frac{\ep}{K+1} \\
	&< \ep  
	\end{align*}
	\end{proof}
	
	\begin{defn} \ld{}
	Let $(X, d_X)$ and $(Y, d_Y)$ be metric spaces and $f: X \rightarrow Y$ and $x_0 \in X$. Then $f$ is said to be \tbf{locally Lipchitz at $x_0$} if there exists $U \in \MN_{x_0}$ such that $f$ is Lipschitz on $U$.
	\end{defn}
	
	\begin{defn} \ld{}
	Let $(X, d_X)$ and $(Y, d_Y)$ be metric spaces and $f: X \rightarrow Y$. Then $f$ is said to be \tbf{locally Lipchitz} if for each $x_0 \in X$, $f$ is locally Lipschitz at $x_0$.
	\end{defn}
	
	
	\begin{defn} \ld{}
		Let $X, Y$ be metric spaces and $T : X \rightarrow Y$. Then $T$ is said to be an \tbf{isometry} if for each $x_1, x_2 \in X$, $d( Tx_1, Tx_2) = d(x_1,x_2) $.
	\end{defn}
	
	\begin{ex} \lex{}
		Let $X,Y$ be metric spaces and $T:X \rightarrow Y$ and isometry. Then $T$ is injective.
	\end{ex}
	
	\begin{proof}
		Let $x_1, x_2 \in X$. Suppose that $Tx_1=Tx_2$. Then $0= d( Tx_1, Tx_2) = d(x_1,x_2)$. So $x_1 = x_2$. Hence $T$ is injective.
	\end{proof}
	
	\begin{note}
		Let $X,Y$ be metric spaces and $T:X \rightarrow Y$ an isometry. Then $T$ is clearly continuous. If $T$ is surjective, then $T^{-1}$ is an isometry and therefore continuous. Hence $T$ is a homeomorphism.
	\end{note}
	
	\begin{defn} \ld{}
	Let $(M,d)$ be a metric space. Then $(M,d)$ is said to be a \tbf{Polish space} if $(M,d)$ is complete and separable. 
	\end{defn}
	
	
	
	\begin{ex} \lex{}
	Let $(X, d)$ be a compact metric space, $E \subset X$ closed, $U \subset X$ open. Suppose that $E \subset U$. Then there exists $\del >0$ such that for each $x \in E$, $B(x, \del) \subset U$.
	\end{ex}	
	
	\begin{proof}
	Since $X$ is compact, $E$ and $U^c$ are compact. Then there exist $x_0 \in E$ and $y_0 \in U^c$ such that $d(E, U^c) = d(x_0,y_0)$. Since $E \cap U^c = \varnothing$, $x_0 \neq y_0$ and $d(E, U^c) >0$. Put $\ep = d(E, U^c)$ and $\del = \frac{\ep}{2}$.  Let $x \in E$, $w \in B(x, \del)$ and $y \in U^c$. Then 
	\begin{align*}
	d(y, w) 
	&\geq d(y, x) - d(x, w) \\
	&> \ep - \del \\
	&= \ep - \frac{\ep}{2} \\
	&= \frac{\ep}{2} \\
	&> 0
\end{align*}	  
	So $y \neq w$. Since and $y \in U^c$ and $w \in B(x, \del)$ are arbitrary, $B(x, \del) \subset U$.
	\end{proof}
	
	\begin{defn} \ld{}
	Let $S$ be a set, $(M, d)$ a metric space and $B(S, M) = \{f: S \rightarrow M: f \text{ is bounded} \}$. We define the \tbf{supremum metric}, denoted $d_u:B(S,M) \times B(S,M) \rightarrow \Rg$, by $$d_u(f, g) = \sup_{x \in X}d(f(x), g(x)) $$ 
	\end{defn}
	
	\begin{ex} \lex{211111111}
	Let $X$ be a set, $(Y, d_Y)$, $(Z, d_Z)$ metric spaces, $(f_n)_{n \in \N} \subset B(X, Y)$, $f \in B(X, Y)$ and $g \in C(Y, Z)$. Suppose that $g$ is uniformly continuous. If $f_n \convt{u} f$, then $g \circ f_n \convt{u} g \circ f$. 
	\end{ex}
	
	\begin{proof}
	Suppose that $f_n \convt{u} f$. Let $\ep >0$. Uniform continuity of $g$ implies that there exists $\del >0$ such that for each $y_1, y_2 \in Y$, $d_Y(y_1, y_2) < \del$ implies that $d_Z(g(y_1), g(y_2)) < \ep/2$.  Uniform convergence implies that there exists $N \in \N$ such that for each $n \in \N$, $n \geq \N$ implies that $d_u(f_n, f) < \del/2$. Let $n \in \N$. Suppose that $n \geq N$. Let $x \in X$. Then $d_Y(f_n(x), f(x)) < \del$. This implies that $d_Z(g(f_n(x)), g(f(x))) < \ep/2$. Hence $\sup\limits_{x \in X} d_Z(g \circ f_n(x), g \circ f(x)) \leq \ep/2$. Thus $d_u(g \circ f_n , g \circ f) < \ep$. So $g \circ f_n \convt{u} g \circ f$.
	\end{proof}
	
	\begin{defn} \ld{}
	Let $(X, d)$ be a metric space. Define
	\begin{enumerate}
	\item $\Aut(X) = \{\sig:X \rightarrow X: \sig \text{ is a homeomorphism}\}$
	\item $\Aut(X, d) = \{\sig:X \rightarrow X: \sig \text{ is an isometric isomorphism}\}$
	\end{enumerate}
	\end{defn}
	
	\begin{ex} \lex{}
	Let $(X, d)$ be a compact metric space, $E \subset X$ closed, $U \subset X$ open. Suppose that $E \subset U$. Let $(f_n)_{n \in \N} \in \Aut(X)$, $f \in \Aut(X)$.  Suppose that $f_n \convt{u} f$. Then there exists $N \in \N$ such that for each $n \geq N$, $f(E) \subset f_n(U)$.
	\end{ex}
	
	\begin{proof}
	Since $f$ is a homeomorphism, $E$ is closed and $U$ is open, $f(E)$ is compact and $f(U)$ is open and $f(E) \subset f(U)$. Then $d(f(E), f(U^c)) >0$. Put $\ep = d(f(E), f(U^c))$. Choose $\del = \ep/2$. Then there exists $N \in \N$ such that for each $n \in \N$, $n \geq N$ implies that $\sup\limits_{z \in X} d(f(z), f_n(z)) < \del$. Let $n \geq N$, $x \in E$ and $w \in B(f(x), \del)$. For the sake of contradiction, suppose that $w \in f_n(U^c)$. Then there exist $p \in U^c$ such that $w = f_n(p)$. Put $z = f(p) \in f(U^c)$. Then 
	\begin{align*}
	\ep 
	&\leq d(f(x), z) \\ 
	&\leq d(f(x), w) + d(w, z) \\
	& = d(f(x), w) + d(f_n(p), f(p))  \\
	& < \del + \del \\
	& = \ep
	\end{align*}
	which is a contradiction. So $w \in f_n(U)$. Hence $B(f(x), \del) \subset f_n(U)$
	\end{proof}
	
	
	
	
	
	
	
	
	
	
	\newpage
	\subsection{Product Spaces}
	
	
	
	
	
	
	
	
	
	
	
	
	
	\newpage
	\section{Topology}
	
	\subsection{Introduction}
	
	\begin{defn} \ld{31001}
	Let $X$ be a set and $\MT \subset \MP(X)$. Then $\MT$ is said to be a \tbf{topology on $X$} if 
	\begin{enumerate}
	\item $X$, $\varnothing \in \MT$ 
	\item for each $(U_{\al})_{\al \in A} \subset \MT$, $$\bigcup_{\al \in A}U_{\al} \in \MT$$
	\item for each $(U_j)_{j=1}^n \subset \MT$, $$\bigcap_{j=1}^n U_{j} \in \MT$$
	\end{enumerate}
	\end{defn}		
	
	\begin{ex} \lex{31002} 
		Let $X$ be a set and $(\MT_{i})_{i \in I}$ a collection of topologies on $X$. Then $\bigcap\limits_{i \in I}\MT_i$ is a topology on $X$.
	\end{ex}
	
	\begin{proof}\
	\begin{enumerate}
	\item Since for each $i \in I$, $X, \varnothing \in \MT_{i}$, we have that $X, \varnothing \in \bigcap\limits_{i \in I}\MT_i$.
	\item Let $(U_{\al})_{\al \in A} \subset \bigcap\limits_{i \in I}\MT_i$. Then for each $i \in I$, $(U_{\al})_{\al \in A} \subset T_i$. So for each $i \in I$, $\bigcup\limits_{\al \in A}U_{\al} \in \MT_i$. Thus $\bigcup\limits_{\al \in A}U_{\al} \in \bigcap\limits_{i \in I}\MT_i$.
	\item Let $(U_{j})_{j=1}^n \subset \bigcap\limits_{i \in I}\MT_i$. Then for each $i \in I$, $(U_{j})_{j=1}^n \subset T_i$. So for each $i \in I$, $\bigcap\limits_{j=1}^n U_{j} \in \MT_i$. Thus $\bigcap\limits_{j=1}^n U_{j} \in \bigcap\limits_{i \in I}\MT_i$.
	\end{enumerate}
	So $\bigcap\limits_{i \in I}\MT_i$ is a topology on $X$.
	\end{proof}
	
	\begin{defn} \ld{31003}
	Let $X$ be a set and $\ME \subset \MP(X)$. Set 
	\begin{equation*}
	\MS = \{\MT \subset \MP(X): \MT \text{ is a topology 	on $X$ and $\ME \subset \MT$}\}
	\end{equation*}	 
We define the \tbf{topology generated by $\ME$} on $X$, denoted $\tau(\ME)$, by $$\tau(\ME) = \bigcap_{\MT \in \MS} \MT$$
	\end{defn}
	
	\begin{defn} \ld{31004}
	Let $(X,d)$ be a metric space. We define the \tbf{metric topology on X}, denoted $\MT_d$, by $$\MT_d = \tau(\{B(x, \del): x \in X, \del >0\})$$
	\end{defn}
	
	\begin{defn} \ld{31005}
	Let $X$ be a set and $\MT \subset \MP(X)$ a topology on $X$, $x \in X$ and $\MB_x \subset \MT$. Then $\MB_x$ is said to be a \tbf{local basis for $\MT$ at $x$} if 
	\begin{enumerate}
	\item for each $U \in \MB_x$, $x \in U$
	\item for each $V \in \MT$, if $x \in V$, then there exists $U \in \MB_x$ such that $U \subset V$
	\end{enumerate}
	\end{defn}
	
	\begin{ex} \lex{31006}
	Let $(X,d)$ be a metric space and $x \in X$. Set $\MB_x = \{B(x, \del): \del > 0\}$. Then $\MB_x$ is a local basis for $\MT_d$ at $x$. \\
	\tbf{FINISH!!! right now not well defined.}
	\end{ex}	
	
	\begin{proof}
	Clear.
	\end{proof}
	
	\begin{defn} \ld{31006.5}
	Let $X$ be a set and $\MT \subset \MP(X)$ a topology on $X$ and $\MB \subset \MT$. Then $\MB$ is said to be a \tbf{basis for $\MT$} if for each $V \in \MT$ and $x \in V$, there exists $U \in \MB$ such that $x \subset U \subset V$.
	\end{defn}

	\begin{ex} \lex{31007}
	Let $X$ be a set and $\MT \subset \MP(X)$ a topology on $X$ and $\MB \subset \MT$. Then $\MB$ is a basis for $\MT$ iff for each $x \in X$, there exists $\MB_x \subset \MB$ such that $\MB_x$ is a local basis for $\MT$ at $x$. 
	\end{ex}

	\begin{proof}
		Suppose that $\MB$ is a basis for $\MT$. Let $x \in X$. Define $\MB_x = \{U \in \MB: x \in U\}$. 
		\begin{enumerate}
			\item By definition, for each $U \in \MB_x$, $x \in U$
			\item Let $V \in \MT$. Suppose that $x \in V$. Since $\MB$ is a basis, there exists $U \in \MB$ such that $x \in U \subset V$. By definition, $U \in \MB_x$.
		\end{enumerate}
		Hence $\MB_x$ is a local basis for $\MT$ at $x$. \\
		Conversely, suppose that for each $x \in X$, there exists $\MB_x \subset \MB$ such that $\MB_x$ is a local basis for $\MT$ at $x$. Let $V \in \MT$ and $x \in V$. By assumption, there exists $\MB_x \subset \MB$ such that $\MB_x$ is a local basis for $\MT$ at $x$. Since $\MB_x$ is a local basis for $\MT$ at $x$, there exists $U \in \MB_x \subset \MB$ such that $x \in U \subset V$. Hence $\MB$ is a basis for $\MT$. 
	\end{proof}
	
	\begin{ex} \lex{31008}
	Let $X$ be a set and $\MT \subset \MP(X)$ a topology on $X$ and $\MB \subset \MT$. Then $\MB$ is a basis for $\MT$ iff for each $V \in \MT$, there exists a collection $\MC \subset \MB$ such that $$V = \bigcup\limits_{U \in \MC} U$$
	\end{ex}
	
	\begin{proof}
	Suppose that $\MB$ is a basis for $\MT$. Let $V \in \MT$. Since since $\MB$ is a basis for $\MT$, for each $x \in V$, there exists $U_x \in \MB$ such that $x \in U_x \subset V$. Then $(U_x)_{x \in U} \subset \MB$ satisfies $V = \bigcup\limits_{x \in U} U_x$. \\
	Conversely, suppose that for each $V \in \MT$, there exists a collection $\MC \subset \MB$ such that $V = \bigcup\limits_{U \in \MC} U$. Let $V \in \MT$ and $x \in V$. By assumption, there exists a collection $\MC \subset \MB$ such that $V = \bigcup\limits_{U \in \MC} U$. Since $x \in V$, there exists $U \in \MC$ such that $x \in U$. Hence there exists $U \in \MB$ such that $x \in U \subset V$. Then $\MB$ is a basis for $\MT$.
	\end{proof}
	
	\begin{ex} \lex{31008.1}
	Let $X$ be a set and $\MT_1, \MT_2 \subset \MP(X)$ topologies on $X$ and $\MB \subset \MT_1$. Suppose that $\MT_1 \subset \MT_2$. If $\MB$ is a basis for $\MT_2$, then $\MB$ is a basis for $\MT_1$.  
	\end{ex}
	
	\begin{proof}
	Suppose that $\MB$ is a basis for $\MT_2$. Let $V \in \MT_1$. Then $V \in \MT_2$. Since $\MB$ is a basis for $\MT_2$, the previous exercise implies that there exists a collection $(U_{\al})_{\al \in A} \subset \MB$ such that $V = \bigcup\limits_{\al \in A} U_{\al}$. Thus the previous exercise implies that $\MB$ is a basis for $\MT_1$. 
	\end{proof}
	
	\begin{ex} \lex{31009}
	Let $X$ be a set and $\MB \subset \MP(X)$. Then there exists a topology $\MT$ on $X$ such that $\MB$ is a basis for $\MT$ iff 
	\begin{enumerate}
	\item for each $x \in X$, there exists $V \in \MB$ such that $x \in V$
	\item for each $U, V \in \MB$, if $x \in U \cap V$, then there exists $W \in \MB$ such that $x \in W \subset U \cap V$
	\end{enumerate}
	\end{ex}
	
	\begin{proof}
	Suppose that there exists a topology $\MT$ on $X$ such that $\MB$ is a basis for $\MT$. Then conditions $(1)$ and $(2)$ are clear by \rd{31005} and \rd{31007}. \\
	Conversely, suppose that $(1)$ and $(2)$ are satisfied. Define $\MT \subset \MP(X)$ by 
	$$\MT =  \{ U \subset X: \text{ for each $x \in U$, there exists $V \in \MB$ such that $x \in V \subset U$} \}$$
	Trivially $\varnothing \in \MT$. By condition $(1)$, $X \in \MT$. Let $(U_{\al})_{\al \in A} \subset \MT$ and $x \in \bigcup\limits_{\al \in A}U_{\al}$. Then there exists $\al \in A$ such that $x \in U_{\al}$. Hence there exists $V \in \MB$ such that 
	\begin{align*}
	x 
	& \in V \\
	& \subset U_{\al} \\
	& \subset \bigcup\limits_{\al \in A}U_{\al}
	\end{align*}
	So $\bigcup\limits_{\al \in A}U_{\al} \in \MT$. Let $(U_j)_{j=1}^n \subset \MT$ and $x \in \bigcap\limits_{j=1}^n U_j$. Then in particular, $U_1, U_2 \in \MT$ and $x \in  U_1 \cap U_2$. Then for $j \in \{1, 2\}$, there exists $V_j \in \MB$ such that $x \in V_j \subset U_j$. This implies that $x \in V_1 \cap V_2$ and by condition $(2)$, there exists $W \in \MB$ such that
	\begin{align*}
	x 
	& \in W \\
	& \subset V_1 \cap V_2 \\
	& \subset U_1 \cap U_2
	\end{align*}
	Therefore $U_1 \cap U_2 \in \MT$. Proceeding inductively, we obtain that $\bigcap\limits_{j=1}^n U_j \in \MT$.
	\end{proof}
	
	\begin{ex} \lex{31010}
	Let $X$ be a set and $\ME \subset \MP(X)$. Define $\MB \subset \MP(X)$ by 
	$$\MB = \{X, \varnothing\} \cup  \bigg \{\bigcap_{j=1}^n V_j: (V_j)_{j=1}^n \subset \ME \bigg \}$$ 
	Then 
	\begin{enumerate}
	\item $\MB$ is a basis for $\tau(\ME)$ 
	\item $$\tau(\ME) = \bigg \{ \bigcup_{\al \in A} V_{\al}: (V_{\al})_{\al \in A} \subset \MB \bigg \}$$ That is, each element of $\tau(\ME)$ is either $X, \varnothing$ or a union of finite intersections of elements of $\ME$. 
	\end{enumerate}
	
	\end{ex}
	
	\begin{proof}\
	\begin{enumerate}
	\item Referring to \rex{31009}, since $X \in \MB$, condition $(1)$ is satisfied and since for each $U, V \in \MB$, $U \cap V \in \MB$, condition $(2)$ is satisfied. Hence there exists a topology $\MT$ on $X$ such that $\MB$ is a basis for $\MT$. Since $\MB \subset \MT$ and $\tau(\ME) = \tau(\MB)$, we have that $\tau(\ME) \subset \MT$. Since $\MB$ is a basis for $\MT$ and $\MB \subset \tau(\ME)$, \rex{31008.1} implies that $\MB$ is a basis for $\tau(\ME)$.
	\item \rex{31008} implies that $$\tau(\ME) = \bigg \{ \bigcup_{\al \in A} V_{\al}: (V_{\al})_{\al \in A} \subset \MB \bigg \}$$
	\end{enumerate}
	
	\end{proof}
	
	\begin{defn} \ld{31011}
	Let $X$ be a set and $\MT$ a topology on $X$. Then $(X, \MT)$ is said to be a \tbf{topological space}. Let $U \subset X$. Then $U$ is said to be \tbf{open} if $U \in \MT$ and $U$ is said to be \tbf{closed} if $U^c$ is open. We define $\MF_T = \{C \subset X: C^c \in \MT  \}$. 
	\end{defn}
	
	\begin{defn} \ld{31012}
	Let $X$ be a topological space and $S,N \subset X$. Then $N$ is said to be a \tbf{neighborhood} of $S$ if there exists $U \subset X$ such that $U$ is open and $S \subset U \subset N$. For $S \in X$, we denote the set of neighborhoods of $S$ by $\MN_S$.
	\end{defn}
	
	\begin{defn} \ld{31013}
	Let $X$ be a topological space and $A \subset X$. Set $\MU_{A} = \{U \subset X:U \subset A \text{ and $U$ is open}\}$ and $\MC_{A} = \{U \subset X: A \subset U \text{ and $U$ is closed}\}$. \\
	We define the \tbf{interior of A}, denoted $\Int A$, by $$\Int A = \bigcup_{U \in \MU_{A}} U$$ 
	We define the \tbf{closure of A}, denoted $\cl A$, by $$\cl A = \bigcap_{U \in \MC_{A}} U$$ 
	\end{defn}
	
	\begin{defn} \ld{31014}
	Let $X$ be a topological space and $A \subset X$. Then 
	\begin{enumerate}
	\item $A$ is open iff $A = \Int A$ 
	\item $A$ is closed iff $A = \cl A$
	\end{enumerate}
	\end{defn}
	
	\begin{proof}
	Clear.
	\end{proof}
	
	\begin{ex} \lex{31015}
	Let $X$ be a topological space and $A \subset X$. Then $( \Int A )^c = \cl A^c $.
	\end{ex}	
	
	\begin{proof}
	
	\end{proof}
	
	
	\begin{ex} \lex{31016}
	Let $X$ be a topological space, $A \subset X$ and $x \in X$. Then $A \in \MN_x$ iff $x \in \Int A$.
	\end{ex}
	
	\begin{proof}
	Suppose that $A \in \MN_x$. Then there exists $U \subset X$ such that $U$ is open and $x \in U \subset A$. By definition, $U \subset \Int A$. Conversely, suppose that $x \in \Int A$. Then by definition, $\Int A \in \MN_x$.
	\end{proof}
	
	\begin{ex} \lex{31017}
	Let $X$ be a topological space and $A \subset X$. Then $A$ is open iff for each $x \in A$, there exists $U \in \MN_x$ such that $U$ is open and $U \subset A$.
	\end{ex}
	
	\begin{proof}
	Suppose that $A$ is open. Let $x \in A$. Then $A \in \MN_x$, $A$ is open and $A \subset A$. Conversely, suppose that or each $x \in A$, there exists $U_x \in \MN_x$ such that $U$ is open and $U_x \subset A$. Then $$A = \bigcup\limits_{x \in A}U_x$$ is open. 
	\end{proof}
	
	\begin{defn} \ld{31018}
	Let $X$ be a topological space, $A \subset X$ and $x \in X$. Then $x$ is said to be a \tbf{limit point of $A$} if for each $U \in \MN_x$, $$A \cap (U \setminus \{x\}) \neq \varnothing$$  
	We define $A' = \{x \in A: \text{$x$ is a limit point of $A$}\}$.
	\end{defn}
	
	\begin{ex} \lex{31019}
	Let $X$ be a topological space and $A \subset X$. Then $\cl A = A \cup A'$. 
	\end{ex}	
	
	\begin{proof}
	Let $x \in A'$. For the sake of contradiction, suppose that $x \not \in \cl A$. Then there exists $C \subset X$ such thath $C$ is closed, $A \subset C$ and $x \not \in C$. Hence $x \in C^c \subset A^c$. Since $C^c$ is open, $x \in \Int A^c$. Since $x \in A'$ and $\Int A^c \in \MN_x$, $[\Int A^c \setminus \{x\}] \cap A \neq \varnothing$. This is a contradiction since $\Int A^c \setminus \{x\} \subset A^c$. So $x \not \in \cl A$ and $A' \subset \cl A$. Since $A \subset \cl A$, we have that $A \cup A' \subset \cl A $.\\ Conversely, let $x \in \cl A$. For the sake of contradiction, suppose that $x \not \in A \cup A'$. Then $x \in A^c \cap (A')^c$. Since $x \in (A')^c$, there exists $U \in \MN_x$ such that $(U \setminus \{x\}) \cap A = \varnothing$. Hence $U \setminus \{x\} \subset A^c$. Since $x \in A^c$, 
	\begin{align*}
	x
	& \in \Int U \\
	& \subset U \\ 
	&= (U \setminus \{x\}) \cup \{x\} \\
	& \subset A^c
\end{align*}	
	Since $A \subset (\Int U)^c$ which is closed, $x \in \cl A$ implies that $x \in (\Int U)^c$ which is a contradiction. So $x \in A \cup A'$ and $\cl A \subset A \cup A'$. Therefore $\cl A = A \cup A'$.
	\end{proof}
	
	
	
	
	
	
	
	
	
	
	
	
	
	
	
	
	
	
	
	
	\newpage
	\subsection{Continuous Maps}	
	
	\begin{defn} \ld{}
	Let $(X,\MA)$ and $(Y,\MB)$ be topological spaces and $f:X \rightarrow Y$. Then $f$ is said to be \tbf{continuous} if for each $B \in \MB$, $f^{-1}(B) \in \MA$.
	\end{defn}
	
	\begin{defn} \ld{}
	Let $(X,\MA)$ and $(Y,\MB)$ be topological spaces, $f:X \rightarrow Y$ and $x \in X$. Then $f$ is said to be \tbf{continuous at $x$} if for each $V \in \MN_{f(x)}$, there exists $U \in \MN_x$ such that $f(U) \subset V$. 
	\end{defn}		
	
	\begin{ex} \lex{}
	Let $(X,\MA)$ and $(Y,\MB)$ be topological spaces, $f:X \rightarrow Y$ and $x \in X$. Then $f$ is continuous at $x$ iff for each $V \in \MN_{f(x)}$, $f^{-1}(V) \in \MN_{x}$.\\
	\tbf{Hint:} for $U \in \MN_x$ and $V \in \MN_{f(x)}$, consider $f^{-1}(f(U))$ and $f(f^{-1}(V))$
	\end{ex}
	
	\begin{proof}
	Suppose that $f$ is continuous at $x$. Let $V \in \MN_{f(x)}$. Then there exists $U \in \MN_x$ such that $f(U) \subset V$. Thus
	\begin{align*}
	x 
	&\in \Int U \\
	& \subset U \\
	&\subset f^{-1}(f(U)) \\
	&\subset f^{-1}(V)
	\end{align*}
	So $f^{-1}(V) \in \MN_x$.\\
	Conversely, suppose that for each $V \in \MN_{f(x)}$, $f^{-1}(V) \in \MN_{x}$. Let $V \in \MN_{f(x)}$. Hence $f^{-1}(V) \in \MN_{x}$. Set $U = f^{-1}(V)$. Then 
	\begin{align*}
	f(U) 
	&= f(f^{-1}(V)) \\
	& \subset V
	\end{align*}
	Thus $f$ is continuous at $x$.
	\end{proof}
	
	\begin{ex} \lex{}
	Let $(X,\MA)$ and $(Y,\MB)$ be topological spaces and $f:X \rightarrow Y$. Then $f$ is continuous iff for each $x \in X$, $f$ is continuous at $x$.
	\end{ex}
	
	\begin{proof}
	Suppose that $f$ is continuous. Let $x \in X$. Let $V \in \MN_{f(x)}$. Then $\Int V \in \MB$ and $f(x) \in \Int V$. Set $U = f^{-1}(\Int V)$. By continuity, $U \in \MA$ and by construction, $x \in U$. Hence $U \in \MN_x$. Then 
	\begin{align*}
	f(U)
	&= f(f^{-1}(\Int V))\\
	& \subset \Int V\\
	& \subset V
\end{align*}	 	
So $f$ is continuous at $x$. \\
Conversely, suppose that for each $x \in X$, $f$ is continuous at $x$. Let $B \in \MB$. Let $x \in f^{-1}(B)$. Then $B \in \MN_{f(x)}$. Continuity at $x$ implies that $f^{-1}(B) \in \MN_x$. Then $x \in \Int (f^{-1}(B))$. Since $x \in f^{-1}(B)$ is arbitrary, $f^{-1}(B) \subset \Int (f^{-1}(B))$. Hence $f^{-1}(B) = \Int (f^{-1}(B))$ which implies that $f^{-1}(B) \in \MA$. So $f$ is continuous.
	\end{proof}
	
	\begin{defn} \ld{}
	Let $(X, \MA)$ and $(Y,\MB)$ be topological spaces and $f: X \rightarrow Y$. We define the 
	\begin{enumerate}
	\item \tbf{push-forward of $\MA$}, denoted $f_*\MA$, by 
	$$f_*\MA = \{B \subset Y: f^{-1}(B) \in \MA\}$$ 
	\item  \tbf{pull-back of $\MB$}, denoted $f^*\MB$, by  
	$$f^*\MB = \{f^{-1}(B):  B \in \MB \}$$
	\end{enumerate}
	\end{defn}
	
	\begin{ex} \lex{} 
		Let $(X,\MA)$ and $(Y,\MB)$ be topological spaces and $f: X \rightarrow Y$. Then 
		\begin{enumerate}
			\item $f_*\MA$ is a topology on $Y$
			\item $f^*\MB$ is a topology on $X$
		\end{enumerate}
	\end{ex}
	
	\begin{proof}\
		\begin{enumerate}
			\item 
			\begin{itemize}
			\item Since $f^{-1}(Y) = X \in \MA$ and $f^{-1}(\varnothing) = \varnothing \in \MA$, $Y, \varnothing \in f_*\MA$.
			\item Let $(U_{\al})_{\al \in A} \subset f_*\MA$. Then for each $\al \in A$, $f^{-1}(U_{\al}) \in \MA$. This implies that 
			\begin{align*}
			f^{-1}\bigg( \bigcup\limits_{\al \in A}U_{\al} \bigg) 
			&=  \bigcup\limits_{\al \in A} f^{-1}(U_{\al}) \\
			& \in \MA
			\end{align*}
			Hence $\bigcup\limits_{\al \in A}U_{\al} \in f_*\MA$.
			\item Let $(U_{j})_{j=1}^n \subset f_*\MA$. Then for each $j \in {1, \ldots, n}$, $f^{-1}(U_{j}) \in \MA$. This implies that 
			\begin{align*}
			f^{-1}\bigg( \bigcap\limits_{j=1}^n U_{j} \bigg) 
			&=  \bigcap\limits_{j=1}^n f^{-1}(U_{j}) \\
			& \in \MA
			\end{align*}
			Hence $\bigcap\limits_{j=1}^n U_{j} \in f_*\MA$.
			\end{itemize}
			So $f_*\MA$ is a topology on $Y$.
			\item Similar to (1).
		\end{enumerate}
	\end{proof}	
	
	\begin{ex} \lex{}
	Let $(X,\MA)$ and $(Y,\MB)$ be topological spaces, $f:X \rightarrow Y$ and $\ME \subset \MP(Y)$. Suppose that $\MB = \tau(\ME)$. Then $f$ is continuous iff for each $B \in \ME$, $f^{-1}(B) \in \MA$.
	\end{ex}
	
	\begin{proof}
	Suppose that $f$ is continuous. Since $\ME \subset \MB$, clearly for each $B \in \ME$, $f^{-1}(B) \in \MA$. \\
	Conversely, suppose that for each $B \in \ME$, $f^{-1}(B) \in \MA$. Then $\ME \subset f_*\MA$. Since $f_*\MA$ is a topology on $Y$, we have that $\MB = \tau(\ME) \subset f_*\MA$. So $f$ is continuous.
	\end{proof}
	
	\begin{defn} \ld{}
	Let $X$ be a set, $(Y_{\al}, \MB_{\al})_{\al \in A}$ a collection of topological spaces and $\MF \in \prod \limits_{\al \in A}Y_{\al}^X$ (i.e. $\MF = (f_{\al})_{\al \in A}$ where for each $\al \in A$, $f_{\al}:X \rightarrow Y_{\al}$). We define the \tbf{initial topology generated by $\MF$} on $X$, denoted $\tau_X(\MF)$, by 
	\begin{align*}
	\tau_X(\MF) 
	&= \tau(\{f_{\al}^{-1}(B): B \in \MB_{\al} \text{ and } \al \in A \})
\end{align*}	 
	\end{defn}
	
	\begin{note}
	The initial topology topology generated by $\MF$ is also called the \tbf{weak topology generated by $\MF$} and if $\MF = \{f\}$, then $\tau_X(\MF) = f^*\MB$.
	\end{note}
	
	\begin{note}
	Essentially, $\tau_X(\MF)$ is the smallest topology on $X$ such that for each $\al \in A$, $f_{\al}:X \rightarrow Y_{\al}$ is continuous. 
	\end{note}

	\begin{ex}
		Let $(Y_{\al}, \MB_{\al})_{\al \in A}$ be a a collection of topological spaces, $X$ a set, $(Z, \MC)$ a topological space, $\MF = (f_{\al})_{\al \in A} \in \prod \limits_{\al \in A}Y_{\al}^X$ and $g: Z \rightarrow X$. Then $g$ is $\MC$-$\tau_X(\MF)$ continuous iff for each $\al \in A$, $f_{\al} \circ g$ is $\MC$-$\MB_{\al}$ continuous:
		\[ \begin{tikzcd}
			Y_{\al}	
			& X  \arrow[l, "f_{\al}"'] \\
			& Z \arrow[ul, "g \circ f_{\al}"]  \arrow[u, "g"']
		\end{tikzcd}
		\]
	\end{ex}
	
	\begin{proof}
		If $g$ is $\MC$-$\tau_X(\MF)$ continuous, then clearly for each $\al \in A$, $ f_{\al} \circ g$ is $\MC$-$\MB_{\al}$ continuous. \\
		Conversely, suppose that for each $\al \in A$, $f_{\al} \circ g$ is $\MC$-$\MB_{\al}$ continuous. Let $\al \in A$ and $V \in \MB_{\al}$. Continuity implies that,
		\begin{align*}
			g^{-1}(f_{\al}^{-1}(V)) 
			& = (f_{\al} \circ g)^{-1}(V) \\
			& \in \MC
		\end{align*}
		Since $\al \in A$ and $V \in \MB_{\al}$ are arbitrary, we have that for each $\al \in A$ and $V \in \MB_{\al}$, $g^{-1}(f_{\al}^{-1}(V)) \in \MC$. Since $\tau_X(\MF) = \tau(\{f_{\al}^{-1}(V): \al \in A \text{ and } V \in \MB_{\al})$, the previous exercise implies that $g$ is $\MC$-$\tau_X(\MF)$  continuous.
	\end{proof}
	
	\begin{defn} \ld{}
	Let $(X_{\al}, \MA_{\al})_{\al \in A}$ be a a collection of topological spaces, $Y$ a set and $\MF \in \prod \limits_{\al \in A}Y^{X^{\al}}$ (i.e. $\MF = (f_{\al})_{\al \in A}$ where for each $\al \in A$, $f_{\al}:X_{\al} \rightarrow Y$). We define the \tbf{final topology generated by $\MF$} on $X$, denoted $\tau_Y(\MF)$, by 
	\begin{align*}
	\tau_Y(\MF) 
	&= \tau(\{V \subset Y: \text{ for each $\al \in A$, $f_{\al}^{-1}(V) \in \MA_{\al}$}\})
\end{align*}	 
	\end{defn}
	
	\begin{note}
	If $\MF = \{f\}$, then $\tau_Y(\MF) = f_*\MA$.
	\end{note}
	
	\begin{note}
	Essentially, $\tau_X(\MF)$ is the largest topology on $X$ such that for each $\al \in A$, $f_{\al}:X_{\al} \rightarrow Y$ is continuous. 
	\end{note}
	
	\begin{ex}
	Let $(X_{\al}, \MA_{\al})_{\al \in A}$ be a a collection of topological spaces, $Y$ a set, $(Z, \MC)$ a topological space, $\MF = (f_{\al})_{\al \in A} \in \prod \limits_{\al \in A}Y^{X_{\al}}$ and $g: Y \rightarrow Z$. Then $g$ is $\tau_Y(\MF)$-$\MC$ continuous iff for each $\al \in A$, $g \circ f_{\al}$ is $\MA_{\al}$-$\MC$ continuous:
	\[ \begin{tikzcd}
	X_{\al} \arrow[r, "f_{\al}"] \arrow[dr, "g \circ f_{\al}"'] 	
	& Y  \arrow[d, "g"] \\
	& Z 
\end{tikzcd}
	\]
	\end{ex}
	
	\begin{proof}
	If $g$ is $\tau_Y(\MF)$-$\MC$ continuous, then clearly for each $\al \in A$, $g \circ f_{\al}$ is $\MA_{\al}$-$\MC$ continuous. \\
	Conversely, suppose that for each $\al \in A$, $g \circ f_{\al}$ is $\MA_{\al}$-$\MC$ continuous. Let $\al \in A$ and $V \in \MC$. Continuity implies that 
	\begin{align*}
		f_{\al}^{-1}(g^{-1}(V)) 
		& = (g \circ f_{\al})^{-1}(V) \\
		& \in \MA_{\al}
	\end{align*}
	Since $\al \in A$ is arbitrary, we have that by definition, $g^{-1}(V) \in \tau_Y(\MF)$. Since $V \in \MC$ is arbitrary, $g$ is $\tau_Y(\MF)$-$\MC$ continuous.
	\end{proof}
	
	\begin{defn} \ld{}
		Let $(X,\MA)$ and $(Y,\MB)$ be topological spaces and $f:X \rightarrow Y$. Then 
		\begin{enumerate}
			\item $f$ is said to be \tbf{open} if for each $A \in \MA$, $f(A) \in \MB$.
			\item $f$ is said to be \tbf{closed} if for each $A \subset X$, if $A^c \in \MA$, then $f(A)^c \in \MB$. 
		\end{enumerate}
	\end{defn}

	\begin{ex}
		Let $(X, \MT), (Y, \MS)$ be topological spaces, $\MB \subset \MT$ a basis for $\MT$ and $f: X \rightarrow Y$. Then $f$ is open iff for each $U \in \MB$, $f(U) \in \MS$.\\
		\tbf{Hint:} $f\bigg( \bigcup\limits_{\al \in A} A_{\al} \bigg) =  \bigcup\limits_{\al \in A} f(A_{\al})$.
	\end{ex}

	\begin{proof}
		Clearly if $f$ is open, then for each $U \in \MB$, $f(U) \in \MS$.\\
		Conversely, suppose that for each $U \in \MB$, $f(U) \in \MS$. Let $U \in \MT$. Then there exists $(U_{\al})_{\al \in A} \subset \MB$ such that $U =  \bigcup\limits_{\al \in A} U_{\al}$. Then 
		\begin{align*}
			f(U) 
			& = \bigcup\limits_{\al \in A} f(U_{\al}) \\
			& \in \MS
		\end{align*}
		Since $U \in \MT$ is arbitrary, $f$ is open.
	\end{proof}
	
	\begin{ex}\lex{} \tbf{Doob-Dynkin Lemma:} \\
	Let $(X_1, \MT_1)$, $(X_2, \MT_2)$ and $(X_3, \MT_3)$ be topological spaces and $f: X_1 \rightarrow X_2$ and $g:X_1 \rightarrow X_3$. Suppose that $f$ is surjective and $\MT_1$-$\MT_2$ continuous and $g$ is $\MT_1$-$\MT_3$ continuous and $(X_3, \MT_3)$ is a $T_1$ space. Then $g$ is $f^*\MT_2$-$\MT_3$ continuous iff there exists a unique $\phi: X_2 \rightarrow X_3$ such that $\phi$ is $\MT_2$-$\MT_3$ continuous and $g = \phi \circ f$. \\
	\tbf{Hint:} For each $t \in X_3$, set $A_t = g^{-1}(\{t\}) \in \MF_{(f^* \MT_2)}$ and choose $B_t \in \MT_2$ such that $A_t = f^{-1}(B_t)$. Set $\phi(y) = t$ for $y \in B_t \cap f(X_1)$ and $t \in g(X_1)$.
	\end{ex}
	
	\begin{proof}
	
	Suppose that there exists a unique $\phi: X_2 \rightarrow X_3$ such that $\phi$ is $\MT_2$ - $\MT_3$ measurable and $g = \phi \circ f$. Since $f$ is $f^* \MT_2$ - $\MT_2$ continuous, we have that $g = \phi \circ f$ is $f^*\MT_2$-$\MT_3$ continuous.  \\
	Conversely, suppose that $g$ is $f^*\MT_2$-$\MT_3$ continuous. \\
	\begin{itemize}
	\item \tbf{(Existence)} \\
	For each $t \in X_3$, set $A_t = g^{-1}(\{t\})$ Since $(X_3, \MT_3)$ is a $T_1$ space, for each $t \in X_3$, $A_t \in \MF_{f^*\MT_2}$ and thus, there exists $B_t \in \MF_{\MT_2}$ such that $A_t = f^{-1}(B_t)$. \\
	Note that 
	\begin{itemize}
	\item for each $t \in g(X_1)$, there exists $x \in A_t$ such that $g(x) = t$. Hence $f(x) \in B_t$.\\
	\item for $t_1, t_2 \in g(X_1)$, $t_1 \neq t_2$ implies that
	\begin{align*}
	f^{-1}(B_{t_1} \cap B_{t_2}) 
	&= A_{t_1} \cap A_{t_2} \\
	&= g^{-1}(\{t_1\} \cap \{t_2\}) \\
	&= \varnothing
	\end{align*}	 
	and since $f$ is surjective, 
	\begin{align*}
	B_{t_1} \cap  B_{t_2} 
	& = f(f^{-1}(B_{t_1} \cap  B_{t_2} )) \\
	&= f(\varnothing) \\
	&= \varnothing
	\end{align*}
	\item we have that 
	\begin{align*}
	f^{-1} \bigg( \bigcup_{t \in g(X_1)} B_t\bigg) 
	&=  \bigcup_{t \in g(X_1)} A_t \\
	&= \bigcup_{t \in g(X_1)} g^{-1}(\{t\}) \\
	&= g^{-1}(g(X_1)) \\
	&= X_1
	\end{align*}
	Since $f$ is surjective, we have that 
	\begin{align*}
	X_2
	&= f(X_1) \\
	&= f \bigg( f^{-1} \bigg( \bigcup_{t \in g(X_1)} B_t\bigg)  \bigg) \\
	&= \bigcup_{t \in g(X_1)} B_t
	\end{align*}
	\end{itemize}
	Therefore, 
	\begin{itemize}
	\item for each $t \in g(X_1)$, $B_t \neq \varnothing$
	\item $(A_t)_{t \in g(X_1)}$ is a partion of $X_1$
	\item $(B_t)_{t \in g(X_1)}$ is a partition of $X_2$\\
\end{itemize}		
	 Define $\phi:X_2 \rightarrow X_3$ by $\phi(y) = t$ for $t \in g(X_1)$ and $y \in B_t $. Then the previous observations imply that $\phi$ is well defined and $\phi(X_2) = g(X_1)$. Since for each $t \in g(X_1)$ and $x \in A_t$, $f(x) \in B_t$ and $g(x) = t$, we have that $\phi \circ f (x) = t = g(x)$. So $\phi \circ f = g$. \\ \\
	To show that $\phi$ is continuous, let $C \in \MT_3$. Choose $B \in \MT_2$ such that $g^{-1}(C) = f^{-1}(B)$.
	Let $y \in \phi^{-1}(C) \subset X_2$. Set $t = \phi(y) \in C$ and choose $x \in X_1$ such that $y = f(x)$. Since 
	\begin{align*}
	g(x) 
	&= \phi \circ f (x) \\
	&= \phi(y) \\
	&= t \\
	&\in C
\end{align*}		
	 $x \in g^{-1}(C) = f^{-1}(B)$. Therefore, $y = f(x) \in B$. So $\phi^{-1}(C) \subset B$. \\
	Let $y \in B$. Choose $x \in X_1$ such that $f(x) = y$. Then $x \in f^{-1}(B) = g^{-1}(C)$. So 
	\begin{align*}
	\phi(y) 
	&= \phi \circ f (x) \\
	&= g(x) \\
	&\in C
	\end{align*}	 
	and $y \in \phi^{-1}(C)$. So $B \subset \phi^{-1}(C)$. 
	Hence $\phi^{-1}(C) = B \in \MT_2$ and $\phi$ is $\MT_2$ - $\MT_3$ continuous.\\
	\item \tbf{(Uniqueness)} \\
	Let $\psi: X_2 \rightarrow X_3$. Suppose that $\psi$ is $\MT_2$-$\MT_3$ continuous and $g = \psi \circ f$. Let $y \in X_2$. Then there exists $x \in X_1$ such that $y = f(x)$. Then 
	\begin{align*}
	\psi(y) 
	&= \psi \circ f(x) \\
	&= g(x) \\
	&= \phi \circ f(x) \\
	&= \phi(y)
	\end{align*}
	So $\psi = \phi$.
	\end{itemize}
 
	\end{proof}

	\begin{ex} \lex{}
	Let $(X_1, \MT_1)$, $(X_2, \MT_2)$ and $(X_3, \MT_3)$ be topological spaces and $f: X_1 \rightarrow X_2$ and $g:X_1 \rightarrow X_3$. Suppose that $f$ is $\MT_1$-$\MT_2$ continuous and $g$ is $\MT_1$-$\MT_3$ continuous and $(X_3, \MT_3)$ is a $T_1$ space. Then $g$ is $f^*\MT_2$-$\MT_3$ continuous iff there exists a unique $\phi: f(X_1) \rightarrow X_3$ such that $\phi$ is $\MT_2 \cap f(X_1)$ - $\MT_3$ continuous and $g = \phi \circ f$. \\
	\end{ex}
	
	\begin{proof}
	A previous exercise implies that $f: X_1 \rightarrow f(X_1)$ is $\MT_1$ - $\MT_2 \cap f(X_1)$ continuous. Now apply the previous exercise. 
	\end{proof}
	
	\begin{defn}
		Let $X$ be a topological space, $x_0 \in X$ and $f:X \rightarrow \R$. We define the \tbf{limit inferior of $f$ as $x \rightarrow x_0$ (resp. limit inferior of $f$ as $x \rightarrow x_0$)}, denoted $\liminf\limits_{x \rightarrow x_0}f(x)$ (resp. $\liminf\limits_{x \rightarrow x_0}f(x)$), by 
		$$\liminf_{x \rightarrow x_0} f(x) = \sup_{V \in \MN_{x_0}} \inf_{x \in V \setminus \{x_0\}} f(x)$$
		resp. 
		$$\limsup_{x \rightarrow x_0} f(x) = \inf_{V \in \MN_{x_0}} \sup_{x \in V \setminus \{x_0\}} f(x)$$
	\end{defn}

	\begin{ex}
		Let $X$ be a topological space, $x_0 \in X$ and $f:X \rightarrow \R$. Then $f$ is continuous at $x_0$ iff $\liminf\limits_{x \rightarrow x_0}f(x) = \limsup\limits_{x \rightarrow x_0}f(x) = f(x_0)$ 
	\end{ex}

	\begin{proof}
		Suppose that \\
		\tbf{FINISH!!!}
	\end{proof}














\newpage
\subsection{Nets}	

	\begin{defn} \ld{33001}
	Let $A$ be a set and $\leq$ a relation on $A$. Then $(A, \leq)$ is said to be a \tbf{directed set} if , 
	\begin{enumerate}
	\item for each $\al \in A$, $\al \leq \al$
	\item for each $\al, \be, \gam \in A$, $\al \leq \be$ and $\be \leq \gam$ implies that $\al \leq \gam$
	\item for each $\al, \be \in A$, there exists $\gam \in A$ such that $\al, \be \leq \gam$
	\item $A \neq \varnothing$
	\end{enumerate}
	\end{defn}
	
	\begin{defn} \ld{33002}
	Let $X$ be a set. Define the \tbf{reverse inclusion ordering} on $\MN_x$, denoted $\leq$, by $U \leq V$ iff $V \subset U$. 
	\end{defn}
	
	\begin{ex} \lex{33003}
	Let $X$ be a topological space and $x \in X$. Then $\MN_x$ ordered by reverse inclusion is a directed set.
	\end{ex}
	
	\begin{proof}\
	\begin{enumerate}
	\item Clearly, for each $U \in \MN_x, U \leq U$.
	\item Let $U,V,W \in \MN_x$. Suppose that $U \leq V$ and $V \leq W$. Then $W \subset V \subset U$ which implies that $W \subset U$ and hence $U \leq W$.
	\item Let $U,V \in \MN_x$. Set $W = U \cap V$. Then $W \in \MN_x$ and $U,V \leq W$. 
	\end{enumerate}
	So $\MN_x$ is a directed set. 
	\end{proof}

	\begin{defn}
		Let $X$ be a metric space and $x_0 \in X$. Define the \tbf{reverse distance from $x_0$ ordering} on $X \setminus \{x_0\}$, denoted $\leq_{x_0}$, by $x \leq_{x_0} y$ iff $d(x, x_0) \geq d(y, x_0)$.
	\end{defn}

	\begin{ex}
		 Let $X$ be a metric space and $x_0 \in X$. Then $(X \setminus \{x_0\}, \leq_{x_0})$ is a directed set. 
	\end{ex}

	\begin{proof}\
		\begin{enumerate}
			\item Let $x \in X \setminus \{x_0\}$. Since $d(x, x_0) \geq d(x, x_0)$, $x \leq_{x_0} x$.
			\item Let $x, y, z \in X \setminus \{x_0\}$. Suppose that $x \leq_{x_0} y$ and $y \leq_{x_0} z$. Then $d(x, x_0) \geq d(y, x_0)$ and $d(y, x_0) \geq d(z, x_0)$. Hence $d(x, x_0) \geq d(z, x_0)$ so that $x \leq z$.
			\item Let $x,y \in X \setminus \{x_0\}$. Set 
			\begin{align*}
				z 
				&= \argmin\limits_{a \in \{x, y\}} d(a, x_0) \\ 
				& \in X \setminus \{x_0\} 
			\end{align*}
			Then $x, y \leq_{x_0} z$.
		\end{enumerate}
	\end{proof}

	\begin{defn}
		Let $(A, \leq_A)$ and $(B, \leq_B)$ be directed sets. We define the \tbf{product directed set of $(A, \leq_A)$ and $(B, \leq_B)$}, denoted $(A \times B, \leq)$, by 
		$$(a_1, b_1) \leq (a_2, b_2) \text{ iff } a_1 \leq a_2 \text{ and } b_1 \leq b_2$$
	\end{defn}

	\begin{ex}
		Let $(A, \leq_A)$ and $(B, \leq_B)$ be directed sets. Then the product directed set of $(A, \leq_A)$ and $(B, \leq_B)$ is a directed set.
	\end{ex}

	\begin{proof}\
		\begin{enumerate}
			\item Let $(a, b) \in A \times B$. Then $a \leq_A a$ and $b \leq_B b$. So $(a, b) \leq (a ,b)$.
			\item Let $(a_1, b_1), (a_2, b_2), (a_3, b_3) \in A \times B$. Suppose that $(a_1, b_1) \leq (a_2, b_2)$ and $(a_2, b_2) \leq (a_3, b_3)$. Then $a_1 \leq_A a_2$, $a_2 \leq_A a_3$, $b_1 \leq_B b_2$ and $b_2 \leq_B b_3$. Therefore $a_1 \leq_A a_3$ and $b_1 \leq_B b_3$. Hence $(a_1, b_1) \leq (a_3, b_3)$.
			\item Let $(a_1, b_1), (a_2, b_2) \in A \times B$. Then there exist $a \in A$ and $b \in B$ such that $a_1, a_2 \leq_A a$ and $b_1, b_2 \leq_B b$. Hence $(a_1, b_1), (a_2, b_2) \leq (a, b)$.
		\end{enumerate}
		So $(A \times B, \leq)$ is directed.
	\end{proof}
	
	\begin{defn} \ld{33004}
	Let $X$ be a topological space, $A$ a directed set and $x:A \rightarrow Y$. Then $x$ is said to be a \tbf{net} in $X$. We typically write $(x_{\al})_{\al \in A}$. 
	\end{defn}
	
	\begin{defn} \ld{33005}
	Let $X$ be a topological space, $(x_{\al})_{\al \in A} \subset X$ a net and $U \subset X$.
	Then $(x_{\al})_{\al \in A}$ is said to be 
	\begin{itemize}
	\item \tbf{eventually in $U$} if there exists $\be \in A$ such that for each $\al \in A$ $\al \geq \be$ implies that $x_{\al} \in U$
	\item \tbf{frequently in $U$}  if for each $\al \in A$, there exists $\be \in A$ such that $\be \geq \al $ and $x_{\be} \in U$
	\end{itemize}
	\end{defn}
	
	\begin{defn} \ld{33006}
	Let $X$ be a topological space, $(x_{\al})_{\al \in A} \subset X$ a net and $x \in X$. Then $(x_{\al})_{\al \in A}$ is said to \tbf{converge to $x$}, denoted $x_{\al} \rightarrow x$, if for each $U \in \MN_x$, $(x_{\al})_{\al \in A}$ is eventually in $U$. 
	\end{defn}	
	
	\begin{defn} \ld{33007}
	Let $X$ be a topological space and $(x_{\al})_{\al \in A} \subset X$ a net. Then $(x_{\al})_{\al \in A}$ is said to \tbf{converge} if there exists $x \in X$ such that $x_{\al} \rightarrow x$. 
	\end{defn}	

	\begin{ex}
		Let $X$ be a metric space and $x_0 \in X$. Set $A = X \setminus \{x_0\}$. Order $A$ by reverse distance from $x_0$. Define $(x_{\al})_{\al \in A} \subset X$ by $x_{\al} = \al$. Then $x_{\al} \rightarrow x_0$.
	\end{ex}

	\begin{proof}
		Let $U \in \MN_{x_0}$. Since $x_0 \in \Int U$, there exists $\del > 0$ such that $B(x_0, \del) \subset \Int U$. Choose $\be \in B^*(x_0, \del)$. Let $\al \in A$. Suppose that $\al \geq \be$. Then $d(\al, x_0) \leq d(\be, x_0) < \del$. Hence  
		\begin{align*}
			x_{\al} 
			&= \al  \\
			& \in B^*(x_0, \del) \\
			&\subset U
		\end{align*}
		Since $U \in \MN_{x_0}$ is arbitrary, $x_{\al} \rightarrow x_0 $
	\end{proof}
	
	\begin{ex} \lex{33008}
	Let $X$ be a topological space, $S \subset X$ and $x \in X$. Then $x \in S'$ iff there exists a net $(x_{\al})_{\al \in A} \subset S \setminus \{x\}$ such that $x_{\al} \rightarrow x$. 
	\end{ex}

	\begin{proof}
	Suppose that $x \in S'$. Set $A = \MN_x$, ordered by reverse inclusion.  Since $x \in S'$, for each $\al \in A$, there exists $x_\al \in (\al \setminus \{x\}) \cap S.$ Then $(x_{\al})_{\al \in A} \subset S \setminus \{x\}$. Let $V \in \MN_x$. Choose $\be = V$. Let $\al \in \MN_x$. Suppose that $\al \geq \be$. Then 
	\begin{align*}
	x_{\al} 
	&\in (\al \setminus \{x\}) \cap S \\
	& \subset \al \\
	& \subset \be \\
	&= V
\end{align*}	
	So $(x_{\al})_{\al \in \MN_x}$ is eventually in $V$. Since $V \in \MN_x$ is arbitrary, $x_{\al } \rightarrow x$. \\
	Conversely, suppose that there exists a net $(x_{\al})_{\al \in A} \subset S \setminus \{x\}$ such that $x_{\al} \rightarrow x$. Let $U \in \MN_x$. Since $(x_{\al})_{\al \in A}$ is eventually in $U$, there exits $\be \in A$ such that $x_{\be} \in U$. Then $x_{\be} \in (U \setminus \{x\}) \cap S$ and $(U \setminus \{x\}) \cap S \neq \varnothing$. Since $U \in \MN_x$ is arbitrary, $x \in S'$.
	\end{proof}
	
	\begin{ex} \lex{33009}
	Let $X$ be a topological space, $S \subset X$ and $x \in X$. Then $x \in \cl S$ iff there exists a net $(x_{\al})_{\al \in A} \subset S$ such that $x_{\al} \rightarrow x$. 
	\end{ex}

	\begin{proof}
	Suppose that $x \in \cl S$. Since $\cl S = S \cup S'$, $x \in S$ or $x \in S'$. If $x \in S$, define $(x_n)_{n \in \N} \subset S$ by $x_n = x$. Then $x_n \rightarrow x$. If $x \in S'$, the previous exercise implies that there exists a net $(x_{\al})_{\al \in A} \subset S \setminus \{x\} \subset S$ such that $x_{\al} \rightarrow x$. 
	\end{proof}

	\begin{ex} \tbf{Topology in Terms of Nets: } \\
		Let $X$ be a topological space and $U \subset X$. Then $U$ is open iff for each net $(x_{\al})_{\al \in A} \subset X$ and $x \in U$, $x_{\al} \rightarrow x$ implies that $(x_{\al})_{\al \in A} $ is eventually in $U$.
	\end{ex}

	\begin{proof}
		Suppose that $U$ is open. Let $(x_{\al})_{\al \in A} \subset X$ be a net and $x \in U$. Suppose that $x_{\al} \rightarrow x$. Since $U \in \MN_x$, $(x_{\al})_{\al \in A}$ is eventually in $U$. \\
		Conversely, suppose that for each net $(x_{\al})_{\al \in A} \subset X$ and $x \in U$, $x_{\al} \rightarrow x$ implies that $(x_{\al})_{\al \in A} $ is eventually in $U$. For the sake of contradiction, suppose that $U^c$ is not closed. Then there exist a net $(x_{\al})_{\al \in A} \subset U^c$ and $x \in U$ such such that $x_{\al} \rightarrow x$. By assumption, $(x_{\al})_{\al \in A}$ is eventually in $U$. This is a contradiction, so $U^c$ is closed and hence $U$ is open. 
	\end{proof}
	
	\begin{ex} \lex{33010}
	Let $(X,\MA)$ and $(Y,\MB)$ be topological spaces, $f:X \rightarrow Y$ and $x \in X$. Then $f$ is continuous at $x$ iff for each net $(x_{\al})_{\al \in A} \subset X$, $x_{\al} \rightarrow x$ implies that $f(x_{\al}) \rightarrow f(x)$. 
	\end{ex}
	
	\begin{proof}
	Suppose that $f$ is continuous at $x$. Let $(x_{\al})_{\al \in A} \subset X$ be a net. Suppose that $x_{\al} \rightarrow x$. Let $V \in \MN_{f(x)}$. Continuity implies that $f^{-1}(V) \in \MN_{x}$. Since  $x_{\al} \rightarrow x$, $(x_{\al})_{\al \in A}$ is eventually in $f^{-1}(V)$. So there exists $\be \in A$ such that for each $\al \in A$, $\al \geq \be$ implies that $x_{\al} \in f^{-1}(V)$. Let $\al \in A$. Suppose that $\al \geq \be$. Then $f(x_{\al}) \in V$. So $(f(x_{\al}))_{\al \in A}$ is eventually in $V$. Since $V \in \MN_{f(x)}$ is arbitrary, $f(x_{\al}) \rightarrow f(x)$.\\
	Conversely, suppose that $f$ is not continuous at $x$. Then there exists $V \in \MN_{f(x)}$ such that $f^{-1}(V) \not \in \MN_x$. Then $x \not \in \Int (f^{-1}(V))$. So $x \in (\Int (f^{-1}(V)))^c = \cl f^{-1}(V^c)$. This implies that there exists a net $(x_{\al})_{\al \in A} \subset f^{-1}(V^c)$ such that $x_{\al} \rightarrow x$. Since for each $\al \in A$, $f(x_{\al}) \in V^c$, $f(x_{\al})$ is not eventually in $V$. So $f(x_{\al}) \not \rightarrow f(x)$. 
\end{proof}		
	
	\begin{ex} \lex{33011}
	Let $(Y_{\al}, \MB_{\al})_{\al \in A}$ be a collection of topological spaces, $X$ a set and $\MF \in \prod \limits_{\al \in A}Y_{\al}^X$ with $\MF = (f_{\al})_{\al \in A}$. Equip $X$ with $\tau_X(\MF)$. Let $(x_{\gam})_{\gam \in \Gam} \subset X$ be a net and $x \in X$. Then $x_{\gam} \rightarrow x$ iff for each $\al \in A$, $f_{\al}(x_{\gam}) \rightarrow f_{\al}(x)$.  
	\end{ex}
	
	\begin{proof}
	Suppose that $x_{\gam} \rightarrow x$. Let $\al \in A$. Since $f_{\al}$ is continuous, the previous exercise implies that $f_{\al}(x_{\gam}) \rightarrow f_{\al}(x)$. \\
	Conversely, Suppose that for each $\al \in A$,  $f_{\al}(x_{\gam}) \rightarrow f_{\al}(x)$. Let $U \in \MN_x$. Since $\Int U \in \tau_X(\MF)$, \rex{31010} implies there exist $V_1 \in \MB_{\al_1}, \ldots, V_n \in \MB_{\al_n}$ such that $\bigcap\limits_{j=1}^n f_{\al_j}^{-1}(V_j) \subset \Int U$ and $x \in \bigcap\limits_{j=1}^n f_{\al_j}^{-1}(V_j)$. Let $j \in \{1, \ldots, n\}$. Since $f_{\al_j}^{-1}(V_j) \in \MN_x$, $V_j \in \MN_{f(x)}$. By assumption, $f_{\al_j}(x_{\gam})$ is eventually in $V_j$. Thus there exist there exist $\gam'_j \in \Gam$ such that for each $\gam \geq \gam'_j$, $f_{\al_j}(x_{\gam}) \in V_j$, or equivalently, $x_{\gam} \in f_{\al_j}^{-1}(V_j)$. Since $\Gam$ is directed, there exists $\gam' \in \Gam$ such that for each $j \in \{1, \ldots, n\}$, $\gam' \geq \gam'_j$. Let $\gam \in \Gam$. Suppose that $\gam \geq \gam'$. Then 
	\begin{align*}
	x_{\gam} 
	& \in \bigcap\limits_{j=1}^n f_{\al_j}^{-1}(V_j) \\
	& \subset \Int U \\
	& \subset U
\end{align*}	
	So $(x_\gam)_{\gam \in \Gam}$ is eventually in $U$. Since $U \in \MN_x$ is arbitrary, $x_{\gam} \rightarrow x$.  
	\end{proof}
	
	
	\begin{ex} \lex{33012}
	Let $X$ be a set and $\MT_1$, $\MT_2$ topologies on $X$. Then the following are equivalent:
	\begin{enumerate}
		\item $\MT_1 = \MT_2$
		\item for each net $(x_{\al})_{\al \in A} \subset X$ and $x \in X$, $x_{\al} \rightarrow x$ in $\MT_1$ iff $x_{\al} \rightarrow x$ in $\MT_2$.
	\end{enumerate}
	\end{ex}

	\begin{proof}\
		\begin{itemize}
			\item $(1) \implies (2)$: \\
			Clear. \\
			\item $(2) \implies (1)$: \\
			Let $U \in \MT_1$ and $x \in U^c$. Since $U^c$ is closed in $\MT_1$, there exists a net $(x_{\al})_{\al \in A} \subset U^c$ such that $x_{\al} \rightarrow x$ in $\MT_1$. By assumption, $x_{\al} \rightarrow x$ in $\MT_2$. So $U^c$ is closed in $\MT_2$ and $U \in \MT_2$. Hence $\MT_1 \subset \MT_2$. \\
			Similarly, $\MT_2 \subset \MT_1$.
		\end{itemize}
	\end{proof}
	
	\begin{ex} \lex{33013}
		Let $X, Y$ be topological spaces and $\phi: X \rightarrow Y$ a homeomorphism. Then for each $E \subset X$, 
		\begin{enumerate}
			\item $\cl \phi(E) = \phi(\cl E)$  \item $\Int \phi(E) = \phi\Int (E)$  
		\end{enumerate} 
	\end{ex}
	
	\begin{proof}\
		\begin{enumerate}
			\item Let $E \subset X$. Since $E \subset \cl E$, we have that $\phi(E) \subset \phi(\cl E)$. Since $\cl E$ is closed, $\phi(\cl E)$ is closed and thus $\cl \phi(E) \subset \phi(\cl E)$. Conversely, let $x \in \phi(\cl E)$. Then $\phi^{-1}(x) \in \cl E$. Then there exists a net $( y_{\al} )_{\al \in A} \subset E$ such that $y_{\al} \rightarrow \phi^{-1}(x)$. Then $( \phi(y_{\al}) )_{\al \in A } \subset \phi(E)$ and $\phi(y_{\al}) \rightarrow x$. Thus $x \in \cl \phi(E)$ and $\phi(\cl E) \subset \cl \phi(E)$.
			\item Similar
		\end{enumerate} 
	\end{proof}

	\begin{defn}
		Let $X$ be a topological space, $(x_{\al})_{\al \in A} \subset X$ a net and $x \in X$. Then $x$ is said to be a \tbf{cluster point of} $(x_{\al})_{\al \in A}$ if for each $U \in \MN_x$, $(x_{\al})_{\al \in A}$ is frequently in $U$.
	\end{defn}
	
	\begin{defn} \ld{33014}
	Let $X$ be a topological space, $(x_{\al})_{\al \in A}$, $(y_{\be})_{\be \in B} \subset X$ nets and $\phi:B \rightarrow A$. 
	Then $((y_{\be})_{\be \in B}, \phi)$ is said to be a \tbf{subnet of $(x_{\al})_{\al \in A}$} if 
	\begin{enumerate}
		\item for each $\beta \in B$, $y_{\beta} = x_{\phi(\beta)}$
		\item for each $\al_0 \in A$, there exists $\be_0 \in B$ such that for each $\be \in B$, $\be \geq \be_0$ implies that $\phi(\be) \geq \al_0$
	\end{enumerate}
	\end{defn}
	
	\begin{note}
		We usually supress $\phi$ and write $\al_{\be}$ in place of ${\phi(\beta)}$.
	\end{note}
	
	\begin{ex} \lex{33015}
	Let $X$ be a topological space, $(x_{\al})_{\al \in A} \subset X$ a net and $x \in X$. Then $x$ is a cluster point of $(x_{\al})_{\al \in A}$ iff there exists a subnet $(x_{\al_{\be}})_{\be \in B}$ of $(x_{\al})_{\al \in A}$ such that $x_{\al_{\be}} \rightarrow x$. \\
	\tbf{Hint:} Order $\MN_x$ by reverse inclusion and consider the product directed set $B = A \times \MN_x$. If $x$ is a cluster point of $(x_{\al})_{\al \in A}$, then for each $\be = (\gam, U) \in B$, there exists $\al_{\be} \in A$ such that $\al_{\be} \geq \gam$ and $\al_{\be} \in U$. 
	\end{ex}

	\begin{proof}
		Suppose that $x$ is a cluster point of $(x_{\al})_{\al \in A}$. Set $B = A \times \MN_x$. Since $x$ is a cluster point of $(x_{\al})_{\al \in A}$, for each $(\gam, U) \in B$, there exists $\al_{(\gam, U)} \in A$ such that $\al_{(\gam, U)} \geq \gam$ and $x_{\al_{(\gam, U)}} \in U$. Let $\al_0 \in A$. Choose $\be_0 = (\al_0, X) \in B$. Let $\be = (\gam, U) \in B$. Suppose that $\be \geq \be_0$. Then $\gam \geq \al_0$ and 
		\begin{align*}
			\al_{\be}
			&= \al_{(\gam, U)} \\
			& \geq \gam \\
			& \geq \al_0
		\end{align*}
		So that $(x_{\al_{\be}})_{\be \in B}$ is a subnet of $(x_{\al})_{\al \in A}$. Let $U_0 \in \MN_x$. Choose $\al_0 \in A$ and set $\be_0 = (\al_0, U_0)$. Let $\be = (\gam, U) \in B$. Suppose that $\be \geq \be_0$. Then 
		\begin{align*}
			x_{\al_{\be}} 
			&= x_{\al_{(\gam, U)}} \\
			& \in U \\
			&\subset U_0
		\end{align*} 
		Since $U_0 \in \MN_x$ is arbitrary, $x_{\al_{\be}} \rightarrow x$. \\
		Conversely, suppose that there exists a subnet $(x_{\al_{\be}})_{\be \in B}$ of $(x_{\al})_{\al \in A}$ such that $x_{\al_{\be}} \rightarrow x$. Let $U \in \MN_x$ and $\al \in A$. Since $x_{\al_{\be}} \rightarrow x$, there exists $\be_1 \in B$ such that for each $\be \in B$, $\be \geq \be_1$ implies that $x_{\al_{\be}} \in U$. Since $(x_{\al_{\be}})_{\be \in B}$ is a subnet of $(x_{\al})_{\al \in A}$, there exists $\be_2 \in B$ such that for each $\be \in B$, $\be \geq \be_2$ implies that $\al_{\be} \geq \al$. Since $B$ is directed, there exists $\be \in B$ such that $\be_1, \be_2 \leq \be$. Then $x_{\be} \geq \al$ and $x_{\be} \in U$. Since $\al \in A$ is arbitrary,  $(x_{\al_{\be}})_{\be \in B}$ is frequently in $U$. Since $U \in \MN_x$ is arbitrary, $x$ is a cluster point of $(x_{\al_{\be}})_{\be \in B}$. 
	\end{proof}
	
	\begin{ex}
		Let $X$ be a topological space, $(x_{\al})_{\al \in A} \subset X$ a net and $x \in X$. If $x_{\al} \rightarrow x$, then for each subnet $(x_{\al_{\be}})_{\be \in B}$ of $(x_{\al})_{\al \in A}$, $x_{\al_{\be}} \rightarrow x$.
	\end{ex}

	\begin{proof}
		Suppose that $x_{\al} \rightarrow x$. Let $(x_{\al_{\be}})_{\be \in B}$ be a subnet of $(x_{\al})_{\al \in A}$ and $U \in \MN_x$. Since $x_{\al} \rightarrow x$, there exists $\al_0 \in A$ such that for each $\al \geq \al_0$, $x_{\al} \in U$. Since $(x_{\al_{\be}})_{\be \in B}$ is a subnet of $(x_{\al})_{\al \in A}$, there exists $\be_0 \in B$ such that for each $\be \in B$, $\be \geq \be_0$ implies that $\al_{be} \geq \al_0$. Then for each $\be \in B$, $\be \geq \be_0$ implies that $x_{\al_{\be}} \in U$. Since $U \in \MN_x$ is arbitrary, $x_{\al_{\be}} \rightarrow x$.
	\end{proof}
	
	
	
	
	
	\begin{defn}
		Let $(x_{\al})_{\al \in A} \subset \R$ a net. We define the \tbf{limit inferior (resp. limit superior) of $(x_{\al})_{\al \in A}$}, denoted $\liminf x_{\al}$ (resp. $\limsup x_{\al}$), by 
		$$\liminf x_{\al} = \sup_{\beta \in A } \inf_{\al \geq \beta} x_{\al}$$ 
		resp. 
		$$\limsup x_{\al} = \inf_{\beta \in A } \sup_{\al \geq \beta} x_{\al}$$  
	\end{defn}

	\begin{ex}
		Let $(x_{\al})_{\al \in A} \subset \R$ a net. Then $$\liminf x_{\al} \leq \limsup x_{\al}$$
	\end{ex}

	\begin{proof}
		\tbf{FINISH!!!}c
	\end{proof}

	\begin{ex}
		Let $(x_{\al})_{\al \in A} \subset \R$ a net and $x \in \R$. Then $x_{\al} \rightarrow x$ iff $$\liminf x_{\al} = \limsup x_{\al} = x$$
	\end{ex}

	\begin{proof}
		Suppose that $x_{\al} \rightarrow x$. Let $\ep >0$. Then there exist $\beta \in A$ such that for each $\al \in A$, $\al \geq \beta$ implies that $x_{\al} \in B(x, \ep)$. So $\inf\limits_{\al \geq \beta} x_{\al} \geq x - \ep$ and $\sup\limits_{\al \geq \beta} \leq x + \ep$. Therefore 
		\begin{align*}
			\liminf x_{\al} 
			&= \sup_{\beta \in A} \inf_{\al \geq \be} x_{\al} \\
			& \geq x - \ep \\
		\end{align*}
		and 
		\begin{align*}
			\limsup x_{\al} 
			&= \inf_{\beta \in A} \sup_{\al \geq \be} x_{\al} \\
			& \leq x + \ep \\
		\end{align*}
		Since $\ep >0$ is arbitrary, $$\limsup x_{\al} \leq x \leq \liminf x_{\al}$$
		Since $\liminf x_{\al} \leq \limsup x_{\al}$, we have that $\liminf x_{\al} = \limsup x_{\al} = x$.
	\end{proof}
	

	\begin{ex}
		Let $X$ be a topological space, $f:X \rightarrow \R$, $(x_{\al})_{\al \in A} \subset X$ a net and $x \in X$. Suppose that $x_{\al} \rightarrow x$ and for each $\al \in A$, $x_{\al} \neq x$. Then 
		\begin{enumerate}
			\item $\liminf\limits_{t \rightarrow x} f(t) \leq \liminf f(x_{\al})$
			\item $\limsup\limits_{t \rightarrow x} f(t) \geq \limsup f(x_{\al})$
		\end{enumerate}
	\end{ex}

	\begin{proof}\
		\begin{enumerate}
			\item Let $V \in \MN_{x}$. Then there exists $\beta \in A$ such that for each $\al \geq \beta$, $x_{\al} \in V \setminus \{x\}$. Thus 
			$$\inf_{t \in V \setminus \{x\}} \leq \inf_{\al \geq \beta}f(x_{\al})$$
			which implies that $$\inf_{t \in V \setminus \{x\}} f(t)  \leq \sup_{\beta \in A} \inf_{\al \geq \beta} f(x_{\al})$$
			and since $V \in \MN_x$ is arbitrary, we have that
			\begin{align*}
				\liminf_{t \rightarrow x} f(t) 
				&= \sup_{V \in \MN_{x}} \inf_{t \in V \setminus \{x\}} f(t) \\
				& \leq \sup_{\beta \in A} \inf_{\al \geq \beta} f(x_{\al}) \\
				&= \liminf f(x_{\al})
			\end{align*} 
			\item Similar to $(1)$.
		\end{enumerate}
	\end{proof}































\newpage
\subsection{Subspace Topology}

\begin{defn}
	Let $X$ be a set and $A \subset X$. We define the \tbf{inclusion map from $A$ to $B$}, denoted $\iota: A \rightarrow X$, by $\iota(x) = x$. 
\end{defn}

\begin{defn}
	Let $(X, \MT)$ be a topological space and $A \subset X$. We define the \tbf{subspace topology on $A$}, denoted $\MT \cap A$, by $$\MT \cap A = \iota^*(\MT)$$
\end{defn}

\begin{ex}
	Let $(X, \MT)$ be a topological space and $A \subset X$. Then $$\MT \cap A = \{U \cap A: U \in \MT\}$$
\end{ex}

\begin{proof}
	Clear.
\end{proof}

\begin{ex}
	universal property
\end{ex}

\begin{proof}
	\tbf{FINISH!!!}
\end{proof}

\begin{ex}
	Let $(X, \MT)$ be a topological space, $A \subset X$, $(x_{\gam})_{\gam \in \Gam} \subset A$ a net and $x \in A$. Then $x_{\gam} \rightarrow x$ in $(A,\MT \cap A)$ iff $x_{\gam} \rightarrow x$ in $(X,\MT)$.
\end{ex}

\begin{proof}
	Suppose that $x_{\gam} \rightarrow x$ in $(A,\MT \cap A)$. Since $\iota: A \rightarrow X$ is continuous, 
	\begin{align*}
		x_{\gam} 
		&= \iota(x_{\gam}) \rightarrow \iota(x) \\
		&= x
	\end{align*}
	So that $x_{\gam} \rightarrow x$ in $(X,\MT)$. \\
	Conversely, suppose that $x_{\gam} \rightarrow x$ in $(X,\MT)$. Let $V \in \MN_{x}$ in $(A, \MT \cap A)$. Then $x \in \Int V$ in  $(A, \MT \cap A)$. Hence there exists $U \in \MT$ such that $\Int V = U \cap A$. Thus $U \in \MN_x$ in $(X, \MT)$. This implies that $(x_{\gam})_{\gam \in \Gam}$ is eventually in $U$. Then $(x_{\gam})_{\gam \in \Gam}$ is eventually in $U \cap A = \Int V \subset V$. So $x_{\gam} \rightarrow x$ in $(A, \MT \cap A)$.  
\end{proof}
















	
	
	
	
	
	
	
	
	
	
	
	
	
	\newpage
	\subsection{Product Topology}
	
	\begin{defn}
	Let $(X_{\al})_{\al \in A}$ be a collection of topological spaces. We define the \tbf{product topology} on $\prod_{\al \in A}X_{\al}$ to be the initial (weak) topology on $\prod\limits_{\al \in A} X_{\al}$ generated by the projection maps $(\pi_{\al})_{\al \in A}$.
	\end{defn}

	\begin{ex}
		Let $(X_{\al}, \MT_{\al})_{\al \in A}$ be a collection of topological spaces. Define $$\MB = \bigg \{\prod_{\al \in A}B_{\al}: \text{ for each $\al \in A$, } B_{\al} \in \MT_{\al} \text{ and for finitely many $\al \in A$, } B_{\al} \neq X_{\al} \bigg\}$$
		Then $\MB$ is a basis for the product topology on $\prod_{\al \in A}X_{\al}$.
	\end{ex}

	\begin{proof}
		Set $X = \prod_{\al \in A}X_{\al}$. Denote the product topology on $X$ by $\MT_X$. Set 
		$$\ME = \{\pi_{\al}^{-1}(B_{\al}): \al \in A, B_{\al} \in \MT_{\al}\}$$ 
		By definition, $\MT_{X} = \tau_{X}(\ME)$. Let $\al \in A$ and $B_{\al} \in \MT_{\al}$. For $\be \in A$, set 
		\[
		C_{\be} = 
		\begin{cases}
			B_{\be} & \be = \al \\
			X_{\be} & \be \neq \al
		\end{cases}
		\]
		Then 
		\begin{align*}
			\pi_{\al}^{-1}(B_{\al}) = \prod_{\be \in A} C_{\be}  
		\end{align*}
		Hence $\MB = \bigg \{\bigcap_{j=1}^n V_j:(V_j)_{j=1}^n \subset \ME \bigg \} \subset \MT_X$. A previous exercise implies that $\MB$ is a basis for $\MT_X$.
	\end{proof}

	\begin{ex}
		Let $(X_j, \MT_j)_{j =1}^n$ be a collection of topological spaces. Set 
		$$\MB = \bigg \{\prod\limits_{j=1}^n A_j: \text{for each $j \in \{1, \ldots, n\}$, } A_j \in \MT_{j} \bigg \}$$ 
		Then $\MB$ is a basis for the product topology on $\prod_{j=1}^n X_j$.
	\end{ex}

	\begin{proof}
		Clear by previous exercise.
	\end{proof}

	\begin{ex}
		Let $(X_{\al}, \MT_{\al})_{\al \in A}$ be a collection of topological spaces and for each $\al \in A$, $\MB_{\al}$ a basis for $\MT_{\al}$. Set $X = \prod_{\al \in A} X_{\al}$ and denote the product topology on $X$ by $\MT_X$. Set 
		\begin{align*}
			\MB = 
			& \bigg \{\prod_{\al \in A} U_{\al}: \text{there exists $J \subset A$ such that $ \# J < \infty$, } \\
			& \text{ for each $\al \in J$, $U_{\al} \in \MB_{\al}$ and for each $\al \in J^c$, $U_{\al} = X_{\al}$ } \bigg \}
		\end{align*} 
		Then $\MB$ is a basis for $\MT_X$.
	\end{ex}

	\begin{proof}
		Set 
		$$\MB' = \bigg \{\prod_{\al \in A}V_{\al}: \text{ for each $\al \in A$, } V_{\al} \in \MT_{\al} \text{ and for finitely many $\al \in A$, } V_{\al} \neq X_{\al} \bigg\}$$
		The previous exercise implies that $\MB'$ is a basis for $\MT_X$. Then $\MB \subset \MB' \subset \MT_X$. Let $V \in \MT$ and $x \in V$. Write $x = (x_{\al})_{\al \in A}$. Since $\MB'$ is a basis for $\MT_X$, for each $\al \in A$, there exists $V_{\al} \in \MT_{\al}$ such that for finitely many $\al \in A$, $V_{\al} \neq X_{\al}$ and $x \in \prod\limits_{\al \in A} V_{\al} \subset V$. Define $J \subset A$ by $J = \{\al \in A: V_{\al} \neq X_{\al}\}$. Then $\#J < \infty$. Let $\al \in J$. Then $x_{\al} \in V_{\al}$. Since $\MB_{\al}$ is a basis for $\MT_{\al}$, there exists $U'_{\al} \in \MB_{\al}$ such that $x_{\al} \in U'_{\al} \subset V_{\al}$. For $\al \in A$, define $U_{\al} \in \MT_{\al}$ by 
		\[
		U_{\al} =
		\begin{cases}
			U'_{\al} & \al \in J \\
			X_{\al} & \al \in J^c
		\end{cases}
		\]
		Set $U = \prod_{\al \in A} U_{\al}$. Then $U \in \MB$ and 
		\begin{align*}
			x 
			& \in U \\
			& = \prod_{\al \in A} U_{\al} \\
			& \subset \prod\limits_{\al \in A} V_{\al} \\
			& \subset V
		\end{align*}
		Hence $\MB$ is a basis for $\MT_X$.
	\end{proof}

	\begin{ex}
		Let $X$ be a topological space, $(Y_{\al}, \MT_{\al})_{\al \in A}$ a collection of topological spaces and $f: X \rightarrow \prod_{\al \in A}Y_{\al}$. Then $f$ is continuous iff for each $\al \in A$, $\pi_{\al} \circ f$ is continuous.
	\end{ex}

	\begin{proof}
		Immediate by a previous exercise about the initial topology.
	\end{proof}
	
	\begin{ex}
		Let $(X_{\al}, \MT_{\al})_{\al \in A}$ and $(Y_{\al}, \MS_{\al})_{\al \in A}$ be collections of topological spaces and $(f_{\al})_{\al \in A} \in \prod\limits_{\al \in A} Y_{\al}^{X_{\al}}$, i.e. for each $\al \in A$, $f_{\al}:X_{\al} \rightarrow Y_{\al}$. Set $X = \prod\limits_{\al \in A} X_{\al}$ and $Y = \prod\limits_{\al \in A}Y_{\al}$. Define $f: X \rightarrow Y$ by $(f(x))_{\al} = f_{\al}(x_{\al})$. Then 
		\begin{enumerate}
			\item if for each $\al \in A$, $f_{\al}$ is continuous, then $f$ is continous
			\item if $A$ is finite and for each $\al \in A$, $f_{\al}$ is open, then $f$ is open iff for each $\al \in A$, $f_{\al}$ is continuous
		\end{enumerate}
	\end{ex}

	\begin{proof} Denote the $\al$-th projection maps on $X$ and $Y$ by $\pi^X_{\al}$ and $\pi^Y_{\al}$ respectively. Denote the product topologies on $X$ and $Y$ by $\MT$ and $\MS$ respectively. Let $\al \in A$ and $x \in X$. Then
		\begin{align*}
			\pi^Y_{\al} \circ f(x) 
			& = (f(x))_{\al} \\
			& = f_{\al}(x_{\al}) \\
			& = f_{\al} \circ \pi^X_{\al}(x) 
		\end{align*}
		Since $\al \in A$ and $x \in X$ are arbitrary, for each $\al \in A$, $\pi^Y_{\al} \circ f = f_{\al} \circ \pi^X_{\al}$.
		\begin{enumerate}
			\item Suppose that for each $\al \in A$, $f_{\al}$ is continuous. Let $\al \in A$. Then $f_{\al} \circ \pi^X_{\al}$ is continuous. Hence $\pi^Y_{\al} \circ f$ is continuous. Since $\al \in A$ is arbitrary, the previous exercise implies that $f$ is continuous.
			\item Suppose that $A$ is finite and for each $\al \in A$, $f_{\al}$ is open. Set 
			$$\MB_X = \bigg\{ \prod\limits_{\al \in A} U_{\al} : \text{for each $\al \in A$, } U_{\al} \in \MT_{\al}\bigg\}$$ 
			$$\MB_Y = \bigg\{ \prod\limits_{\al \in A} V_{\al} : \text{for each $\al \in A$, } U_{\al} \in \MS_{\al}\bigg\}$$ 
			A previous exercise implies that $\MB_X$ is a basis for $\MT$ and $\MB_Y$ is a basis for $\MS$. For each $\al \in A$, let $U_{\al} \in \MT_{\al}$. Then for each $\al \in A$, $f_{\al}(U_{\al}) \in \MS_{\al}$. Hence
			\begin{align*}
				f\bigg( \prod\limits_{\al \in A} U_{\al}\bigg) 
				& = \prod_{\al \in A} f_{\al}(U_{\al}) \\
				& \in \MB_{Y} \\
				& \subset \MS
			\end{align*}
			Thus for each $U \in \MB_Y$, $f(U) \in \MS$. An exercise about open maps in the section on continuous maps implies that $f$ is open.
		\end{enumerate}
	\end{proof}

	\begin{ex}
		Let $X_1, X_2,Y_1,Y_2$ be topological spaces and $f_1:X_1 \rightarrow Y_1$, $f_2:X_2 \rightarrow Y_2$. If $f_1$ and $f_2$ are open, then $f_1 \times f_1$ is open.
	\end{ex}
	
	\begin{proof}
		Let $A_1 \subset X_1, A_2 \subset X_2$ be open. Then $f_1 \times f_2(A_1 \times A_2) = f_1(A_1) \times f_2(A_2)$ which is open in $Y_1 \times Y_2$. Since $\MB = \{A_1 \times A_2:  \text{$A_1 \subset X_1$ and $A_2 \subset X_2$ are open} \}$ is a basis for the product topology on $X_1 \times X_2$, an exercise in the section on continuous maps implies that $f_1 \times f_2$ is open.
	\end{proof}
	
	\begin{ex}
		Let $X$ and $Y$ be topological spaces and $U \subset X \times Y$ open. Then for each $(x_0,  y_0) \in U$, $U^{x_0}$ and $U^{y_0}$ are open.
	\end{ex}
	
	\begin{proof}
		Let $(x_0, y_0) \in U$. Define $\phi: X \rightarrow X \times Y$ by $\phi(x) = (x, y_0)$. Since $\pi_X \circ \phi = \id_X$ and $\pi_Y \circ \phi$ is constant, $\pi_X \circ \phi$ and $\pi_Y \circ \phi$ are continous. Therefore, $\phi$ is continuous. Then $U^{y_0}$ is open since $U$ is open and $\phi^{-1}(U) = U^{y_0}$. Similarly, $U_{x_0}$ is open.
	\end{proof}

	\begin{ex}
		Let $X$, $Y$ and $Z$ be topological spaces, $U \subset X \times Y$ open and $f: U \rightarrow Z$. Equip $U$ with the subspace topology. Suppose that $f$ is continuous. Let $(x_0, y_0) \in U$. Equip $U_{x_0}$ and $U^{y_0}$ with the subspace topology. Then $f_{x_0}:U_{x_0} \rightarrow Z$ and $f^{y_0}: U^{y_0} \rightarrow Z$ are continuous.
	\end{ex}

	\begin{proof}
		Let $(x_0, y_0) \in U$. Let $V \subset Z$. Suppose that $V$ is open. Continuity of $f$ implies that $f^{-1}(V)$ is open in $U$. Since $U$ is open in $X \times Y$, $f^{-1}(V)$ is open in $X \times Y$. A previous exercise in the section on product sets implies that $(f^{y_0})^{-1}(V) = (f^{-1}(V))^{y_0}$. The previous exercise implies that $(f^{-1}(V))^{y_0}$ is open in $X$. So $(f^{y_0})^{-1}(V)$ is open in $X$. Since $(f^{y_0})^{-1}(V) \subset U^{y_0}$, $(f^{y_0})^{-1}(V)$ is open in $U^{y^0}$. Thus $f^{y_0}: U^{y_0} \rightarrow Z$ is continuous. Similarly, $f_{x_0}: U_{x_0} \rightarrow Z$ is continuous.
	\end{proof}
	

	
	
	
	
	
	
	
	
	
	
	
	
	
	
	
	
	
	
	
	
	
	
	
	
	
	
	
	
	
	
	
	
	\newpage
	\subsection{Quotient Topology}
	
	\begin{defn} \ld{34001}
	Let $(X, \MA)$, $(Y, \MB)$ be topological spaces and $f:X \rightarrow Y$. Then $f$ is said to be a \tbf{$\MA$-$\MB$ quotient map} if 
	\begin{enumerate}
	\item $f$ is surjective
	\item $\MB$ is the final topology on $Y$ generated by $f$, i.e. for each $V \subset Y$, $V \in \MB$ iff $f^{-1}(V) \in \MA$.
	\end{enumerate}
	\end{defn}
	
	\begin{note}
	We typically avoid specifying the topologies when they are clear from the context.
	\end{note}
	
	\begin{ex} \lex{34002}
	Let $(X, \MA)$, $(Y, \MB)$ be topological spaces and $f:X \rightarrow Y$. If $f$ is a quotient map, then $f$ is continuous.
	\end{ex}
	
	\begin{proof}
	Suppose that $f$ is a quotient map. Let $V \subset Y$. Suppose that $V$ is open. By definition, $f^{-1}(V)$ is open. Hence $f$ is continuous.  
	\end{proof}
	
	\begin{ex} \lex{34003}
	Let $(X, \MA)$, $(Y, \MB)$ be topological spaces and $f:X \rightarrow Y$. Suppose that $f$ is continuous and surjective. Then $f$ is a quotient map iff 
	\begin{equation*}
	\text{for each $C \subset Y$, $C$ is closed iff $f^{-1}(C)$ is closed} 
	\end{equation*}	
	\end{ex}
	
	\begin{proof}\
	\begin{itemize}
	\item ($\implies$) \\
	Suppose that $f$ is a quotient map.\\
	Let $C \subset Y$. If $C$ is closed, then continuity implies that $f^{-1}(C)$ is closed.\\ 
	Conversely, suppose that $f^{-1}(C)$ is closed. Then $f^{-1}(C^c) = (f^{-1}(C))^c$ is open. Since $f$ is a quotient map, $f(f^{-1}(C^c))$ is open. Surjectivity implies that $f(f^{-1}(C^c)) = C^c$. So $C$ is closed. 
	\item ($\impliedby$) \\
	Suppose that for each $C \subset Y$, $C$ is closed iff $f^{-1}(C)$ is closed. \\
	Let $V \subset Y$. If $V$ is open. Continuity implies that $f^{-1}(V)$ is open.\\ 
	Conversely, suppose that $f^{-1}(V)$ is open. Then $ f^{-1}(V^c) = (f^{-1}(V))^c$ is closed. Therefore, $f(f^{-1}(V^c))$ is closed. Surjectivity implies that $V^c = f(f^{-1}(V^c))$. So $U$ is open.
	\end{itemize}
	\end{proof}
	
	\begin{ex} \lex{34004}
	Let $(X, \MA)$, $(Y, \MB)$ be topological spaces and $f:X \rightarrow Y$. Suppose that $f$ is continuous and surjective. If $f$ is open or closed, then $f$ is a quotient map. 
	\end{ex}
	
	\begin{proof}\
	\begin{itemize}	
	\item Suppose that $f$ is open. Let $V \subset Y$. \\
	Suppose that $V$ is open. Then continuity implies that $f^{-1}(V)$ is open. Conversely, suppose that $f^{-1}(V)$ is open. Since $f$ is open $f(f^{-1}(V))$ is open. Surjectivity implies that $V = f(f^{-1}(V))$. So $V$ is open. By definition, $f$ is a quotient map.\\
	\item   
	Suppose that $f$ is open. Then similarly to above, $f$ is a quotient map.
	\end{itemize}
	\end{proof}
	
	\begin{ex} \lex{34005}
	Let $(X, \MA)$, $(Y, \MB)$ be topological spaces and $f:X \rightarrow Y$. Suppose that $f$ is a quotient map. Then $f$ is open iff 
	\begin{equation*}
	\text{for each $U \subset X$, $U$ is open implies that $f^{-1}(f(U))$ is open} 
	\end{equation*}
	\end{ex}
	
	\begin{proof}\
	\begin{itemize}	
	\item ($\implies$) \\
	Suppose that $f$ is open.\\
	Let $U \subset X$. Suppose that $U$ is open. Since $f$ is open, $f(U)$ is open. Continuity implies that $f^{-1}(f(U))$ is open.\\ 
	\item ($\impliedby$) \\
	Suppose that for each $U \subset X$, $U$ is open implies that $f^{-1}(f(U))$ is open. \\
	Since $f$ is a quotient map, $f(U)$ is open. So $f$ is open.
	
	\end{itemize}
	\end{proof}
	
	\begin{ex} \lex{34008}
	Let $(X, \MA)$, $(Y, \MB)$ be topological spaces, and $f:X \rightarrow Y$. Suppose that $f$ is surjective and continuous. If $f$ is open or closed, then $f$ is a quotient map.
	\end{ex}
	
	\begin{proof}
	By continuity, $\MB \subset f_*\MA$. 
	\begin{itemize}
	\item Suppose that $f$ is open. Let $V \in f_*\MA$. By definiiton, $f^{-1}(V) \in \MA$. Since $f$ is open, $f(f^{-1}(V)) \in \MB$. Surjectivity implies that $V = f(f^{-1}(V))$. So $f_*\MA = \MB$ and $f$ is a $\MA$-$\MB$ quotient map.
	\item The case is similar if $f$ is closed.
	\end{itemize}
	\end{proof}
	
	
	\begin{defn} \ld{34006}
	Let $(X, \MT)$ be a topological space, $Y$ a set and $f:X \rightarrow Y$. Suppose that $f$ is surjective.	
	 We call $f_* \MT$ a \tbf{quotient topology} on $Y$.  
	\end{defn}
	
	\begin{ex} \lex{34007}
	Let $(X, \MT)$ be a topological space, $Y$ a set and $f:X \rightarrow Y$. Suppose that $f$ is surjective. Then $f: X \rightarrow Y$ is a $\MT$-$f_*\MT$ quotient map. 
	\end{ex}
	
	\begin{proof}
	Clear.
	\end{proof}
	
	
	\begin{ex} \lex{34008}
	Let $(X, \MT)$ be a topological space, $\sim$ an eqivalence relation on $X$ and $\pi:X \rightarrow X/\sim$ the projection map given by $x \mapsto \bar{x}$. Then $\pi$ is a $\MT$-$\pi_*\MT$ quotient map. 
	\end{ex}
	
	\begin{proof}
	Since $\pi$ is surjective, the previous exercise implies that $\pi$ is a $\MT$-$\pi_*\MT$ quotient map. 
	\end{proof}

	\begin{defn}
		Let $(X, \MT)$ be a topological space, $\sim$ an eqivalence relation on $X$ and $\pi:X \rightarrow X/\sim$ the projection map given by $x \mapsto \bar{x}$. We define the \tbf{quotient topology on $X / \sim$} on $X/ \sim$, denoted $\MT_{X/ \sim}$, by $$\MT_{X/ \sim} = \pi_{*}\MT$$
	\end{defn}

	\begin{defn}
		Let $(X, \MT_X)$, $(Y, \MT_Y)$ be topological spaces, $\sim_X$ an equivalence relation on $X$, $\sim_Y$ and equivalence relation on $Y$ and $f : X \rightarrow Y $. Then $f$ is said to be \tbf{$(\sim_X, \sim_Y)$-invariant} if for each $x, y \in X$, $x \sim_X y$ implies that $f(x) \sim_Y f(y)$.
	\end{defn}
	
	\begin{ex} \lex{34008}
	Let $(X, \MT_X)$, $(Y, \MT_Y)$ be topological spaces, $\sim_X, \sim_Y$ eqivalence relations on $X$ and $Y$ respectively, $\pi_X:X \rightarrow X/\sim_X$, $\pi_Y:Y \rightarrow Y/\sim_Y$ the projection map and $f:X \rightarrow Y$ continuous. If $f$ is $(\sim_X, \sim_Y)$-invariant, then there exists a unique $\bar{f}:X / {\sim}_X \rightarrow Y/ {\sim_Y}$ such that $ \bar{f} \circ \pi_X = \pi_Y \circ f$, i.e. the following diagram commutes:
	\[ 
	\begin{tikzcd}
		X  \arrow[r, "f"]  \arrow[d, "\pi_X"']  & Y   \arrow[d, "\pi_Y"]\\
		X/ {\sim_X} \arrow[r, "\bar{f}"'] &  Y / {\sim_Y} \\
	\end{tikzcd}
	\]
	and $\bar{f}$ is $((\pi_X)_*\MT_X, (\pi_Y)_*\MT_Y)$-continuous.
	\end{ex}
	
	\begin{proof}\
	Suppose that $f$ is is $(\sim_X, \sim_Y)$-invariant. 
	\begin{itemize}
		\item \tbf{Existence:} \\
			Define $\bar{f}: X / {\sim_Y} \rightarrow Y/{\sim_Y}$ by $\bar{f}(\bar{x}) = \ol{f(x)}$. Let $a,b \in X$. Then 
		\begin{align*}
			\bar{a} = \bar{b}
			& \implies a \sim_X b \\
			& \implies f(a) \sim_Y f(b) \\
			& \implies \overline{f(a)} = \overline{f(b)} \\
			& \implies \bar{f}(\bar{a}) = \bar{f}(\bar{b}) 
		\end{align*}
		So $\bar{f}$ is well defined. By construction $\bar{f} \circ \pi_X = \pi_Y \circ f$.
		\item \tbf{Uniqueness:} \\
		Let $g: X / {\sim}_X \rightarrow Y/ {\sim_Y}$. Suppose that $g \circ \pi_X = \pi_Y \circ f$, i.e. the following diagram commutes:
		\[ 
		\begin{tikzcd}
			X  \arrow[r, "f"]  \arrow[d, "\pi_X"']  & Y   \arrow[d, "\pi_Y"]\\
			X/ {\sim_X} \arrow[r, "\bar{f}"'] &  Y / {\sim_Y} \\
		\end{tikzcd}
		\]
		\item  \tbf{Continuity:} \\
		Let $V \in (\pi_Y)_* \MT_Y$. Continuity of $f$ and $\pi_Y$ implies that 
		\begin{align*}
			\pi_X^{-1}(\bar{f}^{-1}(V))
			& = (\bar{f} \circ \pi_X)^{-1}(V) \\
			& = (\pi_Y \circ f)^{-1}(V) \\
			& = f^{-1}(\pi_Y^{-1}(V)) \\
			& \in \MT_X
		\end{align*}
		By definition of the quotient topology, $\bar{f}^{-1}(V) (\pi_X)_*\MT_X$. So $\bar{f}$ is $((\pi_X)_*\MT_X, (\pi_Y)_*\MT_Y)$-continuous.
	\end{itemize}

	\end{proof}
	
	
	
	\begin{ex}
		Let $G$ be a group, $X$ a topological space and $\phi: G \times X \rightarrow X$ a group action. Suppose that for each $g \in G$, the map $\phi_g \in \Sym(X)$ defined by $\phi_g(x) = g \cdot x$ is continuous. Then $\pi: X \rightarrow X / G$ is open. 
	\end{ex}

	\begin{proof}
		Suppose that for each $g \in G$, $\phi_g$ is continuous. Let $g \in G$. Since $(\phi_g)^{-1} = \phi_{g^{-1}}$, $\phi_g$ is a homeomorphism. Hence for each $g \in G$ and $U \subset X$, $g \cdot U$ is open. Let $U \subset X$. Suppose that $U$ is open. Then $\pi^{-1}(\pi(U)) = \bigcup_{g \in G} g \cdot U$ is open. A previous exercise implies that $\pi$ is open.
	\end{proof}

	
	
	
	
	
	
	
	
	
	
	
	
	
	
	
	
	
	
	
	
	
	
	
	
	
	
	
	\newpage
	\subsection{Separation and Countability}
	
	\begin{defn}
		Let $X$ be a topological space. Then $X$ is said to be $\mathbf{T_1}$ if for each $x,y \in X$, if $x \neq y$, then there exists $U \in \MN_x$ such that $U$ is open and $y \not \in U$.
	\end{defn}
	
	\begin{ex}
		Let $X$ be a topological space. Then $X$ is $T_1$ iff for each $x \in X$, $\{x\}$ is closed. 
	\end{ex}
	
	\begin{proof}
		Suppose that $X$ is $T_1$. Let $a \in \{x\}^x$. Then there exists $U_{a} \in |MN_a$ such that $U_a$ is open and $U_{a} \subset \{x\}^c$. Therefore 
		$$\{x\}^c = \bigcap_{a \in \{x\}^c} U_a$$ 
		which is open. Hence $\{x\}$ is closed. \\
		Conversely, suppose that for each $x \in X$, $\{x\}$ is closed. Let $x, y \in X$. Suppose that $x \neq y$. Since $\{y\}$ is closed, $\{y\}^c$ is open and $x \in \{y\}^c$. 
	\end{proof}

	\begin{ex}
		Let $(X_{\al}, \MT_{\al})_{\al \in A}$ be a collection of topological spaces. Set $X = \prod\limits_{\al \in A} X_{\al}$ and denote the product topology on $X$ by $\MT_X$. If for each $\al \in A$, $(X_{\al}, \MT_{\al})$ is $T_1$, then $(X, \MT_X)$ is $T_1$.
	\end{ex}

	\begin{proof}
		Suppose that for each $\al \in A$, $(X_{\al}, \MT_{\al})$ is $T_1$. Let $(x_{\al})_{\al \in A}, (y_{\al})_{\al \in A} \in X$. Suppose that $(x_{\al})_{\al \in A} \neq (y_{\al})_{\al \in A}$. Then there exists $\al_0 \in A$ such that $x_{\al_0} \neq y_{\al_0}$. Then there exists $U_{\al_0} \in \MT_{\al_0}$ such that $x_{\al_0} \in U_{\al_0}$ and $y_{\al_0} \not \in U_{\al_0}$. Set $U = \pi_{\al_0}^{-1}(U_{\al_0})$. Then $U \in \MT_X$, $(x_{\al})_{\al \in A} \in U$ and $(y_{\al})_{\al \in A} \not \in U$. Since $(x_{\al})_{\al \in A}, (y_{\al})_{\al \in A} \in X$ are arbitrary, $(X, \MT_X)$ is $T_1$.
	\end{proof}
	
	\begin{defn}
		Let $X$ be a topological space. Then $X$ is said to be $\mathbf{T_2}$ or \tbf{Hausdorff} if for each $x,y \in X$, if $x \neq y$, then there exist $U \in \MN_x$ and $V \in \MN_y$ such that $U$ and $V$ are open and $U \cap V = \varnothing$.
	\end{defn}
	
	\begin{ex}
		Let $X$ be a topological space. If $X$ is Hausdorff, then $X$ is $T_1$.
	\end{ex}
	
	\begin{proof}
		Clear.
	\end{proof}
	
	\begin{ex}
		Let $X$ be a topological space. Then the following are equivalent: 
		\begin{enumerate}
			\item $X$ is Hausdorff
			\item for each net $(x_{\al})_{\al \in A} \subset X$ and $x,y \in X$, if $x_{\al} \rightarrow x$ and $x_{\al} \rightarrow y$, then $x = y$.
			\item The diagonal $\Del_X = \{(x,x):x \in X\}$ is closed in $X \times X$.
		\end{enumerate} 
	\end{ex}
	
	\begin{proof}\
		\begin{itemize}
			\item $(1) \implies (2)$: \\
			Suppose that $X$ is Hausdorff. Let $(x_{\al})_{\al \in A} \subset X$ be a net and $x,y \in X$. Suppose that $x_{\al} \rightarrow x$ and $x_{\al} \rightarrow y$. For the sake of contradiction, suppose that $x \neq y$. Then there exist $U \in \MN_x$ and $V \in \MN_y$ such that $U$ and $V$ are open and $U \cap V = \varnothing$. Since $x_{\al} \rightarrow x$, $(x_{\al})_{\al \in A}$ is eventually in $U$ and there exists $\be_x \in A$ such that for each $\al \in A$, $\al \geq \be_x$ implies that $x_{\al} \in U$. Since $x_{\al} \rightarrow y$, $(x_{\al})_{\al \in A}$ is eventually in $V$ and there exists $\be_y \in A$ such that for each $\al \in A$, $\al \geq \be_y$ implies that $x_{\al} \in V$. Since $A$ is directed, there exists $\be \in A$ such that $\be \geq \be_x, \be_y$. Hence 
			\begin{align*}
				x_{\be} 
				& \in U \cap V \\
				& = \varnothing
			\end{align*}
			which is a contradiction. So $x = y$.
			\item $(2) \implies (3)$: \\
			Let $(x_{\al}, y_{\al})_{\al \in A} \subset \Del_X$ be a net and $(x,y) \in X \times X$. Then for each $\al \in A$, $x_{\al} = y_{\al}$. Suppose that $(x_{\al}, y_{\al}) \rightarrow (x,y)$. So $x_{\al} \rightarrow x$ and $x_{\al} \rightarrow y$. Hence $x = y$ and $(x,y) \in \Del_X$. Thus $\Del_X$ is closed.
			\item $(3) \implies (1)$: \\
			Suppose that $\Del_X$ is closed. Let $x,y \in X$. Suppose that $x \neq y$. Then $(x,y) \in \Del_X^c$. Recall that $\MB = \{A \times B: A,B \subset X \text{ and $A$, $B$ are open}\}$ is a basis for the product topology on $X \times X$. Since $\Del_X^c$ is open and $(x,y) \in \Del_X^c$, there exist $A \times B \in \MB$ such that $(x,y) \in A \times B \subset \Del_X^c$. Suppose that $A \cap B \neq \varnothing$. Then there exists $z \in A \cap B$. Hence $(z,z) \in A \times B$. This is a contradiction since $A  \times B \subset \Del_X^c$. Thus $x \in A$, $y \in B$ and $A \cap B = \varnothing$ and $A$, $B$ are open. Since $x,y \in X$ are arbitrary, $X$ is Hausdorff. 
		\end{itemize}
	\end{proof}
	
	\begin{ex}
		Let $X$ be a topological space and $\sim$ an equivalence relation on $X$. If $\pi: X \rightarrow X / \sim$ is open, then $X / \sim$ is Hausdorff iff $\sim$ is closed in $X \times X$.
	\end{ex}
	
	\begin{proof}
		Suppose that $\pi:X \rightarrow X / \sim$ is open. 
		\begin{itemize}
			\item $(\implies)$: \\
			Suppose that $X/\sim$ is Hausdorff. Let $(x_{\al}, y_{\al})_{\al \in A} \subset \, \sim$ be a net and $(x,y) \in X \times X$. Suppose that $x_{\al}, y_{\al} \rightarrow (x,y)$. Then $x_{\al} \rightarrow x$ and $y_{\al} \rightarrow y$. Since $\pi:X \rightarrow X / \sim$ is continuous, $\pi(x_{\al}) \rightarrow \pi(x)$ and $\pi(y_{\al}) \rightarrow \pi(y)$. Since for each $\al \in A$, $x_{\al} \sim y_{\al}$, we have that 
			\begin{align*}
				\pi(x_{\al}) 
				& = \pi(y_{\al})\\
				& \rightarrow \pi(y)
			\end{align*}
			Since $X/ \sim$ is Hausdorff, $\pi(x) = \pi(y)$. Hence $x \sim y$ and $(x,y) \in \, \sim$. Thus $\sim$ is closed in $X \times X$.\\
			\item $(\impliedby)$: \\
			Conversely, suppose that $\sim$ is closed in $X \times X$ is closed. Let $\bar{x}, \bar{y} \in X / \sim$. Suppose that $\bar{x} \neq \bar{y}$. Then $(x,y) \in \, \sim^c$. Recall that $\MB =\{A \times B: A,B \subset X \text{ and $A, B$ are open} \}$ is a basis for $X \times X$. Since  $\sim^c$ is open and $(x,y) \in \, \sim^c$, there exist $A,B \subset X$ such that $A,B$ are open and $(x,y) \in A \times B \subset \, \sim^c$. Thus $x \in A$ and $y \in B$. Since $\pi$ is open, $\pi(A) = \bar{A}$ and $\pi(B) = \bar{B}$ are open. Suppose for the sake of contradiction that $\pi(A) \cap \pi(B) \neq \varnothing$. Then there exists $z \in X$ such that $\bar{z} \in \pi(A) \cap \pi(B)$. Therefore there exist $z_A \in A$ and $z_B \in B$ such that $z_A \sim z \sim z_B$. Then $(z_A, z_B) \in A \times B$ and $(z_A, z_B) \in \, \sim$. This is a contradiction since $A \times B \subset \, \sim^c$. So $\pi(A) \cap \pi(B) = \varnothing$. Thus $\bar{x} \in \pi(A)$, $\bar{y} \in \pi(B)$, $\pi(A)$, $\pi(B)$ are open and $\pi(A) \cap \pi(B) = \varnothing$. Since $\bar{x}, \bar{y} \in X / \sim$ are arbitrary, $X / \sim$ is Hausdorff.
		\end{itemize}
	\end{proof}

	\begin{defn}
		Let $(X, \MT)$ be a topological space. Then $(X, \MT)$ is said to be \tbf{second-countable} if there exists $\MB \subset \MT$ such that 
		\begin{enumerate}
			\item $\MB$ is a basis for $\MT$
			\item $\MB$ is countable
		\end{enumerate} 
	\end{defn}

	\begin{ex}
		Let $(X_{\al}, \MT_{\al})_{\al \in A}$ be a collection of topological spaces. Set $X = \prod\limits_{\al \in A} X_{\al}$ and denote the product topology on $X$ by $\MT_X$. Suppose that $A$ is countable. If for each $\al \in A$, $(X_{\al}, \MT_{\al})$ is second-countable, then $(X, \MT_X)$ is second-countable. 
	\end{ex}

	\begin{proof}
		Suppose that for each $\al \in A$, $(X_{\al}, \MT_{\al})$ is second-countable. Then for each  $\al \in A$, there exists $\MB_{\al} \subset \MT_{\al}$ such that $\MB_{\al}$ is a basis for $\MT_{\al}$ and $\MB_{\al}$ is countable. Set 
		\begin{align*}
			\MB = 
			& \bigg \{\prod_{\al \in A} U_{\al}: \text{there exists $J \subset A$ such that $ \# J < \infty$, } \\
			& \text{ for each $\al \in J$, $U_{\al} \in \MB_{\al}$ and for each $\al \in J^c$, $U_{\al} = X_{\al}$ } \bigg \}
		\end{align*}  
		An exercise in the section on the product topology implies that $\MB$ is a basis for $\MT_X$. Since $A$ is countable, $\MB$ is countable. Hence $\MT_X$ is second-countable. 
	\end{proof}


	
	
	
	
	
	
	
	
	
	
	
	
	
	
	
	\newpage
	\subsection{Compactness}
	
	\begin{defn} \ld{}
		Let $X$ be a topological space and $E \subset X$. Then $E$ is said to be \tbf{precompact} if $\cl E$ is compact.
	\end{defn}







	
	
	
	
	
	
	
	
	
	
	
	
	
	
	
	
	
	
	\newpage
	\subsection{Semi-continuity}
	
	\begin{defn} \ld{}
	Let $X$ be a topological space, $f: X \rightarrow \Ru$ and $x_0 \in X$. Then $f$ is said to be \tbf{lower semicontinuous at $x_0$} if $$\liminf_{x \rightarrow x_0}f(x) \geq f(x_0)$$ and $f$ is said to be \tbf{lower semicontinuous} if for each $x_0 \in X$, $f$ is lower semicontinuous at $x_0$. 
	\end{defn}
	
	\begin{ex} \lex{}
	Let $X$ be a topological space and $f: X \rightarrow \Ru$. Then $f$ is \lsc iff for each $\al \in \R$, $f^{-1}((\al, \infty])$ is open. 
	\end{ex}
	
	\begin{proof}
	Suppose that $f$ is \lsc. Let $\al \in \R$ and $x_0 \in f^{-1}(\al, \infty]$. Put $\ep = f(x_0) - \al$. By definition, $$\sup_{V \in N_{x_0}} \inf_{x \in V \setminus \{x_0\}} f(x) \geq f(x_0)$$ Choose $V_{\ep} \in N_{x_0}$ such that 
	\begin{align*}
	\inf_{x \in V_{\ep} \setminus \{x_0\}} f(x)  
	&> f(x_0) - \ep \\
	&= \al
\end{align*}
Then $V_{\ep}^o \in \MN_{x_0}$ is open and 
	\begin{align*}
		V_{\ep}^o 
		& \subset V_{\ep} \\
		&\subset f^{-1}((\al, \infty])
	\end{align*} 
	So $f^{-1}((\al, \infty])$ is open. \\
	Conversely, suppose that for each $\al \in \R$, $f^{-1}((\al, \infty])$ is open. Let $x_0 \in X$. Put $\al = f(x_0)$. For $n \in \N$, define $V_n = f^{-1}((f(x_0)-1/n, \infty]) $. Then for each $n \in \N$, $V_n \in \MN_{x_0}$ and 
	\begin{align*}
	\liminf_{x \rightarrow x_0} f(x) 
	&= \sup_{V \in \MN_{x_0}} \inf_{x \in V \setminus \{x_0\}} f(x) \\
	& \geq \sup_{n \in \N} \inf_{x \in V_n \setminus \{x_0\}} f(x) \\
	& \geq \sup_{n \in \N} f(x_0)-1/n \\
	&= f(x_0) \\
	\end{align*}
	So $f$ is \lsc.
	\end{proof}

	\begin{defn}
		Let $X$ be a topological space and $f: X \rightarrow \R$. We define the \tbf{epigraph of $f$}, denoted $\epi f$, by 
		$$\epi f = \{(x, y) \in X \times \R: f(x) \leq y\}$$
	\end{defn}

	\begin{ex}
		Let $X$ be a topological space and $f: X \rightarrow \R$. Then $f$ is lower semicontinuous iff $\epi f$ is closed.
	\end{ex}

	\begin{proof}
		Suppose that $f$ is lower semicontinuous. Let $(x_{\al}, y_{\al})_{\al \in A} \subset \epi f$ be a net and $(x, y) \in X \times \R$. Then for each $\al \in A$, $f(x_{\al}) \leq y_{\al}$. Suppose that $(x_{\al}, y_{\al}) \rightarrow (x, y)$. Then $x_{\al} \rightarrow x$ and $y_{\al} \rightarrow y$. Therefore 
		\begin{align*}
			f(x) 
			& \leq \liminf_{t \rightarrow x} f(t) \\
			& \leq \liminf f(x_{\al}) \\
			& \leq \liminf y_{\al} \\
			&= y 
		\end{align*}
	So $(x, y) \in \epi f$ and $\epi f$ is closed. \\
	Conversely, suppose that $\epi f$ is closed. 
	\end{proof}

	\begin{ex}
		Let $X$ be a topological space and $ (f_{\lam})_{\lam \in \Lam} \subset (-\infty, \infty]^X$. Suppose that for each $\lam \in \Lam$, $f_\lam$ is \lsc. Set $f = \sup\limits_{\lam \in \Lam} f_{\lam}$. Then $f$ is \lsc.
	\end{ex}

	\begin{proof}
		Let $\al \in \R$ and $x \in X$. Then 
		\begin{align*}
			x \in f^{-1}((\al, \infty])
			& \iff \sup_{\lam \in \Lam} f_{\lam}(x) > \al \\
			& \iff \text{there exists $\lam \in \Lam$ such that } f_{\lam}(x) > \al \\
			& \iff \text{there exists $\lam \in \Lam$ such that } x \in f_{\lam}^{-1}((\al, \infty]) \\
			& \iff x \in \bigcup_{\lam \in \Lam} f_{\lam}^{-1}((\al, \infty])
		\end{align*}
		Since for each $\lam \in \Lam$, $f_{\lam}^{-1}((\al, \infty])$ is open, $f^{-1}((\al, \infty]) = \bigcup\limits_{\lam \in \Lam} f_{\lam}^{-1}((\al, \infty])$ is open. So $f$ is \lsc.
	\end{proof}
























	\newpage
	\section{Locally Convex Spaces}
	
	\subsection{Topological Vector Spaces}
	
	\begin{defn}
		Let $X$ be a vector space and $\MT$ a topology on $X$. Then $X$ is said to be a \tbf{topological vector space} if
		\begin{enumerate}
			\item  addition $X \times X \rightarrow X$ is continuous  \item scalar multiplication $\C \times X \rightarrow X$ is continuous
		\end{enumerate}
	\end{defn}
	
	\begin{note}
		We usually suppress the topology $\MT$.
	\end{note}

	\begin{ex}
		Let $X$ be a topological vector space, $(\lam_{\al})_{\al \in A} \subset \C$, $(x_{\al})_{\al \in A}$, $(y_{\al})_{\al \in A} \subset X$ nets and $\lam \in \C$, $x, y \in X$. If $\lam_{\al} \rightarrow \lam$, $x_{\al} \rightarrow x$ and $y_{\al} \rightarrow y$, then $x_{\al} + \lam_{\al}y_{\al} \rightarrow x + \lam y$.
	\end{ex}

	\begin{proof}
		Clear since addition and scalar multiplication are continuous.
	\end{proof}
	
	\begin{ex}
		Let $X$ be a topological vector space, $a \in X$ and $\lam \in \C^{\times}$. Define $f,g: X \rightarrow X$ by $f(x) = x + y$ and $g(x) = \lam x$. Then $f$ and $g$ are homeomorphisms. 
	\end{ex}
	
	\begin{proof}
		Since $X$ is a topological vector space, $f$ and $g$ are continuous. Clearly $f$ and $g$ are bijections with $f^{-1}(x) = x - y$ and $g^{-1}(x) = \lam^{-1}x$. Again, since $X$ is a topological vector space, $f^{-1}$ and $g^{-1}$ are continuous.
	\end{proof}

	\begin{ex}
		Let $X$ be a topological vector space, $x,y \in X$ and $U \in \MN_x$. If $U$ is open, then there exists $r >0$ such that for each $t \in \R$, $|t| \leq r$ implies that $x+ ty \in U$.
	\end{ex}

	\begin{proof}
		Suppose that $U$ is open. For the sake of contradiction, suppose that for each $r > 0$, there exists $t \in \R$ such that $t \leq r$ and $x+ ty \not \in U$. Then for each $n \in \N$, there exists $t_n \in \R$ such that $|t_n| \leq 1/n$ and $x + t_ny \in U^c$. Since $t_n \rightarrow 0$, 
		\begin{align*}
			x + t_ny 
			& \rightarrow x + 0y \\
			&= x
		\end{align*}
		Since $U^c$ is closed, $x \in U^c$. This is a contradiction. Hence there exists $r >0$ such that for each $t \in \R$, $|t| \leq r$ implies that $x+ ty \in U$.
	\end{proof}

	\begin{ex}
		Let $X$ be a topological vector space and $A$, $B \subset X$. If $A$ is open, then $A + B$ is open.
	\end{ex}
	
	\begin{proof} \
		Suppose that $A$ is open. Then for each $b \in B$, $A + b$ is open. Since 
		$$A + B = \bigcup_{b \in B} A + b$$
		we have that $A + B$ is open.
	\end{proof}

	\begin{ex}
		Let $X$ be a topological vector space and $A,B \subset X$. Suppose that $A$ is compact, $B$ is closed and $A \cap B = \varnothing$. Then there exists $U \in \MN_0$ such that $U$ is open and $(A + U) \cap B = \varnothing$. 
	\end{ex}
	
	\begin{proof}
		Set $\Gam = \{U \in \MN_0: U \text{ is open}\}$ and order $\Gam$ by reverse inclusion, so that $\Gam$ is a directed set. For the sake of contradiction, suppose that for each $U \in \Gam$, $(A + U) \cap B \neq \varnothing$. Then for each $\gam \in \Gam$, there exist $a_{\gam} \in A$ and $u_{\gam} \in \gam$ such that $a_{\gam} + u_{\gam} \in B$. Let $V \in \MN_0$. Since $\Int V \in \Gam$
		\begin{align*}
			u_{\Int V} 
			& \in \Int V \\
			& \subset V
		\end{align*}
		Since $V \in \MN_0$ is arbitrary, $u_{\gam} \rightarrow 0$.  Since $A$ is compact, there exists $a \in A$ and a subnet $(a_{\gam_{\ze}})_{\ze \in Z}$ of $(a_{\gam})_{\gam \in \Gam}$ such that $a_{\gam_{\ze}} \rightarrow a$. Then $a_{\gam_{\ze}} + u_{\gam_{\ze}} \rightarrow a$. Since $(a_{\gam_{\ze}} + u_{\gam_{\ze}})_{\ze \in Z} \subset B$ and $B$ is closed, we have that $a \in B$. This is a contradiction since $A \cap B = \varnothing$. So there exists $U \in \MN_0$ such that $U$ is open and $(A+U) \cap B = \varnothing$.
	\end{proof}

	\begin{ex}
		Let $X$ be a topological vector space and $U \in \MN_0$. If $U$ is open, then there exists $V \in \MN_0$ such that $V$ is open and $V+V \subset U$.
	\end{ex}

	\begin{proof}
		Suppose that $U$ is open. Set $\Gam = \{V \in \MN_0: V \text{ is open}\}$ and order $\Gam$ by reverse inclusion, so that $\Gam$ is a directed set. For the sake of contradiction, suppose that for each $V \in \MN_0$, if $V$ is open, then $V + V \not \subset U$. Then for each $\gam \in \Gam$, there exists $x_{\gam}, y_{\gam} \in \gam$ such that $x_{\gam}+y_{\gam} \in U^c$. Let $W \in \MN_0$. Set $\be = \Int V$. Then $\be \in \Gam$. Then for each $\gam \geq \be$, 
		\begin{align*}
			x_{\gam},y_{\gam} 
			& \in \gam \\
			& \subset \be \\
			& \subset W
		\end{align*}
		So that $(x_{\gam})_{\gam \in \Gam}$ and $(y_{\gam})_{\gam \in \Gam}$ are eventually in $W$. Since $W \in \MN_0$ is arbitrary, $x_{\gam} \rightarrow 0$ and	$y_{\gam} \rightarrow 0$. Therefore $x_{\gam} + y_{\gam} \rightarrow 0$. Since for each $\gam \in \Gam$, $x_{\gam} + y_{\gam} \in U^c$ and $U^c$ is closed, $0 \in U^c$. This is a contradiction, so there exists $V \in \MN_0$ such that $V$ is open and $V+V \subset U$.
		\end{proof}
	

	\begin{defn} \ld{55001}\
		Let $X$ be a vector space over $\C$ and $T :X \rightarrow \C$. Then $T$ is said to be a \tbf{linear functional on} $X$ if $T$ is linear. We define the \tbf{algebraic dual space of} $X$, denoted $X^*$, by $ X^* = \{ T:X \rightarrow \C: T \text{ is linear} \} $
	\end{defn}
	
	\begin{note}
		We define $X^*$ similarly when $X$ is a vector space  over $\R$.
	\end{note}
	
	\begin{defn} \ld{55001}\
		Let $X$ be a topological vector space over $\C$ and $T :X \rightarrow \C$. We define the \tbf{dual space of} $X$, denoted $X^*$, by $ X^* = \{ T:X \rightarrow \C: T \text{ is linear and continuous} \} $
	\end{defn}
	
	\begin{note}
		We define $X^*$ similarly when $X$ is a vector space  over $\R$.
	\end{note}

	\begin{ex}
		Let $X$ be a topological vector space. Then $X^*$ is a vector space. 
	\end{ex}

	\begin{proof}
		Clear.
	\end{proof}

	\begin{ex}
		Let $X$, $Y$ be topological vector spaces and $\phi:X \rightarrow Y$. Suppose that $\phi$ is linear. Then $\phi$ is continuous iff $\phi$ is continuous at $0$.
	\end{ex}
	
	\begin{proof}
		If $\phi$ is continuous, then $\phi$ is continuous at $0$.\\
		Conversely, suppose that $\phi$ is continous at $0$. Let $(x_{\al})_{\al \in A} \subset X$ be a net and $x \in X$. Suppose that $x_{\al} \rightarrow x$. Then $x_{\al} - x \rightarrow 0$. Hence 
		\begin{align*}
			\phi(x_{\al}) - \phi(x) 
			&= \phi(x_{\al} - x) \\
			&\rightarrow \phi(0) \\
			&= 0
		\end{align*}
		Therefore $\phi(x_{\al}) \rightarrow \phi(x)$ and $\phi$ is continuous at $x$. Since $x \in X$ is arbitrary, $\phi$ is continuous. 
	\end{proof}

\begin{ex}
	Let $X$ be a topological vector space and $\phi :X \rightarrow \C$ linear. Then $\phi \in X^*$ iff $|\phi|$ is continuous. 
\end{ex}

\begin{proof}
	Suppose that $\phi$ is continuous. Since  $|\cdot|:\C \rightarrow \Rg$ is continuous, $|\phi|$ is continuous. \\
	Conversely, suppose that $|\phi|$ is continuous. Let $(x_{\al})_{\al \in A} \subset X$ be a net and $x \in X$. Suppose that $x_{\al} \rightarrow x$. Then $x_{\al} - x \rightarrow 0$. Therefore 
	\begin{align*}
		|\phi(x_{\al}) - \phi(x)| 
		&= |\phi(x_{\al} - x)| \\
		& \rightarrow |\phi(0)| \\
		&= 0
	\end{align*} 
	So $\phi(x_{\al}) \rightarrow \phi(x)$ and $\phi$ is continuous.
\end{proof}

	\begin{ex}
		Let $X$ be a real topological vector space and $\phi \in X^*$. If $\phi$ is not constant, then $\phi$ is open. \\
		\tbf{Hint:} There exists $x_* \in X$  such that $\phi(x_*) = 1$ and for each $U \subset X$ open and $x \in U$, there exists $r >0$ such that for each $t \in \R$, $|t| \leq r$ implies that $x + tx_* \in U$. 
	\end{ex}
	
	\begin{proof}
		Suppose that $\phi$ is not constant. Then there exists $x_0 \in X$ such that $\phi(x_0) \neq 0$. Set $x_* = \phi(x_0)^{-1} x_0$. Then
		\begin{align*}
			\phi(x_*)
			&= \phi(\phi(x_0)^{-1}x_0) \\
			&= \phi(x_0)^{-1}\phi(x_0) \\
			&= 1
		\end{align*}
		Let $U \subset X$ be open and $y \in \phi(U)$. Then there exists $x \in U$ such that $\phi(x) = y$. Sine $U$ is open, a previous exercise implies that there exists $r > 0$ such that for each $t \in \R$, $\|t\| \leq r$ implies that $x + tx_* \in U$. Let $t \in (-r, r)$. Then $\phi(x+ tx_*) \in \phi(U)$. Since 
		\begin{align*}
			\phi(x+ tx_*) 
			&= \phi(x) + t\phi(x_*) \\
			&= y + t
		\end{align*}
	we have that $(y-r, y+r) \subset \phi(U)$. Since $y \in U$ is arbitrary, $\phi(U)$ is open thus $\phi$ is open. 
	\end{proof}

	\begin{defn}
		Let $X$ be a vector space and $\phi: X \rightarrow \C$. Then $\phi$ is said to be \tbf{real-linear} if for each $x,y \in X$ and $\lam \in \R$, $\phi(x+ \lam y) = \phi(x) + \lam \phi(y)$.
	\end{defn}

	\begin{ex}
		Let $X$ be a topological vector space and $\phi \in X^*$. Then $\Re \phi$ is continuous and real-linear. 
	\end{ex}

	\begin{proof}
		Clear.
	\end{proof}

	\begin{ex}
		Let $X$ be a topological vector space and $f:X \rightarrow \R$. If $f$ is continuous and real-linear, then there exists a unique $\phi \in X^*$ such that $\Re \phi = f$. \\
		\tbf{Hint:} For each $z \in \C$, $z = \Re(z) - i \Re(iz)$
	\end{ex}
	
	\begin{proof}
		Suppose that $f$ is continuous and real-linear. Define $\phi:X \rightarrow \C$ by $\phi(x) = f(x) -if(ix)$. Then $\phi$ is continuous. Let $x,y \in X$ and $\lam \in C$. Write $\lam = a + bi$. Then 
		\begin{align*}
			\phi(x + \lam y)
			&= f(x + \lam y) -if(i(x + \lam y)) \\
			&= f(x + ay + i by) -if(ix + iay - by) \\
			&= f(x) + af(y) + bf(iy) - if(ix) - iaf(iy) +ibf(y) \\
			&= [f(x) - if(ix)] + a[f(y) - if(iy)] + ib[f(y) -if(iy)] \\
			&= \phi(x) + a\phi(y) +ibf(y) \\
			&= \phi(x) + \lam \phi(y) 
		\end{align*}
	So $\phi$ is linear and $\phi \in X^*$. Let $\psi \in X^*$. Suppose that $f = \Re \psi$. 
	Then for each $x \in X$,
	\begin{align*}
		\phi(x) 
		&= f(x) - if(ix) \\
		&= \Re \psi(x) -i \Re \psi(ix) \\
		&= \Re \psi(x) - \Re i\psi(x) \\
		&= \Re \psi(x) + \Im \psi(x) \\
		&= \psi(x)
	\end{align*} 
	So $\psi = \phi$ and $\phi$ is unique.
	\end{proof}


	





















	
	\newpage	
	\subsection{Sublinear Functionals}
	
	\begin{defn} \ld{55003}
		Let $X$ be a real vector space and $p:X \rightarrow \R$. Then $p$ is said to be a \tbf{sublinear functional} if for each $x,y \in X$, $\lam \geq 0$, 
		\begin{enumerate}
			\item $p(x+y) \leq p(x) + p(y)$
			\item $p(\lam x ) = \lam p(x)$
		\end{enumerate}  
	\end{defn}
	
	\begin{ex} \lex{55004}
		Let $X$ be a vector space and $p: X \rightarrow \R$ be a sublinear functional. Then $p(0) = 0$.
	\end{ex}
	
	\begin{proof} Set $\lam = 0$. Then 
		\begin{align*}
			0
			&= \lam p(0) \\
			&= p(\lam 0) \\
			&= p(0)
		\end{align*}
	\end{proof}
	
	\begin{proof}
		Clear
	\end{proof}
	
	\begin{ex} \lex{55008}
		Let $X$ be a vector space and $p:X \rightarrow \R$ a sublinear functional. Then for each $x, y \in X$
		\begin{enumerate}
			\item $-p(-x) \leq p(x)$
			\item $- p(y-x) \leq p(x) - p(y) \leq p(x-y)$
		\end{enumerate}
	\end{ex}
	
	\begin{proof}
		Let $x, y \in X$.
		\begin{enumerate}
			\item We have
			\begin{align*}
				0
				&= p(0) \\ 
				&= p(x - x) \\
				& \leq p(x) + p(-x)
			\end{align*}
			So $-p(-x) \leq p(x)$.
			\item We have
			\begin{align*}
				p(x)
				&= p(x -y + y) \\
				& \leq p(x-y) + p(y)
			\end{align*}
			So $p(x) - p(y) \leq p(x-y)$. Switching $x$ and $y$ gives us $p(y) - p(x) \leq p(y-x)$ and multiplying both sides by $-1$ yields $-p(y-x) \leq p(x) - p(y)$ \\ 
			Putting these two together, we see that $$-p(y-x) \leq p(x) - p(y) \leq p(x-y)$$
		\end{enumerate}
	\end{proof}
	
	
	
	\begin{thm}\tbf{Hahn-Banach Theorem for Sublinear Functionals}\\
		Let $X$ be a vector space, $p:X \rightarrow \R$ a sublinear functional, $M \subset X$ a subspace and $f:M \rightarrow \R$ a linear functional. If for each $x \in M$, $ f(x)  \leq p(x)$, then there exists a linear functional $F:X \rightarrow \R$ such that for each $x \in X$, $F(x) \leq p(x)$ and $F|_{M}=f$.
	\end{thm}
	
	\begin{ex} \lex{55011}
		Let $X$ be a vector space and $p:X \rightarrow \R$ a sublinear functional. Then there exists a linear functional $F: X \rightarrow \R$ such that for each $x \in X$, $F(x) \leq p(x)$.
	\end{ex}
	
	\begin{proof}
		Take $M = \{0\}$ and $f \equiv 0$ and apply the Hahn-Banach theorem.
	\end{proof}	
	
	\begin{ex} \lex{55012} \tbf{Equivalency of linearity (General Case)}
		Let $X$ be a vector space and $p:X \rightarrow \R$ a sublinear functional. Then the following are equivalent:
		\begin{enumerate}
			\item there exists a unique $F \in X^*$ such that $F \leq p$
			\item for each $x \in X$, $-p(-x) = p(x)$
			\item $p$ is linear
		\end{enumerate}	
		\tbf{Hint:} If there exists $x \in X$ such that $-p(-x) \neq p(x)$, define $f_1,f_2 :\spn(x) \rightarrow \R$ by $f_1(tx) = t p(x)$ and $f_2(tx) = -tp(-x)$
	\end{ex}	
	
	\begin{proof} \
		\begin{itemize}
			\item $(1) \implies (2)$: \\ 
			Suppose that there exists a unique $F \in X^*$ such that $F \leq p$. For the sake of contradiction, suppose that there exists $x \in X$ such that $-p(-x) \neq p(x)$. Define $f_1,f_2: \spn(x) \rightarrow \R$ by $$f_1(tx) = t p(x)$$ and $$f_2(tx) = -tp(-x)$$ Let $y \in \spn(x)$. Then there exists $t \in \R$ such that $y = tx$. Then for each $k \in \R$,
			\begin{align*}
				f_1(ky)
				&= f_1(ktx) \\
				&= ktp(x) \\
				&= k f_1(tx) \\
				&= k f_1(y)
			\end{align*}
			Similarly, $f_2(ky) = kf_2(y)$ and so $f_1, f_2 \in \spn(x)^*$. 
			If $t \geq 0$, then 
			\begin{align*}
				f_1(y) 
				&= f_1(tx) \\
				&= tp(x) \\
				&= p(tx) \\
				&= p(y) 
			\end{align*}
			If $t <0$, then 	
			\begin{align*}
				f_1(y) 
				&= f_1(tx) \\
				&= tp(x) \\
				&= -|t|p(x) \\
				&= -p(|t|x) \\
				&= -p(-tx) \\
				& \leq p(tx) \\
				&= p(y)  
			\end{align*}
			So $f_1 \leq p$ on $\spn(x)$. Similarly, $f_2 \leq p$ on $\spn(x)$. The Hahn-Banach theorem implies that there exist $F_1, F_2 \in X^*$ such that $F_1, F_2 \leq p$ and $F_1 = f_1, F_2 = f_2$ on $\spn(x)$. By the assumption of uniqueness, $F_1 = F_2$. This is a contradiction since 
			\begin{align*}
				F_1(x) 
				&= p(x) \\
				& \neq -p(-x) \\
				& = F_2(x) 
			\end{align*}		
			So for each $x \in X$, $-p(-x) = p(x)$. 
			\item $(2) \Rightarrow (3)$: \\
			Suppose that for each $x \in X$, $-p(-x) = p(x)$. The previous exercise implies that there exists $F \in X^*$ such that $F \leq p$. Let $x \in X$. Then 
			\begin{align*}
				-F(x) 
				&= F(-x) \\
				& \leq p(-x) \\
				&= -p(x)
			\end{align*}	
			So $p(x) \leq F(x)$ and $p \leq F$. Therefore $p = F$ and $p$ is linear.  
			\item $(3) \implies (1)$: \\ 
			Suppose that $p$ is linear. Let $F \in X^*$. Suppose that $F \leq p$. Let $x \in X$. Then as in the case for $(2) \implies (3)$, we have that
			\begin{align*}
				-F(x) 
				&= F(-x) \\
				& \leq p(-x) \\
				&= -p(x)
			\end{align*}	 
			which implies that $p = F$. So $p$ is the unique linear function $F \in X^*$ such that $F \leq p$.
		\end{itemize}
	\end{proof}
	






















	\newpage
	\subsection{Seminorms}
	\begin{defn} \ld{55005}
		Let $X$ be a vector space and $p:X \rightarrow \R$. Then $ p$ is said to be a \tbf{seminorm} if for each $x,y \in X$, $\lam \in \R$, 
		\begin{enumerate}
			\item $p(x+y) \leq p(x) + p(y)$
			\item $p(\lam x) = |\lam| p(x)$
		\end{enumerate}  
	\end{defn}
	
	\begin{ex} \lex{55006}
		Let $X$ be a vector space and $p: X \rightarrow \R$ be a seminorm, then $p$ is a sublinear functional.
	\end{ex}
	
	\begin{proof}
		Clear
	\end{proof}

	\begin{ex}
		Let $X$ be a vector space and $\phi \in X^*$. Then $|\phi|$ is a seminorm on $X$.
	\end{ex}
	
	\begin{proof}
		Clear.
	\end{proof}

	\begin{ex}
		Let $X,Y$ be a vector spaces, $T \in L(X, Y)$ and $p$ a seminorm on $Y$. Then $p \circ T$ is a seminorm on $X$.
	\end{ex}
	
	\begin{proof}
		Clear.
	\end{proof}
	
	\begin{ex} \lex{55007}
		Let $X$ be a vector space and $p: X \rightarrow \R$ be a seminorm. Then $p \geq 0$. 
	\end{ex}
	
	\begin{proof}
		Let $x \in X$. Then 
		\begin{align*}
			0 
			&= p(0) \\ 
			&= p(x - x) \\
			&\leq  p(x) + p(-x) \\
			&= p(x) + p(x) \\
			&= 2p(x)
		\end{align*}
		So $p(x) \geq 0$.
	\end{proof}

	\begin{ex} \tbf{Reverse Triangle Inequality:} \\
		Let $X$ be a vector space and $p:X \rightarrow \Rg$ be a seminorm on $X$. Then for each $x ,y \in X$, $|p(x) - p(y)| \leq p(x - y)$.  
	\end{ex}
	
	\begin{proof}
		Let $x, y \in X$. Then 
		\begin{align*}
			p(x)
			&= p(x -y + y) \\
			&\leq p(x - y) + p(y) 
		\end{align*}
		So $p(x) - p(y) \leq p(x - y)$. 
		Similarly, $p(y) \leq p(y - x) + p(y)$ and so $p(x) - p(y) \leq p(x - y)$. Therefore $|p(x) - p(y)| \leq p(x - y)$.
	\end{proof}
	
	\begin{ex}
		Let $X$ be a vector space, $p: X \rightarrow \Rg$ a seminorm and $\phi \in X^*$. Then $\phi \leq p$ iff $|\phi| \leq p$. 
	\end{ex}
	
	\begin{proof}
		Suppose that $\phi \leq p$. Let $x \in X$. Then 
		\begin{align*}
			-\phi(x)
			& = \phi(-x) \\
			& \leq p(-x) \\
			& = p(x) \\
		\end{align*}
		So $-p(x) \leq \phi(x)$. Hence $-p \leq \phi \leq p$. Thus $|\phi| \leq p$. \\
		Conversely, if $|\phi| \leq p$, then clearly $\phi \leq p$.
	\end{proof}
	
	\begin{defn}
		Let $X$ be a vector space and $p:X \rightarrow \Rg$ be a seminorm on $X$. We define the \tbf{kernel of $p$}, denoted $\ker p$, by $\ker p = p^{-1}(\{0\})$.
	\end{defn}
	
	\begin{ex}
		Let $X$ be a vector space and $p: X \rightarrow \Rg$ a seminorm. Then $\ker p$ is a subspace of $X$.
	\end{ex}
	
	\begin{proof}
		Let $x, y \in \ker p$ and $\lam \in \C$. Then $p(x) = p(y) = 0$. Thus 
		\begin{align*}
			p(x + \lam y) 
			&\leq p(x) + p(\lam y) \\
			&= p(x) + |\lam|p(y) \\
			&= 0
		\end{align*} 
		So $x + \lam y \in N$ and $N$ is a subspace.
	\end{proof}
	
	
	\begin{defn}
		Let $X$ be a vector space and $p:X \rightarrow [0, \infty)$ a seminorm on $X$. We define the \tbf{norm induced by $p$}, denoted $\bar{p} : X / \ker p \rightarrow  \Rg$, by $$\bar{p}(\bar{x}) = p(x)$$
	\end{defn}
	
	\begin{ex}
		Let $X$ be a vector space and $p:X \rightarrow [0, \infty)$ a seminorm on $X$. Then $\bar{p} : X / \ker p \rightarrow  \Rg$ is well defined and a norm. 
	\end{ex}
	
	\begin{proof}
		Let $x, y \in X$. Suppose that $\bar{x} = \bar{y}$. Then there exists $n \in \ker p$ such that $x = y + n$. Therefore, 
		\begin{align*}
			\bar{p}(\bar{x}) 
			&= p(x) \\
			&= p(y + n) \\
			&\leq p(y) + p(n) \\
			&= p(y) \\
			&= \bar{p}(\bar{y})
		\end{align*}
		and 
		\begin{align*}
			\bar{p}(\bar{y}) 
			&= p(y) \\
			&= p(x - n) \\
			&\leq p(x) + p(n) \\
			&= p(x) \\
			&= \bar{p}(\bar{x})
		\end{align*}
		So $\bar{p}(\bar{x}) = \bar{p}(\bar{y})$ and $\bar{p}: X / \ker p \rightarrow \Rg$ is well defined. Let $x \in X$. Suppose that $\bar{x} = \bar{0}$. Then there exists $n \in \ker p$ such that $x = n$. Therefore 
		\begin{align*}
			\bar{p}(\bar{x}) 
			&= p(x) \\
			&= p(n) \\
			&= 0
		\end{align*}
		So $\bar{p}$ is a norm.
	\end{proof}

	
	\begin{defn}
		Let $X$ be a vector space, $p:X \rightarrow \Rg$ a seminorm on $X$, $x \in X$ and $r >0$. We define the 
		\begin{itemize}
			\item \tbf{open semiball of $p$ at $x$ of radius $r$}, denoted $B_p(x, r)$, by $$B_p(x, r) = \{y \in X: p(x - y) < r\}$$
			\item \tbf{closed semiball of $p$ at $x$ of radius $r$}, denoted $\bar{B}_p(x, r)$, by $$\bar{B}_p(x, r) = \{y \in X: p(x - y) \leq r\}$$
		\end{itemize}
	\end{defn}

	\begin{ex} 
		Let $X$ be a vector space, $p:X \rightarrow \Rg$ a seminorm on $X$, $x \in X$ and $r >0$. Then $B_p(x, r) =  x + rB_p(0, 1)$. 
	\end{ex}

	\begin{proof}
		Let $y \in B_p(x, r)$. Then  
		\begin{align*}
			p(r^{-1}(y - x)) 
			&= r^{-1}p(y - x) \\
			&< r^{-1} r \\
			&= 1
		\end{align*}
		So $r^{-1}(y - x) \in B_p(0, 1)$. By definition, there exists $u \in B_p(0,1)$ such that $r^{-1}(y - x) = u$, which implies that 
		\begin{align*}
			y 
			&=  x + ru \\
			&\in x + rB_p(0, 1)
		\end{align*} 
		Conversely, let $y \in x + rB_p(0, 1)$. By definition, there exists $u \in B_p(0,1)$ such that $y = x + ru$. Then 
		\begin{align*}
			p(y - x) 
			&= p(ru) \\
			&= rp(u) \\
			&< r
		\end{align*}
	So $y \in B_p(x, r)$ 
	\end{proof}

	\begin{ex}
		Let $X$ be a vector space and $p,q:X \rightarrow [0, \infty)$ seminorms on $X$. Then $p \leq q$ iff $B_q(0,1) \subset B_p(0,1)$.  
	\end{ex}

	\begin{proof}
		Suppose that $p \leq q$. Let $x \in B_q(0,1)$. Then 
		\begin{align*}
			p(x) 
			& \leq q(x) \\
			& < 1
		\end{align*}
		So $x \in B_p(0,1)$. \\
		Conversely, suppose that $B_q(0,1) \subset B_p(0,1)$. Let $x \in X$. If $p(x) = 0$, then $p(x) \leq q(x)$. Suppose that $p(x) > 0$. For the sake of contradiction, suppose that $p(x) > q(x)$. Then 
		\begin{align*}
			q \bigg( \frac{x}{p(x)}\bigg) 
			&= \frac{q(x)}{p(x)} \\
			& < 1
		\end{align*}
		Therefore, $x/p(x) \in B_q(0,1) \subset B_p(0,1)$. By definition,
		\begin{align*}
			\frac{p(x)}{p(x)}
			& =  p\bigg( \frac{x}{p(x)}\bigg)\\
			&< 1
		\end{align*} 
		which is a contradiction. So $p(x) \leq q(x)$. Since $x \in X$ is arbitrary, $p \leq q$.
	\end{proof}

	\begin{ex}
		Let $X$ be a topological vector space and $p:X \rightarrow \Rg$ a continuous seminorm. Then 
		\begin{enumerate}
			\item $B_p(0,1)$ is open 
			\item $\bar{B}_p(0,1)$ is closed
		\end{enumerate}
	\end{ex}

	\begin{proof}\
		\begin{enumerate}
			\item Let $(x_{\al})_{\al \in A}$ be a net in $B_p(0,1)^c$ and $x \in X$. Suppose that $x_{\al} \rightarrow x$. Then $p(x_{\al}) \rightarrow p(x)$. Since for each $\al \in A$, $p(x_{\al}) \geq 1$, $p(x) \geq 1$. Hence $x \in B_p(0,1)^c$. So $B_p(0,1)^c$ is closed which implies that $B_p(0,1)$ is open.
			\item Let $(x_{\al})_{\al \in A}$ be a net in $\bar{B}_p(0,1)$ and $x \in X$. Suppose that $x_{\al} \rightarrow x$. Then $p(x_{\al}) \rightarrow p(x)$. Since for each $\al \in A$, $p(x_{\al}) \leq 1$, $p(x) \leq 1$. Hence $x \in \bar{B}_p(0,1)$. So $\bar{B}_p(0,1)$ is closed.
		\end{enumerate}
	\end{proof}

	\begin{ex}
		Let $X$ be a topological vector space and $p:X \rightarrow \Rg$ a seminorm. Then the following are quivalent:
		\begin{enumerate}
			\item $p$ is continuous
			\item $B_p(0,1)$ is open
			\item $\bar{B}_p(0,1) \in \MN_0$ 
			\item $p$ is continuous at $0$. 
		\end{enumerate}
	\end{ex}

	\begin{proof}\
		\begin{itemize}
			\item $(1) \implies (2)$: \\
			Clear from previous exercise. \\
			\item $(2) \implies (3)$: \\
			Clear since $B_p(0,1) \subset \bar{B}_p(0,1)$. \\
			\item $(3) \implies (4)$: \\
			Let $(x_{\al})_{\al \in A} \subset X$ be a net. Suppose that $x_{\al} \rightarrow 0$. Let $U \subset \R$. Suppose that $U \in \MN_0$. Then there exists $\ep > 0$ such that $\bar{B}(0, \ep) \subset U$. Since the map $f_{\ep}: X \rightarrow X$ defined by $f_{\ep}(x) = \ep x$ is a homeomorphism, $\bar{B}_p(0,\ep) = \ep \bar{B}_p(0,1) \in \MN_0$. Hence there exists $\beta \in A$ such that for each $\al \geq \beta$, $x_{\al} \in \bar{B}_p(0, \ep)$. Let $\al \in A$. Suppose that $\al \geq \beta$. By definition, $p(x_{\al}) \leq \ep$. So $p(x_{\al}) \in \bar{B}(0,\ep) \subset U$. Hence $(p(x_{\al}))_{\al \in A}$ is eventually in $U$. Since $U \in \MN_0$ is arbitrary, $p(x_{\al}) \rightarrow 0$. So $p$ is continuous at $0$. \\
			\item $(4) \implies (1)$: \\
			Let $(x_{\al})_{\al \in A} \subset X$ be a net and $x \in X$. Suppose that $x_{\al} \rightarrow x$. Then $x_{\al} - x \rightarrow 0$. Therefore $p(x_{\al} - x) \rightarrow 0$. The reverse triangle inequality implies that $p(x_{\al}) \rightarrow p(x)$. So $p$ is continuous.
		\end{itemize}
	\end{proof}

	\begin{ex}
		Let $X$ be a topological vector space and $p:X \rightarrow \Rg$ a seminorm. Then $p$ is continuous iff there exists a continuous seminorm $q: X \rightarrow \Rg$ such that $p \leq q$. 
	\end{ex}

	\begin{proof}
		Suppose that $p$ is continuous. Set $q = p$. Then $q$ is a continuous and $p \leq q$\\
		Conversely, suppose that there exists a continuous seminorm $q:X \rightarrow \Rg$ such that $p \leq q$. Then $\bar{B}_q(0,1) \subset \bar{B}_p(0,1)$. The previous exercise tells us that 
		\begin{align*}
			q \text{ is continuous} 
			& \iff \bar{B}_q(0,1) \in \MN_0 \\
			& \implies \bar{B}_p(0,1) \in \MN_0 \\
			& \iff p \text{ is continuous} 
		\end{align*}
	\end{proof}

	\begin{thm}\tbf{Hahn-Banach Theorem for Seminorms}\\
		Let $X$ be a vector space, $p:X \rightarrow \R$ a seminorm, $M \subset X$ a subspace and $f:M \rightarrow \C$ a linear functional. If for each $x \in M$, $\vert f(x) \vert \leq p(x)$, then there exists a linear functional $F:X \rightarrow \C$ such that for each $x \in X$, $\vert F(x) \vert \leq p(x)$ and $F|_{M}=f$.
	\end{thm}	






















	
	\newpage
	\subsection{Minkowski Functionals}

	\begin{defn}
		Let $X$ be a vector space and $A \subset X$. Then $A$ is said to be \tbf{convex} if for each $x, y \in A$, $t \in [0,1]$, $tx +(1-t)y \in A$.
	\end{defn}

	\begin{ex}
		Let $X$ be a vector space and $\MA \subset \MP(X)$, Suppose that for each $A \in \MA$, $A$ is convex. Then $$\bigcap_{A \in \MA} A$$ is convex.
	\end{ex}

	\begin{proof}
		Let $x,y \in \bigcap\limits_{A \in \MA} A$ and $t \in [0,1]$. Then for each $A \in \MA$, $x,y \in A$. Let $A \in \MA$. Since $A$ is convex, $tx + (1-t)y \in A$. Since $A \in \MA$ is arbitrary, $tx + (1-t)y \in \bigcap\limits_{A \in \MA} A$. So $\bigcap\limits_{A \in \MA} A $ is convex.
	\end{proof}

	\begin{defn}
		Let $X$ be a vector space and $A \subset X$. Set 
		$$\MS = \{S \subset X: S \text{ is convex and } A \subset S\}$$ 
		We define the \tbf{convex hull of $A$}, denoted $\cnv A$, by $$\cnv A = \bigcap_{S \in \MS}S$$
	\end{defn}

	\begin{note}
		We may think of $\cnv A$ as the smallest convex set containing $A$. 
	\end{note}

	\begin{defn}
		Let $X$ be a vector space, $A \subset X$ and $x \in X$. Then $x$ is said to be a \tbf{convex combinations of elements of $A$} if there exist $(a_j)_{j=1}^n \subset A$ and $(t_j)_{j=1}^n \subset [0,1]$ such that $x = \sum\limits_{j=1}^n t_j a_j$ and $\sum\limits_{j=1}^n t_j = 1$. We define $C_A \subset X$ by 
		
		$$C_A = \{x \in X:\text{$x$ is a convex combination of elements of $A$}\}$$
	\end{defn}

	\begin{ex}
		Let $X$ be a vector space and $A \subset X$. 
 		Then 
 		\begin{enumerate}
 			\item $A \subset C_A$
 			\item $C_A$ is convex 
 		\end{enumerate}
	\end{ex}

	\begin{proof}\
		\begin{enumerate}
			\item Let $x \in A$, then 
			\begin{align*}
				x 
				&= 1x \\
				& \in C_A 
			\end{align*}
			So $A \subset C_A$.\\
			\item Let $x, y \in C_A$. and $\lam \in [0,1]$. Then there exist $(a_i)_{i=1}^n$, $(b_j)_{j=1}^m \subset A$ and $(s_i)_{i=1}^n$, $(t_j)_{j=1}^m \subset [0,1]$ such that $x = \sum\limits_{i=1}^n s_i a_i$ and $y = \sum\limits_{j=1}^m t_j b_j$. Then
			\begin{align*}
				\lam x + (1-\lam)y 
				&= \lam [\sum\limits_{i=1}^n s_i a_i] + (1-\lam)[\sum\limits_{j=1}^m t_j b_j] \\
				&= \sum\limits_{i=1}^n \lam s_i a_i + \sum\limits_{j=1}^m (1-\lam) t_j b_j
			\end{align*}
			Since 
			\begin{enumerate}
				\item for each $i \in \{1, \ldots, n\}$ and $j \in \{1, \ldots, m\}$, we have that $\lam s_i \in [0,1]$ and $(1-\lam)t_j \in [0,1]$
				\item 
				\begin{align*}
					\sum\limits_{i=1}^n \lam s_i + \sum\limits_{j=1}^m (1-\lam) t_j
					&= \lam \sum\limits_{i=1}^n s_i +  (1-\lam) \sum\limits_{j=1}^m  t_j \\
					&= \lam + (1-\lam) \\
					&= 1
				\end{align*}
			\end{enumerate}
			we have that $\lam x+(1-\lam) y \in C_A$. So $C_A$ is convex.
		\end{enumerate}
	\end{proof}

	\begin{ex}
		Let $X$ be a vector space and $A \subset X$. Let $(a_j)_{j=1}^n \subset A$ and $(t_j)_{j=1}^n \subset [0,1]$. Suppose that $\sum\limits_{j=1}^n t_j = 1$. If $A$ is convex, then $\sum\limits_{j=1}^n t_ja_j \in A$.\\
		\tbf{Hint:} proceed by induction on $n$
	\end{ex}
	

	\begin{proof}
		Suppose that $A$ is convex. If $n = 2$, then by definition, $\sum\limits_{j=1}^n t_ja_j \in A$. \\
		Suppose that the claim is true for $n - 1$. Since $\sum\limits_{j=1}^n t_j = 1$, then there $k \in \{1, \ldots, n\}$ such that $t_k > 0$. Choose Choose $l \in \{1, \ldots, n\}$ such that $l \neq k$. Set $S = \{1, \ldots, n\} \setminus \{t_l\}$. Then $1 - t_l >0$ and  
		\begin{align*}
			x
			&= \sum\limits_{j=1}^n t_j a_j \\
			&= t_l a_l + \sum_{j \in S} t_ja_j \\
			&= t_la_l + (1-t_l) \sum_{j \in S} \frac{t_j}{1 - t_l}a_j
		\end{align*}
		Since 
		\begin{align*}
			\sum_{j \in S} \frac{t_j}{1-t_l} 
			&= \frac{1-t_l}{1-t_l} \\
			&= 1
		\end{align*}
		our induction hypothesis implies that 
		$$\sum\limits_{j \in S} \frac{t_j}{1-t_l} a_j \in A$$ 
		Since $A$ is convex, by definition we have that 
		\begin{align*}
			x 
			&= t_la_l + (1-t_l) \bigg[ \sum_{j \in S} \frac{t_j}{1 - t_l}a_j \bigg] \\
			& \in A
		\end{align*}
	\end{proof}

	\begin{ex}
		Let $X$ be a vector space and $A \subset X$. Then 
		$$\cnv A = C_A$$
	\end{ex}

	\begin{proof}
	Since $A \subset C_A$ and $C_A$ is convex, $\cnv A \subset C_A$. \\
	Conversely, Let $x \in C_A$. Then there exist $(a_j)_{j=1}^n \subset A$ and $(t_j)_{j=1}^n \subset [0,1]$ such that $x = \sum\limits_{j=1}^n t_j a_j$ and $\sum\limits_{j=1}^n t_j = 1$. Since $A \subset \cnv A$ and $\cnv A$ is convex, the previous exercise implies that $x \in \cnv A$. So $C_A \subset \cnv A$. Hence $\cnv A = C_A$.
	\end{proof}

	\begin{ex}
		Let $X$ be a vector space and $A$, $B \subset X$ convex and $\lam \in \C$. Then 
		\begin{enumerate}
			\item $A + B$ is convex
			\item $\lam A$ is convex
		\end{enumerate}
	\end{ex}
	
	\begin{proof}\
		\begin{enumerate}
			\item Let $x,y \in A + B$ and $t \in [0,1]$. Then there exist $a_x, a_y \in A$, $b_x, b_y \in B$ such that $x = a_x + b_x$ and $y = a_y + b_y$. Since $A$ and $B$ are convex, $ta_x + (1-t)a_y \in A$ and $tb_x + (1-t)b_y \in B$. Hence 
			\begin{align*}
				tx + (1-t)y
				&= ta_x + tb_x + (1-t)a_y + (1-t)b_y \\
				&= [ta_x + (1-t)a_y] + [tb_x + (1-t)b_y] \\
				& \in A + B
			\end{align*}
			So $A + B$ is convex.
			\item Let $x, y \in \lam A$ and $t \in [0,1]$. Then there exist $a_x, a_y \in A$ such that $x = \lam a_x$ and $y = \lam a_y$. Since $A$ is convex, $t a_x + (1-t) a_y \in A$. Therefore 
			\begin{align*}
				tx + (1-t)y 
				&= t \lam a_x + (1-t) \lam a_y \\
				&= \lam [t a_x + (1-t) a_y ] \\
				& \in \lam A
			\end{align*}
			So $\lam A$ is convex.
		\end{enumerate}
	\end{proof}

		\begin{defn}
		Let $X$ be a vector space and $A \subset X$. Then $A$ is said to be \tbf{balanced} if for each $x \in A$, $c \in \C$, $|c| \leq 1$ implies that $cx \in A$.
	\end{defn}

		\begin{ex}
		Let $X$ be a vector space and $\MA \subset \MP(X)$, Suppose that for each $A \in \MA$, $A$ is balanced. Then $$\bigcup_{A \in \MA} A$$ is balanced.
	\end{ex}
	
	\begin{proof}
		Let $x \in \bigcap\limits_{A \in \MA} A$ and $r \in \C$. Suppose that $|r| \leq 1$. Then there exists $B \in \MA$ such that $x \in B$. Since $A$ is balanced, 
		\begin{align*}
			rx 
			&\in B \\
			& \subset \bigcap\limits_{A \in \MA} A
		\end{align*}
		So $\bigcap\limits_{A \in \MA} A$ is balanced.
	\end{proof}

	\begin{defn}
		Let $X$ be a vector space and $A \subset X$. We define the \tbf{balanced hull of $A$}, denoted $\bal A$, by $$\bal A = \bigcup_{\substack{r \in \C \\ |r| \leq 1}} rA$$
	\end{defn}

	\begin{ex}
		Let $X$ be a vector space and $A \subset X$. Then $\bal A$ is balanced. 
	\end{ex}

	\begin{proof}
		Let $x \in \bal A$ and $r \in \C$. Suppose that $|r| \leq 1$. By definition, there exists $s \in \C$ and $a \in A$ such that $|s| \leq 1$ and $x = sa$. Then 
		\begin{align*}
			|rs| 
			&= |r||s| \\
			&\leq 1
		\end{align*}
		which implies that
		\begin{align*}
			rx 
			&= rsa \\
			& \in rsA \\
			& \subset \bigcup_{\substack{q \in \C \\ |q| \leq 1}} qA \\
			&= \bal A
		\end{align*}
		So $\bal A$ is balanced.
	\end{proof}
	
	\begin{note}
		We may think of $\bal A$ as the smallest balanced set containing $A$. 
	\end{note}
	
	\begin{ex}
		Let $X$ be a vector space and $A \subset X$. Suppose that $A \neq \varnothing$. If $A$ is balanced, then $0 \in A$.
	\end{ex}
	
	\begin{proof}
		Clear by definition.
	\end{proof}

	\begin{ex}
		Let $X$ be a vector space, $A \subset X$, $x \in X$ and $\lam \in \C$. Suppose that $A$ is balanced. Then $\lam x \in A$ iff $|\lam| x \in A$.  
	\end{ex}
	
	\begin{proof}
		If $\lam = 0$, then the claim is clearly true. Suppose that $\lam \neq 0$. Set $s = \sgn(\lam)$. Suppose that $\lam x \in A$. Since $A$ is balanced and $|s| = |s^{-1}| = 1$, 
		\begin{align*}
			|\lam|x 
			&= s^{-1} \lam x \\
			& \in A
		\end{align*}
		Conversely, suppose that $|\lam| x \in A$. Then 
		\begin{align*}
			\lam x 
			&= s |\lam| x \\
			& \in A
		\end{align*}
	\end{proof}

	\begin{ex}
		Let $X$ be a vector space and $A \subset X$. If $A$ is balanced, then $\cnv A$ is balanced.
	\end{ex}

	\begin{proof}
		Suppose that $A$ is balanced. Let $x \in \cnv A$ and $r \in \C$. Suppose that $|r| \leq 1$. Then there exist $(a_j)_{j=1}^n \subset A$ and $(t_j)_{j=1}^n \subset [0,1]$ such that $x = \sum\limits_{j=1}^n t_j a_j$ and $\sum\limits_{j=1}^n t_j = 1$. Since $A$ is balanced, for each $j \in \{1, \ldots, n\}$, 
		\begin{align*}
			ra_j 
			&\in A \\
			&\subset \cnv A
		\end{align*}
		Since $\cnv A$ is convex, we have that 
		\begin{align*}
			rx 
			&= r\sum\limits_{j=1}^n t_j a_j\\
			&= \sum\limits_{j=1}^n t_j ra_j\\
			&\in \cnv A
		\end{align*}
		Hence $\cnv A$ is balanced..
	\end{proof}



	\begin{defn}
		Let $X$ be a vector space and $A \subset X$. Then $A$ is said to be \tbf{absorbing} if for each $x \in X$, there exists $r > 0$ such that for each $c \in \R$, $|c| \geq r$ implies that $x \in cA$.
	\end{defn}

	\begin{ex}
		Let $X$ be a topological vector space and $A \in \MN_0$. Then $A$ is absorbing. 
	\end{ex}

	\begin{proof}
		Let $x \in A$. For the sake of contradiction, suppose that for each $r > 0$, there exists $c \in \R$ such that $|c| \geq r$ and $c^{-1}x \in A^c$. Then there exists a sequence $(c_{n})_{n \in \N} \subset \R$ such that for each $n \in \N$, $c_n \geq n$ and $c_n^{-1}x \in A^c$. Since $c_n^{-1} \rightarrow 0$, $c_n^{-1}x \rightarrow 0$. Since $A \in \MN_0$, $(c_n^{-1}x)_{n \in \N}$ is eventually in $A$. This is a contradiction. So there exists $r > 0$ such that for each $c \in \R$, $|c| \geq r$ implies that $x \in cA$. Hence $A$ is absorbing.
	\end{proof}

	\begin{ex}
		
	\end{ex}
	
	\begin{proof}
		
	\end{proof}

	\begin{defn}
		Let $X$ be a vector space and $A \subset X$. For $x \in X$, set $$T^A_x = \{t > 0: x \in tA\}$$ We define the \tbf{Minkowski functional}, denoted $p_A: X \rightarrow \RG$, by $$p_A(x) = \inf T^A_x$$ 
	\end{defn}

	\begin{ex}
		Let $X$ be a vector space and $A \subset X$. Suppose that $A$ is convex, absorbing and $0 \in A$. Then 
		\begin{enumerate}
			\item $p_A:X \rightarrow \Rg$ 
			\item $p(0) = 0$
			\item $p_A$ is a sublinear functional on $X$
		\end{enumerate}
	\end{ex}

	\begin{proof}\
		\begin{enumerate}
			\item Since $A$ is absorbing, there exists $r >0$ such that for each $c \in \R$, $|c| \geq r$ implies that $x \in cA$. Therefore $p_A(x) \leq |c|$ and $p_A: X \rightarrow \Rg$.
			\item Since $0 \in A$, 
			\begin{align*}
				p_A(0) 
				&= \inf T^A_0 \\
				&= 0
			\end{align*}
			\item \begin{itemize}
				\item Let $\ep >0$. Choose $t_x \in T^A_x$ and $t_y \in T^A_y$ such that $t_x < p_A(x) + \ep/2$ and $t_y < p_A(y) + \ep/2$. By definition, $t_x^{-1}x$, $t_y^{-1}y \in A$. Set $\theta = t_x(t_x +t_y)^{-1} \in (0, 1)$. Since $A$ is convex, 
				\begin{align*}
					(t_x +t_y)^{-1}(x+y) 
					&= (t_x +t_y)^{-1}x + (t_x +t_y)^{-1}y \\
					&= \theta t_x^{-1}x + (1 - \theta)t_y^{-1}y \\
					& \in A
				\end{align*}
				Therefore, $t_x + t_y \in T^A_{x+y}$ and 
				\begin{align*}
					p_A(x+y) 
					& \leq t_x + t_y \\
					& < p_A(x) + \frac{\ep}{2} + p_A(y) + \frac{\ep}{2} \\
					& = p_A(x) + p_A(y) + \ep.
				\end{align*}
				Since $\ep >0$ is arbitrary, $p_A(x+y) \leq p_A(x) + p_A(y)$.
				\item If $\lam =0$, then 
				\begin{align*}
					p_A(\lam x) 
					&= p_A(0) \\
					&= 0 \\
					&= |\lam| p_A(x)
				\end{align*}
				Suppose that $\lam > 0$. Let $t >0$. Then
				\begin{align*}
					p_A(\lam x) 
					&= \inf \{t > 0: \lam x \in tA\} \\
					&= \inf \{t > 0: x \in \lam^{-1} tA\} \\
					&= \inf \{\lam s > 0: x \in sA\} \\
					&= \lam \inf \{ s > 0: x \in sA\} \\
					&= \lam p_A(x)
				\end{align*}
			\end{itemize}
		\end{enumerate}
		So $p$ is a sublinear functional on $X$. 
	\end{proof}

	\begin{ex}
		Let $X$ be a vector space and $A \subset X$. Suppose that $A$ is convex, absorbing and $0 \in A$. Then $p_A^{-1}[0, 1) \subset A$. 
	\end{ex}

	\begin{proof}
		Let $x \in p_A^{-1}[0, 1)$. Then $p_A(x) < 1$. By definition, there exists $t \in (0,1)$ such that $x \in tA$. Thus $t^{-1}x \in A$. Since $0 \in A$ and $A$ is convex, we have that 
		\begin{align*}
			x
			&= t (t^{-1}x) + (1-t)0 \\
			& \in A
		\end{align*}
	 	Since $x \in p_A^{-1}[0, 1)$ is arbitrary, $p_A^{-1}[0, 1) \subset A$.
	\end{proof}

		\begin{ex}
		Let $X$ be a topological vector space and $A \subset X$. Suppose that $A$ is open, convex, and $0 \in A$. Then $p_A^{-1}[0, 1) = A$.\\
		\tbf{Hint:} for $x \in A$, consider the sequence $(1 + 1/n)x$
	\end{ex}
	
	\begin{proof}
		Since $A$ is open and $0 \in A$, $A \in \MN_0$ which implies that $A$ is absorbing. The previous exercise implies that $p_A^{-1}[0, 1) \subset A$. \\
		Conversely, let $x \in A$. Since $A$ is open, $A \in \MN_x$. Since $1 + 1/n \rightarrow 1$, $(1 + 1/n)x \rightarrow x$. Therefore, there exits $N \in \N$ such that for each $n \in \N$, $n \geq N$ implies that $(1 + 1/n)x \in A$. In particular, $x \in (1 + 1/N)^{-1}A$. Hence $(1 + 1/N)^{-1} \in T_x^A$ and
		\begin{align*}
			p_A(x)
			&= \leq (1 + 1/N)^{-1} \\
			& < 1
		\end{align*}
		So $x \in p_A^{-1}[0, 1)$ and $A \subset B_{p_A}(0,1)$. \\
	\end{proof}

	\begin{ex}
		Let $X$ be a topological vector space, $A \subset X$ and $x_0 \in A^c$. Suppose that $A$ is convex, $A \in \MN_0$ and $A$ is open. Then there exists $F \in X^*$ such that $\Re F(x_0) =1$ and $\Re F|_A < 1$. \\
		\tbf{Hint:} Assume $X$ is real.
		\begin{enumerate}
			\item \tbf{Existence:} Consider a special $f \in (\R x_0)^*$ and use $p_A$ to apply the Hahn-Banach theorem.
			\item \tbf{Continuity:} for $\ep > 0$, consider the neighborhood $U_{\ep} = \ep A \cap - \ep A$
		\end{enumerate} 
	\end{ex}

	\begin{proof} Assume that $X$ is real.
		\begin{enumerate}
			\item Define $f \in (\R x_0)^*$ by $f(tx_0) = t$. Then $f(x_0) = 1$. Since $A \in \MN_0$, $0 \in A$ and a previous exercise implies that $A$ is absorbing. Since $A$ is convex, absorbing and $0 \in A$, $p_A:X \rightarrow [0, \infty)$ is a sublinear functional on $X$. Since $x_0 \in A^c$, the previous exercise implies that $1 \leq p_A(x_0)$. Let $x \in \R x_0$. Then there exists $t \in \R$ such that $x = tx_0$. 
			\begin{itemize}
				\item If $t \geq 0$, then 
				\begin{align*}
					f(x)
					&= t \\
					&\leq tp_A(x_0) \\
					&= p_A(tx_0) \\
					&= p_A(x)
				\end{align*}
				\item If $t<0$, then $-t >0$ and an exercise from the section on sublinear functionals implies that
				\begin{align*}
					f(x)
					&= t \\
					&= < 0 \\
					& \leq p_A(x)
				\end{align*}
			\end{itemize}
				So $f \leq p_A$ on $\R x_0$. The Hahn-Banach theorem implies that there exists $F:X \rightarrow \R$ such that $F$ is linear, $F|_{\R x_0} = f$ and $F \leq p_A$. The previous exercise implies that  $p_A|_A < 1$. Hence $F|_A < 1$. 
		\item Let $V \in \MN_{0_\R}$. Choose $\ep >0$ such that $B(0,\ep) \subset V$. Set $U_{\ep} = \ep A \cap -\ep A$. Then $U_{\ep} \in \MN_0$. Let $u \in U_{\ep}$. Then $\ep^{-1}u, -\ep^{-1}u \in A$. A previous exercise implies that $p_A^{-1}([0,1)) = A$. Hence 
		\begin{align*}
			\ep^{-1}F(u)
			&= F(\ep^{-1}u) \\
			&\leq p_A(\ep^{-1} u) \\
			&< 1
		\end{align*}
		So $F(u) < \ep$. Similarly, $F(-u) < \ep$. So $-\ep < F(u) < \ep$ and
		\begin{align*}
			F(U_{\ep}) 
			&\subset B(0, \ep) \\
			& \subset V 
		\end{align*}
		Since $V \in \MN_{0_{\R}}$ is arbitrary, $F$ is continuous at $0$. Since $F$ is linear and $F$ is continuous at $0$, $F$ is continuous. Hence $F \in X^*$.
		\end{enumerate}
	If $X$ is complex, then the previous part implies that there exists $G:X \rightarrow \R$ such that $G$ is continuous, real-linear, $G(x_0) = 1$ and $G|_A < 1$. A previous exercise implies that there exists a unique $F \in X^*$ such that $\Re F = G$.
	\end{proof}

	\begin{ex} \tbf{Hahn-Banach Separation Theorem 1:} \\
		Let $X$ be a topological vector space and $A$, $B \subset X$. Suppose that $A$, $B$ are nonempty, convex and disjoint. If $A$ is open, then there exists $\phi \in X^*$ and $c \in \R$ such that for each $x \in A$, $y \in B$, $$\Re \phi(x) < c \leq \Re \phi(y)$$
		\tbf{Hint:} Assume $X$ is real.
		\begin{enumerate}
			\item Choose $a_0 \in A$ and $b_0 \in B$ and set $x_0 = b_0 - a_0$ and $C = A - B + x_0$. Then there exists $\phi \in X^*$ such that $\phi(x_0) = 1$ and $\phi|_C < 1$.
			\item For each $a \in A$, $b \in B$, $\phi(a) < \phi(b)$. Set $c = \sup\limits_{a \in A}\phi(a)$. Since $\phi$ is not constant, $\phi$ is open.
		\end{enumerate}
	\end{ex}
	
	\begin{proof}\
		Assume $X$ is real.
		\begin{enumerate}
			\item Since $A, B$ are nonempty, there exist $a_0 \in A$ and $b_0 \in B$. Set $x_0 = b_0 - a_0$. Previous exercises imply that $A - B$ is open and convex. Set $C = A - B + x_0$. Then $C$ is open and convex. 
			Since  
			\begin{align*}
				0 
				&= a_0 - b_0 + x_0 \\
				&\in C
			\end{align*}
			$C \in \MN_0$. For the sake of contradiction, suppose that $x_0 \in C$. Then there exist $a \in A$, $b \in B$ such that $x_0 = a - b + x_0$. This implies that $a = b$. This is a contradiction since $A \cap B = \varnothing$. Hence $x_0 \not \in C$. The previous exercise implies that there exists a $\phi \in X^*$ such that $\phi(x_0) = 1$ and  $\phi|_C < 1$. 
			\item Let $x \in A$ and $y \in B$. Then 
			\begin{align*}
				\phi(a) - \phi(b) + 1
				&= \phi(a) - \phi(b) + \phi(x_0) \\
				&= \phi(a - b + x_0) \\
				& < 1
			\end{align*}
			So $\phi(a) < \phi(b)$. Set $c = \sup\limits_{a \in A}\phi(a)$. Since $A$ is open and $\phi \in X^*$ is open. Thus for each $x \in A$, $y \in B$, 
			$$\phi(x) < c \leq \phi(y)$$
		\end{enumerate}
	If $X$ is complex, then the previous part implies that there exists $f:X \rightarrow \R$ and $c \in \R$ such that $f$ is continuous, real-linear and for each $x \in A$ and $y \in B$, 
	$$f(x) < c \leq f(y)$$ 
	A previous exercise implies that there exists a unique $\phi \in X^*$ such that $\Re \phi = f$.
	\end{proof}
	
		\begin{defn}
		Let $X$ be a vector space and $A \subset X$. Then $A$ is said to be an \tbf{absorbing disk} if $A$ is convex, absorbing and balanced.
	\end{defn}
	
	\begin{ex}
		Let $X$ be a vector space, $p :X \rightarrow \Rg$ a seminorm on $X$ and $r >0$. Then $B_p(0, r)$ is an absorbing disk.
	\end{ex}
	
	\begin{proof}\
		\begin{enumerate}
			\item Let $a, b \in B_p(0, r)$ and $t \in [0,1]$. Then $p(a - x) < r$ and $p(b) < r$. So 
			\begin{align*}
				p([ta + (1 - t)b]) 
				&\leq p(ta + p((1-t)b) \\
				&= tp(a) + (1-t)p(b) \\
				&< tr + (1-t)r \\
				&= r
			\end{align*}
			So $ta + (1 - t)b \in B_p(0, r)$ and $B_p(0, r)$ is convex.
			\item Let $a \in X$. Set $s = (p(a) + 1)/ r$. Then for each $t \geq s$, $tr \geq p(a)+1$ so that 
			\begin{align*}
				a 
				& \in B_p(0, p(a)+ 1) \\
				& \subset B_p(0, tr) \\
				& = tB_p(0, r) 
			\end{align*} 
			So $B_p(0,r)$ is absorbing.
			\item Let $a \in B_p(0, r)$ and $u \in \C$. Uppose that $|u| \leq 1$. Then
			\begin{align*}
				p(ua)
				&= |u|p(a) \\
				&< |u|r \\
				&\leq r
			\end{align*}
			So $ua \in B_p(0, r)$ and $B_p(0, r)$ is balanced. 
		\end{enumerate}
		Since $B_p(0, r)$ is convex, absorbing and balanced, it is an absorbing disk. 
	\end{proof}
	
	
	\begin{ex}
		Let $X$ be a vector space and $A \subset X$. Suppose that $A$ is an absorbing disk. Then $p_A:X \rightarrow \Rg$ is a seminorm on $X$.
	\end{ex}
	
	\begin{proof} Since $A$ is an absorbing disk, $A$ is convex, absorbing and balanced. So $0 \in A$ and the previous exercise tells us that $p$ is a sublinear functional on $X$. Let $x\in X$ and $\lam \in \C$. \\
		If $\lam =0$, then 
				\begin{align*}
					p_A(\lam x) 
					&= p_A(0) \\
					&= 0 \\
					&= |\lam| p_A(x)
				\end{align*}
		Suppose that $\lam \neq 0$. Since $A$ is balanced, for $t >0$, $\lam t^{-1} x \in A$ iff $|\lam| t^{-1} x \in A$. So
				\begin{align*}
					p_A(\lam x) 
					&= \inf \{t > 0: \lam x \in tA\} \\
					&= \inf \{t > 0: x \in |\lam|^{-1} tA\} \\
					&= \inf \{|\lam| s > 0: x \in sA\} \\
					&= |\lam|\inf \{ s > 0: x \in sA\} \\
					&= |\lam | p_A(x)
				\end{align*}
	So $p$ is a seminorm on $X$.
	\end{proof}
	
	\begin{ex}
		Let $X$ be a topological vector space and $A \subset X$. Suppose that $A$ is an absorbing disk and $A$ is open. Then $B_{p_A}(0, 1) = A$.
	\end{ex}

	\begin{proof}
		Clear by previous exercise.
	\end{proof}

	\begin{ex}
		Let $X$ be a topological vector space and $A \subset X$. Suppose that $A$ is an absorbing disk. Then $p_A:X \rightarrow \Rg$ is continuous iff $A$ is open. 
	\end{ex}

	\begin{proof}
		If $A$ is open, then 
		\begin{align*}
			A 
			&= B_{p_A}(0,1) \\
			& \subset \bar{B}_{p_A}(0,1) \\
		\end{align*}
		which implies that $\bar{B}_{p_A}(0,1) \in \MN_0$. An exercise in the previous section implies that $p_A$ is continuous. \\
		Conversely, if $p_A$ is continuous, then an exercise in the previous section implies that $B_{p_A}(0,1)$ is open. 
	\end{proof}
	

	

	






















	
	
	
	
	
	
	
	

	
	\newpage
	\subsection{Locally Convex Spaces}
	
	\begin{defn}
		Let $X$ be a vector space and $p:X \rightarrow [0, \infty)$ a seminorm on $X$. We equip $X / \ker p$ with the topology induced by the norm $\bar{p}:X / \ker p \rightarrow \Rg$. We define the projection $\pi_p: X \rightarrow X / \ker p$ by $\pi_p(x) = \bar{x} = x + \ker p$.
	\end{defn}
	
	\begin{defn}
		Let $X$ be a vector space and $\MP$ a family of seminorms on $X$. Then $\MP$ is said to \tbf{separate points of $X$} if for each $x \in X$, if $x \neq 0$, then there exists $p \in \MP$ such that $p(x) \neq 0$.
	\end{defn}
	
	\begin{defn}
		Let $X$ be a vector space, $\MT$ a topology on $X$ and $\MP$ a family of seminorms. Then $(X, \MT)$ is said to be a \tbf{locally convex space with associated family of seminorms $\MP$} if 
		\begin{itemize}
			\item $\MP$ separates points of $X$
			\item $\MT = \tau_X(\pi_p : p \in \MP)$
		\end{itemize} 
	\end{defn}
	
	\begin{note}
		We will generally suppress the family $\MP$ of seminorms and the induced topology $\MT$.
	\end{note}
	
	\begin{ex}
		Let $X$ be a locally convex space and $(x_{\al})_{\al \in A} \subset X$ a net and $x \in X$. Then $x_{\al} \rightarrow x$ iff for each $p \in \MP$, $p(x_{\al} - x) \rightarrow 0$.
	\end{ex}
	
	\begin{proof}
		Suppose that $x_{\al} \rightarrow x$. Let $p \in \MP$. 
		By assumption,  
		\begin{align*}
			\bar{x}_{\al} 
			&= \pi_p(x_{\al}) \\
			&\rightarrow \pi_p(x) \\
			&= \bar{x}
		\end{align*}   
		So 
		\begin{align*}
			p(x_{\al} - x) 
			&= \bar{p}(\bar{x}_{\al} - \bar{x}) \\
			& \rightarrow 0
		\end{align*}
		Conversely, suppose that for each $p \in \MP$, $p(x_{\al} - x) \rightarrow 0$. Let $p \in \MP$. Then 
		\begin{align*}
			\bar{p}(\bar{x}_{\al} - \bar{x}) 
			&= p(x_{\al} - x) \\
			& \rightarrow 0
		\end{align*} 
		So $\pi_p(x_{\al}) \rightarrow \pi_p(x)$. Since $p \in \MP$ is arbitrary, $x_{\al} \rightarrow x$. 
	\end{proof}
	
	\begin{ex}
		Let $X$ be a locally convex space. Then for each $p \in \MP$, $p$ is continuous. 
	\end{ex}
	
	\begin{proof}
		Let $(x_{\al})_{\al \in A} \subset X$ be a net and $x \in X$. Suppose that $x_{\al} \rightarrow x$. Let $p \in \MP$. Then $p(x_{\al} - x) \rightarrow 0$. The reverse triangle inequality implies that 
		\begin{align*}
			|p(x_{\al}) - p(x)| 
			& \leq p(x_{\al} - x) \\
			& \rightarrow 0
		\end{align*}
		So $p(x_{\al}) \rightarrow p(x)$ and $p$ is continuous. 
	\end{proof}
	
	\begin{ex}
		Let $X$ be a locally convex space. Then X is a Hausdorff topological vector space.
	\end{ex}
	
	\begin{proof}\
		\begin{enumerate}
			\item Let $(x_{\al})_{\al \in A}$, $(x_{\al})_{\al \in A} \subset X$ and $(\lam_{\al})_{\al \in A} \subset \C$ be nets and $x$,$y \in X$, $\lam \in \C$. Suppose that $x_{\al} \rightarrow x$, $y_{\al} \rightarrow y$ and $\lam_{\al} \rightarrow \lam$. Let $P \in \MP$. Then 
			\begin{align*}
				p([x_{\al} + y_{\al}] - [x + y]) 
				&= p([x_{\al} - x] + [y_{\al} - y]) \\
				&\leq p(x_{\al} - x) + p(y_{\al} - y) \\
				& \rightarrow 0
			\end{align*}
			Since $p \in \MP$ is arbitrary, $x_{\al} + y_{\al} \rightarrow x + y$ and addition $X \times X \rightarrow X$ is continuous. \\
			
			\item Similiarly, 
			\begin{align*}
				p(\lam_{\al} x_{\al} - \lam x) 
				&= p([\lam_{\al} x_{\al} - \lam x_{\al}] + [\lam x_{\al} - \lam x]) \\
				& \leq p(\lam_{\al} x_{\al} - \lam x_{\al}) + p(\lam x_{\al} - \lam x) \\
				&= p([\lam_{\al} - \lam] x_{\al}) + p(\lam [x_{\al} - x]) \\
				&= |\lam_{\al} - \lam|p(x_{\al}) + |\lam|p(x_{\al} - x) \\
				&\rightarrow 0 
			\end{align*}
			So scalar multiplication $ \C \times X \rightarrow X$ is  continuous. \\

			\item Let $x, y \in X$. Suppose that $x \neq y$. Since $\MP$ separates points of $X$, there exists $p \in \MP$ such that $p(x - y) \neq 0$. Thus $\bar{p}(\bar{x} - \bar{y}) \neq 0$. Thus $\bar{x} \neq \bar{y}$. Since $X / \ker p$ is Hausdorff, there exists $U' \in \MN_{\bar{x}}$ and $V' \in \MN_{\bar{y}}$ such that $U' \cap V' = \varnothing$. Set $U = \pi_p^{-1}(U')$ and $V =  \pi_p^{-1}(V')$. Then $U \in \MN_x$, $V \in \MN_y$ and  
			\begin{align*}
				U \cap V 
				&= \pi_p^{-1}(U') \cap  \pi_p^{-1}(V') \\
				&=  \pi_p^{-1}(U' \cap V') \\
				&=  \pi_p^{-1}(\varnothing) \\
				&= \varnothing
			\end{align*}
			So $X$ is Hausdorff.
		\end{enumerate}
	\end{proof}

	
	\begin{ex}
		Let $X$ be a locally convex space and $U \in \MN_0$ open. Then there exist $p \in \MP$ and $r >0$ such that $B_p(0,r) \subset U$.
	\end{ex}

	\begin{proof}
		For the sake of contradiction, suppose that for each $p \in \MP$ and $r >0$, $B_p(0,r) \not \subset U$. Then there exists a sequence $(x_n)_{n \in \N} \subset U^c$ such that for each $p \in \MP$ and $n \in \N$, $p(x_n) < 1/n$. So $x_n \rightarrow 0$. Since $U^c$ is closed, $0 \in U^c$ which is a contradiction. Hence there exist $p \in \MP$ and $r >0$ such that $B_p(0,r) \subset U$.
	\end{proof}

	\begin{ex}
		Let $X$ be a locally convex space. Then for each $U \in \MN_0$, if $U$ is open, then there exists $V \subset U$ such that $V$ is an open absorbing disk.
	\end{ex}

	\begin{proof}
		Let $U \in \MN_0$. Suppose that $U$ is open. The previous exercise implies that there exists $p \in \MP$ and $r > 0$ such that $B_p(0,1) \subset U$. A previous exercise tells us that $B_p(0,1)$ is an open absorbing disk.
	\end{proof}

	\begin{ex}
		Let $(X, \MT)$ be a locally convex space with associated  family of seminorms $\MP$ and $M \subset X$ a subspace. Define $\MP_M = \{p|_M: p \in \MP\}$. Then $(M, \MT \cap M)$ is a locally convex space with associated family of seminorms $\MP_M$. 
	\end{ex}

	\begin{proof}
		Let $(x_{\al})_{\al \in A} \subset M$ be a net and $x \in M$. Suppose that $x_{\al} \rightarrow x$ in $\MT \cap M$. Then an exercise in the section on the subspace topology implies that $x_{\al} \rightarrow x$ in $\MT$. Let $q \in \MP_M$. Then there exists $p \in \MP$ such that $q = p|_M$. Therefore
		\begin{align*}
			q(x_{\al} - x) 
			&= p|_M(x_{\al} - x) \\ 
			&= p(x_{\al} - x) \\
			& \rightarrow 0
		\end{align*}  
		Hence $x_{\al} \rightarrow x$ in $\tau_X(\pi_q:q \in \MP_M)$. \\
		Conversely, suppose that $x_{\al} \rightarrow x$ in $\tau_X(\pi_q:q \in \MP_M)$. Let $p \in \MP$. Then 
		\begin{align*}
			p(x_{\al} - x) 
			&= p|_M(x_{\al} - x) \\
			& \rightarrow 0
		\end{align*}
		Hence $x_{\al} \rightarrow x$ in $\MT$. So $x_{\al} \rightarrow x$ in $\MT \cap M$. Therefore $\MT \cap M = \tau_X(\pi_q: q \in \MP_M)$. 
	\end{proof}
	
	\begin{ex}
		Let $X$ be a locally convex space, $M \subset X$ a subspace and $f \in M^*$. Then there exists $F \in X^*$ such that $F|_M = f$.  
	\end{ex}

	\begin{proof}
		Set $p_f = |f|$. Since $p_f$ is a continuous seminorm, $B_{p_f}(0,1)$ is open in $M$. Therefore, there exists $U \subset X$ open such that $B_{p_f}(0,1) = U \cap M$. A previous exercise implies that there exists $p \in \MP$ and $r >0$ such that $B_p(0, r) \subset U$. Set $A = B_p(0, r)$. Since $A$ is open, $p_A:X \rightarrow \Rg$ is continuous and $A = B_{p_A}(0,1)$. Hence 
		\begin{align*}
			B_{p_A|_M}(0,1) 
			&= A \cap M \subset U \cap M \\
			&= B_{p_f}(0,1) 
		\end{align*}  
		Therefore $p_f \leq p_A|_M$ and $|f| \leq p_A$ on $M$. The Hahn-Banach theorem implies that there exists $F:X \rightarrow \C$ such that $F$ is linear, $F|_M = f$ and $|F| \leq p_A$. Since $p_A$ is continuous, $|F|$ is continuous, which implies that $F$ is continuous. So $F \in X^*$.  
	\end{proof}
	
	\begin{ex} \tbf{Hahn-Banach Separation Theorem 2:}\\
		Let $X$ be a locally convex space and $A$, $B \subset X$. Suppose that $A$, $B$ are nonempty, convex and disjoint. If $A$ is compact and $B$ is closed, then there exists $\phi \in X^*$ and $c_1, c_2 \in \R$ such that for each $x \in A$, $y \in B$, $$\Re \phi(x) < c_1 < c_2 \leq \Re \phi(y)$$
		\tbf{Hint:} Assume $X$ is real. Since $X$ is locally convex, there exists $V \subset U$ such that $V$ is an open absorbing disk and $(A + V) \cap B = \varnothing$. Then apply the first Hahn-Banach separation theorem to $A+V$ and $B$.
	\end{ex}
	
	\begin{proof}
		Assume $X$ is real. Suppose that $A$ is compact and $B$ is closed. A previous exercise implies that there exists $U \in \MN_0$ such that $U$ is open and $(A + U) \cap B = \varnothing$. Since $X$ is locally convex, there exists $V \subset U$ such that $V$ is an open  absorbing disk. Then $(A + V)$ is open and convex. By the first Hahn-Banach separation theorem, there exist $\phi \in X^*$ and $c_2 \in \R$ such that for each $x \in A + V$, $y \in B$, $$\phi(x) < c_2 \leq \phi(y)$$
		Specifically, $c_2 = \sup\limits_{x \in A + V} \phi(x)$. Since $\phi \in X^*$ is not constant, $\phi$ is open and thus $\phi(A + V)$ is open. Continuity of $\phi$ implies that $\phi(A)$ is compact. Therefore, $\sup \phi(A) < \sup \phi(A + V)$. So there exists $c_1 \in \phi(A + V)$ such that $\sup \phi(A) < c_1$. Hence there exists $x_1 \in A + V$ such that $\phi(x_1) = c_1$. Then for each $x \in A$ and $y \in B$, 
		\begin{align*}
			\phi(x) 
			&\leq \sup \phi(A) \\
			&< c_1 \\
			&= \phi(x_1) \\
			&< c_2 \\
			& \leq \phi(y)
		\end{align*}
	If $X$ is complex, then the previous part implies that there exists $f:X \rightarrow \R$ and $c_1, c_2 \in \R$ such that $f$ is continuous, real-linear and for each $x \in A$ and $y \in B$, 
	$$f(x) < c_1 < c_2 \leq f(y)$$ 
	A previous exercise implies that there exists a unique $\phi \in X^*$ such that $\Re \phi = f$.
	\end{proof}


	\begin{ex}
		Let $X$ be a locally convex space and $M \subset X$ a closed subspace. If $M \neq X$, then there exists $\phi \in X^*$ such that $\phi \neq 0$ and $\phi|_M = 0$. 
	\end{ex}

	\begin{proof}
		Assume that $X$ is real. Suppose that $M \neq X$. Then there exists $x_0 \in X$ such that $x_0 \not \in M$. Since $\{x_0\}$ is compact and convex, $M$ is closed and convex and $\{x_0\} \cap M = \varnothing$, the second Hahn-Banach separation theorem implies that there exists $\phi \in X^*$ such that for each $x \in M$, $$\phi(x_0) < \phi(x)$$
		Since $0 \in M$, 
		\begin{align*}
			\phi(x_0)
			&< \phi(0) \\
			&= 0
		\end{align*}
	so that $\phi \neq 0$. For the sake of contradiction, suppose that $\phi|_M \neq 0$. Then there exists $x_1 \in M$ such that $\phi(x_1) \neq 0$. Then for each $t \in \R$, 
	\begin{align*}
		\phi(x_0) 
		&< \phi(tx_1) \\
		&= t \phi(x_1)
	\end{align*}
	Set $t = \frac{\phi(x_0)}{\phi(x_1)}$. Then 
	\begin{align*}
		\phi(x_0) 
		&< t \phi(x_1) \\
		&= \phi(x_0)
	\end{align*}
	which is a contradiction. So $\phi|_M = 0$.
	\end{proof}

	\begin{ex}
		Let $X$ be a locally convex space. Then $X^*$ separates the points of $X$. 
	\end{ex}

	\begin{proof}
		Let $x, y \in X$. The second Hahn-Banach separation theorem implies that there exists $\phi \in X^*$ such that $\phi(x) \neq \phi(y)$. 
	\end{proof}









	
	
	
	
	
	
	
	
	
	
	
	
	
	
	
	
	\newpage
	\subsection{Direct Sums}

	























	\newpage
	\subsection{Quotient Spaces}
	
	\begin{ex}
		Let $X$ be a topological vector space and $M \subset X$ a subspace. Then $\pi: X \rightarrow X / M$ is open. 
	\end{ex}

	\begin{proof}
		Define the action $\phi: M \times X \rightarrow X$ by $m \cdot x = x + m$. Then $o_x = x+ M$. Since for each $m \in M$, the map $x \mapsto x +m$ is continuous, an exercise in the section on the quotient topology implies that $\pi: X \rightarrow X/M$ is open.  
	\end{proof}
	
	\begin{ex}
		Let $(X, \MT)$ be a topological vector space and $M \subset X$ a subspace. Then $(X/M, \MT_{X/M})$ is a topological vector space.
	\end{ex}

	\begin{proof}
		Denote addition on $X$ and $X /M$ by $A: X^2 \rightarrow X$ and $\bar{A}:(X /M)^2 \rightarrow X/ M$ respectively. Similarly, denote scalar multiplication on $X$ and $X /M$ by $\Lam: \C \times X \rightarrow X$ and $\bar{\Lam}:\C \times (X /M) \rightarrow X/ M$ respectively. 
		\begin{itemize}
			\item Let $\bar{x}, \bar{y} \in X /M$. Let $U \in \MN_{\bar{x} + \bar{y}}$. Since $\pi: X \rightarrow X / M$ is continuous, we have that $\pi^{-1}(U) \in \MN_{x+y}$. Since addition $A: X^2 \rightarrow X$ is continuous,
			\begin{align*}
				(\pi \circ A)^{-1}(U) 
				& = A^{-1}(\pi^{-1}(U)) \\
				&\in \MN_{(x,y)}
			\end{align*}
			Since $\MB = \{A \times B: A,B \subset X \text{ and $A,B$ are open}\}$ is a basis for the product topology on $X^2$, there exist $V_x \times V_y \in \MB$ such that $(x,y) \in V_x \times V_y \subset (\pi \circ A)^{-1}(U)$. Thus $V_x \in \MN_x, V_y \in \MN_y$ and $V_x \times V_y \in \MN_{(x,y)}$. Recall that $\pi \times \pi: X^2 \rightarrow (X /M)^2$ is defined by $\pi \times \pi (x,y) = (\pi(x), \pi(y))$. For $x,y \in X$, we have that 
			\begin{align*}
				\bar{A}\circ (\pi \times \pi)(x,y)
				& = \bar{A}(\bar{x}, \bar{y}) \\
				& = \bar{x} + \bar{y} \\
				&= \pi(x) + \pi(y) \\
				&= \pi (x + y) \\
				&= \pi \circ A (x,y) 
			\end{align*}
			So $\bar{A}\circ (\pi \times \pi) = \pi \circ A$.  Since $\pi$ is open, an exercise in the section on the product topology implies that $\pi \times \pi$ is open and therefore $\pi \times \pi(V_x \times V_y) \in \MN_{(\bar{x}, \bar{y})}$. Hence
			\begin{align*}
				\bar{A} \circ (\pi \times \pi) (V_x \times V_y) 
				& \subset \bar{A} \circ (\pi \times \pi)((\pi \circ A)^{-1}(U)) \\
				& = \bar{A} \circ (\pi \times \pi)((\bar{A} \circ (\pi \times \pi))^{-1}(U)) \\
				& \subset U
			\end{align*} 
			So for each $U \in \MN_{\bar{x} + \bar{y}}$, there exists $\pi \times \pi(V_x \times V_y) \in \MN_{(\bar{x}, \bar{y})}$ such that $\bar{A}(\pi \times \pi(V_x \times V_y)) \subset U$. Hence $\bar{A}$ is continuous at $(\bar{x}, \bar{y})$. Since $\bar{x}, \bar{y} \in X/ M$ are arbitrary, $\bar{A}$ is continuous. 
			\item Let $\lam \in \C$ and $\bar{x} \in X / M$. Let $U \in \MN_{\lam \bar{x}}$. Since $\pi$ is continuous, $\pi^{-1}(U) \in \MN_{\lam x}$. Since scalar multiplication $\Lam: \C \times X  \rightarrow X$ is continuous, 
			\begin{align*}
				\Lam^{-1}(\pi^{-1}(U)) 
				& = (\pi \circ \Lam)^{-1}( U) \\
				& \in \MN_{(\lam, x)}
			\end{align*} 
			Since $\MB = \{A \times B: \text{$A \subset \C$, $B \subset X$ and $A,B$ are open}\}$ is a basis for the product topology on $\C \times X$, there exist $V_{\lam} \times V_x \in \MB$ such that $(\lam,x) \in V_x \times V_y \subset  (\pi \circ \Lam)^{-1}(U)$. Thus $V_{\lam} \in \MN_{\lam}, V_x \in \MN_x$ and $V_{\lam} \times V_x \in \MN_{(\lam, x)}$. As in the previous part, $\pi \circ \Lam = \bar{\Lam} \circ (\id_{\C} \times \pi)$ and $\id_{\C}$ is open. Hence $\id_{\C} \times \pi$ is open and $\id_{\C} \times \pi (V_{\lam} \times V_x) \in \MN_{(\lam, \bar{x})}$. As in the previous part we have that 
			\begin{align*}
				\bar{\Lam} \circ (\id_{\C} \times \pi) (V_{\lam} \times V_x) 
				& \subset \bar{\Lam} \circ (\id_{\C} \times \pi)((\pi \circ \Lam)^{-1}(U)) \\
				& = \bar{\Lam} \circ (\id_{\C} \times \pi)((\bar{\Lam} \circ (\id_{\C} \times \pi))^{-1}(U)) \\
				& \subset U
			\end{align*} 
			So for each $U \in \MN_{\lam\bar{x}}$, there exists $\id_{\C} \times \pi(V_{\lam} \times V_x) \in \MN_{(\lam, \bar{x})}$ such that $\bar{\Lam}(\id_{\C} \times \pi(V_{\lam} \times V_x)) \subset U$. Hence $\bar{\Lam}$ is continuous at $(\lam, \bar{x})$. Since $\lam \in \C$ and $\bar{x} \in X/ M$ are arbitrary, $\bar{\Lam}$ is continuous. 
		\end{itemize}
	\end{proof}

	\begin{ex}
		Let $X$ be a topological vector space and $M \subset X$ a subspace. If $M$ is closed, then $X / M$ is Hausdorff. 
	\end{ex}
	
	\begin{proof}
		Suppose that $M$ is closed. Define the action $\phi: M \times X \rightarrow X$ by $m \cdot x = m  + x$. Denote by $\sim$, the equivalence relation induced by $\phi$ (i.e. $x \sim y$ iff $x-y \in M$). A previous exercise implies that $\pi: X \rightarrow X /M$ is open. Let $(x_{\al},y_{\al})_{\al \in A} \subset \, \sim$ be a net and $(x,y) \in X \times X$. Suppose that $(x_{\al},y_{\al}) \rightarrow (x,y)$. Then $x_{\al} \rightarrow x$ and $y_{\al} \rightarrow y$. Therefore $x_{\al} - y_{\al} \rightarrow x -y$. Since for each $\al \in A$, $x_{\al} - y_{\al} \in M$ and $M$ is closed, we have that $x -y \in M$. Hence $(x,y) \in \, \sim$ and $\sim$ is closed. Since $\pi$ is open, a previous exercise in the section on separation and countability implies that $X / M$ is Hausdorff.
	\end{proof}
	
	\begin{ex}
		Let $X$ be a topological vector space and $\phi,\psi \in X^*$. If $\ker  \phi \subset \ker \psi$, then there exists $\lam \in \C$ such that $\psi = \lam \phi$.\\
		\tbf{Hint:} This is just a fact about vector spaces. The isomorphism theorems imply that there exists $g: \Im \phi \rightarrow \Im \psi$ such that $\psi = g \circ \phi$. 
	\end{ex}

	\begin{proof}
		Suppose that $\ker  \phi \subset \ker \psi$. If $\phi = 0$, then 
		\begin{align*}
			X 
			& = \ker \phi\\
			& \subset \ker \psi 
		\end{align*}
		So 
		\begin{align*}
			\psi 
			& = 0 \\
			& = \phi
		\end{align*}
		Suppose that $\phi \neq 0$. Then $\Im \phi = \C$. Let $\pi_{\phi}: X \rightarrow X / \ker \phi$ and $\pi_{\psi}: X \rightarrow X / \ker \psi$ be the canonical projection maps and let $\tilde{\phi}: X / \ker \phi \rightarrow \Im \phi$ and $\tilde{\psi}:X / \ker \psi \rightarrow \Im \psi$ be the unique maps such that $\tilde{\phi} \circ \pi_{\phi} = \phi$ and $\tilde{\psi} \circ \pi_{\psi} = \psi$. Note that $\tilde{\phi}$ and $\tilde{\psi}$ are vector space isomorphisms. Define the linear map $\iota: X /\ker \phi \rightarrow X / \ker \psi$ by $\iota(x + \ker \phi) = x + \ker \psi$. Let $x,y \in X$. If $x + \ker \phi = y + \ker \phi$, then 
		\begin{align*}
			x -y 
			& \in \ker \phi \\
			& \subset \ker \psi 
		\end{align*}
		So 
		\begin{align*}
			\iota(x) 
			& = x + \ker \psi \\
			& = y + \ker \psi \\
			& = \iota(y)
		\end{align*}
		and $\iota$ is well defined. Define $g: \Im \phi \rightarrow \Im \psi$ by $g(y) = \tilde{\psi} \circ \iota \circ \tilde{\phi}^{-1}$. Set $\lam = g(1)$. Since $g: \C \rightarrow \C$ is linear, $g = \lam \id_{\C}$. Thus we have the following commutative diagram: 
		\[ \begin{tikzcd}
			& X_{\al} \arrow[dr, "\pi_{\psi}"] \arrow[dl, "\pi_{\phi}"'] &  \\
			X / \ker \phi \arrow[rr, "\iota"] \arrow[d, "\tilde{\psi}"] & & X / \ker \psi  \arrow[d, "\tilde{\psi}"] \\
			\Im \phi \arrow[rr, "g = \lam \id_{\C}"] & & \Im \psi
		\end{tikzcd}
		\]
		Hence 
		\begin{align*}
			\psi
			& = g \circ \phi \\
			& = \lam \id_{\C} \circ \phi \\
			& = \lam \phi
		\end{align*} 
	\end{proof}

	\begin{ex}
		
	\end{ex}
	
	\begin{ex}
		Let $X, Y$ be topological vector spaces.  and $\phi: X \rightarrow Y$ linear. Then $\ker \phi$ is closed iff $\phi$ is continuous. 
	\end{ex}
	
	\begin{proof}
		Suppose that $\phi$ is continuous. Since $\{0\} \subset Y$ is closed, $\ker \phi = \phi^{-1}(\{0\})$ is closed. Conversely, suppose that $\ker \phi$ is closed. 
	\end{proof}
	
	
	
	
	
	
	
	
	
	
	
	
	
	
	
	
	
	
	
	
	
	
	
	\newpage
	\subsection{Duality}
	
	\begin{defn}
		Let $X,Y$ and $Z$ be topological vector spaces (over the same field) and $b : X \times Y \rightarrow Z$. Then $b$ is said to be a \tbf{pairing of $X$ with $Y$ over $Z$} if $b$ is bilinear. \\
	\end{defn}

	\begin{defn}
		Let $X,Y$ and $Z$ be topological vector spaces and $b : X \times Y \rightarrow Z$ a pairing. We define the \tbf{dual pairing} of $b$, denoted $b^*:Y \times X \rightarrow Z$, by $b^*(y,x) = b(x,y)$. Then $b$ is a pairing. 
	\end{defn}

	\begin{ex}
		Let $X,Y$ and $Z$ be topological vector spaces and $b : X \times Y \rightarrow Z$ a pairing. Then $b^*$ is a pairing.
	\end{ex}

	\begin{proof}
		Clear.
	\end{proof}

	\begin{defn}
		Let $X,Y$ and $Z$ be topological vector spaces and $b : X \times Y \rightarrow Z$ a pairing. We define the \tbf{weak topology on $X$ induced by $b$}, denoted $\sig_b(X, Y)$ by 
		$$\sig_b(X,Y) = \tau_{X}(b(\cdot, y): X \rightarrow Z: y \in Y)$$ 
		We define the \tbf{weak topology on $Y$ induced by $b$}, denoted $\sig_b(Y, X)$, by $\sig_b(Y,X) = \sig_{b^*}(Y, X)$.
	\end{defn}

	\begin{ex}
		Let $X,Y$ and $Z$ be topological vector spaces and $b : X \times Y \rightarrow Z$ a pairing. Then 
		\begin{enumerate}
			\item $(X, \sig_b(X, Y))$ is a topological vector space
			\item $(Y, \sig_b(Y, X))$ is a topological vector space
		\end{enumerate}
	\end{ex}

	\begin{proof}\
		\begin{enumerate}
			\item Let $(u_{\al})_{\al \in A}$, $(v_{\al})_{\al \in A} \subset X$ and $(\lam_{\al})_{\al \in A} \subset \C$ be nets and $u,v \in X$ and $\lam \in \C$. Suppose that $u_{\al} \rightarrow u$, $v_{\al} \rightarrow v$ and $\lam_{\al} \rightarrow \lam$. Let $y \in Y$. Since $Z$ is a topological vector space,
			\begin{align*}
				b(u_{\al} + v_{\al}, y) 
				& = b(u_{\al}, y) + b( v_{\al}, y) \\
				& \rightarrow b(u, y) + b(v, y) \\
				& = b(u+v, y)
			\end{align*}
		and 
			\begin{align*}
				b(\lam_{\al} u_{\al} , y) 
				& = \lam_{\al} b(u_{\al}, y)\\
				& \rightarrow \lam b(u, y) \\
				& = b(\lam u, y)
			\end{align*}
			Since $y \in Y$ is arbitrary, $u_{\al} + v_{\al} \rightarrow u+v$ and $\lam_{\al}u_{\al} \rightarrow \lam u$. Hence addition $X \times X \rightarrow X$ and scalar multiplication $\C \times X$ $\rightarrow X$ are continuous.
			\item Since $\sig_b(X,Y) = \sig_{b^*}(Y,X)$, $(1)$ implies $(2)$. 
		\end{enumerate}
	\end{proof}

	\begin{defn}
		Let $X,Y$ and $Z$ be topological vector spaces and $b : X \times Y \rightarrow Z$ a pairing. Then 
		\begin{itemize}
			\item $Y$ is said to \tbf{separate the points of $X$ via $b$} if
			for each $x \in X$, $x \neq 0$ implies that there exists $y \in Y$ such that $b(x,y) \neq 0$
			\item $X$ is said to \tbf{separate the points of $Y$ via $b$} if $X$ separates the points of $Y$ via $b^*$
		\end{itemize}
	\end{defn}

	\begin{ex}
		Let $X,Y$ and $Z$ be topological vector spaces and $b : X \times Y \rightarrow Z$ a pairing. Suppose that $Z$ is Hausdorff.   
		\begin{enumerate}
			\item if $Y$ separates the points of $X$ via $b$, then $(X, \sig_b(X,Y))$ is Hausdorff
			\item 
		\end{enumerate}
	\end{ex}

	\begin{proof} \
		\begin{enumerate}
			\item Suppose that $Y$ separates the points of $X$ via $b$. Let $x_1, x_2 \in X$. Suppose that $x_1 \neq x_2$. Then $x_1 - x_2 \neq 0$. Hence there exists $y \in Y$ such that 
			\begin{align*}
				b(x_1, y) - b(x_2,y)
				& = b(x_1 - x_2,y) \\
				& \neq 0
			\end{align*}
			 Since $Z$ is Hausdorff, there exist $V_1 \in \MN_{b(x_1, y)}, V_2 \in \MN_{b(x_2, y)}$ such that $V_1$ and $V_2$ are open and $V_1 \cap V_2 = \varnothing$. Set $U_1 = b(\cdot, y)^{-1}(V_1)$ and $U_2 = b(\cdot, y)^{-1}(V_2)$. By definition of $\sig_b(X,Y)$, $b(\cdot, y): X \rightarrow Z$ is continuous. Thus $U_1$, $U_2 \in \sig_b(X,Y)$, $x_1 \in U_1$, $x_2 \in U_2$ and 
			 \begin{align*}
			 	U_1 \cap U_2
			 	& = b(\cdot, y)^{-1}(V_1) \cap b(\cdot, y)^{-1}(V_2) \\
			 	& = b(\cdot, y)^{-1}(V_1 \cap V_2) \\
			 	& = b(\cdot, y)^{-1}(\varnothing) \\
			 	& = \varnothing
			 \end{align*}
		 	Therefore $(X, \sig_b(X,Y))$ is Hausdorff. \\
			\item 
		\end{enumerate}
	\end{proof}

	\begin{defn}
		
	\end{defn}
	
	\begin{defn}
		Let $X$ be a topological vector space and $x \in X$. Define $\hat{x}:X^* \rightarrow \C$ by $\hat{x}(f) = f(x)$. We define $\hat{X} = \{\hat{x}: x \in X\}$.
	\end{defn}

	\begin{defn}
		Let $X$ be a topological vector space. We define the \tbf{weak topology on $X$}, denoted $\MT_w$, by $\MT_w = \tau_X(X^*)$ (i.e. the initial topology on $X$ generated by $X^*$).
	\end{defn}

	\begin{defn}
		Let $X$ be a topological vector space, $(x_{\al})_{\al \in A} \subset X$ and $x \in X$. Then $(x_{\al})_{\al \in A}$ is said to \tbf{converge weakly to $x$}, denoted $x_{\al} \conv{w} x$ if $(x_{\al})_{\al \in A}$ converges to $x$ in the weak topology.
	\end{defn}
	
	\begin{ex} \lex{}
		Let $X$ be a topological vector, $(x_{\al})_{\al \in A} \subset X$ a net and $x \in X$. Then $x_{\al} \conv{w} x$ iff for each $\lam \in X^*$, $ \lam (x_{\al}) \rightarrow \lam(x)$. 
	\end{ex}
	
	\begin{proof}
		Immediate by \rex{33011}.
	\end{proof}
	
	\begin{defn}
		Let $X$ be a topological vector space. We define the \tbf{weak-* topology on $X^*$}, denoted $\MT_{w*}$, by $\MT_{w*} = \tau_X(\hat{X})$ (i.e. the initial topology on $X^*$ generated by $\hat{X}$). 
	\end{defn}
	
	\begin{defn}
		Let $X$ be a topological vector space, $(\lam_{\al})_{\al \in A} \subset X^*$ and $\lam \in X^*$. Then $(\lam_{\al})_{\al \in A}$ is said to \tbf{converge in weak-* to $\lam$}, denoted $\lam_{\al} \conv{w^*} \lam$ if $(\lam_{\al})_{\al \in A}$ converges to $\lam$ in the weak-* topology.
	\end{defn}
	
	\begin{ex} \lex{}
		Let $X$ be a topological vector, $(\lam_{\al})_{\al \in A} \subset X^*$ a net and $\lam \in X^*$. Then $\lam_{\al} \conv{w^*} \lam$ iff for each $x \in X$, $ \lam_{\al} (x) \rightarrow \lam(x)$. 
	\end{ex}
	
	\begin{proof}
		Immediate by \rex{33011}.
	\end{proof}
	
	\begin{ex}
		Let $X$ be a topological vector space. 
		\begin{enumerate}
			\item If $X^*$ separates the points of $X$, then  $(X, \MT_w)$ is a locally convex space 
			\item $(X^*, \MT_{w^*})$ is a locally convex space 
		\end{enumerate}
	\end{ex}
	
	\begin{proof}\
		\begin{enumerate}
			\item Suppose that $X^*$ separates the points of $X$. For $\lam \in X^*$, define $p_{\lam}:X \rightarrow \Rg$ by $p_{\lam} = |\lam|$. Set $\MP_{w} = \{p_{\lam}: \lam \in X^*\}$. Then $\MP_{w}$ separates the points of $X$. Let $(x_{\al})_{\al \in A} \subset X$ be a net and $x \in X$. Suppose that $x_{\al} \conv{w} x$. Let $\lam \in X^*$. Then 
			\begin{align*}
				p_{\lam}(x_{\al} - x) 
				&= |\lam(x_{\al} - x)| \\
				&= |\lam(x_{\al}) - \lam(x)| \\
				& \rightarrow 0
			\end{align*}
			So $x_{\al} \rightarrow x$ in $\tau_X(\pi_p: p \in \MP_w)$. \\
			Conversely, suppose that $x_{\al} \rightarrow x$ in $\tau_X(\pi_p: p \in \MP_w)$. Then for each $x \in X$,
			\begin{align*}
				|\lam(x_{\al}) - \lam(x)|
				&= p_{\lam}(x_{\al} - x) \\
				& \rightarrow 0
			\end{align*}
			So that $\lam(x_{\al}) \rightarrow \lam(x)$ and $x_{\al} \conv{w} x$. Hence $\MT_{w} = \tau_{X}(\pi_p: p \in \MP_{w})$ and $(X, \MT_{w})$ is a locally convex space.  \\
			\item For $x \in X$, define $p_x:X^* \rightarrow \Rg$ by $p_x = |\hat{x}|$. Set $\MP_{w^*} = \{p_x:x \in X\}$. Let $\phi \in X^*$. Suppose that $\phi \neq 0$. Then there exists $x \in X$ such that 
			\begin{align*}
				\hat{x}(\phi)
				& = \phi(x)  \\
				& \neq 0
			\end{align*}
			So $\MP_{w^*}$ separates the points of $X^*$. Let $(\lam_{\al})_{\al \in A} \subset X^*$ be a net and $\lam \in X^*$. Suppose that $\lam_{\al} \conv{w^*} \lam$. Let $x \in X$. Then 
			\begin{align*}
				p_x(\lam_{\al} - \lam) 
				&= |\hat{x}(\lam_{\al} - \lam)| \\
				&= |\hat{x}(\lam_{\al}) - \hat{x}(\lam)| \\
				& \rightarrow 0
			\end{align*}
			So $\lam_{\al} \rightarrow \lam$ in $\tau_{X^*}(\pi_p: p \in \MP_{w^*})$. \\
			Conversely, suppose that $\lam_{\al} \rightarrow \lam $ in $\tau_{X^*}(\pi_p: p \in \MP_{w^*})$. Then for each $x \in X$,
			\begin{align*}
				|\hat{x}(\lam_{\al}) - \hat{x}(\lam)|
				&= p_x(\lam_{\al} - \lam) \\
				& \rightarrow 0
			\end{align*}
			So that $\hat{x}(\lam_{\al}) \rightarrow \hat{x}(\lam)$ and $\lam_{\al} \conv{w^*} \lam$. Hence $\MT_{w^*} = \tau_{X^*}(\pi_p: p \in \MP_{w^*})$ and $(X^*, \MT_{w^*})$ is a locally convex space.  
		\end{enumerate}
	\end{proof}

	\begin{note}
		Let $X$ be a topological vector space. When we equip $X^*$ with the weak-$*$ topology, we write $X^{**}$ in place of $(X^*)^*$.
	\end{note}
	
	
	\begin{ex} \lex{}
		Let $X$ be a topological vector space. Then $X^{**} = \hat{X}$. \\
		\tbf{Hint:} Hahn-Banach theorem
	\end{ex}
	
	\begin{proof}
		Let $f \in X^{**}$. Define $p_{f} = |f|$. Then $p_f$ is a continuous seminorm on $X^*$. Therefore $B_{p_f}(0,1)$ is open. A previous exercise implies that there exists $p \in \MP_{w^*}$ and $r >0$ such that 
		\begin{align*}
			B_{r^{-1}p}(0,1)
			& = B_{p}(0,r) \\
			& \subset B_{p_f}(0,1)
		\end{align*}
		A previous exercise implies that $p_f \leq r^{-1}p$. By definition of $\MP_{w^*}$, there exists $x \in X$ such that $p = |\hat{x}|$. Thus
		\begin{align*}
			p_f
			& = |f| \\
			& \leq r^{-1}p \\
			& = |r^{-1}\hat{x}| \\
		\end{align*} 
	Therefore $\ker \hat{x} \subset \ker f$. An exercise in the section on quotient spaces of locally convex spaces implies that there exists $\lam \in \C$ such that 
	\begin{align*}
		f 
		& = \lam r^{-1}\hat{x} \\
		& \in \hat{X}
	\end{align*}
	So $X^{**} = \hat{X}$.
	\end{proof}




























\newpage
\subsection{Continous Linear Maps}

\begin{defn}
	Let $X,Y$ be topological vector spaces. We define $L(X,Y) = \{T:X \rightarrow Y: T \text{ is linear and continuous}\}$.
\end{defn}

\begin{defn}
	Let $X, Y$ be locally convex spaces with respective associated families of seminorms $\MP$ and $\MQ$ and $p \in \MP$, $q \in \MQ$. We define $\|\cdot\|_{p,q}: L(X, Y) \rightarrow [0, \infty)$ by 
	$$\|T\|_{p,q} = \inf \{C \geq 0: \text{ for each $x \in X$, $q(Tx) \leq Cp(x)$} \}$$
\end{defn}

\begin{ex}
	Let $X, Y$ be locally convex spaces with respective associated families of seminorms $\MP$ and $\MQ$, $p \in \MP$, $q \in \MQ$ and $T \in L(X,Y)$. Then for each $x \in X$, $q(Tx) \leq \|T\|_{p,q}p(x)$. 
\end{ex}

\begin{proof}
	Set $A = \{C \geq 0: \text{ for each $x \in X$, $q(Tx) \leq Cp(x)$} \}$. Let $C \in A$ and $x \in X$. Let $\ep >0$. Then $\ep / [1 + p(x)] > 0$. Hence there exists $C \in A$ such that $$C < \|T\|_{p,q} + \frac{\ep}{1 + p(x)}$$
	Therefore, 
	\begin{align*}
		q(Tx) 
		& \leq Cp(x) \\
		& \leq \bigg[\|T\|_{p,q} + \frac{\ep}{1 + p(x)}\bigg]p(x) \\
		& < \|T\|_{p,q}p(x) + \ep
	\end{align*}
	Since $\ep >0$ is arbitrary, $q(Tx) \leq \|T\|_{p,q}p(x) $. Since $x \in X$ is arbitrary, $\|T\|_{p,q} \in A$. 
\end{proof}

	\begin{ex}
		Let $X, Y$ be locally convex spaces with respective associated families of seminorms $\MP$ and $\MQ$, $p \in \MP$, $q \in \MQ$ and $T \in L(X,Y)$. Then 
		$$\|T\|_{p,q} = \sup \{q(Tx): p(x) = 1\}$$
	\end{ex}

	\begin{proof}
		Let 
	\end{proof}

\begin{ex}
	Let $X, Y$ be locally convex spaces with respective associated families of seminorms $\MP$ and $\MQ$ and $p \in \MP$, $q \in \MQ$. Then $\|\cdot\|_{p,q}$ is a seminorm  on $L(X,Y)$. 
\end{ex}

\begin{proof}
	Let $S, T \in L(X,Y)$ and $\lam \in \C$. 
	\begin{enumerate}
		\item Let $x \in X$. Then 
		\begin{align*}
			q((S+T)(x)) 
			& = q(Sx + Tx) \\
			& \leq q(Sx) + q(Tx) \\
			& \leq \|S\|_{p,q}p(x) + \|T\|_{p,q}p(x) \\
			& =  (\|S\|_{p,q} + \|T\|_{p,q}) p(x)
		\end{align*}
		Since $x \in X$ is arbitrary, $\|S+T\|_{p,q} \leq $
		$\|S + T\|_{p,q} $
		\item Let $x \in X$. Then 
		\begin{align*}
			q((\lam T)(x)) 
			& = q(\lam Tx) \\
			& = |\lam| q(Tx) \\
			& \leq |\lam |\|T\|_{p,q}p(x) \\
		\end{align*}
		Since $x \in X$ is arbitrary, $\|\lam T\|_{p,q} \leq $
	\end{enumerate}
\end{proof}

 

	
	
	
	
	
	
	
	




	
	
	
	
	
	
	
	
	
	
	
	
	
	
	\newpage
	\section{Banach Spaces}
	\subsection{Introduction}
	\begin{note}
		In the following, we will consider vector spaces over $\C$. There are analogous results for real vector spaces as well, just replace every $\C$ with $\R$.
	\end{note}
	
	\begin{defn} \ld{}
		Let $X$ be a normed vector space. Then $X$ is said to be a \tbf{Banach space} if $X$ is complete.  
	\end{defn}
	
	\begin{defn} \ld{}
		Let $X$ be a normed vector space and $(x_i)_{i=1}^n \subset X$. The series $\sum_{i =1}^{\infty}x_i$ is said to \tbf{converge} if the sequence $s_n := \sum_{i=1}^n x_i$ converges. The series $\sum_{i =1}^{\infty}x_i$ is said to \tbf{converge absolutely} if $\sum_{i\in \N}\|x_i \|< \infty$.
	\end{defn}
	
	\begin{ex} \lex{}
		Let $X$ be a normed vector space. Then $X$ is complete iff for each $\seq{x}{i} \subset X$, $\sum_{i =1}^{\infty}x_i$ converges absolutely implies that $\sum_{i=1}^{\infty}x_i$ converges. \\
		\tbf{Hint:} Given a Cauchy sequence $(x_n)_{n \in \N}$, obtain a subsequence $(x_{n_j})_{j \in \N} \subset (x_n)_{n \in \N}$ such that for each $j \in \N$, $\|x_{n_{j+1}} - x_{n_{j}}\| < 2^{-j}$. Define a new sequence $(y_j)_{j \in \N} \subset X$ by 
		\[
		y_j = 
		\begin{cases}
		x_{n_1} & j =1 \\  
		x_{n_j} - x_{n_{j-1}} & j \geq 2	
		\end{cases}
		\] 
	\end{ex}
	
	\begin{proof}
		Suppose that $X$ is complete. Let $\seq{x}{i} \subset X$. Suppose that $\sum_{i=1}^{\infty}x_i$ converges absolutely. Let $\ep >0$. Choose $N \in \N$ such that for each $m,n \in \N$, if $m, n \geq N$ and $m< n$, then $\sum_{m+1}^n \|x_i \|< \ep$. Let $m, n \in \N$. Suppose that $m<n$. Then 
		\begin{align*}
			\|s_n-s_m \|
			&= \bigg \|\sum_{i=1}^n x_i -\sum_{i=1}^m x_i\bigg \|\\
			&= \bigg\|\sum_{i=m+1}^{n} x_i \bigg \| \\
			& \leq \sum_{i=m+1}^n \|x_i \|\\
			& < \ep
		\end{align*}
		
		Thus $(s_n)_{n \in N}$ is Cauchy. Since $X$ is complete, $\sum_{i=1}^{\infty}x_i$ converges. \\
		Conversely, Suppose that for each $\seq{x}{i} \subset X$, $\sum_{i =1}^{\infty}x_i$ converges absolutely implies that $\sum_{i=1}^{\infty}x_i$ converges. Let $\seq{x}{i} \subset X$ be Cauchy. Proceed inductively to create a strictly increasing sequence $(n_i)_{i \in \N} \subset \N$ such that for each $m, n \in \N$, if $m,n \geq n_i$, then $ \|x_m-x_n \|< 2^{-i}$. Define $(y_i)_{i \in \N} \subset X$ by 
		\[ y_i = \begin{cases}
			x_{n_1} & i=1 \\
			x_{n_i} - x_{n_{i-1}} & i \geq 2\\
		\end{cases}\]
		
		Then $\sum_{i=1}^k y_i = x_{n_k}$ and 
		\begin{align*}
			\sum_{i \in \N} \|y_i \|
			&= \|x_{n_1} \|+ \sum_{i \in \N} \|x_{n_i}-x_{n_{i-1}} \|\\
			& \leq \|x_{n_1} \|+ 2\sum_{i \in \N}2^{-i}\\
			& = \|x_{n_1} \|+2
		\end{align*}
		Hence $(x_{n_k})_{k \in \N} = (\sum_{i=1}^k y_i)_{i\in \N}$ converges. Since $(x_i)_{i \in \N}$ is cauchy and has a convergent subsequence, it converges. So $X$ is complete.
	\end{proof}
	
	\begin{ex} \lex{}
		Let $X$ be a normed vector space. Then addition $X \times X \rightarrow X$ and scalar multiplication $\C \times X \rightarrow X$ are continuous and $\|\cdot \|:X \rightarrow \Rg$ is continuous.
	\end{ex}
	
	\begin{proof}
		Let $\ep > 0$. Choose $\del = \frac{\ep}{2}$. Let $(x_1,y_1), (x_2,y_2) \in X \times X$. Suppose that 
		$$\max\{\|x_1-x_2 \|, \|y_1 - y_2 \|\} < \del$$
		Then 
		\begin{align*}
			\|(x_1 + y_1) - (x_2+y_2) \|
			&= \|(x_1-x_2) + (y_1-y_2) \|\\
			& \leq \| x_1-x_2 \|+ \|y_1-y_2 \|\\
			& < 2\del \\
			&= \ep
		\end{align*} 
		Hence addition is uniformly continuous. \vspace{1cm}\\ Let $(\lam_1,x_1) \in \C \times X$ and $\ep >0$. Choose $\del = \min\{\frac{\ep}{2(\vert \lam_1 \vert + \|x_1 \|+ 1)}, \frac{\sqrt{\ep}}{\sqrt{2}}\}$. Let $(\lam_2, x_2) \in \C \times X$. Suppose that $$ \max\{\vert \lam_1-\lam_2 \vert , \|x_1 - x_2 \|\} < \del$$ 
		Then 
		\begin{align*}
			\|\lam_1x_1 - \lam_2x_2 \|
			&= \|\lam_1x_1 - \lam_1x_2 + \lam_1x_2- \lam_2x_2 \|\\
			&= \|\lam_1(x_1-x_2) + (\lam_1-\lam_2)x_2 \|\\
			& \leq \vert \lam_1 \vert \| x_1-x_2 \|+ \vert \lam_1-\lam_2 \vert \|x_2\|\\
			& \leq \vert \lam_1 \vert  \| x_1-x_2 \|+ \vert \lam_1-\lam_2 \vert (\|x_1 -x_2\|+ \|x_1\|)\\
			& < \vert \lam_1 \vert \del  +  \del( \del + \|x_1 \|)\\
			&= (\vert \lam_1 \vert + \|x_1 \|) \del + \del^2 \\
			&< \frac{\ep}{2}+ \frac{\ep}{2}\\
			&= \ep
		\end{align*}
		Since $(\lam_1, x_1) \in \C \times X$ is arbitrary, scalar multiplication is continuous. \vspace{1cm} \\ Let $\ep > 0$. Choose $\del = \ep$. Let $x,y \in X$. Suppose that $\|x-y \|< \del$. Then 
		\begin{align*}
			\big \vert \|x \|- \|y \|\big  \vert
			& \leq \|x - y \|\\
			&< \del\\
			&=\ep
		\end{align*}  
		So $\|\cdot \|: X \rightarrow \Rg$ is uniformly continuous.
	\end{proof}
	
	
	
	
	
	
	
	
	
	
	
	
	
	
	
	\newpage
	\subsection{Bounded Operators}
	
	\begin{defn} \ld{42001} 
		Let $X,Y$ be a normed vector spaces and $T:X \rightarrow Y$ linear. Then $T$ is said to be \tbf{bounded} if $T(\cl B(0,1))$ is bounded. We define $$L(X,Y) = \{T:X \rightarrow Y: T \text{ is linear and bounded}\}$$
		When $X=Y$, we write $L(X)$.
	\end{defn}
	
	\begin{ex} \ld{42001.1} 
		Let $X,Y$ be a normed vector spacesand $T:X \rightarrow Y$ linear. Then $T$ is bounded iff there exists $C \geq 0$ such that for each $x \in X$, $$\|Tx \|\leq C \|x \|$$ 
	\end{ex}
	
	\begin{proof}
	Suppose that $T$ is bounded. If $T = 0$, choose $C = 0$. Suppose that $T \neq 0$. Set $ A = \{\|Tx\|: \|x\| =1\}$. Since $T \neq 0$, there exists $x_0 \in X$ such that $\|x_0\| = 1$ so that $A \neq \varnothing$.  Boundedness of $T$ implies that $A$ is bounded. Set $C = \sup A$. Let $x \in X$. If $x = 0$, then $Tx = 0$ and $\|Tx\| \leq C \|x\|$. Suppose that $x \neq 0$. Then $Tx = \|x\| T(\|x\|^{-1} x)$. Since $\|\|x\|^{-1} x\| = 1$, we have that
	\begin{align*}
	\|Tx\|
	&= \|T(\|x\|^{-1} x)\| \|x\|  \\
	& \leq C\|x\| 
\end{align*}	
Conversely, suppose that there exists $C \geq 0$ such that for each $x \in X$, $\|Tx \|\leq C \|x \|$. Let $x \in \cl B(0,1)$. Then 
	\begin{align*}
	\|Tx\| 
	&\leq C \|x\| \\
	&\leq C
	\end{align*}
So that $T(\cl B(0,1))$ is bounded. 
	\end{proof}
	
	\begin{ex} \lex{42002}
	Set $X = C^{1}([0,1])$ and $Y = C([0,1])$. Equip both $X$ and $Y$ with the sup norm. Define $T:X \rightarrow Y$ by $Tf = f'$. Then $T$ is not bounded.
	\end{ex}
	
	\begin{proof}
	For the sake of contradiction, suppose that $T$ is bounded. Then there exists $C \geq 0$ such that for each $f \in X$, $\|Tf\| \leq C \|f\|$. Choose $n \in \N$ such that $n > C$. Define $f \in X$ by $f(x) = x^n$. Then
	\begin{align*}
	n
	&= \|Tf\| \\
	&\leq C \|f\| \\
	&= C
\end{align*}		
	which is a contradiction. Hence $T$ is not bounded.
	\end{proof}
	
	\begin{ex} \lex{42003}
		Let $X,Y$ be a normed vector spaces and $T:X \rightarrow Y$ a linear map. Then $T$ is bounded iff there exists $r,s>0$ such that $T(B(0,r)) \subset B(0,s)$
	\end{ex}
	
	\begin{proof}
		Suppose that $T$ is bounded. Then there exists $C \geq 0$ such that for each $x \in X$, $\|Tx \|\leq C \|x \|$. Thus $T(B(0,1)) \subset B(0,C+1)$. Conversely. Suppose that there exists $r,s >0$ such that $T(B(0,r)) \subset B(0,s)$. Define $C = \frac{2s}{r}$. Let $x \in X$. Put $\al = \frac{r}{2\|x \|}$ Then $\al x \in B(0,r)$. So $T(\al x ) = \al T(x) \in B(0,s)$. Hence 
		\begin{align*}
			\|T(\al x) \|
			&= \|\al T(x) \|\\
			&= \vert \al \vert \|T(x) \|\\
			& = \frac{r}{2 \|x \|}  \|T(x) \|\\
			& < s.
		\end{align*}
		Thus $$\|Tx \|< \frac{2 s}{r} \|x \|= C \|x \|$$ So $T$ is bounded. 
	\end{proof}
	
	\begin{ex} \lex{42003.1}
	Let $X, Y$ be normed vector spaces and $T:X \rightarrow Y$. Suppose that $T$ is linear. Then there exists $x_0 \in X$ such that $T$ is continuous at $x_0$ iff $T$ is continuous at 0.
	\end{ex}
	
	\begin{proof}
	Suppose that there exists $x_0 \in X$ such that $T$ is continuous at $x_0$. Since $T$ is linear, $T(0) = 0$. Let $(x_n)_{n \in \N} \subset X$. Suppose that $x_n \rightarrow 0$. Then $x_n + x_0 \rightarrow x_0$. Hence 
	\begin{align*}
	T(x_n) + T(x_0)
	&= T(x_n + x_0) \\
	& \rightarrow T(x_0)
	\end{align*}	  
	This implies that 
	\begin{align*}
	T(x_n) 
	&\rightarrow 0 \\
	& = T(0)
	\end{align*}	 
	Therefore $T$ is continuous at $0$. \\
	Conversely, if $T$ is continuous at $0$, then trivially, there exists $x_0 \in X$ such that $T$ is continuous at $x_0$.
	\end{proof}
	
	\begin{ex} \lex{42004}
		Let $X,Y$ be normed vector spaces and $T:X \rightarrow Y$ a linear map. Then the following are equivalent:
		\begin{enumerate}
			\item $T$ is continuous
			\item $T$ is continuous at $x=0$
			\item $T$ is bounded
		\end{enumerate}
	\end{ex}
	
	\begin{proof}\
		\begin{itemize}
		\item $(1) \implies (2)$:\\
		Trivial
		\item $(2) \implies (3)$:\\
		Suppose that $T$ is continuous at $x=0$. Then there exists $\del>0$ such that for each $x \in X$, if $\|x \|< \del$, then $\|Tx \|< 1$. Choose $C = \frac{2}{\del}$. If $x=0$, then $\|Tx \|\leq C \|x \|$. Suppose that $\|x \|\neq 0$. Define $y = \frac{\del}{2 \|x \|}x$. Then $\|y \|< \del$. So 
		\begin{align*}
		1 
		&> \|Ty \|\\
		&= \frac{\del}{2 \|x \|} \|Tx \|
		\end{align*}
		Thus 
		\begin{align*}
			\|Tx \|
			&< \frac{2}{\del} \|x \| \\
			&=C \|x \|
		\end{align*}
		
		Hence $T$ is bounded.
		\item $(3) \implies (1)$\\
		Suppose that $T$ is bounded. Then there exists $C \geq 0$ such that for each $x \in X$, $\|Tx \|\leq C\|x \|$. Let $\ep >0$. Choose $\del = \frac{\ep}{C+1}$. Let $x,y \in X$ Suppose that $\|x-y \|< \del$. Then 
		\begin{align*}
			\|Tx-Ty \|
			& = \|T(x-y) \| \\
			& \leq C \|x-y \|\\
			&< (C+1) \del\\ 
			&= \ep
		\end{align*}
		
		So $T$ is continuous.
		\end{itemize}
	\end{proof}
	
	\begin{defn} \ld{42005}
		Let $X,Y$ be normed vector spaces. Define $\|\cdot\|: L(X,Y)\rightarrow \Rg$ by $$\|T\| = \inf \{C \geq 0: \text{for each }x \in X\text{, } \|Tx \|\leq C\|x\|\}$$ We call $\|\cdot \|$ the \tbf{operator norm on $L(X,Y)$}
	\end{defn}
	
	\begin{ex} \lex{42006}
		Let $X,Y$ be normed vector spaces. If $X\neq \{0\}$, then the operater norm on $L(X,Y)$ is given by: 
		\begin{enumerate}
			\item $\|T\| = \sup\limits_{\|x\|=1}\|Tx\|$
			\item $\|T\| = \sup\limits_{x \neq 0}\|x\|^{-1} \|Tx\|$
			\item $\|T\| = \inf \{C \geq 0: \text{for each }x \in X\text{, } \|Tx \|\leq C\|x\|\}$
		\end{enumerate}
	\end{ex}
	
	\begin{proof} Since $X \neq \{0\}$, the supremums in (1) and (2) are well defined. Let $T \in L(X,Y)$. By linearity of $T$, the sets over which the supremums are taken in (1) and (2) are the same. So (1) and (2) are equal.\\
		Now, set $M = \sup\limits_{\|x \|=1} \|Tx \|$ and $m = \inf \{C \geq 0: \text{ for each }x \in X\text{, } \|Tx \|\leq C \|x \|\}$. Let $x \in X$. If $\|x \|=0$, then $\|Tx \|\leq M \|x \|$. Suppose that $\|x \|\neq 0$. Then 
		\begin{align*}
			\|Tx \|
			&= \bigg(\big\|T(x/\|x\|)\big\|\bigg)\|x \|\\
			& \leq M \|x\|
		\end{align*}
		Hence $M \in \{C \geq 0: \text{ for each }x \in X\text{, } \|Tx \|\leq C \|x \|\}$ and $m \leq M$.
		Let $C \in \{C \geq 0: \text{ for each }x \in X\text{, } \|Tx \|\leq C \|x\|\}$. Suppose that $\|x \|=1$. Then $\Vert Tx\Vert \leq C \|x \|= C$. So $M \leq C$. Therefore $M \leq m$. So $M=m$ and the supremum in (1) is the same as the infimum in (3). 
	\end{proof}
	
	\begin{note}
		From here on, unless stated otherwise, we assume $X \neq 0$.
	\end{note}
	
	\begin{ex} \lex{42007}
		Let $X,Y$ be normed vector spaces and $T \in L(X,Y)$. Then for each $x \in X$, $\|Tx \| \leq \|T\|\|x \|$
	\end{ex}
	
	\begin{proof}
		This is just part of the previous exercise. Let $x \in X$. If $x = 0$, then $\|Tx \|\leq \|T \|\|x \|$. Suppose that $x \neq 0$. Then $\|Tx \|= T(x/\|x\|)\|x\|\leq \|T \|\|x \|$
	\end{proof}
	
	\begin{ex} \lex{42008}
		Let $X, Y$ be normed vector spaces. Then the operator norm is a norm on $L(X,Y)$.
	\end{ex}
	
	\begin{proof}
		Let $S,T \in L(X,Y)$ and $\al \in \C$. For each $x \in X$, we have that 
		\begin{align*}
			\|(S+T)x \|
			&= \|Sx+Tx \|\\
			& \leq \|Sx \|+ \|Tx \|\\
			&\leq \|S \|\|x \|+ \|T \|\|x \|\\
			&= \big(\|S \|+ \|T \|\big) \|x \|
		\end{align*}
		
		So $\|S+T \|\leq \|S \|+ \|T \|$.\vspace{1cm}\\
		
		Using the definition of $\|T \|$, we see that 
		\begin{align*}
			\|\al T \|
			&= \sup_{\|x \|=1} \|(\al T)x \|\\
			&= \sup_{\|x \|=1} \vert \al \vert \|Tx \|\\
			&=\vert \al \vert \sup_{\|x \|=1} \|Tx \|\\
			&=\vert \al \vert \|T \|
		\end{align*} 
		So $\|\al S \|= \vert \al \vert \|S \|$. \vspace{1cm}\\ Suppose that $\|T \|= 0$. Let $x \in X$. Then $\|T x\|\leq \|T \|\|x \|= 0$. So $Tx=0$. Since $x \in X$ is arbitrary, we have that $T=0$. 
	\end{proof}
	
	\begin{ex} \lex{42009}
		Let $X,Y,Z$ be normed vector spaces, $T \in L(X,Y)$ and $S \in L(Y,Z)$. Define $ST:X \rightarrow Z$ by $STx = S(Tx)$. Then $ST \in L(X,Z)$ and $\|ST \|\leq \|S \|\|T \|$. 
	\end{ex}
	
	\begin{proof}
		Clearly $ST$ is linear. Let $x \in X$. Then 
		\begin{align*}
			\|ST x \|
			& = \|S(Tx) \|\\
			& \leq \|S \|\|Tx \|\\
			& \leq \|S \|\|T \|\|x \|
		\end{align*}
		
		So $\|ST \|\leq \|S \|\|T \|$.
	\end{proof}
	
	\begin{defn} \ld{42010}
		Let $X,Y$ be a normed vector spaces and $T \in L(X,Y)$. Then $T$ is said to be \tbf{invertible} or an \tbf{isomorphism} if $T$ is a bijection and $T^{-1} \in L(Y,X)$.
	\end{defn}
	
	\begin{defn} \ld{42011}
		Let $X$ be a normed vector space. Define $GL(X) := \{T \in L(X,X): T \text{ is invertible}\}$.
	\end{defn}
	
	\begin{ex} \lex{42013}
		Let $X,Y$ be normed vector spaces. If $Y$ is complete, then so is $L(X,Y)$.
	\end{ex}
	
	\begin{proof}
		Suppose that $Y$ is complete. Let $(T_n)_{n \in \N} \subset L(X,Y)$. Suppose that $(T_n)_{n \in \N}$ is Cauchy. Since for each $m,n \in \N$, $\big\vert \|T_m \|- \|T_n \|\big\vert \leq \|T_m -T_n \|$, we have that $(\|T_n \|)_{n \in \N} \subset \Rg$ is Cauchy. Hence $\lim\limits_{n \rightarrow \infty}\|T_n \|$ exists. \vspace{1cm} \\ Let $x \in X$ and $m,n \in \N$. Then 
		\begin{align*}
			\|T_m x - T_n x \|
			&= \|(T_m-T_n) x \|\\
			&\leq \|T_m-T_n \|\|x \|
		\end{align*}
		So $(T_nx)_{n \in \N} \subset Y$ is Cauchy and hence converges. Define $T:X \rightarrow Y$ by $Tx = \lim\limits_{n \rightarrow \infty} T_nx$. \vspace{1cm}\\
		Since addition and scalar multiplication are continuous, $T$ is linear. Let $x \in X$ and $\ep>0$. Choose $N \in \N$ such that for each $n \in N$, if $n \geq N$, then $\|Tx - T_n x\|< \ep$. Then for each $n \in \N$, if $n \geq N$ we have that 
		\begin{align*}
			\|Tx\|
			&\leq \|Tx-T_nx \|+ \|T_nx \|\\
			&< \ep + \|T_nx \|\\
			&\leq \ep + \|T_n \|\|x \|
		\end{align*}  
		Thus $\|Tx \|\leq \ep +(\lim\limits_{n \rightarrow \infty} \|T_n \|) \|x \|$. Since $\ep >0$ is arbitrary, $\|Tx \|\leq (\lim\limits_{n \rightarrow \infty} \|T_n \|) \|x \|$. Thus $T \in L(X,Y)$ and $\|T \|\leq \limn \|T_n \|$. \vspace{1cm} \\
		Note that since addition, scalar multiplication and $\|\cdot \|$ are continuous, we have that for each $n \in \N$ and $x \in X$, $\|(T_n-T_m)x \|$ converges to $\|(T_n-T)x \|$ because 
		\begin{align*}
			\lim_{m \rightarrow \infty} \|(T_n-T_m)x \|
			&= \lim_{m \rightarrow \infty} \|T_nx-T_mx \|\\
			&= \|T_nx-\lim_{m \rightarrow \infty}T_mx \|\\
			&=\|T_nx-Tx \|\\
			&= \|(T_n-T)x \|
		\end{align*} 
		Let $\ep >0 $. Choose $N \in \N$ such that for each $m, n \in \N$ if $n,m \geq N$, then $\|T_n - T_m \|< \ep$. Then for each $n \in \N$ if $n \geq N$, then for each $x \in X$, $$\|(T_n-T_m)x\|\leq \|(T_n-T_m)\|\|x \|< \ep \|x\|$$ Combining this with the previous fact, we see that for each $n \in N$, if $n \geq N$, then for each $x \in X$, $$\|(T_n -T) x\|\leq \ep \|x \|$$ In particular, for each $n \in \N$, if $n \geq N$, then $$ \|T_n -T \|= \sup\limits_{\|x \|= 1}\|(T_n - T)x \|\leq \ep$$ This implies that $T_n$ converges to $T$ in $L(X,Y)$. 
		Since $$\bigg | \|T_n \|- \|T \|\bigg | \leq \|T_n - T \|$$ it is clear that $\limn \|T_n \|= \|T \|$
	\end{proof}
	
	
	
	
	
	
	
	
	
	
	
	
	
	
	
	
	
	
	
	
	
	
	
	
	\newpage
	\subsection{Direct Sums}
	
	\begin{defn} \ld{}
	Let $X, Y$ be normed vector spaces and $p \in [1, \infty]$. Let $\| \cdot \|'_p: \R^2 \rightarrow [0, \infty)$ denote the usual $l^p$ norm. We define $\| \cdot \|_p : X \oplus Y \rightarrow \Rg$ by $$\|(x, y) \|_p = \|( \|x\|, \| y \|) \|'_ p$$
	\end{defn}
	
	\begin{ex} \lex{}	
	Let $X, Y$  be normed vector spaces. Then 
	\begin{enumerate}
	\item for each $p \in [1, \infty]$, $\|\cdot\|_p: X \oplus Y \rightarrow \Rg$ is a norm on $X \oplus Y$
	\item  $\{\|\cdot \|_p:  p \in [1, \infty]\}$ are equivalent. 
	\end{enumerate}
	\end{ex}
	
	\begin{proof}\
	\begin{enumerate}
	\item Let $p \in [1, \infty]$, $(x_1,y_1), (x_2,y_2) \in X \oplus Y$ and $\lam \in \C$.
	\begin{itemize}
	\item Clearly if $(x_1, y_1) = (0,0)$, then $\|S\|_p = 0$. Conversely, suppose that $\|(x_1, y_1)\|_p = 0$. Then $\|x_1\| = 0$ and $\|y_1\| = 0$. So $x_1 = 0$ and $y_1 = 0$. Therefore $S = 0$. 
	\item 
	\begin{align*}
	\|\lam (x_1, y_1)\|_p
	&= \|(\|\lam x_1\|, \|\lam y_1\|)\|_p' \\
	&= \|(|\lam|\| x_1\|, |\lam|\|y_1 \|)\|_p' \\
	&= \||\lam| (\| x_1\|,\|y_1\| )\|_p' \\
	&= |\lam| \|(\| x_1\|,\|y_1\|)\|_p' \\
	&= |\lam| \| (x_1, y_1)\|_p
	\end{align*}
	\item 
	\begin{align*}
	\|(x_1, y_1) + (x_2, y_2)\|_p
	&= \|(\|x_1 + x_2\|, \|y_1 + y_2\|)\|_p' \\
	&\leq \|(\|x_1\| + \|x_2\|, \|y_1\| + \|y_2\|)\|_p' \\
	&= \|(\|x_1\|, \|y_1\|) + (\|x_2\|, \|y_2\|)\|_p' \\
	&\leq \|(\|x_1\|, \|y_1\|)\|_p' + \|(\|x_2\|, \|y_2\|)\|_p' \\
	&= \|(x_1, y_1)\|_p + \|(x_2, y_2)\|_p \\ 
	\end{align*}
	\end{itemize}
	\item All norms on $\R^2$ are equivalent.
	\end{enumerate}
\end{proof}		

\begin{ex} \lex{}	
	Let $X, Y$ be Banach spaces. Then $X \oplus Y$ equipped with $\|\cdot \|_p:X \oplus Y \rightarrow [0, \infty)$ is a Banach space. 
	\end{ex}
	
	\begin{proof}
	
	\end{proof}
	
	\begin{ex}
	Let $X, Y$ and $Z$ be Banach spaces and $p \in [0, \infty]$. Equip $Y \oplus Z$ with $\|\cdot\|_p$. Let $T \in L(X, Y \oplus Z)$ with $T = (T_Y, T_Z)$. Then $T_Y \in L(X, Y)$ and $T_Z \in L(X, Z)$.
	\end{ex}
	
	\begin{proof}
	Let $x \in X$. Then $\|T_Y(x)\|, \|T_Z(x)\| \leq $
	\\ \tbf{FINISH!!!}
	\end{proof}

	\begin{defn}
	Let $X, Y$ and $Z$ be Banach spaces and $p \in [0, \infty]$. Let $\| \cdot \|'_p: \R^2 \rightarrow [0, \infty)$ denote the usual $l^p$ norm. Equip $Y \oplus Z$ with $\|\cdot\|_p$. Let $T \in L(X, Y \oplus Z)$ with $T = (T_Y, T_Z)$. Define $\|\cdot \|_p: L(X, Y \oplus Z) \rightarrow \Rg$ by $$\|T\|_p = \|(\|T_Y\|, \|T_Z\|)\|_p'$$
	\end{defn}
	
	\begin{ex}
	Let $X, Y$ and $Z$ be Banach spaces and $p \in [0, \infty]$. Then $\|\cdot \|_p: L(X, Y \oplus Z) \rightarrow \Rg$ is a norm on $L(X, Y \oplus Z)$. 
	\end{ex}
	
	\begin{proof}
	Let $\lam \in \C$ and $S, T \in L(X, Y \oplus Z)$ with $S= (S_Y, S_Z)$ and $T = (T_Y, T_Z)$. 
	\begin{itemize}
	\item Clearly if $S = 0$, then $\|S\|_p = 0$. Conversely, suppose that $\|S\|_p = 0$. Then $\|S_Y\| = 0$ and $\|S_Z\| = 0$. So $S_Y = 0$ and $S_Z = 0$. Therefore $S = 0$. 
	\item 
	\begin{align*}
	\|\lam S\|_p
	&= \|(\|\lam S_Y\|, \|\lam S_Z\|)\|_p' \\
	&= \|(|\lam|\| S_Y\|, |\lam|\|S_Z \|)\|_p' \\
	&= \||\lam| (\| S_Y\|,\|S_Z\| )\|_p' \\
	&= |\lam| \|(\| S_Y\|,\|S_Z\|)\|_p' \\
	&= |\lam| \| S\|_p
	\end{align*}
	\item 
	\begin{align*}
	\|S + T\|_p
	&= \|(\|S_Y + T_Y\|, \|S_Z + T_Z\|)\|_p' \\
	&\leq \|(\|S_Y\| + \|T_Y\|, \|S_Z\| + \|T_Z\|)\|_p' \\
	&= \|(\|S_Y\|, \|S_Z\|) + (\|T_Y\|, \|T_Z\|)\|_p' \\
	&\leq \|(\|S_Y\|, \|S_Z\|)\|_p' + \|(\|T_Y\|, \|T_Y\|)\|_p' \\
	&= \|S\|_p + \|T\|_p \\ 
	\end{align*}
	\end{itemize}
	So $\|\cdot \|_p: L(X, Y \oplus Z) \rightarrow \Rg$ is a norm on $L(X, Y \oplus Z)$. 
	\end{proof}
	
	
	\begin{ex}
	Let $X, Y$ and $Z$ be Banach spaces and $p \in [0, \infty]$. Equip $Y \oplus Z$ with $\|\cdot\|_p$. Let $T \in L(X, Y \oplus Z)$ with $T = (T_Y, T_Z)$. Then $\|T\|\leq 2^{1/p}\|T\|_p$.
	\end{ex}
	
	\begin{proof}
	Let $x \in X$. If $p < \infty$, then
	\begin{align*}
	\|T(x)\|_p
	&= \|(T_Y(x), T_Z(x))\|_p \\
	& \|( \|T_Y(x)\|, \|T_Z(x)\|)\|_p' \\
	&=  \bigg(\|T_Y(x) \|^p +  \|T_Z(x) \|^p \bigg)^{1/p} \\
	& \leq \bigg(\|T_Y\|^p\|x\|^p +  \|T_Z\|^p\|x\|^p \bigg)^{1/p} \\
	& \leq \bigg[ (\|T_Y\|^p+ \|T_Z\|^p)\|x\|^p +  (\|T_Y\|^p + \|T_Z\|^p)\|x\|^p \bigg ]^{1/p} \\
	&= \bigg[ 2(\|T_Y\|^p + \|T_Z\|^p)\|x\|^p \bigg ]^{1/p} \\
	&= 2^{1/p}\|T\|_p\|x\| \\
	\end{align*}
	Hence $\|T\| \leq 2^{1/p}\|T\|_p$
	If $p = \infty$, then 
	\begin{align*}
	\|T(x)\|_{\infty} 
	&= \max(\|T_Y(x) \|, \|T_Z(x)\|) \\
	& \leq \max(\|T_Y\|\|x \|, \|T_Z\|\|x\|) \\
	& \leq \max \bigg[ \max (\|T_Y\|, \|T_Z\|)\|x \|, \max(\|T_Y\| ,\|T_Z\|)\|x\| \bigg] \\
	&= \max(\|T_Y\| ,\|T_Z\|) \|x\|\\
	&= \|T\|_{\infty} \|x\|
	\end{align*}
	Hence 
	\begin{align*}
	\|T\| 
	& \leq \|T\|_{\infty} \\
	&= 2^{1/\infty}\|T\|_{\infty}
	\end{align*}
	\end{proof}
	
	\begin{ex}
	Let $X$ and $X_1, \cdots, X_n$ be Banach spaces and $p \in [0, \infty]$. Equip $\bigoplus\limits_{j=1}^n X_j$ with $\|\cdot\|_p$. Let $T \in L(X, \bigoplus\limits_{j=1}^n X_j)$. Then $\|T\|\leq n^{1/p}\|T\|_p$.
	\end{ex}
	
	\begin{proof}
	Similar to the previous exercise.
	\end{proof}
	
	
	
	
	
	
	
	
	
	
	
	
	
	
	
	
	
	
	
	
	
	
	\newpage
	\subsection{Quotient Spaces}	
	
	\begin{defn} \ld{}
		Let $X$ be a normed vector space and $M \subset X$ a closed subspace. Define $\|\cdot\|:X/M \rightarrow \Rg$ by $$\|x+M\| := \inf_{y \in M}\|x+y\|$$
		
		We call $\|\cdot\|$ the \tbf{subspace norm on $X/M$}
	\end{defn}
	
	\begin{ex} \lex{}
		Let $X$ be a normed vector space and $M \subsetneq X$ a proper, closed subspace of $M$. 
		Then 
		\begin{enumerate}
			\item The previously defined subspace norm on $X/M$ is well defined and is a norm. 
			\item For each $\ep > 0$, there exists $x \in X$ such that $\|x\|=1$ and $\|x+M\| \geq 1-\ep$.
			\item The projection map $\pi:X \rightarrow X/M$ defined by $\pi(x) = x+M$ is continuous and $\|\pi\|=1$. 
			\item If $X$ is complete, then $X/M$ is complete. 
		\end{enumerate} 
	\end{ex}
	
	\begin{proof}\
		\begin{enumerate}
			\item  Let $x, y \in X$ and $\al \in \C$. Suppose that $x+M =y+M$. Then there exists $m \in M$ such that $x=y+m$. Since $M$ is a subspace, the map $T:M \rightarrow M$ given by $Tx = x+m$ is a bijection. So $$\inf_{z \in M} \|y+m+z \|= \inf_{z \in M} \|y+z \|$$ which implies that 
			\begin{align*}
				\|x +M \|
				&= \inf_{z \in M} \|x+z \|\\
				&= \inf_{z \in M} \|y+m+z \|\\
				&= \inf_{z \in M} \|y+z \|\\
				&= \|y+M \|
			\end{align*} 
			So $\|\cdot \|: X/M \rightarrow \Rg$ is well defined.\vspace{.5cm}\\
			We observe that for each $z,w \in M$, $$\|x+y+z \|\leq \|x+w \|+ \|y+w+z \|$$
			Taking infimums over $M$ with respect to $z$ in this inequality implies that for each $w \in M$,
			\begin{align*}
				\inf_{z \in M}\|x+y+z \|
				&\leq \inf_{z \in M} \bigg( \|x+w \|+ \|y+w+z \|\bigg) \\
				&= \|x+w \|+\inf_{z \in M}\|y+w+z \|
			\end{align*}
			Again we use the fact that for each $w \in M$, $$\inf_{z \in M}\|y+w+z \|= \inf_{z \in M}\|y+z \|$$
			This implies that for each $w \in M$, $$\inf_{z \in M}\|x+y+z \|\leq \|x+w \|+ \inf_{z \in M}\|y+z \|$$
			
			Therefore, taking infimums over $M$ with respect to $w$ in this inequality yields
			\begin{align*}
				\|x+y+M \|
				&= \inf_{z \in M} \|x+y +z \|\\
				& \leq \inf_{w \in M} \bigg(\|x+w \|+ \inf_{z \in M}\|y+z \|\bigg)\\
				&= \inf_{w \in M} \|x+w \|+ \inf_{z \in M}\|y+z \|\\
				&= \|x+M \|+ \|y+M \|
			\end{align*}
			\vspace{.5cm}\\
			If $\al =0$, then $\al x = 0$. Choosing $z = 0 \in M$ gives $\|\al x+M \|=0 = \vert \al \vert \|x+M \|$. Suppose that $\al \neq 0$. Then the map $T:M \rightarrow M$ given by $Tx = \al ^{-1}x$ is a bijection and thus $\inf\limits_{z \in M} \|x+\al^{-1}z \|= \inf\limits_{z \in M} \|x+z \|$. Hence we have that
			\begin{align*}
				\|\al x+M \|
				&= \inf_{z \in M} \|\al x +z \|\\
				&= \inf_{z \in M} \vert \al \vert \|x +\al^{-1}z \|\\
				&= \vert \al \vert \inf_{z \in M}\|x +\al^{-1}z \|\\
				&= \vert \al \vert \inf_{z \in M}\|x +z \|\\
				&= \vert \al \vert \|x+M \|
			\end{align*} 
			
			Suppose that $\|x \|=0$. Choose a sequence $(z_n)_{n \in N} \subset M$ such that 
			\begin{align*}
				\lim\limits_{n \rightarrow \infty} \|x - z_n \|
				& = \inf_{z \in M} \|x+ z \|\\
				& = 0
			\end{align*} 
			
			Then $\limn z_n =x$. Since $M$ is closed, $x \in M$. Hence $x+M=0+M$. \vspace{1cm}\\
			\item Since $M$ is a proper subspace, there exists $v \in X$ such that $v \not \in M$. Then $\|v +M \|\neq 0$. Let $\ep >0$. Then $(1-\ep)^{-1}\|v+M \|> \|v+M \|$. So there exists $z \in M$ such that $$0< \|v+M\|\leq \|v+z \|< (1-\ep)^{-1} \|v+M \|$$ Choose $x = \|v+z \|^{-1}(v+z)$. Then $\|x \|=1$ and 
			\begin{align*}
				\|x+M \|
				&= \|v+z \|^{-1} \|v+z +M \|\\
				&= \|v+z \|^{-1} \|v +M \|\\
				&> 1-\ep
			\end{align*}\vspace{.5cm}\\
			\item Let $x \in X$. Taking $z=0$, we we see that $\|\pi(x) \|=\|x+M \|\leq \|x+z \|= \|x \|$. So $\pi$ is bounded and in particular, $$\sup_{\|x \|=1} \|\pi(x) \|\leq 1$$ 
			From (2) we see that $$\sup_{\|x \|=1} \|\pi(x) \|\geq 1$$
			Hence $\|\pi\|= 1$. \vspace{.5cm}\\
			\item Suppose that $X$ is complete. Let $(x_i+M)_{i\in \N} \subset X/M$. Suppose that $\sum\limits_{i\in \N} \|x_i+M \|< \infty$. Let $\ep>0$. Then for each $i \in \N$, there exists $z_i \in M$ such that $\|x_i +z_i \|< \|x_i +M \|+ \ep2^{-i}$. Define the sequence $(a_i)_{i\in \N} \subset X$ by $a_i = x_i +z_i$. Then we have 
			\begin{align*}
				\sum_{i\in \N} \|a_i \|
				&= \sum_{i \in N} \|x_i + z_i \|\\
				&\leq \sum_{i \in N} \bigg (\|x_i +M \|+ \ep2^{-i} \bigg)\\
				&= \sum_{i\in \N} \|x_i+M\|+ \ep
			\end{align*}
			Since $\ep>0$ is arbitrary, it follows that $$\sum_{i\in \N} \|a_i \|\leq \sum_{i\in \N} \|x_i+M\|< \infty$$
			Since $X$ is complete, $\sum\limits_{i=1}^{\infty}a_i$ converges in $X$. Define $(s_n)_{n \in \N} \subset X$ and $s \in X$ by $s_n = \sum\limits_{i =1}^n a_i$ and $s = \sum\limits_{i=1}^\infty a_i $. Since $\limn s_n = s$, and $\pi: X \rightarrow X/M$ is continuous, it follows that $\limn \pi(s_n) = \pi(s)$. Since 
			\begin{align*}
				\pi(s_n) 
				&= \sum_{i=1}^n a_i +M\\
				&= \sum_{i=1}^n x_i +M
			\end{align*} 
			We have that $\sum\limits_{i=1}^{\infty}x_i +M$ converges which implies that $X/M$ is complete.
		\end{enumerate}
	\end{proof}
	
	\begin{ex} \lex{}
		Let $X,Y$ be normed vector spaces and $T \in L(X,Y)$. Then
		\begin{enumerate}
			\item $\ker T$ is closed
			\item there exists a unique map $S :X/ \ker T \rightarrow T(X)$ such that $T = S \circ \pi$. Furthermore $S$ is a bounded linear bijection and $\|S \|= \|T \|$.
		\end{enumerate}
	\end{ex}
	
	\begin{proof}\
		\begin{enumerate}
			\item Since $T$ is continuous and $\ker T = T^{-1}(\{0\})$, we have that $\ker T$ is closed.
			\item Suppose that there exists $S_1,S_2 \in L(X/ \ker T, T(X)) $ such that $T = S_1 \circ \pi$ and  $T = S_2 \circ \pi $. Let $x \in X$. Then $$S_1(x + \ker T) = S_1(\pi(x)) = T(x) = S_2(\pi(x)) = S_2(x + \ker T)$$ So $S_1 = S_2$. Therefore such a map is unique.\\
			Define $S: X / \ker T \rightarrow T(X)$ by $S(x+ \ker T) = T(x)$. Then $S$ is clearly a linear bijection that satisfies $T = S \circ \pi$. Let $x \in X$ and $z \in \ker T$. Then 
			\begin{align*}
				\|S(x+ \ker T) \|
				& = \|T(x) \|\\
				& = \|T(x+z) \|\\
				& \leq \|T \|\|x+ z \|
			\end{align*} 
			Thus $$\|S(x+ \ker T) \|\leq \|T \|\inf_{z \in \ker T}  \|x + z \|= \|T \|\|x + \ker T \|$$
			So $S$ is bounded and $\|S \|\leq \|T \|$. This implies that $$\|T \|= \|S \circ \pi \|\leq \|S \|\|\pi \|= \|S \|$$
			Thus $\|S \|= \|T \|$.
		\end{enumerate}
	\end{proof}
	
	\begin{ex} \lex{}
		Let $X, Y$ be normed vector spaces. Define $\phi: L(X,Y) \times X \rightarrow Y$ by \\$\phi(T,x) = Tx$. Then $\phi$ is continuous.
	\end{ex}
	
	\begin{proof}
		Let $(T_1, x_1) \in L(X,Y) \times X$ and $\ep > 0$. Choose $\del = \min \{\frac{\ep}{2(\|x_1 \|+ \|T_1 \|+1)}, \frac{\sqrt{\ep}}{\sqrt{2}} \}$. Let $(t_2, x_2) \in L(X,Y) \times X$. Suppose that $$\|(T_1, x_1) - (T_2, x_2) \|= \max \{\|T_1 - T_2\|, \|x_1 -x_2 \|\} < \del$$ Then 
		\begin{align*}
			\|\phi(T_1, x_1) - \phi(T_2-x_2) \|
			&= \|T_1 x_ - T_2 x_2 \|\\
			&= \|T_1 x_1 - T_2 x_1 + T_2 x_1 - T_2 x_2 \|\\
			& \leq \|(T_1 - T_2) x_1 \|+ \|T_2(x_1 -x_2) \|\\
			& \leq \|T_1 -T_2 \|\|x_1 \|+ \|T_2 \|\|x_1 -x_2 \|\\
			& \leq \|T_1 -T_2 \|\|x_1 \|+ \big(\|T_1 - T_2 \|+ \|T_1 \|\big)\|x_1 -x_2 \|\\
			& < \del \|x_1 \|+ (\del + \|T_1 \|) \del \\
			&= \del (\|T_1 \|+ \|x_1 \|) + \del^2\\
			& < \frac{\ep}{2} + \frac{\ep}{2}\\
			&= \ep
		\end{align*}
		So $\phi$ is continuous.
	\end{proof}
	
	\begin{ex} \lex{}
		Let $X$ be a normed vector space and $M \subset X$ a subspace. Then $\ol{M}$ is a subspace.
	\end{ex}
	
	\begin{proof}
		Let $x,y \in \ol{M}$ and $\al \in \C$. Then there exist sequences $(x_n)_{n \in \N} \subset M$ and $(y_n)_{n \in \N} \subset M$ such that $x_n \conv{} x$ and $y_n \conv{} y$. Since $M$ is a subspace, $(x_n +y_n)_{n \in \N} \subset M$ and $(\al x_n)_{n \in \N} \subset M$. Since addition and scalar multiplication are continuous, we have that $x_n + y_n \conv{} x+y$ and $\al x_n \conv{} \al x$. Thus $x+y \in \ol{M}$ and $\al x \in \ol{M}$ and hence $\ol{M}$ is a subspace.
	\end{proof}
	
	\newpage
	
	
	
	
	
	
	
	



	
	
	
	
	
	
	
	
	
	
	
	
	

	

	\newpage
	\subsection{Applications of the Hahn-Banach Theorem}
	
	
	
	
	
	
	\begin{defn} \ld{55002}\
	Let $X$ be a normed vector space over $\C$, and $T :X \rightarrow \C$. Then $T$ is said to be a \tbf{bounded linear functional on} $X$ if $T \in L(X, \C)$. We define the \tbf{dual space of} $X$, denoted $X^*$, by $X^* = L(X, \C)$.
	\end{defn}
	
	\begin{note}
	We define $X^*$ similarly when $X$ is a normed vector space over $\R$.
	\end{note}

	\begin{defn} \ld{55009}
		Let $X$ be a normed vector space and $p:X \rightarrow \R$ a sublinear functional. Then $p$ is said to be \tbf{bounded} if there exists $M >0$ such that for each $x \in X$, $p(x) \leq M\|x\|$. 
	\end{defn}
	
	\begin{ex} \lex{55010}
		Let $X$ be a normed vector space and $p:X \rightarrow \R$ a sublinear functional. Then $p$ is bounded iff $p$ is Lipschitz. 
	\end{ex}
	
	\begin{proof}
		Suppose that $p$ is bounded. Then there exists $M >0$ such that for each $x \in X$, $p(x) \leq M\|x\|$. Let $x, y \in X$. Then the previous exercise implies that 
		\begin{align*}
			-M\|x-y\| 
			&= -M\|y-x\| \\
			& \leq -p(y-x) \\
			& \leq p(x)-p(y) \\
			& \leq p(x-y) \\
			& \leq M \| x-y\| 
		\end{align*}
		So that $$|p(x) - p(y)| \leq  M\|x-y\|$$
		and $p$ is Lipschitz.
		Conversely, suppose that $p$ is Lipschitz. Then there exists $M >0 $ such that for each $x ,y \in X$, $|p(x) - p(y)| \leq  M\|x-y\|$. Let $x \in X$. Then 
		\begin{align*}
			p(x) 
			& \leq |p(x)| \\
			& = |p(x) - p(0)| \\
			& \leq M\|x - 0\| \\
			& \leq M\|x\| 
		\end{align*}
		So $p$ is bounded.
	\end{proof}
	
	\begin{ex} \lex{55013}
	Let $X$ be a normed vector space, $p:X \rightarrow \R$ a bounded sublinear functional and $\phi:X \rightarrow \R$ a linear functional. If $\phi \leq p$, then $\phi \in X^*$. 
	\end{ex}
	
	\begin{proof}
	Since $p$ is Lipschitz, there exists $M >0$ such that for each $x \in X$, 
	\begin{align*}
	p(x) 
	&\leq |p(x)| \\
	&\leq M \|x\|
	\end{align*}
	Let $x \in X$. Then 
	\begin{align*}
	\phi(x) 
	&\leq p(x) \\
	&\leq |p(x)| \\
	&\leq M \|x\| 
	\end{align*}
	and therefore  
	\begin{align*}
	- M \|x\| 
	&= -M \|-x\| \\
	& \leq -p(-x) \\
	& \leq - \phi(-x) \\
	&= \phi(x) 
	\end{align*}
	So that $|\phi(x)| \leq  M\|x\|$ and $\phi \in X^*$.
	\end{proof}
	
	\begin{ex} \lex{55014}
	Let $X$ be a normed vector space and $p:X \rightarrow \R$ a bounded sublinear functional. Then there exists $\phi \in X^*$ such that for each $x \in X$, $\phi(x) \leq p(x)$.
	\end{ex}
	
	\begin{proof}
	A previous exercise implies there exists $\phi: X \rightarrow \R$ such that $\phi$ is linear and $\phi \leq p$. The previous exercise implies that $\phi \in X^*$.
	\end{proof}
	
	\begin{ex} \lex{55015} \tbf{Equivalency of linearity (Bounded Case)}\\
	Let $X$ be a normed vector space and $p:X \rightarrow \R$ a bounded sublinear functional. Then the following are equivalent:
	\begin{enumerate}
	\item there exists a unique $\phi \in X^*$ such that $\phi \leq p$
	\item for each $x \in X$, $-p(-x) = p(x)$
	\item $p$ is linear
\end{enumerate}	
	\end{ex}
	
	\begin{proof}
	Basically the same as last time.
	\end{proof}
	
	\begin{ex} \lex{55016}
		Let $X$ be a normed vector space, $M \subset X$ a subspace and $f \in M^*$. Then there exists $F \in X^*$ such that $\|F \|= \|f \|$ and $F|_M = f$.  
	\end{ex}
	
	\begin{proof}
		If $f =0$, Choose $F=0$. Suppose $f \neq 0$. Then $\|f \|\neq 0$ and there exists $x_0 \in M$ such that $x_0  \neq 0$. Thus $\|f \| \neq 0$. Define $p:X \rightarrow \Rg$ by $ p(x) = \|f \|\|x \|$. Then $p$ is a sublinear functional on $X$ and for each $x \in M$, $\vert f(x) \vert \leq p(x)$. So there exists a linear functional $F:X \rightarrow \C$ such that for each $x \in X$, $\vert F(x) \vert \leq p(x) = \|f \|\|x \|$ and $F|_M = f$. Thus $F \in X^*$ with $\|F \|\leq \|f \|$. Also $$\|F \|= \sup_{\substack{ x \in X \\ \|x \|= 1}} \vert F(x) \vert \geq  \sup_{\substack{ x \in M \\ \|x \|= 1}} \vert F(x) \vert = \sup_{\substack{ x \in M \\ \|x \|= 1}} \vert f(x) \vert = \|f \|$$
		
		So $\|F \|= \|f \|$.
	\end{proof}
	
	\begin{ex} \lex{55017}
		Let $X$ be a normed vector space, $M \subsetneq X$ a proper closed subspace and $x \in X \setminus M$. Then there exists $F \in X^*$ such that $F|_M = 0$, $\|F \|=1$ and $ F(x) = \|x+M \|\neq 0$. \\
		\tbf{Hint:} Consider $f:M+\C x \rightarrow \C$ defined by $f(m+\lam x) = \lam \|x +M \|$.
	\end{ex}
	
	\begin{proof}
		Define $f:M+\C x \rightarrow \C$ as above. Clearly $f$ is linear and $f|_M = 0$. Let $m \in M$ and $\lam \in \C$. If $\lam = 0$, then $\vert f(m +\lam x) \vert = 0 \leq \|m+ \lam x \|$. Suppose that $\lam \neq 0$. Then 
		\begin{align*}
			\vert f(m+\lam x) \vert 
			& = \vert \lam \vert \|x+M \|\\
			& =  \|\lam x+M \|\\
			& = \inf_{z \in M} \|z+ \lam x \|\\
			& \leq  \|m+ \lam x  \|\\
		\end{align*} 
		So $f \in (M+\C x )^*$ and $\|f \|\leq 1$. Let $\ep >0$. A previous exercise tells us that there exist $m \in M, \lam \in \C$ such that $\|m+ \lam x \|= 1$ and $\|m+ \lam x +M \|> 1- \ep$. Then 
		\begin{align*}
			\vert f(m + \lam x) \vert
			&= \vert \lam \vert \|x+M\|\\
			&=\|\lam x +M \|\\
			&= \|m + \lam x +M \|\\
			&> 1-\ep
		\end{align*}
		
		So $$ \|f \|= \sup_{\substack{z \in M + \C x \\ \|z \|=1}} \vert f(z) \vert \geq 1$$ Hence $\|f \|=1$. 
		The same exercise also tells us that $f(x) = \|x+M\|\neq 0$. Using the previous exercise, there exists $F \in X^*$ such that $\|F \|= \|f \|= 1$ and $F|_{M+\C x} = f$.
	\end{proof}
	
	\begin{ex} \lex{55018}
		Let $X$ be a normed vector space and $x \in X$. If $x \neq 0$, then there exists $F \in X^*$ such that $\|F \|= 1$ and $F(x) = \|x \|$.
	\end{ex}
	
	\begin{proof}
		Define $f:\C x \rightarrow \C$ by $f(\lam x) = \lam \|x \|$. Then $f$ is linear and $f(x) = \|x \|$. Clearly $$\sup_{\substack{z \in \C x \\ \|z \|=1}}\vert f(z) \vert = 1$$ 
		So $f \in (\C x)^*$ and $\|f \|= 1$. By a previous exercise, there exists $F \in X^*$ such that $\|F \|= \|f \|=1$ and $F|_{\C x} = f$. 
	\end{proof}
	
	\begin{ex} \lex{55019}
	Let $X$ be a normed vector space and $x \in X$. Then $x = 0$ iff for each $\phi \in X^*$, $\phi(x) = 0$.
	\end{ex}
	
	\begin{proof}
	Clear by previous exercise.
\end{proof}		
		
	\begin{ex} \lex{55020}
		Let $X$ be a normed vector space. Then $X^*$ separates the points of $X$. 
	\end{ex}
	
	\begin{proof}
		Let $x, y \in X$. Suppose that $x \neq y$. Then $x-y \neq 0$. The previous exercies implies that there exists $F \in X^*$ such that $\|F \|= 1$ and $$F(x) - F(y) = F(x-y) = \|x-y \|\neq 0$$ Thus $F(x) \neq F(y)$ and $X^*$ separates the points of $X$.
	\end{proof}
	
	
	\begin{ex} \lex{55021}
		Let $X$ be a normed vector space and $f:X \rightarrow \C$ a linear functional on $X$. Then $f$ is bounded iff $\ker f$ is closed. 
	\end{ex}
	
	\begin{proof}
		Suppose that $f$ is continuous. Since $\{0\}$ is closed, we have that $\ker f = f^{-1}(\{0\})$ is closed. Conversely, suppose that $\ker f$ is closed. If $\ker f = X$, then $f =0$ and $f$ is continuous. Suppose that $\ker f \neq X$. Then $\ker f$ is a proper, closed subspace of $X$. A previous exercise tells us that there exists $x \in X$ such that $\|x \|= 1$ and $\|x + \ker f \|> \frac{1}{2}$. Let $y \in X$. Suppose that $\|y \|< \frac{1}{2}$. Then for each $z \in \ker f$, 
		\begin{align*}
			\|z -  (x+y)\|
			& = \|(z-x) -y \|\\
			& \geq \|z-x \|- \|y \|\\
			& > \frac{1}{2} - \frac{1}{2} \\
			&=0
		\end{align*}
		
		So $x+y \not \in \ker f$. Therefore $f(B(x,\frac{1}{2})) \cap \{0\} = \varnothing$. If $f(B(x,\frac{1}{2})) $ is unbounded, then $f(B(x,\frac{1}{2})) = \C$ by linearity. This is a contradiction since $0 \not \in f(B(x,\frac{1}{2}))$. So There exists $s > 0$ such that $f(B(x,\frac{1}{2})) \subset B(0,s)$ and thus $f$ is bounded. 
	\end{proof}
	
	\begin{ex} \lex{55022}
		Let $X$ be a normed vector space. 
		\begin{enumerate}
			\item Let $M \subsetneq X$ be a proper closed subspace of $X$ and $x \in X \setminus M$. Then $M + \C x$ is closed.
			\item Let $M \subset X$ be a finite dimensional subspace of $X$. Then $M$ is closed.
		\end{enumerate}
	\end{ex}
	
	\begin{proof}
		\begin{enumerate}
			\item Let $y \in X$ and $(y_n)_{n \in \N} \subset M+ \C x$. Suppose that $y_n \conv{} y$. If $y \in M$, then $y \in M+ \C x$. Suppose that $y \not \in M$. For each $n \in \N$, there exists $m_n \in M$ and $\lam_n \in \C$ such that $y_n = m_n + \lam_nx$. A previous exercise tells us that there exists $F \in X^*$ such that $\|F \|= 1$, $F|_M = 0$ and $F(x) = \|x+M \|\neq 0$. Since $F$ is continuous, $F(y_n) \conv{} F(y)$. Since for each $n \in \N$, $$F(y_n) = F(m_n + \lam_n x) = F(m_n)+ \lam_n (F_x) = \lam_n F(x)$$ we have that $\lam_n F(x) \conv{} F(y)$. Since $F(x) \neq 0$, this implies that $\lam_n \conv{} F(x)^{-1} F(y)$. It follows that $\lam_n x \conv{}F(x)^{-1}F(y)x$. Since  for each $n \in \N$, $m_n = y_n - \lam_nx$, we know that $m_n \conv{} y-F(x)^{-1}F(y)x$. Since $(m_n)_{n \in \N} \subset M$ and $M$ is closed, we have that $y-F(x)^{-1}F(y)x \in M$ and therefore $y \in M+\C x$. Hence $M+\C x$ is closed. \vspace{.5cm}\\
			\item If $M = X$, then $M$ is closed. Suppose that $M \neq X$. Let $(x_i)_{i=1}^n$ be a basis for $M$. Define $N_0 = \{0\}$ and for each $i =1,2, \cdots, n$, define $N_i = N_{i-1}+\C x_i$. Since $N_0$ is a proper closed subpace of $X$ and $x_1 \in X \setminus N_0$, (1) implies that $N_1$ is closed. Proceed inductively to obtain that $M = N_n$ is closed.
		\end{enumerate}
	\end{proof}
	
	\begin{ex} \lex{55023}
		Let $X$ be an infinite-dimensional normed vector space. 
		\begin{enumerate}
			\item There exists a sequence $(x_n)_{n\in \N} \subset X$ such that for each $m, n \in \N$, $\|x_n \|= 1$ and if $m \neq n$, then $\|x_m - x_n \|> \frac{1}{2}$.
			\item $X$ is not locally compact. 
		\end{enumerate}
	\end{ex}
	
	\begin{proof}\
		\begin{enumerate}
			\item Define $N_0 = \{0\}$. Then $N_0$ is a closed proper subspace of $X$. Choose $x_1 \in X$ such that $\|x_1 \|= 1$. Using the results of previous exercises, we proceed inductively. For each $n \geq 2$ we define $N_{n-1} = \text{span}(x_1, x_2, \cdots, x_{n-1})$. Then $N_{n-1}$ is a closed proper subspace of $X$. Thus we may choose $x_n \in X$ such that $\|x_n \|= 1$ and $\|x_n + N_{n-1} \|>  \frac{1}{2}$. Let $m,n \in \N$. Suppose that $m<n$. Then $x_m \in N_{n-1}$. Thus $\|x_n - x_m \|\geq \|x_n + N_{n-1} \|> \frac{1}{2}$\vspace{.5cm}\\
			\item Suppose that $X$ is locally compact. Then $\cl B(0,1)$ is compact and therefore sequentially compact. Using $(x_n)_{n \in \N} \subset \cl B(0,1)$ defined in (1), we see that there exists a subsequence $(x_{n_k})_{k \in \N}$, $x \in \cl B(0,1)$ such that $x_{n_k} \conv{} x$. Then $(x_{n_k})_{k \in \N}$ is Cauchy. So there exists $N \in N$ such that for each $j, k \in \N$, if $j, k \geq N$, then $\|x_{n_j} - x_{n_k} \|< \frac{1}{2}$. Then $\|x_{n_N} - x_{n_{N+1}} \| < \frac{1}{2}$. This is a contradiction since by construction, $\|x_{n_N} - x_{n_{N+1}} \| > \frac{1}{2}$. Thus $X$ is not locally compact.
		\end{enumerate}
	\end{proof}
	
	
	
	
	
	
	
	
	
	
	
	
	
	
	
	
	
	

	
	
	
	
	
	
	
	
	
	
	
	
	
	
	
	
	
	
	
	
	
	
	
	\newpage
	\subsection{The Baire Category and Closed Graph Theorems}
	
	\begin{thm} \tbf{Open Mapping Theorem:} \\
		Let $X, Y$ be Banach spaces and $T\in L(X,Y)$. If $T$ is surjective, then $T$ is open.
	\end{thm}
	
	\begin{cor}
		Let $X, Y$ be Banach spaces and $T \in L(X,Y)$. If $T$ is a bijection, then $T^{-1} \in L(X,Y)$.
	\end{cor}
	
	\begin{defn} \ld{}
		Let $X,Y$ be sets and $f:X \rightarrow Y$. We define the \tbf{graph of f}, $\Gam(f)$, by $\Gam(f) = \{(x,y) \in X \times Y: f(x) = y\}$.
	\end{defn}
	
	\begin{thm}
		Let $X, Y$ be Banach spaces and $T:X \rightarrow Y$ a linear map. If $\Gam(T)$ is closed, then $T \in L(X,Y)$.  
	\end{thm}
	
	\begin{note}
		We recall that $\Gam(T)$ is closed iff for each $(x_n)_{n \in \N} \subset X$, $x \in X$ and $y \in Y$, $x_n \conv{} x$ and $T(x_n) \conv{} y$ implies that $T(x) = y$. 
	\end{note}
	
	\begin{thm}
		
		Let $X, Y$ be Banach spaces and $S \subset L(X,Y)$. If for each $x \in X$, $$\sup_{T \in S} \|Tx \|< \infty$$ then $$\sup_{T \in S} \|T \|< \infty$$
	\end{thm}
	
	\begin{ex} \lex{}
		Let $\mu$ be counting measure on $(N, \MP(\N))$. Define $h: \N \rightarrow \N$ and $ \nu$ on $(N, \MP(\N))$ by $h(n) = n$ and $d \nu = h d \mu$. Define $X=L^1(\nu)$ and $Y = L^1(\mu)$. Equip both $X$ and $Y$ with the $L^1$ norm with respect to $\mu$. 
		\begin{enumerate}
			\item We have that $X$ is a proper subspace of $Y$ and therefore $X$ is not complete.
			\item Define $T: X \rightarrow Y$ by $Tf(n) = nf(n)$. Then $T$ is linear, $\Gam(T)$ is closed, and $T$ is unbounded.
			\item Define $S:Y \rightarrow X$ by $Sg(n) = \frac{1}{n}g(n)$. Then $S \in L(Y,X)$, $S$ is surjective and $S$ is not open. 
		\end{enumerate}
	\end{ex}
	
	\begin{proof}\
		\begin{enumerate}
			\item Note that for each $f: \N \rightarrow \C$, 
			\begin{align*}
				{\|f \|}_{\mu, 1}
				&= \sum_{n=1}^{\infty} \vert f(n) \vert  \\
				& \leq \sum_{n=1}^{\infty} n \vert f(n) \vert  \\
				& = \|f \|_{\nu,1} 
			\end{align*} 
			Hence $X$ is a subspace of $Y$. Define $f : \N \rightarrow \C$ by $f(n) = \frac{1}{n^2}$. Then $$\|f \|_{\mu, 1} = \sum_{n=1}^{\infty} \frac{1}{n^2} < \infty$$ So  $f \in Y$. 
			However 
			$$\|f \|_{\nu, 1} = \sum_{n=1}^\infty \frac{1}{n} = \infty$$ 
			So $f \not \in X$. Thus $X$ is a proper subspace of $Y$. Let $g \in Y$ and $\ep >0$. Since the simple functions are dense in $L^1(\mu)$, there exists $\phi \in L^1(\mu)$ such that $\phi$ is simple and $\|g - \phi \|_{\mu ,1} < \ep$. Then there exist $(c_i)_{i=1}^k \subset \C$ and $ (E_i)_{i=1}^k \subset \MP(\N)$ such that for each $i,j \in  \{1,2,\cdots, k\}$, $E_i$ is finite, $i \neq j$ implies that $E_i \cap E_j = \varnothing$ and  
			$$\phi = \sum_{i=1}^kc_i \chi_{E_i}$$ 
			Define $c = \max\{\vert c_i \vert: i=1,2,\cdots k\}$ and $m = \max \bigg[ \bigcup\limits_{i=1}^k E_i \bigg]$. Then 
			\begin{align*}
				\|\phi \|_{\nu,1} 
				&=  \sum_{n=1}^m n \vert \phi(n) \vert \\
				& \leq \sum_{n=1}^m  mc \\
				& = c m^2 \\
				& < \infty
			\end{align*}
			Hence $\phi \in X$ and $X$ is dense in $Y$. Since $X$ is a dense, proper subspace, it is not closed. Since $Y$ is complete and $X \subset Y$ is not closed, we have that $X$ is not complete.
			\item Clearly $T$ is linear. Let $(f_j)_{j \in \N} \subset X$, $f \in X$ and $g \in Y$. Suppose that $f_j \conv{L^1(\mu)} f$ and $Tf_j \conv{L^1(\mu)} g$. 
			
			Note that for each $j \in \N$ and $n \in \N$, $$\vert f_j(n) - f(n) \vert \leq \sum_{n =1}^{\infty}\vert f_j(n) - f(n) \vert = \|f_j-f \|_{\mu, 1}$$ and $$\vert nf_j(n) - g(n) \vert \leq \sum_{n =1}^{\infty}\vert nf_j(n) - g(n) \vert = \|Tf_j - g\|_{\mu, 1}$$  
			Thus for each $n \in \N$, $f_j(n) \conv{j} f(n)$ and $nf_j(n) \conv{j} g(n)$. This implies that for each $n \in \N$, $nf(n) = g(n)$. Thus $Tf = g$ which implies that $\Gam(T)$ is closed. Suppose, for the sake of contradiction, that $T$ is bounded. Then there exists $C \geq 0$ such that for each $f \in X$, $\|Tf \|_{\mu,1} \leq C \|f \|_{\mu, 1}$. Choose $n \in \N$ such that $n > C$. Define $f: \N \rightarrow \C$ by $f = \chi_{\{n\}}$. As established above, $S^+ \subset L^1(\mu)$. Then $\|f \|_{\mu,1} = 1$ and
			\begin{align*}
				\|Tf \|_{\mu,1}
				& = n \\
				&> C\\
				& = C \|f \|_{\mu,1}
			\end{align*}
			which is a contradiction. So $T$ is unbounded.
			\item Clearly $S$ is linear. Let $g \in Y$. Then \begin{align*}
				\|Sg \|_{\mu,1} 
				&= \sum_{n =1}^{\infty} \frac{1}{n} \vert g(n) \vert \\
				& \leq  \sum_{n =1}^{\infty} \vert g(n) \vert \\
				& = \|g \|_{\mu,1}
			\end{align*}
			So $S$ is bounded and $\|S \|\leq 1$. Thus $S \in L(Y,X)$. Let $f \in X$. Define $g: \N \rightarrow \C$ by $g(n) = nf(n)$. By defnition, $g \in Y$ and we have that
			\begin{align*}
				Sg(n) 
				&= \frac{1}{n}g(n) \\
				& = f(n)
			\end{align*}
			Hence $Sg =f$ and thus $S$ is surjective. Let $g \in Y$. Suppsose that $Sg = 0$. Then $$\sum_{n=1}^{\infty} \frac{1}{n}\vert g(n)\vert =\|Sg \| = 0$$ Thus for each $n \in \N$, $g(n) = 0$. Hence $\ker S = \{0\}$ and $S$ is injective. Note that for each $A \subset Y$, $S(A)= T^{-1}(A)$. If $S$ is open, then $T$ is continuous which as shown above is a contradiction. So $g$ is not open. 
		\end{enumerate}
	\end{proof}
	
	\begin{ex} \lex{}
		Let $X = C^1([0,1])$ and $Y=C([0,1])$. Equip both $X$ and $Y$ with the uniform norm. 
		\begin{enumerate}
			\item Then $X$ is not complete
			\item Define $T: X \rightarrow Y$ by $Tf = f'$. Then $\Gam(T)$ is closed and $T$ is not bounded. 
		\end{enumerate}
	\end{ex}
	
	\begin{proof}
		\begin{enumerate}
			\item Recall that for each $a,b \geq 0$ and $p \in \N$, $$(a^{\frac{1}{p}}+b^{\frac{1}{p}})^p = \sum_{n=0}^p  {p \choose n} a^{\frac{n}{p}}b^{\frac{p-n}{p}} \geq a + b$$ Thus $(a+b)^{\frac{1}{p}} \leq a^{\frac{1}{p}}+b^{\frac{1}{p}}$.\\
			For each $n \in \N$, define $f_n: [0,1] \rightarrow \C$ by $f_n(x) = \sqrt{(x-\frac{1}{2})^2+ \frac{1}{n^2}}$. Then $(f_n)_{n \in \N} \subset X$. Define $f:[0,1] \rightarrow \C$ by $f(x) = \vert x-\frac{1}{2}\vert$. Then $f \in Y \cap X^c$. Note that for each $n \in \N$, $f \leq f_n$. Our observation above implies that for each $x \in X$,
			\begin{align*}
				f_n(x) 
				&= \bigg[ (x-\frac{1}{2})^2 + \frac{1}{n^2} \bigg]^{\frac{1}{2}}\\
				& \leq \vert x-\frac{1}{2} \vert + \frac{1}{n}
			\end{align*}
			Thus $0 \leq f_n - f \leq \frac{1}{n} $. This implies that $f_n \convt{u} f$. Since $f \not \in X$, $X$ is not complete. \vspace{.5cm}\\
			\item Let $(f_n)_{n \in \N} \subset X$, $f \in X$ and $g \in Y$. Suppose that $f_n \convt{u} f$ and $Tf_n \convt{u} g$. Let $x \in [0,1]$. Then $f_n(x) \conv{} f(x)$ and $f_n(0) \conv{} f(0)$ and $f_n' \convt{u} g$. Applying the DCT to this sequence of integrable functions that converges uniformly to an integrable function on a finite measure space (a previous exercise) we have that
			\begin{align*}
				f_n(x) - f_n(0) 
				&= \int_{[0,x]} f_n' dm \\
				& \rightarrow \int_{[0,x]} g dm \\ 
			\end{align*} 
			Since $f_n(x) - f_n(0) \conv{} f(x) - f(0)$, we know that $$f(x) - f(0) = \int_{[0,x]} g dm$$ Thus $Tf = g$ and $\Gam(T)$ is closed. \\
			By \rex{42002}, $T$ is not bounded.
		\end{enumerate}
	\end{proof}
	
	\begin{ex} \lex{}
		Let $X, Y$ be Banach spaces and $T \in L(X,Y)$. Then $X/\ker T \cong T(X)$ iff $T(X)$ is closed.
	\end{ex}
	
	\begin{proof}
		Since $X$ is a banach space and $T$ is continuous, we have that $\ker T$ is closed and $X/ \ker T$ is a Banach space. Suppose that $X/ \ker T \cong T(X)$. Then $T(X)$ is complete. Since $Y$ is complete, this implies that $T(X)$ is closed. \\
		Conversely Suppose that $T(X)$ is closed. Then $T(X)$ is complete. Define $S: X/ \ker T \rightarrow T(X)$ by $S(x + \ker T) = T(x)$. A previous exercise tells us that the map $S: X/ \ker T \rightarrow T(X)$ defined by $S(x + \ker T) = T(x)$ is a bounded linear bijection. Since $T(X)$ is complete and $S$ is surjective, $S^{-1}$ is bounded and thus $S$ is an isomorphism.   
	\end{proof}
	
	\begin{ex} \lex{}
		Let $X$ be a separable Banach space. Define $B_X = \{x \in X: \|x \|< 1\}$. Let $(x_n)_{n \in \N} \subset B_X $ a dense subset of the unit ball and $\mu$ the counting measure on $(\N, \MP(\N))$. Define $T: L^1(\mu) \rightarrow X$ by $$Tf = \sum_{n=1}^{\infty}f(n)x_n$$ Then 
		\begin{enumerate}
			\item $T$ is well defined and $T \in L(L^1(\mu), X)$
			\item $T$ is surjective
			\item There exists a closed subspace $K \subset L^1(\mu)$ such that $L^1(\mu)/K \cong X$ 
		\end{enumerate} 
	\end{ex}
	
	\begin{proof}
		\begin{enumerate}
			\item Let $f \in L^1(\mu)$. Since $X$ is complete and 
			\begin{align*}
				\sum_{n=1}^{\infty}\|f(n)x_n \|
				& = \sum_{n=1}^{\infty} \vert f(n) \vert \|x_n \|\\
				& \leq \sum_{n=1}^{\infty} \vert f(n) \vert \\
				&< \infty 
			\end{align*}
			we have that $\sum_{n=1}^{\infty} f(n)x_n $ converges and thus $Tf \in X$. Hence $T$ is well defined. \vspace{.5cm}\\
			Clearly $T$ is linear. Let $f \in L^1(\mu)$. Then
			\begin{align*}
				\|Tf \|
				&= \| \sum_{n=1}^{\infty} f(n)x_n \|\\
				& \leq \sum_{n=1}^{\infty} \|f(n)x_n \|\\
				& \leq \sum_{n=1}^{\infty} \vert f(n) \vert \\
				&= \|f \|_1
			\end{align*}
			So $T$ is bounded with $\|T \|\leq 1$.\vspace{.5cm}\\
			\item Let $x \in X$. Suppose that $\|x \|< 1$. Then $x \in B_X$. So there exists $n_1 \in \N$ such that $\|x - x_{n_1} \|< \frac{1}{2}$. Then $2(x-x_{n_1}) \in B_X$. Since for each $j \in \N$, $B_X\setminus (x_n)_{n=1}^j$ is dense in $B_X$, there exists $n_2 \in \N$ such that $x_{n_2} \not \in (x_n)_{n=1}^{n_1}$ and $\|2(x- x_{n_1}) - x_{n_2} \|< \frac{1}{2}$ which implies that $\|x- (x_{n_1} - \frac{1}{2}x_{n_2}) \|< \frac{1}{4}$. \vspace{.5cm}\\ 
			Proceed inductively to obtain a subsequence $(x_{n_k})_{k \in \N}$ such that for each $k \geq 2$, $x_{n_k} \not \in (x_n)_{n=1}^{n_{k-1}}$ and $\|x - \sum_{j=1}^k 2^{1-j}x_{n_j} \|< \frac{1}{2^k}$. Then $x = \sum_{k=1}^{\infty}2^{1-k}x_{n_k}$. \vspace{.5cm} \\ 
			Define $f:\N \rightarrow \C$ by $f = \sum_{k=1}^{\infty}2^{1-k}\chi_{\{n_k\}}$. Then $\|f \|_1 = \sum_{k=1}^{\infty}2^{1-k}< \infty$, so $f \in L^1(\mu)$ and $Tf = \sum_{k=1}^{\infty}2^{1-k}x_{n_k} = x$. Now, suppose that $\|x \|\geq 1$, then $\frac{1}{2\|x \|}x \in B_X$. The above argument shows that there exists $f \in L^1(\mu)$ such that $Tf = \frac{1}{2\|x \|}x$. Then $2 \|x \|f \in L^1(\mu)$ and $T(2 \|x \|f) = 2 \|x \|Tf =x$. \\
			So for each $x \in X$, there exists $f \in L^1(\mu)$ such that $Tf = x$ and thus $T$ is surjective. 
			\item Since $X$ is a Banach space and $T$ is surjective, the previous exercise implies that $L^1(\mu)/\ker T \cong X$. 
		\end{enumerate}
	\end{proof}
	
	



























	\newpage
	\subsection{Duality}

	\begin{note}
		Let $X$ be a normed vector space. Then $X^*$ is a normed vector space. In general the weak-$*$ topology on $X^*$ is not necessarily the same as the norm topology on $X^*$. In the context of normed vector spaces, we will write $X^{**}$ to denote $(X^*)^*$ when $X^*$ is equipped with the norm topology and $\hat{X}$ to denote $(X^*)^*$ when $X^*$ is equipped with the weak-$*$ topology. 
	\end{note}
	
	\begin{ex} \lex{}
		Let $X$ be a normed vector space and $x \in X$. Define $\hat{x}:X^* \rightarrow \C$ by $\hat{x}(f) = f(x)$. Then $\hat{x} \in X^{**}$ and $\|\hat{x} \|= \|x \|$. \\
		\tbf{Hint:} Hahn-Banach theorem
	\end{ex}
	
	\begin{proof}
		Let $f,g \in X^*$ and $\lam \in \C$. Then $$\hat{x}(f+\lam g) = (f+ \lam g)(x) = f(x) + \lam g(x) = \hat{x}(f) + \lam \hat{x}(g)$$
		So $\hat{x}$ is linear. For each $f \in X^*$, $$\vert \hat{x}(f) \vert = \vert f(x) \vert \leq \|x \|\|f \|$$ Hence $\hat{x} \in X^{**}$ with $\|\hat{x} \|\leq \|x \|$. If $x=0$, then $\hat{x} = 0$ and $\|\hat{x} \|= \|x \|$. Suppose that $x \neq 0$. Then a previous exercise implies that there exists $F \in X^*$ such that $\|F \|=1$ and $F(x) = \|x \|$. Then we have that $$\sup_{\substack{f \in X^* \\ \|f \|= 1 } } \vert \hat{x}(f) \vert  = \sup_{\substack{f \in X^* \\ \|f \|= 1 }}  \vert f(x) \vert \geq \vert F(x) \vert = \|x \|$$
		Hence $\|\hat{x} \|= \|x \|$.
	\end{proof}

	\begin{ex}
		Let $X$ be a normed vector space. If $X$ is separable, then there exist $(\phi_n)_{n \in \N} \subset X^*$ such that for each $n \in \N$, $\|\phi_n\| = 1$ and for each $x \in X$, 
		$$\|x\| = \sup_{n \in \N} |\phi_n(x)|$$  
		\tbf{Hint:} Let $(x_n)_{n \in \N} \subset X$ be a dense subset. A previous exercise on the Hahn-Banach theorem implies that for each $n$, there exists $\phi_n \in X^*$ such that $\|\phi_n\| = 1$ and $\phi_n(x_n) = \|x_n\|$. Then for each $x \in X$, $$\|x\| = \sup_{n \in \N} |\phi_n(x)|$$. 
	\end{ex}
	
	\begin{proof}
		Suppose that $X$ is separable. Then there exists $(x_n)_{n \in \N} \subset X$ such that $(x_n)_{n \in \N}$ is dense in $X$. A previous exercise on the Hahn-Banach theorem implies that for each $n$, there exists $\phi_n \in X^*$ such that $\|\phi_n\| = 1$ and $\phi_n(x_n) = \|x_n\|$. Let $x \in X$. Then 
		\begin{align*}
			\|x\| 
			&= \|\hat{x}\| \\
			&= \sup_{\substack{\phi \in X^* \\ \|\phi\| = 1}} \|\hat{x}(\phi)\| \\
			&= \sup_{\substack{\phi \in X^* \\ \|\phi\| = 1}} \|\phi(x)\| \\
			& \geq \sup_{n \in \N} \|\phi_n(x)\| \\
		\end{align*}
		Let $\ep > 0$. Choose $N \in \N$ such that $\|x - x_N\| < \ep/2$. Then
		\begin{align*}
			\|x\| 
			& \leq \|x - x_N\| + \|x_N\| \\
			& = \|x - x_N\| + |\phi_N(x_N)| \\
			& \leq \|x - x_N\| + |\phi_N(x_N - x)| + |\phi_N(x)| \\
			& \leq \|x - x_N\| + \|\phi_N\|\|x_N - x\| + |\phi_N(x)| \\
			& \leq 2\|x - x_N\| + |\phi_N(x)| \\
			& < 2 \frac{\ep}{2} +  |\phi_N(x)| \\
			& \leq \ep + \sup_{n \in \N} |\phi_n(x)|
		\end{align*}
		Since $\ep >0$ is arbitrary, $\|x\| \leq \sup\limits_{n \in \N} |\phi_n(x)|$. So $\|x\| = \sup\limits_{n \in \N} |\phi_n(x)|$.
	\end{proof}
	
	
	\begin{ex} \lex{}
		Let $X$ be a normed vector space. Define $\phi : X \rightarrow X^{**}$ by $\phi(x) = \hat{x}$. Then $\phi$ is a linear isometry. 
	\end{ex}
	
	\begin{proof}
		Let $x,y \in X$ and $\lam \in \C$. Then for each $f \in X^*$, we have that 
		\begin{align*}
			\phi(x+ \lam y)(f) 
			&= \widehat{x+ \lam y}(f) \\
			&= f(x+\lam y) \\
			&= f(x) + \lam f(y) \\
			&= \hat{x}(f) + \lam \hat{y}(f)\\
			&= \phi(x)(f) + \lam \phi(y)(f)
		\end{align*} 
		So $\phi(x+ \lam y) = \phi(x) + \lam \phi(y)$ and $\phi$ is linear. The previous exercise tells us that 
		\begin{align*}
			\|\phi(x) - \phi(y) \|
			&= \|\phi(x-y)\|\\
			&= \|\widehat{x-y} \|= \|x-y \|
		\end{align*}
		So $\phi$ is an isometry.
	\end{proof}
	
	\begin{defn} \ld{}
		Let $X$ be a normed vector space and define $\phi:X \rightarrow X^{**}$ as above. We define $\widehat{X} = \phi(X) \subset X^{**}$. Since $\widehat{X}$ and $X$ are isomorphic, we may identify $X$ as a subset of $X^{**}$. 
	\end{defn}
	
	\begin{defn} \ld{}
		Let $X$ be a normed vector space and define $\phi:X \rightarrow X^{**}$ as above. Then $X$ is said to be \tbf{reflexive} if $\phi$ is surjective. In this case $\phi$ is then an isomorphism
	\end{defn}

	\begin{defn}
	Let $X,Y$ be normed vector spaces and $T \in L(X,Y)$. Define the \tbf{adjoint of $T$}, denoted  $T^*:Y^* \rightarrow X^*$, by $T^*(f) = f \circ T$. 
	\end{defn}
	
	\begin{ex} \lex{}
		Let $X,Y$ be normed vector spaces and $T \in L(X,Y)$. 
		\begin{enumerate}
			\item Then $T^* \in L(Y^*, X^*)$.
			\item Applying the result from (1) twice, we have that $T^{**} \in L(X^{**},Y^{**})$. We have that for each $x \in X$, $T^{**}(\hat{x}) = \widehat{T(x)}$.
			\item $T^*$ is injective iff $T(X)$ is dense in $Y$.
			\item If $T^*(Y^*)$ is dense in $X^*$, then $T$ is injective. The converse is true if $X$ is reflexive.
		\end{enumerate}
	\end{ex}
	
	\begin{proof}\
		\begin{enumerate}
			\item Let $f \in Y^*$. Then $\|T^* (f) \|= \|f \circ T \|\leq  \|T \| \|f \|$. So $T^* \in L(Y^*, X^*)$ with $\|T^* \|\leq \|T \|$.\vspace{.5cm}\\
			\item Let $x \in X$. Let $f \in Y^*$. Then 
			\begin{align*}
				T^{**}(\hat{x})(f) 
				&= \hat{x} \circ T^{*}(f) \\
				&= \hat{x}(T^* (f)) \\
				&= \hat{x}(f \circ T) \\
				&= f \circ T (x) \\
				&= f(T(x)) \\
				&= \widehat{T(x)}(f)
			\end{align*} 
			Hence $T^{**}(\hat{x}) = \widehat{T(x)}$.\vspace{.5cm}\\
			\item Suppose that $T(X)$ is not dense in $Y$. Then $\cl T(X) \neq Y$. So $T(X)$ is a proper closed subspace of $Y$ and there exists $y \in Y$ such that $y \not \in \cl T(X)$. By a previous exercise, there exists $f \in Y^*$ such that $f(y) = \|y+\cl T(X) \|\neq 0$, $\|f \|=1$ and $f|_{\cl T(X)} = 0$. Let $x \in X$. Then $T^*(f)(x) = f \circ T(x) = 0$. Hence $T^*(f) = 0 = T^*(0)$. Since $f \neq 0$, $T^*$ is not injective.\\ Now suppose that $T(X)$ is dense in $Y$. Let $f,g \in Y^*$. Define $h \in Y^*$ by $h = f-g$ Suppose that $T*(f) = T^*(g)$ Then $T^*(h) = 0$. So for each $x \in X$, $h(T(x)) = 0$. Let $y \in Y$ and $\ep >0$. By continuity, there exists $\del > 0 $ such that for each $y' \in Y$, if $\|y - y' \|< \del$, then $\|h(y) - h(y') \|< \ep$. Since $T(X)$ is dense in $Y$, there exists $x \in X$ such that $\|y - T(x) \|< \del$. Thus 
			\begin{align*}
				\|h (y) \|
				&\leq \|h(y) - h(T(x)) \|+ \|h(T(x)) \|\\
				& = \|h(y) - h(T(x)) \| \\
				& < \ep
			\end{align*} 
			Since $\ep > 0$ is arbitrary, $\|h(y) \|= 0$. This implies that $h(y) = 0$ and therefore $f(y) = g(y) $. Since $y \in Y$ is arbitrary, $f=g$ and $T^*$ is injective. \vspace{.5cm}\\
			\item For the sake of contradiction, suppose that $T^*(Y^*)$ is dense in $X^*$ and $T$ is not injective. Then there exist $x_1, x_2 \in X$ such that $x_1 \neq x_2$ and $T(x_1) = T(x_2)$. Define $x = x_1-x_2$. Then $x \neq 0$ and $T(x) = 0$. A previous exercise implies that there exists $F \in X^*$ such that $F(x) = \|x\|\neq 0$ and $\|F \|= 1$. Let $\ep >0$. Choose $g \in Y^*$ such that $\|F - T^*(g) \|< \ep$. Then 
			\begin{align*}
				\|x \|
				&= \vert F(x) \vert \\
				&\leq \vert F(x) - T^*(g)(x) \vert + \vert T^*(g)(x) \vert \\
				& < \ep \|x \|+ \vert g(T(x)) \vert\\
				&= \ep \|x \|
			\end{align*}
			
			Since $\ep > 0$ is arbitrary, we have that $\|x \|=0$ which is a contradiction. Hence if $T^*(Y^*) $ is dense in $X^*$, then $T$ is injective. \vspace{.5cm}\\ 
			Now, suppose that $X$ is reflexive and $T$ is injective. Let $\phi_1, \phi_2 \in X^{**}$. Suppose that $T^{**}(\phi_1) = T^{**}(\phi_2)$. Then $T^{**}(\phi_1 - \phi_2) = 0$. Since $X$ is reflexive, there exist $x_1, x_2 \in X$ such that $\phi_1 = \hat{x_1}$ and $\phi_2 = \hat{x_2}$. Define $x = x_1 - x_2$. Then $T^{**}(\hat{x}) = 0$. So for each $f \in Y^*$, 
			\begin{align*}
				T^{**}(\hat{x})(f) 
				&= \hat{x} \circ T^*(f)\\
				&= \hat{x}( T^*(f))\\
				&= \hat{x} (f \circ T)\\
				&= f \circ T(x)\\
				&= f(T(x))\\
				&= 0 
			\end{align*}
			Suppose that $T(x) \neq 0$. Then a previous exercise implies that there exists $g \in Y^*$ such that $g(T(x)) = \|T(x) \|\neq 0$ and $\|g \| = 1$. This is a contradiction since $g(T(x)) = 0$. So $T(x) = 0$. Since $T$ is injective, this implies that $x = 0$. Hence $\hat{x}=0$ and thus $\phi_1 = \phi_2$. Thus $T^{**}$ is injective. By (3), we have that $T^*(Y^*)$ is dense in $X^*$.
		\end{enumerate}
	\end{proof}
	
	\begin{ex} \lex{}
		Let $X$ be a normed vector space. Then $X$ is reflexive iff $X^*$ is reflexive. 
	\end{ex}
	
	\begin{proof}
		Suppose that $X$ is reflexive. Let $\al \in X^{***}$. Define $f :X \rightarrow \C$ by $f(x) = \al(\hat{x})$. Clearly $f$ is linear and a previous exercise tells us that for each $x \in X$, 
		\begin{align*}
			\vert f(x) \vert 
			& \leq \|\al \|\|\hat{x} \|\\
			&= \|\al \|\|x \|
		\end{align*}
		So $f \in X^*$.
		Let $\phi \in X^{**}$. Since $X$ is reflexive, there exists $x \in X$ such that $\phi = \hat{x}$. Then 
		\begin{align*}
			\al(\phi)
			&= \al(\hat{x})\\
			&= f(x)\\
			&= \hat{x}(f)\\
			&= \hat{f}(\hat{x})\\
			&= \hat{f}(\phi)
		\end{align*}
		Hence $\al = \hat{f}$. Thus the map $X^* \rightarrow X^{***}$ given by $f \mapsto \hat{f} $ is surjective and so $X^{*}$ is reflexive.\vspace{.5cm}\\
		Conversely, suppose that $X^*$ is reflexive. Since $\phi:X \rightarrow X^{**}$ given by $\phi(x) = \hat{x}$ is an isometry, $\widehat{X} \subset X^{**}$ is closed. For the sake of contradiction, suppose that $\widehat{X} \neq X^{**}$. Then there exists $\al \in X^{**}$ such that $\al \not \in \widehat{X}$. Thus there exists $F \in X^{***}$ such that $\|F \|= 1$, $F(\al) = \|\al + \widehat{X} \|\neq 0$ and $F|_{\widehat{X}}=0$. Since $X^*$ is reflexive, there exists $f \in X^*$ such that $F = \hat{f}$. A previous exercise tells us that $\|f \|= \|\hat{f} \|= \|F \|= 1$. Since for each $x \in X$, $f(x) = \hat{x}(f) = \hat{f}(\hat{x}) = F(\hat{x}) = 0$, we have that $f = 0$. Thus $\|f \|= 0$, a contradiction. So $\widehat{X} = X^{**}$ and $X$ is reflexive.
		
	\end{proof}
	
	
	\begin{defn}
	Let $X$ be a normed vector space, $M \subset X$ and $N \subset X^*$. We define the \tbf{annihilator} of $M$ and the annihilator of $N$, denoted by $M^{\perp} \subset X^*$ and $^{\perp}N \subset X$ respectively, by 
	\begin{align*}
	 M^{\perp} &= \{\phi \in X^*: \text{for each $x \in M$, $\phi(x) = 0$}\} \\
	 ^{\perp}N &= \{x \in X: \text{for each $\phi \in N$, $\phi(x) = 0$}\}
	\end{align*}
	\end{defn}	
	
	\begin{ex}
	Let $X$ be a normed vector space, $M \subset X$ and $N \subset X^*$. Then 
	\begin{enumerate}
	\item $$M^{\perp} = \bigcap_{x \in M} \ker \hat{x}$$
	\item $$^{\perp}N = \bigcap_{\phi \in N} \ker \phi $$
	\end{enumerate}
	\end{ex}
	
	\begin{proof}\
	\begin{enumerate}
	\item 
	\begin{align*}
	M^{\perp} 
	&= \{\phi \in X^*: \text{for each $x \in M$, $\phi(x) = 0$}\} \\
	&= \bigcap_{x \in M} \{\phi \in X^*: \phi(x) = 0\} \\
	&= \bigcap_{x \in M} \{\phi \in X^*: \hat{x}(\phi) = 0\} \\
	&= \bigcap_{x \in M} \ker \hat{x}
	\end{align*}
	\item 
	\begin{align*}
	^{\perp}N 
	&= \{x \in X: \text{for each $\phi \in N$, $\phi(x) = 0$}\} \\
	&= \bigcap_{\phi \in N} \{x \in X: \phi(x) = 0\} \\
	&= \bigcap_{\phi \in N} \ker \phi
	\end{align*}
	\end{enumerate}
	\end{proof}
	
	\begin{ex}
	Let $X$ be a normed vector space, $M \subset X$ and $N \subset X^*$. Then 
	\begin{enumerate}
	\item $M^{\perp}$ is weak-* closed
	\item $^{\perp} N$ is closed
	\end{enumerate}
	\end{ex}
	
	\begin{proof}\
	\begin{enumerate}
	\item Let $(\phi_n)_{n \in \N} \subset M^{\perp}$ and $\phi \in X^*$. Suppose that $\phi_n \conv{w^*} \phi$. Then for each $x \in X$, $\phi_n(x) \rightarrow \phi(x)$. Let $x \in M$. By definition, for each $n \in \N$, $\phi_n(x) = 0$. Thus $\phi_n(x) \rightarrow 0$ which implies that $\phi(x) = 0$ and $\phi \in \ker \hat{x}$. Since $x \in M$ is arbitrary, 
	\begin{align*}
	\phi 
	&\in \bigcap_{x \in M} \ker \hat{x} \\
	&= M^{\perp}
	\end{align*}
	\item Let $(x_n)_{n \in \N} \subset {^{\perp} N}$ and $x \in X$. Suppose that $x_n \rightarrow x$. Let $\phi \in N$. Continuity implies that $\phi(x_n) \rightarrow \phi(x)$. By definition, for each $n \in \N$, $\phi(x_n) = 0$. Thus $\phi(x_n) \rightarrow 0$ which implies that $\phi(x) = 0$. So $x \in \ker \phi$. Since $\phi \in N$ is arbitrary, 
	\begin{align*}
	x 
	&\in \bigcap_{\phi \in N} \ker \phi \\
	&= {^{\perp}N}
	\end{align*}
	\end{enumerate}
	\end{proof}
	
	\begin{ex}
	Let $X$ be a normed vector space, $M \subset X$ and $N \subset X^*$. Then 
	\begin{enumerate}
	\item $^{\perp}(M^{\perp}) = \cl M$, i.e. the norm closure of $M$
	\item $({^{\perp}N})^{\perp} = \text{cl}_{w^*}(N)$, i.e. the weak-* closure of $N$.
	\end{enumerate}
	\end{ex}
	
	\begin{proof}\
	\begin{enumerate}
	\item Let $x \in M$, then by definition, for each $\phi \in M^{\perp}$, $\phi(x) = 0$. Again by definition, $x \in {^{\perp}(M^{\perp})}$. So $M \subset {^{\perp}(M^{\perp})}$. Since ${^{\perp}(M^{\perp})}$ is closed, $\cl M \subset {^{\perp}(M^{\perp})}$. For the sake of contradiction, suppose that ${^{\perp}(M^{\perp})} \not \subset \cl M$. Then there exists $x \in {^{\perp}(M^{\perp})}$ such that $x \not \in \cl M$. \rex{55017} implies that there exists $\phi \in X^*$ such that $\phi|_{\cl M} = 0$, $\|\phi\| = 1$ and $\phi(x) = \|x + \cl M\| > 0$. By definition, $\phi \in M^{\perp}$. Since $\phi(x) \neq 0$, we have that $x \not \in {^{\perp}(M^{\perp})}$. This is a contradiction and so ${^{\perp}(M^{\perp})} \subset \cl M$.
	\item 
	\end{enumerate}
	\end{proof}
	
	
	
	







	
	
	
	






\newpage
\subsection{Compact Operators}


\begin{defn}

\end{defn}























	\newpage
	\subsection{Multilinear Maps}	
	\begin{defn} \ld{43001}
	Let $X_1, \cdots, X_n, Y$ be normed vector spaces and $T: \prod\limits_{i=1}^n X_i \rightarrow Y$ multilinear. Then $T$ is said to be \tbf{bounded} if there exists $C \geq 0$ such that for each $x_1, \cdots, x_n \in X$, $$\|T(x_1, \cdots, x_n)\| \leq C \|x_1\| \cdots \|x_n\|$$
	We define $$L^n (X_1, \dots, X_n; Y) = \bigg\{T: \prod\limits_{i=1}^n X_i \rightarrow Y: T \text{ is multilinear and bounded}\bigg \}$$ 
	If $X_1 = \cdots = X_n = X$, we write $L^n(X,Y)$ in place of $L^n (X, \dots, X; Y) $. If $X_1 = \cdots = X_n = Y =  X$, we write $L^n(X)$. 
	\end{defn}
	
	\begin{note}
	For the remainder of this section we will primarily consider $L^2(X_1, X_2; Y)$ to avoid notational clutter, but all results immediately generalize to $L^n(X_1, \ldots, X_n;Y)$
	\end{note}
	
	\begin{ex} \lex{43002}
	Let $X_1, X_2$ and $Y$ be normed vector spaces and $T: X_1 \times X_2 \rightarrow Y$ bilinear. Then the following are equivalent:
	\begin{enumerate}
			\item $T$ is continuous
			\item $T$ is continuous at $(0,0)$
			\item $T$ is bounded
		\end{enumerate}
	\end{ex}
	
	\begin{proof}\
		\begin{itemize}
		\item $(1) \implies (2)$:\\
		Trivial
		\item  $(2) \implies (3)$:\\ 
		Suppose that $T$ is continuous at $(0, 0)$. For the sake of contradiction, suppose that $T$ is not bounded. Then for each $C \geq 0$, there exist $(x_1, x_2) \in X_1 \times X_2$ such that $\|T(x_1, x_2)\| > C\|x_1\| \|x_2\|$. Hence there exist $(a_n)_{n \in \N} \subset X_1$ and $(b_n)_{n \in \N} \subset X_2$ such that for each $n \in \N$, $ \| T(a_n, b_n) \| > n^2 \|a_n\| \|b_n\|$. Hence for each $n \in \N$, $\|a_n\|$, $\|b_n\| > 0$. Define $$(a'_n)_{n \in \N} \subset X_1$$ and $(b'_n)_{n \in \N} \subset X_2$ by $a_n' = \frac{a_n}{n\|a_n\|}$ and $b_n' = \frac{b_n}{n\|b_n\|}$. Then $(a_n', b_n') \rightarrow (0,0)$. Continuiuty implies that $T(a_n',b_n') \rightarrow 0$. By construction, for each $n \in \N$,
		\begin{align*}
		\|T(a_n',b_n')\| 
		&= \frac{1}{n^2 \|a_n\| \|b_n\|} T(a_n, b_n) \\
		& > \frac{n^2 \|a_n\| \|b_n\|}{n^2 \|a_n\| \|b_n\|} \\
		&= 1
		\end{align*}
		which is a contradiction. So $T$ is bounded.
		\item  $(3) \implies (1)$:\\ 
		Suppose that $T$ is bounded. Then there exists $C > 0$ such that for each $(x_1, x_2) \in X_1 \times X_2$, $\| T(x_1, x_2) \| \leq C\|x_1\| \|x_2\|$. Let $(a, b) \in X_1 \times X_2$ and $(a_n, b_n)_{n \in \N} \subset X_1 \times X_2$. Suppose that $(a_n, b_n) \rightarrow (a,b)$. Then $a_n \rightarrow a$, $b_n \rightarrow b$ and $(a_n)_{n \in \N}$, $(b_n)_{n \in \N}$ are bounded. So there exists $B \geq 0$ such that for each $n \in \N$ $\|b_n\| \leq B$. Hence 
		\begin{align*}
		\|T(a_n,b_n) - T(a,b) \|
		&= \|T(a_n,b_n) - T(a, b_n) + T(a, b_n) - T(a,b) \| \\
		& \leq \|T(a_n,b_n) - T(a, b_n) \| + \|T(a, b_n) - T(a,b) \| \\
		&= \|T(a_n - a,b_n) \| + \|T(a, b_n - b)\| \\
		& \leq C(\|a_n - a\| \|b_n\| + \|a\|\|b_n - b\|) \\
		& \leq C(\|a_n - a\| B + \|a\|\|b_n - b\|) \\
		& \rightarrow 0
		\end{align*}
		Thus $T$ is continuous. 
		\end{itemize}
	\end{proof}
	
	\begin{defn} \ld{43005}
	Let $X_1, X_2$ and $Y$ be normed vector spaces and $T \in L^2(X_1, X_2;Y)$. We define the \tbf{operator norm} on $L^2(X_1, X_2;Y)$, denoted $\|\cdot\|: L^2(X_1, X_2; Y) \rightarrow [0, \infty)$, by  $$\|T\| =  \inf \{C \geq 0: \text{for each } (x_1, x_2) \in X_1 \times X_2\text{, } \|T(x_1, x_2) \|\leq C\|x_1\|\|x_2\|\}$$
	\end{defn}
	
	\begin{ex} \lex{42006}
		Let $X_1, X_2$ and $Y$ be normed vector spaces. If $X_1 \neq \{0\}$ and $ X_2 \neq \{0\}$, then the operater norm on $L(X,Y)$ is given by: 
		\begin{enumerate}
			\item $\|T\| = \sup\limits_{\|x_1\|=1 ,\|x_2\| = 1 }\|T(x_1, x_2)\|$
			\item $\|T\| = \sup\limits_{x_1 \neq 0, x_2 \neq 0 } \|x_1\|^{-1} \|x_2\|^{-1}\|T(x_1, x_2)\|$
			\item $\|T\| = \inf \{C \geq 0: \text{for each } (x_1, x_2) \in X_1 \times X_2\text{, } \|T(x_1, x_2) \|\leq C\|x_1\|\|x_2\|\}$
		\end{enumerate}
	\end{ex}
	
	\begin{proof} Since $X_1 \neq \{0\}$ and $ X_2 \neq \{0\}$, the supremums in (1) and (2) are well defined. Let $T \in L^2(X_1, X_2; Y)$. Bilinearity of $T$ implies that the sets over which the supremums are taken in (1) and (2) are the same. So (1) and (2) are equal.\\
		Now, set 
		$$M = \sup\limits_{\|x_1\|=1 ,\|x_2\| = 1 }\|T(x_1, x_2)\|$$ 
		and 
		$$m = \inf \{C \geq 0: \text{for each } (x_1, x_2) \in X_1 \times X_2\text{, } \|T(x_1, x_2) \|\leq C\|x_1\|\|x_2\|\}$$ 
		Let $(x_1,x_2) \in X_1 \times X_2$. If $\|x_1 \|=0$ or $\|x_2 \|=0$, then $T(x_1, x_2) = 0$ and $\|T(x_1, x_2) \|\leq M \|x_1 \| \|x_2\|$. Suppose that $\|x_1 \| \neq 0$ and $ \|x_2\|\neq 0$. Then 
		\begin{align*}
			\|T(x_1, x_2)\|
			&= \bigg(\big\|T(\|x_1\|^{-1} x_1, \|x_2\|^{-1} x_2 )\big\|\bigg)\|x_1 \| \|x_2\|\\
			& \leq M \|x_1\| \|x_2\|
		\end{align*}
		Hence $M \in \{C \geq 0: \text{ for each }x \in X\text{, } \|Tx \|\leq C \|x \|\}$ and $m \leq M$.
		Let $C \in \{C \geq 0: \text{for each } (x_1, x_2) \in X_1 \times X_2\text{, } \|T(x_1, x_2) \|\leq C\|x_1\|\|x_2\|\}$. Suppose that $\|x_1 \|=1$ and $\|x_2\| = 1$. Then $\| T (x_1, x_2) \| \leq C \|x_1 \| \|x_2 \|= C$. So $M \leq C$. Therefore $M \leq m$. So $M=m$ and the supremum in (1) is the same as the infimum in (3). 
	\end{proof}
	
	\begin{ex}
	Let $X_1, X_2$ and $Y$ be normed vector spaces. Then $\|\cdot\|: L^2(X_1, X_2; Y) \rightarrow [0, \infty)$ is a norm. 
	\end{ex}
	
	\begin{proof}
	
	\end{proof}
	
	\begin{ex} \lex{43003}
	Let $X_1, X_2, Y$ be normed vector spaces and $T_1 \in L(X_1, L(X_2, Y))$. Define $T:X_1 \times X_2 \rightarrow Y$ by $T(x_1, x_2) = T_1(x_1)(x_2)$. Then $T \in L^2(X_1, X_2; Y)$.  
	\end{ex}
	
	\begin{proof}
	It is straightforward to show that $T$ is multilinear. For $x_1 \in X_1$ and $x_2 \in X_2$, 
	\begin{align*}
	\|T(x_1, x_2)\| 
	&= \|T_1(x_1)(x_2)\| \\
	& \leq \|T_1(x_1)\| \|x_2\| \\
	& \leq \|T_1\| \|x_1\| \|x_2\|
	\end{align*}
	So $T \in L^2(X_1, X_2;Y)$.
	\end{proof}
	
	\begin{ex} \lex{43004}
	Let $X_1, X_2, Y$ be normed vector spaces and $T \in L^2(X_1, X_2; Y)$. Define the map $T_1 : X_1 \rightarrow  Y^{X_2}$ by  $T_1(x_1)(\cdot) = T(x_1, \cdot)$. Then $T_1 \in L(X_1, L(X_2, Y))$.
	\end{ex}
	
	\begin{proof}
	Let $x_1 \in X_1$. By definition of $T$, $T_1(x_1)$ is linear. Since $T$ is bounded, there exists $C \geq 0$ such that for each $a_1 \in X_1$, $a_2 \in X_2$, $T(a_1, a_2) \leq C\|a_1\| \|a_2\|$. Then for each $x_2 \in X_2$,
	\begin{align*}
	\| T_1(x_1)(x_2) \|
	&= \|T(x_1, x_2) \| \\
	&\leq (C\|x_1\|) \|x_2\| 
	\end{align*}
	So $T_1(x_1) \in L(X_2,Y)$ with $\|T_1(x_1)\| \leq C\|x_1\|$. Since $x_1 \in X_1$ was arbitrary, $T_1: X_1 \rightarrow L(X,Y)$. By definition of $T$, $T_1$ is linear. The preceeding argument tells us that for each $x_1 \in X_1$, $$\|T_1(x_1)\| \leq C\|x_1\|$$ 
	So $T_1 \in L(X_1, L(X_2, Y))$ with $\|T_1\| \leq C$.  
	\end{proof}	
	
	
	
	
	\begin{ex} \lex{43005}
	Let $X_1, X_2$ be normed vector spaces. Define a map $\phi: L^2(X_1, X_2;Y) \rightarrow L(X_1, L(X_2, Y))$ by $\phi(T)(x_1)(x_2) = T(x_1, x_2)$. Then $T$ is an isometric isomorphism.
	\end{ex}
	
	\begin{proof}
	. 
	\end{proof}
	
	\begin{defn}
	Let $X_1, X_2$ be normed vector spaces, $\phi_1 \in X_1^*$ and $\phi_2 \in X_2^*$. Define $\phi_1 \otimes \phi_2: X_1 \times X_2$ by $\phi_1 \otimes \phi_2(x_1, x_2) = \phi_1(x_1)\phi_2(x_2)$. 
	\end{defn}
	
	\begin{ex}
	Let $X_1, X_2$ be normed vector spaces, $\phi_1 \in X_1^*$ and $\phi_2 \in X_2^*$. Then $\phi_1 \otimes \phi_2 \in L^2(X_1, X_2; \C)$.
	\end{ex}
	
	\begin{proof}
	Clear.
	\end{proof}
	
	\begin{ex}
	Let $X_1, X_2$ be normed vector spaces and $(x_1, x_2) \in X_1 \times X_2$. If for each $\phi_1 \in X_1^*$ and $\phi_2 \in X_2^*$, $\phi_1 \otimes \phi_2 (x_1, x_2) = 0$, then $x_1 = 0$ or $x_2 = 0$. 
	\end{ex}
	
	\begin{proof}
	Suppose that $x_1 \neq 0$ and $x_2 \neq 0$. The previous section implies that there exist $\phi_1 \in X_1^*$ and $\phi_2 \in X_2^*$ such that $\phi_1(x_1) = \|x_1\| \neq 0$ and $\phi_2(x_2) = \|x_2\| \neq 0$. Then 
	\begin{align*}
	\phi_1 \otimes \phi_2 (x_1, x_2) 
	& = \phi_1(x_1) \phi_2(x_2) \\
	& \neq 0
	\end{align*}
	\end{proof}


























\newpage
	\subsection{Banach Algebras}
	
	\begin{defn} \ld{}
		Let $X$ be a Banach space and an associative algebra. Then $X$ is said to be a \tbf{Banach algebra} if for each $S,T \in X$, $\|ST \|\leq \|S \|\|T \|$. 
	\end{defn}
	
	\begin{defn} \ld{}
	Let $X$ be a Banach algebra and $I \in X$. Then $I$ is said to be an \tbf{identity} if for each $T \in X$, $IT = TI = T$. 
	\end{defn}
	
	\begin{defn} \ld{}
	Let $X$ be a Banach algebra. and $I \in X$. Then $I$ is said to be an \tbf{identity} if $I \neq 0$ and for each $T \in X$, $IT = TI = T$.
	\end{defn}
	
	\begin{defn} \ld{}
	Let $X$ be a Banach algebra. Then $X$ is said to be \tbf{unital} if there exists $I \in X$ such that $I$ is an identity.
	\end{defn}	
		
		\begin{ex} \lex{}
		Let $X$ be a unital Banach algebra. Then there exists a unique $I \in X$ such that $I$ is an identity.
		\end{ex} 
		
		\begin{proof}
		Clear.
		\end{proof}
		
		\begin{note}
		We denote the unique identity element by $I$.
		\end{note}
		
		\begin{defn} \ld{}
		Let $X$ be a unital Banach algebra and $T,S \in X$. Then $S$ is said to be an 
		\tbf{inverse} of $T$ if $TS=ST = I$.
		\end{defn}

		\begin{defn} \ld{}
		Let $X$ be a unital Banach algebra and $T \in X$. 			Then $T$ is said to be 
		\tbf{invertible} if there exists $S \in X$ such 		that $S$ is an inverse of $T$.
		\end{defn}
		
		\begin{ex} \lex{}
		Let $X$ be a unital Banach algebra and $T \in X$. If $T$ is invertible, then there exists a unique $S \in X$ such that $S$ is an inverse of $T$.
		\end{ex}
		
		\begin{proof}
		Clear.
		\end{proof}
		
		\begin{note}
		We denote the unique inverse of $T$ by $T^{-1}$.
		\end{note}
		
			
	\begin{ex} \lex{}\tbf{Fundamental Example:} \\
	Let $X$ be a Banach space. Then $GL(X)$ is a unital Banach algebra.
	\end{ex}
	
	\begin{proof}
	Clear.
	\end{proof}
		
		\begin{defn} \ld{}
		Let $X$ be a unital Banach algebra. We define $GL(X) = \{T \in X: T \text{ is invertible}\}$.
		\end{defn}
		
		\begin{ex} \lex{}
		Let $X$ be a unital Banach algebra. Then $GL(X)$ is a group.  
	\end{ex}
	
	\begin{proof}
	Clear.
	\end{proof}
	
	\begin{ex} \lex{}
		Let $X$ be a unital Banach algebra. Then $1 \leq \|I \|$. 
	\end{ex}
	
	\begin{proof}
		Since $I \neq 0$, $\|I \|\neq 0$. By definition, $$\|I \|= \|I I \|\leq \|I \|\|I \|$$ Hence $1 \leq \|I \|$.
	\end{proof}
	
	\begin{ex} \lex{}
		Let $X$ be a Banach algebra. Then mulitplication is continuous. 
	\end{ex}
	
	\begin{proof}
		Let $(S_1,T_1) \in X \times X$ and $\ep > 0$. Choose $\del = \min\{\frac{\ep}{2(\|S_1 \|+ \|T_1 \|+1)}, \frac{\sqrt{\ep}}{\sqrt{2}}\}$. Let $(S_2, T_2) \in X \times X$. Suppose that 
		$$\|(S_1, T_1) - (S_2, T_2) \|= \max \{ \|S_2 -S_2 \|, \|T_1 - T_2 \|\} < \del$$ 
		Then 
		\begin{align*}
			\|S_1T_1 - S_2T_2 \|
			&= \|S_1T_1 - S_2T_1 +S_2T_1 - S_2T_2 \|\\
			& \leq \|S_1 -S_2 \|\|T_1 \|+ \|S_2 \|\|T_1 - T_2 \|\\
			& \leq \|S_1 -S_2 \|\|T_1 \|+ \big( \|S_1-S_2 \|+ \|S_1 \|\big) \|T_1 - T_2 \|\\
			& \leq \del \|T_1 \|+(\del + \|S_1 \|) \del \\
			&= \del (\|S_1 \|+ \|T_1 \|) + \del^2 \\
			& < \frac{\ep}{2} + \frac{\ep}{2}\\
			&= \ep
		\end{align*}
	\end{proof}
	
	
	\begin{ex} \lex{}
		Let $X$ be a unital Banach algebra. Then 
		\begin{enumerate}
			\item For each $T \in X$, if $\|I- T \|< 1$, then $T \in GL(X)$ and $$T^{-1} = \sum_{n=0}^{\infty}(I-T)^n$$
			\item For each $S,T \in X$, if $S \in GL(X)$  and $\|S-T \|< \|S^{-1} \|^{-1}$, then $T \in GL(X)$. 
			\item $GL(X)$ is open.
		\end{enumerate}
	\end{ex}
	
	\begin{proof}\
		\begin{enumerate}
			\item Let $T \in X$. Suppose that $\|I-T \|< 1$. Then $$\sum_{n=0}^{\infty} \|(I -T)^n \| \leq \sum_{n=0}^{\infty} \|I -T \|^{n} < \infty$$ Since $X$ is a complete, $\sum\limits_{n=0}^{\infty}(I-T)^n$ converges in $X$.\\
			Define $(S_k)_{k=0}^{\infty} \subset X$ and $S \in X$ by $S_k = \sum\limits_{n=0}^{k} (I-T)^n$ and \\ $S = \sum\limits_{n=0}^{\infty}(I-T)^n$. Then for each $k \in \N$,
			\begin{align*}
				S_k T
				&= S_k - S_k(I-T) \\
				&= (I-T)^0 - (I-T)^{k+1} \\
				&= I - (I-T)^{k+1}
			\end{align*}
			and $\|S_kT - I \|\leq \|I-T \|^{k+1}$. Since multiplication on Banach algebras is continuous, we have that $$ST = (\lim_{k \rightarrow \infty} S_k)T = \lim\limits_{k \rightarrow \infty}S_kT = I$$
			Similarly $TS = I$. Thus $T \in GL(X)$ and $T^{-1} = S \in X$. \vspace{.5cm}\\
			\item  Let $S, T \in X$. Suppose that $S \in GL(X)$ and $\|S-T \|< \|S^{-1} \|^{-1}$. Then 
			\begin{align*}
				\|I - S^{-1}T \|
				& = \|S^{-1}(S - T) \|\\
				& \leq \|S^{-1} \|\|S -T \|\\
				&< 1
			\end{align*}
			So $S^{-1}T \in GL(X)$. Thus $T = S (S^{-1}T) \in GL(X)$. \vspace{.5cm}\\
			\item Let $T \in GL(X)$. Choose $\del = \|T^{-1}\|^{-1}$. By (2), $B(T, \del) \subset GL(X)$.
		\end{enumerate}
	\end{proof}	
	
	
	
	
	











	
	
	
	
	
	
	
	
	
	
	
	
	
	
	
	
	\newpage
	\section{Hilbert Spaces}
	
	\subsection{Introduction}
	
	\begin{defn} \ld{}
		Let $H$ be a vector space and $\l \cdot, \cdot \r: H \rightarrow \C$. Then $\l \cdot, \cdot \r$ is said to be an \tbf{inner product} on $H$ if for each $x,y,z \in H$and $c \in \C$
		\begin{enumerate}
			\item $\l x , y + cz\r = \l x , y \r + c\l x , z\r $
			\item $\l x , y \r = \l y , x\r^*$
			\item $\l x , x \r \geq 0$
			\item if $\l x ,x \r = 0$, then $x = 0$.  
		\end{enumerate}
	\end{defn}
	
	\begin{note}
	In mathematics, inner products are conventionally defined to be linear in the first argument. However, in my opinion, the convention in physics of defining inner products to be linear in the second argument makes more sense.
	\end{note}
	 
	\begin{ex} \lex{}
	Let $H$ be an inner product space, $(x_j)_{j =1}^n$, $(y_j)_{j =1}^n \subset H$ and $(\al_j)_{j=1}^n$, $(\be_j)_{j=1}^n \subset \C$. Then $$\bigg \l \sum_{i=1}^n \al_i x_i , \sum_{j=1}^n \be_j y_j \bigg \r = \sum_{i=1}^n \sum_{j=1}^n \al_i^*\be_j \l x_i , y_j \r $$
\end{ex}

\begin{proof}
Clear.
\end{proof}

\begin{defn} \ld{}
Let $H$ be an inner product space. Define the \tbf{induced norm}, denoted $\|\cdot \|: H \rightarrow \C$, by $$\|x\| = \l x, x\r^{\frac{1}{2}}$$
\end{defn}

\begin{ex} \lex{} \tbf{Cauchy-Schwarz Inequality}\\
Let $H$ be an inner product space. Then for each $x,y \in H$, $| \l x, y\r | \leq \|x\| \| y\|$ and $| \l x, y\r | = \|x\| \| y\|$ iff $x \in \spn(y)$. \\
\tbf{Hint:} For $x, y \in H$, put $z = \sgn\l x, y \r^*y$ and Consider $f: \R \rightarrow \Rg$ defined by $f(t) = \|x - tz\|^2$
\end{ex}

\begin{proof}
Let $x,y \in H$. If $y = 0$, then the claim holds trivially. Suppose that $y \neq 0$. Put $z = \sgn\l x, y \r^*y$. So $\l x, z\r = |\l x,y \r |$ and $\|z\| = \|y\|$. Define $f: \R \rightarrow \Rg$ by $$f(t) = \|x - tz\|^2$$. Then for each $t \in \R$, 
\begin{align*}
0 
& \leq f(t) \\
&=  \|x - tz\|^2 \\
&= \|x\|^2 + |t|^2\|z\|^2 - 2 \Re(t \l x,z \r) \\
&= \|x\|^2 + t^2\|y\|^2 - 2 t |\l x,y \r| \\
\end{align*} 
Thus $f$ is a quadratic with a minimum at $t_0 = \frac{|\l x, y \r|}{\|y\|^2}$. Hence 
\begin{align*}
0 
&\leq f(t_0) \\
&= \|x\|^2 +  \frac{|\l x, y \r|}{\|y\|^2} - 2\frac{|\l x, y \r|}{\|y\|^2} \\
& = \|x\|^2 -  \frac{|\l x, y \r|}{\|y\|^2}
\end{align*}
Which implies that $$| \l x, y\r |^2 \leq \|x\|^2 \| y\|^2$$ and hence the claim holds. Clearly if $x \in \spn(y)$, then equality holds. Conversely, if equality holds, then $x-z = 0$ which implies that $x \in spn(y)$.
\end{proof}

\begin{ex} \lex{}
Let $H$ be an inner product space. Then the induced norm, $\| \cdot\|: H \rightarrow \C$, is a norm. 
\end{ex}

\begin{proof}Let $x,y \in H$ and $c \in \C$. Then
\begin{enumerate}
\item By definition, if $\|x\| = 0$, then $\l x, x \r =0$, which implies that $x =0$.
\item Note that 
\begin{align*}
\| cx \|^2 
&= \l cx, cx \r \\
&= c*c \l x, x\r \\
&= |c|^2\| x \|^2
\end{align*}
So $\| cx \| = |c|\|x\|$
\item The Cauchy-Schwarz inequality implies that
\begin{align*}
\|x + y\|^2 
&= \|x\|^2 + \|y\|^2 + 2 \Re(\l x, y\r) \\
& \leq \|x\|^2 + \|y\|^2 + 2 |\l x, y\r | \\
& \leq \|x\|^2 + \|y\|^2 + 2 \|x\| \|y\| \\
&= (\|x\| + \|y\|)^2
\end{align*}
Hence $\|x + y\| \leq \|x\| + \|y\|$.
\end{enumerate}
\end{proof}

\begin{defn} \ld{}
	Let $H$ be an inner product space, $x, y \in H$ and $S \subset H$. Then
	\begin{enumerate}
	\item $x$ and $y$ are said to be \tbf{orthogonal} if $\l x,y\r = 0$. 
	\item $S$ is said to be \tbf{orthogonal} if for each $x,y \in S$, $x,y$ are orthogonal. 
	\end{enumerate}
\end{defn}

\begin{ex} \lex{}\tbf{(Pythagorean theorem):}\\
	Let $H$ be an inner product space and $(x_j)_{j =1}^n \subset H$ an orthogonal set. Then $$\bigg \|\sum\limits_{j = 1}^n x_j  \bigg \|^2 = \sum\limits_{j =1}^n \|x_j \|^2$$
\end{ex}

\begin{proof}
	We have that
	\begin{align*}
		\bigg \| \sum\limits_{j = 1}^n  x_j\bigg \|^2
		&= \bigg \l \sum\limits_{i =1}^n x_i , \sum\limits_{j =1}^n x_j \bigg \r \\
		&= \sum\limits_{i =1}^n \sum\limits_{j =1}^n \l x_j , x_j \r \\
		&= \sum\limits_{j =1}^n \l x_j , x_j \r \\
		&= \sum\limits_{j =1}^n \| x_j \|^2
	\end{align*}
\end{proof}

\begin{ex} \lex{}
	Let $H$ be an inner product space and $S \subset H$. Suppose that $0 \not \in S$. If $S$ is orthogonal, then $S$ is linearly independent.
\end{ex}

\begin{proof}
	Let $x_1, \cdots, x_n \in S$ and $c_1, \cdots, c_n \in \C$. Suppose that $\sum\limits_{j =1}^n c_j x_j = 0 $. Since $(c_j x_j)_{j=1}^n$ is orthogonal, the Pythagorean theorem implies that 
	\begin{align*}
		0
		&= \bigg \| \sum_{i=1}^n c_i x_i \bigg \| \\
		&= \sum_{j=1}^n  |c_j|^2 \| x_j\| 
	\end{align*}
	So for each $j \in \{ 1 , \cdots, n\}$, $c_j = 0$ and $S$ is linearly independent.
\end{proof}

\begin{defn} \ld{}
	Let $H$ be an inner product space and $S \subset H$. Then $S$ is said to be \tbf{orthonormal} if $S$ is orthogonal and for each $x \in S$, $\|x \| = 1$.
\end{defn}

\begin{ex} \lex{}\tbf{Bessel's Inequality:}\\
Let $H$ be an inner product space and $S \subset H$. If $S$ is orthonormal, then for each $x \in H$, $$\sum_{u \in S} | \l u, x \r |^2  \leq \|x\|$$
and in particular, $\{u \in S: \l u, x\r \neq 0\}$ is countable.
\end{ex}

\begin{proof}
Suppose that $S$ is orthonormal. Let $x \in H$ and $F \subset S$ finite. Then the Pythagorean theorem implies that  
\begin{align*}
0 
& \leq \bigg \|x - \sum_{u \in F} \l u, x \r u \bigg \|^2 \\
&= \|x\|^2 + \bigg \| \sum_{u \in F} \l u, x \r u \bigg \|^2 - 2 \Re \bigg \l x, \sum_{u \in F} \l u, x \r u \bigg \r  \\
&= \|x\|^2 +  \sum_{u \in F} |\l u, x \r|^2 \|u\|^2 - 2 \sum_{u \in F} | \l u, x \r|^2  \\
&= \|x\|^2 -  \sum_{u \in F} |\l u, x \r|^2 
\end{align*}
So $$\sum_{u \in F} | \l u, x \r |^2  \leq \|x\|$$
By definition of the sum, $$\sum_{u \in S} | \l u, x \r |^2  \leq \|x\|$$
Basic integration theory then tells us that $\{u \in S: \l u, x\r \neq 0\}$ is countable.
\end{proof}

\begin{defn} \ld{}
	Let $H$ be an inner product space. Then $H$ is said to be a \tbf{Hilbert space} if $H$ is a complete with respect to the induced norm on $H$.
\end{defn}

\begin{ex} \lex{}
Let $H$ be a Hilbert space and $S \subset H$. Suppose that  $S$ is orthonormal. Then the followong are equivalent: 
\begin{enumerate}
\item For each $x \in H$, if for each $u \in S$, $\l u, x \r = 0$, then $x =0$.
\item For each $x \in H$, there exist $(u_j)_{j\in \N} \subset S$ such that $x = \sum\limits_{j \in \N} \l u_j, x\r u_j$ and for each $u \not \in (u_j)_{j\in \N}$, $\l u, x\r =0$.
\item For each $x \in H$, $\|x\|^2 = \sum\limits_{u \in S} | \l u, x \r |^2$.
\end{enumerate}
\end{ex}

\begin{proof}\
\begin{itemize}
		\item $(1) \implies (2)$:\\
		Suppose that for each $x \in H$, if for each $u \in S$, $\l u, x \r = 0$, then $x =0$. Let $x \in H$. Put $S_{*} = \{u \in S: \l u, x \r \neq 0 \}$. The previous exercise implies that $S_{*}$ is countable. Write $S_* = (u_j)_{j=1}^n$. The previous exercise tells us that $\sum\limits_{j \in \N} |\l u_j, x \r|^2 \leq \|x\|^2$ and hence converges. Thus for $\ep >0$, there exist $N \in \N$ such that for each  $m,n \in \N$, $m, n \geq N$ implies that if $m < n$, then $$\sum_{m+1}^{n} |\l u_j, x \r |^2 < \ep$$
		Define $(y_n)_{n \in \N} \subset H$ by $$y_n = \sum_{j=1}^{n} \l u_j, x \r u_j$$ 
		Then for each $m,n \in \N$, $m, n \geq N$ implies that if $m < n$, then 
		\begin{align*}
		\|y_n - y_m\|^2 
		& = \bigg \|\sum_{1}^{n} \l u_j, x \r u_j  - \sum_{1}^{m} \l u_j, x \r u_j  \bigg \|^2 \\
		&= \bigg \|\sum_{m+1}^{n} \l u_j, x \r u_j \bigg \|^2 \\
		&= \sum_{m+1}^{n} |\l u_j, x \r |^2 \\
		& < \ep
		\end{align*}
		So $(y_n)_{n \in \N}$ is Cauchy. Since $H$ is complete, there exists $y \in H$ such that $y_n \rightarrow y$. By definition, $$y = \sum\limits_{j \in \N}\l u_j, x \r u_j $$
		Continuity of $\l \cdot, \cdot\r: H \times H \rightarrow \C$ implies that 
		\begin{enumerate}
		\item for each $u \in S \setminus  S_*$, 
		\begin{align*}
		\l u, x - y\r 
		&= \l u, x \r  - \l u, y \r \\
		&=  \l u, x \r - \limn \l u, y_n\r \\
		&=  \l u, x \r - \limn \sum_{j=1}^n \l u_j, x\r \l u, u_j \r \\
		&= 0 - 0 \\
		&=0
		\end{align*}
		\item for each $k \in \N$, 
		\begin{align*}
		\l u_k, x - y\r 
		&= \l u_k, x \r  - \l u_k, y \r \\
		&= \l u_k, x \r - \limn \l u_k, y_n\r \\
		&= \l u_k, x \r - \limn \sum_{j=1}^n \l u_j, x\r \l u_k, u_j \r \\
		&= \l u_k, x \r  - \l u_k, x\r \\
		&= 0
		\end{align*}
		\end{enumerate}
		So for each $u \in S$, $\l u, x-y\r =0$. By assumption, $x-y = 0$ and hence 
		$$x = \sum\limits_{j \in \N}\l u_j, x \r u_j$$
		
		\item $(2) \implies (3)$:\\
		Suppose that for each $x \in H$, there exist $(u_j)_{j\in \N} \subset S$ such that $x = \sum\limits_{j \in \N} \l u_j, x\r u_j$ and for each $u \not \in (u_j)_{j\in \N}$, $\l u, x\r =0$. Then continuity of $\|\cdot\|:H \rightarrow \Rg$ implies that
		\begin{align*}
		\|x\|^2 
		&= \bigg \|  \limn \sum_{j=1}^n \l u_j, x\r u_j \bigg \|^2 \\
		&= \limn \bigg \|\sum_{j=1}^n \l u_j, x\r u_j \bigg \|^2 \\
		&= \limn \sum_{j=1}^n |\l u_j, x\r |^2 \\
		&= \sum_{j \in \N}|\l u_j, x\r |^2  \\
		&= \sum_{u \in S}|\l u, x\r |^2 
		\end{align*}
		\item $(3) \implies (4)$:\\
		Suppose that for each $x \in H$, $\|x\|^2 = \sum\limits_{u \in S} | \l u, x \r |^2$. Let $x \in H$. Suppose that for each $u \in S$, $\l u,x \r = 0$. Then 
		\begin{align*}
		\| x\|^2 
		&= \sum_{u \in S}  | \l u, x \r |^2 \\
		&= 0
		\end{align*}
		So $x =0$
\end{itemize}
\end{proof}

\begin{defn} \ld{}
	Let $H$ be a Hilbert space and $S \subset H$. Then $S$ is said to be an \tbf{orthonormal basis of $H$} if 
	\begin{enumerate}
	\item $S$ is orthonormal
	\item for each $x \in H$, if for each $u \in S$, $\l u, x \r = 0$, then $x =0$
	\end{enumerate}
\end{defn}





\newpage

\subsection{Operators and Functionals}

\begin{defn} \ld{}\tbf{(Adjoint of an Operator):} \\
	Let $H$ be a Hilbert space and $A,B \in L(H)$. Then $B$ is said to be the \tbf{adjoint} of $A$ if for each $x_1$, $x_2 \in H$, $$\l x_1 , Ax_2 \r = \l B x_1 , x_2 \r$$  In this case, we write $$B = A^{*}$$
\end{defn}

\begin{note}
	In physics, the adjoint of $A$ is typically denoted by $A^{\dagger}$.
\end{note}

\begin{ex} \lex{}
	Let $H$ be a Hilbert space, $A, B \in L(H)$ and $\lam \in \C$, then \begin{enumerate}
		\item $(A^{*})^{*} = A$
		\item $(A + B)^{*} = A^{*} + B^{*}$
		\item $(AB)^{*} = B^{*}A^{*}$
		\item $(\lam A)^{*} = \lam^*A^{*}$
		\item $A$ and $B$ commute iff $A^{*}$ and $B^{*}$ commute.
	\end{enumerate}
\end{ex}

\begin{proof} Let $x_1$, $x_2 \in H$. Then
	\begin{enumerate}
		\item 
		\begin{align*}
			\l A x_1 , x_2 \r
			&= \l x_2 , A x_1 \r^*\\
			&= \l A^{*}x_2 ,  x_1 \r^* \hspace{.5cm} \text{(by definition)}\\
			&= \l  x_1 , A^{*}x_2 \r
		\end{align*}
		\item 
		\begin{align*}
			\l x_1 , (A+B) x_2 \r 
			&= \l x_1 , A x_2 \r + \l x_1 , B x_2 \r \\
			&= \l A^{*} x_1 , x_2 \r + \l B^{*} x_1 , x_2 \r \\
			&= \l (A^{*} + B^{*})  x_1 , x_2 \r  \\
		\end{align*}
		\item 
		\begin{align*}
			\l x_1 , AB x_2 \r  
			&= \l A^{*}x_1 , B x_2 \r \\
			&= \l B^{*} A^{*} x_1 , x_2 \r 
		\end{align*}
		\item 
		\begin{align*}
			\l x_1 , \lam A x_2 \r 
			&= \lam \l x_1 , A x_2 \r \\
			&= \lam \l A^{*}x_1 , x_2 \r \\
			&= \l \lam^* A^{*} x_1 , x_2 \r 
		\end{align*}
		\item If $A$ and $B$ commute, then 
		\begin{align*}
			A^{*}B^{*}
			&= (BA)^{*} \\
			&= (AB)^{*} \\
			&= B^{*}A^{*}
		\end{align*}
		Conversely, if $A^{*}$ and $B^{*}$ commute then 
		\begin{align*}
			AB
			&= (B^{*}A^{*})^{*} \\
			&= (A^{*}B^{*})^{*} \\
			&= BA
		\end{align*}
	\end{enumerate}
\end{proof}

\begin{defn} \ld{}
	Let $H$ be a Hilbert space and $Q \in L(H)$. Then $Q$ is said to be \tbf{self-adjoint} if $$Q = Q^{*}$$
\end{defn}

\begin{ex} \lex{}
	Let $H$ be a Hilbert space and $Q \in L(H)$. If $Q$ is a self-adjoint then 
	\begin{enumerate}
		\item the eigenvalues of $Q$ are real.
		\item the eigenvectors of $Q$ corresponding to distinct eigenvalues are orthogonal.
	\end{enumerate}
\end{ex}

\begin{proof}
	Suppose that $Q$ is self-adjoint.
	\begin{enumerate}
		\item Let $\lam$ be an eigenvalue of $Q$ with corresponding eigenvector $x$. Then 
		\begin{align*}
			\lam \l x , x\r
			&= \l x , Q x\r \\
			&= \l Q x , x\r \\
			&= \lam^* \l x , x\r
		\end{align*}
		Thus $\lam = \lam^*$ and is real
		
		\item Let $\lam_1$ and $\lam_2$ be eigenvalues of $Q$ with corresponding eigenvectors $x_1$ and $x_2$. Suppose that $\lam_1 \neq \lam_2$. Then 
		\begin{align*}
			\lam_2 \l x_1 ,  x_2\r
			&= \l x_1 , Q x_2\r\\
			&= \l Q x_1 ,  x_2\r\\
			&= \lam_1 \l x_1 ,  x_2\r
		\end{align*}
		So $(\lam_2 - \lam_1)\l x_1 ,  x_2\r = 0$. Which implies that $\l x_1 ,  x_2\r=0$
	\end{enumerate}
\end{proof}

\begin{ex} \lex{}
	Let $H$ be a Hilbert space, $A, B \in L(H)$ and $ \lam \in \R$. Suppose that $A, B$ are self-adjoint. If $A$ and $B$ commute and then $\lam AB$ is self-adjoint.
\end{ex}

\begin{proof}
	\begin{align*}
		(\lam AB)^{*}
		&= \lam^* (AB)^{*} \\
		&= \lam B^{*} A^{*} \\
		&= \lam B A \\
		&= \lam AB
	\end{align*}
\end{proof}

\begin{defn} \ld{}\tbf{(Adjoint of a Vector):} \\
	Let $H$ be a Hilbert space and $x \in H$. We define the \tbf{adjoint} of $x$, denoted $x^* \in H^*$, by $x^* y = \l x, y \r$. 
\end{defn}

\begin{note}
	In mathematics, where linearity of the inner product is in the first argument, $x^{*}$ is typically referred to by $u_{x} \in H^{*} $ where $u_{x}(y) = \l y, x\r$. In physics, where the inner product with linearity in the second argument, $x^{*} \phi$ is usually written in the so-called ``bra-ket" notation as $\l x | \phi \r$ which works smoothly since it aligns with the linearity of $u_{x}(\phi_1 + \lam \phi_2)$ and the conjugate-linearity of $u_{x_1 + \lam x_2}(\phi)$. In this way, it generalizes the notation for $\l x, y\r = x^T y$ for $\R^n$ to $\l x, y\r = x^*y$ for $\C^n$. 
\end{note}

\begin{ex} \lex{}
	Let $H$ be a Hilbert space, $x, y \in H$ and $\lam \in \C$. Then 
	\begin{enumerate}
		\item $(x + y)^* =  x^* + y^*$
		\item $(\lam x)^* = \lam^* x^*$
	\end{enumerate}
\end{ex}

\begin{proof}
	Clear.
\end{proof}

\begin{defn} \ld{}
	Let $H$ be a Hilbert space, $x, y \in H$ and $A \in L(H)$. We define 
	\begin{enumerate}
		\item $x^* A \in H^*$ by $(x^*A) y = x^*(A y)$
		\item $x y^* \in L(H)$ by $(x y^*) z = (y^*z) x$
	\end{enumerate}
\end{defn}

\begin{ex} \lex{}
	Let $H$ be a Hilbert space, $A \in L(H)$ and $x \in H$. Then $$(A x)^*= x^*A^*$$
\end{ex}

\begin{proof}
	Let $y \in H$. Then 
	\begin{align*}
		(Ax)^*	y 
		&= \l Ax, y \r \\
		&= \l x, A^* y \r \\
		&= x^*A^* y
	\end{align*}
\end{proof}

\begin{defn} \ld{}\tbf{(Commutator):} \\
	Let $H$ be a Hilbert space and $A, B \in L(H)$. The \tbf{commutator} of $A$ and $B$, denoted $[A,B]$, is defined by $$[A,B] = AB - BA$$
\end{defn}

\begin{ex} \lex{}
	Let $H$ be a Hilbert space and $A,B, C \in L(H)$. Then 
	\begin{enumerate}
		\item $[AB,C] = A[B,C] + [A,C]B$
		\item $[A, BC] = B[A, C] + [A,B]C$
	\end{enumerate}
\end{ex}

\begin{proof} \
	\begin{enumerate}
		\item 
		\begin{align*}
			[AB,C]
			&= ABC - CAB\\
			&= ABC - ACB + ACB -CAB\\
			&= A(BC - CB) + (AC-CA)B\\
			&= A[B,C] + [A,C]B
		\end{align*}
		\item Similar to (1).
	\end{enumerate}
\end{proof}

%\begin{defn} \ld{}\tbf{(Tensor Product):} \\ 
%	Let $H_1, H_2$ be Hilbert spaces. Define $$\otimes: H_1 \times H_2 \rightarrow \C^{H_1 \times H_2} ,\hspace{.5cm} (x, \phi) \mapsto x \otimes \phi$$ by $$x \otimes \phi(x,y) = \l x, x  \r \l y, \phi\r$$ 
%\end{defn}
%
%\begin{note}
%	For the remainder of this section, we assume that $H_1$ and $H_2$ are Hilbert spaces.
%\end{note}
%
%\begin{ex} \lex{}
%	We have that $\otimes: H_1 \times H_2 \rightarrow \C^{H_1 \times H_2}$ is bilinear.
%\end{ex}
%
%\begin{proof}
%	Clear.
%\end{proof}
%
%\begin{defn} \ld{}
%	Define $T(H_1, H_2) = \spn \{x \otimes \phi: x \in H_1 \text{ and } \phi \in H_2\}$ and define $\l \cdot, \cdot \r : T(H_1, H_2) \rightarrow \C $ by $$\l x_1 \otimes \phi_1 , x_2 \otimes \phi_2 \r = \l x_1, x_2 \r \l \phi_1 ,  \phi_1\r$$ and extending sesquilinearly so that $$ \l \sum_{i=1}^m \al_i x_i \otimes \phi_i , \sum_{j=1}^n \beta_j \Xi_j \otimes \Gam_j\r = \sum_{i=1}^m \sum_{j=1}^n \al_i^* \beta_j \l x_i \otimes \phi_i,  \Xi_j \otimes \Gam_j \r$$
%\end{defn}
%
%\begin{ex} \lex{}
%	We have that  $\l \cdot, \cdot \r : T(H_1, H_2) \rightarrow \C $ is an inner product on $T(H_1, H_2)$.
%\end{ex}
%
%\begin{proof} 
%	Clear.
%\end{proof}
%
%\begin{defn} \ld{}
%	Define $H_1 \otimes H_2$ to be the completion of $T(H_1, H_2)$.
%\end{defn}
%	
%
%	\begin{ex} \lex{}
%		Let $H_1, H_2$ be Hilbert spaces. If $(x_j)_{j\in \N}$ is an orthonormal basis for $H_1$ and $(\phi_j)_{j \in \N}$ is an orthonormal basis for $H_2$, then $(x_i \otimes \phi_j)_{i,j \in \N}$ is an orthonormal basis for $H_1 \otimes H_2$. 
%	\end{ex}
%
%	\begin{proof}
%		Since 
%		\begin{align*}
%			\l x_{i_1} \otimes \phi_{j_1} , x_{i_2} \otimes \phi_{j_2} \r 
%			&= \l x_{i_1} , x_{i_2} \r \l \phi_{j_1} , \phi_{j_2} \r \\
%			&= \del_{i_1, i_2} \del_{j_1, j_2}
%		\end{align*}
%		we have that $(x_i \otimes \phi_j)_{i,j \in \N}$ is orthonormal. Let $x = \sum\limits_{i \in \N} a_i x_i \in H_1$ and $\phi = \sum\limits_{j \in \N} b_j \phi_j \in H_2$. Then $\phi \otimes x = \sum\limits_{i \in \N} \sum\limits_{j \in \N} a_ib_j x_i \otimes \phi_j$, so $(x_i \otimes \phi_j)_{i,j \in \N}$ is a dense subset of $T(H_1, H_2)$, which is dense in $H_1 \otimes H_2$. Hence $(x_i \otimes \phi_j)_{i,j \in \N}$ is dense in $H_1 \otimes H_2$ and is a basis.
%	\end{proof}
%
%	\begin{note}
%		If $H_1$ and $H_2$ are function spaces over sets $S_1$ and $S_2$ respectively, then $H_1 \otimes H_2$ can be identified with the function space over $S_1 \times S_2$ given by  $f_1 \otimes f_2(s_1, s_2) = f_1(s_1)f_2(s_2)$.
%	\end{note}
%
%	\begin{defn} \ld{}
%		Let $A$ and $B$ be operators on $H_1$ and $H_2$ respectively. We define the operator $A \otimes B$ on $H_1 \otimes H_2$ by setting $$A \otimes B (x \otimes \phi) = A x \otimes B \phi$$ and extending linearly to $T(H_1, H_2)$ and then extending continuously to $H_1 \otimes H_2$
%	\end{defn}



















\newpage
	\subsection{Tensor Products}
	
	\begin{note}
	This section assumes familiarity with the algebraic tensor product of two vector spaces. See section ??? of \cite{algebra} for details. 
	\end{note}	
	
	\begin{defn}
	Let $X, Y$ and $Z$ be Banach spaces and $\phi \in L^2(X,Y ; Z) $. Then $(Z, \phi)$ is said to be a \tbf{tensor product} of $X$ with $Y$ if 
	\begin{enumerate}
	\item $\spn \phi(X \times Y)$ is dense in $Z$
	\item for each Banach space $W$ and $\psi \in L^2(X,Y;W)$, there exists a unique $\psi' \in L(Z, W)$ such that $\psi' \circ \phi = \psi$, i.e. such that the following diagram commutes: 
	\[ \begin{tikzcd}
	X \times Y \arrow[r, "\phi"] \arrow[dr, "\psi"'] 	
	& Z  \arrow[d, "\psi'"] \\
	& W 
\end{tikzcd}
	\]
	\end{enumerate}
	If $(Z, \phi)$ is a tensor product of $X$ with $Y$. We often write $Z = X \otimes Y$ and for each $x\in X$, $y \in Y$, we often write $\phi(x,y) = x \otimes y$.
	\end{defn}	
	
	\begin{ex}
	Let $X$ and $Y$ be Banach spaces, $U \subset X$ and $V \subset Y$. Set $W = \{u \otimes v: u \in U \text{ and } v \in V\} \subset X \otimes Y$. If $U$ and $V$ are linearly independent, then $W$ is linearly independent.\\
	\tbf{Hint:} For $\phi \in X^*$, $\psi \in Y^*$, define $T \in L^2(X, Y; \C)$ by $T(x,y) = \phi(x)\psi(y)$.
	\end{ex}	
	
	\begin{proof}
	Let $w = \sum\limits_{u \in U} \sum\limits_{v \in V}\lam_{u,v} u \otimes v$. Suppose that $w = 0$. Let $\phi \in X^*$ and $\psi \in Y^*$. Define $T \in L^2(X, Y; \C)$ by $T(x,y) = \phi(x)\psi(y)$. By definition of the tensor product, there exists a unique $T' \in L(X \otimes Y, \C)$ such that for each $x \in X$ and $y \in Y$, $T'(x \otimes y) = T(x,y)$. Then 
	\begin{align*}
	0 
	&= T'(w) \\
	&= T'(\sum\limits_{u \in U} \sum\limits_{v \in V}\lam_{u,v} u \otimes v) \\
	&= \sum\limits_{u \in U} \sum\limits_{v \in V}\lam_{u,v} T'(u \otimes v) \\
	&= \sum\limits_{u \in U} \sum\limits_{v \in V}\lam_{u,v} T(u, v) \\
	&= \sum\limits_{u \in U} \sum\limits_{v \in V}\lam_{u,v} \phi(u)\psi(v) \\
	&= \phi\bigg( \sum\limits_{u \in U} \sum\limits_{v \in V}\lam_{u,v}  \psi(v) u \bigg)
	\end{align*}
	Since $\phi \in X^*$ is arbitary, a previous exercise in the section on linear functionals implies that 
	\begin{align*}
	0 
	&= \sum\limits_{u \in U} \sum\limits_{v \in V}\lam_{u,v}  \psi(v) u \\
	&= \sum\limits_{u \in U} \bigg (\sum\limits_{v \in V}\lam_{u,v}  \psi(v)  \bigg) u
	\end{align*}
	Linear independence of $U$ implies that for each $u \in U$, 
	\begin{align*}
	0 
	&= \sum\limits_{v \in V}\lam_{u,v}  \psi(v) \\
	&= \psi \bigg( \sum\limits_{v \in V}\lam_{u,v}  v \bigg )
	\end{align*}
	Since $\psi \in Y^*$ is arbitary, for each $u \in U$, $$\sum\limits_{v \in V}\lam_{u,v}  v = 0$$
	Linear independence of $V$ implies that for each $u \in U, v \in V$, $\lam_{u,v} = 0$. Hence $W$ is linearly independent. 
	\end{proof}
	
	\begin{ex} \tbf{Uniqueness:}\\
	Let $X, Y$ and $Z$ be Banach spaces and $\phi \in L^2(X,Y ; Z)$. Suppose that $(Z, \phi)$ is a tensor product of $X$ with $Y$. Then $(Z, \phi)$ is unique up to isomorphism. 
	\end{ex}
	
	\begin{proof}
	Let $W$ be a Banach space and $\psi \in  L^2(X,Y ; W)$. Suppose that $(W, \psi)$ is a tensor product of $X$ with $Y$. Since $(Z, \phi)$ is a tensor product of $X$ with $Y$, there exists a unique $\psi' \in L(Z, W)$ such that $\psi' \circ \phi = \psi$. Since $(W, \psi)$ is a tensor product of $X$ with $Y$, there exists a unique $\phi' \in L(W,Z)$ such that $\phi' \circ \psi = \phi$. Thus the following diagram commutes: 
	\[ \begin{tikzcd}
	& W  \arrow[d, "\phi'"] \\
	X \times Y \arrow[r, "\phi"] \arrow[dr, "\psi"'] \arrow[ur, "\psi"]	
	& Z  \arrow[d, "\psi'"] \\
	& W 
\end{tikzcd}
	\]
	On the other hand, since $(W, \psi)$ is a tensor product of $X$ with $Y$, there exists a unique $\Psi \in L(W)$ such that $\Psi \circ \psi = \psi$. Thus the following diagram commutes:  
	\[ \begin{tikzcd}
	X \times Y \arrow[r, "\psi"] \arrow[dr, "\psi"'] 	
	& W  \arrow[d, "\Psi"] \\
	& W 
\end{tikzcd}
	\]
	Since $I_W \in L(W)$ and $I_W \circ \psi = \psi$, uniqueness of $\Psi$ implies that $\Psi = I_W$. 
	From the first diagram, we see that $\psi' \circ \phi'$ satisfies $(\psi' \circ \phi') \circ \psi  = \psi$. Since $\psi' \circ \phi' \in L(W)$, uniqueness of $\Psi$ implies that $\Psi = \psi' \circ \phi'$. Thus $\psi' \circ \phi'  = I_W$. \\
	Similarly, we could have initially considered the following diagram: 
	\[ \begin{tikzcd}
	& Z  \arrow[d, "\psi'"] \\
	X \times Y \arrow[r, "\psi"] \arrow[dr, "\phi"'] \arrow[ur, "\phi"]	
	& W  \arrow[d, "\phi'"] \\
	& Z
\end{tikzcd}
	\]
	Playing a similar game, we could use the fact that there exists a unique $\Phi \in L(Z)$ such that $\Phi \circ \phi = \phi$ to obtain the following diagram:
	\[ \begin{tikzcd}
	X \times Y \arrow[r, "\phi"] \arrow[dr, "\phi"'] 	
	& Z  \arrow[d, "\Phi"] \\
	& Z 
\end{tikzcd}
	\]
	As before, uniqueness enables us to conclude that $\phi' \circ \psi' = I_Z$. Thus $\psi'$ and $\phi'$ are isomorphisms and $Z \cong W$.
	\end{proof}
	
	
	\begin{note}
	The following definitions and exercises will cover the explicit construction of a tensor product of Banach spaces.
	\end{note}	
	
	\begin{defn}
	Let $X$ and $Y$ be Banach spaces. Define $X \otimes^{\text{alg}} Y = \spn \{ x \otimes y: x \in X \text{ and } y \in Y \}$ to be the algebraic tensor product of $X$ with $Y$ (see section ??? of \cite{algebra} for details). 
	\end{defn}
	
	\begin{ex}
	Let $X$ and $Y$ be Banach spaces and $x \otimes y \in X \otimes^{\text{alg}} Y$. If for each $\phi \in X^*$ and $\psi \in Y^*$, $\phi \otimes \psi(x,y) = 0$, then $x \otimes y = 0$.
	\end{ex}
	
	\begin{proof}
	The previous section tells us that for each $\phi \in X^*$ and $\psi \in Y^*$, $\phi \otimes psi(x,y) = 0$, then $x = 0$ or $y = 0$. This implies that $x \otimes y = 0$.
	\end{proof}
	
	\begin{defn}\tbf{The Projective Norm:} \\
	Define $\|\cdot \|_{\pi}:X \otimes^{\text{alg}} Y \rightarrow \Rg$ by $$\|u\|_{\pi} = \inf \bigg \{ \sum_{j=1}^n \|x_j\| \|y_j\|: (x_j)_{j=1}^n \subset X \text{, }  (y_j)_{j=1}^n \subset Y \text{ and } u = \sum_{j=1}^n x_j \otimes y_j  \bigg \}$$
	\end{defn}
	
	\begin{ex}
	Let $X$ and $Y$ be Banach spaces. Then $\| \cdot \|_{\pi}: X \otimes^{\text{alg}} Y \rightarrow \Rg$ is a norm on $X \otimes^{\text{alg}} Y$.
	\end{ex}
	
	\begin{proof}\
	\begin{itemize}
	\item Let $\lam \in \C$, $u \in X \otimes^{\text{alg}} Y$. If $\lam = 0$, then $\lam u = 0  u = 0 \otimes 0$ and clearly $\|\lam u\|_{\pi} = 0 = |\lam|\|u\|_{\pi}$. Suppose that $\lam \neq 0$. Let $\ep >0$. Then there exist $(x_j)_{j=1}^n \subset X$ and $(y_j)_{j=1}^n \subset Y$ such that $u = \sum\limits_{j=1}^n x_j \otimes y_j $ and $\sum\limits_{j=1}^n \|x_j\| \|y_j\| < \|u\|_{\pi} + \ep/|\lam| $. Then $\lam u = \sum\limits_{j=1}^n (\lam x_j) \otimes y_j $.
	Therefore
	\begin{align*}
	\|\lam u\|_{\pi} 
	& \leq \sum\limits_{j=1}^n \|\lam x_j\| \|y_j\| \\
	& \leq |\lam| \sum\limits_{j=1}^n \| x_j\| \|y_j\| \\
	& < |\lam| \bigg( \|u\|_{\pi} + \frac{\ep}{|\lam|} \bigg) \\
	& = |\lam| \|u\|_{\pi} + \ep
	\end{align*}
	Since $\ep >0$ is arbitrary, $\|\lam u\|_{\pi} \leq |\lam| \|u\|_{\pi} $. For the sake of contradiction, suppose that $\|\lam u\|_{\pi} < |\lam| \|u\|_{\pi} $. Then there exists $(x_j)_{j=1}^n \subset X$ and $(y_j)_{j=1}^n \subset Y$ such that $\lam u = \sum\limits_{j=1}^n x_j \otimes y_j $ and $\sum\limits_{j=1}^n \|x_j\| \|y_j\| < |\lam| \|u\|_{\pi}$. Hence $u = \sum\limits_{j=1}^n (\lam^{-1}x_j) \otimes y_j$. This implies that 
	\begin{align*}
	\|u \|_{\pi} 
	& \leq \sum\limits_{j=1}^n \|\lam^{-1}x_j\| \|y_j\| \\
	&= |\lam |^{-1}\sum\limits_{j=1}^n \|x_j\| \|y_j\| \\
	&< |\lam |^{-1} |\lam| \|u\|_{\pi} \\
	&= \|u\|_{\pi}
	\end{align*}
	which is a contradiction. Therefore $\|\lam u\|_{\pi} \geq |\lam| \|u\|_{\pi}$ which implies that $\|\lam u\|_{\pi} = |\lam| \|u\|_{\pi}$
	\item Let $u, v \in X \otimes^{\text{alg}} Y$ and $\ep >0$. Then there exist $(x_j)_{j=1}^n$, $(a_k)_{k=1}^m \subset X$ and $(y_j)_{j=1}^n$, $(b_k)_{k=1}^m \subset Y$ such that $u = \sum\limits_{j=1}^n x_j \otimes y_j $, $v = \sum\limits_{k=1}^m a_k \otimes b_k $, $\sum\limits_{j=1}^n \|x_j\| \|y_j\| < \|u\|_{\pi} + \ep/2$ and $\sum\limits_{k=1}^m \|a_k\| \|b_k\| < \|u\|_{\pi} + \ep/2$. Then $u+v = \sum\limits_{j=1}^n x_j \otimes y_j + \sum\limits_{k=1}^m a_k \otimes b_k $ which implies that 
	\begin{align*}
	\|u + v\|_{\pi} 
	& \leq \sum\limits_{j=1}^n \|x_j\| \|y_j\| + \sum\limits_{k=1}^m \|a_k\| \|b_k\| \\
	&< \|u\|_{\pi} + \ep/2 + \|v\|_{\pi} + \ep/2 \\
	&= \|u\|_{\pi} + \|v\|_{\pi} + \ep
	\end{align*}
	Since $\ep >0$ is arbitrary, $\|u + v\|_{\pi} \leq \|u\|_{\pi} + \|v\|_{\pi}$.
	\item Let $u \in X \otimes^{\text{alg}} Y$. Suppose that $\|u\| = 0$. Let $\phi \in X^*$ and $\psi \in Y^*$ and $\ep >0$. Then there exist $(x_j)_{j=1}^n \subset X$ and $(y_j)_{j=1}^n \subset Y$ such that $u = \sum\limits_{j=1}^n x_j \otimes y_j $ and $$\sum\limits_{j=1}^n \|x_j\| \|y_j\| < \frac{\ep}{\|\phi\| \|\psi\| + 1}$$
	Then 
	\begin{align*}
	\sum_{j=1}^n |\phi \otimes \psi(x_j,y_j)|
	&=  \sum_{j=1}^n |\phi(x_j)\psi(y_j)| \\
	& \leq  \sum_{j=1}^n \|\phi\|\|x_j\| \|\psi\|\|y_j\| \\
	&= \|\phi\|\|\psi\|\sum_{j=1}^n \|x_j\| \|y_j\| \\
	& < \|\phi\|\|\psi\|\frac{\ep}{\|\phi\| \|\psi\| + 1} \\
	& < \ep
	\end{align*}
	Then for each $j \in \{1, \ldots, n\}$, $|\phi \otimes \psi(x_j,y_j)| < \ep$.
	\tbf{FINISH!!!} Try using sequences and continuity and a common refinement of representation and averaging
	\end{itemize}
	\end{proof}
	
	\begin{ex} \tbf{Existence:} \\
	
	\end{ex}
	
	\begin{proof}
	
	\end{proof}	
	
	
	
	
	
	
	
	
	
	
	
	
	
	
	
	
	
	
	
	
	
	
	
	
	\newpage
	\section{Differentiation}
	
	\subsection{The Gateaux Derivative}
	
	\begin{defn} \ld{61001}
	Let $X,Y$ be a Banach spaces, $A \subset X$ open, $f:A \rightarrow Y$, $x_0 \in A$ and $x \in X$. Then $f$ is said to be 
	\begin{enumerate}
	\item \tbf{right-hand-differentiable at $x_0$ in the direction $x$} if the limit
	$$  \lim_{t \rightarrow 0^+} \frac{f(x_0 +tx) - f(x_0)}{t}$$
	exists. If $f$ is right-hand-differentiable at $x_0$ in the direction $x$, we define the \tbf{right-hand derivative} of $f$ at $x_0$ in the direction $x$, denoted by  $d^+ f(x_0; x)$, to be the above limit. 
	
	\item \tbf{left-hand-differentiable at $x_0$ in the direction $x$} if the limit
	$$  \lim_{t \rightarrow 0^-} \frac{f(x_0 +tx) - f(x_0)}{t}$$
	exists. If $f$ is right-hand-differentiable at $x_0$ in the direction $x$, we define the \tbf{left-hand derivative} of $f$ at $x_0$ in the direction $x$, denoted by  $d^- f(x_0; x)$, to be the above limit. 
	
	\item \tbf{differentiable at $x_0$ in the direction $x$} if the limit
	$$  \lim_{t \rightarrow 0} \frac{f(x_0 +tx) - f(x_0)}{t}$$
	exists. If $f$ is differentiable at $x_0$ in the direction $x$, we define the \tbf{derivative} of $f$ at $x_0$ in the direction $x$, denoted by $d f(x_0; x)$, to be the above limit. 
	\end{enumerate}
	\end{defn}	
	
	\begin{ex} \lex{61002}
	Let $X, Y$ be Banach spaces, $A \subset X$ open, $f:A \rightarrow \R$ and $x_0 \in A$. Then $df(x_0; 0) = 0$.
	\end{ex}
	
	\begin{proof}
	Clear.
	\end{proof}
	
	\begin{defn} \ld{}\tbf{The Gateaux Derivative:}\\
	Let $X,Y$ be Banach spaces, $A \subset X$ open, $f:A \rightarrow Y$ and $x_0 \in A$. Then $f$ is said to be 
	\begin{enumerate}
	\item \tbf{right-hand Gateaux differentiable} at $x_0$ if for each $x \in X$, $d^+ f(x_0; x)$ exits. We define the \tbf{right-hand Gateaux derivative} of $f$ at $x_0$, denoted $d^+ f(x_0) : X \rightarrow \R$, to be $$d^+ f(x_0)(x) = d^+ f(x_0;x)$$ 
	
	\item \tbf{left-hand Gateaux differentiable} at $x_0$ if for each $x \in X$, $d^- f(x_0; x)$ exits. We define the \tbf{left-hand Gateaux derivative} of $f$ at $x_0$, denoted $d^- f(x_0) : X \rightarrow \R$, to be $$d^- f(x_0)(x) = d^- f(x_0;x)$$
	
	\item \tbf{Gateaux differentiable} at $x_0$ if for each $x \in X$, $d f(x_0; x)$ exits. We define the \tbf{Gateaux derivative} of $f$ at $x_0$, denoted $d f(x_0) : X \rightarrow \R$, to be $$d f(x_0)(x) = d f(x_0;x)$$
	\end{enumerate}
	\end{defn}
	
	\begin{defn} \ld{61003}
Let $Y$ be a Banach space, $A \subset \R$ open and $f:A \rightarrow Y$. Then $f$ is said to be \tbf{Gateaux differentiable} if for each $x \in A$, $f$ is Gateaux differentiable at $x$. If $f$ is Gateaux differentiable, we define $df:A \rightarrow Y^X$ by $x_0 \mapsto df(x_0)$.
\end{defn}	
	
	\begin{ex} \lex{61004}
	Let $X, Y$ be Banach spaces, $A \subset X$ open, $f,g :A \rightarrow Y$, $\lam \in \R$ and $x_0 \in A$. If $f, g$ are Gateaux differentiable at $x_0$, then $f + \lam g$ is Gateaux differentiable at $x_0$ and $d[f+\lam g](x_0) = df(x_0) + \lam dg(x_0)$.
	\end{ex}
	
	\begin{proof}
	Similar to the case of the derivative from Calc I. 
	\end{proof}		
	
	\begin{ex} \lex{61005}
	Let $X, Y$ be Banach spaces, $A \subset X$ open, $f:A \rightarrow Y$ and $x_0 \in A$. Suppose that $f$ is Gateaux differentiable at $x_0$. Then for each $\lam \in \R$ and $x \in X$, $$df(x_0)(\lam x) = \lam df(x_0)(x)$$
	\end{ex}
	
	\begin{proof}
	Let $\lam \in \R$ and $x \in X$. Then 
	\begin{align*}
	df(x_0)(\lam x) 
	&= \lim_{t \rightarrow 0} \frac{f(x_0 + t \lam x) - f(x_0)}{t} \\
	&= \lim_{t \rightarrow 0} \lam \frac{f(x_0 + t \lam x) - f(x_0)}{\lam t} \\
	&= \lam \lim_{t \rightarrow 0}  \frac{f(x_0 + t \lam x) - f(x_0)}{\lam t} \\
	&= \lam \lim_{t \rightarrow 0}  \frac{f(x_0 + t x) - f(x_0)}{t} \\
	&= \lam df(x_0)(x) 
	\end{align*}
	\end{proof}
	
	\begin{ex} \lex{61006}
	Let $X, Y$ be Banach spaces, $A \subset X$ open, $f:A \rightarrow Y$. If $f$ is constant, then $f$ is Gateaux differentiable and for each $x_0\in A, x \in X$, $$df(x_0)(x) = 0$$
	\end{ex}
	
	\begin{proof}
	Suppose that $f$ is constant. Then there exists $c \in Y$ such that for each $x \in A$, $f(x) = c$. Let $x_0 \in A, x \in X$. Then 
	\begin{align*}
	df(x_0)(x) 
	&= \lim_{t \rightarrow 0} \frac{f(x_0 + t x) - f(x_0)}{t} \\
	&= \lim_{t \rightarrow 0} \frac{c - c}{t} \\
	&= 0
	\end{align*}
	\end{proof}
	
	\begin{ex} \lex{61007}
	Let $X, Y$ be Banach spaces, $A \subset X$ open, $f:A \rightarrow Y$. If $f$ is linear, then $f$ is Gateaux differentiable and for each $x_0\in A, x \in X$, $$df(x_0)(x) = f(x)$$
	\end{ex}
	
	\begin{proof}
	Suppose that $f$ is linear. Let $x_0\in A, x \in X$. Then 
	\begin{align*}
	df(x_0)(x) 
	&= \lim_{t \rightarrow 0} \frac{f(x_0 + t x) - f(x_0)}{t} \\
	&= \lim_{t \rightarrow 0} \frac{f(x_0) + t f(x) - f(x_0)}{t} \\
	&= f(x)
	\end{align*}
	\end{proof}		
	
	\begin{ex} \lex{61008}
	There exist Banach spaces $X,Y$, and $f:X \rightarrow Y$ such that $f$ is Gateaux differentiable and $f$ is nowhere continuous. \\
	\tbf{Hint:} use \rex{61007}
	\end{ex}
	
	\begin{proof}
	Set $X = C^1([0, 1])$ and $Y = C([0,1])$. Equip both $X$ and $Y$ with the sup norm. Define $T: X \rightarrow Y$ by $Tf = f'$. Then \rex{42002} implies that $T$ is not bounded. Since $T$ is linear, \rex{61007} implies that $T$ is Gateaux differentiable. Since $T$ is not bounded, \rex{42004} implies that $T$ is not continuous at $0$. Then \rex{42003.1} tells us that $T$ is nowhere continuous. 
	\end{proof}
	
	\begin{ex} \lex{61008.5}
	Set $A = \{(x,y) \in \R^2: y = -x^2 \text{ and } x \neq 0\}$. Define $f: \R^2 \setminus A \rightarrow \R$ by 
	\[f(x,y) = 
	\begin{cases}
	0 & (x,y) = (0,0) \\
	\frac{x^4y}{x^6 + y^3} & \text{otherwise}
	\end{cases}	
	\]
	Then $f$ is Gateaux differentiable at $(0,0)$ and $f$ is not continuous at $(0,0)$. \\
	\tbf{Hint:} Consider the set $B = \{(x, x^2:x \in \R)\} \subset \R^2 \setminus A$. 
	\end{ex}
	
	\begin{proof}
	
	\end{proof}
	
	
	
	\begin{ex} \lex{61009}
	Let $Y$ be a Banach space, $A \subset \R$ open, $f:A \rightarrow Y$ and $x_0 \in A$. Suppose that $f$ is Gateaux differentiable at $x_0$. Then $df(x_0) \in L(\R,Y)$.
	\end{ex}
	
	\begin{proof}
	Let $x,y ,\lam \in \R$. 
	\begin{enumerate}
	\item The previous exercise implies 
	\begin{align*}
	df(x_0)(x + \lam y) 
	&= df(x_0)((x+\lam y)1)  \\
	&= (x+\lam y)df(x_0)(1) \\
	&= xdf(x_0)(1) + \lam y df(x_0)(1) \\
	&= df(x_0)(x) + \lam df(x_0)(y)
	\end{align*}
	So $df(x_0):\R \rightarrow Y$ is linear.
	\item Since 
	\begin{align*}
	\|df(x_0)(x)\| 
	&= \|xdf(x_0)(1)\| \\
	&= |x| \|df(x_0)(1)\| \\
	\end{align*}	
	We have that $df(x_0):\R \rightarrow Y$ is bounded with $\|df(x_0)\| \leq \|df(x_0)(1)\|$. 
	\end{enumerate}
	\end{proof}
	
	\begin{ex} \lex{61010}
	Let $X$ be a Banach space, $A \subset X$ open, $f:A \rightarrow \R$ and $x_0 \in A$. If $f$ is Gateaux differentiable at $x_0$ and $f$ has a local extremum at $x_0$, then $df(x_0) = 0$.
	\end{ex}	
	
	\begin{proof}
	Suppose that $f$ is Gateaux differentiable at $x_0$ and $f$ has a local minimum point at $x_0$. Then there exists $\del >0 $ such that $B(x_0, \del) \subset A$ and for each $y \in B(x_0, \del)$, $f(x_0) \leq f(y)$. \\
	For the sake of contradiction, suppose that $df(x_0) \neq 0$. Then there exists $x \in X$ such that $x \neq 0$ and $df(x_0)(x) \neq 0$. \\
	First, suppose that $df(x_0)(x) < 0$. Choose $\ep = -df(x_0)(x) >0$. Then there exists $t_0 >0$ such that for each $t \in B^*(0, t_0)$, $x_0 + tx \in B(x_0, \del)$ and $$\bigg | \frac{f(x_0 + tx) - f(x_0)}{t} - df(x_0)(x) \bigg | < \ep$$ 
	This implies that for each $t \in B^*(0, t_0)$,
	\begin{align*}
	\frac{f(x_0 + tx) - f(x_0)}{t}  
	&< \ep + df(x_0)(x) \\
	&= 0
	\end{align*} 
	and hence $f(x_0 + tx) < f(x_0)$, which is a contradiction. \\
	Now, suppose that $df(x_0)(x) > 0$. Then 
	\begin{align*}
	df(x_0)(-x) 
	&= -df(x_0)(x) \\
	& < 0
	\end{align*}
	Similarly to above, this implies that there exists $t_0 >0$ such that for each $t \in B^*(0, t_0)$, $x_0 - tx \in B(x_0, \del)$ and $f(x_0 - tx) < f(x_0)$ which is a contradiction. So $df(x_0)(x) = 0$ and $df(x_0) = 0$. \\
	If $f$ has a local maximum at $x_0$, then $-f$ has a local minimum point at $x_0$. Then 
	\begin{align*}
	df(x_0)
	&= -d[-f](x_0) \\
	&= -0 \\
	&= 0
\end{align*}	 
	\end{proof}
	
	\begin{ex} \lex{61011}
	Let $X, Y, Z$ be a Banach spaces, $A \subset X$ open, $B \subset Y$ open, $f:A \rightarrow Y$, $g:B \rightarrow Z$ and $x_0 \in A$. Suppose that $f$ is affine. If $g$ is Gateaux differentiable at $f(x_0)$, then $g \circ f$ is Gateaux differentiable at $x_0$ and $$d(g \circ f)(x_0)(x) = dg(f(x_0))(df(x_0)(x))$$
	\end{ex}
	
	\begin{proof}
	Suppose that $g$ is Gateaux differentiable at $f(x_0)$. Since $f$ is affine, there exists $h:A \rightarrow Y$ and $c \in Y$ such that $h$ is linear and $f = h+c$. Then 
	\begin{align*}
	df(x_0) 
	&= dh(x_0) \\
	&= h
	\end{align*}
	Let $x \in X$. Choose $\del >0$ such that for each $t \in B(0, \del) \subset \R$, $f(x_0) + th(x) \in B$. Then for each $t \in B^*(0, \del)$,
	\begin{align*}
	g \circ f(x_0 + tx)
	&= g \bigg ( f(x_0) + t \frac{f(x_0 + tx) - f(x_0)}{t} \bigg) \\
	&=  g ( f(x_0) + th(x)) 
	\end{align*}
	This implies that
	\begin{align*}
	d(g \circ f)(x_0)
	&= \lim_{t \rightarrow 0 }\frac{g \circ f (x_0 + tx) - g(f(x_0))}{t} \\  
	&= \lim_{t \rightarrow 0} \frac{g ( f(x_0) + th(x)) - g(x_0)}{t} \\
	&= dg(f(x_0))(h(x)) \\
	&= dg(f(x_0))(df(x_0)(x)) \\
	\end{align*}
	\end{proof}
	
	
	
	
	
	
	
	
	
	
	
	
	
	
	
	\newpage
	\subsection{The Frechet Derivative}
	
	\begin{note}
		Let $X$ be a vector space over $\C$, $Y$ a vector space over $\R$ and $T:X \rightarrow Y$. Since all vector spaces over $\C$ are also vector spaces over $\R$ we will consider $T$ linear if $T$ is $\R$-linear. 
	\end{note}
	
	\begin{ex} \lex{62001}
	Let $X,Y$ be a normed vector spaces and $\phi: X \rightarrow Y$ linear. If $\phi(h) = o(\|h\|)$ as $h \rightarrow 0$, then $\phi = 0$. 
	\end{ex}
	
	\begin{proof}
	Let $h_0 \in X$. If $h_0 = 0$, then $\phi(h_0) = 0$. Suppose that $h_0 \neq 0$. Define $(h_n)_{n \in \N} \subset X$ by $$h_n = \frac{h_0}{n}$$ Then $h_n \rightarrow 0$. By continuity of $\phi$ and our initial assumption we have that 
	\begin{align*}
	\| h_0 \|^{-1} \phi ( h_0 ) 
	&= \phi \bigg( \frac{h_0}{\| h_0 \|} \bigg) \\
	&= \phi \bigg( \frac{h_n}{\| h_n \|} \bigg) \\
	&= \frac{\phi(h_n)}{\| h_n \|} \\
	& \rightarrow 0
	\end{align*}
	which implies that $\| h_0 \|^{-1}\phi ( h_0 ) = 0$. So $\phi(h_0) = 0$ and hence $\phi = 0$.
	\end{proof}
	
	\begin{ex} \lex{62002}
	Let $X, Y$ be a normed vector spaces, $A \subset X$ open, $f:A \rightarrow Y$ and $x_0 \in A$. Suppose that there exists $\phi: X \rightarrow Y$ such that $\phi$ is linear and $$f(x_0 + h) = f(x_0) + \phi(h) + o(\|h\| ) \hspace{.5cm} \text{ as } h \rightarrow 0$$ then $\phi$ is unique. 
	\end{ex}
	
	\begin{proof}
	Suppose that there exists $\psi : X \rightarrow Y$ such that $\psi$ is linear and such that
	$$f(x_0 + h) = f(x_0) + \psi(h) + o(\|h\| ) \hspace{.5cm} \text{ as } h \rightarrow 0$$ 
	Then $\phi(h) - \psi(h) = o(h)$. Since $\phi - \psi$ is linear, the previous exercise implies that $\phi = \psi$.
	\end{proof}
	
	\begin{note}
	Recall that for Banach spaces $X$ and $Y$, there isomorphic isometry $$L(X, L(X, \cdots, L(X, Y)) \cdots) \rightarrow L^n(X, Y)$$ given by $\phi \mapsto \psi_{\phi}$ where $$\psi_{\phi}(x_1, x_2, \cdots, x_n) = \phi(x_1)(x_2),\cdots,(x_n)$$
	\end{note}	
	
	\begin{defn} \ld{62003}\tbf{Frechet Derivative:} \\
	Let $X, Y$ be a banach spaces, $A \subset X$ open, $f:A \rightarrow Y$ and $x_0 \in A$. 
	\begin{enumerate}
	\item 
	\begin{itemize}
	\item Then $f$ is said to be \tbf{ Frechet differentiable at $x_0$} if there exists $Df(x_0) \in L(X,Y)$ such that, $$f(x_0 + h) = f(x_0) + Df(x_0)(h) + o(\|h\| ) \hspace{.5cm} \text{ as } h \rightarrow 0$$  
	\item If $f$ is Frechet differentiable at $x_0$, we define the \tbf{ Frechet derivative of $f$ at $x_0$} to be $Df(x_0)$. 
	\item We say that $f$ is \tbf{ Frechet differentiable} if for each $x \in A$, $f$ is Frechet differentiable at $x$. 
	\item If $f$ is Frechet differentiable, we define the \tbf{ Frechet derivative of $f$}, denoted $Df:A \rightarrow L(X, Y)$, by $x \mapsto D^{(1)}f(x)$. 
	\end{itemize}
	\item Continuing inductively, we set $D^0f = f$ and for $n \geq 2$, 
	\begin{itemize}
	\item $f$ is said to be \tbf{$n$-th order Frechet differentiable at $x_0$} if $f$ is $(n-1)$-th order Frechet differentiable and $D^{n-1}f$ is Frechet differentiable at $x_0$. 
	\item If $f$ is $n$-th order Frechet differentiable at $x_0$, we define $D^nf(x_0) \in L^n( X, Y)$ by 
	$$D^nf(x_0) = D[D^{n-1}f](x_0)$$ 
	\item We say that $f$ is \tbf{$n$-th order Frechet differentiable} if $f$ is $(n-1)$-th order Frechet differentiable and for each $x \in A$, $D^{n-1}f$ is Frechet differentiable at $x$. 
	\item If $f$ is $n$-th order Frechet differentiable, we define the \tbf{$n$-th order Frechet derivative of $f$}, denoted $D^nf:A \rightarrow  L^n(X, Y)$ by $x \mapsto D^{n}f(x)$ \\
	\end{itemize}		
	\item If $f$ is $n$-th order differentiable, then $f$ is said to be \tbf{continuously $n$-th order differentiable} if $D^nf$ is continuous. We define $$C^n(A,Y) = \{f:A \rightarrow Y: f \text{ is continuously $n$-th order differentiable}\}$$
	
	\end{enumerate}
	\end{defn}
	
	\begin{ex} \lex{62004}
	Let $X, Y$ be a banach spaces, $A \subset X$ open, $f,g:A \rightarrow Y$, $\lam \in \R$ and $x_0 \in A$. If $f$ and $g$ are Frechet differentiable at $x_0$, then $f+ \lam g$ is Frechet differentiable at $x_0$ and $D(f+\lam g)(x_0) = Df(x_0) + \lam Dg(x_0)$.
	\end{ex}
	
	\begin{proof}
	Suppose that $f$ and $g$ are Frechet differentiable at $x_0$. Then $$f(x_0 + h) = f(x_0) + Df(x_0)(h) + o(\|h\| ) \hspace{.5cm} \text{ as } h \rightarrow 0$$  and $$g(x_0 + h) = g(x_0) + Dg(x_0)(h) + o(\|h\| ) \hspace{.5cm} \text{ as } h \rightarrow 0$$  
	This implies that 
	\begin{align*}
	(f+\lam g)(x_0 + h) 
	&= f(x_0 + h) +\lam g(x_0 + h) \\
	&= f(x_0) + Df(x_0)(h) + o(\|h\| ) + \lam g(x_0) + \lam Dg(x_0)(h) + o(\|h\| ) \\
	&= (f+\lam g)(x_0) + [Df(x_0)+ \lam Dg(x_0)](h) + o(\|h\|) \hspace{.5cm} \text{ as } h \rightarrow 0
	\end{align*}
	Since $Df(x_0)+\lam Dg(x_0) \in L(X,Y)$, $f+\lam g$ is Frechet differentiable at $x_0$ and $D(f+\lam g)(x_0) = Df(x_0) + \lam Dg(x_0)$. 
	\end{proof}
	
	\begin{ex}\lex{62004.5}
	Let $X, Y$ be a banach spaces, $A \subset X$ open, $f:A \rightarrow Y$ and $x_0 \in A$. If $f$ is Frechet differentiable at $x_0$, then $f$ is continuous at $x_0$. 
	\end{ex}
	
	\begin{proof}
	Suppose that $f$ is Frechet differentiable at $x_0$. Then $f(x) - f(x_0) = Df(x_0)(x - x_0) + o(\|x - x_0\|)$ as $x \rightarrow x_0$. Hence $\|f(x) - f(x_0)\| \leq \| Df(x_0)\| \|x - x_0 \| + o(\|x - x_0\|)$ as $x \rightarrow x_0$. This implies that $f(x) \rightarrow f(x_0)$ as $x \rightarrow x_0$ and therefore $f$ is continuous at $x_0$.
	\end{proof}
	
	\begin{ex} \lex{62005}
	Let $X, Y$ be a banach spaces, $A \subset X$ open, $f:A \rightarrow Y$ and $x_0 \in A$. If $f$ is Frechet differentiable at $x_0$, then $f$ is Gateaux differentiable at $x_0$ and $df(x_0) = Df(x_0)$.
	\end{ex}
	
	\begin{proof}
	Suppose that $f$ is Frechet differentiable at $x_0$. Then $f(x_0 + h) = f(x_0) + Df(x_0)(h) + o(\|h\| )$ as $h \rightarrow 0$. Let $x \in X$. Then $f(x_0 + tx) - f(x_0) = tDf(x_0)(x) + o(t)$ as $t \rightarrow 0$. This implies that $f$ is differentiable at $x_0$ in the direction $x$ and 
	\begin{align*}
	df(x_0)(x) 
	&= \lim_{t \rightarrow 0} \frac{f(x_0 + tx) - f(x_0)}{t} \\
	&= Df(x_0)(x)
	\end{align*}
	Since $x \in X$ is arbitrary, $f$ is Gateaux differentiable at $x_0$ and $df(x_0) = Df(x_0)$.
	\end{proof}
	
	\begin{ex} \lex{62006}
	Let $X$ be a Banach space, $A \subset X$ open, $f:A \rightarrow \R$ and $x_0 \in A$. If $f$ is Frechet differentiable at $x_0$ and $f$ has a local extremum at $x_0$, then $Df(x_0) = 0$.
	\end{ex}	
	
	\begin{proof}
	Suppose that $f$ is Frechet differentiable at $x_0$ and $f$ has a local extremum at $x_0$. Two previous exercises imply that $f$ is Gateaux differentiable at $x_0$ and 
	\begin{align*}
	Df(x_0) 
	&= df(x_0) \\
	&= 0
	\end{align*}	
	\end{proof}
	
	\begin{defn} \ld{62007}
	Let $X, Y$ be a banach spaces, $A \subset X$ open, $f:A \rightarrow Y$ and $x_0 \in A$. Suppose that $f$ is Frechet differentiable at $x_0$. Define $R_f(x_0): A - x_0 \rightarrow Y$ by $$R_f(x_0)(h) = f(x_0 + h) - f(x_0) - Df(x_0)(h)$$
	\end{defn}
	
	\begin{ex} \lex{62008}
	Let $X, Y$ be a banach spaces, $A \subset X$ open, $f:A \rightarrow Y$ and $x_0 \in A$. If $f$ is Frechet differentiable at $x_0$, then $$f(x_0+h) - f(x_0) = O(\|h\|) \hspace{.1cm} \text{ as } h \rightarrow 0$$ 
	\end{ex}
	
	\begin{proof}
	Suppose that $f$ is Frechet differentiable at $x_0$. Then $R_f(h) = o(\|h\|)$ as $h \rightarrow 0$. Hence there exists $\del >0$ such that $B(0, \del) \subset A - x_0$ and for each $h \in B(0, \del)$, $\|R_f(h)\| \leq \|h\|$. Hence for each $h \in B(0, \del)$
	\begin{align*}
	\|f(x_0+h) - f(x_0) \| 
	&= \|Df(x_0)(h) + R_f(x_0)(h)\| \\
	& \leq \|Df(x_0)(h)\| + \|R_f(x_0)(h)\|  \\
	& \leq \|Df(x_0)\| \|(h)\| + \|h\| \\
	& = (\|Df(x_0)\| + 1) \|h\|
	\end{align*}
	\end{proof}
	
	\begin{ex} \lex{62009}\tbf{Chain Rule:}\\
	Let $X, Y, Z$ be a Banach spaces, $A \subset X$ open, $B \subset Y$ open, $f:A \rightarrow Y$, $g:B \rightarrow Z$ and $x_0 \in A$. Suppose that $f(x_0) \in B$. If $f$ is Frechet differentiable at $x_0$ and $g$ is Frechet differentiable at $f(x_0)$, then $g \circ f$ is Frechet differentiable at $x_0$ and $$D(g \circ f)(x_0) = Dg(f(x_0)) \circ Df(x_0)$$
	\end{ex}
	
	\begin{proof}
	Suppose that $f$ is Frechet differentiable at $x_0$ and $g$ is Frechet differentiable at $f(x_0)$. 
	
	\begin{itemize}
	\item The previous exercise implies that there exists $\del^* >0$ and $K > 0$ such that for each $h \in B(0, \del^*)$, $\| f(x_0 + h) - f(x_0) \| \leq K \|h\|$. Let $\ep >0$. Since $R_g(f(x_0))(h') = o(\|h'\|)$ as $h' \rightarrow 0$, there exists $\del' >0$ such that for each $h' \in B(0, \del')$, $\|R_g(f(x_0))(h')\| \leq \frac{\ep}{K} \|h'\|$. \\ Choose $\del = \min(\del' / K, \del^*)$. Let $h \in B(0, \del)$. Then 
	\begin{align*}
	\| f(x_0 + h) - f(x_0) \| 
	& \leq K \|h\| \\
	&< \del' 
	\end{align*}
	This implies that 
	\begin{align*}
	\|R_g(f(x_0))(f(x_0 + h) - f(x_0))\| 
	& \leq \frac{\ep}{K} \|f(x_0 + h) - f(x_0)\| \\
	& \leq \frac{\ep}{K} K\|h\| \\
	& \leq \ep \|h\|
	\end{align*}
	So $R_g(f(x_0))(f(x_0 + h) - f(x_0)) = o(\|h\|)$ as $h \rightarrow 0$. \\
	\item Since $\|Dg(f(x_0))(R_f(x_0)(h))\| \leq \|Dg(f(x_0)) \| \|R_f(x_0)(h)\|$ and $R_f(x_0)(h) = o(h)$ as $h \rightarrow 0$, we have that $Dg(f(x_0))(R_f(x_0)(h)) = o(h)$ as $h \rightarrow 0$. \\
	\item Combining the previous two observations, we have that $Dg(f(x_0))(R_f(x_0)(h)) + R_g(f(x_0))(f(x_0 + h) - f(x_0)) = o(\|h\|)$ as $h \rightarrow 0$. \\
	\item All together, we obtain 
	\begin{align*}
	g \circ f(x_0 + h) 
	&=  g (f(x_0)) + f(x_0 + h) - f(x_0)) \\
	&= g(f(x_0)) + Dg(f(x_0))(f(x_0 + h) - f(x_0)) + R_g(f(x_0))(f(x_0 + h) - f(x_0)) \\
	&= g(f(x_0)) + Dg(f(x_0))(Df(x_0)(h) + R_f(x_0)(h)) \\
	& \hspace{2.05cm} +  R_g(f(x_0))(f(x_0 + h) - f(x_0)) \\
	&= g(f(x_0)) + Dg(f(x_0))(Df(x_0)(h)) + Dg(f(x_0))(R_f(x_0)(h)) \\
	& \hspace{2.05cm} +  R_g(f(x_0))(f(x_0 + h) - f(x_0)) \\
	&= g \circ f(x_0) + Dg(f(x_0)) \circ Df(x_0)(h) + o(\|h\|) \text{ as } h \rightarrow 0
	\end{align*}
	So $g \circ f$ is Frechet differentiable at $x_0$ and $D(g \circ f)(x_0) = Dg(f(x_0)) \circ Df(x_0)$.
	\end{itemize}
	\end{proof}
	
	\begin{ex} \lex{62010}
	Let $Y$ be a Banach space, $A \subset \R$ open and $f:A \rightarrow Y$. Then $f$ is Gateaux differentiable iff $f$ is Frechet differentiable.
	\end{ex}
	
	\begin{proof}
	Suppose that $f$ is Gateaux differentiable. Let $x_0 \in A$. A previous exercise implies that $df(x_0) \in L(\R, Y)$. By defintion, $$  \lim_{h \rightarrow 0} \bigg \| \frac{f(x_0 + h) - f(x_0)}{h} - df(x_0)(1) \bigg \| = 0$$ 
	This is equivalent to saying that $$f(x_0 + h) = f(x_0) + df(x_0)(h) + o(|h|) \hspace{.5cm} \text{ as } h \rightarrow 0$$
	So $f$ is Frechet differentiable at $x_0$ and $Df(x_0) = df(x_0)$.
	\end{proof}
	
	
	
	
	
	
	
	
	\newpage
	\subsection{The Calc I Derivative}
	\begin{defn} \ld{}\tbf{Calc I Derivative:}\\
	Let $Y$ be a Banach space, $A \subset \R$ or $\C$ open, $f:A \rightarrow Y$ and $x_0 \in A$. 
	\begin{enumerate}
	\item 
	\begin{itemize}
	\item If $f$ is Frechet differentiable at $x_0$, we define the \tbf{calc I derivative of $f$ at $x_0$}, denoted $$f'(x_0) \text{ or } \dv{f}{t}{(x_0)}$$ by
	\begin{align*}
	f'(x_0) 
	&= \lim_{t \rightarrow 0} \frac{f(x_0 + t) - f(x_0)}{t} \\
	&= df(x_0)(1) \\
	&= Df(x_0)(1)
	\end{align*}
	\item If $f$ is Frechet differentiable, we define $f':A \rightarrow Y$ by $x \mapsto f'(x)$. 
	\end{itemize}
	\item Continuing inductively, we set $f^{(0)} = f$ and for $n \geq 1$,
	\begin{itemize}
	\item  if $f^{(n-1)}$ is Frechet differentiable at $x_0$, we define the \tbf{$(n)$-th order calc I derivative of $f$ at $x_0$}, denoted $f^{(n)}(x_0)$, by $$f^{(n)}(x_0) = [f^{(n-1)}]'(x_0)$$ 
	\item if $f^{(n-1)}$ is Frechet differentiable, we define $f^{(n)}:A \rightarrow Y$ by $$f^{(n)} = [f^{(n - 1)}]'$$ 
	\end{itemize}
	\end{enumerate}
	\end{defn}	
	
	\begin{ex} \lex{}
	Let $Y$ be a Banach space, $A \subset \R$ open and $f:A \rightarrow Y$. If $f$ is $n$-th order Frechet differentiable, then for each $x_0 \in A$ and $k \in \{1, \cdots, n\}$, $$f^{(k)}(x_0) = D^kf(x_0)(1^{\oplus k})$$
	\end{ex}
	
	\begin{proof}
	Let $x_0 \in A$. We proceed by induction. The base case is true by definition. Let $k \in \{1, \cdots, n\}$. Suppose the claim is true for $k-1$. Then $$f^{(k-1)}(x_0) = D^{k-1}f(x_0)(1^{\oplus (k-1)})$$
	Since $f$ is $n$-th order Frechet differentiable, $$D^{k-1}f(x_0+h) = D^{k-1}f(x_0) + D^kf(x_0)(h) + o(\|h\|) \hspace{.5cm} \text{ as } h \rightarrow 0$$ 
	This implies that 
	\begin{align*}
	f^{(k-1)}(x_0+h) 
	&=  D^{k-1}f(x_0+h)(1^{\oplus (k-1)}) \\
	&= D^{k-1}f(x_0)(1^{\oplus (k-1)}) + D^kf(x_0)(h)(1^{\oplus (k-1)}) + o(\|h\|) \hspace{.5cm} \text{ as } h \rightarrow 0
	\end{align*}
	Therefore for each $h \in \R$, $$Df^{(k-1)}(x_0)(h) = D^kf(x_0)(h)(1^{\oplus (k-1)})$$
	and by definition,
	\begin{align*}
	f^{(k)}(x_0) 
	&= [f^{(k-1)}]'(x_0) \\
	&= Df^{(k-1)}(x_0)(1) \\
	&=  D^kf(x_0)(1^{\oplus k})
	\end{align*}
	\end{proof}
	
	
	
	
	\begin{ex} \lex{}
	Let $X,Y$ be Banach spaces, $A \subset X$ open, $f \in C^n(A, Y), x_0 \in A$, and $h \in X$. Suppose that $\{x_0 +th: t \in [0,1]\} \subset A$. Define and $g:(0,1) \rightarrow Y$ by $$g(t) = f(x_0 + th)$$
	Then for each $k \in \{1 \dots, n\}$ and $t \in (0,1)$, $$g^{(k)}(t) = D^kf(x_0 + th)(h^{\oplus k})$$
	\end{ex}
	
	\begin{proof}
	We proceed by induction. It is straightforward to show that the claim is true for $k=1$.\\
	Let $k \in \{1 \dots, n\}$. Suppose that $g^{(k-1)}(t) = D^{k-1}f(x_0 + th)(h^{\oplus (k-1)})$. Since $f \in C^k(A, Y)$, $$D^{k-1}f(x_0 + s_0h + th)= D^{k-1}f(x_0 + s_0h) + D^k f(x_0 + s_0h)(th) + o(\|t\|) \hspace{.2cm} \text{ as }t \rightarrow 0 $$ 
	The previous exercise implies that 
	\begin{align*}
	g^{(k-1)}(s_0 + t)
	&= D^{k-1}g(s_0+t)(1^{\oplus (k-1)}) \\
	&= D^{k-1}f(x_0 + s_0h + th)(h^{\oplus (k-1)}) \\
	&= D^{k-1}f(x_0 + s_0h)(h^{\oplus (k-1)}) + D^kf(x_0 + s_0h)(th)(h^{\oplus (k-1)}) + o(\|t\|) \hspace{.2cm} \text{ as }t \rightarrow 0
	\end{align*}
	Hence $$Dg^{(k-1)}(s_0)(t) = D^kf(x_0 + s_0h)(th)(h^{\oplus (k-1)})$$ 
	and 
	\begin{align*}
	g^{(k)}(t)
	&= Dg^{(k-1)}(t)(1) \\
	&= D^kf(x_0 + th)(h^{\oplus k})
	\end{align*}
	\end{proof}
	
	
	
	
	
	
	
	\newpage
	\subsection{Mean Value Theorem}	
	
	\begin{ex} \lex{64001}
	Let $X$ be a Banach space, $A \subset X$ open and convex, and $f:A \rightarrow \R$. If $f$ is continuous and Gateaux differentiable, then for each $x,y \in A$, there exists $t^* \in (0,1)$ such that $f(x) - f(y) = df(t^*x + (1-t^*)y)(x - y)$. 
	\end{ex}
	
	\begin{proof}
	Suppose that $f$ is continuous and Gateaux differentiable. Let $x,y \in A$. Define $h: [0,1] \rightarrow X$ by $h(t) = tx +(1-t)y$. Set $g = f \circ h: [0,1] \rightarrow \R$. Then $g$ is continuous on $[0,1]$ and \rex{61011} implies that $g$ is Gateaux differentiable on $(0,1)$. Then \rex{62010} \rex{61011} and the mean value theorem implies that there exists $t^* \in (0,1)$ such that
	\begin{align*}
	f(x) - f(y)
	&= g(1) - g(0) \\
	&=g'(t^*) \\ 
	&= dg(t^*)(1) \\
	&= df(h(t^*))(dh(t^*)(1)) \\
	&= df(h(t^*))(h'(t^*)) \\
	&= df(t^*x + (1-t^*)y)(x -y)
	\end{align*}
	\end{proof}
	
	\begin{ex} \lex{64002}
	Let $X$ be a Banach space, $A \subset X$ open and convex, and $f:A \rightarrow \R$. If $f$ is Frechet differentiable, then for each $x,y \in A$, there exists $t^* \in (0,1)$ such that $f(x) - f(y) = Df(t^*x + (1-t^*)y)(x - y)$. 
	\end{ex}
	
	\begin{proof}
	Suppose that $f$ is Frechet differentiable. Then $f$ is continuous and Gateaux differentiable. Now apply the previous exercise.	
	\end{proof}
	
	\begin{ex} \lex{64003}\tbf{Mean Value Theorem:}\\
	Let $X, Y$ be a Banach spaces, $A \subset X$ open and convex and $f:A \rightarrow Y$. If $f$ is Frechet differentiable, then for each $x,y \in A$, there exists $t^* \in (0,1)$ such that $$\|f(x) - f(y)\| \leq \|Df(t^*x + (1-t^*)y)\|\|x-y\|$$
	\tbf{Hint:} For $x,y \in A$ with $f(x) \neq f(y)$, using a Hahn-Banach argument, find $\lam \in Y^*$ such that $\|\lam\| = 1 $ and $\lam (f(x) - f(y)) = \|f(x) - f(y)\|$.
	\end{ex}
	
	\begin{proof}
	Suppose that $f$ is Frechet differentiable. Let $x,y \in A$. The claim is clearly true when $f(x) = f(y)$. Suppose that $f(x) \neq f(y)$. An exercise in the section on linear functionals implies that there exists $\lam \in Y^*$ such that $\lam(f(x)-f(y)) = \| f(x) - f(y)\|$ and $\|\lam \| = 1$
	Define $g:[0,1] \rightarrow \R$ by $$g(t) = \lam(f(tx +(1-t)y))$$ Then $g$ is continuous and (Frechet) differentiable on $(0,1)$ with $$Dg(t)(h) = \lam \circ Df(tx+(1-t)y)((x-y)h)$$ which implies that
	\begin{align*}
	g'(t) 
	&= Dg(t)(1)\\
	&= \lam \circ Df(tx+(1-t)y)((x-y))
	\end{align*}
	The mean value theorem implies that there exists $t^* \in (0,1)$ such that 
	\begin{align*}
	\|f(x) - f(y)\|
	&= \lam(f(x) - f(y)) \\
	&= g(1) - g(0) \\
	&= g'(t^*)\\
	&= \lam \circ Df(t^*x+(1-t^*)y)((x-y))
	\end{align*}
	Taking absolute values, we see that 
	\begin{align*}
	\|f(x) - f(y)\|
	&= |\lam \circ Df(t^*x+(1-t^*)y)((x-y))| \\
	& \leq \|\lam \| \|Df(t^*x+(1-t^*)y)\|\|x-y\| \\
	& \leq \|Df(t^*x+(1-t^*)y)\|\|x-y\|
	\end{align*}
	\end{proof}
	
	\begin{ex} \lex{64004}
	Let $X, Y$ be Banach spaces, $A \subset X$ open and convex and $f:A \rightarrow Y$. Suppose that $f$ is Frechet differentiable. If for each $x \in A$, $Df(x) = 0$, then $f$ is constant.
	\end{ex}
	
	\begin{proof}
	Suppose that for each $x \in A$, $Df(x) = 0$. Let $x,y \in A$. Then the mean value theorem implies that there exists $t \in (0, 1)$ such that 
	\begin{align*}
	\|f(x) - f(y)\| 
	&\leq \|Df(tx + (1-t)y)\| \|x-y\| \\
	&= 0
	\end{align*}
	So $f(x) = f(y)$. 
	\end{proof}
	
	\begin{ex} \lex{64005}
	Let $X, Y$ be Banach spaces, $A \subset X$ open and convex and $f,g:A \rightarrow Y$. Suppose that $f$ and $g$ are Frechet differentiable. If $Df = Dg$, then there exists $c \in Y$ such that $f = g+c$.
	\end{ex}
	
	\begin{proof}
		Suppose that $Df = Dg$. Then $D(f-g) = 0$ and the previous exercise implies that $f -g$ is constant.
	\end{proof}		
	
	\begin{ex} \lex{64006}
	Let $X, Y$ be a Banach spaces, $A \subset \R$ open and $f:A \rightarrow Y$. Suppose that $f$ is Frechet differentiable. Then $f' \in C(A,Y)$ iff $f \in C^1(A,Y)$.
	\end{ex}
	
	\begin{proof}
	Suppose that $f' \in C(A, Y)$. Let $x,y \in A$ and $h \in \R$. Then 
	\begin{align*}
	\|(Df(x)- Df(y))(h)\| 
	&= \|Df(x)(h) - Df(y)(h)\| \\
	&=  \|hf'(x) - hf'(y)\| \\
	&= \|h(f'(x) - f'(y))\| \\
	&= \|f'(x) - f'(y)\||h|
	\end{align*}
	So $\|Df(x) - Df(y)\| \leq \|f'(x) - f'(y)\|$. Hence continuity of $f'$ implies continuity of $Df$ and $f \in C^1(A, Y)$.
	Conversely, suppose that $f \in C^1(A, Y)$. Let $x,y \in A$. Then 
	\begin{align*}
	\|f'(x) - f'(y)\| 
	&= \|Df(x)(1) - Df(y)(1)\| \\
	&= \|(Df(x) - Df(y))(1)\| \\
	& \leq \| Df(x) - Df(y)\|
	\end{align*}
	Hence continuity of $Df$ implies continuity of $f'$ and $f' \in C(A, Y)$.
	\end{proof}

	\begin{ex}
		Let $X,Y$ be Banach spaces, $A \subset X$ open and convex and $f:A \rightarrow Y$. Suppose that $f$ is Frechet differentiable. Then $f$ is Lipschitz iff $Df$ is bounded.
	\end{ex}

	\begin{proof}
		Suppose that $f$ is Lipschitz. Then there exists $M > 0$ such that for each $x, y \in A$, $\|f(y) - f(x)\| \leq M \|y - x\|$. Let $x \in A$ and $h \in X$. Suppose that $\|h\| = 1$. Then $Df(y-x) = f(y) - f(x) + o(\|x-y\|)$. Hence 
		\begin{align*}
			\|Df(x)(th)\| 
			& \leq \|f(x +th) - f(x)\| + o(\|th\|) \\
			& \leq M \|th\|  + o(\|th\|)  \text{ as $t \rightarrow 0$}\\
			& = M |t|  + o(|t|)  \text{ as $t \rightarrow 0$}\\
		\end{align*}
		Hence $\|Df(x)(h)\| \leq M + o(1) $ as $t \rightarrow 0$ which implies that $\|Df(x)(h)\| \leq M $. Thus 
		\begin{align*}
			\|Df(x)\| 
			& = \sup \{\|Df(x)(h)\|: h \in X \text{ and $\|h\| =1 $}\} \\
			& \leq M
		\end{align*}
		Since $x \in A$ is arbitrary, $Df$ is bounded.\\
		Conversely, suppose that $Df$ is bounded. Then there exists $M > 0$ such that for each $x \in A$, $\|Df(x)\| \leq M$. Let $x,y \in A$. The mean value theorem implies that there exists $t^* \in (0,1)$ such that 
		\begin{align*}
			\|f(x) - f(y)\| 
			& \leq \|Df(t^*x + (1-t^*)y)\|\|x-y\| \\
			& \leq M \|x-y\|
		\end{align*}
		Therefore $f$ is Lipschitz.
	\end{proof}
	
	
	
	
	
	
	
	
	
	
	\newpage
	\subsection{Taylor's Theorem}
	
	\begin{ex} \lex{}
	Let $Y$ be a separable Banach space, $f:[a,b] \rightarrow Y$ continuous so that $f$ is Bochner-integrable. Define $F:(a,b) \rightarrow Y$ by  $$F(x) = \int_{(a, x]}f dm$$ Then $F \in C^1((a,b), Y)$ and for each $x_0 \in (a,b)$ and $F'(x_0) = f(x_0)$.
	\end{ex}
	
	\begin{proof}
	Let $x_0 \in (a,b)$ and $h \in (0, b-x_0)$. Then continuity implies that
	\begin{align*}
	\frac{1}{\|h\|} \bigg | \int_{(x_0, x_0 + h]}f - f(x_0) dm \bigg |
	& \leq  \frac{1}{\|h\|} \max_{x \in (x_0, x_0+h]} |f(x) - f(x_0)| \|h\| \\
	&= \max_{x \in [x_0, x_0+h]} |f(x) - f(x_0)| \\
	& \rightarrow 0  \text{ as } h \rightarrow 0
\end{align*}	  
So $$\int_{(x_0, x_0 + h]}f - f(x_0) dm = o(\|h\|) \hspace{1cm}\text{ as }h \rightarrow 0$$ 
	Therefore 
	\begin{align*}
	F(x_0 + h)
	&= \int_{(a, x_0 + h]} f dm  \\
	&= \int_{(a, x_0]} f dm + \int_{(x_0, x_0 + h]} fdm \\
	&= \int_{(a, x_0]} f dm + hf(x_0) + \int_{(x_0, x_0 + h]} f - f(x_0) dm \\ 
	&= F(x_0 ) + hf(x_0) + o(\|h\|) \hspace{1cm }\text{ as } h \rightarrow 0\\
	\end{align*}
	The case is similar for $h \in (x_0 - b, 0)$. Since the map $h \mapsto f(x_0)h$ is bounded, $F$ is Frechet differentiable at $x_0$ and $DF(x_0)(h) = f(x_0)h$. This implies that $F'(x_0) = f(x_0)$ and a previous exercise implies tells us that continuity of $f$ implies continuity of $DF$. So $F \in C^1(A, Y)$.
	\end{proof}
	
	\begin{ex} \lex{}\tbf{Fundamental Theorem of Calculus:}
	Let $Y$ be a separable Banach space and $f \in C^1((a,b), Y)$. Then for each $x, x_0 \in (a,b)$, $x_0 < x$ implies that 
	\begin{enumerate}
	\item $f'$ is Bochner integrable on $(x_0, x]$ 
	\item  $$f(x) - f(x_0) = \int_{(x_0, x]}f'dm$$ 
	\end{enumerate}
	\end{ex}

	\begin{proof}\
	\begin{enumerate}
	\item Since $f \in C^1((a,b), Y)$, a previous exercise tells us that $f' \in C_Y(a,b)$. Let $x, x_0 \in (a,b)$. Suppose that $x_0 < x$. Choose $c,d \in (a,b)$ such that $a < c < x_0< x< d < b$. Then $f'$ is continuous on $[c,d]$ and hence Bochner-integrable on $(c,d]$ and $(x_0,x]$. 
	\item Define $g: (c,d) \rightarrow Y$ by $$g(\xi) = \int_{(c, \xi]}f'dm$$
	Then the previous exercise implies that $g \in C^1_Y(c,d)$ and for each $t \in (c, d)$, $g'(t) = f'(t)$. Let $t \in (c,d)$ and $h \in \R$. Then
	\begin{align*}
	Dg(t)(h) 
	&= hg'(t) \\
	&= hf'(t) \\
	&= Df(t)(h)
	\end{align*}
	So $Dg = Df$ on $(c,d)$. A previous exercise implies that there exists $c \in Y$ such that $f = g + c$ on $(c, d)$. Then 
	\begin{align*}
	f(x) - f(x_0)
	&= g(x)+c - (g(x_0)+c) \\
	&= g(x) - g(x_0) \\
	&= \int_{(c, x]}f'dm - \int_{(c, x_0]}f'dm\\
	&= \int_{(x_0, x]}f'dm
	\end{align*}
	\end{enumerate}
	\end{proof}
	
	
	\begin{ex} \lex{}
	Let $Y$ be a Banach space, $A \subset \R$ open and $g:A \rightarrow Y$. If $g$ is $n$-th order Frechet differentiable, then 
	$$\dv{t} \sum_{k=0}^{n-1} \frac{(1-t)^k}{k!}g^{(k)}(t) = \frac{(1-t)^{n-1}}{(n-1)!}g^{(n)}(t)$$
	\end{ex}
	
	\begin{proof}
	Taking the derivative yields a telescoping series.
	\end{proof}
	
	
	
	\begin{ex} \lex{} \tbf{Taylor's Theorem I:}\\
	Let $X$ be a Banach space, $Y$ a separable Banach space, $A \subset X$ open and convex, $f\in C^{n+1}(A, Y)$, $x_0 \in A$, and $h \in A - x_0$. Then $$f(x_0 + h) = \sum_{k=0}^{n} \frac{1}{k!} D^k f(x_0)(h^{\oplus k}) + R(x_0)(h)$$ 
	where $R(x_0): A - x_0 \rightarrow Y$ is defined by $$R(x_0)(h) = \frac{1}{n!}\int_{(0,1)} (1-t)^{n}D^{n+1}f(x_0 + th)(h^{\oplus (n+1)})d m(t)$$
	and $R(x_0)(h) = o(\|h\|^{n})$ as $h \rightarrow 0$.\\
	\tbf{Hint:} Define $g: (0,1) \rightarrow Y$ by $$g(t) = f(x_0 +t h)$$ Then use the previous exercise and the fundamental theorem of calculus.
	\end{ex}
	
	\begin{proof}
	Let $h \in X$. Suppose that $x_0 + h \in A$. Define $g: (0,1) \rightarrow Y$ by 
	$$g(t) = f(x_0 +t h)$$ 
	For each $k \in \{1, \dots, n+1\}$, a previous exercise implies that $g^{(k)}(t) = D^kf(x_0 + th)(h^{\oplus k})$, so $g^{(k)}(0) = D^kf(x_0)(h^{\oplus k})$. The previous exercise and the fundamental theorem of calculus tell us that 
	\begin{align*}
	f(x_0 +h) - \sum_{k=0}^{n} \frac{1}{k!}D^kf(x_0)(h^{\oplus k})
	&= g(1) - \sum_{k=0}^{n} \frac{1}{k!}g^{(k)}(0)\\
	&= \int_{(0,1)} \bigg [\dv{t} \sum_{k=0}^{n} \frac{(1-t)^k}{k!}g^{(k)}(t)\bigg ] dm(t) \\
	&= \int_{(0,1)} \frac{(1-t)^{n}}{n!}g^{(n+1)}(t) dm(t) \\
	&= \frac{1}{n!}\int_{(0,1)} (1-t)^{n}D^{n+1}f(x_0 + th)(h^{\oplus (n+1)})d m(t) \\
	&= R(x_0)(h)
	\end{align*}	
	Note that $$\frac{1}{n+1} = \frac{1}{n!}\int_{(0,1)} (1-t)^{n} dm(t)$$ 
	Since $D^{n+1}f$ is continuous at $x_0$, there exists $\del_1 >0$ such that for each $h \in B(0, \del_1)$, $x_0 + h \in A$ and 
	$$\|D^{n+1} f(x_0+h) - D^{n+1}f(x_0)\| < 1 $$  
	Let $\ep >0$. Choose $\del_2 >0$ such that $$\frac{1}{n+1} \bigg( \|D^{n+1}f(x_0 )\|  +  1 \bigg) \del_2 < \ep$$ Set $\del = \min(\del_1, \del_2)$. Let $h \in B(0, \del)$. Then
	\begin{align*}
	\|R(x_0)(h)\| 
	&= \bigg \| \int_{(0,1)} \frac{1}{n!}\int_{(0,1)} (1-t)^{n}D^{n+1}f(x_0 + th)(h^{\oplus (n+1)})d m(t) \bigg\| \\
	&\leq \frac{1}{n!}\int_{(0,1)} \|(1-t)^{n}D^{n+1}f(x_0 + th)(h^{\oplus (n+1)}) \|dm(t)\\
	&\leq \max_{t \in [0,1]}\|D^{n+1}f(x_0 + th)\| \|h\|^{n+1} \frac{1}{n!}\int_{(0,1)} (1-t)^{n} dm(t)  \\
	&\leq \frac{1}{n+1}  \bigg(\|D^{n+1}f(x_0 )\| +  \max_{t \in [0,1]} \|D^{n+1} f(x_0+th) - D^{n+1}f(x_0)\| \bigg)\|h\|^{n+1}  \\
	&< \frac{1}{n+1}\bigg(\|D^{n+1}f(x_0 )\|  +  1 \bigg)\|h\|^{n+1}  \\
	&<\ep \|h\|^n
	\end{align*}
	So $R(x_0)(h) = o(\|h\|^{n})$ as $h \rightarrow 0$.
	\end{proof}
	
	
	\begin{ex} \lex{} \tbf{Taylor's Theorem II:}\\
	Let $X$ be a Banach space, $Y$ a separable Banach space, $A \subset X$ open and convex, $f\in C^{n}(A, Y)$, $x_0 \in A$, and $h \in A - x_0$. Then there exists $R(x_0): A - x_0 \rightarrow Y$ such that $$f(x_0 + h) = \sum_{k=0}^{n} \frac{1}{k!} D^k f(x_0)(h^{\oplus k}) + R(x_0)(h)$$ and $R(x_0)(h) = o(\|h\|^n)$ as $ h \rightarrow 0$. \\
	\tbf{Hint:} use Taylor's theorem and expand the derivative inside the integral.
	\end{ex}
	
	\begin{proof}
	This is clear by definition for $n = 1$. Suppose that $n \geq 2$. Taylor's theorem implies that $$f(x_0 + h) = \sum_{k=0}^{n-2} \frac{1}{k!} D^k f(x_0)(h^{\oplus k}) + S(x_0)(h)$$ 
	where $S(x_0): A - x_0 \rightarrow Y$ is defined by 
	$$S(x_0)(h) = \frac{1}{(n-2)!}\int_{(0,1)} (1-t)^{n-2}D^{n-1}f(x_0 + th)(h^{\oplus (n-1)})d m(t)$$
	
	and $S(x_0; h) = o(\|h\|^{n})$ as $h \rightarrow 0$.
	Define $T^{n-1}(x_0):A-x_0 \rightarrow L^{n-1}(X;Y)$ by 
	$$T^{n-1}(x_0)(h) = D^{n-1}f(x_0 + h) - D^{n-1}f(x_0) - D^nf(x_0)(h)$$ 
	so that 
	$$D^{n-1}f(x_0 + h) = D^{n-1}f(x_0) + D^nf(x_0)(h) + T^{n-1}(x_0)(h)$$ 
	and $T^{n-1}(x_0)(h) = o(\|h\|)$ as $h \rightarrow 0$. \\
	Define $R(x_0): A - x_0 \rightarrow Y$ by 
	$$R(x_0)(h) = \frac{1}{(n-2)!} \int_{(0,1)} (1-t)^{n-2}T^{n-1}(x_0)(th)(h^{\oplus (n-1)}) dm(t) $$
	Note that 
	\begin{itemize}
	\item $$\int_0^1 (1-t)^{n-2} dt = \frac{1}{n-1}$$
	\item $$\int_0^1 (1-t)^{n-2}t dt = \frac{1}{n(n-1)}$$
	\end{itemize}
	Let $\ep >0$. Choose $\del >0$ such that for each $h \in B(0, \del)$, $h \in A - x_0$ and $$\|T^{n-1}(x_0)(h)\| \leq \ep n! \|h\|$$ Let $h \in B(0, \del)$. Then 
	\begin{align*}
	\|R(x_0)(h)\|
	&= \bigg \|  \frac{1}{(n-2)!} \int_{(0,1)} (1-t)^{n-2}T^{n-1}(x_0)(th)(h^{\oplus (n-1)}) dm(t) \bigg \| \\
	& \leq  \frac{1}{(n-2)!} \int_{(0,1)} (1-t)^{n-2}\|T^{n-1}(x_0)(th)(h^{\oplus (n-1)})\| dm(t) \\
	& \leq  \frac{1}{(n-2)!} \int_{(0,1)} (1-t)^{n-2} \|T^{n-1}(x_0)(th) \| \|h\|^{n-1} dm(t) \\
	& \leq  \frac{\ep}{(n-2)!}  n!  \|h\|^n \int_{(0,1)} (1-t)^{n-2} t  dm(t) \\		
	&= \ep \|h\|^n  
	\end{align*}
	So that $R(x_0)(h) = o(\|h\|^n)$ as $h \rightarrow 0$. 
	
	Then 
	\begin{align*}
	S(x_0)(h) 
	&= \frac{1}{(n-2)!}\int_{(0,1)} (1-t)^{n-2}D^{n-1}f(x_0 + th)(h^{\oplus (n-1)})d m(t) \\
	&= \frac{1}{(n-2)!}\int_{(0,1)} (1-t)^{n-2} D^{n-1}f(x_0)(h^{\oplus (n-1)}) dm(t) \\ 
	& \hspace{1.5cm} + \frac{1}{(n-2)!} \int_{(0,1)} (1-t)^{n-2}t D^{n}f(x_0)(h)(h^{\oplus (n-1)}) dm(t) \\
	& \hspace{1.5cm} + \frac{1}{(n-2)!} \int_{(0,1)} (1-t)^{n-2}T^{n-1}(x_0)(th)(h^{\oplus (n-1)}) dm(t) \\
	&= \frac{1}{(n-1)!}D^{n-1}f(x_0)(h^{\oplus (n-1)}) + \frac{1}{n!}D^{n}f(x_0)(h^{\oplus n}) + R_f(x_0)(h)
	\end{align*}
	Hence 
	\begin{align*}
	f(x_0 + h) 
	&= \sum_{k=0}^{n-2} \frac{1}{k!} D^k f(x_0)(h^{\oplus k}) + S(x_0)(h) \\
	&= \sum_{k=0}^{n} \frac{1}{k!} D^k f(x_0)(h^{\oplus k}) + R(x_0)(h) 
	\end{align*}
	
	\end{proof}
	
	
	\begin{ex} \lex{} \tbf{Taylor's Theorem III:}\\
	Let $X$ be a Banach space, $A \subset X$ open and convex, $f \in C^{n}(A)$, $x_0 \in A$, and $h \in A - x_0$. Then there exists $t^* \in (0,1)$ such that $$f(x_0 + h) = \sum_{k=0}^{n-1} \frac{1}{k!} D^k f(x_0)(h^{\oplus k}) + \frac{1}{(n-1)!} (1-t^*)^{n-1}D^{n}f(x_0 + t^*h)(h^{\oplus n})$$ \\
	\tbf{Hint:} use Taylor's theorem and the mean value theorem.
	\end{ex}	
	
	\begin{proof}
	Taylors Theorem implies that 
	$$f(x_0 + h) = \sum_{k=0}^{n-1} \frac{1}{k!} D^k f(x_0)(h^{\oplus k}) + R(x_0)(h)$$ 
	where 
	$$R(x_0)(h) =  \frac{1}{(n-1)!}\int_{(0,1)} (1-t)^{n-1}D^{n}f(x_0 + th)(h^{\oplus n})d m(t)$$ 
	Define $F \in C^1([0,1])$ by $$F(t) = \int_{(0,t]}\frac{1}{(n-1)!}(1-s)^{n-1}D^{n}f(x_0 + sh)(h^{\oplus n})d m(s)$$ Then the fundamental theorem of calculus implies that 
	$$F'(t) = \frac{1}{(n-1)!}(1-t)^{n-1}D^{n}f(x_0 + th)(h^{\oplus n})$$ 
	The mean value theorem implies that there exists $t^* \in (0,1)$ such that 
	\begin{align*}
	R(x_0)(h)
	&= F(1) - F(0) \\
	&= F'(t^*) \\
	&= \frac{1}{(n-1)!}(1-t^*)^{n-1}D^{n}f(x_0 + t^*h)(h^{\oplus n})
	\end{align*}
	\end{proof}	

	
	
	
	\begin{ex} \lex{}
	Let $X$ be a Banach space, $A \subset X$ open and convex and $f\in C^{2}(A)$, $x_0 \in A$. If $f$ has a local minimum at $x_0$, then $D^2f(x_0)$ is positive semidefinite.   
	\end{ex}
	
	\begin{proof}
	Suppose that $f$ has a local minimum at $x_0$, then $Df(x_0) = 0$. Let $x \in X$. Then 
	\begin{align*}
	0 
	& \leq f(x+h) - f(x_0) \\
	&= \frac{1}{2} D^2f(x_0)(h,h) +o(\|h\|^2) \hspace{.2cm} \text{ as } h \rightarrow 0
	\end{align*}
	Let $h \in X$. Then $$0 \leq \frac{1}{2} t^2 D^2f(x_0)(h,h) +o(t^2) \hspace{.2cm} \text{ as } t \rightarrow 0$$
	This implies that $D^2f(x_0)(h,h) \geq 0$. So $D^2f(x_0)$ is positive semidefinite.
	\end{proof}





























	\newpage
	\subsection{Implicit and Inverse Function Theorems}
	\begin{defn}
		Let $(x_0,y_0) \in U$. Then $f$ is said to be \tbf{partial Frechet differentiable with respect to $X$ at $(x_0, y_0)$} if $f^{y_0}$ is Frechet differentiable at $x_0$. \\
		Suppose that $f$ is partial Frechet differentiable with respect to $X$ at $(x_0, y_0)$. We define the \tbf{partial Frechet derivative of $f$ with respect to $X$ at $(x_0, y_0)$}, denoted $D_X f(x_0, y_0) \in L(X, Z)$, by 
		$$D_Xf(x_0, y_0) = Df^{y_0}(x_0)$$
		Suppose that for each $y \in Y$, $f^{y}$ is Frechet differentiable. We define the \tbf{partial Frechet derivative of $f$ with respect to $X$}, denoted $D_X f: X \times Y \rightarrow L(X, Z)$, by 
		$$D_Xf(x, y) = Df^{y}(x)$$
		We define partial Frechet differentiability with respect to $Y$ similarly.
	\end{defn}

	\begin{ex}
		Let $X, Y$ and $Z$ be Banach spaces, $f: X \times Y \rightarrow Z$ and $(x_0, y_0) \in X \times Y$. If $f$ is Frechet differentiable at $(x_0, y_0)$, then $f$ is partial Frechet differentiable at $(x_0, y_0)$ with respect to $X$ and $Y$ and for each $h_X \in X$, $h_Y \in Y$,  
		$$Df(x_0, y_0)(h_X, h_Y) = D_Xf(x_0, y_0) (h_X) + D_Yf(x_0, y_0) (h_Y)$$
	\end{ex}

	\begin{proof}
		Suppose that $f$ is Frechet differentiable at $(x_0, y_0)$. Then 
		$$f[(x_0,y_0) + (h_X, h_Y)] = f(x_0, y_0) + Df(x_0, y_0)(h_x, h_Y) + o(\|(h_X, h_Y)\|_{X \oplus Y}) \text{ as } (h_X, h_Y) \rightarrow (0,0)$$ Since there exist $C_1, C_2 > 0$ such that for each $h_X \in X$ and $h_Y \in Y$, $C_1(\|x\| + \|y\|) \leq \|(h_x, h_y)\|_{X \oplus Y} \leq C_2(\|x\| + \|y\|)$, we have that 
		$$f^{y_0}(x_0 + h_Y) = f^{y_0}(x_0) + Df(x_0, y_0)(h_X, 0) + o(\|h_X\|) \text{ as } h_X \rightarrow 0$$
		Therefore $f^{y_0}:X \rightarrow Z$ is Frechet differentiable at $x_0$ and $Df^{y_0}(x_0) = Df(x_0, y_0)(h_X, 0)$. Hence $f$ is partial Frechet differentiable at $(x_0, y_0)$ with respect to $X$ and for each $h_X \in X$, $D_Xf(x_0, y_0)(h_x) = Df(x_0, y_0)(h_X, 0)$. Similarly, $f$ is partial Frechet differentiable at $(x_0, y_0)$ with respect to $Y$ and for each $h_Y \in Y$, $D_Yf(x_0, y_0)(h_Y) = Df(x_0, y_0)(0, h_Y)$. Let $h_X \in X$ and $h_Y \in Y$. Then
		\begin{align*}
			Df(x_0, y_0)(h_X, h_Y) 
			& = Df(x_0, y_0)[(h_X, 0) + (0, h_Y)] \\
			& = Df(x_0, y_0)(h_X, 0) + Df(x_0, y_0)(0, h_Y) \\
			& = D_Xf(x_0, y_0)(h_x) +  D_Yf(x_0, y_0)(h_Y)
		\end{align*}
	\end{proof}

	\begin{ex}
		Let $X, Y$ and $Z$ be Banach spaces, $U \subset X \times Y$ open, $f: U \rightarrow Z$ and $n \in \N$. If $f$ is $C^1(U, Z)$, then $D_Xf$,$D_Yf \in C(U, Z)$.
	\end{ex}

	\begin{proof}
		Suppose that $f$ is $C^1(U, Z)$. Then $Df \in C(U, Z)$. Define $\phi_X:X \rightarrow X \times Y$ and $\phi_Y:Y \rightarrow X \times Y$ by $\phi_X(x) = (x, 0)$ and $\phi_Y(y) = (0, y)$. Then $\phi_X \in L(X, X \times Y)$ and $\phi_Y \in L(Y, X \times Y)$. The previous exercise implies that for each $(x,y) \in U$, $D_Xf(x,y) = Df(x, y) \circ \phi_X $. Let $(x,y)$, $(x_0, y_0) \in U$. Then 
		\begin{align*}
			\|D_Xf(x,y) - D_Xf(x_0, y_0)\|
			& = \|Df(x, y) \circ \phi_X - Df(x_0, y_0) \circ \phi_X\| \\
			& = \|(Df(x, y) - Df(x_0, y_0)) \circ \phi_X\| \\
			& \leq \|Df(x, y) - Df(x_0, y_0) \| \|\phi_X\| \\
		\end{align*}
	\end{proof}
	

	\begin{ex}
		Let $X, Y$ and $Z$ be Banach spaces, $U \subset X \times Y$ open, $F: U \rightarrow Z$, $(x_0, y_0) \in U$. Suppose that $F$ is partial Frechet differentiable with respect to $Y$ on $U$ and $F$ and $D_YF$ continuous at $(x_0, y_0)$. Then there  
	\end{ex}

	\begin{proof}
		Set $L = D_YF(x_0, y_0)$. Define $G: U \rightarrow Z$ by $G(x,y) = y - L^{-1}F(x,y)$. Then $G(x_0, y_0) = y_0$ and since $F \in C^1(U, Z)$, $G \in C^1(U, Z)$. The previous exercise implies that $D_YG \in C(U, Z)$. Note that for each $(x,y) \in U$,
		\begin{align*}
			D_YG(x, y) 
			& = \id_Y - L^{-1} D_YF(x,y) \\
			& = L^{-1}(L - D_YF(x,y))
		\end{align*} 
		which implies that $D_YG(x_0, y_0) = 0$. Set $\ep = 1/2$. Since $U$ is open and $D_YG$ is continuous at $(x_0, y_0)$ there exist $\del_X$, $\del_Y > 0$ such that for each $x \in B(x_0, \del_X)$ and $y \in B(y_0, \del_Y)$, $(x, y) \in U$ and  
		\begin{align*}
			\|D_YG(x, y)\| 
			& =  \|D_YG(x, y) - D_YG(x_0, y_0)\| \\
			& < \ep
		\end{align*}
		Set $A = B(x_0, \del_X)$ and $B = B(y_0, \del_Y)$. Let $x \in A$ and $y_1, y_2 \in B$. Define $l: [0,1] \rightarrow B$ by $l(t) = ty_1 + (1-t)y_2$. The mean value theorem implies that 
		\begin{align*}
			\|G(x, y_1) - G(x, y_2)\| 
			& \leq \sup_{t \in [0,1]} \|D_YG(x, l(t))\|\|y_1 - y_2\| \\
			& \leq \ep \|y_1 - y_2\| \\
			& = \frac{1}{2}\|y_1 - y_2\| 
		\end{align*} 
		Hence, for each $x \in X$ and $y \in Y$, 
		$\|G(x, y)\| \leq \frac{1}{2}\|y_1 - y_2\|$  
		For $x \in A$, define $T_x: B \rightarrow B$ by $T_x(y) = G(x,y)$. 
	\end{proof}


	
	
	
	
	
	
	
	
	
	
	
	
	
	
	
	
	
	
	
	
	
	
	
	
	\newpage
	\subsection{The Gradient}
	
	\begin{defn} \ld{}
	Let $H$ be a Hilbert space, $f: H \rightarrow \C$ and $x_0 \in H$. Suppose that $f$ is Frechet differentiable at $x_0$. Then $Df(x_0) \in H^*$. We define the \tbf{gradient of $f$ at $x_0$}, denoted $\nabla f(x_0) \in H$, via the Riesz representation theorem to be the unique element of $H$ satisfying $$\l \nabla f(x_0), y \r = Df(x_0)(y) \hspace{.3cm} \text{ for each } y \in H$$
	\end{defn}
	
	
	
	
	
	
	
	
	
	
	

	
	
	
	
	
	
	\newpage
	\section{Convexity}
	
	\subsection{Introduction}

	\begin{note}
	In this section, we assume all vector spaces are real.
	\end{note}

	\begin{defn} \ld{91001}
	Let $X$ be a vector space and $A \subset X$. Then $A$ is said to be $\tbf{convex}$ if for each $x, y \in A$, and $t \in [0,1]$,  $tx + (1-t)y \in A$. 
	\end{defn}	
	
	\begin{defn} \ld{91002}
	Let $X$ be a vector space and $f:A \rightarrow R$. Then $f$ is said to be \tbf{convex} if for each $x,y \in A$, $t \in \ui$, $$f(tx + (1-t)y) \leq tf(x) + (1-t)f(y)$$
	\end{defn}
	
	\begin{defn} \ld{91003}
	Let $X$ be a vector space and $f:A \rightarrow R$. Then $f$ is said to be \tbf{strictly convex} if for each $x,y \in A$, $t \in (0,1)$, $x \neq y$ implies that $$f(tx + (1-t)y) < tf(x) + (1-t)f(y)$$
	\end{defn}
	
	\begin{ex} \lex{91004}
	Let $X$ be a vector space, $f \in X^*$ and $g: X \rightarrow \R$ constant. Then $f$ and $g$ are convex. 
	\end{ex}
	
	\begin{proof}
		Let $x, y \in X$ and $t \in \ui$. Put $c = g(0)$. Then $$f(tx + (1-t)y) = tf(x) + (1-t)f(y)$$ and 
		\begin{align*}
		g(tx + (1-t)y) 
		&= c\\ 
		&= tc + (1-t)c \\
		&= tg(x) + (1-t)g(y)
		\end{align*}
		So $f$ and $g$ are convex.
	\end{proof}		

	\begin{ex} \tbf{Star-shapedness:}
		Let $f:\Rg \rightarrow \R$ be convex. If $f(0) \leq 0$, then for each $x \in \Rg$, $t \in [0,1]$, $f(tx) \leq tf(x)$.
	\end{ex}

	\begin{proof}
		Suppose that $f(0) \leq 0$. Let $x \in \Rg$ and $t \in [0,1]$. Then 
		\begin{align*}
			f(tx)
			&= f(tx + (1-t)0) \\
			& \leq tf(x) + (1-t)f(0) \\
			& \leq tf(x)
		\end{align*}
	\end{proof}

	\begin{ex} \tbf{Superadditivity:}\\
		Let $f:\Rg \rightarrow \Rg$ be convex. If $f(0) = 0$, then for each $x,y \in \Rg$, $$f(x) + f(y) \leq f(x+y)$$
		\tbf{Hint:} $f(x) = f \bigg( \frac{x}{x+y}(x+y)\bigg)$
	\end{ex}

	\begin{proof}
		Suppose that $f(0) = 0$. Let $x, y \in \Rg$. If $x+y = 0$, then $x=y=0$ and $f(x) + f(y) = 0 = f(x+y)$. Suppose that $x+y \neq 0$. Then the previous exercise implies that 
		\begin{align*}
			f(x) + f(y) 
			&= f \bigg( \frac{x}{x+y}(x+y)\bigg) + f \bigg( \frac{y}{x+y}(x+y)\bigg) \\
			& \leq \frac{x}{x+y}f(x+y) + \frac{y}{x+y}f(x+y) \\
			&= f(x+y)
		\end{align*}
	\end{proof}
	
	\begin{ex} \lex{91005}
	Let $X$ be a vector space, $A \subset X$ convex, $f,g:A \rightarrow \R$ and $\lam \geq 0$. If $f,g$ are convex, then 
	\begin{enumerate}
	\item $f + g$ is convex 
	\item $\lam f$ is convex
	\end{enumerate}
	\end{ex}
	
	\begin{proof}
	Suppose that $f$ and $g$ are convex. Let $x,y \in A$ and $t \in [0,1]$. Then 
	\begin{align*}
	(f + \lam g)(tx + (1-t)y) 
	&= f(tx + (1-t)y) + \lam g(tx + (1-t)y) \\
	& \leq tf(x) + (1-t)f(y) +  t \lam g(x) + (1-t)\lam g(y) \\
	&= t(f(x) + \lam g(x)) + (1-t)(f(y) + \lam g(y))\\
	& = t(f + \lam g)(x) + (1-t)(f + \lam g)(y)
\end{align*}		 
	\end{proof}
	
	
	\begin{defn} \ld{91006}
	Let $X$ be a vector space and $f: X \rightarrow \R$. Then $f$ is said to be \tbf{affine} if there exists $\phi \in X^*$, $a \in \R$ constant such that $f = \phi + a$.\\
	\end{defn}
	
	\begin{ex} \lex{91007}
	Let $X$ be a vector space and $f: X \rightarrow \R$. If $f$ is affine, then $f$ is convex.
	\end{ex}
	
	\begin{proof}
	Suppose that $f$ is affine. Then there exists $\phi \in X^*$, $a \in R$ constant such that $f = \phi + a$. Then $\phi$ is convex and $g: X \rightarrow \R$ defined by $g(x) = a$ is convex. So $f = \phi + g$ is convex.
	\end{proof}
	
	\begin{ex} \lex{91008}
	Let $X$ be a vector space, $A \subset X$ convex, $f:\R \rightarrow \R$ and $g: A \rightarrow \R$. If $f$ is convex and increasing and $g$ is convex, then $f \circ g$ is convex.
	\end{ex}	
	
	\begin{proof}
	Let $t \in [0,1]$ and $x, y \in A$. Then convexity of $g$ implies that $$g(tx +(1-t)y) \leq tg(x) + (1-t)g(y)$$ and we have
	\begin{align*}
	f\circ g(tx +(1-t)y) 
	&= f(g(tx +(1-t)y)) \\
	& \leq f(tg(x) + (1-t)g(y)) \hspace{2cm} (f \text{ increasing)}\\
	& \leq tf(g(x)) + (1-t)f(g(y)) \hspace{2cm}  (f \text{ convex)}\\	
	&= tf \circ g(x) + (1-t)f \circ g(y)
\end{align*}	 
So $f \circ g$ is convex.
	\end{proof}
	
	\begin{ex} \lex{91009}
	Let $X$ be a vector space, $A \subset X$ convex, $f:A \rightarrow \R$ convex and $x_0 \in A$. Then $f$ has a local minimum point at $x_0$ iff $f$ has a global minimum point at $x_0$.
	\end{ex}	
	
	\begin{proof}
	If $f$ has a global minimum point at $x_0$, then $f$ has a local minimum point at $x_0$. Conversely, suppose that $f$ has a local minimum point at $x_0$. Then there exists $\del >0$ such that for each $x \in B(x_0, \del) \cap A$, $f(x_0) \leq f(x)$. For the sake of contradiction, suppose that $f$ does not have a global minimum point at $x_0$. Then there exits $x' \in A$ such that $f(x') < f(x_0)$. Put $t_0 = \min(\frac{\del}{\|x' - x_0\| + 1}, 1) >0$. Let $t \in (0, t_0)$, then
	\begin{align*}
	\|(tx' + (1-t)x_0) - x_0\| 
	&= t\|x' -x_0 \| \\
	& <   \frac{\|x' -x_0 \|\del}{\|x' -x_0\| + 1} \\
	& < \del
	\end{align*} 
	so that $tx' + (1-t)x_0 \in B(x_0, \del) \cap A$ and hence $f(x_0) \leq f(tx' + (1-t)x_0)$.  Therefore  
	\begin{align*}
	f(x_0) 
	& \leq f(tx' + (1-t)x_0) \\
	& \leq tf(x') + (1-t)f(x_0)  \hspace{.5cm} (\text{convexity of }f)\\
	& < tf(x_0) + (1-t)f(x_0) \\
	&= f(x_0)
	\end{align*}
	which is a contradiction. Hence $f$ has a global minimum point at $x_0$.
	\end{proof}
	
	\begin{ex} \lex{91010}
	Let $X$ be a vector space, $A \subset X$ convex, $f:A \rightarrow \R$ strictly convex and $x_0 \in X$. If $f$ has a local minimum point at $x_0$, then $f$ has a unique global minimum point at $x_0$.  
	\end{ex}
	
	\begin{proof}
	Suppose that $f$ has a local minimum point at $x_0$. The previous exercise implies that $f$ has a global minimum point at $x_0$. For the sake of contradiction suppose that there exists $x_1 \in X$ such that $f$ has a global minimum point at $x_1$ and $x_0 \neq x_1$. This implies $f(x_0) = f(x_1)$. Set $t = 1/2$. Strict convexity implies that 
	\begin{align*}
	f(tx_0 + (1-t)x_1) 
	&< tf(x_0) + (1-t)f(x_1)  \\
	&= f(x_0) 
	\end{align*}
	which is a contradiction since $f$ has a global minimum point at $x_0$.
	\end{proof}

	
	\begin{defn} \ld{91011}
	Let $X, Y$ be vector spaces, $A \subset X \oplus Y$. For $y \in Y$, define $$A^y = \{x \in X: (x,y) \in A \}$$ and $f^y:A^y \rightarrow \R$ by $$f^y(x) = f(x,y)$$
	\end{defn}
	
	\begin{ex} \lex{91012}
	Let $X, Y$ be vector spaces, $A \subset X \oplus Y$ convex and $f:A \rightarrow \R$ convex. Then for each $y \in \pi_2(A)$,
	\begin{enumerate}
	\item $A^y$ is convex 
	\item $f^y$ is convex  
	\end{enumerate}	  
	where $\pi_2: X\times Y \rightarrow Y$, the canonical projection of $X \times Y$ onto $Y$ given by $\pi_2(x,y) = y$.
	\end{ex}
	
	\begin{proof}
	Let $y \in \pi_2(A)$, $x_1, x_2 \in A^y$ and $t \in [0,1]$. Then by definition, $(x_1, y)$, $(x_2, y) \in A$.
	\begin{enumerate}
	\item  Convexity of $A$ implies that $(tx_1 + (1-t)x_2, y) \in A$. Hence $tx_1 + (1-t)x_2 \in A^y$ and $A^y$ is convex. 
	\item Convexity of $f$ implies that 
	\begin{align*}
	f^y(tx_1 + (1-t)x_2)
	&= f(tx_1 + (1-t)x_2, y) \\
	&= f(t(x_1, y) + (1-t)(x_2,y)) \\
	& \leq tf(x_1, y) + (1-t) f(x_2,y) \\
	&= tf^y(x_1) + (t-t)f^y(x_2)
\end{align*}	  
	and so $f^y$ is convex.
	\end{enumerate}
	\end{proof}
	
	\begin{ex} \lex{91013}
	Let $X$, $Y$ be vector spaces and $A\subset X, B \subset Y$. If $A$ and $B$ are convex, then $A \times B \subset X \oplus Y$ is convex.
	\end{ex}	
	
	\begin{proof}
	Suppose that $A$ and $B$ are convex. Let $(x_1,y_1), (x_2,y_2) \in A \times B$ and $t \in [0,1]$. Convexity of $A$ and $B$ implies that $tx_1 + (1-t)x_2 \in A$ and $ty_1 + (1-t)y_2 \in B$. Therefore 
	\begin{align*}
	t(x_1,y_1) + (1-t)(x_2,y_2) 
	&= (tx_1 + (1-t)x_2, ty_1 + (1-t)y_2) \\
	& \in A \times B
\end{align*}	 
	\end{proof}
	
	\begin{ex} \lex{91014}
	Let $X, Y$ be vector spaces and $A \subset X$, $B \subset Y$ convex (implying that $A \times B$ is convex)  and $f:A \times B \rightarrow \R$ convex. Suppose that for each $y \in B$, $\{f(x, y): x \in A\}$ is bounded below. Then $\inf\limits_{y \in B}f^y$ is convex
	\end{ex}
	
	\begin{proof}
	Put $g = \inf\limits_{y \in B}f^y$. 
	Let $x_1, x_2 \in A$, $y_1, y_2 \in B$ and $t \in [0,1]$. Put $y'= ty_1 + (1-t)y_2$. Then convexity of $f$ implies that
	\begin{align*}
	g(tx_1 + (1-t)x_2) 
	& \leq f^{y'}(tx_1 + (1-t)x_2) \\
	&= f(tx_1 + (1-t)x_2, ty_1 + (1-t)y_2)\\
	&= f(t(x_1,y_1) + (1-t)(x_2, y_2)) \\
	& \leq tf(x_1, y_1) + (1-t)f(x_2, y_2) \\
	&= tf^{y_1}(x_1) + (1-t)f^{y_2}(x_2) \\
	\end{align*}
	Since $y_1 \in B$ is arbitrary, we have that $$g(tx_1 + (1-t)x_2) \leq tg(x_1) + (1-t)f^{y_2}(x_2)$$ Similarly, since $y_2 \in B$ is arbitrary, we have that $$g(tx_1 + (1-t)x_2) \leq tg(x_1) + (1-t)g(x_2)$$ and $f$ is convex.
	\end{proof}	

	\begin{ex} \lex{91015}
	Let $X$ be a vector space, $A \subset X$ convex and $ (f_{\lam})_{\lam \in \Lam} \subset \R^A$. Suppose that for each $\lam \in \Lam$, $f_\lam$ is convex. Define 
	\begin{enumerate}
		\item $A^* = \{x \in A: \sup\limits_{\lam \in \Lam} f_{\lam}(x) < \infty\}$
		\item $f^*:A^* \rightarrow \R$ by $f^*(x) =  \sup\limits_{\lam \in \Lam} f_{\lam}(x)$
	\end{enumerate}
Then
	\begin{enumerate}
		\item $A^*$ is convex 
		\item $f^*$ is convex
	\end{enumerate}
	\end{ex}
	
	\begin{proof}
		\begin{enumerate}
			\item Let $x, y \in A$ and $t \in [0,1]$. By definition, $ \sup\limits_{\lam \in \Lam} f_{\lam}(x)$,  $\sup\limits_{\lam \in \Lam} f_{\lam}(y) < \infty$. Therefore 
			\begin{align*}
				 \sup\limits_{\lam \in \Lam} f_{\lam}(tx + (1-t)y) 
				&\leq \sup\limits_{\lam \in \Lam} [tf_{\lam}(x) + (1-t)f_{\lam}(y) ]  \\
				& \leq t \sup\limits_{\lam \in \Lam} f_{\lam}(x) + (1-t) \sup\limits_{\lam \in \Lam} f_{\lam}(y) \\
				& < \infty
			\end{align*}
			So $tx + (1-t)y \in A$.
			\item  By definition, the previous part implies that for each $x,y \in A^*$, $f^*(tx + (1-t)y) \leq t f^*(x) + (1-t)f^*(y)$. So $f^*:A^* \rightarrow \R$ is convex.
		\end{enumerate}
	\end{proof}
	
	
	
	
	\begin{ex} \lex{91016}
	Let $X$ be a normed vector space, $A \subset X$ open and convex, $f:A \rightarrow \R$ convex and $x_0 \in A$. If $f$ is continuous at $x_0$, then $f$ is locally Lipschitz at $x_0$. \\
	\tbf{Hint:} Given $x_1, x_2$ near $x_0$ Choose a $z$ near $x_0$ s.t. $x_1$ is a convex combination of $x_2$ and $z$. Then repeat but with $x_2$ as a convex combination of $x_1$ and $z$
	\end{ex}
	
	\begin{proof}
	By continuity, $f$ is locally bounded at $x_0$. So there exist $M, \del >0$ such that $B(x_0, \del) \subset A$ and for each $x \in B(x_0, \del)$, $|f(x)| \leq M$. Put $\del' = \frac{\del}{2}$ and choose $U = B(x_0, \del')$. Then $U \subset A$ and $U \in \MN_{x_0}$. \\
	Let $x_1, x_2 \in U$. Suppose that $x_1 \neq x_2$. Define $\al = \|x_1 - x_2\| >0$, $p = \frac{\al}{\al + \del'}$, $q = 1-p$ and $z = p^{-1}(x_1 - qx_2)$. Then $x_1 = pz + qx_2$ and 
	\begin{align*}
	\|z - x_1\| 
	&= \|(p^{-1} - 1)x_1 - p^{-1}qx_2\| \\
	&= \frac{1-p}{p} \al \\
	&= \frac{\del'}{\al} \al \\
	&= \del ' 
	\end{align*}
	Therefore 
	\begin{align*}
	\|z - x_0\| 
	& \leq \|z - x_1\| + \|x_1 - x_0\| \\
	& <  \del '  + \del '  \\
	&= \del
\end{align*}	  
	So $z \in B(x_0, \del)$, which implies that 
	\begin{align*}
	f(z) - f(x_2) 
	& \leq |f(z) - f(x_2)|\\ 
	&\leq |f(z)| + |f(x_2)| \\
	&\leq 2M
\end{align*}		
	Since $x_1 = pz + qx_2$, convexity of $f$ implies that $f(x_1) \leq pf(z) + qf(x_2)$. Hence 
	\begin{align*}
	f(x_1) - f(x_2) 
	& \leq pf(z) -pf(x_2) \\
	&= p(f(z) - f(x_2)) \\
	& \leq p 2M \\
	&= \frac{\al}{\al + \del'} 2M \\
	& \leq \al 2M \\
	&= 2M \|x_1 - x_2 \|
	\end{align*}
	Similarly, choosing $z = p^{-1}(x_2 - qx_1)$, yields $f(x_2) - f(x_1) \leq 2M \|x_1 - x_2 \|$ which implies that $$|f(x_1) - f(x_2)| \leq 2M \|x_1 - x_2 \|$$ and $f$ is Lipschitz on $U$. 
 	\end{proof}


	
	
	
	
	
	
	
	
	
	
	
	
	
	
	
	
	\newpage
	\subsection{The Subdifferential}
	
	\begin{ex} \lex{}
	Let $X$ be a Banach space, $A \subset X$ open and convex, $f:A \rightarrow \R$ convex, $x_0 \in A$ and $x \in X$. Define $T = \{ t \in \R: x_0+tx \in A\}$. Then there exist $a, b \in (0, \infty]$ such that $T = (-a, b)$.
	\end{ex}
	
	\begin{proof}
	Continuity of scalar multiplication and addition implies that $T$ is an open neighborhood of $0$. Let $t > 0$ and $s \in [0,t]$. Then $\frac{s}{t} \in [0, 1]$ and by convexity of $A$, $x_0 + tx \in A$ implies that
	\begin{align*}
	x_0 + sx 
	&= \frac{s}{t}(x_0 + tx) + \bigg(1-\frac{s}{t} \bigg )x_0\\
	& \in A
	\end{align*} 
	Thus $[0,t] \subset T$. Similarly, $x_0 - tx \in A$ implies that $[-t, 0] \subset T$. \\
	Define $a,b \in (0, \infty]$ by $a = \sup \{t > 0: x_0 -tx \in A\}$ and $b = \sup\{t > 0: x_0 +tx \in A\}$. Then $(-a, b) = T$.
	\end{proof}
	
	\begin{defn} \ld{}
	Let $X$ be a Banach space, $A \subset X$ open and convex, $f:A \rightarrow \R$ convex, $x_0 \in A$ and $x \in X$. Define $T$ as in the previous exercise and choose $t_0 >0$ such that $(-t_0, t_0) \subset T$. For $t \in (0,t_0)$, define the difference quotient $q: (-t_0, t_0) \setminus \{0\} \rightarrow \R$ by$$q(t) = \frac{f(x_0 + tx) - f(x_0)}{t}$$ 
	\end{defn}	
	
	\begin{ex} \lex{}
	Let $X$ be a Banach space, $A \subset X$ open and convex, $f:A \rightarrow \R$ convex, $x_0 \in A$ and $x \in X$. Define $t_0$ as above.
	Then
	\begin{enumerate}
	\item $q(t)$ is increasing on $(0, t_0)$
	\item $q(-t)$ decreasing on $(0, t_0)$
\end{enumerate}	 
	 \tbf{Hint:} As an example, look at the graph of $f(x) = x^2$. For the algebra, start at the desired end inequality and work backwards
	\end{ex}	
	
	\begin{proof}\
	\begin{enumerate}
	\item Let $s, t \in (0, t_0)$ and suppose that $s \leq t$. Then $x_0 +sx$, $x_0 + tx \in A$. Note that since $0 < s \leq t$, $\frac{s}{t} \in (0, 1]$ and $1- \frac{s}{t} = \frac{t-s}{t} \in (0, 1]$. Also, since $A$ is convex, we have that $$ \bigg( \frac{t-s}{t} \bigg) x_0 +  \bigg(\frac{s}{t} \bigg) (x_0 + tx)  \in A$$
	Convexity of $f$ implies that 
	\begin{align*}
	f(x_0 + sx)
	&= f\bigg ( \bigg( \frac{t-s}{t} \bigg) x_0 +  \bigg(\frac{s}{t} \bigg) (x_0 + tx) \bigg) \\
	& \leq \bigg( \frac{t-s}{t} \bigg) f(x_0) + \bigg(\frac{s}{t} \bigg) f(x_0 + tx)
	\end{align*}
	This implies that $$tf(x_0 + sx) \leq (t-s) f(x_0) + s f(x_0 + tx)$$
	and after rearranging, we get $$t f(x_0 + sx) - tf(x_0) \leq s f(x_0 + tx) - sf(x_0)$$
	and so finally, dividing both sides by $st$, we obtain 
	\begin{align*}
	q(s)
	&= \frac{f(x_0 + sx) - f(x_0)}{s} \\
	& \leq \frac{f(x_0 + tx) - f(x_0)}{t} \\
	&= q(t) 
	\end{align*}
	as desired.
	\item Similar to $(1)$.
	\end{enumerate}
	\end{proof}
	
	\begin{ex} \lex{}
	Let $X$ be a Banach space, $A \subset X$ open and convex, $f:A \rightarrow \R$ convex, $x_0 \in A$ and $x \in X$. Define $t_0$ as before. Then for each $t \in (0, t_0)$, $$q(-t) \leq q(t)$$ \\
	\tbf{Hint:} for sufficiently small $t$, convexity of $f$ implies that $f(x_0) \leq \frac{1}{2} f(x_0 - 2tx) + \frac{1}{2} f(x_0 + 2tx)$
	\end{ex}
	
	\begin{proof}
	Choose $t_0$ as in the previous exercise. Since convexity of $f$ implies that for each $t \in (0, t_0/2)$,
	$$f(x_0) \leq \frac{1}{2} f(x_0 - 2tx) + \frac{1}{2} f(x_0 + 2tx)$$
	we have that for each $t \in (0, t_0/2)$,
	\begin{align*}
	q(-2t) 
	&= \frac{f(x_0 - 2tx) - f(x_0)}{-2t} \\
	&\leq \frac{f(x_0 + 2tx) - f(x_0)}{2t} \\
	&= q(2t)
	\end{align*}	 
	So for each $t \in (0, t_0)$, $q(-t) \leq q(t)$.
	\end{proof}	
	
	\begin{ex} \lex{}
	Let $X$ be a Banach space, $A \subset X$ open and convex, $f:A \rightarrow \R$ convex and $x_0 \in A$. Then 
	\begin{enumerate}
	\item $f$ is left-hand and right-hand Gateaux differentiable at $x_0$ with $d^-f(x_0) \leq d^+f(x_0)$ 
	\item for each $x \in X$, $d^-f(x_0)(x) = - d^+f(x_0)(-x)$
	\end{enumerate}
	\end{ex}	
	
	\begin{proof}\
	\begin{enumerate}
	\item Let $x \in X$. Choose $t_0 >0$ as in the previous two exercises. Let $t, u \in (0,t_0)$. Choose $s \in (0, \min(u, t))$. The previous two exercises imply that 
	\begin{align*}
	q(-u) 
	& \leq q(-s) \\ 
	&\leq q(s) \\
	&\leq q(t)
	\end{align*} and therefore $q(t)$ is an upper bound for $\{q(-u): u \in (0,t_0)\}$ and $d^-f(x_0)(x) = \sup\limits_{u \in (0,t_0)}q(-u)$ exists with $d^-f(x_0)(x) \leq q(t)$.\\
	Since $t \in (0, t_0)$ is arbitrary, $d^-f(x_0)(x)$ is a lower bound for $\{q(t): t \in (0, t_0)\}$. Therefore $$d^+f(x_0)(x) = \inf_{t \in (0,t_0)}q(t)$$ exists with $d^+f(x_0)(x) \geq d^-f(x_0)(x)$. 
	\item By definition, we have 
	\begin{align*}
	d^-f(x_0)(x)
	&= \lim_{t \rightarrow 0^+} \frac{f(x_0 + -tx) - f(x_0)}{-t} \\
	&= - \lim_{t \rightarrow 0^+} \frac{f(x_0 + -tx) - f(x_0)}{t} \\
	&= - d^+f(x_0)(-x)
	\end{align*}
	\end{enumerate}
	\end{proof}
	
	\begin{ex} \lex{}
	Let $X$ be a Banach space, $A \subset X$ open and convex, $f:A \rightarrow \R$ convex and $x_0 \in A$. Then $d^+f(x_0):X \rightarrow \R$ is a sublinear functional.
	\end{ex}	
	
	\begin{proof}
	Let $x,y \in X$ and $k \geq 0$. If $k = 0$, then clearly
	\begin{align*}
	d^+f(x_0)(kx)
	&= k d^+(x_0)(x)
	\end{align*}
	If $k >0$. Then 
	\begin{align*}
	d^+f(x_0)(kx)
	&= \lim_{t \rightarrow 0^+} \frac{f(x_0 + tkx) - f(x_0)}{t} \\
	&= k\lim_{t \rightarrow 0^+} \frac{f(x_0 + tkx) - f(x_0)}{tk}\\
	&= kd^+f(x_0)(x)
	\end{align*}
	Define $t_0 >0$ as before and let $t \in (0, \frac{t_0}{2})$. Note that $$x_0 + tx + ty = \frac{1}{2}(x_0 + 2tx) + \frac{1}{2}(x_0 + 2ty)$$ 
	Convexity of $f$ implies that $$f(x_0 + tx + ty) \leq \frac{1}{2}f(x_0 + 2tx) + \frac{1}{2}f(x_0 + 2ty)$$
	which implies that $$\frac{f(x_0 + tx + ty) - f(x_0)}{t} \leq \frac{f(x_0 + 2tx) - f(x_0)}{2t} + \frac{f(x_0 + 2ty) - f(x_0)}{2t}$$
	Therefore 
	\begin{align*}
	d^+f(x_0)(x+y) 
	&= \lim_{t \rightarrow 0^+} \frac{f(x_0 + t(x + y)) - f(x_0)}{t} \\
	&= \lim_{t \rightarrow 0^+} \frac{f(x_0 + tx + ty) - f(x_0)}{t} \\
	& \leq \lim_{t \rightarrow 0^+} \bigg[ \frac{f(x_0 + 2tx) - f(x_0)}{2t} + \frac{f(x_0 + 2ty) - f(x_0)}{2t} \bigg ] \\
	&= \lim_{t \rightarrow 0^+}  \frac{f(x_0 + 2tx) - f(x_0)}{2t} + \lim_{t \rightarrow 0^+} \frac{f(x_0 + 2ty) - f(x_0)}{2t} \\
	&= d^+f(x_0)(x) + d^+f(x_0)(y) 
	\end{align*}
	\end{proof}
	
	\begin{ex} \lex{}
	Let $X$ be a Banach space, $A \subset X$ open and convex, $f:A \rightarrow \R$ convex and $x_0 \in A$. Then for each $x \in A$, $$d^+f(x_0)(x-x_0) \leq f(x) - f(x_0)$$
	\end{ex}	
	
	\begin{proof}
	Let $x \in A$. Define $T = \{t \in \R: x_0 + t(x-x_0) \in A\}$ similarly to earlier. Clearly $1 \in T$ and  
	\begin{align*}
	d^+f(x_0)(x - x_0) 
	&= \inf_{t \in (0,1]} \frac{f(x_0 + t(x-x_0)) - f(x_0)}{t} \\
	& \leq f(x) - f(x_0)
	\end{align*}
	\end{proof}
	
	\begin{ex} \lex{}
	Let $X$ be a Banach space, $A \subset X$ open and convex, $f:A \rightarrow \R$ convex and $x_0 \in A$. If $f$ is continuous at $x_0$, then $d^+f(x_0)$ is Lipschitz (equivalently bounded). 
	\end{ex}	
	
	\begin{proof}
	Suppose that $f$ is continuous at $x_0$. A previous exercise about convex functions tells us that $f$ is locally Lipschitz at $x_0$, so there exists $\del,M >0$ such that for each $x_1, x_2 \in B(x_0, \del)$, $|f(x_1) - f(x_2)| \leq M\|x_1 - x_2\|$. Let $x \in X$ and define $t_0 = \frac{\del}{\|x\| + 1}$ so that for each $t \in (0, t_0)$,
	\begin{align*}
	\|(x_0 +tx) - x_0\|
	& = t\|x\| \\
	& \leq t_0 \| x\| \\
	&= \frac{\del \| x\|}{\|x\| + 1} \\
	& < \del
	\end{align*}  
	and $x_0 +tx \in B(x_0, \del)$.
	Then for each $t \in (0, t_0)$, 
	\begin{align*}
	d^+f(x_0)(x) 
	& \leq \frac{f(x_0 + tx) - f(x_0)}{t} \\
	& \leq \frac{|f(x_0 + tx) - f(x_0)|}{t} \\
	& \leq t^{-1}M \| (x_0 + tx) - x_0\| \\
	&= M\|x\|
	\end{align*}
	Thus $d^+f(x_0)$ is a bounded sublinear functional and  a previous exercise in the section on sublinear functionals implies this is eqivalent to $d^+f(x_0)$ being Lipschitz.
	\end{proof}
	
	\begin{ex} \lex{}
	Let $X$ be a Banach space, $A \subset X$ open and convex, $f:A \rightarrow \R$ convex and $x_0 \in A$. If $f$ is continuous at $x_0$, then there exists $\phi \in X^*$ such that $\phi \leq d^+f(x_0)$.
	\end{ex}	
	
	\begin{proof}
	Suppose that $f$ is continuous at $x_0$. The previous exercise implies that $d^+f(x_0)$ is Lipschitz (equivalently bounded). A previous exercise in the section discussing sublinear functionals tells us that boundedness of $d^+f(x_0)$ implies that there exists $\phi \in X^*$ such that $\phi \leq d^+f(x_0)$.
	\end{proof}
	
	\begin{defn} \ld{}\tbf{Subdifferential:}\\
	Let $X$ be a Banach space, $A \subset X$ open and convex, $f:A \rightarrow \R$ convex and $x_0 \in A$. We define the \tbf{subdifferential of $f$ at $x_0$}, denoted $\p f(x_0)$, to be $$\p f(x_0) = \{ \phi \in X^*: \text{for each } x \in A, f(x_0) + \phi(x-x_0) \leq f(x)\}$$
	\end{defn}
	
	\begin{ex} \lex{}
	Let $X$ be a Banach space, $A \subset X$ open and convex, $f:A \rightarrow \R$ convex and $x_0 \in A$. If $f$ is continuous at $x_0$, then $\p f(x_0) \neq \varnothing$.
	\end{ex}
	
	\begin{proof}
	Suppose that $f$ is continuous at $x_0$. The previous exercise tells us that there exists $\phi \in X^*$ such that $\phi \leq d^+f(x_0)$. Let $x \in A$. A previous exercise implies that
	\begin{align*}
	\phi(x-x_0) 
	& \leq d^+f(x_0)(x - x_0) \\
	& \leq f(x) - f(x_0)
	\end{align*}
	Then $f(x_0) + \phi(x-x_0) \leq f(x)$.
	\end{proof}
	
	\begin{ex} \lex{}
	Let $X$ be a Banach space, $A \subset X$ open and convex, $f:A \rightarrow \R$ convex, $\phi \in X^*$ and $x_0 \in A$. Then 
	\begin{enumerate}
	\item for each $x \in A$, $$\phi(x-x_0) \leq f(x) - f(x_0)$$ iff  $$\phi \leq d^+f(x_0)$$ 
	\item  $\p f(x_0) = \{ \phi \in X^*: \phi \leq d^+ f(x_0)\}$
	\end{enumerate}
	\end{ex}	
	
	\begin{proof}\
	\begin{enumerate}
	\item Suppose that for each $x \in A$, $\phi(x-x_0) \leq f(x) - f(x_0)$. Let $x \in X$. Define $t_0$ as before. Then for each $t \in (0, t_0)$, 
	\begin{align*}
	t\phi(x)
	&= \phi((x_0 + tx) - x_0) \\
	& \leq f(x_0 + tx) - f(x_0)
	\end{align*}	 
	This implies that $\phi(x) \leq d^+f(x_0)(x)$.\\
	Conversely, suppose that $\phi \leq d^+f(x_0)$. Let $x \in A$. A previous exercise implies that, 
	\begin{align*}
	\phi(x-x_0) 
	& \leq d^+f(x_0)(x-x_0) \\
	&\leq f(x) - f(x_0)
	\end{align*}
	\item Clear.
	\end{enumerate}
	\end{proof}
	
	\begin{ex} \lex{}
	Let $X$ be a Banach space, $A \subset X$ open and convex, $f:A \rightarrow \R$ convex and $x_0 \in A$. If $f$ is continuous at $x_0$, then the following are equivalent:
	\begin{enumerate}
	\item $f$ is Gateaux differentiable at $x_0$    
	\item $d^+f(x_0)$ is linear 
	\item $\# \p f(x_0) = 1$
	\end{enumerate}
	\end{ex}	
	
	\begin{proof}
	Suppose that $f$ is continuous at $x_0$. Then $d^+f(x_0)$ is Lipschitz and bounded.
	\begin{itemize}
	\item $(1) \implies (2)$: \\ 
	Suppose that $f$ is Gateaux differentiable at $x_0$. Let $x \in X$. Then a previous exercise implies that 
	\begin{align*}
	-df^+(x_0)(-x) 
	&= df^-f(x_0)(x) \\
	&= df^+f(x_0)(x)
	\end{align*}
	An exercise in the section on sublinear functionals implies that $df^+f(x_0)$ is linear.
	\item $(2) \implies (3)$: \\  
	Suppose that $df^+f(x_0)$ is linear. Let $\phi \in \p f(x_0)$. The previous exercise implies that $\phi \leq df^+f(x_0)$. Equivalence of linearity in the section on sublinear functionals implies that $d^+f(x_0) = \phi$. 
	\item $(3) \implies (1)$: \\  
	Suppose that $\# \p f(x_0) = 1$. Since $\p f(x_0) = \{ \phi \in X^*: \phi \leq d^+ f(x_0) \}$, equivalence of linearity in the section on sublinear functionals implies that $d^+ f(x_0)$ is linear. This implies that $d^+ f(x_0) = d^- f(x_0)$ and which implies that $f$ is Gateaux differentiable at $x_0$.
	\end{itemize}
	\end{proof}
	
	\begin{ex}
	Let $X$ be a Banach space, $A \subset X$ open and convex, $f,g:A \rightarrow \R$ convex, $\lam \geq 0$ and $x_0 \in A$. Then $$\p f(x_0) + \lam \p g(x_0) \subset \p[f + \lam g](x_0)$$
	\end{ex}
	
	\begin{proof}
	Let $\zeta \in \p f(x_0) + \lam \p g(x_0)$. Then there exist $\phi \in \p f(x_0)$ and $\psi \in \p g(x_0)$ such that $\zeta = \phi + \lam \psi$. A previous exercise implies that $\phi \leq d^+f(x_0)$ and $\lam \psi \leq \lam d^+g(x_0) = d^+[\lam g](x_0)$. Hence 
	\begin{align*}
	\zeta
	&= \phi + \lam \psi \\
	&\leq d^+f(x_0) + d^+[\lam g](x_0) \\
	&= d^+[f + \lam g](x_0)
	\end{align*}
	So $\zeta \in \p [f+\lam g](x_0)$
	\end{proof}
	
	\begin{ex} \lex{}
	Let $X$ be a Banach space, $A \subset X$ open and convex, $f:A \rightarrow \R$ convex and $x_0 \in A$. If $f$ is continuous at $x_0$, then $f$ has a global minimum point at $x_0$ iff $0 \in \p f(x_0)$.
	\end{ex}
	
	\begin{proof}
	Suppose that $f$ has a global minimum point at $x_0$. Let $x \in X$. Then 
	\begin{align*}
	d^+f(x_0)(x) 
	&= \lim_{t \rightarrow 0^+} \frac{f(x_0 + tx) - f(x_0)}{t} \\
	& \geq 0
	\end{align*}
	So $0 \leq df^+(x_0)$ and $0 \in \p f(x_0)$.\\
	Conversely, suppose that $0 \in \p f(x_0)$. Let $x \in A$. Then 
	\begin{align*}
	0
	& = 0(x - x_0) \\
	& \leq f(x) - f(x_0)
	\end{align*}
	So that $f(x_0) \leq f(x)$ which implies that $f$ has a global minimum point at $x_0$.
	\end{proof}
	
	\begin{ex}
	et $X$ be a Banach space, $A \subset X$ open and convex, $f:A \rightarrow \R$ convex and $x_0 \in A$. If $f$ is Frechet differentiable at $x_0$, then $\p f(x_0) = \{Df(x_0)\}$. 
	\end{ex}	
	
	\begin{proof}
	Clear.
	\end{proof}
	
	\begin{ex}
	Let $X$ be a Banach space, $A \subset X$ open and convex, $f:A \rightarrow \R$ convex and $x_0 \in A$. Suppose that $f$ is Frechet differentiable at $x_0$. If $Df(x_0) = 0$, then $f$ has a global minimum point at $x_0$. 
	\end{ex}
	
	\begin{proof}
	Suppose that $Df(x_0) = 0$. Since $\p f(x_0) = \{Df(x_0)\}$, a previous exercise implies that $f$ has a global minimum point at $x_0$. 
	\end{proof}
	
	\begin{ex}
	Let $X$ be a Banach space, $A \subset X$ open and convex, $f:A \rightarrow \R$ convex and $x_0 \in A$. Suppose that $f$ is Frechet differentiable at $x_0$. Then for each $x \in A$, $f(x) \geq f(x_0) + Df(x_0)(x - x_0)$
	\end{ex}
	
	\begin{proof}
	Since $Df(x_0) \in \p f(x_0)$, for each $x \in A$, $Df(x_0)(x - x_0) \leq f(x) - f(x_0)$.
	\end{proof}
	
	\begin{ex}
	Let $X$ be a Banach space, $A \subset X$ open and convex, $f:A \rightarrow \R$. Suppose that $f$ is Frechet differentiable. Then $f$ is convex iff for each $x_0, x \in A$, $f(x) \geq f(x_0) + Df(x_0)(x - x_0)$.
	\end{ex}
	
	\begin{proof}
	Suppose that $f$ is convex. Then the previous exercise implies that for each $x_0,x \in A$, $f(x) \geq f(x_0) + Df(x_0)(x - x_0)$. Conversely, suppose that for each $x_0,x \in A$, $f(x) \geq f(x_0) + Df(x_0)(x - x_0)$. Let $x_0, x, y \in A$. Then $f(x) \geq f(x_0) + Df(x_0)(x - x_0)$ and $f(y) \geq f(x_0) + Df(x_0)(y - x_0)$. \\
	\tbf{FINISH!!!}
	\end{proof}
	
	\begin{ex} \lex{}
	Let $X$ be a Banach space, $A \subset X$ open and convex, and $f \in C^2(A)$. Then $f$ is convex iff for each $x_0 \in A$, $D^2f(x_0)$ is positive semidefinite.\\
	\tbf{Hint:} Define $g:A \rightarrow \R$ by $g(x) = f(x) - Df(x_0)(x - x_0)$ and show $g$ is convex and use Taylor's Theorem
	\end{ex}
	
	\begin{proof}
	Suppose that $f$ is convex. Let $x_0 \in X$. Define $g:A \rightarrow \R$ by $g(x) = f(x) - Df(x_0)(x - x_0)$. Since $g$ is the sum of a convex function and an affine function, $g$ is convex. Since $f \in C^2(A)$, we have that $g \in C^2(A)$ and it is straightforward to show that for each $x \in A$, $Dg(x) = Df(x) - Df(x_0)$ and $D^2g(x) = D^2f(x)$. In particular, $Dg(x_0) = 0$. Hence $g$ has a global minimum point at $x_0$. This implies that $D^2f(x_0)$ is positive semidefinite. 
	Conversely, suppose that for each $x_0 \in A$, $D^2f(x_0)$ is positive semidefinite. Let  \\
	\tbf{FINISH!!!}
	\end{proof}
	
	\newpage 
	\subsection{Conjugacy}
	
	\begin{defn} \ld{}
	Let $X$ be a Banach space, $A \subset X$ and $f:A \rightarrow \R$. Define 
	\begin{enumerate}
		\item $A^* \subset X^*$ and $f^*: A^* \rightarrow \R$ 
		\item $A^{**} \subset X$ and $f^{**}:A^{**} \rightarrow \R$
	\end{enumerate}
	by 
	\begin{enumerate}
		\item $$A^* = \bigg \{\phi \in X^*: \sup_{x \in A} \bigg[ \phi(x) - f(x) \bigg] < \infty \bigg  \}$$ and $$f^*(\phi) = \sup_{x \in A} \bigg[ \phi(x) - f(x) \bigg] $$  
		\item $$A^{**} = \bigg \{x \in X: \sup_{\phi \in A^*} \bigg[ \hat{x}(\phi) - f^*(\phi) \bigg] < \infty \bigg \}$$ and $$f^{**}(x) = \sup_{\phi \in A^*} \bigg[ \hat{x}(\phi) - f^*(\phi) \bigg]$$
	\end{enumerate}
	\end{defn} 

	\begin{note}
		If $X$ is a Hilbert space, we may define $A^* \subset X$ and $f^*: A^* \rightarrow \R$ via the Riesz representation theorem by $$A^* = \bigg \{y \in X: \sup_{x \in A} \bigg[ \l y, x \r - f(x) \bigg] < \infty \bigg  \}$$ and $f^*: A^* \rightarrow \R$ and $$ f^*(y) = \sup_{x \in A} \bigg[ \l y, x \r - f(x) \bigg] $$
	\end{note}
	
	\begin{ex} \lex{}
	Let $X$ be a Banach space, $A \subset X$ and $f:A \rightarrow \R$. Then 
	\begin{enumerate}
		\item $A^*$ is convex and $f^*:A^* \rightarrow \R$ is convex and weak* lower semicontinuous.
		\item $A^{**}$ is convex and $f^{**}:A^{**} \rightarrow \R$ is convex and weakly lower semicontinuous.
	\end{enumerate} 
	\end{ex}
	
	\begin{proof} \
		\begin{enumerate}
			\item For $x \in A$, define $g_x: X^* \rightarrow \R$ by $g_x(\phi) = \hat{x}(\phi) - f(x)$. Then for each $x \in A$, $g_x$ is convex and weak* \lsc \, since it is affine and weak* continuous. \rex{91015} implies that $A^* = \{\phi \in X^*: \sup\limits_{x \in A} g_x(\phi) < \infty \}$ is convex and  $f^* = \sup\limits_{x \in A} g_x$ is convex.
			\item For $\phi \in A^*$, define $h_{\phi}: X \rightarrow \R$ by $h_{\phi}(x) = \phi(x) - f^{*}(\phi)$. Then for each $\phi \in A^*$, $g_{\phi}$ is convex and weakly \lsc \, since it is affine and weakly continuous. \rex{91015} implies that $A^{**} = \{x \in X: \sup\limits_{\phi \in A^*} h_{\phi}(x) < \infty \}$ is convex and   $f^{**} = \sup\limits_{\phi \in A^*} h_{\phi}$ is convex. 
		\end{enumerate}		 
	\end{proof}
	
	\begin{ex} \lex{}
		Let $X$ be a Banach space, $A \subset X$ and $f:A \rightarrow \R$. Then for each $x \in A$ and $\phi \in A^*$, $f^*(\phi) \geq \phi(x) - f(x)$.	
	\end{ex}
	
	\begin{proof}
	Clear by definition.
	\end{proof}
	
	\begin{ex} \lex{}
	Let $X$ be a Banach space, $A \subset X$ and $f:A \rightarrow \R$. Then $A \subset A^{**}$.
	\end{ex}

	\begin{proof}
		Let $x \in A$. Then the previous exercise implies that
		\begin{align*}
			\sup_{\phi \in A^*} [\phi(x) - f^{*}(\phi)] 
			& \leq f(x) \\
			& < \infty  
		\end{align*}
		So $x \in A^{**}$.
	\end{proof}

	\begin{ex}
		Let $X$ be a Banach space, $A \subset X$ convex, $f:A \rightarrow \R$ convex and lower semicontinuous and $x_0 \in A$. 
		\begin{enumerate}
			\item if $x_0 \in A$, then for each $\ep >0$, there exists $\phi \in A^*$ such that for each $x \in A$, $f(x) > f(x_0) + \phi(x - x_0) - \ep$
			\item if $x_0 \not \in A$, then for each $M \in \R$, there exists $\phi \in A^*$ such that for each $x \in A$, $f(x) > M + \phi(x - x_0)$
		\end{enumerate}
	\tbf{Hint:} Apply second Hahn-Banach separation theorem to $\{(x_0, f(x_0) - \ep)\}$ and $\epi f$.
	\end{ex}

	\begin{proof}\
		\begin{enumerate}
			\item Suppose that $x_0 \in A$. Let $\ep >0$. Since $f$ is convex and lower semicontinuous, $\epi f \subset X \times \R$ is convex and closed, $\{(x_0, f(x_0) - \ep)\} \subset X \times \R$ is convex and compact and $\{(x_0, f(x_0) - \ep)\} \cap \epi f = \varnothing$. Thus, there exists $\lam \in \R$, $\psi \in X^*$ and $k \in \R$ such that for each $x \in A$ and $r \geq f(x)$, 
			$$\psi(x) + \lam r < k < \psi(x_0) + \lam (f(x_0) - \ep)$$
			Taking $(x, r) = (x_0, f(x_0))$ implies that $0 < -\lam \ep $ and hence that $\lam < 0$. Set $\phi = |\lam|^{-1}\psi$. For $x \in A$, set $r = f(x)$. Then 
			\begin{align*}
				& \hspace{1.2cm} \psi(x) -|\lam| f(x) < \psi(x_0) - |\lam| (f(x_0) - \ep) \\
				& \iff |\lam|^{-1} \psi(x) - f(x) < |\lam|^{-1}\psi(x_0) - (f(x_0) - \ep) \\
				& \iff \phi(x) - f(x) < \phi(x_0) - (f(x_0) - \ep) \\
				& \iff f(x) > f(x_0) + \phi(x-x_0) - \ep 
			\end{align*}
			Since for each $x \in A$, $\phi(x) - f(x) < \phi(x_0) - f(x_0) + \ep$, we have that 
			\begin{align*}
				\sup_{a \in A}[\phi(x) - f(x)] 
				& \leq \phi(x_0) - f(x_0) + \ep \\
				&< \infty
			\end{align*}
			So $\phi \in A^*$.
			\item Suppose that $x_0 \not \in A$. Let $M \in \R$. Repeat the previous argument for $(x_0, M)$ and $\epi f$.
		\end{enumerate}
	\end{proof}

	\begin{ex}
		Let $X$ be a Banach space, $A \subset X$ convex and $f:A \rightarrow \R$ convex and lower semicontinuous. Then 
		\begin{enumerate}
			\item $A = A^{**}$
			\item $f = f^{**}$
		\end{enumerate}
	\end{ex}

	\begin{proof}\
		\begin{enumerate}
			\item A previous exercise implies that $A \subset A^{**}$. Let $x_0 \in X$. Suppose that $x_0 \not \in A$. Let $M \in \R$. The previous exercise implies that there exists $\phi_0 \in A^*$ such that for each $x \in A$, $f(x) > M + \phi_0(x - x_0)$. Then 
			\begin{align*}
				\phi_0(x_0) - f^*(\phi_0) 
				&= \phi_0(x_0) - \sup_{x \in A}[\phi_0(x) - f(x)] \\
				&= \phi_0(x_0) + \inf_{x \in A}[f(x) - \phi_0(x)] \\
				& \geq \phi_0(x_0) + (M - \phi_0(x_0)) \\
				&= M
			\end{align*}
		Therefore 
		\begin{align*}
			\sup\limits_{\phi \in A^*}[\phi(x_0) - f^*(\phi)] 
			& \geq \phi_0(x_0) - f^*(\phi_0) \\
			& \geq M
		\end{align*}
		Since $M \in \R$ is arbitrary, $$\sup\limits_{\phi \in A^*}[\phi(x_0) - f^*(\phi)] = \infty $$ and $x_0 \not \in A^{**}$. So $A^c \subset (A^{**})^c$, which implies that $A^{**} \subset A$. Thus $A^{**} = A$.
		\item Part $(1)$ and a previous exercise imply that $f^{**} \leq f$. Suppose that $f \not \leq f^{**}$. Then there exists $x_0 \in A$ such that $f(x_0) > f^{**}(x_0)$. Choose $\ep > 0$ such that $f(x_0) > f^{**}(x_0) + 2 \ep$. A previous exercise implies that there exists $\phi \in A^*$ such that for each $x \in A$, $f(x) > f(x_0) + \phi(x - x_0) - \ep$. Choose $a \in A$ such that $f^*(\phi) - \ep < \phi(a) - f(a)$. Then 
		\begin{align*}
			f(x_0)
			& > f^{**}(x_0) + 2 \ep \\
			& \geq \phi(x_0) - f^*(\phi) + 2 \ep \\
			& > \phi(x_0 -a) + f(a) + \ep \\
			& > \phi(x_0 -a) + f(x_0) + \phi(a - x_0) - \ep + \ep \\
			&= f(x_0) 
		\end{align*}
		which is a contradiction. So $f \leq f^{**}$ and hence $f = f^{**}$. 
		\end{enumerate}
		
	\end{proof}
	

	
	
	
	
	
	
	
	
	
	\begin{defn} \ld{}
	Let 
	\end{defn}
	
	\begin{defn} \ld{}
	$\partial f$
	\end{defn}	
	
	\begin{ex} \lex{}
	
	\end{ex}
	
	
	
	
	
	
	
	
	
	
	
	
	
	
	
	
	
	
	
	
	
	\newpage
	\section{Topological Groups}
	
	
	
	\subsection{Topological Groups}
	\begin{note}
	This section establishes some basic results about topological groups and gives examples of common topological groups in analysis, specifically automorphism groups of metric spaces.  
	\end{note}
	
	\begin{defn} \ld{00000} 
		Let $G$ be a group and $\MT$ a topology on $G$. Then $(G, \MT)$ is said to be a \tbf{topological group} if the maps \begin{enumerate}
			\item $G \times G \rightarrow G$ given by $(x,y) \mapsto xy$
			\item  $G \rightarrow G$ given by $x \mapsto x^{-1}$ 
		\end{enumerate} are continuous.
	\end{defn}

	\begin{note}
		For the remainder of this chapter, measurablility is in reference to $(G, \MB(\MT))$. That is, the measurable sets are the Borel sets.
	\end{note}
	
	\begin{defn} \ld{00000} 
		Let $G$ be a topological group. We define $$\Homeo(G) = \{\phi:G \rightarrow G: \phi \text{ is a homeomorphism}\}$$
	\end{defn}
	
	\begin{note}
	Let $G$ be a topological group. Then $\Homeo(G)$ is a group.
	\end{note}
	
	\begin{defn} \ld{00000} 
		Let $G$ be a group. Define $\iota:G \rightarrow G$ by $\iota(x) = x^{-1}$. 
	\end{defn}
	
	\begin{ex} \lex{00000} 
		Let $G$ be a topological group. Then $\iota \in \Homeo(G)$.
	\end{ex}

	\begin{proof}
		By assumption $\iota$ is continuous. We know from basic group theory that $\iota$ is a bijection with $\iota^{-1} = \iota$. 
	\end{proof}

	\begin{defn} \ld{00000} 
		Let $G$ be a group and $S \subset G$, then $S$ is said to be \tbf{symmetric} if $\iota(S) = S$, ( i.e. $S^{-1} = S$).
	\end{defn}
	
	\begin{defn} \ld{00000} 
		Let $G$ be a topological group and $\phi:G \rightarrow G$. Then $\phi$ is said to be an \tbf{automorphism} of $G$ if $\phi$ is a homomorphism and a homeomorphism. We define $$\Aut(G) = \{\phi:G \rightarrow G: \phi \text{ is an automorphism}\}$$
	\end{defn}
	
	\begin{ex} \lex{00000} 
	Let $G$ be a topological group. Then $\iota \in \Aut(G)$ iff $G$ is abelian. 
	\end{ex}
	
	\begin{proof}
	Basic group theory tells us that $\iota$ is a homomorphism iff $G$ is abelian.
	\end{proof}
	
	\begin{defn} \ld{00000} 
		Let $G$ be a group and $g \in G$. Define $l_g:G \rightarrow G$ and $r_g:G \rightarrow G$ by $l_g(x) = gx$ and $r_g(x) = xg^{-1}$. 
	\end{defn}
	
	\begin{ex} \lex{00000} 
		Let $G$ be a topological group and $g \in G$. Then $l_g, r_g \in \Homeo(G)$.
	\end{ex}
	
	\begin{proof}
		By assumption $l_g$ and $r_g$ are continuous. We know from basic group theory that $l_g$ and $r_g$ are bijections with $l_g^{-1} = l_{g^{-1}}$ and $r_g^{-1} = r_{g^{-1}}$ so $l_g$ and $r_g$. are homeomorphisms. 
	\end{proof}
	
	\begin{ex} \lex{00000} 
	Let $G$ be a toplogical group. Define $\phi, \psi:G \rightarrow \Homeo(G)$ by $\phi(g) = l_g$ and $\psi(g) = r_g$. Then $\phi, \psi$ are homomorphisms.
	\end{ex}
	
	\begin{proof}
	Let $g_1, g_2 \in G$. Then $$l_{g_1} \circ l_{g_2}(x) = l_{g_1}(g_2 x) = g_1 g_2 x= l_{g_1 g_2}(x)$$ and $$r_{g_1} \circ r_{g_2} (x) = r_{g_1}(x g_2^{-1})= xg_2^{-1}g_1^{-1} = x(g_1g_2)^{-1} = r_{g_1g_2}(x)$$ 
	\end{proof}
	
	\begin{ex} \lex{00000} 
		Let $G$ be a topological group. Then for each $U \subset G$ and $g \in G$, if $U$ is open, then $gU$, $Ug$ and $U^{-1}$ are open. 
	\end{ex}
	\begin{proof}
		Let $U \subset G$ and $g \in G$. Suppose that $U$ is open. Since $l_g, r_g$ and $\iota$ are homeomorphisms, $l_g(U) = gU$, $r_g(U) = Ug$ and $\iota(U) = U^{-1}$ are open. 
	\end{proof}
	
	\begin{defn} \ld{00000} 
		Let $G$ be a topological group, $y \in G$ and $f \in L^0$.  Define $L_y, R_y: L^0(G) \rightarrow L^0(G)$ by $L_y f = f \circ l_y^{-1}$ and $R_y f = f \circ r_y^{-1}$, that is, $L_yf(x) = f(y^{-1}x)$ and $R_yf(x) = f(xy)$.
	\end{defn}
	
	\begin{ex} \lex{00000} 
	Let $G$ be a topological group and $y \in G$. Then $L_y, R_y \in \Sym(L^0(G))$. 
	\end{ex}
	
	\begin{proof}
	It is straight forward to show that $L_y^{-1} = L_{y^{-1}}$ and $R_y^{-1} = R_{y^{-1}}$. 
	\end{proof}
	
	\begin{ex} \lex{00000} 
	Let $G$ be a topological group. Define $\phi, \psi: G \rightarrow \Sym(L^0(G))$ by $\phi(y) = L_y$ and $\psi(y) = R_y$. Then $\phi$ and $\psi$ are homomorphisms.
	\end{ex}

	\begin{proof}
		Let $y,z \in G$ and $f \in L^0(G)$. Then 
		\begin{align*}
			L_y \circ L_z(f)
			& = L_y (L_z (f))  \\
			& = L_y (f \circ l_z^{-1})  \\
			& = (f \circ l_z^{-1}) \circ l_y^{-1} \\
			&= f \circ (l_z^{-1} \circ l_y^{-1}) \\
			& = f \circ (l_y \circ l_z)^{-1}  \\
			& = f \circ l_{yz}^{-1} \\
			&= L_{yz} (f)
		\end{align*}
		
		The case is similar for $R_y$ and $ R_z$.
	\end{proof}
	
	\begin{ex} \lex{00000} 
		Let G be a topological group, $U \in \MB(G)$ and $y \in G$. Then $L_y\chi_U = \chi_{yU}$ and $R_y\chi_U = \chi_{Uy^{-1}}$. 
	\end{ex}
	
	\begin{proof}
		Let $x \in G$. Then 
		\begin{align*}
			L_y\chi_U(x) = 1
			& \iff y^{-1}x \in U\\
			& \iff x \in yU \\
			& \iff \chi_{yU}(x) = 1
		\end{align*}
		The case is similar for $R_y$
	\end{proof}
	
	\begin{ex} \lex{00000} 
		Let G be a topological group, $y \in G$ and $f \in L^0(G)$. Then $\supp(L_yf) = y\supp(f)$ and $\supp(R_yf) = \supp(f)y^{-1}$
	\end{ex}
	
	\begin{proof}
		Put $A = \{x \in G: L_yf(x) \neq 0 \}$ and $B = \{x \in G: f(x) \neq 0 \}$. Then 
		\begin{align*}
			x \in A
			& \iff L_yf(x) \neq 0 \\
			& \iff f(y^{-1}x) \neq 0 \\
			& \iff y^{-1}x \in B \\
			& \iff x \in yB
		\end{align*}
		Thus $A = yB$ which implies that $\cl A = y \cl B$. Therefore $\supp(L_yf) = y\supp(f)$.
	\end{proof}
	
	\begin{ex} \lex{00000} 
		Let $G$ be a topological group and $y \in G$. Then $L_y, R_y$ are linear and if we restrict to the bounded measurable functions, then  $L_y, R_y \in L(B(G))$ and $\|L_y\|, \|R_y\| = 1$. 
	\end{ex}
	
	\begin{proof}
		Let $f, g \in L^0(G)$ and $\lam \in \C$. Then 
		\begin{align*}
			L_y(\lam f+g)(x)
			& = (\lam f+g)(y^{-1}x) \\
			& = \lam f(y^{-1}x) + g(y^{-1}x) \\
			& = \lam L_yf(x) + L_yg(x)
		\end{align*}
		So $L_y$ is linear. Next, we restrict to $B(G) \cap L^0$. We note that $$\{|f(y^{-1}x)|: x \in y\supp(f)\} = \{|f(x)|: x \in \supp(f)\}$$ This implies that 
		\begin{align*}
			\|L_yf \|_u 
			& = \sup_{x \in \supp(L_yf)} |L_yf(x)| \\
			& = \sup_{x \in y\supp(f)} |f(y^{-1}x)| \\
			& = \sup_{x \in \supp(f)} |f(x)| \\ 
			& = \|f\|_u
		\end{align*} 
		So $L_y$ is bounded. Hence $L_y \in L(L^0)$. The case is similar for $R_y$.
	\end{proof}
	
	\begin{defn} \ld{00000} 
		Let $G$ be a topological group. We say that $G$ is a \tbf{locally compact group} if $G$ is locally compact and Hausdorff.
	\end{defn}
	
	
	
	
	
	
	
	
	
	
	
	
	
	
	
	
	
	
	
	
	
	
	
	
	\newpage
	\subsection{Automorphism Groups of Metric Spaces}
	
	\begin{defn} \ld{}
	Let $(X, \tau)$ be a topological space. Define $$\Aut(X) = \{\sig:X\rightarrow X: \sig \text{ is a homeomorphism} \}$$ 
	\end{defn}	
	
	\begin{ex} \lex{}
	Let $(X, d)$ be a compact metric space. Then $(\Aut(X), d_{u} )$ is a topological group.
	\end{ex}
	
	\begin{proof}
	Let $(\sig_n)_{n \in \N}, (\tau_n)_{n \in \N} \subset \Aut(X)$ and $\sig,\tau \in \Aut(X)$. Suppose that $\sig_n \convt{u} \sig$ and $\tau_n \convt{u} \tau$.
	\begin{enumerate}
	\item Let $\ep >0$. Since $X$ is compact and $\sig$ is continuous, $\sig$ is uniformly continuous. Then there exists $\del >0$ such that for each $x, y \in X$, $d(x,y) < \del$ implies that $d(\sig(x), \sig(y)) \leq \ep/2$.  Choose $N_\sig \in \N$ such that for each $n \in \N$, $ n \geq \N$ implies that $d_u(\sig_n, \sig) < \ep/2$. Choose $N_\tau \in \N$ such that for each $n \in \N$, $ n \geq \N$ implies that $d_u(\tau_n, \tau) < \del$. Put $N = \max(N_\sig, N_\tau)$. Let $n \in \N$ and $x \in X$. Suppose that $n \geq N$. Then 
	\begin{align*}
		d(\sig_n \circ \tau_n (x) ,\sig \circ \tau (x) ) 
		&\leq  d(\sig_n(\tau_n(x)),  \sig(\tau_n(x))) + d( \sig(\tau_n (x)), \sig( \tau (x))) \\
		& < \ep / 2 +\ep / 2 \\
		&= \ep 
	\end{align*}
	So $d_u(\sig_n \circ \tau_n, \sig\circ \tau) \leq \ep$ and $\circ: \Aut(X)^2 \rightarrow \Aut(X)$ is continuous. 
	\item Suppose that $\sig = \id_X$. Let $\ep >0$. Then there exists $N \in \N$ such that for each $n \in \N$, $n \geq N$ implies that $d_u(\sig_n, \id_X) < \ep$. Let $n \in \N$. Suppose that $n \geq N$. Then 
	\begin{align*}
	\sup_{x \in X} d(\sig^{-1}_n(x), x) 
	&= \sup_{x \in \sig_n(X)}d(\sig^{-1}_n(x), x) \\
	&= \sup_{x \in X}d(\sig^{-1}_n(\sig_n(x)), \sig_n(x)) \\
	&= \sup_{x \in X}d(x, \sig_n(x)) \\
	&< \ep
	\end{align*}
	So $\sig^{-1}_n \convt{u} \id_X$. Now suppose that $\sig \neq \id_X$. Since $\sig_n \convt{u} \sig$, part $(1)$ implies that $\sig^{-1} \circ \sig_n \convt{u} \id_X$. Applying the result from above, we get that $\sig_n^{-1} \circ \sig \convt{u} \id_X$. Applying part $(1)$ again implies that $\sig_n^{-1}  \convt{u}  \sig^{-1}$. So the map $\sig \mapsto \sig^{-1}$ is continuous. 
	\end{enumerate}
	Hence $\Aut(X)$ is a topological group. 
	\end{proof}
	
	\begin{defn} \ld{}
	Let $(X, d)$ be a metric space. Define 
	$$\Aut(X, d) = \{\sig:X\rightarrow X: \sig \text{ is an isometric isomorphism} \}$$  
	\end{defn}
	
	\begin{ex} \lex{}
	Let $(X, d)$ be a compact metric space. Then $(\Aut(X, d), d_u)$ is a compact subgroup of $(\Aut(X), d_u)$.
	\end{ex}
	
	\begin{proof}
	Clearly, $(\Aut(X, d), d_u)$ is a topological subgroup. To show compactness, use the Arzela Ascoli theorem.
	\end{proof}
	
	\begin{defn} \ld{}
	Let $(X, \tau)$ be a topological space and $\mu: \MB(X) \rightarrow \R$ a Borel measure. Define $$\Aut(X, \mu) = \{\sig \in \Aut(X): \sig_* \mu = \mu\}$$ 
	\end{defn}	
	
	\begin{ex} \lex{}
	Let $(X,d)$ be a compact metric space and $\mu: \MB(X) \rightarrow \R$ an outer-regular Borel measure. Then $\Aut(X, \mu)$ is a closed subgroup of $\Aut(X)$.
	\end{ex}
	
	\begin{proof}
	It is clear that $\Aut(X, \mu)$ is a subgroup of $\Aut(X)$. Let $(\sig_n)_{n \in \N} \subset \Aut(X, \MB(X), \mu)$ and $\sig \in \Aut(X)$. Suppose that $\sig_n \convt{u} \sig$. Let $E \subset X$ be closed, $U \subset X$ open and suppose that $E \subset U$. An exercise in the section on metric spaces tells us that there exists $N \in \N$ such that for each $n \in \N$, $n \geq N$ implies that $\sig(E) \subset \sig_n(U)$. Then 
	\begin{align*}
	\mu(\sig(E)) 
	&\leq \mu(\sig_N(U)) \\
	&= \mu(U) 
	\end{align*}
	Therefore, since $\mu$ is outer regular, $\mu(\sig(E)) \leq \mu(E)$. Since $\sig_n^{-1} \convt{u} \sig^{-1}$, we may apply the above argument to obtain that 
	\begin{align*}
	\mu(E) 
	&= \mu(\sig^{-1}(\sig (E))) \\
	&\leq  \mu(\sig(E))
\end{align*}	 
Hence $\mu(E) = \mu(\sig(E))$. Applying the whole argument above thus far to $\sig^{-1}$, we see that $\mu(E) = \mu(\sig^{-1}(E))$. Since $E \subset X$ is an arbitrary closed set and $\MB(X) = \sig(E \subset X: E \text{ is closed})$, we have that $\mu = \sig_*\mu$. Thus $\sig \in \Aut(X, \mu)$ which implies that $\Aut(X, \mu)$ is closed. 
	\end{proof}
	
	\begin{defn} \ld{}
	Let $(X,d)$ be a compact metric space and $\mu: \MB(X) \rightarrow \R$ an outer-regular Borel measure. Define $\Aut(X, d, \mu) = \Aut(X, d) \cap \Aut(X, \mu)$.
	\end{defn}
	
	\begin{ex} \lex{}
	Let $(X,d)$ be a compact metric space and $\mu: \MB(X) \rightarrow \R$ an outer-regular Borel measure. Then $\Aut(X, d, \mu)$ is compact.
	\end{ex}
	
	\begin{proof}
	Since $\Aut(X, d)$ is compact and $\Aut(X, \mu)$ is closed, $\Aut(X, d, \mu)$ is compact.
	\end{proof}
	
	
	
	
	
	
	
	
	
	
	
	
	
	

	
	
	
	
	
	
	
	
	
	
	\newpage
	\section{Group Actions on Metric Spaces}
	
	\subsection{Introduction}
	\begin{note}
	For a set $X$, a group $G$ and a (left) group action $\phi: G \times X \rightarrow X$, we will write $\phi(g, x)$ as $g \cdot x$. We denote the projection map by $\pi: X \rightarrow X/G$.
	\end{note}	
	
	\begin{defn} \ld{00000} 
		Let $X$ be a set, $G$ a group, $\phi: G \times X \rightarrow X$ a group action and $g \in G$. Define $l_g:X \rightarrow X$ by 
		\begin{equation*}
		l_g(x) = g \cdot x
		\end{equation*}
	\end{defn}
	
	\begin{defn}
	Let $X$ be a topological space, $G$ a group and $\phi: G \times X \rightarrow X$ a group action. Then $\phi$ is said to be $X$-continuous if for each $g \in G$, $l_g$ is continuous.
	\end{defn}
	
	\begin{ex}
	Let $X$ be a topological space, $G$ a group and $\phi: G \times X \rightarrow X$ an $X$-continuous group action. Then for each $g \in G$, $l_g \in \Homeo(X)$.
	\end{ex}
	
	\begin{proof}
	Let $g \in G$, then $l_g$ and $l_{g}^{-1} = l_{g^{-1}}$ are continuous, so $l_g \in \Homeo(G)$. 
	\end{proof}
	
	\begin{defn} \ld{}
	Let $(X, d)$ be a metric space, $G$ a group, and $\phi: G \times X \rightarrow X$ a group action. Then $\phi$ is said to be an \tbf{isometric group action} if for each $g \in G$, $l_g:X \rightarrow X$ is an isometry. 
	\end{defn}
	
	\begin{ex}
	Let $(X, d)$ be a metric space, $G$ a group, and $\phi: G \times X \rightarrow X$ an isometric group action. Then $\phi$ is $X$-continuous.
	\end{ex}
	
	\begin{proof}
	Clear since isometries are continuous.
\end{proof}		
	
	\begin{defn}
	Let $X$ be a set, $G$ a group and $\phi: G \times X \rightarrow X$ an $X$-continuous group action. Let $g \in G$. Define $L_g:\C^X \rightarrow \C^X$ by 
	\begin{align*}
	L_g(f)(x) 
	&= f \circ l_g^{-1} \\
	&= f \circ l_{g^{-1}}
	\end{align*}
	\end{defn}
	
	
	\begin{defn}
	Let X be a set, $G$ a group, $\phi: G \times X \rightarrow X$ a group action and $f:X \rightarrow \C$. Then $f$ is said to be \tbf{$G$-invariant} if for each $g \in G$, $L_g f = f$.
	\end{defn}
	
	\begin{ex}
	Let X be a set, $G$ a group, $\phi: G \times X \rightarrow X$ a group action and $f:X \rightarrow \C$. Then $f$ is $G$-invariant iff for each $g \in G$ $x \in X$, $f(g \cdot x) = f(x)$.  
	\end{ex}
	
	\begin{proof}
	Clear.
	\end{proof}
	
	\begin{defn}
	Let X be a set, $G$ a group, $\phi: G \times X \rightarrow X$ a group action and $f:X \rightarrow \C$. Suppose that $f$ is $G$-invariant. Define $\bar{f}:X/ G \rightarrow \C$ by $\bar{f}(\bar{x}) = f(x)$. 
	\end{defn}
	
	\begin{ex}
	Let X be a set, $G$ a group, $\phi: G \times X \rightarrow X$ a group action and $f:X \rightarrow \C$. Suppose that $f$ is $G$-invariant. Then $f = \bar{f} \circ \pi$. 
	\end{ex}
	
	\begin{proof}
	Clear.
	\end{proof}
	
	
	
	
	
	
	
	
	
	\newpage
	\subsection{Induced Metrics on Orbit Spaces}
	
	\begin{note}
	This section establishes the criteria for the existence of a metric on the orbit space of a metric space under a group action. 
	\end{note}
	
	\begin{defn} \ld{}
	Let $(X, d)$ be a metric space, $G$ a group, and $\phi: G \times X \rightarrow X$ a group action. We define 
	$\bar{d}: X/G \times X / G \rightarrow \Rg$ by 
	$$\bar{d}(\bar{x}, \bar{y}) = \inf_{\substack{a \in \bar{x} \\ b \in \bar{y}}} d(a,b) $$
	\end{defn}
	
	\begin{ex} \lex{}
	Let $(X, d)$ be a metric space, $G$ a group, and $\phi: G \times X \rightarrow X$ an isometric group action. Then for each $x, y \in X$, $$\bar{d}(\bar{x}, \bar{y}) = \inf_{g \in G} d(g \cdot x, y)$$
	\end{ex}
	
	\begin{proof}
	Let $x, y \in X$, $a \in \bar{x}$ and $b \in \bar{y}$. Then there exists there exists $g_a, g_b \in G$ such that $a = g_a \cdot x$ and $b = g_b \cdot y$. Set $g = g_b^{-1}g_a$. Since the map $z \mapsto g_b^{-1} \cdot z$ is an isometry, 
	\begin{align*}
	d(a,b) 
	&= d(g_a \cdot x, g_b \cdot y) \\
	&= d(g_b^{-1}g_a \cdot x, y)\\
	&= d(g\cdot x, y)
	\end{align*}
	Let $\ep >0$. Then there exist $a^* \in \bar{x}$ and $b^* \in \bar{y}$ such that $d(a^*,b^*) < \bar{d}(\bar{x},\bar{y}) + \ep$. The above argument implies that that there exists $g^* \in G$ such that 
	\begin{align*} 
	\inf_{g \in G} d(g \cdot x, y) 
	& \leq d(g^* \cdot x, y) \\
	&= d(a^*, b^*) \\
	& < \bar{d}(\bar{x}, \bar{y}) + \ep
\end{align*}	 
	Since $\ep >0$ is arbitrary, $$\inf_{g \in G} d(g \cdot x, y) \leq \bar{d}(\bar{x}, \bar{y})$$
	Conversely, since $\{(g \cdot x, y): g \in G\} \subset \{(a,b): a \in \bar{x}, b \in \bar{y}\}$, we have that 
	$$\inf_{g \in G} d(g \cdot x, y) \geq \bar{d}(\bar{x}, \bar{y})$$ 
	\end{proof}
	
	\begin{ex} \lex{}
	Let $(X, d)$ be a metric space, $G$ a group, and $\phi: G \times X \rightarrow X$ an isometric group action. Then for each $x, y, z \in X$, $$\bar{d}(\bar{x}, \bar{y}) \leq \bar{d}(\bar{x}, \bar{z}) + \bar{d}(\bar{z}, \bar{y})$$
	\end{ex}
	
	\begin{proof}
	Let $x, y, z \in X$. An exercise in section $(2.1)$ implies that $d(\bar{x}, \bar{y}) \leq d(\bar{x}, z) + d(z, \bar{y})$. The previous exercise implies that 
	\begin{align*}
	d(\bar{x}, z) 
	&= \inf_{a \in \bar{x}} d(a, z) \\
	&= \inf_{g \in G} d(g \cdot x, z) \\
	&= \bar{d}(\bar{x}, \bar{z})
	\end{align*}
	Similarly, $d(z, \bar{y}) = \bar{d}(\bar{z}, \bar{y})$. Then 
	\begin{align*}
	d(\bar{x}, \bar{y}) 
	&\leq d(\bar{x}, z) + d(z, \bar{y}) \\
	&= \bar{d}(\bar{x}, \bar{z}) + \bar{d}(\bar{z}, \bar{y})
	\end{align*}
	\end{proof}
	
	\begin{ex} \lex{}
	Let $(X, d)$ be a metric space, $G$ a group, and $\phi: G \times X \rightarrow X$ an isometric group action. If for each $x \in X$, $\bar{x}$ is closed, then for each $x, y \in X$, $\bar{d}(\bar{x}, \bar{y}) =0$ implies that $\bar{x} = \bar{y}$.
	\end{ex}
	
	\begin{proof}
	Suppose that for each $x \in X$, $\bar{x}$ is closed. Let $x,y \in X$. Suppose that $\bar{d}(\bar{x} , \bar{y}) = 0$. Then $\inf\limits_{ g \in G} d(g \cdot x, y) = 0$. Hence there exists $(g_n)_{n \in N} \subset G$ such that $g_n \cdot x \rightarrow y$. Since $(g_n \cdot x)_{n \in \N} \subset \bar{x}$ and $\bar{x}$ is closed, $y \in \bar{x}$. Thus $\bar{x} = \bar{y}$. 
	\end{proof}
	
	\begin{ex} \lex{}
	Let $(X, d)$ be a metric space, $G$ a group, and $\phi: G \times X \rightarrow X$ an isometric group action. If for each $x \in X$, $\bar{x}$ is closed, then $\bar{d}$ is a metric on $X/G$.
	\end{ex}
	
	\begin{proof}
	Clear by preceeding exercises.
	\end{proof}
	
	\begin{ex} \lex{}
	Let $(X, d)$ be a metric space, $(G, \tau)$ a topological group, and $\phi: G \times X \rightarrow X$ an isometric group action. Suppose that $G$ is compact and for each $x \in X$, the map $g \mapsto g \cdot x$ is continuous. Then $\bar{d}$ is a metric on $X/G$. 
	\end{ex}
	
	\begin{proof}
	Let $x \in X$. Since $G$ is compact and the map $g \mapsto g \cdot x$ is continuous, $\bar{x} = G \cdot x$ is compact and therefore closed. The previous exercise implies that $\bar{d}$ is a metric.
	\end{proof}
	
	\begin{ex} \lex{}
	Let $(X, d)$ be a metric space, $G$ a group, and $\phi: G \times X \rightarrow X$ an isometric group action. Suppose that $\bar{d}$ is a metric on $X/G$. Then the projection map $\pi: X \rightarrow X/G$ is Lipschitz and therefore continuous.
	\end{ex}
	
	\begin{proof}
	Let $x,y \in X$. Then
	\begin{align*}
	\bar{d}(\pi(x), \pi(y)) 
	&= \bar{d}(\bar{x}, \bar{y}) \\
	&= \inf_{g \in G} d(g \cdot x, y)\\
	& \leq d(x,y)  \\
	\end{align*}
	\end{proof}
	
	\begin{ex} \lex{}
	Let $(X, d)$ be a metric space, $G$ a group, and $\phi: G \times X \rightarrow X$ an isometric group action. Suppose that $\bar{d}$ is a metric on $X/G$. Let $(x_n)_{n \in \N} \subset X$ and $x \in X$. Then $\bar{x}_n \conv{\bar{d}} \bar{x}$ iff there exists a sequence $(g_n)_{n \in \N}$ such that $g_n \cdot x_n \conv{d} x$.
	\end{ex}
	
	\begin{proof} 
	Suppose that $\bar{x}_n \conv{\bar{d}} \bar{x}$. For $n \in \N$, choose $g_n \in G$ such that $d(g_n \cdot x_n, x) < \bar{d}(\bar{x}_n, \bar{x}) + 2^{-n}$. Then $d(g_n \cdot x_n, x) \rightarrow 0$ and $g_n \cdot x_n \conv{d} x$.  \\
	Conversely, suppose that that there exists a sequence $(g_n)_{n \in \N}$ such that $g_n \cdot x_n \conv{d} x$. Since $\pi:X \rightarrow X/G$ is continuous, we have that
	\begin{align*}
	g_n \cdot x_n \conv{d} x
	& \implies \pi(g_n \cdot x_n) \conv{\bar{d}} \pi(x)\\
	& \implies \bar{x}_n  \conv{\bar{d}} \bar{x}
	\end{align*}
	\end{proof}		
	
	\begin{ex} \lex{}
	Let $X$ be a set, $d_1, d_2: X^2 \rightarrow \Rg$ metrics, $G$ a group and $\phi: G \times X \rightarrow X$ an isometric group action. Suppose that $d_1$ and $d_2$ are topologically equivalent. 
	\begin{enumerate}
	\item Then $\bar{d}_1$ is a metric on $X/G$ iff $\bar{d}_2$ is a metric on $X/G$
	\item If $\bar{d}_1$ and $\bar{d}_2$ are metrics, then $\bar{d}_1$ and $\bar{d}_2$ are topologically equivalent. 
	\end{enumerate}
	\end{ex}
	
	\begin{proof}\
	\begin{enumerate}
	\item 
	\begin{itemize}
	\item $\implies$ Suppose that $\bar{d}_1$ is a metric. Let $x,y \in X$. Suppose that $\bar{d}_2(\bar{x}, \bar{y}) = 0$. Then there exist $(g_n)_{n \in \N} \subset G$ such that $d_2(g_n \cdot x, y) \rightarrow 0$. Since $d_1$ and $d_2$ are topologically equivalent, $d_1(g_n \cdot x, y) \rightarrow 0$. Thus $\bar{d}_1(\bar{x}, \bar{y}) = 0$. Since $\bar{d}_1$ is a metric, $\bar{x} = \bar{y}$. Hence $\bar{d}_2$ is a metric. 
	\item $\impliedby$ Similar.
	\end{itemize}
	\item Suppose that $\bar{d}_1$ and $\bar{d}_2$ are metrics. Let $(\bar{x}_n)_{n \in \N} \subset X/G$ and $\bar{x} \in X/G$. 
	\begin{itemize}
	\item Suppose that $\bar{x}_n \conv{\bar{d}_1} \bar{x}$. Then there exists a sequence $(g_n)_{n \in \N}$ such that $g_n \cdot x_n \conv{d_1} x$. Since $d_1$ and $d_2$ are topologically equivalent, $g_n \cdot x_n \conv{d_2} x$. This implies that $\bar{x}_n \conv{\bar{d}_2} \bar{x}$. 
	\item Suppose that $\bar{x}_n \conv{\bar{d}_2} \bar{x}$. Then similarly to above, $\bar{x}_n \conv{\bar{d}_1} \bar{x}$.
	\end{itemize}
	\end{enumerate}
	\end{proof}	
	
	\begin{ex} \lex{}
	Let $X$ be a set, $d_1, d_2: X^2 \rightarrow \Rg$ metrics on $X$, $G$ a group and $\phi: G \times X \rightarrow X$ an isometric group action. Suppose that $d_1$ and $d_2$ are equivalent. If $\bar{d}_1$ and $\bar{d}_2$ are metrics, then $\bar{d}_1$ and $\bar{d}_2$ are equivalent.
	\end{ex}
	
	\begin{proof} Suppose that $\bar{d}_1$ and $\bar{d}_2$  are metrics. Since $d_1$ $d_2$ are equivalent, there exist $C_1, C_2 >0$ such that for each $x,y \in X$, $C_1d_1(x,y) \leq d_2(x,y) \leq C_2d_1(x,y)$. Let $x,y \in X$. Then
	\begin{align*}
	C_1\bar{d}_1(\bar{x}, \bar{y}) 
	&= C_1 \inf_{g \in G} d_1(g \cdot x, y) \\
	&=  \inf_{g \in G} C_1 d_1(g \cdot x, y) \\
	&\leq \inf_{g \in G} d_2(g \cdot x, y) \\
	&= \bar{d}_2(\bar{x}, \bar{y}) \\
	\end{align*}	 
	and 
	\begin{align*}
	\bar{d}_2(\bar{x}, \bar{y}) 
	&= \inf_{g \in G} d_2(g \cdot x, y) \\	
	& \leq \inf_{g \in G} C_2 d_1(g \cdot x, y) \\
	&= C_2 \inf_{g \in G}  d_1(g \cdot x, y) \\
	&= C_2 \bar{d}_1(\bar{x}, \bar{y})
	\end{align*}
	So that $C_1 \bar{d}_1 \leq \bar{d}_2 \leq C_2 \bar{d}_1$
	\end{proof}
	
	\begin{ex}
	Let $(X,d)$ be a metric space, $G$ a group and $\phi: G \times X \rightarrow X$ an isometric group action. Suppose that $\bar{d}$ is a metric. Then $\pi:X \rightarrow X/G$ is a quotient map.
	\end{ex}
	
	\begin{proof}\
	\begin{itemize}
	\item Clearly $\pi$ is surjective. 
	\item Let $C \subset X/G$. Suppose that $C$ is closed. Since $\pi$ is continuous, if $\pi^{-1}(C)$ is closed. \\
	Conversely, suppose that $\pi^{-1}(C)$ is closed. Let $(\bar{x}_{\al})_{\al} \subset C$ be a net and $\bar{x} \in X/G$. Suppose that $\bar{x}_{\al} \rightarrow \bar{x}$. Then there exists $(g_{\al})_{\al \in A} \subset G$ such that $g_{\al} \cdot x_{\al} \rightarrow x$. Since $(g_{\al} \cdot x_{\al})_{\al \in A} \subset \pi^{-1}(C)$, $x \in \pi^{-1}(C)$. Hence $\bar{x} \in C$ and $C$ is closed. Then \rex{34003} implies that $\pi$ is a quotient map.
	\end{itemize}
	\end{proof}
	
	\begin{ex}
	Let $(X,d)$ be a metric space, $G$ a group and $\phi: G \times X \rightarrow X$ an isometric group action. Suppose that $\bar{d}$ is a metric. Then $\pi:X \rightarrow X/G$ is open.
	\end{ex}
	
	\begin{proof}
	Let $U \subset X$. Suppose that $U$ is open. Then 
	\begin{equation*}
	\pi^{-1}(\pi(U)) = \bigcup_{g \in G} g \cdot U
	\end{equation*}		
	Since for each $g \in G$, $l_g \in \Homeo(X)$, we have that for each $g \in G$, $g \cdot U$ is open. Therefore $\bigcup\limits_{g \in G} g \cdot U$ is open. Hence $\pi^{-1}(\pi(U))$ is open. Then \rex{34005} implies that $\pi$ is open.
	\end{proof}
	
	\begin{ex}
	Let $(X,d)$ be a metric space, $G$ a group and $\phi: G \times X \rightarrow X$ an isometric group action. Suppose that $\bar{d}$ is a metric. Then $\bar{d}$ metrizes the quotient topology $\pi_*\tau(d)$ on $X/G$.
	\end{ex}
	
	\begin{proof}
	Immediate by the previous exercise and \rex{34008}.
	\end{proof}
	
	\begin{ex}
	Let $(X, d)$ be a metric space, $G$ a group, and $\phi: G \times X \rightarrow X$ an isometric group action. Let $f: X \rightarrow \C$. Suppose that $f$ is $G$-invariant and $\bar{d}$ is a metric. If $f \in C(X)$, then $\bar{f} \in C(X/G)$.  \\
	\tbf{Hint:} \rex{34008}
	\end{ex}
	
	\begin{proof}
	Suppose that $f \in C(X)$. \rex{34008} implies that $\bar{f}: X \rightarrow \C$ is the unique map such that $\bar{f} \circ \pi = f$ and $\bar{f}$ is continuous. 
	\end{proof}
	
	\begin{ex}
	Let $(X, d)$ be a metric space, $G$ a group, and $\phi: G \times X \rightarrow X$ an isometric group action. Let $f: X \rightarrow \C$. Suppose that $f$ is $G$-invariant and $\bar{d}$ is a metric. If $f \in C(X)$, then $\bar{f} \in C(X/G)$.  \\
	\tbf{Hint:} \rex{34008}
	\end{ex}
	
	\begin{proof}
	Suppose that $f \in C(X)$. \rex{34008} implies that $\bar{f}: X \rightarrow \C$ is the unique map such that $\bar{f} \circ \pi = f$ and $\bar{f}$ is continuous. 
	\end{proof}
	
	
	
	
	
	
	
	
	
	
	
	
	
	
	
	
	
	
	
	
	
	

	
	
	
	
	
	
	
	
	
	
	\newpage
	\subsection{Fundamental Examples}
	\begin{note}
	This section uses results from the previous two sections to establish metrics on some fundamental orbit spaces of metric spaces under a group action. 
	\end{note} 
		
	
	\begin{ex} \lex{} \tbf{Procrustes Distance:} \\
	Consider the metric space $(\C^{n \times d}, \|\cdot\|_F)$, topological group $(U_d, \|\cdot\|_F)$ and  the (right) action $\phi: X \times U_d \rightarrow X$ by $X \cdot U = XU$. Then 
	\begin{enumerate}
	\item $\phi$ is a continuous isometric group action 
	\item $U_d$ is compact 
	\item $\bar{d}$ is a metric on $\C^{n \times d}/ U_d$
	\end{enumerate}
	\end{ex}
	
	\begin{proof}
	Clear.
	\end{proof}		
	
	\begin{ex} \lex{}
	Let $X$ be a compact metric space and $\mu:\MB(X) \rightarrow \RG$ a Borel measure. Define the (right) group action $\phi: L^1(\mu) \times \Aut(X, \mu) \rightarrow L^1(\mu) $ by $$f \cdot \sig = f \circ \sig$$ Then $\phi$ is an isometric group action. 
	\end{ex}
	
	\begin{proof}
	Let $\sig \in \Aut(X, \mu)$ and $f \in L^1(\mu)$. 
Then 
	\begin{align*}
	\|f \cdot \sig\|_1
	&=  \int_X |f \circ \sig| d\mu \\
	&=  \int_X |f| \circ \sig d\mu \\
	&=  \int_{\sig(X)} |f| d \sig_* \mu  \\
	&=  \int_{\sig(X)} |f| d \mu \\
	&=  \int_{X} |f| d \mu \\
	&= \|f\|_1 
	\end{align*}	 
	\end{proof}
	
	\begin{ex} \lex{}
	Let $X$ be a compact metric space and $\mu:\MB(X) \rightarrow \RG$ a Radon measure. Define the (right) group action $\phi: L^1(\mu) \times \Aut(X, \mu) \rightarrow L^1(\mu) $ by 
	$$f \cdot \sig = f \circ \sig$$	
	Then for each $f \in L^1(\mu)$, the map $\sig \mapsto  f \cdot \sig$ is continuous.  
	\end{ex}
	
	\begin{proof}
	Let $f \in L^1(\mu)$, $(\sig_n)_{n \in \N} \subset \Aut(X, \mu)$ and $\sig \in \Aut(X, \mu)$. Suppose that $\sig_n \convt{u} \sig$. Since $\mu$ is Radon, $C_c(X)$ is dense in $L^1(\mu)$ and therefore, there exists $\phi \in C_c(X)$ such that $\|\phi - f\| < \ep/3$. Since $X$ is compact and $\mu$ is Radon, $\mu(X) < \infty$. Since $\phi$ is uniformly continuous, \rex{211111111} implies that $\phi \circ \sig_n \convt{u} \phi \circ \sig$.  So there exists $N \in \N$ such that for each $n \in \N$, $n \geq N$ implies that $\|\phi \circ \sig_n - \phi \circ \sig\|_u < \frac{\ep}{3 (\mu(X)+1)}$. Let $n \in \N$. Suppose that $n \geq \N$. Then 
	\begin{align*}
	\|f \circ \sig_n - f \circ \sig\|_1 
	&\leq \|f \circ \sig_n - \phi \circ \sig_n \|_1 + \|\phi \circ \sig_n - \phi \circ \sig\|_1 + \|\phi \circ \sig - f \circ \sig\|_1 \\
	& = \|(f - \phi) \circ \sig_n \|_1 + \|\phi \circ \sig_n - \phi \circ \sig\|_1 + \|(\phi - f) \circ \sig\|_1 \\
	&= \|f - \phi  \|_1 + \|\phi \circ \sig_n - \phi \circ \sig\|_1 + \|\phi - f \|_1 \\
	&= \|f - \phi  \|_1 + \|\phi \circ \sig_n - \phi \circ \sig\|_u \mu(X) + \|\phi - f \|_1 \\
	&< \frac{\ep}{3} + \frac{\ep}{3} + \frac{\ep}{3} \\
	&= \ep
	\end{align*}
	So that $f \circ \sig_n \convt{u} f \circ \sig$ which implies that the map $\sig \mapsto  f \cdot \sig$ is continuous. 
	\end{proof}
	
	\begin{ex} \lex{} \tbf{Cut Distance:} \\
	Let $X$ be a compact metric space and $\mu:\MB(X) \rightarrow \RG$ a Radon measure. Define the (right) group action $\phi: L^1(\mu) \times \Aut(X, \mu) \rightarrow L^1(\mu) $ by 
	$$f \cdot \sig = f \circ \sig$$	
	Then 
	\begin{enumerate}
	\item $\phi$ is an isometric group action 
	\item $\Aut(X, d, \mu)$ is compact 
	\item for each $f \in L^1(\mu)$, the map $\sig \mapsto  f \cdot \sig$ is continuous.  
	\item $\bar{d}$ is a metric on $L^1(\mu) / \Aut(X, d, \mu)$
	\end{enumerate}
	\end{ex}
	
	\begin{proof}
	Clear by the preceeding exercises.
	\end{proof}
	
	\begin{note}
	The preceeding distance is not quite the Cut distance, as the Cut norm only considers a subset of measurable sets for a function of two variables, but with some work, maybe I can show it is a distance.
	\end{note}
	
	
	
	
	
	
	
	
	
	
	
	\newpage
	\section{Appendix}
	
	\subsection{Summation}
	
	\begin{defn} \ld{}
		Let $f:X \rightarrow \Rg$, Then we define $$\sum_{x \in X} f(x) := \sup_{\substack{F \subset X \\ F \text{ finite}}} \sum_{x \in F} f(x)$$ This definition coincides with the usual notion of summation when $X$ is countable. For $f:X \rightarrow \C$, we can write $f = g +ih$ where $g,h:X \rightarrow \R$. If $$\sum_{x \in X}|f(x)| < \infty,$$ then the same is true for $g^+,g^-,h^+,h^-$. In this case, we may define $$\sum_{x \in X} f(x)$$ in the obvious way.
	\end{defn} 
	
	The following note justifies the notation $\sum_{x \in X}f(x)$ where $f:X \rightarrow \C$.
	
	\begin{note}
		Let $f:X \rightarrow \C$ and $\al:X \rightarrow X$ a bijection. If $\sum_{x \in X}|f(x)|< \infty$, then $\sum_{x \in X}f( \al (x)) = \sum_{x \in X}f(x) $.
	\end{note}
	
	\newpage	
	
	\subsection{Asymptotic Notation}
	
	\begin{defn} \ld{}
	Let $X$ be a topological space, $Y, Z$ be normed vector spaces, $f:X \rightarrow Y$, $g: X \rightarrow Z$ and $x_0 \in X \cup \{\infty\}$. Then we write $$f = o(g) \hspace{.5cm} \text{ as } x \rightarrow x_0$$ if for each $\ep >0$, there exists $U \in \MN_{x_0}$ such that for each $x \in U$, $$\|f(x)\| \leq \ep\|g(x)\|$$
	\end{defn}
	
	\begin{ex} \lex{}
	Let $X$ be a topological space, $Y, Z$ be normed vector spaces, $f:X \rightarrow Y$, $g: X \rightarrow Z$ and $x_0 \in X \cup \{\infty\}$. If there exists $U \in \MN_{x_0}$ such that for each $x \in U \setminus \{x_0\}$, $g(x) > 0$, then $$f = o(g) \text{ as } x \rightarrow x_0 \hspace{.25cm} \text{ iff } \hspace{.25cm}  \lim_{x \rightarrow x_0} \frac{\| f(x) \|}{\| g(x) \|} = 0$$
	\end{ex}	
	
	\begin{ex} \lex{}
	Let $X$ and $Y$ a be normed vector spaces, $A \subset X$ open and $f:A \rightarrow Y$. Suppose that $0 \in A$. If $f(h) = o(\|h\|)$ as $h \rightarrow 0$, then for each $h \in X$,  $f(th) = o(|t|)$ as $t \rightarrow 0$.
	\end{ex}	
	
	\begin{proof}
	Suppose that $f(h) = o(\|h\|)$ as $h \rightarrow 0$.  Let $h \in X$ and $\ep >0$. Choose $\del' >0 $ such that for each $h' \in B(0, \del')$, $h' \in A$ and 
	$$\|f(h')\| \leq \frac{\ep}{\|h\|+1} \|h'\|$$ 
	Choose $\del >0$ such that for each $t \in B(0,\del)$, $th \in B(0,\del')$. Let $t \in B(0,\del)$. Then 
	\begin{align*}
	\|f(th)\| 
	&\leq \frac{\ep}{\|h\|+1} |t|\|h\| \\
	&< \ep |t|
	\end{align*}
	So $f(th) = o(|t|)$ as $t \rightarrow 0$.
	\end{proof}		
	
	
	
	
	\begin{defn} \ld{}
	Let $X$ be a topological space, $Y, Z$ be normed vector spaces, $f:X \rightarrow Y$, $g: X \rightarrow Z$ and $x_0 \in X \cup \{\infty\}$. Then we write $$f = O(g) \hspace{.5cm} \text{ as } x \rightarrow x_0$$ if there exists $U \in \MN_{x_0}$ and $M \geq 0$ such that for each $x \in U$, $$\|f(x)\| \leq M\|g(x)\|$$
	\end{defn}
	
	
	
	
	
	
	
	
	
	\newpage
	\begin{thebibliography}{4}
\bibitem{algebra} \href{https://github.com/carsonaj/Mathematics/blob/master/Introduction\%20to\%20Algebra/Introduction\%20to\%20Algebra.pdf}{Introduction to Algebra}

\bibitem{analysis}  \href{https://github.com/carsonaj/Mathematics/blob/master/Introduction\%20to\%20Analysis/Introduction\%20to\%20Analysis.pdf}{Introduction to Analysis}	

\bibitem{foranal}  \href{https://github.com/carsonaj/Mathematics/blob/master/Introduction\%20to\%20Fourier\%20Analysis/Introduction\%20to\%20Fourier\%20Analysis.pdf}{Introduction to Fourier Analysis}

\bibitem{measure}  \href{https://github.com/carsonaj/Mathematics/blob/master/Introduction\%20to\%20Measure\%20and\%20Integration/Introduction\%20to\%20Measure\%20and\%20Integration.pdf}{Introduction to Measure and Integration}



\end{thebibliography}


	
	
	
	
	
	
	
	
	
	
	
	
	
	
	
	
	
	
	
	
	
	\end{document}
	
	