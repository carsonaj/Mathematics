%% filename: amsbook-template.tex
%% version: 1.1
%% date: 2014/07/24
%%
%% American Mathematical Society
%% Technical Support
%% Publications Technical Group
%% 201 Charles Street
%% Providence, RI 02904
%% USA
%% tel: (401) 455-4080
%%      (800) 321-4267 (USA and Canada only)
%% fax: (401) 331-3842
%% email: tech-support@ams.org
%% 
%% Copyright 2006, 2008-2010, 2014 American Mathematical Society.
%% 
%% This work may be distributed and/or modified under the
%% conditions of the LaTeX Project Public License, either version 1.3c
%% of this license or (at your option) any later version.
%% The latest version of this license is in
%%   http://www.latex-project.org/lppl.txt
%% and version 1.3c or later is part of all distributions of LaTeX
%% version 2005/12/01 or later.
%% 
%% This work has the LPPL maintenance status `maintained'.
%% 
%% The Current Maintainer of this work is the American Mathematical
%% Society.
%%
%% ====================================================================

%    AMS-LaTeX v.2 driver file template for use with amsbook
%
%    Remove any commented or uncommented macros you do not use.

\documentclass{book}

%    For use when working on individual chapters
%\includeonly{}

%    For use when working on individual chapters
%\includeonly{}

%    Include referenced packages here.
\usepackage[left =.5in, right = .5in, top = 1in, bottom = 1in]{geometry} 
\usepackage{amsmath}
\usepackage{amsthm}
\usepackage{amssymb}
\usepackage{setspace}
\usepackage{mathtools}
\usepackage{tikz}  
\usepackage{tikz-cd}
\usepackage{tkz-fct}
\usepackage{pgfplots}
\usepackage{environ}
\usepackage{tikz-cd} 
\usepackage{enumitem}
\usepackage{color}   %May be necessary if you want to color links
%\usepackage{xr}

\usepackage{hyperref}
\hypersetup{
	colorlinks=true, %set true if you want colored links
	linktoc=all,     %set to all if you want both sections and subsections linked
	linkcolor=black,  %choose some color if you want links to stand out
	urlcolor=cyan
}
\usepackage[symbols,nogroupskip,sort=none]{glossaries-extra}

\pgfplotsset{every axis/.append style={
		axis x line=middle,    % put the x axis in the middle
		axis y line=middle,    % put the y axis in the middle
		axis line style={<->,color=black}, % arrows on the axis
		xlabel={$x$},          % default put x on x-axis
		ylabel={$y$},          % default put y on y-axis
}}


\theoremstyle{definition}
\newtheorem{definition}{Definition}[subsection]
\newtheorem{defn}[definition]{Definition}
\newtheorem{note}[definition]{Note}
\newtheorem{ax}[definition]{Axiom}
\newtheorem{thm}[definition]{Theorem}
\newtheorem{lem}[definition]{Lemma}
\newtheorem{prop}[definition]{Proposition}
\newtheorem{cor}[definition]{Corollary}
\newtheorem{conj}[definition]{Conjecture}
\newtheorem{ex}[definition]{Exercise}
\newtheorem{exmp}[definition]{Example}
\newtheorem{soln}[definition]{Solution}

\setcounter{tocdepth}{3}

% hide proofs
\newif\ifhideproofs
%\hideproofstrue %uncomment to hide proofs
\ifhideproofs
\NewEnviron{hide}{}
\let\proof\hide
\let\endproof\endhide
\fi

% lower-case greek
\newcommand{\al}{\alpha}
\newcommand{\be}{\beta}
\newcommand{\gam}{\gamma}
\newcommand{\del}{\delta}
\newcommand{\ep}{\epsilon}
\newcommand{\ze}{\zeta} 
\newcommand{\kap}{\kappa} 
\newcommand{\lam}{\lambda}  
\newcommand{\sig}{\sigma} 
\newcommand{\omi}{\omicron}
\newcommand{\up}{\upsilon}
\newcommand{\om}{\omega}

% upper-case greek
\newcommand{\Gam}{\Gamma}
\newcommand{\Del}{\Delta}
\newcommand{\Lam}{\Lambda} 
\newcommand{\Sig}{\Sigma} 
\newcommand{\Om}{\Omega}

% blackboard bold
\newcommand{\C}{\mathbb{C}}
\newcommand{\E}{\mathbb{E}}
\newcommand{\F}{\mathbb{F}}
\renewcommand{\H}{\mathbb{H}}
\newcommand{\K}{\mathbb{K}}
\newcommand{\N}{\mathbb{N}}
\renewcommand{\O}{\mathbb{O}}
\newcommand{\Q}{\mathbb{Q}}
\newcommand{\R}{\mathbb{R}}
\renewcommand{\S}{\mathbb{S}}
\newcommand{\T}{\mathbb{T}}
\newcommand{\V}{\mathbb{V}}
\newcommand{\Z}{\mathbb{Z}}

% math caligraphic
\newcommand{\MA}{\mathcal{A}}
\newcommand{\MB}{\mathcal{B}}
\newcommand{\MC}{\mathcal{C}}
\newcommand{\MD}{\mathcal{D}}
\newcommand{\ME}{\mathcal{E}}
\newcommand{\MF}{\mathcal{F}}
\newcommand{\MG}{\mathcal{G}}
\newcommand{\MH}{\mathcal{H}}
\newcommand{\MI}{\mathcal{I}}
\newcommand{\MJ}{\mathcal{J}}
\newcommand{\MK}{\mathcal{K}}
\newcommand{\ML}{\mathcal{L}}
\newcommand{\MM}{\mathcal{M}}
\newcommand{\MN}{\mathcal{N}}
\newcommand{\MO}{\mathcal{O}}
\newcommand{\MP}{\mathcal{P}}
\newcommand{\MQ}{\mathcal{Q}}
\newcommand{\MR}{\mathcal{R}}
\newcommand{\MS}{\mathcal{S}}
\newcommand{\MT}{\mathcal{T}}
\newcommand{\MU}{\mathcal{U}}
\newcommand{\MV}{\mathcal{V}}
\newcommand{\MW}{\mathcal{W}}
\newcommand{\MX}{\mathcal{X}}
\newcommand{\MY}{\mathcal{Y}}
\newcommand{\MZ}{\mathcal{Z}}

% mathfrak
\newcommand{\MFX}{\mathfrak{X}}
\newcommand{\MFg}{\mathfrak{g}}
\newcommand{\MFh}{\mathfrak{h}}

% tilde 
\newcommand{\tMA}{\tilde{\MA}}
\newcommand{\tMB}{\tilde{\MB}}
\newcommand{\tU}{\tilde{U}}
\newcommand{\tV}{\tilde{V}}
\newcommand{\tphi}{\tilde{\phi}}
\newcommand{\tpsi}{\tilde{\psi}}
\newcommand{\tF}{\tilde{F}}

\newcommand{\iid}{\stackrel{iid}{\sim}}





% label/reference
% internal label/reference
\newcommand{\lex}[1]{\label{ex:#1}}
\newcommand{\rex}[1]{Exercise \ref{ex:#1}}

\newcommand{\ld}[1]{\label{defn:#1}}
\newcommand{\rd}[1]{Definition \ref{defn:#1}}

\newcommand{\lax}[1]{\label{ax:#1}}
\newcommand{\rax}[1]{Axiom \ref{ax:#1}}

\newcommand{\lfig}[1]{\label{fig:#1}}
\newcommand{\rfig}[1]{Figure \ref{fig:#1}}

% external reference
\newcommand{\extrex}[2]{Exercise \ref{#1-ex:#2}}

\newcommand{\extrd}[2]{Definition \ref{#1-defn:#2}}

\newcommand{\extrax}[2]{Axiom \ref{#1-ax:#2}}

\newcommand{\extrfig}[2]{Figure \ref{#1-fig:#2}}

% external documents (EDIT HERE)
%\externaldocument[analysis-]{"/home/carson/Desktop/Github/Mathematics/Introduction to Analysis/Introduction to Analysis.tex"}




% math operators
\DeclareMathOperator{\supp}{supp}
\DeclareMathOperator{\sgn}{sgn}
\DeclareMathOperator{\spn}{span}
\DeclareMathOperator{\Iso}{Iso}
\DeclareMathOperator{\Eq}{Eq}
\DeclareMathOperator{\id}{id}
\DeclareMathOperator{\Aut}{Aut}
\DeclareMathOperator{\Endo}{End}
\DeclareMathOperator{\Homeo}{Homeo}
\DeclareMathOperator{\Sym}{Sym}
\DeclareMathOperator{\Alt}{Alt}
\DeclareMathOperator{\cl}{cl}
\DeclareMathOperator{\Int}{Int}
\DeclareMathOperator{\bal}{bal}
\DeclareMathOperator{\cyc}{cyc}
\DeclareMathOperator{\cnv}{conv}
\DeclareMathOperator{\epi}{epi}
\DeclareMathOperator{\dom}{dom}
\DeclareMathOperator{\cod}{cod}
\DeclareMathOperator{\codim}{codim}
\DeclareMathOperator{\Obj}{Obj}
\DeclareMathOperator{\Derivinf}{Deriv^{\infty}}
\DeclareMathOperator{\Hom}{Hom}
\DeclareMathOperator*{\argmax}{arg\,max}
\DeclareMathOperator*{\argmin}{arg\,min}
\DeclareMathOperator{\diam}{\text{diam}}
\DeclareMathOperator{\rnk}{\text{rank}}
\DeclareMathOperator{\tr}{\text{tr}}
\DeclareMathOperator{\prj}{\text{proj}}
\DeclareMathOperator{\nab}{\nabla}
\DeclareMathOperator{\diag}{\text{diag}}
\DeclareMathOperator*{\ind}{\text{ind}}
\DeclareMathOperator*{\ar}{\text{arity}}
\DeclareMathOperator*{\cur}{\text{cur}}
\DeclareMathOperator*{\Part}{\text{Part}}
\DeclareMathOperator{\Var}{\text{Var}}
\DeclareMathOperator*{\FIP}{\text{FIP}} 
\DeclareMathOperator*{\Fun}{\text{Fun}} 
\DeclareMathOperator*{\Rel}{\text{Rel}} 
\DeclareMathOperator*{\Cons}{\text{Cons}} 
\DeclareMathOperator*{\Sg}{\text{Sg}} 
\DeclareMathOperator*{\ot}{\otimes}
\DeclareMathOperator{\uni}{Uni}

% Algebra
\DeclareMathOperator{\inv}{\text{inv}}
\DeclareMathOperator{\mult}{\text{mult}}
\DeclareMathOperator{\smult}{\text{smult}}

% category theory
\DeclareMathOperator*{\Set}{\text{\tbf{Set}}}
\DeclareMathOperator*{\BanAlg}{\text{\tbf{BanAlg}}}
\DeclareMathOperator*{\Meas}{\text{\tbf{Meas}}}
\DeclareMathOperator*{\TopMeas}{\text{\tbf{TopMeas}}}
\DeclareMathOperator*{\Msrpos}{\text{\tbf{Msr}}_{+}}
\DeclareMathOperator*{\TopMsrpos}{\text{\tbf{TopMsr}}_{+}}
\DeclareMathOperator*{\TopRadMsrpos}{\text{\tbf{TopRadMsr}}_{+}}
\DeclareMathOperator*{\TopRadMsrone}{\text{\tbf{TopRadMsr}}_{1}}
\DeclareMathOperator*{\MsrC}{\text{\tbf{Msr}}_{\C}} 
\DeclareMathOperator*{\TopMsrC}{\text{\tbf{TopMsr}}_{\C}} 
\DeclareMathOperator*{\TopRadMsrC}{\text{\tbf{TopRadMsr}}_{\C}} 
\DeclareMathOperator*{\Maninf}{\text{\tbf{Man}}^{\infty}} 
\DeclareMathOperator*{\ManBndinf}{\text{\tbf{ManBnd}}^{\infty}} 
\DeclareMathOperator*{\Man0}{\text{\tbf{Man}}^{0}}
\DeclareMathOperator*{\Buninf}{\text{\tbf{Bun}}^{\infty}} 
\DeclareMathOperator*{\VecBuninf}{\text{\tbf{VecBun}}^{\infty}} 
\DeclareMathOperator*{\Field}{\text{\tbf{Field}}} 
\DeclareMathOperator*{\Mon}{\text{\tbf{Mon}}} 
\DeclareMathOperator*{\Grp}{\text{\tbf{Grp}}}
\DeclareMathOperator*{\Semgrp}{\text{\tbf{Semgrp}}}
\DeclareMathOperator*{\LieGrp}{\text{\tbf{LieGrp}}} 
\DeclareMathOperator*{\Alg}{\text{\tbf{Alg}}} 
\DeclareMathOperator*{\Vect}{\text{\tbf{Vect}}} 
\DeclareMathOperator*{\Mod}{\text{\tbf{Mod}}}
\DeclareMathOperator*{\Rep}{\text{\tbf{Rep}}} 
\DeclareMathOperator*{\URep}{\text{\tbf{URep}}}
\DeclareMathOperator*{\Ban}{\text{\tbf{Ban}}} 
\DeclareMathOperator*{\Hilb}{\text{\tbf{Hilb}}} 
\DeclareMathOperator*{\Prob}{\text{\tbf{Prob}}} 
\DeclareMathOperator*{\PrinBuninf}{\text{\tbf{PrinBun}}^{\infty}}

\DeclareMathOperator*{\Top}{\text{\tbf{Top}}}
\DeclareMathOperator*{\TopField}{\text{\tbf{TopField}}} 
\DeclareMathOperator*{\TopMon}{\text{\tbf{TopMon}}} 
\DeclareMathOperator*{\TopGrp}{\text{\tbf{TopGrp}}}
\DeclareMathOperator*{\TopVect}{\text{\tbf{TopVect}}} 
\DeclareMathOperator*{\TopEq}{\text{\tbf{TopEq}}}

\DeclareMathOperator*{\VectR}{\text{\tbf{Vect}}_{\R}}
\DeclareMathOperator*{\VectC}{\text{\tbf{Vect}}_{\C}} 
\DeclareMathOperator*{\VectK}{\text{\tbf{Vect}}_{\K}}
\DeclareMathOperator*{\Cat}{\text{\tbf{Cat}}}
\DeclareMathOperator*{\0}{\mbf{0}}
\DeclareMathOperator*{\1}{\mbf{1}}


\DeclareMathOperator*{\Cone}{\text{\tbf{Cone}}}

\DeclareMathOperator*{\Cocone}{\text{\tbf{Cocone}}}


% Algebra
\DeclareMathOperator{\End}{\text{End}} 
\DeclareMathOperator{\rep}{\text{Rep}} 




% notation
\renewcommand{\r}{\rangle}
\renewcommand{\l}{\langle}
\renewcommand{\div}{\text{div}}
\renewcommand{\Re}{\text{Re} \,}
\renewcommand{\Im}{\text{Im} \,}
\newcommand{\Img}{\text{Img} \,}
\newcommand{\grad}{\text{grad}}
\newcommand{\tbf}[1]{\textbf{#1}}
\newcommand{\tcb}[1]{\textcolor{blue}{#1}}
\newcommand{\tcr}[1]{\textcolor{red}{#1}}
\newcommand{\mbf}[1]{\mathbf{#1}}
\newcommand{\ol}[1]{\overline{#1}}
\newcommand{\ub}[1]{\underbar{#1}}
\newcommand{\tl}[1]{\tilde{#1}}
\newcommand{\p}{\partial}
\newcommand{\Tn}[1]{T^{r_{#1}}_{s_{#1}}(V)}
\newcommand{\Tnp}{T^{r_1 + r_2}_{s_1 + s_2}(V)}
\newcommand{\Perm}{\text{Perm}}
\newcommand{\wh}[1]{\widehat{#1}}
\newcommand{\wt}[1]{\widetilde{#1}}
\newcommand{\defeq}{\vcentcolon=}
\newcommand{\Con}{\text{Con}}
\newcommand{\ConKos}{\text{Con}_{\text{Kos}}}
\newcommand{\trl}{\triangleleft}
\newcommand{\trr}{\triangleright}
\newcommand{\alg}{\text{alg}}
\newcommand{\Triv}{\text{Triv}}
\newcommand{\Der}{\text{Der}}
\newcommand{\cnj}{\text{conj}}

\newcommand{\lcm}{\text{lcm}}
\newcommand{\Imax}{\MI_{\text{max}}}


\DeclareMathOperator*{\Rl}{\text{Re}}
\DeclareMathOperator*{\Imn}{\text{Imn}}



% limits
\newcommand{\limfn}{\liminf \limits_{n \rightarrow \infty}}
\newcommand{\limpn}{\limsup \limits_{n \rightarrow \infty}}
\newcommand{\limn}{\lim \limits_{n \rightarrow \infty}}
\newcommand{\convt}[1]{\xrightarrow{\text{#1}}}
\newcommand{\conv}[1]{\xrightarrow{#1}} 
\newcommand{\seq}[2]{(#1_{#2})_{#2 \in \N}}

% intervals
\newcommand{\RG}{[0,\infty]}
\newcommand{\Rg}{[0,\infty)}
\newcommand{\Rgp}{(0,\infty)}
\newcommand{\Ru}{(\infty, \infty]}
\newcommand{\Rd}{[\infty, \infty)}
\newcommand{\ui}{[0,1]}

% integration \newcommand{\dm}{\, d m}
\newcommand{\dmu}{\, d \mu}
\newcommand{\dnu}{\, d \nu}
\newcommand{\dlam}{\, d \lambda}
\newcommand{\dP}{\, d P}
\newcommand{\dQ}{\, d Q}
\newcommand{\dm}{\, d m}
\newcommand{\dsh}{\, d \#}

% abreviations 
\newcommand{\lsc}{lower semicontinuous}

% misc
\newcommand{\as}[1]{\overset{#1}{\sim}}
\newcommand{\astx}[1]{\overset{\text{#1}}{\sim}}
\newcommand{\io}{\text{ i.o.}}
%\newcommand{\ev}{\text{ ev.}}
\newcommand{\Ll}{L^1_{\text{loc}}(\R^n)}

\newcommand{\loc}{\text{loc}}
\newcommand{\BV}{\text{BV}}
\newcommand{\NBV}{\text{NBV}}
\newcommand{\TV}{\text{TV}}

\newcommand{\op}[1]{\mathcal{#1}^{\text{op}}}


% Glossary - Notation
\glsxtrnewsymbol[description={finite measures on $(X, \MA)$}]{n000001}{$\MM_+(X, \MA)$}
\glsxtrnewsymbol[description={velocity}]{v}{\ensuremath{v}}


\makeindex

\begin{document}
	
	\frontmatter
	
	\title{Introduction to Analysis}
	
	%    Remove any unused author tags.
	
	%    author one information
	\author{Carson James}
	\thanks{}
	
	\date{}
	
	\maketitle
	
	%    Dedication.  If the dedication is longer than a line or two,
	%    remove the centering instructions and the line break.
	%\cleardoublepage
	%\thispagestyle{empty}
	%\vspace*{13.5pc}
	%\begin{center}
	%  Dedication text (use \\[2pt] for line break if necessary)
	%\end{center}
	%\cleardoublepage
	
	%    Change page number to 6 if a dedication is present.
	\setcounter{page}{4}
	
	\tableofcontents
	\printunsrtglossary[type=symbols,style=long,title={Notation}]
	
	%    Include unnumbered chapters (preface, acknowledgments, etc.) here.
	%\include{}
	
	\mainmatter
	%    Include main chapters here.
	%\include{}
	
	\chapter*{Preface}
	\addcontentsline{toc}{chapter}{Preface}

	\begin{flushleft}
		\href{https://creativecommons.org/licenses/by-nc-sa/4.0/legalcode.txt}{cc-by-nc-sa}
	\end{flushleft}

	\newpage
	
	
	
	
	
	
	
	
	
	
	
	
	
	\chapter{Set Theory}
	
	\tcr{figure out how to structure discussion on orderings, the issue is that projective systems of sets rely on directed sets, directed sets need products of sets, order theory and the specialization topology need topology and separation axioms, maybe each section should list prereqs and put sections with more prereqs towards end of chapter, then we could have a chapter on orderings that cover directed sets that comes after the set theory chapter and in the section on projective systems, include an intro with the prereq material from the section on directed sets, or maybe just make orderings a separate set of notes about order theory and reference these notes. Honestly chapters should be indexed by a poset instead of a totally ordered set.}
	
	\section{Relations}
	
	\section{Sets}
	
	
	\begin{defn} \ld{def:set_theory:sets:0001}
		Let $X$ be a set, $E \subset X$ and $\MU \subset \MP(X)$. Then $\MU$ is said to be a \tbf{cover of $E$ in $X$} if $E \subset \bigcup\limits_{U \in \MU} U$. 
	\end{defn}

	\begin{note}
		When the context is clear, we say $\MU$ is a cover of $E$.
	\end{note}
	
	\begin{ex} \lex{ex:set_theory:sets:0002}
		Let $X$ be a set, $A \subset X$ and $\MU, \MV \subset \MP(X)$. Suppose that $\MU$ and $\MV$ are covers of $A$ and $\MV$ is finite. If for each $\MU_0 \subset \MU$, $\MU_0$ is finite implies that $\MU_0$ is not a cover of $A$, then there exists $V \in \MV$ such that for each $\MU_0 \subset \MU$, $\MU_0$ is finite implies that $\MU_0$ is not a cover of $V$.
	\end{ex}
	
	\begin{proof}
		Suppose that for each $\MU_0 \subset \MU$, $\MU_0$ is finite implies that $\MU_0$ is not a cover of $A$. For the sake of contradiction, suppose that for each $V \in \MV$, there exists $\MU_{\MV} \subset \MU$ such that $\MU_{\MV}$ is finite and $\MU_{\MV}$ is a cover of $V$. Define $\MU_0 \subset \MU$ by $\MU_0 \defeq \bigcup\limits_{V \in \MV} \MU_{\MV}$. Since $\MV$ is finite and for each $V \in \MV$, $\MU_{\MV}$ is finite, we have that $\MU_0$ is finite. By assumption, $\MU_0$ is not a cover of $A$. Since $\MV$ is a cover of $A$, 
		\begin{align*}
			A
			& \subset \bigcup\limits_{V \in \MV} V \\
			& \subset \bigcup\limits_{V \in \MV} \bigg[ \bigcup\limits_{U \in \MU_{\MV}} U \bigg] \\
			& = \bigcup\limits_{U \in \MU_0} U.
		\end{align*} 
		and therefore $\MU_0$ is a cover of $A$, which is a contradiction. Thus there exists $V \in \MV$ such that for each $\MU_0 \subset \MU$, $\MU_0$ is finite implies that $\MU_0$ is not a cover of $V$.
	\end{proof}
	
	
	
	
	
	
	
	
	
	
	
	
	
	
	
	
	
	
	
	
	
	
	
	
	
	
	
	
	
	
	
	
	
	
	
	
	
	
	
	
	
	
	\section{Functions}
	
	
	\subsection{Introduction}
	
	\begin{ex} \lex{ex:set_theory:functions:0001}
		Let $X, Y$ be sets, $f:X \rightarrow Y$ and $A \subset X$. Then $A \subset f^{-1}(f(A))$.
	\end{ex}

	\begin{proof}
		Let $x \in A$. Then $f(x) \in f(A)$. Set $B \defeq f(A)$. Since $f(x) \in B$, we have that 
		\begin{align*}
			x 
			& \in f^{-1}(B) \\
			& = f^{-1}(f(A))
		\end{align*}
		Since $x \in A$ is arbitrary, we have that $A \subset f^{-1}(f(A))$. 
	\end{proof}
	
	\begin{ex} \lex{ex:set_theory:functions:0002}
		Let $X, Y$ be sets, $f:X \rightarrow Y$ and $B \subset Y$. Then $f(f^{-1}(B)) = B \cap f(X)$. 
	\end{ex}

	\begin{proof}
		Let $y \in f(f^{-1}(B))$. Then there exists $x \in f^{-1}(B)$ such that $f(x) = y$. Thus 
		\begin{align*}
			y
			& = f(x) \\
			& \in B
		\end{align*}
		Since 
		\begin{align*}
			f^{-1}(B) 
			& = \{x \in X: f(x) \in B\} \\
			& \subset X
		\end{align*}
		\begin{align*}
			y
			& = f(x) \\
			& \in f(X) \\
		\end{align*}
			Hence $y \in B \cap f(X)$. Since $y \in f(f^{-1}(B))$ is arbitrary, $f(f^{-1}(B)) \subset  B \cap f(X)$.\\
			Conversely, let $y \in B \cap f(X)$. Since $y \in f(X)$, there exists $x \in X$ such that $f(x) = y$. Since $y \in B$, $x \in f^{-1}(B)$. Hence 
			\begin{align*}
				y
				& = f(x) \\
				& \in f(f^{-1}(B)) \\
			\end{align*}
			Since $y \in B \cap f(X)$ is arbitrary, $B \cap f(X) \subset f(f^{-1}(B))$. Thus $f(f^{-1}(B)) =  B \cap f(X)$.
		\end{proof}
	
		\begin{ex} \lex{ex:set_theory:functions:0003}
			Let $X, Y$ be sets, $f:X \rightarrow Y$ and $\MA \subset \MP(X)$. Suppose that $\MA \neq \varnothing$. Then 
			$$f\bigg( \bigcup\limits_{A \in \MA} A\bigg) = \bigcup\limits_{A \in \MA} f(A)$$ 
		\end{ex}
	
		\begin{proof}\
			\begin{itemize}
				\item Let $y \in f\bigg( \bigcup\limits_{A \in \MA} A \bigg)$. Then there exists $x \in \bigcup\limits_{A \in \MA} A$ such that $y = f(x)$. Then there exists $A_0 \in \MA$ such that $x \in A_0$. Therefore 
				\begin{align*}
					y
					& = f(x) \\
					& \in f(A_0) \\
					& \subset \bigcup_{A \in \MA} f(A).
				\end{align*}
				Since $y \in f\bigg( \bigcup\limits_{A \in \MA} A \bigg)$ is arbitrary, we have that for each $y \in f\bigg( \bigcup\limits_{A \in \MA} A \bigg)$, $y \in \bigcup_{A \in \MA} f(A)$. Hence $f\bigg( \bigcup\limits_{A \in \MA} A\bigg) \subset \bigcup\limits_{A \in \MA} f(A)$. 
				\item Let $y \in \bigcup\limits_{A \in \MA} f(A)$. Then there exists $A_0 \in \MA$ such that $y \in f(A_0)$. Therefore, there exists $x \in A_0$ such that $y = f(x)$. Since 
				\begin{align*}
					x
					& \in A_0 \\
					& \subset \bigcup\limits_{A \in A} A ,
				\end{align*}
				we have that 
				\begin{align*}
					y
					& = f(x) \\
					& \in f\bigg( \bigcup\limits_{A \in \MA} A \bigg).
				\end{align*}
				Since $y \in \bigcup\limits_{A \in \MA} f(A)$ is arbitrary, we have that for each $y \in \bigcup\limits_{A \in \MA} f(A)$, $y \in f\bigg( \bigcup\limits_{A \in \MA} A \bigg)$. Hence $f\bigg( \bigcup\limits_{A \in \MA} A\bigg) \subset f\bigg( \bigcup\limits_{A \in \MA} A \bigg)$.  
			\end{itemize}
		\end{proof}
	
		\begin{ex} \lex{ex:set_theory:functions:0003.1}
			Let $X, Y$ be sets, $f:X \rightarrow Y$ and $\MA \subset \MP(X)$. Then 
			$$f\bigg( \bigcap_{A \in \MA} A\bigg) \subset \bigcap_{A \in \MA} f(A)$$ 
		\end{ex}
	
		\begin{proof}
			Let $y \in f\bigg( \bigcap\limits_{A \in \MA} A \bigg)$. Then there exists $x \in \bigcap\limits_{A \in \MA} A$ such that $y = f(x)$. Thus, for each $A \in \MA$, $x \in A$. Hence for each $A \in \MA$, 
			\begin{align*}
				y
				& = f(x) \\
				& \in f(A).
			\end{align*}
			Therefore $y \in \bigcap\limits_{A \in \MA} f(A)$. Since $y \in f\bigg( \bigcap\limits_{A \in \MA} A \bigg)$ is arbitrary, we have that for each $y \in f\bigg( \bigcap\limits_{A \in \MA} A \bigg)$, $y \in \bigcap\limits_{A \in \MA} f(A)$. Thus $f\bigg( \bigcap\limits_{A \in \MA} A\bigg) \subset \bigcap\limits_{A \in \MA} f(A)$.
		\end{proof}
	
		\begin{ex} \lex{ex:set_theory:functions:0004}
			Let $X, Y$ be sets, $f:X \rightarrow Y$, $A \subset X$ and $B \subset Y$. Then 
			$$f(A \cap f^{-1}(B)) = f(A) \cap B$$ 
		\end{ex}
		
		\begin{proof}
			Let $y \in f(A \cap f^{-1}(B))$. Then there exists $a \in A \cap f^{-1}(B)$ such that $f(a) = y$. Then $a \in A$ and $f(a) \in B$. Hence 
			\begin{align*}
				y
				& = f(a) \\
				& \in f(A) \cap B.
			\end{align*}
			Since $y \in f(A \cap f^{-1}(B))$ is arbitrary, we have that $f(A \cap f^{-1}(B)) \subset f(A) \cap B$. \\
			Conversely, let $y \in f(A) \cap B$. Then $y \in B$ and there exists $a \in A$ such that $f(a) = y$. Since $f(a) \in B$, we have that $a \in f^{-1}(B)$. Therefore 
			\begin{align*}
				y
				& = f(a) \\
				& \in f(A \cap f^{-1}(B)).
			\end{align*} 
			Since $y \in f(A \cap f^{-1}(B))$ is arbitrary, we have that $f(A \cap f^{-1}(B)) \subset f(A \cap f^{-1}(B))$. \\
			Hence $f(A \cap f^{-1}(B)) = f(A \cap f^{-1}(B))$. 
		\end{proof}
	
	
	\begin{ex} \lex{ex:set_theory:functions:0005}
		Let $X$ be a set, $K \subset X$, $(X_\al)_{\al \in A}$ a collection of sets and for each $\al \in A$, $f_{\al}: X \rightarrow X_{\al}$. For $\al \in A$, set $K_{\al} \defeq f_{\al}(K)$. Then $K = \bigcap\limits_{\al \in A} f_{\al}^{-1}(K_{\al})$.
	\end{ex}

	\begin{proof}\
		\begin{itemize}
			\item Let $x \in K$ and $\al \in A$. By definition,  
			\begin{align*}
				f_{\al}(x) 
				& \in f_{\al}(K) \\
				& = K_{\al}
			\end{align*}
			Thus $x \in f_{\al}^{-1}(K_{\al})$. Since $\al \in A$ is arbitrary, we have that for each $\al \in A$, $x \in f_{\al}^{-1}(K_{\al})$. Thus $x \in \bigcap\limits_{\al \in A} f_{\al}^{-1}(K_{\al})$. Since $x \in K$ is arbitrary, $K \subset \bigcap\limits_{\al \in A} f_{\al}^{-1}(K_{\al})$.
			\item Let $\al \in A$. \rex{ex:set_theory:functions:0001} implies that 
			\begin{align*}
				K
				& \subset f_{\al}^{-1}(f_{\al}(K)) \\
				& = f_{\al}^{-1}(K_{\al}).
			\end{align*}
			Since $\al \in A$ is arbitrary, for each $\al \in A$, $K \subset f_{\al}^{-1}(K_{\al})$. Thus $K \subset \bigcap\limits_{\al \in A} f_{\al}^{-1}(K_{\al})$.
		\end{itemize}
		Therefore $K = \bigcap\limits_{\al \in A} f_{\al}^{-1}(K_{\al})$.
	\end{proof}
		
		
	
	
	
	
	
	
	
	
	
	
	
	
	
	
	
	
	\subsection{Bijections}
	
	\begin{defn} \ld{ld:set_theory:bijections:0001}
		Let $X, Y$ be sets and $f:X \rightarrow Y$. Then $f$ is said to be a \tbf{surjection} if for each $y \in Y$, there exists $x \in X$ such that $f(x) = y$.
	\end{defn}

	\begin{ex} \lex{ex:set_theory:bijections:0002}
		Let $X, Y$ be sets and $f:X \rightarrow Y$. Suppose that $f$ is a surjection. Then for each $B \subset Y$, $f(f^{-1}(B)) = B$. 
	\end{ex}

	\begin{proof}
		Let $B \subset Y$. Since $f$ is surjective, $f(X) = Y$. \tcr{A previous exercise} implies that 
		\begin{align*}
			f(f^{-1}(B))
			& = B \cap f(X) \\
			& = B \cap Y \\
			& = B
		\end{align*}
	\end{proof}

	\begin{ex} \lex{ex:set_theory:bijections:0003}
		Let $X, Y, Z$ be sets, $f:X \rightarrow Y$ and $g,h:Y \rightarrow Z$. Suppose that $f$ is a surjection. If $g \circ f = h \circ f$, then $g = h$.
	\end{ex}

	\begin{proof}
		Suppose that $g \circ f = h \circ f$. Let $y \in Y$. Since $f$ is a surjection, there exists $x \in X$ such that $y = f(x)$. Then 
		\begin{align*}
			g(y)
			& = g \circ f(x) \\
			& = h \circ f(x) \\
			& = h(y).
		\end{align*}
		Since $y \in Y$ is arbitrary, we have that for each $y \in Y$, $g(y) = h(y)$. Hence $g = h$.
	\end{proof}


	\begin{defn} \ld{def:set_theory:bijections:0004}
		Let $X, Y$ be sets and $f:X \rightarrow Y$. Then $f$ is said to be a \tbf{injection} if for each $a, b \in X$, $f(a) = f(b)$ implies that $a = b$.
	\end{defn}

	\begin{ex} \lex{ex:set_theory:bijections:0004.1}
		Let $X, Y$ be sets and $f:X \rightarrow Y$. Suppose that $f$ is a injection. Then for each $A \subset X$, $f^{-1}(f(A)) = A$. 
	\end{ex}

	\begin{proof}
		\begin{itemize}
			\item Let $x \in f^{-1}(f(A))$. Then $f(x) \in f(A)$. Thus there exists $a \in A$ such that $f(x) = f(a)$. Since $f$ is injective, $x = a$. Thus $x \in A$. Since $x \in f^{-1}(f(A))$ is arbitrary, we have that for each $x \in f^{-1}(f(A))$, $x \in A$. Hence $f^{-1}(f(A)) \subset A$.
			\item \rex{ex:set_theory:functions:0001} implies that $A \subset f^{-1}(f(A))$.
		\end{itemize}
		Since $f^{-1}(f(A)) \subset A$ and $A \subset f^{-1}(f(A))$, we have that $f^{-1}(f(A)) = A$. 
	\end{proof}

	\begin{ex} \lex{ex:set_theory:bijections:0005}
		Let $X, Y$ be sets, $f:X \rightarrow Y$ and $\MA \subset \MP(X)$. Suppose that $\MA \neq \varnothing$ and $f$ is injective. Then 
		$$f\bigg( \bigcap\limits_{A \in \MA} A\bigg) = \bigcap\limits_{A \in \MA} f(A)$$ 
	\end{ex}

	\begin{proof}
		\begin{itemize}
			\item \rex{ex:set_theory:functions:0003.1} implies that $f \bigg( \bigcap\limits_{A \in \MA} A\bigg) \subset \bigcap\limits_{A \in \MA} f(A)$. 
			\item Let $y \in \bigcap\limits_{A \in \MA} f(A)$. Then for each $A \in \MA$, $y \in f(A)$. Thus, for each $A \in \MA$, there exists $x_A \in A$ such that $y = f(x_A)$. Since $\MA \neq \varnothing$, there exists $A_0 \in \MA$. Let $A \in \MA$. Then 
			\begin{align*}
				f(x_A)
				& = y \\
				& = f(x_{A_0}).
			\end{align*}
			Since $f$ is injective, $x_A = x_{A_0}$. Since $A \in \MA$ is arbitrary, we have that for each $A \in \MA$, $x_A = x_{A_0}$. Thus for each $A \in \MA$, $x_{A_0} \in A$. Hence $x_{A_0} \in \bigcap\limits_{A \in \MA} A$ and therefore 
			\begin{align*}
				y
				& = f(x_{A_0}) \\
				& \in f \bigg( \bigcap\limits_{A \in \MA} A \bigg).
			\end{align*}
			Since $y \in  \bigcap\limits_{A \in \MA} f(A)$ is arbitrary, we have that for each $y \in  \bigcap\limits_{A \in \MA} f(A)$, $y \in f \bigg( \bigcap\limits_{A \in \MA} A \bigg)$. Thus $\bigcap\limits_{A \in \MA} f(A) \subset f \bigg( \bigcap\limits_{A \in \MA} A \bigg)$. 
		\end{itemize} 
		Since $f \bigg( \bigcap\limits_{A \in \MA} A\bigg) \subset \bigcap\limits_{A \in \MA} f(A)$ and $\bigcap\limits_{A \in \MA} f(A) \subset f \bigg( \bigcap\limits_{A \in \MA} A \bigg)$, we have that $f\bigg( \bigcap\limits_{A \in \MA} A\bigg) = \bigcap\limits_{A \in \MA} f(A)$.
	\end{proof}

	\subsection{Nets}
	
	\begin{defn} \ld{def:set_theory:nets:0001}
		Let $X$ be a topological space, $A$ a directed set and $x:A \rightarrow Y$. Then $x$ is said to be a \tbf{net} in $X$. We typically write $(x_{\al})_{\al \in A}$. 
	\end{defn}
	
	\begin{defn} \ld{def:set_theory:nets:0008.1}
		Let $X$ be a set, $A$ a directed set and $\al_0 \in A$. We define the \tbf{$\al_0$-tail operator}, denoted $L_{\al_0}: X^A \rightarrow X^{[\al_0, \infty)}$, by $L(x) \defeq x|_{[\al_0, \infty)}$, i.e. $L_{\al_0}((x_{\al})_{\al \in A}) \defeq (x_{\al})_{\al \in [\al_0, \infty)}$.
	\end{defn}
	
	\begin{defn} \ld{def:set_theory:nets:0003}
		Let $X$ be a topological space, $(x_{\al})_{\al \in A}$, $(y_{\be})_{\be \in B} \subset X$ nets and $\phi:B \rightarrow A$. 
		Then $((y_{\be})_{\be \in B}, \phi)$ is said to be a \tbf{subnet of $(x_{\al})_{\al \in A}$} if 
		\begin{enumerate}
			\item for each $\beta \in B$, $y_{\beta} = x_{\phi(\beta)}$
			\item for each $\al_0 \in A$, there exists $\be_0 \in B$ such that for each $\be \in B$, $\be \geq \be_0$ implies that $\phi(\be) \geq \al_0$
		\end{enumerate}
	\end{defn}
	
	\begin{note}
		We usually supress $\phi$ and write $\al_{\be}$ in place of ${\phi(\beta)}$.
	\end{note}
	
	
	
	
	
	
	
	
	
	
	
	
	
	
	\subsection{Sequences}











































	\newpage
	
	\section{Products of Sets}
	
	\begin{defn} \ld{def:set_theory:products:0001}
	Let $(X_{\al})_{\al \in A}$ be a collection of sets. We define 
	\begin{itemize}
		\item the \tbf{Cartesian product} of $(X_{\al})_{\al \in A}$, denoted $\prod\limits_{\al \in A} X_{\al}$, by 
		$$\prod_{\al \in A}X_{\al} \defeq \{f:A \rightarrow \bigcup_{\al \in A} X_{\al}: \text{ for each $\al \in A$, $f(\al) \in X_{\al}$}\}$$
		\item the \tbf{$\al$-th projection map of $\prod_{\al \in A} X_{\al}$ onto $X_{\al}$}, denoted $\pi_{\al}:\prod\limits_{\al \in A}X_{\al} \rightarrow X_{\al}$, by 
		$$\pi_{\al}(f) \defeq f(\al)$$
	\end{itemize}
	\end{defn}

	\begin{ex} \lex{ex:set_theory:products:0003}
		Let $(A_{\lam})_{\lam \in \Lam}$ be a collection of sets and $B$ a set. Then 
		$$\bigg( \bigcup\limits_{\lam \in \Lam} A_{\lam} \bigg) \times B = \bigcup\limits_{\lam \in \Lam} (A_{\lam} \times B)$$
	\end{ex}

	\begin{proof}
		Let $(x,y) \in \bigg( \bigcup\limits_{\lam \in \Lam} A_{\lam} \bigg) \times B$. Then $x \in \bigcup\limits_{\lam \in \Lam} A_{\lam}$ and $y \in B$. Therefore, there exists $\lam \in \Lam$ such that $x \in A_{\lam}$. Hence 
		\begin{align*}
			(x,y) 
			& \in A_{\lam} \times B \\
			& \subset \bigcup\limits_{\lam \in \Lam} (A_{\lam} \times B)
		\end{align*}
		Thus $\bigg( \bigcup\limits_{\lam \in \Lam} A_{\lam} \bigg) \times B \subset  \bigcup\limits_{\lam \in \Lam} (A_{\lam} \times B)$. \\
		Conversely, let $(x,y) \in \bigcup\limits_{\lam \in \Lam} (A_{\lam} \times B)$. Then there exists $\lam \in \Lam$ such that $(x,y) \in A_{\lam} \times B$. Then 
		\begin{align*}
			x 
			& \in A_{\lam} \\
			& \subset \bigcup_{\lam \in \Lam} A_{\lam}
		\end{align*}
		and $y \in B$. Hence $(x,y) \in \bigg( \bigcup\limits_{\lam \in \Lam} A_{\lam} \bigg) \times B$. So $ \bigcup\limits_{\lam \in \Lam} (A_{\lam} \times B) \subset \bigg( \bigcup\limits_{\lam \in \Lam} A_{\lam} \bigg) \times B$.
	\end{proof}

	\begin{defn} \ld{def:set_theory:products:0004}  
		Let $(X_{\al})_{\al \in A}$ and $(Y_{\al})_{\al \in A}$ be collections of sets and $(f_{\al})_{\al \in A} \in \prod\limits_{\al \in A} Y_{\al}^{X_{\al}}$, i.e. for each $\al \in A$, $f_{\al}:X_{\al} \rightarrow Y_{\al}$. We define the \tbf{product of $(f_{\al})_{\al \in A}$}, denoted $\prod_{\al \in A} f_{\al}: \prod\limits_{\al \in A} X_{\al} \rightarrow  \prod\limits_{\al \in A} Y_{\al}$ by 
		$$\bigg( \bigg[\prod_{\al \in A} f_{\al} \bigg] (x) \bigg)_{\be} = f_{\be}(x_{\be})$$
	\end{defn}
	
	\begin{ex}  \lex{ex:set_theory:products:0005}
		Let $(X_{\al})_{\al \in A}$ and $(Y_{\al})_{\al \in A}$ be collections of sets and $(f_{\al})_{\al \in A} \in \prod\limits_{\al \in A} Y_{\al}^{X_{\al}}$, i.e. for each $\al \in A$, $f_{\al}:X_{\al} \rightarrow Y_{\al}$. Denote the $\al$-th projection maps on $\prod\limits_{\al \in A} X_{\al}$ and $\prod\limits_{\al \in A} Y_{\al}$ by $\pi^X_{\al}$ and $\pi^Y_{\al}$ respectively. Then for each $\al \in A$, $\pi^Y_{\al} \circ \bigg[ \prod\limits_{\al \in A} f_{\al} \bigg]  = f_{\al} \circ \pi^X_{\al}$, i.e. the following diagram commutes:
		\[ 
		\begin{tikzcd}
			\prod\limits_{\al \in A} X_{\al}  \arrow[r, "\prod\limits_{\al \in A} f_{\al}"]  \arrow[d, "\pi^X_{\al}"'] & \prod\limits_{\al \in A} Y_{\al} \arrow[d, "\pi^Y_{\al}"]\\
			X_{\al} \arrow[r, "f_{\al}"'] &  Y_{\al} \\
		\end{tikzcd}
		\]
	\end{ex}
	
	\begin{proof}
		Set $X \defeq \prod\limits_{\al \in A} X_{\al}$, $Y \defeq \prod\limits_{\al \in A}Y_{\al}$ and define $f: X \rightarrow Y$ by $f \defeq \prod\limits_{\al \in A} f_{\al}$. Let $\al \in A$ and $x \in X$. Then
		\begin{align*}
			\pi^Y_{\al} \circ f(x) 
			& = (f(x))_{\al} \\
			& = f_{\al}(x_{\al}) \\
			& = f_{\al} \circ \pi^X_{\al}(x) 
		\end{align*}
		Since $\al \in A$ and $x \in X$ are arbitrary, for each $\al \in A$, $\pi^Y_{\al} \circ f = f_{\al} \circ \pi^X_{\al}$.
	\end{proof}

	\begin{ex}  \lex{ex:set_theory:products:0005.1}
		Let $(X_{\al})_{\al \in A}$ and $(Y_{\al})_{\al \in A}$ be collections of sets and $(f_{\al})_{\al \in A} \in \prod\limits_{\al \in A} Y_{\al}^{X_{\al}}$, i.e. for each $\al \in A$, $f_{\al}:X_{\al} \rightarrow Y_{\al}$. Suppose that for each $\al \in A$, $f_{\al}$ is a bijection. Then $\prod\limits_{\al \in A} f_{\al}$ is a bijection and $\bigg[ \prod\limits_{\al \in A} f_{\al} \bigg]^{-1} = \prod\limits_{\al \in A} f_{\al}^{-1}$.
	\end{ex}
	
	\begin{proof}
		Set $X \defeq \prod\limits_{\al \in A} X_{\al}$, $Y \defeq \prod\limits_{\al \in A}Y_{\al}$ and define $f: X \rightarrow Y$ and $g: Y \rightarrow X$ by $f \defeq \prod\limits_{\al \in A} f_{\al}$ and $g \defeq \prod\limits_{\al \in A} f_{\al}^{-1}$. Denote the $\al$-th projection maps on $\prod\limits_{\al \in A} X_{\al}$ and $\prod\limits_{\al \in A} Y_{\al}$ by $\pi^X_{\al}$ and $\pi^Y_{\al}$ respectively. Let $\al \in A$. Then
		\begin{align*}
			\pi_{\al}^X \circ g \circ f
			& = g_{\al} \circ \pi_{\al}^Y \circ f \\
			& = g_{\al} \circ f_{\al} \circ \pi_{\al}^X \\
			& = f_{\al}^{-1} \circ f_{\al} \circ \pi_{\al}^X \\
			& = \pi_{\al}^X
		\end{align*}
		and 
		\begin{align*}
			\pi_{\al}^Y \circ f \circ g
			& = f_{\al} \circ \pi_{\al}^X \circ g \\
			& = f_{\al} \circ g_{\al} \circ \pi_{\al}^Y \\
			& = f_{\al} \circ f_{\al}^{-1} \circ \pi_{\al}^Y  \\
			& = \pi_{\al}^Y 
		\end{align*}
		Since $\al \in A$ is arbitrary, we have that for each $\al \in A$, $\pi_{\al}^X \circ g \circ f = \pi_{\al}^X$ and $\pi_{\al}^Y \circ f \circ g = \pi_{\al}^Y$. Hence $g \circ f = \id_X$ and $f \circ g = \id_Y$. Thus $f$ is a bijection and $f^{-1} = g$. 
	\end{proof}

	\begin{defn} \ld{def:set_theory:products:00010}
		Let $X$ be a a topological space. We define 
		\begin{itemize}
			\item the \tbf{diagonal of $X \times X$}, denoted $\Del_X \subset X \times X$, by $\Del_X \defeq \{(x,x):x \in X\}$,
			\item the \tbf{diagonal of $X^{\N}$}, denoted $\Del_{X^{\N}} \subset X^{\N}$, by $\Del_{X^{\N}} \defeq \{x \in X^{\N}: \text{ for each $m,n \in \N$, $\pi_m(x) = \pi_n(x)$}\}$.
		\end{itemize}
	\end{defn}

	\begin{ex} \lex{ex:set_theory:products:0005.2}
		Let $X$ be a set, $(B_n)_{n \in \N} \subset \MP(X)$, $(Z_n)_{n \in \N}$ a collection of sets and $(f_n) \in \prod\limits_{n \in \N} B_n^{Z_n}$. Suppose that for each $n \in \N$, $f_n$ is a bijection. Set $Z_0 \defeq \prod\limits_{n \in \N} Z_n$. Define $f_0: Z_0 \rightarrow X^{\N}$ and $Z \subset Z_0$ by $f_0 \defeq \prod\limits_{n \in \N} f_n$ and $Z \defeq f_0^{-1}(\Del_{X^{\N}})$. Then 
		\begin{enumerate}
			\item for each $m,n \in \N$, $f_m \circ \pi_m|_Z = f_n \circ \pi_n|_Z$,
			\item for each $n \in \N$, $f_n \circ \pi_n(Z) = \bigcap\limits_{n \in \N} B_n$.
		\end{enumerate}
	\end{ex}
	
	\begin{proof}\
		\begin{enumerate}
			\item Let $m,n \in \N$ and $z \in Z$. By construction, $f_0(z) \in \Del_{X^{\N}}$. Therefore
			\begin{align*}
				f_n (\pi_n(z)) 
				& = \pi_n (f_0(z)) \\
				& = \pi_m (f_0(z)) \\
				& = f_m (\pi_m(z)).
			\end{align*}
			Since $z \in Z$ is arbitrary, we have that $f_m \circ \pi_m|_Z = f_n \circ \pi_n|_Z$. Since $m,n\in \N$ are arbitrary, we have that for each $m,n \in \N$, $f_m \circ \pi_m|_Z = f_n \circ \pi_n|_Z$.
			\item Set $B \defeq \bigcap\limits_{n \in \N} B_n$. Let $n \in \N$. 
			\begin{itemize}
				\item Let $z \in Z$. The previous part implies that   
				\begin{align*}
					f_n (\pi_n(z)) 
					& = f_m (\pi_m(z)) \\
					& \in B_m.
				\end{align*}
				Since $m \in \N$ is arbitrary, we have that for each $m \in \N$, $f_n (\pi_n(z)) \in B_m$. Hence 
				\begin{align*}
					f_n (\pi_n(z)) 
					& \in \bigcap_{m \in \N} B_m \\
					& = B.
				\end{align*} 
				Since $z \in Z$ is arbitrary, we have that $f_n \circ \pi_n(Z) \subset B$.
				\item Let $x \in B$. Then for each $m \in \N$, $x \in B_m$. Define $z \in Z_0$ by $z_m \defeq f_m^{-1}(x)$. Then for each $m \in \N$, 
				\begin{align*}
					\pi_m \circ f_0(z) 
					& = f_m \circ \pi_m (z) \\
					& = f_m(z_m) \\
					& = f_m(f_m^{-1}(x)) \\
					& = x 
				\end{align*}
				In particular, for each $m \in \N$, 
				\begin{align*}
					\pi_m \circ f_0(z) 
					& = x \\
					& = \pi_n \circ f_0(z). 
				\end{align*}
				Thus $f_0(z) \in \Del_{X^{\N}}$ and $z \in Z$. Therefore 
				\begin{align*}
					x 
					& = \pi_n \circ f_0(z) \\
					& = f_n \circ \pi_n(z) \\
					& \in f_n \circ \pi_n (Z).  
				\end{align*}  
				Since $x \in B$ is arbitrary, we have that $B \subset f_n \circ \pi_n (Z)$.  
			\end{itemize}
		\end{enumerate}
	\end{proof}
	

	\begin{defn} \ld{def:set_theory:products:0005.3}
		Let $X$ be a set, $(Y_{\al})_{\al \in A}$ a collection of sets and $(f_{\al})_{\al \in A} \in \prod\limits_{\al \in A} Y_{\al}^X$, i.e. for each $\al \in A$, $f_{\al}:X \rightarrow Y_{\al}$. Set $Y = \prod\limits_{\al \in A}Y_{\al}$. We define the \tbf{tuple of $(f_{\al})_{\al \in A}$}, denoted $(f_{\al})_{\al \in A}: X \rightarrow Y$, by $(f_{\al})_{\al \in A}(x) \defeq (f_{\al}(x))_{\al \in A}$.
	\end{defn}

	\begin{ex} \lex{ex:set_theory:products:0005.4}
		Let $X$ be a set, $(Y_{\al})_{\al \in A}$ a collection of sets, $(f_{\al})_{\al \in A} \in \prod\limits_{\al \in A} Y_{\al}^X$ and $E \subset X$. Then $(f_{\al})_{\al \in A}|_E = (f_{\al}|_E)_{\al \in A}$.
	\end{ex}

	\begin{proof}
		Clear. \tcr{add details}.
	\end{proof}

	\begin{defn} \ld{def:set_theory:products:0006}  \tcr{Define slice maps here in terms of inclusion maps}
		Let $X, Y$ be sets and $U \subset X \times Y$. For each $(x_0, y_0) \in U$, we define $U_{x_0} = \{y \in Y: (x_0,y) \in U\}$ and $U^{y_0} = \{x \in X: (x,y_0) \in U\}$.
	\end{defn}

	\begin{defn} \ld{def:set_theory:products:0007}  
		Let $X, Y$ and $Z$ be sets, $U \subset X \times Y$ and $f: U \rightarrow Z$. For each $(x_0, y_0) \in U$, we define $f_{x_0}: U_{x_0} \rightarrow Z$ and $f^{y_0}: U^{y_0} \rightarrow Z$ by $f_{x_0} = f(x_0, \cdot)$ and $f^{y_0} = f(\cdot, y_0)$.
	\end{defn}
	
	\begin{ex} \lex{ex:set_theory:products:0008}
		Let $X, Y$ and $Z$ be sets, $U \subset X \times Y$, $f: U \rightarrow Z$ and $(x_0, y_0) \in U$. Then for each $V \subset Z$, $(f_{x_0})^{-1}(V) = (f^{-1}(V))_{x_0}$ and $(f^{y_0})^{-1}(V) = (f^{-1}(V))^{y_0}$.
	\end{ex}

	\begin{proof}
		Let $V \subset Z$. Then for each $x \in U^{y_0}$,
		\begin{align*}
			x \in (f^{y_0})^{-1}(V) 
			& \iff f^{y_0}(x) \in V \\
			& \iff f(x, y_0) \in V \\
			& \iff (x, y_0) \in f^{-1}(V) \\
			& \iff x \in (f^{-1}(V))^{y_0}
		\end{align*}
		So $(f^{y_0})^{-1}(V) = (f^{-1}(V))^{y_0}$. Similarly, $(f_{x_0})^{-1}(V) = (f^{-1}(V))_{x_0}$. 
	\end{proof}
	
	\begin{defn} \ld{def:set_theory:products:0009}
		Let $X, Y, Z$ be sets. We define the \tbf{currying operator}, denoted $\cur: Z^{X \times Y} \rightarrow (Z^Y)^X$, by 
		$\cur(f)(x)(y) = f(x, y)$. 
	\end{defn}
	
	
	
	
	
	
	
	
	
	
	
	
	
	
	
	
	
	
	
	
	
	
	
	
	
	
	
	
	
	
	
	
	
	
	
	
	
	
	
	
	
	
	
	
	
	
	
	\newpage
	\section{Coproducts of Sets}
	
	\begin{defn} \ld{def:set_theory:coproducts:0001}
		Let $(X_{\al})_{\al \in A}$ be a collection of sets. We define the \tbf{disjoint union} of $(X_{\al})_{\al \in A}$, denoted $\coprod\limits_{\al \in A}X_{\al}$, by 
		$$\coprod_{\al \in A}X_{\al} = \{(\al, x): x \in X_{\al} \}$$
	\end{defn}
	
	\begin{defn} \ld{def:set_theory:coproducts:0002}
		Let $(X_{\al})_{\al \in A}$ be a collection of sets. For $\al \in A$, we define the \tbf{$\al$-th embedding map if $X_{\al}$ into $\coprod_{\al \in A} X_{\al}$}, denoted $\iota_{\al}: X_{\al} \rightarrow \coprod\limits_{\al \in A}X_{\al}$, by 
		$$\iota_{\al}(x) = (\al, x)$$
	\end{defn}

	\begin{defn} \ld{ex:set_theory:coproducts:0002.1}
		Let $(X_{\al}, \MT_{\al})_{\al \in A}$ and $(Y_{\al}, \MS_{\al})_{\al \in A}$ be collections of topological spaces and $(f_{\al})_{\al \in A} \in \prod\limits_{\al \in A} Y_{\al}^{X_{\al}}$, i.e. for each $\al \in A$, $f_{\al}:X_{\al} \rightarrow Y_{\al}$. Set $X = \coprod\limits_{\al \in A} X_{\al}$ and $Y = \coprod\limits_{\al \in A}Y_{\al}$. We define the \tbf{coproduct of $(f_{\al})_{\al \in A}$}, denoted $\coprod\limits_{\al \in A} f_{\al}: X \rightarrow Y$, by $\coprod\limits_{\al \in A} f_{\al} (\be, x) = (\be, f_{\be}(x))$.
	\end{defn}

	\begin{ex} \lex{ex:set_theory:coproducts:0003}
		Let $(X_{\al})_{\al \in A}$ and $(Y_{\al})_{\al \in A}$ be collections of sets and $(f_{\al})_{\al \in A} \in \coprod\limits_{\al \in A} Y_{\al}^{X_{\al}}$, i.e. for each $\al \in A$, $f_{\al}:X_{\al} \rightarrow Y_{\al}$. Set $f \defeq \coprod\limits_{\al \in A} f_{\al}$. Then for each $\al \in A$, $f \circ \iota^X_{\al} = \iota^Y_{\al} \circ f_{\al}$, i.e. the following diagram commutes:
		\[ 
		\begin{tikzcd}
			\coprod\limits_{\al \in A} X_{\al}  \arrow[r, "f"]  & \coprod\limits_{\al \in A} Y_{\al} \\
			X_{\al} \arrow[u, "\iota^X_{\al}"] \arrow[r, "f_{\al}"'] &  Y_{\al} \arrow[u, "\iota^Y_{\al}"'] \\
		\end{tikzcd}
		\]
	\end{ex}

	\begin{proof}
		Let $\al \in A$ and $x \in X_{\al}$. Then
		\begin{align*}
			f \circ \iota^X_{\al}(x)
			& = f (\al, x) \\
			& = (\al, f_{\al}(x)) \\
			& = \iota^Y_{\al} \circ f_{\al}(x). 
		\end{align*}
		Since $x \in X$ are arbitrary, $f \circ \iota^X_{\al} = \iota^Y_{\al} \circ f_{\al}$. Since $\al \in A$ is arbitrary, we have that for each $\al \in A$, $f \circ \iota^X_{\al} = \iota^Y_{\al} \circ f_{\al}$.
	\end{proof}
	
	\begin{ex}  \lex{ex:set_theory:coproducts:0004}
		Let $(X_{\al})_{\al \in A}$ and $(Y_{\al})_{\al \in A}$ be collections of sets and $(f_{\al})_{\al \in A} \in \prod\limits_{\al \in A} Y_{\al}^{X_{\al}}$, i.e. for each $\al \in A$, $f_{\al}:X_{\al} \rightarrow Y_{\al}$. Suppose that for each $\al \in A$, $f_{\al}$ is a bijection. Then $\coprod\limits_{\al \in A} f_{\al}$ is a bijection and $\bigg[ \coprod\limits_{\al \in A} f_{\al} \bigg]^{-1} = \coprod\limits_{\al \in A} f_{\al}^{-1}$.
	\end{ex}
	
	\begin{proof}
		Set $X \defeq \coprod\limits_{\al \in A} X_{\al}$, $Y \defeq \coprod\limits_{\al \in A}Y_{\al}$ and define $f: X \rightarrow Y$ and $g: Y \rightarrow X$ by $f \defeq \coprod\limits_{\al \in A} f_{\al}$ and $g \defeq \coprod\limits_{\al \in A} f_{\al}^{-1}$. Denote the $\al$-th embedding maps into $X$ and $Y$ by $\iota^X_{\al}$ and $\iota^Y_{\al}$ respectively. Let $\al \in A$. Then
		\begin{align*}
			g \circ f \circ \iota_{\al}^X
			& = g \circ \iota_{\al}^Y \circ f_{\al} \\
			& = \iota_{\al}^X \circ g_{\al} \circ f_{\al} \\
			& = \iota_{\al}^X \circ f_{\al}^{-1} \circ f_{\al} \\
			& = \iota_{\al}^X 
		\end{align*}
		and 
		\begin{align*}
			f \circ g \circ \iota_{\al}^Y
			& = f \circ \iota_{\al}^X \circ g_{\al} \\
			& = \iota_{\al}^Y \circ f_{\al} \circ g_{\al} \\
			& = \iota_{\al}^Y \circ f_{\al} \circ f_{\al}^{-1} \\
			& = \iota_{\al}^Y 
		\end{align*}
		Since $\al \in A$ is arbitrary, we have that for each $\al \in A$, $g \circ f \circ \iota_{\al}^X = \iota_{\al}^X$ and $f \circ g \circ \iota_{\al}^Y = \iota_{\al}^Y$. Hence $g \circ f = \id_X$ and $f \circ g = \id_Y$. Thus $f$ is a bijection and $f^{-1} = g$. 
	\end{proof}
	
	
	
	
	
	
	
	
	
	
	
	
	
	
	
	
	
	
	
	
	
	
	
	
	
	
	
	
	
	
	
	\newpage
	\section{Quotients of Sets}
	
	\begin{defn}
	Let $X$ be a set and $\sim$ an equivalence relation on $X$. We define the \tbf{quotient set} of $X$ by $\sim$, denoted $X/ {\sim}$, by 
	\begin{equation*}
	X/ {\sim} = \{\bar{x}: x \in X\}
	\end{equation*}
	\end{defn}















	
	
	
	
	
	
	
	
	
	
	
	
	
	\newpage
	\section{Equalizers of Maps}
	
	\begin{defn} \ld{def:set_theory:equalizers:0001}
		Let $X, Y$ be sets and $f,g:X \rightarrow Y$. We define the \tbf{equalizer of $f$ and $g$}, denoted $\Eq(f,g)$, by 
		$$\Eq(f,g) \defeq \{x \in X: f(x) = g(x)\}$$
	\end{defn}
	
	\begin{ex}
		
	\end{ex}	
	
	
	
	
	
	
	
	
	
	
	
	
	
	
	
	
	
	
	
	
	
	
	
	
	
	
	
	
	\newpage
	\section{Projective Limits of Sets}
	
	\begin{note}
		Let $(X_j)_{j \in J}$ be a collection of sets. We denote the $j$-th projection map from $\prod\limits_{j \in J} X_j$ onto $X_j$ by $\prj_j: \prod\limits_{j \in J} X_j \rightarrow X_j$.
	\end{note}
	
	\begin{defn} \ld{def:set_theory:proj_limits:0001}
		Let $(J, {\leq})$ be a directed poset, $(X_j)_{j \in J}$ a collection of sets and for each $(j,k) \in \leq$, $\pi_{j,k}:X_k \rightarrow X_j$. Suppose that for each $j,k,l \in J$, 
		\begin{enumerate}
			\item $\pi_{j,j} = \id_{X_j}$,
			\item $j \leq k$ and $k \leq l$ implies that $\pi_{j,k} \circ \pi_{k,l} = \pi_{j,l}$.
		\end{enumerate}
		Then $((X_j)_{j \in J}, (\pi_{j,k})_{(j,k) \in \leq})$ is said to be a \tbf{projective system of sets}.
	\end{defn}
	
	\begin{defn} \ld{def:set_theory:proj_limits:0002}
		Let $(J, {\leq})$ be a directed poset and $((X_j)_{j \in J}, (\pi_{j,k})_{(j,k) \in \leq})$ a projective system of sets. We define 
		\begin{itemize}
			\item the \tbf{inverse limit} of $((X_j)_{j \in J}, (\pi_{j,k})_{(j,k) \in \leq})$, denoted $\varprojlim\limits_{j \in J} X_j$, by 
			$$\varprojlim\limits_{j \in J} X_j \defeq \bigg\{ x \in \prod\limits_{j \in J} X_j: \text{ for each $(j,k) \in {\leq}$, $\pi_{j,k} \circ \prj_k(x) = \prj_j(x) $} \bigg \}$$
			\item the \tbf{$j$-th projection map of $\varprojlim\limits_{j \in J} X_j$ onto $X_j$}, denoted $\pi_j: \varprojlim\limits_{j \in J} X_j \rightarrow X_j$, by 
			$$\pi_j = \prj_j|_{\varprojlim\limits_{j \in J} X_j}$$
		\end{itemize}
	\end{defn}
	
	\begin{ex} \lex{ex:set_theory:proj_limits:0003}
		Let $(J, {\leq})$ be a directed poset and $((X_j)_{j \in J}, (\pi_{j,k})_{(j,k) \in \leq})$ a projective system of sets. Then for each $j,k \in J$, $j \leq k$ implies that $\pi_{j,k} \circ \pi_k = \pi_j$. 
	\end{ex}
	
	\begin{proof}
		Let $x \in \varprojlim\limits_{j \in J} X_j$. Let $j,k \in J$. Suppose that $j \leq k$.  By definition, 
		\begin{align*}
			\pi_{j,k} \circ \pi_k(x)
			& = \pi_{j,k} \circ \prj_k(x) \\
			& = \prj_j(x) \\
			& = \pi_j(x).
		\end{align*}
		Since $x \in \varprojlim\limits_{j' \in J} X_{j'}$ is arbitrary, we have that $\pi_{j,k} \circ \pi_k = \pi_j$. 
	\end{proof}
	
	\begin{ex} \lex{ex:set_theory:proj_limits:0004}
		Let $(J, {\leq})$ be a directed poset and $((X_j)_{j \in J}, (\pi_{j,k})_{(j,k) \in \leq})$ a projective system of sets. Then 
		$$\varprojlim\limits_{j \in J} X_j = \bigcap\limits_{(j,k) \in {\leq}} \Eq(\pi_{j,k} \circ \prj_k, \prj_j)$$
	\end{ex}
	
	\begin{proof}
		Set $X \defeq $. By definition, 
		\begin{align*}
			\varprojlim\limits_{j \in J}
			& = \bigcap\limits_{(j,k) \in {\leq}} \{x \in X: \pi_{j,k} \circ \prj_k(x) = \prj_j(x)\} \\
			& = \bigcap\limits_{(j,k) \in {\leq}} \Eq( \pi_{j,k} \circ \prj_k,  \prj_j).
		\end{align*}
	\end{proof}
	
	
	
	
	
	
	
	
	
	
	
	
	
	
	
	
	
	
	
	
	
	
	
	
	
	
	
	
	
	
	
	
	
	
	
	
	
	
	
	
	
	
	
	
	
	\newpage
	\chapter{Ordered Sets}
	
	\tcr{This chapter is deprecated and has been moved to intro to algebra notes, need to replace all references in this doc which link to this chapter over to there}
	
	\section{Preorders}
	
	\begin{defn} \ld{def:orderings:prosets:0001} \tbf{Preordered Set:} \\
		Let $X$ be a set and $\leq \, \subset X \times X$ a binary relation on $X$. Then 
		\begin{itemize}
			\item $\leq$ is said to be a \tbf{preorder on $X$} if
			\begin{enumerate}
				\item for each $a \in X$, $a \leq a$
				\item for each $a, b, c \in X$, $a \leq b$ and $b \leq c$ implies that $a \leq c$
			\end{enumerate}
			\item $(X, \leq)$ is said to be a \tbf{preordered set} or \tbf{proset} if $\leq$ is a preorder on $X$.
		\end{itemize}
	\end{defn}
	
	\begin{defn} \ld{def:orderings:prosets:0002}
		Let $(A, \leq_A)$ and $(B, \leq_B)$ be directed sets. We define the 
		\begin{itemize}
			\item \tbf{product preorder of $\leq_A$ and $\leq_B$ on $A \times B$}, denoted $\leq_A \otimes \leq_B$ by $(a_1,b_1) \leq_A \otimes \leq_B (a_2, b_2)$ iff $a_1 \leq_A a_2$ and $b_1 \leq_B b_2$.
			\item \tbf{product proset of $(A, \leq_A)$ and $(B, \leq_B)$}, denoted $(A, \leq_A) \otimes (B, \leq_B)$ by $(A, \leq_A) \otimes (B, \leq_B) \defeq (A \times B, \leq_A \otimes \leq_B)$
		\end{itemize}
	\end{defn}
	
	\begin{ex} \lex{ex:orderings:prosets:0002}
		\tcr{probably need to change notation since $\otimes$ might be reserved for something else.}
		Let $(A, \leq_A)$ and $(B, \leq_B)$ be prosets. Then 
		\begin{enumerate}
			\item $\leq_A \otimes \leq_B$ is a preorder on $A \times B$,
			\item $(A, \leq_A) \otimes (B, \leq_B)$ is a proset.
		\end{enumerate}
	\end{ex}
	
	\begin{proof}\
		\begin{enumerate}
			\item 
			\begin{enumerate}
				\item Let $(a, b) \in A \times B$. Then $a \leq_A a$ and $b \leq_B b$. Therefore $(a, b) \leq_A \otimes \leq_B (a ,b)$.
				\item Let $(a_1, b_1), (a_2, b_2), (a_3, b_3) \in A \times B$. Suppose that $(a_1, b_1) \leq_A \otimes \leq_B (a_2, b_2)$ and $(a_2, b_2) \leq_A \otimes \leq_B (a_3, b_3)$. Then $a_1 \leq_A a_2$, $a_2 \leq_A a_3$, $b_1 \leq_B b_2$ and $b_2 \leq_B b_3$. Therefore $a_1 \leq_A a_3$ and $b_1 \leq_B b_3$. Hence $(a_1, b_1) \leq_A \otimes \leq_B (a_3, b_3)$.
			\end{enumerate}
			Hence $\leq_A \otimes \leq_B$ is a preorder on $A \times B$.
			\item Since $\leq_A \otimes \leq_B$ is a preorder on $A \times B$, $(X, \leq)$ is a proset.
		\end{enumerate}
	\end{proof}
	
	\begin{defn} \ld{def:orderings:prosets:0002}
		Let $(X, \leq)$ be a proset, $A \subset X$ and $x \in X$. Then $x$ is said to be a 
		\begin{itemize}
			\item \tbf{upper bound of $A$} if for each $a \in A$, $a \leq x$
			\item \tbf{lower bound of $A$} if for each $a \in A$, $a \geq x$
			\item \tbf{supremum of $A$} or \tbf{least upper bound of $A$}, denoted $x \in \sup A$, if 
			\begin{enumerate}
				\item $x$ is an upper bound of $A$ 
				\item for each $y \in x$, $y$ is an upper bound of $A$ implies that $x \leq y$ 
			\end{enumerate}
			If $\sup A = \{x\}$, we write $x = \sup A$. 
			\item \tbf{infimum of $A$} or \tbf{greatest lower bound of $A$}, denoted $x \in \inf A$, if 
			\begin{enumerate}
				\item $x$ is an lower bound of $A$ 
				\item for each $y \in x$, $y$ is an lower bound of $A$ implies that $x \geq y$. 
			\end{enumerate}
			If $\inf A = \{x\}$, we write $x = \inf A$. 
		\end{itemize} 
	\end{defn}
	
	\begin{defn} \ld{def:orderings:prosets:0003}
		Let $(X, \leq)$ be a proset, $A \subset X$. Then $A$ is said to be 
		\begin{itemize}
			\item \tbf{bounded above} if there exists $x \in X$ such that $x$ is an upper bound of $A$.
			\item \tbf{bounded below} if there exists $x \in X$ such that $x$ is a lower bound of $A$.
		\end{itemize}
	\end{defn}
	
	
	
	
	
	
	
	
	
	
	
	
	
	
	
	
	
	
	
	
	
	
	
	
	
	
	
	
	
	
	
	
	
	
	
	
	
	
	
	
	
	
	
	
	
	
	
	
	
	
	\newpage
	\section{Directed Sets}
	\begin{defn} \ld{def:orderings:directed_sets:0001} \ld{def:orderings:directed_sets:0001} \tbf{Directed Set:} \\
		Let $A$ be a set and $\leq \, \subset A \times A$ a binary relation on $A$. Then 
		\begin{itemize}
			\item $\leq$ is said to be a \tbf{direction on $A$} if
			\begin{enumerate}
				\item $\leq$ is a preorder on $A$
				\item for each $\al, \be \in A$, there exists $\gam \in A$ such that $\al, \be \leq \gam$
			\end{enumerate}
			\item $(A, \leq)$ is said to be a \tbf{directed set} if
			\begin{enumerate}
				\item $A \neq \varnothing$
				\item $\leq$ is a direction on $A$
			\end{enumerate}
		\end{itemize}
	\end{defn}
	
	\begin{ex} \lex{ex:orderings:directed_sets:0003}
		Let $(A, \leq_A)$ and $(B, \leq_B)$ be directed sets. Then 
		\begin{enumerate}
			\item $\leq_A \otimes \leq_B$ is a direction on $A \times B$,
			\item $(A, \leq_A) \otimes (B, \leq_B)$ is a directed set.
		\end{enumerate}
	\end{ex}
	
	\begin{proof}\
		\begin{enumerate}
			\item 
			\begin{enumerate}
				\item \rex{ex:orderings:prosets:0002} implies that $\leq_A \otimes \leq_B$ is a preorder of $A \times B$.
				\item Let $(a_1, b_1), (a_2, b_2) \in A \times B$. Then there exist $a \in A$ and $b \in B$ such that $a_1, a_2 \leq_A a$ and $b_1, b_2 \leq_B b$. Hence $(a_1, b_1), (a_2, b_2) \leq_A \otimes \leq_B (a, b)$.
			\end{enumerate}
			Hence $\leq_A \otimes \leq_B$ is a direction on $A \times B$.
			\item 
			\begin{enumerate}
				\item Since $A \neq \varnothing$ and $B \neq \varnothing$, we have that $A \times B \neq \varnothing$.
				\item From above, $\leq_A \otimes \leq_B$ is a direction on $A \times B$. 
			\end{enumerate}
			Hence $(A \times B, \leq_A \otimes \leq_B)$ is a directed set.
		\end{enumerate}
	\end{proof}
	
	
	
	
	
	
	
	
	
	
	
	
	
	
	
	
	
	
	
	
	
	
	
	
	
	
	
	
	
	
	
	
	
	
	
	
	
	
	
	
	
	
	
	
	
	
	
	\newpage
	\section{Posets}
	
	\begin{defn} \ld{def:orderings:posets:0001} \tbf{Poset:} \\
		Let $X$ be a set and $\leq \, \subset X \times X$ a binary relation on $X$. Then 
		\begin{itemize}
			\item $\leq$ is said to be a \tbf{partial ordering on $X$} if for each $a,b,c \in X$,
			\begin{enumerate}
				\item $\leq$ is a preorder on $X$
				\item $a \leq b$ and $b \leq a$ implies that $a = b$,
			\end{enumerate}
			\item $(X, \leq)$ is $(X, \leq)$ is said to be a \tbf{partially ordered set} or \tbf{poset} if
			$\leq$ is a partial ordering on $X$.
		\end{itemize}
	\end{defn}

	\begin{ex} \lex{ex:orderings:posets:0002}
		Let $(X, \leq)$ be a poset, $A \subset X$ and $x,y \in X$. If $x$ and $y$ are suprema of $A$, then $x =y$.   
	\end{ex}
	
	\begin{proof}
		Suppose that $x$ and $y$ are suprema of $A$. Then $x$ and $y$ are upper bounds of $A$. Since $x$ is a supremum of $A$ and $y$ is an upper bound of $A$, $x \leq y$. Since $y$ is a supremum of $A$ and $x$ is an upper bound of $A$, $y \leq x$. Thus $x = y$. 
	\end{proof}
	
	\begin{defn} \ld{def:orderings:posets:0003}
		Let $(X, \leq)$ be a poset. Then $(X, \leq)$ is said to satisfy the 
		\begin{itemize}
			\item \tbf{least upper bound (LUB) property} if for each $A \subset X$, if $A \neq \varnothing$ and $A$ is bounded above, then there exists $x \in X$ such that $x = \sup A$
			\item \tbf{greatest lower bound (GLB) property} if for each $A \subset X$, if $A \neq \varnothing$ and $A$ is bounded below, then there exists $x \in X$ such that $x = \inf A$. 
		\end{itemize}
	\end{defn}
	
	\begin{ex}
		LUB iff GLB
	\end{ex}
	
	\begin{proof}
		
	\end{proof}
	
	\begin{defn} \ld{def:orderings:posets:0003} \tbf{Suplattice:} \\
		Let $(L, \leq)$ be a poset. Then $(L, \leq)$ is said to be a \tbf{suplattice} if for each $A \subset L$, there exists $x \in L$ such that $x = \sup A$. 
	\end{defn}

	


























































	\newpage
	\chapter{Real and Complex Numbers}
	
	\section{Real Numbers}
	\begin{note}
		As a starting point, we will take as fact the existence of the \tbf{natural numbers} $$\N = \{1, 2, \cdots\}$$ the \tbf{integers} $$\Z = \{\cdots, -2, -2, 0, 1, 2, \cdots\}$$ and the \tbf{rational numbers} $$\Q = \bigg \{\frac{a}{b}: a \in \Z, b \in \N \bigg \}$$
	\end{note}
	
	\begin{defn} \ld{}
		Let $X$ be a set and $\leq$ a relation on $X$. Then $\leq$ is said to be a \tbf{total order} if for each $a,b,c \in X$,
		\begin{enumerate}
			\item $a \leq a$
			\item $a \leq b$ and $b \leq c$ implies that $a \leq  c$ 
			\item $a \leq b$ and $b \leq a$ implies that $a = b$ 
			\item $a \leq b$ or $b \leq a$
		\end{enumerate}
	\end{defn}

	\begin{ex} \lex{}
		We define the relation $\leq$ on $\Q$ defined by $$\frac{a}{b} \leq \frac{c}{d} \hspace{.2cm} \text{iff} \hspace{.2cm} ad \leq bc$$ Then $\leq$ is a total order of $\Q$.
	\end{ex}

	\begin{proof} Let $\frac{a}{b}, \frac{c}{d}, \frac{e}{f} \in \Q$. Then
		\begin{enumerate}
			\item  $\frac{a}{b} \leq \frac{a}{b}$ since $ab \leq ab$. 
			\item if $\frac{a}{b} \leq \frac{c}{d}$ and $\frac{c}{d} \leq  \frac{e}{f}$, then $ad \leq bc$ and $ cf \leq de$. Multiplying the first inequality by $f$ and the second inequality by $b$, we obtain $adf \leq bcf \leq bde$. Dividing both sides by $d$ yields $af \leq be$. Hence $\frac{a}{b} \leq \frac{e}{f}$. 
			\item if $\frac{a}{b} \leq \frac{c}{d}$ and $\frac{c}{d} \leq \frac{a}{b}$, then $ad \leq bc$ and $bc \leq ab$. This implies that $ad = bc$. Hence $\frac{a}{b} = \frac{c}{d}$.
			\item 
		\end{enumerate}
	\end{proof}
	
	
	\begin{defn}
		Let $A$ be a set
	\end{defn}
	
	\begin{ex}
		Let $A,B$ be sets and $f:A \times B \rightarrow \R$. Then 
		$$\sup_{(a,b) \in A \times B} f(a,b) = \sup_{a \in A} \bigg[ \sup_{b \in B} f(a,b) \bigg] =  \sup_{b \in B} \bigg[ \sup_{a \in A} f(a,b) \bigg]$$
	\end{ex}

	\begin{proof}
		For $(a,b) \in A \times B$, set $s_a = \sup_{b \in B} f(a,b)$ and $t_b = \sup_{a \in A}$. Let $(a,b) \in A \times B$, $A \times \{b\}$. Then $\{a\} \times B \subset A \times B$. Therefore 
		$$\sup_{a \in A} s_a, \sup_{b \in B} t_b \leq \sup_{(x,y) \in A \times B} f(x,y)$$
		Thus $\sup_{}$
	\end{proof}
	
	
	
	
	
	
	
	
	
	
	
	
	
	
	
	
	
	
	
	
	
	
	
	
	
	
	
	
	
	
	\section{Extended Real Numbers}
	
	\tcr{	
	\begin{itemize}
		\item define $\sup$ and $\inf$.  	
	\end{itemize}
	}
	
	\begin{defn}\
		\begin{itemize}
			\item We define the \tbf{extended real numbers}, denoted $\ol{\R}$, by 
			$$ \ol{\R} = \R \cup \{\pm \infty\}$$
			\item For $a,b \in \ol{\R}$, we define
			\begin{align*}
				a \pm \infty = \pm\infty + a & = \pm\infty, & a & \neq \mp\infty \\
				a \cdot (\pm\infty) = \pm\infty \cdot a & = \pm\infty, & a & \in (0, +\infty] \\
				a \cdot (\pm\infty) = \pm\infty \cdot a & = \mp\infty, & a & \in [-\infty, 0) \\
				\frac{a}{\pm\infty} & = 0, & a & \in \mathbb{R} \\
				\frac{\pm\infty}{a} & = \pm\infty, & a & \in (0, +\infty) \\
				\frac{\pm\infty}{a} & = \mp\infty, & a & \in (-\infty, 0) \\
				0 \cdot (\pm\infty) = \pm\infty \cdot 0 & = 0
			\end{align*}
			\item We define $\leq_{\ol{\R}} \subset \ol{\R} \times \ol{\R}$ by 
			$${\leq_{\ol{\R}}}  =  {\leq_{\R}} \cup  \{(a,b) \in \ol{\R} \times \ol{\R}: a = - \infty \text{ or } b = \infty\}$$
		\end{itemize}
	\end{defn}
	
	\begin{defn}
		Let $A \subset \ol{\R}$. We define the \tbf{supremum of $A$}, denoted $\sup A \defeq$.
	\end{defn}
	
	
	
	
	
	
	
	
	
	
	
	
	
	
	
	
	
	
	
	\section{Complex Numbers}
	
	We define $\Re, \Im: \C \rightarrow \R$ by $\Re(a + ib) = a$ and $\Im(a+ib) = b$.
	
	
	
	
	
	
	
	
	
	
	
	
	
	
	
	
	
	
	
	
	
	
	
	
	
	
	
	
	
	
	
	
	
	
	
	
	
	
	
	
	
	
	
	
	
	
	
	
	
	
	
	
	
	
	
	
	
	
	
	
	
	
	
	
	
	
	
	
	
	
	
	
	
	\newpage
	\chapter{Topology}
	
	\section{Introduction}
	
	\begin{defn} \ld{31001}
	Let $X$ be a set and $\MT \subset \MP(X)$. Then $\MT$ is said to be a \tbf{topology on $X$} if 
	\begin{enumerate}
	\item $X$, $\varnothing \in \MT$ 
	\item for each $(U_{\al})_{\al \in A} \subset \MT$, $$\bigcup_{\al \in A}U_{\al} \in \MT$$
	\item for each $(U_j)_{j=1}^n \subset \MT$, $$\bigcap_{j=1}^n U_{j} \in \MT$$
	\end{enumerate}
	\end{defn}		
	
	\begin{ex} \lex{31002} 
		Let $X$ be a set and $(\MT_{i})_{i \in I}$ a collection of topologies on $X$. Then $\bigcap\limits_{i \in I}\MT_i$ is a topology on $X$.
	\end{ex}
	
	\begin{proof}\
	\begin{enumerate}
	\item Since for each $i \in I$, $X, \varnothing \in \MT_{i}$, we have that $X, \varnothing \in \bigcap\limits_{i \in I}\MT_i$.
	\item Let $(U_{\al})_{\al \in A} \subset \bigcap\limits_{i \in I}\MT_i$. Then for each $i \in I$, $(U_{\al})_{\al \in A} \subset T_i$. So for each $i \in I$, $\bigcup\limits_{\al \in A}U_{\al} \in \MT_i$. Thus $\bigcup\limits_{\al \in A}U_{\al} \in \bigcap\limits_{i \in I}\MT_i$.
	\item Let $(U_{j})_{j=1}^n \subset \bigcap\limits_{i \in I}\MT_i$. Then for each $i \in I$, $(U_{j})_{j=1}^n \subset T_i$. So for each $i \in I$, $\bigcap\limits_{j=1}^n U_{j} \in \MT_i$. Thus $\bigcap\limits_{j=1}^n U_{j} \in \bigcap\limits_{i \in I}\MT_i$.
	\end{enumerate}
	So $\bigcap\limits_{i \in I}\MT_i$ is a topology on $X$.
	\end{proof}
	
	\begin{defn} \ld{31003}
	Let $X$ be a set and $\ME \subset \MP(X)$. Set 
	\begin{equation*}
	\MS = \{\MT \subset \MP(X): \MT \text{ is a topology 	on $X$ and $\ME \subset \MT$}\}
	\end{equation*}	 
We define the \tbf{topology generated by $\ME$} on $X$, denoted $\tau(\ME)$, by $$\tau(\ME) = \bigcap_{\MT \in \MS} \MT$$
	\end{defn}

	
	\begin{defn} \ld{31005}
	Let $X$ be a set and $\MT \subset \MP(X)$ a topology on $X$, $x \in X$ and $\MB_x \subset \MT$. Then $\MB_x$ is said to be a \tbf{local basis for $\MT$ at $x$} if 
	\begin{enumerate}
	\item for each $U \in \MB_x$, $x \in U$
	\item for each $V \in \MT$, if $x \in V$, then there exists $U \in \MB_x$ such that $U \subset V$
	\end{enumerate}
	\end{defn}
	
	\begin{defn} \ld{31006}
	Let $X$ be a set and $\MT \subset \MP(X)$ a topology on $X$ and $\MB \subset \MT$. Then $\MB$ is said to be a \tbf{basis for $\MT$} if for each $V \in \MT$ and $x \in V$, there exists $U \in \MB$ such that $x \subset U \subset V$.
	\end{defn}

	\begin{ex} \lex{31007}
	Let $X$ be a set and $\MT \subset \MP(X)$ a topology on $X$ and $\MB \subset \MT$. Then $\MB$ is a basis for $\MT$ iff for each $x \in X$, there exists $\MB_x \subset \MB$ such that $\MB_x$ is a local basis for $\MT$ at $x$. 
	\end{ex}

	\begin{proof}
		Suppose that $\MB$ is a basis for $\MT$. Let $x \in X$. Define $\MB_x = \{U \in \MB: x \in U\}$. 
		\begin{enumerate}
			\item By definition, for each $U \in \MB_x$, $x \in U$
			\item Let $V \in \MT$. Suppose that $x \in V$. Since $\MB$ is a basis, there exists $U \in \MB$ such that $x \in U \subset V$. By definition, $U \in \MB_x$.
		\end{enumerate}
		Hence $\MB_x$ is a local basis for $\MT$ at $x$. \\
		Conversely, suppose that for each $x \in X$, there exists $\MB_x \subset \MB$ such that $\MB_x$ is a local basis for $\MT$ at $x$. Let $V \in \MT$ and $x \in V$. By assumption, there exists $\MB_x \subset \MB$ such that $\MB_x$ is a local basis for $\MT$ at $x$. Since $\MB_x$ is a local basis for $\MT$ at $x$, there exists $U \in \MB_x \subset \MB$ such that $x \in U \subset V$. Hence $\MB$ is a basis for $\MT$. 
	\end{proof}
	
	\begin{ex} \lex{31008}
	Let $X$ be a set and $\MT \subset \MP(X)$ a topology on $X$ and $\MB \subset \MT$. Then $\MB$ is a basis for $\MT$ iff for each $V \in \MT$, there exists a collection $\MC \subset \MB$ such that $$V = \bigcup\limits_{U \in \MC} U$$
	\end{ex}
	
	\begin{proof}
	Suppose that $\MB$ is a basis for $\MT$. Let $V \in \MT$. Since since $\MB$ is a basis for $\MT$, for each $x \in V$, there exists $U_x \in \MB$ such that $x \in U_x \subset V$. Then $(U_x)_{x \in U} \subset \MB$ satisfies $V = \bigcup\limits_{x \in U} U_x$. \\
	Conversely, suppose that for each $V \in \MT$, there exists a collection $\MC \subset \MB$ such that $V = \bigcup\limits_{U \in \MC} U$. Let $V \in \MT$ and $x \in V$. By assumption, there exists a collection $\MC \subset \MB$ such that $V = \bigcup\limits_{U \in \MC} U$. Since $x \in V$, there exists $U \in \MC$ such that $x \in U$. Hence there exists $U \in \MB$ such that $x \in U \subset V$. Then $\MB$ is a basis for $\MT$.
	\end{proof}
	
	\begin{ex} \lex{31008.1}
	Let $X$ be a set and $\MT_1, \MT_2 \subset \MP(X)$ topologies on $X$ and $\MB \subset \MT_1$. Suppose that $\MT_1 \subset \MT_2$. If $\MB$ is a basis for $\MT_2$, then $\MB$ is a basis for $\MT_1$.  
	\end{ex}
	
	\begin{proof}
	Suppose that $\MB$ is a basis for $\MT_2$. Let $V \in \MT_1$. Then $V \in \MT_2$. Since $\MB$ is a basis for $\MT_2$, the previous exercise implies that there exists a collection $(U_{\al})_{\al \in A} \subset \MB$ such that $V = \bigcup\limits_{\al \in A} U_{\al}$. Thus the previous exercise implies that $\MB$ is a basis for $\MT_1$. 
	\end{proof}
	
	\begin{ex} \lex{31009}
	Let $X$ be a set and $\MB \subset \MP(X)$. Define 
	$$\MT_{\MB} = \{ U \subset X: \text{ for each $x \in U$, there exists $V \in \MB$ such that $x \in V \subset U$} \}$$ Then 
	\begin{enumerate}
		\item $\MT_{\MB}$ is a topology on $X$ iff 
		\begin{enumerate}
			\item for each $x \in X$, there exists $V \in \MB$ such that $x \in V$
			\item for each $x \in X$ and $U, V \in \MB$, if $x \in U \cap V$, then there exists $W \in \MB$ such that $x \in W \subset U \cap V$
		\end{enumerate}
		\item if $\MT_{\MB}$ is a topology on $X$, then $\MB$ is a basis for $\MT_{\MB}$
		\item if $\MT_{\MB}$ is a topology on $X$, then $\MT_{\MB} = \tau(\MB)$
	\end{enumerate}
	\end{ex}
	
	\begin{proof} \
	\begin{enumerate}
		\item \begin{itemize}
			\item $(\implies): $ \\
			Suppose that $\MT_{\MB}$ is a topology on $X$. 
			\begin{enumerate}
				\item Let $x \in X$. Since $\MT_{\MB}$ is a topology on $X$, $X \in \MT_{\MB}$. Since $x \in X$, the definition of $\MT_{\MB}$ implies that there exists $V \in \MB$ such that $x \in V \subset X$. 
				\item Let $x \in X$ and $U,V \in \MB$. Suppose that $x \in U \cap V$. Since $\MB \subset \MT$, we have that $U,V \in \MT_{\MB}$. Since $\MT_{\MB}$ is a topology on $X$, $U \cap V \in \MT_{\MB}$. By definition of $\MT_{\MB}$, there exists $W \in \MB$ such that $x \in W \subset U \cap V$.
			\end{enumerate}
			\item $(\impliedby): $ \\
			Conversely, suppose that $(a)$ and $(b)$ are satisfied. 
			\begin{itemize}
				\item Vacuously, $\varnothing \in \MT_{\MB}$. Condition $(a)$ implies that $X \in \MT_{\MB}$. 
				\item Let $(U_{\al})_{\al \in A} \subset \MT_{\MB}$ and $x \in \bigcup\limits_{\al \in A}U_{\al}$. Then there exists $\al \in A$ such that $x \in U_{\al}$. Since $U_{\al} \in \MT_{\MB}$, the definition of $\MT_{\MB}$ implies that there exists $V \in \MB$ such that
				\begin{align*}
					x 
					& \in V \\
					& \subset U_{\al} \\
					& \subset \bigcup\limits_{\al \in A}U_{\al}
				\end{align*}
				Since $x \in \bigcup\limits_{\al \in A}U_{\al}$ is arbitrary, $\bigcup\limits_{\al \in A}U_{\al} \in \MT_{\MB}$. 
				\item Let $U_1, U_2 \MT_{\MB}$ and $x \in U_1 \cap U_2$. The definition if $\MT_{\MB}$ implies that for $j \in \{1, 2\}$, there exists $V_j \in \MB$ such that $x \in V_j \subset U_j$. This implies that $x \in V_1 \cap V_2$ and by condition $(b)$, there exists $W \in \MB$ such that
				\begin{align*}
					x 
					& \in W \\
					& \subset V_1 \cap V_2 \\
					& \subset U_1 \cap U_2
				\end{align*}
				Since $x \in U_1 \cap U_2$ is arbitrary, $U_1 \cap U_2 \in \MT_{\MB}$. 
			\end{itemize}
			Thus $\MT_{\MB}$ is a topology on $X$.
		\end{itemize}
		\item Suppose that $\MT_{\MB}$ is a topology on $X$. Let $U \in \MT_{\MB}$ and $x \in U$. By definition of $\MT_{\MB}$, there exists $V \in \MB$ such that $x \subset V \subset U$. Since $U \in \MT_{\MB}$ and $x \in U$ are arbitrary, $\MB$ is a basis for $\MT_{\MB}$.
		\item Suppose that $\MT_{\MB}$ is a topology on $X$. Since $\MB \subset \MT_{\MB}$, we have that $\tau(\MB) \subset \MT_{\MB}$. Let $U \in \tau(\MB)$. \\
		Conversely, let $U \in \MT_{\MB}$. Since $\MT_{\MB}$ is a topology on $X$, part $(1)$ implies that $\MB$ is a basis for $\MT_{\MB}$. Then there exists $\MC \subset \MB$ such that 
		\begin{align*}
			U 
			& = \bigcup\limits_{V \in \MC} V \\
			& \in \tau(\MB)
		\end{align*}
		So $\MT_{\MB} \subset \tau(\MB)$. Hence $\MT_{\MB} = \tau(\MB)$.
	\end{enumerate}
	\end{proof}
 
	\begin{ex} \lex{31010.1}
		Let $X$ be a set and $\MB \subset \MP(X)$. Then $\MB$ is a basis for $\tau(\MB)$ iff 
		\begin{enumerate}
			\item for each $x \in X$, there exists $V \in \MB$ such that $x \in V$
			\item for each $x \in X$ and $U, V \in \MB$, if $x \in U \cap V$, then there exists $W \in \MB$ such that $x \in W \subset U \cap V$
		\end{enumerate} 
	\end{ex}

	\begin{proof}\
		\begin{itemize}
			\item  $(\implies): $ \\
			Suppose that $\MB$ is a basis for $\tau(\MB)$. 
			\begin{enumerate}
				\item Let $x \in X$. Since $X \in \tau(\MB)$ and $\MB$ is a basis for $\tau(\MB)$, there exists $V \in \MB$ such that $x \in V$.
				\item Let $x \in X$ and $U, V \in \MB$. Suppose that $x \in U \cap V$. Since $\tau(\MB)$ is a topology, $U \cap V \in \tau(\MB)$. Since $\MB$ is a basis for $\tau(\MB)$, there exists $W \in \MB$ such that $x \in W \subset U \cap V$.
			\end{enumerate}
			\item  $(\impliedby): $ \\
			Conversely, suppose that $(1)$ and $(2)$ are satisfied. The previous exercise implies that $\MB$ is a basis for $\tau(\MB)$.
		\end{itemize}
	\end{proof}
	
	\begin{ex} \lex{31010}
	Let $X$ be a set and $\ME \subset \MP(X)$. Define $\MB \subset \MP(X)$ by 
	$$\MB = \{X, \varnothing\} \cup  \bigg \{\bigcap_{j=1}^n V_j: (V_j)_{j=1}^n \subset \ME \bigg \}$$ 
	Then 
	\begin{enumerate}
	\item $\MB$ is a basis for $\tau(\ME)$ 
	\item $$\tau(\ME) = \bigg \{ \bigcup_{\al \in A} V_{\al}: (V_{\al})_{\al \in A} \subset \MB \bigg \}$$ That is, each element of $\tau(\ME)$ is either $X, \varnothing$ or a union of finite intersections of elements of $\ME$. 
	\end{enumerate}
	
	\end{ex}
	
	\begin{proof}\
	\begin{enumerate}
	\item Referring to \rex{31009}, since $X \in \MB$, condition $(1)$ is satisfied and since for each $U, V \in \MB$, $U \cap V \in \MB$, condition $(2)$ is satisfied. Hence there exists a topology $\MT$ on $X$ such that $\MB$ is a basis for $\MT$. Since $\MB \subset \MT$ and $\tau(\ME) = \tau(\MB)$, we have that $\tau(\ME) \subset \MT$. Since $\MB$ is a basis for $\MT$ and $\MB \subset \tau(\ME)$, \rex{31008.1} implies that $\MB$ is a basis for $\tau(\ME)$.
	\item \rex{31008} implies that $$\tau(\ME) = \bigg \{ \bigcup_{\al \in A} V_{\al}: (V_{\al})_{\al \in A} \subset \MB \bigg \}$$
	\end{enumerate}
	
	\end{proof}
	
	\begin{defn} \ld{31011}
	Let $X$ be a set and $\MT$ a topology on $X$. Then $(X, \MT)$ is said to be a \tbf{topological space}. Let $U \subset X$. Then $U$ is said to be \tbf{open in $(X, \MT)$} if $U \in \MT$ and $U$ is said to be \tbf{closed in $(X, \MT)$} if $U^c$ is open in $(X, \MT)$.
	\end{defn}

	\begin{note}
		If we assume $X$ is a topological space, we denote the implicit topology on $X$ by $\MT_X$. 
	\end{note}
	
	\begin{note}
		When the context is clear, we say ``open" or ``closed" instead of ``open in $(X, \MT)$" or ``closed in $(X, \MT)$" respectively.
	\end{note}
	
	\begin{defn} \ld{31012}
	Let $(X, \MT)$ be a topological space and $S,N \subset X$. Then $N$ is said to be a \tbf{$\MT$-neighborhood} of $S$ if there exists $U \in \MT$ such that $S \subset U$ and $U \subset N$. For $S \in X$, we denote the set of $\MT$-neighborhoods of $S$ by $\MN_{\MT}(S)$.
	\end{defn}

	\begin{note}
		When the context is clear, we write $\MN(S)$ in place of $\MN_{\MT}(S)$ and say ``neighborhood" instead of ``$\MT$-neighborhood".
	\end{note}
	
	\begin{defn} \ld{31013}
	Let $(X, \MT)$ be a topological space and $A \subset X$. We define
	\begin{itemize}
		\item the \tbf{collection of open subsets of $A$}, denoted $\MU_A(X, \MT)$, by 
		$$\MU_{A}(X, \MT) \defeq \{U \subset X:U \subset A \text{ and $U$ is open in $(X, \MT)$}\},$$ 
		\item the \tbf{collection of closed supersets of $A$}, denoted $\MC_A(X, \MT)$, by 
		$$\MC_{A}(X, \MT) \defeq \{C \subset X: A \subset C \text{ and $C$ is closed in $(X, \MT)$}\},$$ 
		the \tbf{interior of $A$ in $(X, \MT)$}, denoted $\Int_{\MT} A$, by 
		$$\Int A = \bigcup_{U \in \MU_{A}(X, \MT)} U$$ 
		the \tbf{closure of $A$ in $(X, \MT)$}, denoted $\cl_{\MT} A$, by 
		$$\cl A = \bigcap_{C \in \MC_{A}(X, \MT)} C$$ 
	\end{itemize}
	\end{defn}

	\begin{note}
		When the context is clear, we write $\MU_A$, $\MC_A$, $\Int A$ and $\cl A$ in place of $\MU_A(X, \MT)$, $\MC_A(X, \MT)$, $\Int_{\MT} A$ and $\cl_{\MT} A$ respectively. For intuition, $\Int A$ is the largest open subset of $A$ and $\cl A$ is the smallest closed superset of $A$. 
	\end{note}
	
	\begin{ex} \lex{31014}
	Let $X$ be a topological space and $A \subset X$. Then 
	\begin{enumerate}
	\item $A$ is open iff $A = \Int A$ 
	\item $A$ is closed iff $A = \cl A$
	\end{enumerate}
	\end{ex}
	
	\begin{proof}
	Clear.
	\end{proof}

	\begin{ex} \lex{31014.1}
		Let $X$ be a topological space and $S, N \subset X$. Then $N \in \MN(S)$ iff $S \subset \Int N$. 
	\end{ex}
	
	\begin{proof}\
		\begin{itemize}
			\item $(\implies):$ \\
			Suppose that $N \in \MN(S)$. Then there exists $U \in \MT_X$ such that $S \subset U$ and $U \subset N$. Hence $U \in \MU_N$ and therefore
			\begin{align*}
				S
				& \subset U \\
				& \subset \bigcup_{V \in \MU_N} V \\
				& = \Int N.
			\end{align*}
			\item $(\impliedby):$ \\
			Suppose that $S \subset \Int N$. Since $\Int N \in \MT_X$, $S \subset \Int N$ and $\Int N \subset N$, we have that $N \in \MN(S)$. 
		\end{itemize}
	\end{proof}

	\begin{ex} \lex{31014.2}
		Let $(X, \MT)$ be a topological space, $x \in X$ and $U, V \in \MN_{\MT}(x)$. Then $U \cap V \in \MN_{\MT}(x)$.
	\end{ex}

	\begin{proof}
		Since $U, V \in \MN_{\MT}(x)$, \rex{31014.1} implies that $x \in \Int U$ and $x \in \Int V$. Then 
		\begin{align*}
			x 
			& \in \Int U \cap \Int V \\
			& \subset U \cap V.
		\end{align*} 
		\rex{31014} implies that $\Int U, \Int V \in \MT$. Therefore $\Int U \cap \Int V \in \MT$. Thus there exists $W \in \MT$ such that $x \in W$ and $W \subset U \cap V$. By definition, $U \cap V \in \MN_{\MT}(x)$.
	\end{proof}
	
	\begin{ex} \lex{31015}
	Let $X$ be a topological space and $A \subset X$. Then $( \Int A )^c = \cl A^c $.
	\end{ex}	
	
	\begin{proof}
	Define  $\MU_A = \{U \subset X:U \subset A \text{ and $U$ is open}\}$ and $\MC_{A^c} = \{C \subset X: A^c \subset C \text{ and $C$ is closed}\}$ as in \rd{31013}. 
	We note that 
	\begin{enumerate}
		\item for each $U \subset X$,
		\begin{align*}
			U \in \MU_A
			& \iff U \subset A \text{ and } U \text{ is open} \\
			& \iff A^c \subset U^c \text{ and } U^c \text{ is closed} \\
			& \iff  U^c \in \MC_{A^c}
		\end{align*}
		\item
		\begin{align*}
			C \in \MC_{A^c}
			& \iff A^c \subset C \text{ and } C \text{ is closed} \\
			& \iff C^c \subset A \text{ and } C^c \text{ is open} \\
			& \iff  C^c \in \MU_A
		\end{align*} 
	\end{enumerate}
	Let $C \in \{U^c: U \in \MU_A\}$. Then there exists $U \in \MU_A$ such that $C = U^c$. By $(1)$,  
	\begin{align*}
		C 
		& = U^c \\
		& \in \MC_{A^c}
	\end{align*}
	Since $C \in \{U^c: U \in \MU_A\}$ is arbitrary, $\{U^c: U \in \MU_A\} \subset \MC_{A^c}$. \\
	Conversely, let $C \in \MC_{A^c}$. Then $(2)$ implies that $C^c \in \MU_A$ and therefore 
	\begin{align*}
		C
		& = (C^c)^c \\
		& \in \{U^c: U \in \MU_A\}
	\end{align*} 
	Since $C \in \MC_{A^c}$ is arbitrary, $\MC_{A^c} \subset \{U^c: U \in \MU_A\}$. Hence $\MC_{A^c} = \{U^c: U \in \MU_A\}$ and therefore
	\begin{align*}
		(\Int A)^c 
		& = \bigg( \bigcup_{U \in \MU_{A}} U \bigg)^c \\
		& = \bigcap_{U \in \MU_A} U^c \\
		& = \bigcap_{C \in \MC_{A^c}} C \\
		& = \cl A^c
	\end{align*}
	\end{proof}

	\begin{ex} \lex{31015.1}
		Let $X$ be a topological space and $A \subset X$. Then $( \cl A )^c = \Int A^c $.
	\end{ex}

	\begin{proof}
		Define $B = A^c$. The previous exercise implies that $( \Int B )^c = \cl B^c $. Therefore 
		\begin{align*}
			\Int A^c 
			& = \Int B \\
			& = ( \cl B^c)^c \\
			& = (\cl A) ^c 
		\end{align*}
	\end{proof}
	
	\begin{ex} \lex{31017}
	Let $X$ be a topological space and $A \subset X$. Then $A$ is open iff for each $x \in A$, there exists $U \in \MN(x)$ such that $U$ is open and $U \subset A$.
	\end{ex}
	
	\begin{proof}
	Suppose that $A$ is open. Let $x \in A$. Then $A \in \MN(x)$, $A$ is open and $A \subset A$. Conversely, suppose that for each $x \in A$, there exists $U_x \in \MN(x)$ such that $U_x$ is open and $U_x \subset A$. Since 
	$$A = \bigcup\limits_{x \in A}U_x,$$ 
	we have that $A$ is open. 
	\end{proof}

	\begin{ex} \lex{31017.1}
	Let $X$ be a topological space, $A \subset X$ and $x \in X$. Then $x \in \cl A$ iff for each $U \in \MN(x)$, $U$ is open implies that $A \cap U \neq \varnothing$.
	\end{ex}

	\begin{proof} \
		\begin{itemize}
			\item $(\implies)$ \\
			Suppose that $x \in \cl A$. Let $U \in \MN(x)$. Suppose that $U$ is open. For the sake of contradiction, suppose that $A \cap U = \varnothing$. Then $A \subset U^c$. Since $U^c$ is closed, we have that 
			\begin{align*}
				x 
				& \in \cl A \\
				& \subset U^c
			\end{align*}
			This is a contradiction since $x \in U$. Hence $A \cap U \neq \varnothing$. 
			\item $(\impliedby)$ \\ 
			Suppose that for each $U \in \MN(x)$, $U$ is open implies that $A \cap U \neq \varnothing$. For the sake of contradiction, suppose that $x \not \in \cl A$. By definition of closure, there exists $C \subset X$ such that $C$ is closed, $A \subset X$ and $x \not \in C$. Therefore $C^c$ is open and $x \in C^c$. Thus $C^c \in \MN(x)$. By assumption, $A \cap C^c \neq \varnothing$. This is a contradiction since $A \subset C$. So $x \in \cl A$.
			
		\end{itemize}
	\end{proof}
	
	\begin{defn} \ld{31018}
	Let $(X, \MT)$ be a topological space, $A \subset X$ and $x \in X$. Then $x$ is said to be a \tbf{$\MT$-limit point of $A$} if for each $U \in \MN_{\MT}(x)$, $$A \cap (U \setminus \{x\}) \neq \varnothing$$  
	We define $\lmt_{\MT} A = \{x \in A: \text{$x$ is a limit point of $A$}\}$. 
	\end{defn}

	\begin{note}
		When context is clear, we write $A'$ in place of $\lmt_{\MT}(A)$.
	\end{note}
	
	\begin{ex} \lex{31019}
	Let $X$ be a topological space and $A \subset X$. Then $\cl A = A \cup A'$. 
	\end{ex}	
	
	\begin{proof}
	Let $x \in A'$. For the sake of contradiction, suppose that $x \not \in \cl A$. By definition of closure, there exists $C \subset X$ such thath $C$ is closed, $A \subset C$ and $x \not \in C$. Hence $x \in C^c \subset A^c$. Since $C^c$ is open, $x \in \Int A^c$. Since $x \in A'$ and $\Int A^c \in \MN(x)$, $[\Int A^c \setminus \{x\}] \cap A \neq \varnothing$. This is a contradiction since $\Int A^c \setminus \{x\} \subset A^c$. So $x \in \cl A$ and $A' \subset \cl A$. Since $A \subset \cl A$, we have that $A \cup A' \subset \cl A $.\\ 
	Conversely, let $x \in \cl A$. For the sake of contradiction, suppose that $x \not \in A \cup A'$. Then $x \in A^c \cap (A')^c$. Since $x \in (A')^c$, there exists $U \in \MN(x)$ such that $(U \setminus \{x\}) \cap A = \varnothing$. Hence $U \setminus \{x\} \subset A^c$. Since $x \in A^c$, 
	\begin{align*}
	\Int U 
	& \subset U \\ 
	&= (U \setminus \{x\}) \cup \{x\} \\
	& \subset A^c
\end{align*}	
	Which implies that $A \subset (\Int U)^c$. Since $(\Int U)^c$ is closed, 
	\begin{align*}
		x 
		& \in \cl A \\
		& \subset (\Int U)^c 
	\end{align*}
	which is a contradiction since $x \in \Int U$. So $x \in A \cup A'$. Since $x \in \cl A$ is arbitrary, $\cl A \subset A \cup A'$. Therefore $\cl A = A \cup A'$.
	\end{proof}
	
	\begin{defn} \ld{31020} 
		Let $(X, \MT)$ be a topological space and $A \subset X$. Then $A$ is said to be \tbf{dense} in $(X, \MT)$ if $\cl A = X$. 
	\end{defn}

	\begin{ex} \lex{31021} 
		Let $X$ be a topological space and $A \subset X$. Then $A = \varnothing$ iff $\cl A = \varnothing$. 
	\end{ex}

	\begin{proof}
		Suppose $A = \varnothing$. Since $A$ is closed, 
		\begin{align*}
			\cl A 
			& = A \\
			& = \varnothing
		\end{align*}
	Conversely, suppose that $\cl A = \varnothing$. Since $A \subset \cl A$, $A = \varnothing$. 
	\end{proof}

	\begin{ex} \lex{31022} 
		Let $X$ be a topological space and $A \subset X$. Then $A$ is dense in $X$ iff for each $U \subset X$, $U$ is open and $U \neq \varnothing$ implies that $A \cap U \neq \varnothing$.
	\end{ex}

	\begin{proof}
		Suppose that $A$ is dense in $X$. Let $U \subset X$. Suppose that $U$ is open. For the sake of contradiction, suppose that  $A \cap U = \varnothing$. Then $U \subset A^c$. Thus $A \subset U^c$. Since $U^c$ is closed, we have that  
		\begin{align*}
			X
			& = \cl A \\
			& \subset U^c
		\end{align*}
		Therefore, $X = U^c$ and hence $U = \varnothing$. This is a contradiction. So for each $U \subset X$, $U$ is open and $U \neq \varnothing$ implies that $A \cap U \neq \varnothing$. \\
		Conversely, suppose that for each $U \subset X$, if $U$ is open and $U \neq \varnothing$, then $A \cap U \neq \varnothing$. Set $U =(\cl A)^c$. Then $U$ is open. For the sake of contradiction, suppose that $U \neq \varnothing$. By assumption there exists $x \in X$ such that 
		\begin{align*}
			x 
			& \in A \cap U \\
			& = A \cap (\cl A)^c \\
			& \subset A \cap A^c \\
			& = \varnothing
		\end{align*}
		which is a contradiction. Hence $U = \varnothing$. Then 
		\begin{align*}
			X
			& = U^c \\
			& = \cl A \\
		\end{align*}
		so that $A$ is dense in $X$.
	\end{proof}

	\begin{defn} \ld{31023} 
		Let $X$ be a topological space and $A \subset X$. Then $A$ is said to be \tbf{nowhere dense} in $X$ if $\Int \cl A = \varnothing$. 
	\end{defn}

	\begin{ex} \lex{31024} 
		Let $X$ be a topological space and $A \subset X$. If $A$ is nowhere dense in $X$, then $\cl A$ is nowhere dense.
	\end{ex}
	
	\begin{proof}
		Suppose that $A$ is nowhere dense in $X$. Then 
		\begin{align*}
			\Int \cl \cl A
			& = \Int \cl A \\
			& = \varnothing
		\end{align*}
		Hence $\cl A$ is nowhere dense.
	\end{proof}
	

	\begin{ex} \lex{31025} 
		Let $X$ be a topological space and $A \subset X$. If $A$ is nowhere dense in $X$, then $A^c$ is dense.
	\end{ex}
	
	\begin{proof}
		Suppose that $A$ is nowhere dense in $X$. Let $U \subset X$. Suppose that $U$ is open and nonempty. For the sake of contradiction, suppose that  $A^c \cap U = \varnothing$. Then 
		\begin{align*}
			U 
			& \subset (A^c)^c \\
			& = A \\
			& \subset \cl A
		\end{align*}
		Since $U$ is open, we have that  
		\begin{align*}
			U 
			& \subset \Int \cl A \\
			& = \varnothing 
		\end{align*}
		Therefore, $U = \varnothing$. This is a contradiction since $U$ is nonempty. Hence $A^c \cap U \neq \varnothing$. Since $U$ is arbitrary open nonempty subset of $X$, we have that for each $U \subset X$, if $U$ is open and nonempty, then $A^c \cap U \neq \varnothing$. Thus $A^c$ is dense.  \\
	\end{proof}
	
	\begin{defn} \lex{31026} 
		Let $(X, \MT)$ be a topological space, $A \subset X$ and $x \in X$. Then $x$ is said to be a \tbf{$\MT$-condensation point of $A$} if for each $U \in \MN_{\MT}(x)$, $U \cap A$ is uncountable. 
	\end{defn}

	\begin{note}
		When the context is clear, we say ``condensation point" instead of ``$\MT$-condensation point".
	\end{note}

	\begin{ex} \lex{31026.1}
		Let $(X, \MT)$ be a topological space, $A \subset X$ and $x \in X$. If $x$ is a $\MT$-condensation point of $A$, then $x$ is a $\MT$-limit point of $A$
	\end{ex}

	\begin{proof}
		Suppose that $x$ is a $\MT$-condensation point of $A$. Let $U \in \MN_{\MT}(x)$. Since $x$ is a $\MT$-condensation point of $A$, $U \cap A$ is uncountable. Hence 
		\begin{align*}
			A \cap (U \setminus \{x\})
			& = A \cap (U \cap \{x\}^c) \\
			& = (A \cap U) \cap \{x\}^x
		\end{align*}
		is uncountable. In particular, $A \cap (U \setminus \{x\}) \neq \varnothing$. Since $U \in \MN_{\MT}((x))$ is arbitrary, we have that for each $U \in \MN_{\MT}((x))$, $A \cap (U \setminus \{x\}) \neq \varnothing$. Hence $x \in \lim_{\MT}A$. 
	\end{proof}
	
	\begin{ex} \lex{31027} 
		Let $(X, \MT)$ be a topological space. Define $C \subset X$ by 
		$$C \defeq \{x \in X: \text{$x$ is a $\MT$-condensation point of $X$}\}.$$
		Then 
		\begin{enumerate}
			\item for each $x \in X$, $x \in C^c$ iff there exists $U \in \MT$ such that $x \in U$, $U$ is countable and $U \subset C^c$. 
			\item $C$ is closed.
		\end{enumerate}
	\end{ex}
	
	\begin{proof} \
		\begin{enumerate}
			\item Let $x \in X$. 
			\begin{itemize}
				\item $(\implies):$ \\
				Suppose that $x \in C^c$. By definition, there exists $V \in \MN(x)$ such that $x \in V$ and $V$ is countable. Define $U \in \MT$ by $U \defeq \Int V$. Since $V \in \MN(x)$, we have that $x \in U$. Since $U \subset V$, we have that $U$ is countable. Let $y \in U$. Since $y \in U$ and $U \cap X$ is countable, $y \in C^c$. Since $y \in U$ is arbitrary, we have that $U \subset C^c$.
				\item $(\impliedby):$ \\
				Suppose that there exists $U \in \MT$ such that $x \in U$, $U$ is countable and $U \subset C^c$. Then $U \in \MN(x)$ and $U \cap X$ is countable. Thus $x \in C^c$.
			\end{itemize} 
			\item The previous part implies that for each $x \in C^c$, there exists $U \in \MT$ such that $x \in U$, $U$ is countable and $U \subset C^c$. The axiom of choice implies that there exists $(U_x)_{x \in C^c} \subset \MT$ such that for each $x \in C^c$, $x \in U_x$, $U_x$ is countable and $U_x \subset C^c$. Thus 
			\begin{align*}
				C^c
				& = \bigcup_{x \in C^c} U_x \\
				& \in \MT.
			\end{align*}
			Since $C^c$ is open, we have that $C$ is closed.
		\end{enumerate}
	\end{proof}
	
	
	\begin{ex} \lex{31028} 
		Let $(X, \MT)$ be a topological space $x \in X$. The following are equivalent: 
		\begin{enumerate}
			\item $x$ is a $\MT$-condensation point of $X$,
			\item for each $U \in \MN_{\MT}(x)$, $U$ is uncountable,
			\item for each $U \in \MN_T(x)$, $x$ is a $\MT$-condensation point of $U$,
		\end{enumerate}
	\end{ex}

	\begin{proof}\
		\begin{enumerate}
			\item $(1) \implies (2)$: \\
			Suppose that $x$ is a $\MT$-condensation point of $X$. Let $U \in \MN_{\MT}(x)$. Since $x$ is a $\MT$-condensation point of $X$, $U \cap X$ is uncountable. Since $U = U \cap X$, $U$ is uncountable. Since $U \in \MN_{\MT}(x)$ is arbitrary, we have that for each $U \in \MN_{\MT}(x)$, $U$ is uncountable.
			\item $(2) \implies (3)$: \\
			Suppose that for each $U \in \MN_{\MT}(x)$, $U$ is uncountable. Let $U, V \in \MN_{\MT}(x)$. \rex{31014.2} implies that $U \cap V \in \MN_{\MT}(x)$. Then by assumption, $U \cap V$ is uncountable. Since $V \in \MN_{\MT}(x)$ is arbitrary, we have that for each $V \in \MN_{\MT}(x)$, $U \cap V$ is uncountable. Hence $x$ is a $\MT$-condensation point of $U$. Since $U \in \MN_{\MT}(x)$ is arbitrary, we have that for each $U \in \MN_{\MT}(x)$, $x$ is a $\MT$-condensation point of $U$.
			\item $(3) \implies (1)$: \\
			Suppose that for each $U \in \MN_T(x)$, $x$ is a $\MT$-condensation point of $U$. Let $U \in \MN_T(x)$. Since $X \in \MN_T(x)$, and $x$ is a $\MT$-condensation point of $U$, we have that $U \cap X$ is uncountable. Since $U \in \MN_T(x)$ is arbitrary, we have that for each $U \in \MN_T(x)$, $U \cap X$ is uncountable. Hence $x$ is a $\MT$-condensation point of $X$.
		\end{enumerate}
	\end{proof}
	
	
	\begin{defn} \ld{31029}
		Let $(X, \MT)$ be a topological space $E \subset X$ and $\MU \subset \MP(X)$. Then $\MU$ is said to be an \tbf{open cover} of $E$ in $(X, \MT)$ if 
		\begin{enumerate}
			\item $\MU \subset \MT$ 
			\item $\MU$ is a cover of $E$ in $X$.
		\end{enumerate}
	\end{defn}
	
	
	\begin{defn} \ld{31030} 
		Let $(X, \MT)$ be a topological space and $A \subset X$. Then 
		\begin{itemize}
			\item $A$ is said to be an \tbf{$G_{\del}$-set} if there exist $(U_n)_{n \in \N} \subset \MT$ such that $A = \bigcap\limits_{n \in \N} U_n$
			\item $A$ is said to be an \tbf{$F_{\sig}$-set} if $A^c$ is a $G_{\del}$-set.
		\end{itemize}
	\end{defn}
	
	\begin{ex} \lex{31031}
		Let $(X, \MT)$ be a topological space and $A \subset X$. Then $A$ is an $F_{\sig}$-set iff there exists $(C_n)_{n \in \N} \subset \MP(X)$ such that for each $n \in \N$, $C_n$ is $\MT$-closed and $A = \bigcup\limits_{n \in \N} C_n$.
	\end{ex}

	\begin{proof}
		Clear
	\end{proof}
	
	
	\begin{defn} \ld{31032}
		Let $X$ be a topological space, $E \subset X$.  
		\begin{itemize}
			\item Let $x \in X$. Then $x$ is said to be a \tbf{boundary point of $E$} if for each $U \in \MN(x)$, $U \cap E \neq \varnothing$ and $U \cap E^c \neq \varnothing$.
			\item We define the \tbf{boundary of $E$}, denoted $\p E$, by 
			$$\p E = \{x \in X: \text{ $x$ is a boundary point $E$}\}$$
		\end{itemize}
	\end{defn}
	
	
	
	
	
	
	
	
	
	
	
	
	
	
	
	
	
	
	
	\newpage
	\section{Continuous Maps}	
	
	\begin{defn} \ld{def:continuous_maps:0001}
	Let $(X,\MT_X)$ and $(Y,\MT_Y)$ be topological spaces and $f:X \rightarrow Y$. Then $f$ is said to be \tbf{$(\MT_X, \MT_Y)$-continuous} if for each $B \in \MT_Y$, $f^{-1}(B) \in \MT_X$.
	\end{defn}

	\begin{ex} \lex{ex:continuous_maps:0001.1}
		Let $(X,\MT_X)$ and $(Y,\MT_Y)$ be topological spaces and $f:X \rightarrow Y$. Then $f$ is $(\MT_X, \MT_Y)$-continuous iff for each $C \subset Y$, $C$ is closed in $(Y, \MT_Y)$ implies that $f{-1}(C)$ is closed in $(X, \MT_X)$. 
	\end{ex}

	\begin{proof}\
		\begin{itemize}
			\item $(\implies):$ \\
			Suppose that $f$ is $(\MT_X, \MT_Y)$-continuous. Let $C \subset Y$. Suppose that $C$ is closed in $(Y, \MT_Y)$. Then $C^c \in \MT_Y$. By continuity of $f$,
			\begin{align*}
				f^{-1}(C)^c
				& = f^{-1}(C^c) \\
				& \in \MT_X.
			\end{align*} 
			Thus $f^{-1}(C)$ is closed in $(X, \MT_X)$. Since $C \subset Y$ with $C$ closed in $(Y, \MT_Y)$ is arbtitrary, we have that for each $C \subset Y$, $C$ is closed in $(Y, \MT_Y)$ implies that $f{-1}(C)$ is closed in $(X, \MT_X)$. 
			\item $(\impliedby):$ \\
			Suppose that for each $C \subset Y$, $C$ is closed in $(Y, \MT_Y)$ implies that $f{-1}(C)$ is closed in $(X, \MT_X)$. Let $B \in \MT_Y$. Then $B^c$ is closed in $(Y, \MT_Y)$. Hence $f^{-1}(B^c)$ is closed in $(X, \MT_X)$. Since $f^{-1}(B)^c = f^{-1}(B^c)$, we have that $f^{-1}(B)^c$ is closed in $(X, \MT_X)$. Hence $f^{-1}(B) \in \MT_X$. Since $B \in \MT_Y$ is arbitrary, we have that for each $B \in \MT_Y$, $f^{-1}(B) \in \MT_X$. Hence $f$ is $(\MT_X, \MT_Y)$-continuous.
		\end{itemize}
	\end{proof}
	
	\begin{defn} \ld{def:continuous_maps:0002}
	Let $(X,\MA)$ and $(Y,\MB)$ be topological spaces, $f:X \rightarrow Y$ and $x \in X$. Then $f$ is said to be \tbf{continuous at $x$} if for each $V \in \MN(f(x))$, there exists $U \in \MN(x)$ such that $f(U) \subset V$. 
	\end{defn}		
	
	\begin{ex} \lex{ex:continuous_maps:0003}
	Let $(X,\MA)$ and $(Y,\MB)$ be topological spaces, $f:X \rightarrow Y$ and $x \in X$. Then $f$ is continuous at $x$ iff for each $V \in \MN(f(x))$, $f^{-1}(V) \in \MN(x)$.\\
	\tbf{Hint:} for $U \in \MN(x)$ and $V \in \MN(f(x))$, consider $f^{-1}(f(U))$ and $f(f^{-1}(V))$
	\end{ex}
	
	\begin{proof}
	Suppose that $f$ is continuous at $x$. Let $V \in \MN(f(x))$. Then there exists $U \in \MN(x)$ such that $f(U) \subset V$. Thus
	\begin{align*}
	x 
	&\in \Int U \\
	& \subset U \\
	&\subset f^{-1}(f(U)) \\
	&\subset f^{-1}(V)
	\end{align*}
	So $f^{-1}(V) \in \MN(x)$.\\
	Conversely, suppose that for each $V \in \MN(f(x))$, $f^{-1}(V) \in \MN(x)$. Let $V \in \MN(f(x))$. Hence $f^{-1}(V) \in \MN(x)$. Set $U = f^{-1}(V)$. Then 
	\begin{align*}
	f(U) 
	&= f(f^{-1}(V)) \\
	& \subset V
	\end{align*}
	Thus $f$ is continuous at $x$.
	\end{proof}
	
	\begin{ex} \lex{ex:continuous_maps:0004}
	Let $(X,\MA)$ and $(Y,\MB)$ be topological spaces and $f:X \rightarrow Y$. Then $f$ is continuous iff for each $x \in X$, $f$ is continuous at $x$.
	\end{ex}
	
	\begin{proof}
	Suppose that $f$ is continuous. Let $x \in X$. Let $V \in \MN(f(x))$. Then $\Int V \in \MB$ and $f(x) \in \Int V$. Set $U = f^{-1}(\Int V)$. By continuity, $U \in \MA$ and by construction, $x \in U$. Hence $U \in \MN(x)$. Then 
	\begin{align*}
	f(U)
	&= f(f^{-1}(\Int V))\\
	& \subset \Int V\\
	& \subset V
\end{align*}	 	
So $f$ is continuous at $x$. \\
Conversely, suppose that for each $x \in X$, $f$ is continuous at $x$. Let $B \in \MB$. Let $x \in f^{-1}(B)$. Then $B \in \MN(f(x))$. Continuity at $x$ implies that $f^{-1}(B) \in \MN(x)$. Then $x \in \Int (f^{-1}(B))$. Since $x \in f^{-1}(B)$ is arbitrary, $f^{-1}(B) \subset \Int (f^{-1}(B))$. Hence $f^{-1}(B) = \Int (f^{-1}(B))$ which implies that $f^{-1}(B) \in \MA$. So $f$ is continuous.
	\end{proof}
	
	\begin{defn} \ld{def:continuous_maps:0005}
	Let $(X, \MA)$ and $(Y,\MB)$ be topological spaces and $f: X \rightarrow Y$. We define the 
	\begin{enumerate}
	\item \tbf{push-forward of $\MA$}, denoted $f_*\MA$, by 
	$$f_*\MA = \{B \subset Y: f^{-1}(B) \in \MA\}$$ 
	\item  \tbf{pull-back of $\MB$}, denoted $f^*\MB$, by  
	$$f^*\MB = \{f^{-1}(B):  B \in \MB \}$$
	\end{enumerate}
	\end{defn}
	
	\begin{ex} \lex{ex:continuous_maps:0006}
		Let $(X,\MA)$ and $(Y,\MB)$ be topological spaces and $f: X \rightarrow Y$. Then 
		\begin{enumerate}
			\item $f_*\MA$ is a topology on $Y$
			\item $f^*\MB$ is a topology on $X$
		\end{enumerate}
	\end{ex}
	
	\begin{proof}\
		\begin{enumerate}
			\item 
			\begin{itemize}
			\item Since $f^{-1}(Y) = X \in \MA$ and $f^{-1}(\varnothing) = \varnothing \in \MA$, $Y, \varnothing \in f_*\MA$.
			\item Let $(U_{\al})_{\al \in A} \subset f_*\MA$. Then for each $\al \in A$, $f^{-1}(U_{\al}) \in \MA$. This implies that 
			\begin{align*}
			f^{-1}\bigg( \bigcup\limits_{\al \in A}U_{\al} \bigg) 
			&=  \bigcup\limits_{\al \in A} f^{-1}(U_{\al}) \\
			& \in \MA
			\end{align*}
			Hence $\bigcup\limits_{\al \in A}U_{\al} \in f_*\MA$.
			\item Let $(U_{j})_{j=1}^n \subset f_*\MA$. Then for each $j \in {1, \ldots, n}$, $f^{-1}(U_{j}) \in \MA$. This implies that 
			\begin{align*}
			f^{-1}\bigg( \bigcap\limits_{j=1}^n U_{j} \bigg) 
			&=  \bigcap\limits_{j=1}^n f^{-1}(U_{j}) \\
			& \in \MA
			\end{align*}
			Hence $\bigcap\limits_{j=1}^n U_{j} \in f_*\MA$.
			\end{itemize}
			So $f_*\MA$ is a topology on $Y$.
			\item Similar to (1).
		\end{enumerate}
	\end{proof}	
	
	\begin{ex} \lex{ex:continuous_maps:0007}
	Let $(X,\MA)$ and $(Y,\MB)$ be topological spaces, $f:X \rightarrow Y$ and $\ME \subset \MP(Y)$. Suppose that $\MB = \tau(\ME)$. Then $f$ is continuous iff for each $B \in \ME$, $f^{-1}(B) \in \MA$.
	\end{ex}
	
	\begin{proof}
	Suppose that $f$ is continuous. Since $\ME \subset \MB$, clearly for each $B \in \ME$, $f^{-1}(B) \in \MA$. \\
	Conversely, suppose that for each $B \in \ME$, $f^{-1}(B) \in \MA$. Then $\ME \subset f_*\MA$. Since $f_*\MA$ is a topology on $Y$, we have that $\MB = \tau(\ME) \subset f_*\MA$. So $f$ is continuous.
	\end{proof}
	
	\begin{defn} \ld{def:continuous_maps:0008}
	Let $X$ be a set, $(Y_{\al}, \MT_{\al})_{\al \in A}$ a collection of topological spaces and $\MF \in \prod \limits_{\al \in A}Y_{\al}^X$ (i.e. $\MF = (f_{\al})_{\al \in A}$ where for each $\al \in A$, $f_{\al}:X \rightarrow Y_{\al}$). We define the \tbf{initial topology on $X$ generated by $\MF$}, denoted $\tau_X(\MF)$, by 
	\begin{align*}
	\tau_X(\MF) 
	&= \tau_X(\{f_{\al}^{-1}(B): B \in \MT_{\al} \text{ and } \al \in A \})
	\end{align*}	 
	\end{defn}

	\begin{note}
	The initial topology topology generated by $\MF$ is also called the \tbf{weak topology generated by $\MF$} and if $\MF = \{f\}$, then $\tau_X(\MF) = f^*\MB$.
	\end{note}

	\begin{ex} \lex{ex:continuous_maps:0008.1}
		Let $X$ be a set, $(Y_{\al}, \MT_{\al})_{\al \in A}$ a collection of topological spaces and $\MF \in \prod \limits_{\al \in A}Y_{\al}^X$ (i.e. $\MF = (f_{\al})_{\al \in A}$ where for each $\al \in A$, $f_{\al}:X \rightarrow Y_{\al}$). Then for each $\MT \subset \MP(X)$ if $\MT$ is a topology on $X$ and for each $\al \in A$, $f_{\al}$ is $(\MT, \MT_{\al})$-continuous, then $\tau_X(\MF) \subset \MT$.
	\end{ex}
	
	\begin{proof}
		Let $\MT \subset \MP(X)$. Suppose that $\MT$ is a topology on $X$ and for each $\al \in A$, $f_{\al}$ is $(\MT, \MT_{\al})$-continuous. Set $\MV \defeq \{f_{\al}^{-1}(V): \text{$V \in \MT_{\al}$ and $\al \in A$ }\}$. By definition, $\tau_X(\MF) = \tau_X(\MV)$. Since for each $\al \in A$, $f_{\al}$ is $(\MT, \MT_{\al})$-continuous, we have that for each $\al \in A$ and $V \in \MT_{\al}$, $f_{\al}^{-1}(V) \in \MT$. Hence $\MV \subset \MT$. Therefore
		\begin{align*}
			\tau_X(\MF)
			& = \tau_X(\MV) \\
			& \subset \MT.
		\end{align*}
	\end{proof}
	
	\begin{note}
	Essentially, $\tau_X(\MF)$ is the smallest topology on $X$ such that for each $\al \in A$, $f_{\al}:X \rightarrow Y_{\al}$ is continuous. 
	\end{note}

	\begin{ex} \lex{ex:continuous_maps:0009}
		Let $(Y_{\al}, \MT_{\al})_{\al \in A}$ be a a collection of topological spaces, $X$ a set, $(Z, \MC)$ a topological space, $\MF = (f_{\al})_{\al \in A} \in \prod \limits_{\al \in A}Y_{\al}^X$ and $g: Z \rightarrow X$. Then $g$ is $(\MC,\tau_X(\MF))$-continuous iff for each $\al \in A$, $f_{\al} \circ g$ is $(\MC, \MT_{\al})$-continuous:
		\[ \begin{tikzcd}
			Y_{\al}	
			& X  \arrow[l, "f_{\al}"'] \\
			& Z \arrow[ul, "g \circ f_{\al}"]  \arrow[u, "g"']
		\end{tikzcd}
		\]
	\end{ex}
	
	\begin{proof}
		If $g$ is $(\MC,\tau_X(\MF))$-continuous, then clearly for each $\al \in A$, $ f_{\al} \circ g$ is $(\MC, \MT_{\al})$- continuous. \\
		Conversely, suppose that for each $\al \in A$, $f_{\al} \circ g$ is $(\MC, \MT_{\al})$-continuous. Let $\al \in A$ and $V \in \MT_{\al}$. Continuity implies that,
		\begin{align*}
			g^{-1}(f_{\al}^{-1}(V)) 
			& = (f_{\al} \circ g)^{-1}(V) \\
			& \in \MC
		\end{align*}
		Since $\al \in A$ and $V \in \MT_{\al}$ are arbitrary, we have that for each $\al \in A$ and $V \in \MT_{\al}$, $g^{-1}(f_{\al}^{-1}(V)) \in \MC$. Since $\tau_X(\MF) = \tau(\{f_{\al}^{-1}(V): \al \in A \text{ and } V \in \MT_{\al})$, the previous exercise implies that $g$ is $(\MC,\tau_X(\MF))$-continuous.
	\end{proof}

	\begin{ex} \lex{ex:continuous_maps:0010}
		Let $(X, \MT)$ be a topological space. Set $\MF = \Hom_{\Top}((X, \MT), (X, \MT))$. Then $\tau_X(\MF) = \MT$.
	\end{ex}

	\begin{proof}
		Set $\ME = \{f^{-1}(V): V \in \MB \text{ and } f \in \MF \}$. Since for each $f \in \MF$, $f$ is $(\MT, \MT)$-continuous, $\ME \subset \MT$. \\
		Conversely, since $\id_X \in \MF$, we have that for each $U \in \MT$, 
		\begin{align*}
			U
			& = \id_X^{-1}(U) \\
			& \in \ME 
		\end{align*}
		So that $\MT \subset \ME$. Hence $\ME = \MT$ and 
		\begin{align*}
			\tau_X(\MF)
			& = \tau_X(\ME) \\
			& = \tau_X(\MT) \\
			& = \MT
		\end{align*}
	\end{proof}
	
	\begin{defn} \ld{def:continuous_maps:0011}
	Let $(X_{\al}, \MA_{\al})_{\al \in A}$ be a a collection of topological spaces, $Y$ a set and $\MF \in \prod \limits_{\al \in A}Y^{X^{\al}}$ (i.e. $\MF = (f_{\al})_{\al \in A}$ where for each $\al \in A$, $f_{\al}:X_{\al} \rightarrow Y$). We define the \tbf{final topology on $Y$ generated by $\MF$}, denoted $\tau_Y(\MF)$, by 
	\begin{align*}
	\tau_Y(\MF) 
	& \defeq \tau_Y(\{V \subset Y: \text{ for each $\al \in A$, $f_{\al}^{-1}(V) \in \MA_{\al}$}\})
\end{align*}	 
	\end{defn}
	
	\begin{note}
	If $\MF = \{f\}$, then $\tau_Y(\MF) = f_*\MA$.
	\end{note}

	\begin{ex} \lex{ex:continuous_maps:0011.1}
		Let $(X_{\al}, \MT_{\al})_{\al \in A}$ be a a collection of topological spaces, $Y$ a set and $\MF \in \prod \limits_{\al \in A}Y^{X^{\al}}$ (i.e. $\MF = (f_{\al})_{\al \in A}$ where for each $\al \in A$, $f_{\al}:X_{\al} \rightarrow Y$). Then for each $\MT \subset \MP(Y)$ if $\MT$ is a topology on $Y$ and for each $\al \in A$, $f_{\al}$ is $(\MT_{\al}, \MT)$-continuous, then $\MT \subset \tau_Y(\MF)$.
	\end{ex}

	\begin{proof}
		Let $\MT \subset \MP(Y)$. Suppose that $\MT$ is a topology on $Y$ and for each $\al \in A$, $f_{\al}$ is $(\MT_{\al}, \MT)$-continuous. Set $\MV \defeq \{V \subset Y: \text{ for each $\al \in A$, $f_{\al}^{-1}(V) \in \MT_{\al}$}\}$. By definition, $\tau_Y(\MF) = \tau_Y(\MV)$. Let $V \in \MT$. By assumption, for each $\al \in A$, $f_{\al}$ is $(\MT_{\al}, \MT)$-measurable. Thus for each $\al \in A$, $f_{\al}^{-1}(V) \in \MT_{\al}$. Therefore $V \in \MV$. Since $V \in \MT$ is arbitrary, we have that 
		\begin{align*}
			\MT 
			& \subset \MV \\
			& \subset \tau_Y(\MV) \\
			& = \tau_Y(\MF).
		\end{align*}
	\end{proof}
	
	\begin{note}
	Essentially, $\tau_X(\MF)$ is the largest topology on $X$ such that for each $\al \in A$, $f_{\al}:X_{\al} \rightarrow Y$ is continuous. 
	\end{note}
	
	\begin{ex} \lex{ex:continuous_maps:0012}
	Let $(X_{\al}, \MT_{\al})_{\al \in A}$ be a a collection of topological spaces, $Y$ a set, $(Z, \MC)$ a topological space, $\MF = (f_{\al})_{\al \in A} \in \prod \limits_{\al \in A}Y^{X_{\al}}$ and $g: Y \rightarrow Z$. Then $g$ is $(\tau_Y(\MF), \MC)$-continuous iff for each $\al \in A$, $g \circ f_{\al}$ is $(\MT_{\al}, \MC)$-continuous:
	\[ \begin{tikzcd}
	X_{\al} \arrow[r, "f_{\al}"] \arrow[dr, "g \circ f_{\al}"'] 	
	& Y  \arrow[d, "g"] \\
	& Z 
\end{tikzcd}
	\]
	\end{ex}
	
	\begin{proof}
	If $g$ is $(\tau_Y(\MF), \MC)$-continuous, then clearly for each $\al \in A$, $g \circ f_{\al}$ is $(\MT_{\al}, \MC)$-continuous. \\
	Conversely, suppose that for each $\al \in A$, $g \circ f_{\al}$ is $(\MT_{\al}, \MC)$-continuous. Let $\al \in A$ and $V \in \MC$. Continuity implies that 
	\begin{align*}
		f_{\al}^{-1}(g^{-1}(V)) 
		& = (g \circ f_{\al})^{-1}(V) \\
		& \in \MT_{\al}
	\end{align*}
	Since $\al \in A$ is arbitrary, we have that by definition, $g^{-1}(V) \in \tau_Y(\MF)$. Since $V \in \MC$ is arbitrary, $g$ is $(\tau_Y(\MF), \MC)$-continuous.
	\end{proof}
	
	\begin{defn} \ld{def:continuous_maps:0013}
		Let $(X,\MA)$ and $(Y,\MB)$ be topological spaces and $f:X \rightarrow Y$. Then 
		\begin{enumerate}
			\item $f$ is said to be \tbf{open} if for each $A \in \MA$, $f(A) \in \MB$.
			\item $f$ is said to be \tbf{closed} if for each $A \subset X$, if $A^c \in \MA$, then $f(A)^c \in \MB$. 
		\end{enumerate}
	\end{defn}

	\begin{ex} \lex{ex:continuous_maps:0014}
		Let $(X, \MT), (Y, \MS)$ be topological spaces, $\MB \subset \MT$ a basis for $\MT$ and $f: X \rightarrow Y$. Then $f$ is open iff for each $U \in \MB$, $f(U) \in \MS$.\\
		\tbf{Hint:} $f\bigg( \bigcup\limits_{\al \in A} A_{\al} \bigg) =  \bigcup\limits_{\al \in A} f(A_{\al})$.
	\end{ex}

	\begin{proof}
		Clearly if $f$ is open, then for each $U \in \MB$, $f(U) \in \MS$.\\
		Conversely, suppose that for each $U \in \MB$, $f(U) \in \MS$. Let $U \in \MT$. Then there exists $(U_{\al})_{\al \in A} \subset \MB$ such that $U =  \bigcup\limits_{\al \in A} U_{\al}$. Then 
		\begin{align*}
			f(U) 
			& = \bigcup\limits_{\al \in A} f(U_{\al}) \\
			& \in \MS
		\end{align*}
		Since $U \in \MT$ is arbitrary, $f$ is open.
	\end{proof}

	\begin{ex} \lex{ex:continuous_maps:0015}
		Let $(X, \MT_X)$, $(Y, \MT_Y)$ be topological spaces, $f:X \rightarrow Y$ and $\MB_X$ a basis for $\MT_X$.  Suppose that $f$ is surjective, continuous and open. Then $\{f(A): A \in \MB_X\}$ is a basis for $\MT_Y$.
	\end{ex}
	
	\begin{proof}
		Set $\MB_{Y} = \{f(A):A \in A \in \MT_X\}$. Since $f$ is open, $\MB_{Y} \subset \MT_Y$. Let $V \in \MT_Y$. Set $U = f^{-1}(V)$. Since $f$ is continuous, $U \in \MT_X$. Since $\MB_X$ is a basis for $\MT_X$, there exist $\MB_X' \subset \MB_X$ such that $U = \bigcup\limits_{A \in \MB_X'}A$. Define $\MB_Y' \subset \MB_Y$ by $\MB_Y' = \{f(A): A \in \MB_X'\}$. Since $f$ is surjective, we have that $f(f^{-1}(V)) = f(V)$ and therefore
		\begin{align*}
			V
			& = f(f^{-1}(V)) \\
			& = f(U) \\
			& = f \bigg( \bigcup_{A \in \MB_X'} B \bigg) \\
			& = \bigcup_{A \in \MB_X'} f(A) \\
			& = \bigcup_{B \in \MB_Y'} B \\
		\end{align*} 
		Since $V \in \MT_Y$ is arbitrary, we have that for each $V \in \MT_Y$, there exists $\MB_Y' \subset \MB_Y$ such that $V = \bigcup\limits_{B \in \MB_Y'} B$. Thus $\MB_Y$ is a basis for $\MT_Y$.
	\end{proof}
	
	\begin{ex} \lex{ex:continuous_maps:0016} \tbf{Doob-Dynkin Lemma:} \\
	Let $(X_1, \MT_1)$, $(X_2, \MT_2)$ and $(X_3, \MT_3)$ be topological spaces and $f: X_1 \rightarrow X_2$ and $g:X_1 \rightarrow X_3$. Suppose that $f$ is surjective and $\MT_1$-$\MT_2$ continuous and $g$ is $\MT_1$-$\MT_3$ continuous and $(X_3, \MT_3)$ is a $T_1$ space. Then $g$ is $f^*\MT_2$-$\MT_3$ continuous iff there exists a unique $\phi: X_2 \rightarrow X_3$ such that $\phi$ is $\MT_2$-$\MT_3$ continuous and $g = \phi \circ f$. \\
	\tbf{Hint:} For each $t \in X_3$, set $A_t = g^{-1}(\{t\}) \in \MF_{(f^* \MT_2)}$ and choose $B_t \in \MT_2$ such that $A_t = f^{-1}(B_t)$. Set $\phi(y) = t$ for $y \in B_t \cap f(X_1)$ and $t \in g(X_1)$.
	\end{ex}
	
	\begin{proof}
	Suppose that there exists a unique $\phi: X_2 \rightarrow X_3$ such that $\phi$ is $\MT_2$ - $\MT_3$ measurable and $g = \phi \circ f$. Since $f$ is $f^* \MT_2$ - $\MT_2$ continuous, we have that $g = \phi \circ f$ is $f^*\MT_2$-$\MT_3$ continuous.  \\
	Conversely, suppose that $g$ is $f^*\MT_2$-$\MT_3$ continuous. \\
	\begin{itemize}
	\item \tbf{(Existence)} \\
	For each $t \in X_3$, set $A_t = g^{-1}(\{t\})$ Since $(X_3, \MT_3)$ is a $T_1$ space, for each $t \in X_3$, $A_t \in \MF_{f^*\MT_2}$ and thus, there exists $B_t \in \MF_{\MT_2}$ such that $A_t = f^{-1}(B_t)$. \\
	Note that 
	\begin{itemize}
	\item for each $t \in g(X_1)$, there exists $x \in A_t$ such that $g(x) = t$. Hence $f(x) \in B_t$.\\
	\item for $t_1, t_2 \in g(X_1)$, $t_1 \neq t_2$ implies that
	\begin{align*}
	f^{-1}(B_{t_1} \cap B_{t_2}) 
	&= A_{t_1} \cap A_{t_2} \\
	&= g^{-1}(\{t_1\} \cap \{t_2\}) \\
	&= \varnothing
	\end{align*}	 
	and since $f$ is surjective, 
	\begin{align*}
	B_{t_1} \cap  B_{t_2} 
	& = f(f^{-1}(B_{t_1} \cap  B_{t_2} )) \\
	&= f(\varnothing) \\
	&= \varnothing
	\end{align*}
	\item we have that 
	\begin{align*}
	f^{-1} \bigg( \bigcup_{t \in g(X_1)} B_t\bigg) 
	&=  \bigcup_{t \in g(X_1)} A_t \\
	&= \bigcup_{t \in g(X_1)} g^{-1}(\{t\}) \\
	&= g^{-1}(g(X_1)) \\
	&= X_1
	\end{align*}
	Since $f$ is surjective, we have that 
	\begin{align*}
	X_2
	&= f(X_1) \\
	&= f \bigg( f^{-1} \bigg( \bigcup_{t \in g(X_1)} B_t\bigg)  \bigg) \\
	&= \bigcup_{t \in g(X_1)} B_t
	\end{align*}
	\end{itemize}
	Therefore, 
	\begin{itemize}
	\item for each $t \in g(X_1)$, $B_t \neq \varnothing$
	\item $(A_t)_{t \in g(X_1)}$ is a partion of $X_1$
	\item $(B_t)_{t \in g(X_1)}$ is a partition of $X_2$\\
\end{itemize}		
	 Define $\phi:X_2 \rightarrow X_3$ by $\phi(y) = t$ for $t \in g(X_1)$ and $y \in B_t $. Then the previous observations imply that $\phi$ is well defined and $\phi(X_2) = g(X_1)$. Since for each $t \in g(X_1)$ and $x \in A_t$, $f(x) \in B_t$ and $g(x) = t$, we have that $\phi \circ f (x) = t = g(x)$. So $\phi \circ f = g$. \\ \\
	To show that $\phi$ is continuous, let $C \in \MT_3$. Choose $B \in \MT_2$ such that $g^{-1}(C) = f^{-1}(B)$.
	Let $y \in \phi^{-1}(C) \subset X_2$. Set $t = \phi(y) \in C$ and choose $x \in X_1$ such that $y = f(x)$. Since 
	\begin{align*}
	g(x) 
	&= \phi \circ f (x) \\
	&= \phi(y) \\
	&= t \\
	&\in C
\end{align*}		
	 $x \in g^{-1}(C) = f^{-1}(B)$. Therefore, $y = f(x) \in B$. So $\phi^{-1}(C) \subset B$. \\
	Let $y \in B$. Choose $x \in X_1$ such that $f(x) = y$. Then $x \in f^{-1}(B) = g^{-1}(C)$. So 
	\begin{align*}
	\phi(y) 
	&= \phi \circ f (x) \\
	&= g(x) \\
	&\in C
	\end{align*}	 
	and $y \in \phi^{-1}(C)$. So $B \subset \phi^{-1}(C)$. 
	Hence $\phi^{-1}(C) = B \in \MT_2$ and $\phi$ is $\MT_2$ - $\MT_3$ continuous.\\
	\item \tbf{(Uniqueness)} \\
	Let $\psi: X_2 \rightarrow X_3$. Suppose that $\psi$ is $\MT_2$-$\MT_3$ continuous and $g = \psi \circ f$. Let $y \in X_2$. Then there exists $x \in X_1$ such that $y = f(x)$. Then 
	\begin{align*}
	\psi(y) 
	&= \psi \circ f(x) \\
	&= g(x) \\
	&= \phi \circ f(x) \\
	&= \phi(y)
	\end{align*}
	So $\psi = \phi$.
	\end{itemize}
 
	\end{proof}

	\begin{ex} \lex{ex:continuous_maps:0017}
	Let $(X_1, \MT_1)$, $(X_2, \MT_2)$ and $(X_3, \MT_3)$ be topological spaces and $f: X_1 \rightarrow X_2$ and $g:X_1 \rightarrow X_3$. Suppose that $f$ is $\MT_1$-$\MT_2$ continuous and $g$ is $\MT_1$-$\MT_3$ continuous and $(X_3, \MT_3)$ is a $T_1$ space. Then $g$ is $f^*\MT_2$-$\MT_3$ continuous iff there exists a unique $\phi: f(X_1) \rightarrow X_3$ such that $\phi$ is $\MT_2 \cap f(X_1)$ - $\MT_3$ continuous and $g = \phi \circ f$. \\
	\end{ex}
	
	\begin{proof}
	A previous exercise implies that $f: X_1 \rightarrow f(X_1)$ is $\MT_1$ - $\MT_2 \cap f(X_1)$ continuous. Now apply the previous exercise. 
	\end{proof}
	
	\begin{defn} \ld{def:continuous_maps:0018}
		Let $X$ be a topological space, $x_0 \in X$ and $f:X \rightarrow \R$. We define the \tbf{limit inferior of $f$ as $x \rightarrow x_0$ (resp. limit inferior of $f$ as $x \rightarrow x_0$)}, denoted $\liminf\limits_{x \rightarrow x_0}f(x)$ (resp. $\liminf\limits_{x \rightarrow x_0}f(x)$), by 
		$$\liminf_{x \rightarrow x_0} f(x) = \sup_{V \in \MN(x_0)} \inf_{x \in V \setminus \{x_0\}} f(x)$$
		resp. 
		$$\limsup_{x \rightarrow x_0} f(x) = \inf_{V \in \MN(x_0)} \sup_{x \in V \setminus \{x_0\}} f(x)$$
	\end{defn}

	\begin{ex} \lex{ex:continuous_maps:0019}
		Let $X$ be a topological space, $x_0 \in X$ and $f:X \rightarrow \R$. Then $f$ is continuous at $x_0$ iff $\liminf\limits_{x \rightarrow x_0}f(x) = \limsup\limits_{x \rightarrow x_0}f(x) = f(x_0)$ 
	\end{ex}

	\begin{proof}
		Suppose that \\
		\tbf{FINISH!!!}
	\end{proof}


	





































\newpage
\section{Nets}	

\subsection{Common Directed Sets}

	\begin{note}
		We recall the definition of a directed set from \rd{def:orderings:directed_sets:0001}.
	\end{note}
	
	\begin{defn} \ld{def:nets:0002}
	Let $X$ be a set. Define the \tbf{reverse inclusion ordering} on $\MN(x)$, denoted $\leq$, by $U \leq V$ iff $V \subset U$. 
	\end{defn}
	
	\begin{ex} \lex{ex:nets:0003}
	Let $X$ be a topological space and $x \in X$. Then $\MN(x)$ ordered by reverse inclusion is a directed set.
	\end{ex}
	
	\begin{proof}\
	\begin{enumerate}
	\item Clearly, for each $U \in \MN(x), U \leq U$.
	\item Let $U,V,W \in \MN(x)$. Suppose that $U \leq V$ and $V \leq W$. Then $W \subset V \subset U$ which implies that $W \subset U$ and hence $U \leq W$.
	\item Let $U,V \in \MN(x)$. Set $W = U \cap V$. Then $W \in \MN(x)$ and $U,V \leq W$. 
	\end{enumerate}
	So $\MN(x)$ is a directed set. 
	\end{proof}

	\begin{defn}
		Let $(A, \leq)$ be a directed set and $\al_0 \in A$. We define $[\al_0, \infty) \subset A$, by $[\al_0, \infty) \defeq \{\al \in A: \al \geq \al_0\}$ and $\leq_{[\al_0, \infty)} \defeq \leq \cap ([\al_0, \infty) \times [\al_0, \infty))$.
	\end{defn}

	\begin{ex} \lex{ex:nets:0003.1}
		Let $(A, \leq)$ be a directed set and $\al_0 \in A$. Then $([\al_0, \infty), \leq_{[\al_0, \infty)})$ is a directed set. 
	\end{ex}

	\begin{proof} Set $B \defeq [\al_0, \infty)$.
		\begin{enumerate}
			\item Let $\al \in B$. Since $B \subset A$, $\al \in A$. Since $(A, \leq)$ is a directed set, $\al \leq \al$. Since $\al \in B$, we have that $\al \leq_B \al$. Since $\al \in B$ is arbitrary, we have that for each $\al \in B$, $\al \leq_B \al$. 
			\item Let $\al, \be, \gam \in B$. Suppose that $\al \leq_B \be$ and $\be \leq \gam$. Since $\leq_B \subset \leq$, we have that $\al \leq \be$ and $\be \leq \gam$. Since $(A, \leq)$ is a directed set, $\al \leq \gam$. Since $\al, \gam \in B$, we have that $\al \leq_B \gam$. Since $\al, \be, \gam \in B$ are arbitrary, we have that for each $\al, \be, \gam \in B$, $\al \leq_B \be$ and $\be \leq_B \gam$ implies that $\al \leq_B \gam$.
			\item Let $\al, \be \in B$. Since $B \subset A$, there exists $\gam \in A$ such that $\al, \be \leq \gam$. Since $\al \in B$, $\al \geq \al_0$. Since $\gam \geq \al$, we have that $\gam \geq \al_0$. Hence $\gam \in B$. Since $\al, \be, \gam \in B$, we have that $\al, \be \leq_B \gam$. Since $\al,\be \in B$ are arbitrary, we have that for each $\al, \be \in B$, there exists $\gam \in B$ such that $\al, \be \leq_B \gam$.  
			\item Since $\al_0 \in B$, $B \neq \varnothing$.
		\end{enumerate}
		So $(B, \leq_B)$ is a directed set. 
	\end{proof}

	\begin{defn} \ld{def:nets:0004} \tcr{move to metric space chapter}
		Let $X$ be a metric space and $x_0 \in X$. Define the \tbf{reverse distance from $x_0$ ordering} on $X \setminus \{x_0\}$, denoted $\leq_{x_0}$, by $x \leq_{x_0} y$ iff $d(x, x_0) \geq d(y, x_0)$.
	\end{defn}

	\begin{ex} \lex{ex:nets:0005} \tcr{move to metrix space chapter}
		 Let $X$ be a metric space and $x_0 \in X$. Then $(X \setminus \{x_0\}, \leq_{x_0})$ is a directed set. 
	\end{ex}

	\begin{proof}\
		\begin{enumerate}
			\item Let $x \in X \setminus \{x_0\}$. Since $d(x, x_0) \geq d(x, x_0)$, $x \leq_{x_0} x$.
			\item Let $x, y, z \in X \setminus \{x_0\}$. Suppose that $x \leq_{x_0} y$ and $y \leq_{x_0} z$. Then $d(x, x_0) \geq d(y, x_0)$ and $d(y, x_0) \geq d(z, x_0)$. Hence $d(x, x_0) \geq d(z, x_0)$ so that $x \leq z$.
			\item Let $x,y \in X \setminus \{x_0\}$. Set 
			\begin{align*}
				z 
				&= \argmin\limits_{a \in \{x, y\}} d(a, x_0) \\ 
				& \in X \setminus \{x_0\} 
			\end{align*}
			Then $x, y \leq_{x_0} z$.
		\end{enumerate}
	\end{proof}


	
	
	
	
	
	
	
	
	
	
	
	
	
	
	
	
	
	
	
	
	
	
	
	
	
	
	
	\subsection{Nets in Topological Spaces}
	
	\tcr{maybe move the basic definition of net to the set theory section and also introduce sequences there}
	
	\begin{defn} \ld{def:nets:0009}
	Let $X$ be a topological space, $(x_{\al})_{\al \in A} \subset X$ a net and $U \subset X$.
	Then $(x_{\al})_{\al \in A}$ is said to be 
	\begin{itemize}
	\item \tbf{eventually in $U$} if there exists $\be \in A$ such that for each $\al \in A$ $\al \geq \be$ implies that $x_{\al} \in U$
	\item \tbf{frequently in $U$}  if for each $\al \in A$, there exists $\be \in A$ such that $\be \geq \al $ and $x_{\be} \in U$
	\end{itemize}
	\end{defn}

	\begin{ex} \lex{ex:nets:0009.1}
		Let $X$ be a topological space, $(x_{\al})_{\al \in A} \subset X$ a net and $U \subset X$. Then $(x_{\al})_{\al \in A}$ is eventually in $U$ iff there exists $\al_0 \in A$ such that $(x_{\al \in [\al_0, \infty)})_{\al \in A} \subset U$.
	\end{ex}

	\begin{proof}\
		\begin{itemize}
			\item $(\implies):$ \\
			Suppose that $(x_{\al})_{\al \in A}$ is eventually in $U$. Then there exists $\al_0 \in A$ such that for each $\al \in A$, $\al \geq \al_0$ implies that $x_{\al} \in U$. Then $(x_{\al})_{\al \in [\al_0, \infty)} \subset U$.
			\item $(\impliedby):$ \\
			Suppose that there exists $\al_0 \in A$ such that $(x_{\al})_{\al \in [\al_0, \infty)} \subset U$. Then for each $\al \in A$, $\al \geq \al_0$ implies that $x_{\al} \in U$. Hence $(x_{\al})_{\al \in A}$ is eventually in $U$.
		\end{itemize}
	\end{proof}

	\begin{ex}
		\tcr{(make exercise about the tail net and being frequently in $U$)}
	\end{ex}
	
	\begin{proof}
		
	\end{proof}
	
	\begin{defn} \ld{def:nets:0010}
	Let $X$ be a topological space, $(x_{\al})_{\al \in A} \subset X$ a net and $x \in X$. Then $(x_{\al})_{\al \in A}$ is said to \tbf{converge to $x$}, denoted $x_{\al} \rightarrow x$, if for each $U \in \MN(x)$, $(x_{\al})_{\al \in A}$ is eventually in $U$. 
	\end{defn}	
	
	\begin{defn} \ld{def:nets:0011}
	Let $X$ be a topological space and $(x_{\al})_{\al \in A} \subset X$ a net. Then $(x_{\al})_{\al \in A}$ is said to \tbf{converge} if there exists $x \in X$ such that $x_{\al} \rightarrow x$. 
	\end{defn}	

	\begin{ex} \lex{ex:nets:0012}
		Let $X$ be a metric space and $x_0 \in X$. Set $A = X \setminus \{x_0\}$. Order $A$ by reverse distance from $x_0$. Define $(x_{\al})_{\al \in A} \subset X$ by $x_{\al} = \al$. Then $x_{\al} \rightarrow x_0$.
	\end{ex}

	\begin{proof}
		Let $U \in \MN(x_0)$. Since $x_0 \in \Int U$, there exists $\del > 0$ such that $B(x_0, \del) \subset \Int U$. Choose $\be \in B^*(x_0, \del)$. Let $\al \in A$. Suppose that $\al \geq \be$. Then $d(\al, x_0) \leq d(\be, x_0) < \del$. Hence  
		\begin{align*}
			x_{\al} 
			&= \al  \\
			& \in B^*(x_0, \del) \\
			&\subset U
		\end{align*}
		Since $U \in \MN(x_0)$ is arbitrary, $x_{\al} \rightarrow x_0 $
	\end{proof}
	
	\begin{ex} \lex{ex:nets:0013}
	Let $X$ be a topological space, $S \subset X$ and $x \in X$. Then $x \in S'$ iff there exists a net $(x_{\al})_{\al \in A} \subset S \setminus \{x\}$ such that $x_{\al} \rightarrow x$. 
	\end{ex}

	\begin{proof}
	Suppose that $x \in S'$. Set $A = \MN(x)$, ordered by reverse inclusion.  Since $x \in S'$, for each $\al \in A$, there exists $x_\al \in (\al \setminus \{x\}) \cap S.$ Then $(x_{\al})_{\al \in A} \subset S \setminus \{x\}$. Let $V \in \MN(x)$. Choose $\be = V$. Let $\al \in \MN(x)$. Suppose that $\al \geq \be$. Then 
	\begin{align*}
	x_{\al} 
	&\in (\al \setminus \{x\}) \cap S \\
	& \subset \al \\
	& \subset \be \\
	&= V
\end{align*}	
	So $(x_{\al})_{\al \in \MN(x)}$ is eventually in $V$. Since $V \in \MN(x)$ is arbitrary, $x_{\al } \rightarrow x$. \\
	Conversely, suppose that there exists a net $(x_{\al})_{\al \in A} \subset S \setminus \{x\}$ such that $x_{\al} \rightarrow x$. Let $U \in \MN(x)$. Since $(x_{\al})_{\al \in A}$ is eventually in $U$, there exits $\be \in A$ such that $x_{\be} \in U$. Then $x_{\be} \in (U \setminus \{x\}) \cap S$ and $(U \setminus \{x\}) \cap S \neq \varnothing$. Since $U \in \MN(x)$ is arbitrary, $x \in S'$.
	\end{proof}
	
	\begin{ex}  \lex{ex:nets:0014}
	Let $X$ be a topological space, $S \subset X$ and $x \in X$. Then $x \in \cl S$ iff there exists a net $(x_{\al})_{\al \in A} \subset S$ such that $x_{\al} \rightarrow x$. 
	\end{ex}

	\begin{proof}
	Suppose that $x \in \cl S$. Since $\cl S = S \cup S'$, $x \in S$ or $x \in S'$. If $x \in S$, define $(x_n)_{n \in \N} \subset S$ by $x_n = x$. Then $x_n \rightarrow x$. If $x \in S'$, the previous exercise implies that there exists a net $(x_{\al})_{\al \in A} \subset S \setminus \{x\} \subset S$ such that $x_{\al} \rightarrow x$. 
	\end{proof}
	
	\begin{ex} \lex{ex:nets:0016}
		Let $X$ be a topological space and $E \subset X$. Then 
		\begin{enumerate}
			\item $\p E = \cl E \cap \cl E^c$ 
			\item $\p E = \cl E \setminus \Int E$
		\end{enumerate}
	\end{ex}
	
	\begin{proof}\
		\begin{enumerate}
			\item Let $x \in \p E$. Then for each $U \in \MN(x)$, $U \cap E \neq \varnothing$ and $U \cap E^c \neq \varnothing$. The axiom of choice implies that there exist nets $(a_U)_{U \in \MN(x)} \subset E$ $(b_U)_{U \in \MN(x)} \subset E^c$ such that for each $U \in \MN(x)$, $a_U, b_U \in U$. Then $a_U, b_U \rightarrow x$. Hence $x \in (\cl E) \cap (\cl E^c)$. Since $x \in \p E$ is arbitrary, we have that $\p E \subset (\cl E) \cap (\cl E^c)$. \\
			Conversely, let $x \in (\cl E) \cap (\cl E^c)$. Then there exists $(a_{\al})_{\al \in A} \subset E$ and $(b_{\be})_{\be \in B} \subset E^c$ such that $a_{\al} \rightarrow x$ and $b_{\be} \rightarrow x$. Let $U \in \MN(x)$. Then there exists $\al_0 \in A$ and $\be_0 \in B$ such that for each $\al \in A$ and $\be \in B$, $\al \geq \al_0$ implies that $a_{\al} \in A$ and $\be \geq \be_0$ implies that $b_{\be} \in U$. In particular, $a_{\al_0} \in U \cap E$ and $b_{\be_0} \in U \cap E^c$. Hence $U \cap E \neq \varnothing$ and $U \cap E^c \neq \varnothing$. Since $U \in \MN(x)$ is arbitrary, we have that for each $U \in \MN(x)$, $U \cap E \neq \varnothing$ and $U \cap E^c \neq \varnothing$. Thus $x \in \p E$. Since $x \in (\cl E) \cap (\cl E^c)$ is arbitrary, $(\cl E) \cap (\cl E^c) \subset \p E$. \\
			Therefore $\p E = (\cl E) \cap (\cl E^c)$.
			\item \rex{31015.1} and part $(1)$ implies that 
			\begin{align*}
				\p E
				& = (\cl E) \cap (\cl E^c) \\
				& = (\cl E) \cap (\Int E)^c \\
				& = (\cl E) \setminus \Int E
			\end{align*}
		\end{enumerate}
	\end{proof}
	

	\begin{ex} \lex{ex:nets:0017} \tbf{Topology in Terms of Nets: } \\
		Let $X$ be a topological space and $U \subset X$. Then $U$ is open iff for each net $(x_{\al})_{\al \in A} \subset X$ and $x \in U$, $x_{\al} \rightarrow x$ implies that $(x_{\al})_{\al \in A} $ is eventually in $U$.
	\end{ex}

	\begin{proof}
		Suppose that $U$ is open. Let $(x_{\al})_{\al \in A} \subset X$ be a net and $x \in U$. Suppose that $x_{\al} \rightarrow x$. Since $U \in \MN(x)$, $(x_{\al})_{\al \in A}$ is eventually in $U$. \\
		Conversely, suppose that for each net $(x_{\al})_{\al \in A} \subset X$ and $x \in U$, $x_{\al} \rightarrow x$ implies that $(x_{\al})_{\al \in A} $ is eventually in $U$. For the sake of contradiction, suppose that $U^c$ is not closed. Then there exists $x \in \cl U^c$ such that $x \not \in U^c$. Thus $x \in U$. Since $x \in \cl U^c$, a previous exercise implies that there exists a net $(x_{\al})_{\al \in A} \subset U^c$ such that $x_{\al} \rightarrow x$. By assumption, $(x_{\al})_{\al \in A}$ is eventually in $U$. This is a contradiction since $(x_{\al})_{\al \in A} \subset U^c$. Hence $U^c$ is closed and hence $U$ is open. 
	\end{proof}

	\begin{ex} \lex{ex:nets:0018}
		Let $X$ be a topological space, $U \in \MT$ and $E \subset X$. If $U \cap \cl E \neq \varnothing$, then $U \cap E \neq \varnothing$. 
	\end{ex}

	\begin{proof}
		Suppose that $U \cap \cl E \neq \varnothing$. Then there exists $x \in X$ such that $x \in U \cap \cl E$. Since $x \in \cl E$, there exists a net $(x_{\al})_{\al \in A} \subset E$ such that $x_{\al} \rightarrow x$. Since $U \in \MN(x)$, $(x_{\al})_{\al \in A}$ is eventually in $U$. Thus there exists $\al_0 \in A$ such that for each $\al \geq \al_0$, $x_{\al} \in U$. In particular $x_{\al_0} \in U \cap E$. Hence $U \cap E \neq \varnothing$. 
	\end{proof}
	
	\begin{ex} \lex{ex:nets:0019}
	Let $(X,\MA)$ and $(Y,\MB)$ be topological spaces, $f:X \rightarrow Y$ and $x \in X$. Then $f$ is continuous at $x$ iff for each net $(x_{\al})_{\al \in A} \subset X$, $x_{\al} \rightarrow x$ implies that $f(x_{\al}) \rightarrow f(x)$. 
	\end{ex}
	
	\begin{proof}
	Suppose that $f$ is continuous at $x$. Let $(x_{\al})_{\al \in A} \subset X$ be a net. Suppose that $x_{\al} \rightarrow x$. Let $V \in \MN(f(x))$. Continuity implies that $f^{-1}(V) \in \MN(x)$. Since  $x_{\al} \rightarrow x$, $(x_{\al})_{\al \in A}$ is eventually in $f^{-1}(V)$. So there exists $\be \in A$ such that for each $\al \in A$, $\al \geq \be$ implies that $x_{\al} \in f^{-1}(V)$. Let $\al \in A$. Suppose that $\al \geq \be$. Then $f(x_{\al}) \in V$. So $(f(x_{\al}))_{\al \in A}$ is eventually in $V$. Since $V \in \MN(f(x))$ is arbitrary, $f(x_{\al}) \rightarrow f(x)$.\\
	Conversely, suppose that $f$ is not continuous at $x$. Then there exists $V \in \MN(f(x))$ such that $f^{-1}(V) \not \in \MN(x)$. Then $x \not \in \Int (f^{-1}(V))$. So $x \in (\Int (f^{-1}(V)))^c = \cl f^{-1}(V^c)$. This implies that there exists a net $(x_{\al})_{\al \in A} \subset f^{-1}(V^c)$ such that $x_{\al} \rightarrow x$. Since for each $\al \in A$, $f(x_{\al}) \in V^c$, $f(x_{\al})$ is not eventually in $V$. So $f(x_{\al}) \not \rightarrow f(x)$. 
\end{proof}		
	
	\begin{ex} \lex{ex:nets:0020}
	Let $(Y_{\al}, \MB_{\al})_{\al \in A}$ be a collection of topological spaces, $X$ a set and $\MF \in \prod \limits_{\al \in A}Y_{\al}^X$ with $\MF = (f_{\al})_{\al \in A}$. Equip $X$ with $\tau_X(\MF)$. Let $(x_{\gam})_{\gam \in \Gam} \subset X$ be a net and $x \in X$. Then $x_{\gam} \rightarrow x$ iff for each $\al \in A$, $f_{\al}(x_{\gam}) \rightarrow f_{\al}(x)$.  \tcr{(maybe reword without $\MF$ and similar instances elsewhere)}
	\end{ex}
	
	\begin{proof}
	Suppose that $x_{\gam} \rightarrow x$. Let $\al \in A$. Since $f_{\al}$ is continuous, the previous exercise implies that $f_{\al}(x_{\gam}) \rightarrow f_{\al}(x)$. \\
	Conversely, Suppose that for each $\al \in A$,  $f_{\al}(x_{\gam}) \rightarrow f_{\al}(x)$. Let $U \in \MN(x)$. Since $\Int U \in \tau_X(\MF)$, \rex{31010} implies there exist $V_1 \in \MB_{\al_1}, \ldots, V_n \in \MB_{\al_n}$ such that $\bigcap\limits_{j=1}^n f_{\al_j}^{-1}(V_j) \subset \Int U$ and $x \in \bigcap\limits_{j=1}^n f_{\al_j}^{-1}(V_j)$. Let $j \in \{1, \ldots, n\}$. Since $f_{\al_j}^{-1}(V_j) \in \MN(x)$, $V_j \in \MN(f(x))$. By assumption, $f_{\al_j}(x_{\gam})$ is eventually in $V_j$. Thus there exist there exist $\gam'_j \in \Gam$ such that for each $\gam \geq \gam'_j$, $f_{\al_j}(x_{\gam}) \in V_j$, or equivalently, $x_{\gam} \in f_{\al_j}^{-1}(V_j)$. Since $\Gam$ is directed, there exists $\gam' \in \Gam$ such that for each $j \in \{1, \ldots, n\}$, $\gam' \geq \gam'_j$. Let $\gam \in \Gam$. Suppose that $\gam \geq \gam'$. Then 
	\begin{align*}
	x_{\gam} 
	& \in \bigcap\limits_{j=1}^n f_{\al_j}^{-1}(V_j) \\
	& \subset \Int U \\
	& \subset U
\end{align*}	
	So $(x_\gam)_{\gam \in \Gam}$ is eventually in $U$. Since $U \in \MN(x)$ is arbitrary, $x_{\gam} \rightarrow x$.  
	\end{proof}
	
	
	\begin{ex} \lex{ex:nets:0021}
	\tcr{reorganize to an iff}\\
	Let $X$ be a set and $\MT_1$, $\MT_2$ topologies on $X$. Then the following are equivalent:
	\begin{enumerate}
		\item $\MT_1 = \MT_2$
		\item for each net $(x_{\al})_{\al \in A} \subset X$ and $x \in X$, $x_{\al} \rightarrow x$ in $\MT_1$ iff $x_{\al} \rightarrow x$ in $\MT_2$.
	\end{enumerate}
	\end{ex}

	\begin{proof}\
		\begin{itemize}
			\item $(1) \implies (2)$: \\
			Clear. \\
			\item $(2) \implies (1)$: \\
			Let $U \in \MT_1$ and $x \in U^c$. Since $U^c$ is closed in $\MT_1$, there exists a net $(x_{\al})_{\al \in A} \subset U^c$ such that $x_{\al} \rightarrow x$ in $\MT_1$. By assumption, $x_{\al} \rightarrow x$ in $\MT_2$. So $U^c$ is closed in $\MT_2$ and $U \in \MT_2$. Hence $\MT_1 \subset \MT_2$. \\
			Similarly, $\MT_2 \subset \MT_1$.
		\end{itemize}
	\end{proof}
	
	\begin{ex} \lex{ex:nets:0022}
		Let $X, Y$ be topological spaces and $\phi: X \rightarrow Y$ a homeomorphism. Then for each $E \subset X$, 
		\begin{enumerate}
			\item $\cl \phi(E) = \phi(\cl E)$  
			\item $\Int \phi(E) = \phi( \Int E)$  
		\end{enumerate} 
	\end{ex}
	
	\begin{proof}\
		\begin{enumerate}
			\item Let $E \subset X$. Since $E \subset \cl E$, we have that $\phi(E) \subset \phi(\cl E)$. Since $\cl E$ is closed, $\phi(\cl E)$ is closed and thus $\cl \phi(E) \subset \phi(\cl E)$. Conversely, let $x \in \phi(\cl E)$. Then $\phi^{-1}(x) \in \cl E$. Then there exists a net $( y_{\al} )_{\al \in A} \subset E$ such that $y_{\al} \rightarrow \phi^{-1}(x)$. Then $( \phi(y_{\al}) )_{\al \in A } \subset \phi(E)$ and $\phi(y_{\al}) \rightarrow x$. Thus $x \in \cl \phi(E)$ and $\phi(\cl E) \subset \cl \phi(E)$.
			\item Similar
		\end{enumerate} 
	\end{proof}

	\begin{defn} \ld{def:nets:0024}
		Let $X$ be a topological space, $(x_{\al})_{\al \in A} \subset X$ a net and $x \in X$. Then $x$ is said to be a \tbf{cluster point} or \tbf{accumulation point} of $(x_{\al})_{\al \in A}$ if for each $U \in \MN(x)$, $(x_{\al})_{\al \in A}$ is frequently in $U$.
	\end{defn}

	\begin{note}
		We recall the definition of a subnet from \rd{def:set_theory:nets:0003}
	\end{note}
	
	\begin{ex} \lex{ex:nets:0026}
	Let $X$ be a topological space, $(x_{\al})_{\al \in A} \subset X$ a net and $x \in X$. Then the following are equivalent: 
	\begin{enumerate}
		\item $x$ is a cluster point of $(x_{\al})_{\al \in A}$
		\item there exists a subnet $(x_{\al_{\be}})_{\be \in B}$ of $(x_{\al})_{\al \in A}$ such that $x_{\al_{\be}} \rightarrow x$
		\item $x \in \bigcap\limits_{\al \in A} \cl \{x_{\be}: \be \geq \al\}$
	\end{enumerate}
	\tbf{Hint:} Order $\MN(x)$ by reverse inclusion and consider the product directed set $B = A \times \MN(x)$. If $x$ is a cluster point of $(x_{\al})_{\al \in A}$, then for each $\be = (\gam, U) \in B$, there exists $\al_{\be} \in A$ such that $\al_{\be} \geq \gam$ and $\al_{\be} \in U$. 
	\end{ex}

	\begin{proof}\
		\begin{itemize}
			\item $(1) \implies (2)$: \\
			Suppose that $x$ is a cluster point of $(x_{\al})_{\al \in A}$. Set $B = A \times \MN(x)$. Since $x$ is a cluster point of $(x_{\al})_{\al \in A}$, for each $(\gam, U) \in B$, there exists $\al_{(\gam, U)} \in A$ such that $\al_{(\gam, U)} \geq \gam$ and $x_{\al_{(\gam, U)}} \in U$. Let $\al_0 \in A$. Choose $\be_0 = (\al_0, X) \in B$. Let $\be = (\gam, U) \in B$. Suppose that $\be \geq \be_0$. Then $\gam \geq \al_0$ and 
			\begin{align*}
				\al_{\be}
				&= \al_{(\gam, U)} \\
				& \geq \gam \\
				& \geq \al_0
			\end{align*}
			So that $(x_{\al_{\be}})_{\be \in B}$ is a subnet of $(x_{\al})_{\al \in A}$. Let $U_0 \in \MN(x)$. Choose $\al_0 \in A$ and set $\be_0 = (\al_0, U_0)$. Let $\be = (\gam, U) \in B$. Suppose that $\be \geq \be_0$. Then 
			\begin{align*}
				x_{\al_{\be}} 
				&= x_{\al_{(\gam, U)}} \\
				& \in U \\
				&\subset U_0
			\end{align*} 
			Since $U_0 \in \MN(x)$ is arbitrary, $x_{\al_{\be}} \rightarrow x$.
			\item $(2) \implies (3)$: \\
			Suppose that that there exists a subnet $(x_{\al_{\be}})_{\be \in B}$ of $(x_{\al})_{\al \in A}$ such that $x_{\al_{\be}} \rightarrow x$. Let $\al \in A$. Then there exists $\be_0 \in B$ such that for each $\be \in B$, $\be \geq \be_0$ implies that $\al_{\be} \geq \al$. Therefore, for each $\be \in B$, $\be \geq \be_0$ implies that $x_{\al_{\be}} \in E_{\al}$. So $(x_{\al_{\be}})_{\be \in B}$ is eventually in $E_{\al}$. Since $x_{\al_{\be}} \rightarrow x$ and $(x_{\al_{\be}})_{\be \in B}$ is eventually in $E_{\al}$, \rex{ex:nets:0014} implies that $x \in \cl E_{\al}$. Since $\al \in A$ is arbitrary, we have that $x \in \bigcap\limits_{\al \in A} \cl E_{\al}$. 
			\item $(3) \implies (1)$: \\
			Suppose that that $x \in \bigcap\limits_{\al \in A} \cl E_{\al}$. Let $U \in \MN(x)$. Since 
			\begin{align*}
				x 
				& \in [\Int U] \cap \bigcap\limits_{\al \in A} \cl E_{\al} \\
				& = \bigcap\limits_{\al \in A} ([\Int U] \cap \cl E_{\al} )
			\end{align*}
			we have that for each $\al \in A$, $[\Int U] \cap \cl E_{\al} \neq \varnothing$. \rex{ex:nets:0018} implies that for each $\al \in A$,
			\begin{align*}
				\varnothing 
				& \neq [\Int U] \cap E_{\al} \\
				& \subset U \cap E_{\al}
			\end{align*} 
			Let $\al \in A$. Since $U \cap E_{\al} \neq \varnothing$, there exists $x_0 \in X$ such that $x_0 \in U \cap E_{\al}$. Since $x_0 \in E_{\al}$, there exists $\al_0 \in A$ such that $\al_0 \geq \al$ and 
			\begin{align*}
				x_{\al_0} 
				& = x_0 \\
				& \in U
			\end{align*}
			Thus $(x_{\al})_{\al \in A}$ is frequently in $U$. Since $U \in \MN(x)$ is arbitrary, we have that for each $U \in \MN(x)$, $(x_{\al})_{\al \in A}$ is frequently in $U$. Thus $x$ is a cluster point of $(x_{\al})_{\al \in A}$.
		\end{itemize}
	\end{proof}
	
	\begin{ex} \lex{ex:nets:0027}
		Let $X$ be a topological space, $(x_{\al})_{\al \in A} \subset X$ a net and $x \in X$. If $x_{\al} \rightarrow x$, then for each subnet $(x_{\al_{\be}})_{\be \in B}$ of $(x_{\al})_{\al \in A}$, $x_{\al_{\be}} \rightarrow x$.
	\end{ex}

	\begin{proof}
		Suppose that $x_{\al} \rightarrow x$. Let $(x_{\al_{\be}})_{\be \in B}$ be a subnet of $(x_{\al})_{\al \in A}$ and $U \in \MN(x)$. Since $x_{\al} \rightarrow x$, there exists $\al_0 \in A$ such that for each $\al \geq \al_0$, $x_{\al} \in U$. Since $(x_{\al_{\be}})_{\be \in B}$ is a subnet of $(x_{\al})_{\al \in A}$, there exists $\be_0 \in B$ such that for each $\be \in B$, $\be \geq \be_0$ implies that $\al_{be} \geq \al_0$. Then for each $\be \in B$, $\be \geq \be_0$ implies that $x_{\al_{\be}} \in U$. Since $U \in \MN(x)$ is arbitrary, $x_{\al_{\be}} \rightarrow x$.
	\end{proof}
	
	\begin{ex} \lex{ex:nets:0028}
		Let $X$ be a topological space, $(x_{\al})_{\al \in A} \subset X$ a net and $x \in X$. Then $x_{\al} \rightarrow x$ iff for each subnet $(x_{\al_{\be}})_{\be \in B}$ of $(x_{\al})_{\al \in A}$, there exists a subnet $(x_{\al_{\be_{\gam}}})_{\gam \in \Gam}$ of $(x_{\al_{\be}})_{\be \in B}$ such that $x_{\al_{\be_{\gam}}} \rightarrow x$.  
	\end{ex}

	\begin{proof}
		\tcb{FINISH!!!}
	\end{proof}
	
	
	\begin{defn} \ld{def:nets:0029}
		Let $(x_{\al})_{\al \in A} \subset \ol{\R}$ a net. 
		\begin{itemize}
			\item We define the \tbf{limit inferior} of $(x_{\al})_{\al \in A}$, denoted $\liminf \limits_{\al \in A} x_{\al} \in \RG$, by 
			$$\liminf\limits_{\al \in A} x_{\al} = \sup_{\beta \in A } \bigg[  \inf_{\al \geq \beta} x_{\al} \bigg]$$ 
			\item We define the \tbf{limit superior} of $(x_{\al})_{\al \in A}$, denoted $\limsup\limits_{\al \in A} x_{\al} \in \RG$, by
			$$\limsup\limits_{\al \in A} x_{\al} = \inf_{\beta \in A } \bigg[ \sup_{\al \geq \beta} x_{\al} \bigg]$$   
		\end{itemize}
	\end{defn}

	\begin{ex} \lex{ex:nets:0030}
		Let $(x_{\al})_{\al \in A} \subset \R$ a net. Then
		 $$\liminf\limits_{\al \in A} x_{\al} \leq \limsup\limits_{\al \in A} x_{\al}$$
	\end{ex}

	\begin{proof}
		Set $s \defeq \liminf\limits_{\al \in A} x_{\al}$ and $S \defeq \limsup\limits_{\al \in A} x_{\al}$. Let $\ep > 0$. Then there exists $\be_1, \be_2 \in A$ such that for each $\al \in A$, $\al \geq \be_1$ implies that $\sup\limits_{\al \geq \be_1} x_{\al} < S + \ep/2$ and $\inf\limits_{\al \geq \be_2} x_{\al} > s - \ep/2$. Set $\be_0 \defeq \max(\be_1 , \be_2)$. Then 
		\begin{align*}
			s - \frac{\ep}{2}
			& < \inf\limits_{\al \geq \be_2} x_{\al} \\
			& \leq \inf\limits_{\al \geq \be_0} x_{\al} \\
			& \leq \sup\limits_{\al \geq \be_0} x_{\al} \\
			& \leq \sup\limits_{\al \geq \be_1} x_{\al} \\
			& < S + \frac{\ep}{2}
		\end{align*}
		Therefore $-\ep < S-s $. Since $\ep > 0$ is arbitrary, we have that $0 \leq S - s$. Hence
		\begin{align*}
			\liminf\limits_{\al \in A} x_{\al}
			& = s \\
			& \leq S \\
			& = \limsup\limits_{\al \in A} x_{\al}
		\end{align*}
	\end{proof}

	\begin{ex} \lex{ex:nets:0031}
		Let $(x_{\al})_{\al \in A} \subset \R$ a net and $x \in \R$. Then $x_{\al} \rightarrow x$ iff $$\liminf x_{\al} = \limsup x_{\al} = x$$
	\end{ex}

	\begin{proof}
		Suppose that $x_{\al} \rightarrow x$. Let $\ep >0$. Then there exist $\beta \in A$ such that for each $\al \in A$, $\al \geq \beta$ implies that $x_{\al} \in B(x, \ep)$. So $\inf\limits_{\al \geq \beta} x_{\al} \geq x - \ep$ and $\sup\limits_{\al \geq \beta} \leq x + \ep$. Therefore 
		\begin{align*}
			\liminf x_{\al} 
			&= \sup_{\beta \in A} \bigg[ \inf_{\al \geq \be} x_{\al} \bigg] \\
			& \geq x - \ep \\
		\end{align*}
		and 
		\begin{align*}
			\limsup x_{\al} 
			&= \inf_{\beta \in A} \bigg[ \sup_{\al \geq \be} x_{\al} \bigg] \\
			& \leq x + \ep \\
		\end{align*}
		Since $\ep >0$ is arbitrary, $$\limsup x_{\al} \leq x \leq \liminf x_{\al}$$
		Since $\liminf x_{\al} \leq \limsup x_{\al}$, we have that $\liminf x_{\al} = \limsup x_{\al} = x$.
	\end{proof}

	\begin{ex} \lex{ex:nets:0032}
		Let $(x_{\al})_{\al \in A} \subset \R$ a net and $x \in \R$. Then
		\begin{enumerate}
			\item $\liminf\limits_{\al \in A} -x_{\al} = - \limsup\limits_{\al \in A} x_{\al}$
			\item $(x_{\al})_{\al \in A} \subset \R \setminus \{0\}$ implies that $\liminf\limits_{\al \in A} x_{\al}^{-1} = \bigg(\limsup\limits_{\al \in A} x_{\al} \bigg)^{-1}$.
			\item \tcb{ generalize to any order reversing bijection of a totally ordered set (including $\ol{\R}$)}
		\end{enumerate}
	\end{ex}

	\begin{proof}\
		\begin{enumerate}
			\item We have that
			\begin{align*}
				\liminf_{\al \in A} - x_{\al}
				& = \sup_{\be \in A} \bigg[ \inf_{\al \geq \be} - x_{\al} \bigg] \\
				& =  \sup_{\be \in A} \bigg[ - \sup_{\al \geq \be}  x_{\al} \bigg] \\
				& = - \inf_{\al \in A } \bigg[ \sup_{\al \geq \be} x_{\al} \bigg] \\
				& = - \limsup_{\al \in A} x_{\al}
			\end{align*}
			\item Suppose that $(x_{\al})_{\al \in A} \subset \R \setminus \{0\}$. Then 
			\begin{align*}
				\liminf_{\al \in A}  x_{\al}^{-1}
				& = \sup_{\be \in A} \bigg[ \inf_{\al \geq \be}  x_{\al}^{-1} \bigg] \\
				& =  \sup_{\be \in A} \bigg[ \sup_{\al \geq \be}  x_{\al} \bigg]^{-1} \\
				& =  \bigg( \inf_{\al \in A } \bigg[ \sup_{\al \geq \be}  x_{\al} \bigg] \bigg)^{-1} \\
				& =  \bigg( \limsup_{\al \in A} x_{\al} \bigg)^{-1}
			\end{align*}
		\end{enumerate}
	\end{proof}
	

	\begin{ex} \lex{ex:nets:0033}
		Let $X$ be a topological space, $f:X \rightarrow \R$, $(x_{\al})_{\al \in A} \subset X$ a net and $x \in X$. Suppose that $x_{\al} \rightarrow x$ and for each $\al \in A$, $x_{\al} \neq x$. Then 
		\begin{enumerate}
			\item $\liminf\limits_{t \rightarrow x} f(t) \leq \liminf f(x_{\al})$
			\item $\limsup\limits_{t \rightarrow x} f(t) \geq \limsup f(x_{\al})$
		\end{enumerate}
	\end{ex}

	\begin{proof}\
		\begin{enumerate}
			\item Let $V \in \MN(x)$. Then there exists $\be_0 \in A$ such that for each $\al \in A$, $\al \geq \be_0$ implies that $x_{\al} \in V \setminus \{x\}$. Thus 
			\begin{align*}
				\inf_{t \in V \setminus \{x\}} f(t) 
				& \leq \inf_{\al \geq \be_0}f(x_{\al}) \\
				& \leq \sup_{\be \in A} \bigg[ \inf_{\al \geq \be} f(x_{\al}) \bigg] \\
				& = \liminf_{\al \in A} f(x_{\al})
			\end{align*}
			and since $V \in \MN(x)$ is arbitrary, we have that
			\begin{align*}
				\liminf_{t \rightarrow x} f(t) 
				&= \sup_{V \in \MN(x)} \bigg[ \inf_{t \in V \setminus \{x\}} f(t) \bigg] \\
				& \leq \liminf_{\al \in A} f(x_{\al})
			\end{align*} 
			\item Similar to $(1)$.
		\end{enumerate}
	\end{proof}































\newpage
\section{Subspace Topology}

\subsection{Introduction}

\begin{defn} \ld{def:topology:subspaces:0001}
	Let $X$ be a set and $A \subset X$. We define the \tbf{inclusion map from $A$ to $B$}, denoted $\iota: A \rightarrow X$, by $\iota(x) = x$. 
\end{defn}

\begin{defn} \ld{def:topology:subspaces:0002}
	Let $(X, \MT)$ be a topological space and $A \subset X$. We define the \tbf{subspace topology on $A$}, denoted $\MT \cap A$, by 
	$$\MT \cap A = \iota^* \MT. $$
\end{defn}

\begin{note}
	\rex{ex:continuous_maps:0006} implies that $\MT \cap A$ is a topology on $A$. 
\end{note}

\begin{ex} \lex{ex:topology:subspaces:0003}
	Let $(X, \MT)$ be a topological space and $A \subset X$. Then 
	\begin{enumerate}
		\item $\MT \cap A = \{U \cap A: U \in \MT\}$,
		\item for each $E \subset A$, $E \in \MT \cap A$ iff there exists $U \in \MT$ such that $E = U \cap A$. 
	\end{enumerate}
\end{ex}

\begin{proof}\
	\begin{enumerate}
		\item Since for each $U \subset X$, $\iota^{-1}(U) = U \cap A$, we have that 
		\begin{align*}
			\MT \cap A
			& = \iota^* \MT \\
			& = \{\iota^{-1}(U): U \in \MT\} \\
			& = \{ U \cap A: U \in \MT\} \\
		\end{align*}
		\item Clear
	\end{enumerate}
\end{proof}

\begin{ex} \lex{ex:topology:subspaces:0003.1}
	Let $X$ be a set, $\ME \subset \MP(X)$ and $A \subset X$. Then 
	$$\tau_X(\ME) \cap A = \tau_A(\ME \cap A).$$ 
	\tbf{Hint:} $\tau_X(\ME) \subset (\iota_A)_* \tau_A(\ME \cap A)$
\end{ex}

\begin{proof}\
	\begin{itemize}
		\item Clearly $\ME \cap A \subset \tau_X(\ME) \cap A$. Since $\tau_X(\ME) \cap A$ is a topology on $A$, we have that 
		\begin{align*}
			\tau_A(\ME \cap A) 
			& \subset \tau_A[\tau_X(\ME) \cap A] \\
			& = \tau_X(\ME) \cap A.
		\end{align*} 
		\item Let $V_0 \in \ME$. Then
		\begin{align*}
			\iota_A^{-1}(V_0) 
			& = V_0 \cap A \\
			& \in \ME \cap A \\
			& \subset \tau_A(\ME \cap A).
		\end{align*}
		Hence $V_0 \in (\iota_A)_* \tau_A(\ME \cap A)$. Since $V_0 \in \ME$ is arbitrary, we have that $\ME \subset (\iota_A)_* \tau_A(\ME \cap A)$. Thus 
		\begin{align*}
			\tau_X(\ME)
			& \subset \tau_X[(\iota_A)_* \tau_A(\ME \cap A)] \\
			& = (\iota_A)_* \tau_A(\ME \cap A).
		\end{align*}
		Let $V \in \tau_X(\ME) \cap A$. Then there exists $V_0 \in \tau_X(\ME)$ such that $V = V_0 \cap A$. Therefore
		\begin{align*}
			V_0
			& \in \tau_X(\ME) \\
			& \subset (\iota_A)_* \tau_A(\ME \cap A).
		\end{align*}
		By definition of $(\iota_A)_* \tau_A(\ME \cap A)$, we have that 
		\begin{align*}
			V
			& = V_0 \cap A \\
			& = \iota_A^{-1}(V_0) \\
			& \in \tau_A(\ME \cap A).
		\end{align*}
		Since $V \in \tau_X(\ME) \cap A$ is arbitrary, we have that $\tau_X(\ME) \cap A \subset \tau_A(\ME \cap A)$. 
	\end{itemize}
	Since $\tau_A(\ME \cap A) \subset \tau_X(\ME) \cap A$ and $\tau_X(\ME) \cap A \subset \tau_A(\ME \cap A)$, we have that $\tau_X(\ME) \cap A = \tau_A(\ME \cap A)$.
\end{proof}

\begin{ex} \lex{ex:topology:subspaces:0004}
	Let $(X, \MT)$ be a topological space, $A \subset X$ and $B \subset A$. Then $\MT \cap B = (\MT \cap A) \cap B$.
\end{ex}

\begin{proof}\
	\begin{itemize}
		\item Let $U \in (\MT \cap A) \cap B$. Then there exists $U_A \in \MT \cap A$ such that $U = U_A \cap B$. Similarly, there exists $U_X \in \MT$ such that $U_A = U_X \cap A$. Therefore 
		\begin{align*}
			U
			& = U_A \cap B \\
			& = (U_X \cap A) \cap B \\
			& U_X \cap (A \cap B) \\
			& = U_X \cap B \\
			& \in \MT \cap B
		\end{align*}
		Since $U \in (\MT \cap A) \cap B$ is arbitrary, we have that $(\MT \cap A) \cap B \subset \MT \cap B$. \\
		\item Conversely, let $U \in \MT \cap B$. Then there exists $U_X \in \MT$ such that $U = U_X \cap B$. Then 
		\begin{align*}
			U
			& = U_X \cap B \\
			& = U_X \cap (A \cap B) \\
			& = (U_X \cap A) \cap B \\
			& \in (\MT \cap A) \cap B
		\end{align*}
		Since $U \in U_X \cap B$ is arbitrary, we have that $U_X \cap B \subset (\MT \cap A) \cap B$. 
	\end{itemize}
	Therefore $\MT \cap B = (\MT \cap A) \cap B$.
\end{proof}

\begin{note}
	The previous exercise just indicates that the subspace topology on $B$ does not depend on choice of the containing space as long as each containing space is a subspace of the original space \tcr{maybe rephrase this in terms of projective system or something}.
\end{note}

\begin{ex} \lex{ex:topology:subspaces:0004.1}
	Let $(X, \MT)$ be a topological space, $A \subset X$ and $S \subset A$. Then $\MN_{\MT \cap A}(x) = \MN_{\MT}(x) \cap A$. 
\end{ex}

\begin{proof}\
	\begin{itemize}
		\item Let $U \in \MN_{\MT \cap A}(S)$. \rex{31014.1} implies that $S \subset \Int_{\MT \cap A} U$. Since $\Int_{\MT \cap A} U \in \MT \cap A$, there exists $V \in \MT$ such that $\Int_{\MT \cap A} U = V \cap A$. Define $E \subset A$ by $E \defeq U \setminus \Int_{\MT \cap A} U$. Then 
		\begin{align*}
			U
			& = (\Int_{\MT \cap A} U)  \cup E \\
			& = (V \cap A)  \cup (E  \cap A) \\
			& = [(V \cap A) \cup E] \cap [(V \cap A) \cup A] \\
			& = [(V \cup E) \cap (A \cup E)] \cap [(V \cup A) \cap (A \cup A)] \\
			& = [(V \cup E) \cap (A \cup E)] \cap [(V \cup A) \cap A] \\
			& = [(V \cup E) \cap (A \cup E)] \cap A \\
			& = (V \cup E) \cap [(A \cup E) \cap A]\\
			& = (V \cup E) \cap A.
		\end{align*}
		Since $V \in \MT$, $S \subset V$ and $V \subset V \cup E$, we have that $V \cup E \in \MN_{\MT}(S)$. Thus 
		\begin{align*}
			U
			& = (V \cup E) \cap A \\
			& \in \MN_{\MT}(S) \cap A.
		\end{align*}
		Since $U \in \MN_{\MT \cap A}(S)$ is arbirary, we have that for each $U \in \MN_{\MT \cap A}(S)$, $U \in \MN_{\MT}(S) \cap A$. Hence $\MN_{\MT \cap A}(S) \subset \MN_{\MT}(S) \cap A$. 
		\item Let $U \in \MN_{\MT}(S) \cap A$. Then there exists $V \in \MN_{\MT}(S)$ such that $U = V \cap A$. We note that $S \subset (\Int_{\MT} V) \cap A$ and $(\Int_{\MT} V) \cap A \in \MT \cap A$. Define $E \subset X$ by $E \defeq V \setminus \Int_{\MT} V$. Then $[(\Int_{\MT} V) \cap A] \cup (E \cap A) \in \MN_{\MT \cap A}(S)$ and therefore
		\begin{align*}
			U
			& = V \cap A \\
			& = [(\Int_{\MT} V) \cup E] \cap A \\
			& = [(\Int_{\MT} V) \cap A] \cup (E \cap A) \\
			& \in \MN_{\MT \cap A}(S).
		\end{align*}
		Since $U \in \MN_{\MT}(S) \cap A$ is arbirary, we have that for each $U \in \MN_{\MT}(S) \cap A$, $U \in \MN_{\MT \cap A}(S)$. Hence $\MN_{\MT \cap A}(S) \subset \MN_{\MT \cap A}(S)$. 
	\end{itemize}
\end{proof}

\begin{ex} \lex{ex:topology:subspaces:0004.2}
	Let $(X, \MT)$ be a topological space, $E \subset X$, $A \subset E$ and $x \in E$. Then $x$ is a $\MT \cap E$-condensation point of $A$ iff $x$ is a $\MT$-condensation point of $A$.
\end{ex}

\begin{proof}\
	\begin{itemize}
		\item $(\implies):$ \\
		Suppose that $x$ is a $\MT \cap E$-condensation point of $A$. Let $U \in \MN_{\MT}(x)$. \rex{ex:topology:subspaces:0004.1} implies that $U \cap E \in \MN_{\MT \cap E}(x)$. Since $x$ is a $\MT \cap E$-condensation point of $A$, we have that $(U \cap E) \cap A$ is uncountable. Since $A \subset E$, we have that
		\begin{align*}
			U \cap A
			& = U \cap (A \cap E) \\
			& = (U \cap E) \cap A 
		\end{align*}
		and therefore $U \cap A$ is uncountable. Since $U \in \MN_{\MT}(x)$ is arbitrary, we have that for each $U \in \MN_{\MT}(x)$, $U \cap A$ is uncountable. Hence $x$ is a $\MT$-condensation point of $A$. 
		\item $(\impliedby):$ \\
		Suppose that $x$ is a $\MT$-condensation point of $A$. Let $U \in \MN_{\MT \cap E}(x)$. \rex{ex:topology:subspaces:0004.1} implies that $U \in \MN_{\MT}(x) \cap E$. Thus there exists $V \in \MN_{\MT}(x)$ such that $U = V \cap E$. Since $x$ is a $\MT$-condensation point of $A$, we have that $V \cap A$ is uncountable. Since 
		\begin{align*}
			U \cap A
			& = (V \cap E) \cap A \\
			& = V \cap (A \cap E) \\ 
			& = V \cap A,
		\end{align*}
		we have that $U \cap A$ is uncountable. Since $U \in \MN_{\MT \cap E}(x)$ is arbitrary, we have that for each $U \in \MN_{\MT \cap E}(x)$, $U \cap A$ is uncountable. Hence $x$ is a $\MT \cap E$-condensation point of $A$. 
	\end{itemize}
\end{proof}



\begin{ex} \lex{ex:topology:subspaces:0005}
	Let $(X, \MT)$ be a topological space, $A \subset X$, $(x_{\gam})_{\gam \in \Gam} \subset A$ a net and $x \in A$. Then $x_{\gam} \rightarrow x$ in $(A,\MT \cap A)$ iff $x_{\gam} \rightarrow x$ in $(X,\MT)$.
\end{ex}

\begin{proof}\
	\begin{itemize}
		\item Suppose that $x_{\gam} \rightarrow x$ in $(A, \MT \cap A)$. Since $\iota: A \rightarrow X$ is $(\MT \cap A, \MT)$-continuous, 
		\begin{align*}
			x_{\gam} 
			&= \iota(x_{\gam}) \\
			& \rightarrow \iota(x) \\
			&= x
		\end{align*}
		So that $x_{\gam} \rightarrow x$ in $(X, \MT)$.  
		\item Conversely, suppose that $x_{\gam} \rightarrow x$ in $(X, \MT)$. Let $V \in \MN(x)$ in $(A, \MT \cap A)$. Then $x \in \Int V$ in  $(A, \MT \cap A)$. Hence there exists $U \in \MT$ such that $\Int V = U \cap A$. Thus $U \in \MN(x)$ in $(X, \MT)$. Since $x_{\gam} \rightarrow x$ in $(X, \MT)$ and $x \in U$, we have that that $(x_{\gam})_{\gam \in \Gam}$ is eventually in $U$. Then $(x_{\gam})_{\gam \in \Gam}$ is eventually in $U \cap A = \Int V \subset V$. So $x_{\gam} \rightarrow x$ in $(A, \MT \cap A)$. 
	\end{itemize}
\end{proof}

\begin{ex} \lex{ex:topology:subspaces:0005.0001}
	Let $(X, \MT_X)$, $(Y, \MT_Y)$ be topological spaces, $B \subset Y$ and $f:X \rightarrow B$. Then $f$ is $(\MT_X, \MT_Y \cap B)$-continuous iff $f$ is $(\MT_X, \MT_Y)$-continuous. 
\end{ex}

\begin{proof}\
	 \begin{itemize}
	 	\item $(\implies):$ \\
	 	Suppose that $f$ is $(\MT_X, \MT_Y \cap B)$-continuous. Let $(x_{\gam})_{\gam \in \Gam} \subset X$ a net and $x \in X$. Suppose that $x_{\gam} \rightarrow x$. Since $f$ is $(\MT_X, \MT_Y \cap B)$-continuous, we have that $f(x_{\gam}) \rightarrow f(x)$ in $(B, \MT_Y \cap B)$. \rex{ex:topology:subspaces:0005} implies that $f(x_{\gam}) \rightarrow f(x)$ in $(Y, \MT_Y)$. Since $(x_{\gam})_{\gam \in \Gam} \subset A$ is an arbitrary net and $x \in A$ is arbitrary, we have that $f$ is $(\MT_X, \MT_Y)$-continuous.
	 	\item $(\impliedby):$ \\
	 	Similar to $(1)$.
	 \end{itemize}
\end{proof}

\begin{ex} \lex{ex:topology:subspaces:0006}
	Let $(X, \MA)$, $(Y, \MB)$ be topological spaces and $f:X \rightarrow Y$. Then $f$ is $(\MA, \MB)$-continuous iff for each $x \in X$, there exists $U \in \MA$ such that $x \in U$ and $f|_U$ is $(\MA \cap U, \MB)$-continuous.
\end{ex}

\begin{proof}\
	\begin{itemize}
		\item $(\implies)$: \\
		Suppose that $f$ is continuous. Let $x \in X$. Define $U \in \MA$ by $U \defeq X$. Then $x \in U$ and $f|_U$ is $(\MA \cap U, \MB)$-continuous. 
		\item $(\impliedby)$: \\
		Suppose that for each $x \in X$, there exists $U \in \MA$ such that $x \in U$ and $f|_U$ is $(\MA \cap U, \MB)$-continuous. Let $x \in X$ and $V \in \MB$. Suppose that $f(x) \in V$. By assumption, there exists $U_0 \in \MA$ such that $x \in U_0$ and $f|_{U_0}$ is $(\MA \cap U_0, \MB)$-continuous. Define $U \subset X$ by $U = f|_{U_0}^{-1}(V)$. Since $V \in \MB$ and $f|_{U_0}$ is $(\MA \cap U_0, \MB)$-continuous, $U \in \MA \cap U_0$. Since $U_0 \in \MA$, $\MA \cap U_0 \subset \MA$ and therefore $U \in \MA$. We note that $x \in U$. Since $f|_{U_0}^{-1}(V) = U_0 \cap f^{-1}(V)$, we have that 
		\begin{align*}
			f(U)
			& = f(f|_{U_0}^{-1}(V)) \\
			& = f(U_0 \cap f^{-1}(V)) \\
			& \subset V
		\end{align*} 
		Since $V \in \MB$ with $f(x) \in V$ is arbitrary, we have that for each $V \in \MB$, $f(x) \in V$ implies that there exists $U \in \MA$ such that $x \in U$ and $f(U) \subset V$. Thus $f$ is continuous at $x$. Since $x \in X$ is arbitrary, $f$ is continuous. 
	\end{itemize}
\end{proof}

\begin{ex} \lex{ex:topology:subspaces:0007}
	Let $(X, \MT)$ be a topological space, $A \subset X$ and $F \subset A$. Then $F$ is closed in $A$ iff there exists $C \subset X$ such that $C$ is closed in $X$ and $F = C \cap A$. 
\end{ex}

\begin{proof}
	Suppose that $F$ is closed in $A$. Then $A \setminus F$ is open in $A$. Hence there exists $U \in \MT$ such that $A \setminus F = U \cap A$. Set $C = U^c$. Then $C$ is closed in $X$ and  
	\begin{align*}
		F
		& = A \setminus (A \setminus F) \\
		& = A \setminus (U \cap A) \\
		& = A \cap [(U \cap A)^c] \\
		& = A \cap (U^c \cup A^c) \\
		& = (A \cap U^c) \cup (A \cap A^c) \\
		& = A \cap U^c \\
		& = A \cap C
	\end{align*}
Conversely, suppose that there exists $C \subset X$ such that $C$ is closed in $X$ and $F = A \cap C$. Since $C^c \in \MT$, we have that  
\begin{align*}
	A \setminus F
	& = A \cap F^c \\
	& = A \cap [(A \cap C)^c] \\
	& = A \cap (A^c \cup C^c) \\
	& = (A \cap A^c) \cup (A \cap C^c) \\
	& = A \cap C^c \\
	& \in \MT \cap A
\end{align*}
Thus $A \setminus F$ is open in $A$ which implies that $F$ is closed in $A$.
\end{proof}

\begin{ex} \lex{ex:topology:subspaces:0008}
	Let $(X, \MA)$ and $(Y, \MB)$ be topological spaces and $f:X \rightarrow Y$. Then $f$ is $(\MA, \MB)$-continuous iff $f$ is $(\MA, \MB \cap f(X))$-continuous. \\ 
	\tcr{duplicate of \rex{ex:topology:subspaces:0005.0001}, maybe keep statement in \rex{ex:topology:subspaces:0005.0001}, but keep or add this proof in \rex{ex:topology:subspaces:0008}}
\end{ex}

\begin{proof}\
	\begin{itemize}
		\item $(\implies):$ \\
		Suppose that $f$ is $(\MA, \MB)$-continuous. Let $B \in \MB \cap f(X)$. Then there exists $V \in \MB$ such that $B = V \cap f(X)$. Then 
		\begin{align*}
			f^{-1}(B)
			& = f^{-1}(V \cap f(X)) \\
			& = f^{-1}(V) \cap f^{-1}(f(X)) \\
			& = f^{-1}(V) \cap X \\
			& = f^{-1}(V) \\
			& \in \MA
		\end{align*}
		Since $B \in \MB \cap f(X)$ is arbitrary, $f$ is  $(\MA, \MB \cap f(X))$-continuous. 
		\item $(\impliedby):$ \\
		Conversely, suppose that $f$ is  $(\MA, \MB \cap f(X))$-continuous. Let $V \in \MB$. Then $V \cap f(X) \in \MB \cap f(X)$ and 
		\begin{align*}
			f^{-1}(V)
			& = f^{-1}(V \cap f(X)) \\
			& \in \MA  
		\end{align*} 
		Since $V \in \MB$ is arbitrary, $f$ is  $(\MA, \MB)$-continuous. 
	\end{itemize}
\end{proof}

\begin{ex} \lex{ex:topology:subspaces:0009} \tbf{Basis for Subspace Topology:} \\
	Let $(X, \MT)$ be a topological space, $\MB \subset \MT$ a basis for $\MT$ on $X$ and $A \subset X$. Then $\MB \cap A$ is a basis for $\MT \cap A$. 
\end{ex}

\begin{proof}
	Let $E \in \MT \cap A$. Then there exists $E' \in \MT$ such that $E = E' \cap A$. Since $\MB$ is a basis for $\MT$ on $X$, there exists $\MU' \subset \MB$ such that $E' = \bigcup\limits_{U' \in \MU'} U'$. Define $\MU \subset \MT \cap A$ by $\MU \defeq \MU' \cap A$. Then 
	\begin{align*}
		\MU
		& = \MU' \cap A \\
		& \subset \MB \cap A
	\end{align*}
	and
	\begin{align*}
		E
		& = E' \cap E \\
		& = \bigg( \bigcup\limits_{U' \in \MU} U' \bigg)  \cap A \\
		& = \bigcup\limits_{U' \in \MU'} (U' \cap A) \\
		& = \bigcup\limits_{U \in \MU' \cap A} U \\
		& = \bigcup\limits_{U \in \MU} U 
	\end{align*}
	Since $E \in \MT \cap A$ is arbitrary, we have that for each $E \in \MT \cap A$, there exists $\MU \subset \MB \cap A$ such that $E = \bigcup\limits_{U \in \MU} U $. Hence $\MB \cap A$ is a basis for $\MT \cap A$ on $A$.
\end{proof}

\begin{ex} \lex{ex:topology:subspaces:0009.1}
	Let $(X, \MT)$, $(Y, \MS)$ be topological spaces and $f:X \rightarrow Y$. If $f$ is $(\MT, \MS)$-open, then for each $U \in \MT$, $f|_U$ is $(\MT \cap U, \MS)$-open.
\end{ex}

\begin{proof}
	Suppose that $f$ is $(\MT, \MS)$-open. Let $U \in \MT$ and $V \in \MT \cap U$. Since $U \in \MT$, 
	\begin{align*}
		V
		& \in \MT \cap U \\
		& \subset \MT. 
	\end{align*}
	Since $f$ is $(\MT, \MS)$-open and $V \subset U$, we have that 
	\begin{align*}
		f|_U(V)
		& = f(V) \\
		& \in \MS.
	\end{align*}
	Since $V \in \MT \cap U$ is arbitrary, we have that for each $V \in \MT \cap U$, $f|_U(V) \in \MS$. Thus $f|_U$ is $(\MT \cap U, \MS)$. Since $U \in \MT$ is arbitrary, we have that for each $U \in \MT$, $f|_U$ is $(\MT \cap U, \MS)$. 
\end{proof}

\begin{ex} \lex{ex:topology:subspaces:0010}
	Let $(X, \MT)$, $(Y, \MS)$ be topological spaces and $f:X \rightarrow Y$. Then $f$ is $(\MT, \MS)$-open iff for each $x \in X$, there exists $U \in \MT$ such that $x \in U$ and $f|_U$ is $(\MT \cap U, \MS)$-open.
\end{ex}

\begin{proof}\
	\begin{itemize}
		\item $(\implies)$: \\ 
		Suppose that $f$ is $(\MT, \MS)$-open. Let $x \in X$. Set $U \defeq X$. Then $X \in \MT$, $x \in U$ and $f|_U$ is $(\MT \cap U, \MS)$-open. 
		\item $(\impliedby):$ \\
		Suppose that for each $x \in X$, there exists $U \in \MT$ such that $x \in U$ and $f|_U$ is $(\MT \cap U, \MS)$-open. Let $V \in \MT$. By assumption, for each $x \in V$, there exists $U_x \in \MT$ such that $f|_{U_x}$ is $(\MT \cap U_x, \MS)$-open. Then for each $x \in V$, $V \cap U_x \in \MT \cap U_x$ and $V = \bigcup\limits_{x \in V} (V \cap U_x)$. Since for each $x \in V$, $f|_{U_x}$ is $(\MT \cap U_x, \MS)$-open, we have that 
		\begin{align*}
			f(V)
			& = f \bigg( \bigcup_{x \in V} (V \cap U_x) \bigg) \\
			& = \bigcup_{x \in V} f (V \cap U_x) \\
			& = \bigcup_{x \in V} f|_{U_x}(V \cap U_x) \\
			& \in \MS
		\end{align*}
		Since $V \in \MT$ is arbitrary, we have that for each $V \in \MT$, $f(V) \in \MS$. Hence $f$ is open.
	\end{itemize}
\end{proof}

\begin{ex} \lex{ex:topology:subspaces:0010.01}
	Let $(X, \MT_X)$, $(Y, \MT_Y)$ be topological spaces, $f:X \rightarrow Y$ and $E \subset X$. Suppose that $f$ is $(\MT_X, \MT_Y)$-closed. If $E$ is closed in $(X, \MT_Y)$, then $f|_E$ is $(\MT_X \cap E, \MT_Y)$-closed.
\end{ex}

\begin{proof}
	Suppose that $E$ is closed in $(X, \MT_Y)$. Let $F \subset E$. Suppose that $F$ is closed in $(E, \MT_X \cap E)$. \rex{ex:topology:subspaces:0007} implies that there exists $C \subset X$ such that $C$ is closed in $(X, \MT_X)$ and $F = C \cap E$. Since $C, E$ are closed in $(X, \MT_X)$, $F$ is closed in $(X, \MT_X)$. Since $f$ is closed, $f(F)$ is closed in $(Y, \MT_Y)$. Since $f|_E(F) = f(F)$, we have that $f|_E(F)$ is closed in $(Y, \MT_Y)$. Since $F \subset E$ with $F$ closed in $(E, \MT_X \cap E)$ is arbitrary, we have that for each $F \subset E$, $F$ is closed in $(E, \MT_X \cap E)$ implies that $f|_E(F)$ is closed in $(Y, \MT_Y)$. Hence $f|_E$ is $(\MT_X \cap E, \MT_Y)$-closed.
\end{proof}

\begin{ex} \lex{ex:topology:subspaces:0010.1}
	Let $(X, \MT_X), (A, \MT_A)$ be topological spaces. Suppose that $A \subset X$. If $\iota_A|^A$ is a $(\MT_A, \MT_X \cap A)$-homeomorphism, then $\MT_A = \MT_X \cap A$. 
\end{ex}

\begin{proof} Suppose that $\iota_A|^A$ is a $(\MT_A, \MT_X \cap A)$-homeomorphism.
	\begin{itemize}
		\item Let $U \in \MT_A$. Since $\iota_A(U) = U$ and $\iota_A|^A$ is $(\MT_A, \MT_X \cap A)$-open, we have that 
		\begin{align*}
			U 
			& = \iota_A(U) \\
			& = \iota_A|^A(U) \\
			& \in \MT_X \cap A.
		\end{align*}
		Since $U \in \MT_A$ is arbitrary, we have that $\MT_A \subset \MT_X \cap A$.  
		\item Let $U \in \MT_X \cap A$. Since $\iota_A|^{A}$ is $(\MT_A, \MT_X \cap A)$-continuous and $U \subset A$, we have that we have that 
		\begin{align*}
			U
			& = \iota_A^{-1}(U) \\
			& = (\iota_A|^A)^{-1}(U) \\
			& \in \MT_A.
		\end{align*}
		Since $U \in \MT_X \cap A$ is arbitrary, we have that  $\MT_X \cap A \subset \MT_A$. 
	\end{itemize}
	Hence $\MT_A = \MT_X \cap A$. 
\end{proof}


\begin{ex} \lex{ex:topology:subspaces:0011}
	universal property
\end{ex}

\begin{proof}
	\tcr{FINISH!!!}
\end{proof}

\begin{ex} \lex{ex:topology:subspaces:0012}
	Let $(X, \MT)$ be a topological space, $\MU \subset \MT$, $A \subset X$ and $E \subset A$. If $\MU$ is an open cover of $E$ in $(X, \MT)$, then $\MU \cap A$ is an open cover of $E$ in $(A, \MT \cap A)$. 
\end{ex}

\begin{proof}
	Suppose that $\MU$ is an open cover of $E$ in $(X, \MT)$. 
	\begin{enumerate}
		\item Since $\MU \subset \MT$, $\MU \cap A \subset \MT \cap A$. 
		\item Since $\MU$ is a cover of $E$ in $X$, $E \subset \bigcup\limits_{U \in \MU} U$. Since $E \subset A$, we have that
		\begin{align*}
			E
			& = E \cap A \\
			& \subset \bigg[ \bigcup\limits_{U \in \MU} U \bigg] \cap A \\
			& = \bigcup\limits_{U \in \MU} [U \cap A] \\
			& = \bigcup\limits_{U \in \MU \cap A} U.
		\end{align*}
		Hence $\MU \cap A$ is a cover of $E$ in $A$. 
	\end{enumerate}
	Since $\MU \cap A \subset \MT \cap A$ and $\MU \cap A$ is a cover of $E$ in $A$, we have that $\MU \cap A$ is an open cover of $E$ in $(A, \MT \cap A)$. 
\end{proof}



















\subsection{Discrete Subsets}

\begin{defn} \ld{def:topology:subspaces:0012}
	Let $(X, \MT)$ be a topological space, $A \subset X$ and $x \in A$. Then $x$ is said to an \tbf{isolated point of $A$} if there exists $U \in \MT$ such that $U \cap A = \{x\}$. 
\end{defn}

\begin{ex} \lex{ex:topology:subspaces:0013}
	Let $(X, \MT)$ be a topological space, $A \subset X$ and $x \in A$. Then $x$ is an isolated point of $A$ iff $\{x\} \in \MT \cap A$.
\end{ex}

\begin{proof}
	Suppose that $x$ is an isolated point of $A$. Then there exists $U \in \MT$ such that 
	\begin{align*}
		\{x\}
		& = U \cap A \\
		& \in \MT \cap A
	\end{align*} 
	Conversely, suppose that $\{x\} \in \MT \cap A$. Then there exists $U \in \MT$ such that $\{x\} = U \cap A$. Hence $x$ is an isolated point of $A$.
\end{proof}

\begin{defn} \ld{def:topology:subspaces:0014}
	Let $(X, \MT)$ be a topological space and $A \subset X$. Then $A$ is said to be \tbf{discrete} if for each $x \in A$, $x$ is an isolated point of $A$.
\end{defn}

\begin{ex} \lex{ex:topology:subspaces:0015}
	Let $(X, \MT)$ be a topological space and $A \subset X$. Then $A$ is discrete iff $\MT \cap A = \MP(A)$. 
\end{ex}

\begin{proof}
	Suppose that $A$ is discrete. Then for each $x \in A$, $\{x\} \in \MT \cap A$. Let $U \in \MP(A)$. Then 
	\begin{align*}
		U
		& = \bigcup_{x \in U} \{x\} \\
		& \in \MT \cap A
	\end{align*}
	Since $U \in \MP(A)$ is arbitrary, we have that $\MP(A) \subset \MT \cap A$. Since $\MT \cap A \subset \MP(A)$, we have that $\MT \cap A = \MP(A)$. \\
	Conversely, suppose that $\MT \cap A = \MP(A)$. Let $x \in A$. Then
	\begin{align*}
		\{x\}
		& \in \MP(A) \\
		& = \MT \cap A
	\end{align*}	
	Hence $x$ is an isolated point of $A$. Since $x \in A$ is arbitrary, $A$ is discrete.
\end{proof}












































\subsection{Topological Embeddings}

\begin{defn} \ld{def:topology:subspaces:0016}
	Let $(X, \MT_X), (Y, \MT_Y)$ be a topological space and $f : X \rightarrow Y$. Then $f$ is said to be a \tbf{$\Top$}-embedding if $f: X \rightarrow f(X)$ is a $(\MT_X, \MT_Y \cap f(X))$-homeomorphism.
\end{defn}

\begin{ex} \lex{ex:topology:subspaces:0017}
	Let $(X, \MT_X), (Y, \MT_Y)$ be a topological space and $f : X \rightarrow Y$. Suppose that $f$ is $(\MT_X, \MT_Y)$-continuous and injective. Then 
	\begin{enumerate}
		\item $f$ is open implies that $f$ is a $\Top$-embedding
		\item $f$ is closed implies that $f$ is a $\Top$-embedding
	\end{enumerate}
\end{ex}

\begin{proof}\
	\begin{enumerate}
		\item Suppose that $f$ is open. Since $f$ is $(\MT_X, \MT_Y)$-continuous, $f$ is $(\MT_X, \MT_Y \cap f(X))$-continuous. Since $f$ is injective, $f:X \rightarrow f(X)$ is bijective. Since $f$ is $(\MT_X, \MT_Y)$-open, $f$ is $(\MT_X, \MT_Y \cap f(X))$-open. Thus $f:X \rightarrow f(X)$ is a $(\MT_X, \MT_Y \cap f(X))$-homeomorphism.
		\item Similar to $(1)$ by \rex{ex:continuous_maps:0001.1}. 
	\end{enumerate}
\end{proof}




































	
	
	
	
	
	
	
	
	
	
	
	
	
	\newpage
	\section{Product Topology}
	
	\subsection{Bases of the product topology}
	
	\begin{defn} \ld{def:product_topology:0001}
	Let $(X_{\al}, \MT_{\al})_{\al \in A}$ be a collection of topological spaces. We define the \tbf{product topology} on $\prod_{\al \in A}X_{\al}$, denoted $\bigotimes\limits_{\al \in A} \MT_{\al}$, by 
	$$\bigotimes\limits_{\al \in A} \MT_{\al} = \tau(\pi_{\al}: {\al \in A})$$
	i.e. $\bigotimes\limits_{\al \in A} \MT_{\al} $ is the initial (weak) topology on $\prod\limits_{\al \in A} X_{\al}$ generated by the projection maps $(\pi_{\al})_{\al \in A}$.
	\end{defn}

	\begin{ex} \lex{ex:product_topology:0002}
		Let $(X_{\al}, \MT_{\al})_{\al \in A}$ be a collection of topological spaces. Define 
		$$\MB = \bigg \{\prod_{\al \in A}B_{\al}: \text{ for each $\al \in A$,  $B_{\al} \in \MT_{\al}$ and $\# \{\al \in A: B_{\al} \neq X_{\al}\} < \infty$} \bigg\}$$
		Then $\MB$ is a basis for $\bigotimes\limits_{\al \in A} \MT_{\al}$.
	\end{ex}

	\begin{proof}
		Set $X = \prod_{\al \in A}X_{\al}$ and $\MT_X = \bigotimes_{\al \in A} \MT_{\al}$. Set 
		$$\ME = \{\pi_{\al}^{-1}(B_{\al}): \al \in A, B_{\al} \in \MT_{\al}\}$$ 
		By definition, $\MT_{X} = \tau_{X}(\ME)$. Let $\al \in A$ and $B_{\al} \in \MT_{\al}$. For $\be \in A$, set 
		\[
		C_{\be} = 
		\begin{cases}
			B_{\be} & \be = \al \\
			X_{\be} & \be \neq \al
		\end{cases}
		\]
		Then 
		\begin{align*}
			\pi_{\al}^{-1}(B_{\al}) = \prod_{\be \in A} C_{\be}  
		\end{align*}
		Hence $\MB = \bigg \{\bigcap_{j=1}^n V_j:(V_j)_{j=1}^n \subset \ME \bigg \} \subset \MT_X$. \rex{31010} implies that $\MB$ is a basis for $\MT_X$.
	\end{proof}

	\begin{ex} \lex{ex:product_topology:0002.1}
		Let $(X_{\al}, \MT_{\al})_{\al \in A}$ be a collection of topological spaces. Then for each $\al \in A$, $\pi_{\al}: \prod\limits_{\al \in A} X_{\al} \rightarrow X_{\al}$ is open.
	\end{ex}

	\begin{proof}
		Let $\al \in A$. Define 
		$$\MB = \bigg \{\prod_{\al \in A}B_{\al}: \text{ for each $\al \in A$,  $B_{\al} \in \MT_{\al}$ and $\# \{\al \in A: B_{\al} \neq X_{\al}\} < \infty$} \bigg\}$$
		Then \rex{ex:product_topology:0002} implies that $\MB$ is a basis for $\bigotimes\limits_{\al \in A} \MT_{\al}$. Let $U \in \bigotimes\limits_{\al \in A} \MT_{\al}$. Then for each $\al' \in A$, there exist $B_{\al'} \in \MT_{\al'}$ such that $U = \prod_{\al' \in A}B_{\al'}$ and $\# \{\al' \in A: B_{\al'} \neq X_{\al'}\} < \infty$. Then 
		\begin{align*}
			\pi_{\al}(U)
			& = B_{\al} \\
			& \in \MT_{\al}
		\end{align*}
		Since $U \in  \bigotimes\limits_{\al \in A} \MT_{\al}$ is arbitrary, we have that for each $U \in  \bigotimes\limits_{\al \in A} \MT_{\al}$, $\pi_{\al}(U) \in \MT_{\al}$. \rex{ex:continuous_maps:0014} implies that $\pi_{\al}$ is open. Since $\al \in A$ is arbitrary, we have that for each $\al \in A$, $\pi_{\al}$ is open.
	\end{proof}

	\begin{ex}  \lex{ex:product_topology:0003}
		Let $(X_{\al}, \MT_{\al})_{\al \in A}$ be a collection of topological spaces, $x \in \prod_{\al \in A}X_{\al}$ and for each $\al \in A$, $\MB_{x_{\al}} \subset \MT_{\al}$. Suppose that for each $\al \in A$, $\MB_{x_{\al}}$ is a local basis for $\MT_{\al}$ at $x_{\al}$. Define 
		$$\MB_x = \bigg \{\prod_{\al \in A} U_{\al}: \text{ [for each $\al \in A$, $U_{\al} \in \MT_{\al}$ and $U_{\al} \neq X_{\al}$ implies that $U_{\al} \in \MB_{x_{\al}}$] and $\# \{\al \in A: U_{\al} \neq X_{\al}\} < \infty$} \bigg\}$$
		Then $\MB_x$ is a local basis for $\bigotimes\limits_{\al \in A} \MT_{\al}$ at $x$.
	\end{ex}

	\begin{proof}Set $X = \prod_{\al \in A}X_{\al}$ and $\MT = \bigotimes\limits_{\al \in A} \MT_{\al}$.
		\begin{enumerate}
			\item By construction, for each $V \in \MB_x$, $x \in V$. 
			\item Let $U \in \MT$. Suppose that $x \in U$. Set 
			$$\MB = \bigg \{\prod_{\al \in A}B_{\al}: \text{ for each $\al \in A$,  $B_{\al} \in \MT_{\al}$ and $\# \{\al \in A: B_{\al} \neq X_{\al}\} < \infty$} \bigg\}$$
			The previous exercise implies that $\MB$ is a basis for $\bigotimes\limits_{\al \in A} \MT_{\al}$. Thus for each $\al \in A$, there exists $B_{\al} \in \MT_{\al}$ such that $\# \{\al \in A: B_{\al} \neq X_{\al}\} < \infty$ and $x \in \prod\limits_{\al \in A} B_{\al} \subset U$. Set $J = \{\al \in A: B_{\al} \neq X_{\al}\}$. Since $x \in \prod\limits_{\al \in A} B_{\al}$, for each $\al \in A$, $x_{\al} \in B_{\al}$. Since for each $\al \in A$, $\MB_{x_{\al}}$ is a local basis for $\MT_{\al}$ at $x_{\al}$, the axiom of choice implies that there exists $(U_{\al})_{\al \in A} \in \prod\limits_{\al \in A} \MT_{\al}$ such that for each if $\al \in J$, $U_{\al} \in \MB_{x_{\al}}$ and $x_{\al} \in U_{\al} \subset B_{\al}$ and  for each $\al \in J^c$, $U_{\al} = X_{\al}$. By definition, $\prod\limits_{\al \in A} U_{\al} \in \MB_x$. By construction, 
			\begin{align*}
				x 
				& \in \prod_{\al \in A} U_{\al} \\
				& \subset \prod_{\al \in A} B_{\al} \\
				& \subset U
			\end{align*}
			Since $U \in \MT$ such that $x \in U$ is arbitrary, we have that $\MB_x$ is a local basis for $\MT$ at $x$. 
		\end{enumerate}
	\end{proof}

	\begin{ex}  \lex{ex:product_topology:0004}
		Let $(X_j, \MT_j)_{j =1}^n$ be a collection of topological spaces. Set 
		$$\MB = \bigg \{\prod\limits_{j=1}^n A_j: \text{for each $j \in \{1, \ldots, n\}$, } A_j \in \MT_{j} \bigg \}$$ 
		Then $\MB$ is a basis for the product topology on $\prod_{j=1}^n X_j$.
	\end{ex}

	\begin{proof}
		Clear by previous exercise.
	\end{proof}

	\begin{ex}  \lex{ex:product_topology:0005} \tbf{Basis for Product Topology:} \\
		Let $(X_{\al}, \MT_{\al})_{\al \in A}$ be a collection of topological spaces and for each $\al \in A$, $\MB_{\al}$ a basis for $\MT_{\al}$. Set $X = \prod_{\al \in A} X_{\al}$ and denote the product topology on $X$ by $\MT_X$. Set 
		\begin{align*}
			\MB = 
			& \bigg \{\prod_{\al \in A} U_{\al}: \text{there exists $J \subset A$ such that $ \# J < \infty$, } \\
			& \text{ for each $\al \in J$, $U_{\al} \in \MB_{\al}$ and for each $\al \in J^c$, $U_{\al} = X_{\al}$ } \bigg \}
		\end{align*} 
		Then $\MB$ is a basis for $\MT_X$.
	\end{ex}

	\begin{proof}
		Set 
		$$\MB' = \bigg \{\prod_{\al \in A}V_{\al}: \text{ for each $\al \in A$,  $V_{\al} \in \MT_{\al}$ and $\# \{\al \in A: V_{\al} \neq X_{\al}\} < \infty$} \bigg\}$$
		The previous exercise implies that $\MB'$ is a basis for $\MT_X$. Then $\MB \subset \MB' \subset \MT_X$. Let $V \in \MT$ and $x \in V$. Write $x = (x_{\al})_{\al \in A}$. Since $\MB'$ is a basis for $\MT_X$, for each $\al \in A$, there exists $V_{\al} \in \MT_{\al}$ such that for finitely many $\al \in A$, $V_{\al} \neq X_{\al}$ and $x \in \prod\limits_{\al \in A} V_{\al} \subset V$. Define $J \subset A$ by $J = \{\al \in A: V_{\al} \neq X_{\al}\}$. Then $\#J < \infty$. Let $\al \in J$. Then $x_{\al} \in V_{\al}$. Since $\MB_{\al}$ is a basis for $\MT_{\al}$, there exists $U'_{\al} \in \MB_{\al}$ such that $x_{\al} \in U'_{\al} \subset V_{\al}$. For $\al \in A$, define $U_{\al} \in \MT_{\al}$ by 
		\[
		U_{\al} =
		\begin{cases}
			U'_{\al} & \al \in J \\
			X_{\al} & \al \in J^c
		\end{cases}
		\]
		Set $U = \prod_{\al \in A} U_{\al}$. Then $U \in \MB$ and 
		\begin{align*}
			x 
			& \in U \\
			& = \prod_{\al \in A} U_{\al} \\
			& \subset \prod\limits_{\al \in A} V_{\al} \\
			& \subset V
		\end{align*}
		Hence $\MB$ is a basis for $\MT_X$.
	\end{proof}






























	\subsection{Characteristics of the Product Topology}

	\begin{ex}  \lex{ex:product_topology:0006}
		Let $(X_{\al}, \MT_{\al})_{\al \in A}$ be a collection of topological spaces and for each $\al \in A$, $E_{\al} \subset X_{\al}$. Then 
		$$\cl \bigg( \prod_{\al \in A} E_{\al} \bigg) = \prod_{\al \in A} \cl E_{\al}$$ 
		\tbf{Hint:} \rex{31017.1}
	\end{ex}

	\begin{proof}
		Since for each $\al \in A$, $\pi_{\al}: \prod_{\al \in A}X_{\al} \rightarrow X_{\al}$ is continuous and $\cl E_{\al}$ is closed, we have that for each $\al \in A$, $ \pi_{\al}^{-1}(\cl E_{\al})$ is closed and thus 
		$$\prod_{\al \in A} \cl E_{\al} = \bigcap_{\al \in A} \pi_{\al}^{-1}(\cl E_{\al})$$
		is closed. Since for each $\al \in A$, $E_{\al} \subset \cl E_{\al}$, we have that 
		$$ \prod_{\al \in A}  E_{\al}  \subset \prod_{\al \in A} \cl E_{\al}$$ 
		which implies that 
		$$ \cl \bigg( \prod_{\al \in A}  E_{\al} \bigg)  \subset \prod_{\al \in A} \cl E_{\al}$$ 
		Conversely, let $x \in \prod\limits_{\al \in A} \cl E_{\al}$ and $U \in \MN(x)$. Suppose that $U$ is open. Set 
		$$\MB = \bigg \{\prod_{\al \in A}B_{\al}: \text{ for each $\al \in A$,  $B_{\al} \in \MT_{\al}$ and $\# \{\al \in A: B_{\al} \neq X_{\al}\} < \infty$} \bigg\}$$
		A previous exercise implies that $\MB$ is a basis for the product topology on $\prod_{\al \in A}X_{\al}$. So for each $\al \in A$, there exists $U_{\al} \in \MT_{\al}$ such that $\# \{\al \in A: U_{\al} \neq X_{\al}\} < \infty$ and $x \in \prod\limits_{\al \in A} U_{\al} \subset U$. Then for each $\al \in A$, $x_{\al} \in \cl E_{\al} \cap U_{\al}$. Let $\al \in A$. Since $x_{\al} \in \cl E_{\al}$ and $U_{\al} \in \MN(x_{\al})$ is an open neighborhood of $x_{\al}$, \rex{31017.1} implies that $E_{\al} \cap U_{\al} \neq \varnothing$. Since $\al \in A$ is arbitrary, for each $\al \in A$, $E_{\al} \cap U_{\al} \neq \varnothing$. The axiom of choice implies that there exists
		\begin{align*}
			y 
			& \in \prod\limits_{\al \in A}  (E_{\al} \cap U_{\al} ) \\
			& = \bigg( \prod\limits_{\al \in A} E_{\al} \bigg) \cap \prod_{\al \in A} U_{\al} \\
			& \subset \bigg( \prod\limits_{\al \in A} E_{\al} \bigg) \cap U
		\end{align*}
		Hence $\bigg( \prod\limits_{\al \in A} E_{\al} \bigg) \cap U \neq \varnothing$. Since $U \in \MN(x)$ is an arbirtary open neighborhood of $x$, \rex{31017.1} implies that $x \in \cl \prod\limits_{\al \in A} E_{\al}$. Since $x \in \prod\limits_{\al \in A} \cl E_{\al}$ is arbitrary, $\prod\limits_{\al \in A} \cl E_{\al} \subset  \cl \prod\limits_{\al \in A} E_{\al}$. \\
		Hence $\prod\limits_{\al \in A} \cl E_{\al} = \cl \prod\limits_{\al \in A} E_{\al}$
	\end{proof}

	\begin{ex} \lex{ex:product_topology:0006.1}
		Let $(X_{\al}, \MT_{\al})_{\al \in A}$ be a collection of topological spaces, $(a_{\gam})_{\gam \in \Gam} \subset \prod_{\al \in A} X_{\al}$ a net and $a \in \prod_{\al \in A} X_{\al}$. Then $a_{\gam} \rightarrow a$ in $(\prod_{\al \in A} X_{\al}, \bigotimes_{\al \in A} \MT_{\al})$ iff for each $\al \in A$, $\pi_{\al}(a_{\gam}) \conv{\gam} \pi_{\al}(a)$ in $(X_{\al}, \MT_{\al})$. 
	\end{ex}

	\begin{proof} 
		Clear by \rex{ex:nets:0020}.
	\end{proof}

	\begin{ex} \lex{ex:product_topology:0006.2}
		Define $(e_n)_{n \in \N} \subset \R^{\N}$ by $\pi_k(e_n) \defeq \del_{n,k}$. Then $e_n \rightarrow 0$ in $((\R^{\N}), \MT_{\R}^{\otimes \N})$.
	\end{ex}

	\begin{proof}
		Let $k \in \N$. Then 
		\begin{align*}
			\pi_k(e_n) 
			& = \del_{n,k} \\
			& \conv{n} 0. 
		\end{align*}
		Since $k \in \N$ is arbitrary, we have that for each $k \in \N$, $\pi_k(e_n) \rightarrow 0$. \rex{ex:product_topology:0006.1} implies that $e_n \rightarrow 0$ in $((\R^{\N}), \MT_{\R}^{\otimes \N})$.
	\end{proof}



















	
	
	
	
	
	
	
	
	\subsection{Maps and the Product Topology}
	
	\begin{ex}  \lex{ex:product_topology:0007}
		Let $X$ be a topological space, $(Y_{\al}, \MT_{\al})_{\al \in A}$ a collection of topological spaces and $f: X \rightarrow \prod_{\al \in A}Y_{\al}$. Then $f$ is continuous iff for each $\al \in A$, $\pi_{\al} \circ f$ is continuous.
	\end{ex}

	\begin{proof}
		Immediate by a \rex{ex:continuous_maps:0009}.
	\end{proof}
	
	\begin{ex} \lex{ex:product_topology:0010}
		Let $(X_{\al}, \MT_{\al})_{\al \in A}$ and $(Y_{\al}, \MS_{\al})_{\al \in A}$ be collections of topological spaces and $(f_{\al})_{\al \in A} \in \prod\limits_{\al \in A} Y_{\al}^{X_{\al}}$, i.e. for each $\al \in A$, $f_{\al}:X_{\al} \rightarrow Y_{\al}$. If for each $\al \in A$, $f_{\al}$ is continuous, then $\prod\limits_{\al \in A} f_{\al}$ is continous.
	\end{ex}

	\begin{proof}
		Set $X \defeq \prod\limits_{\al \in A} X_{\al}$, $Y \defeq \prod\limits_{\al \in A}Y_{\al}$ and define $f: X \rightarrow Y$ by $f \defeq \prod\limits_{\al \in A} f_{\al}$. Denote the $\al$-th projection maps on $X$ and $Y$ by $\pi^X_{\al}$ and $\pi^Y_{\al}$ respectively. Set $\MT \defeq \bigotimes\limits_{\al \in A}\MT_{\al}$ and $\MS \defeq \bigotimes\limits_{\al \in A} \MS_{\al}$. Suppose that for each $\al \in A$, $f_{\al}$ is continuous. \rex{ex:set_theory:products:0005} implies that for each $\al \in A$, $\pi^Y_{\al} \circ f = f_{\al} \circ \pi^X_{\al}$. Let $\al \in A$. Then $f_{\al} \circ \pi^X_{\al}$ is continuous. Hence $\pi^Y_{\al} \circ f$ is continuous. Since $\al \in A$ is arbitrary, we have that for each $\al \in A$, $\pi^Y_{\al} \circ f$ is continuous. \rex{ex:product_topology:0007} implies that $f$ is continuous. 
	\end{proof}

	\begin{ex} \lex{ex:product_topology:0011}
		Let $(X_{\al}, \MT_{\al})_{\al \in A}$ and $(Y_{\al}, \MS_{\al})_{\al \in A}$ be collections of topological spaces and $(f_{\al})_{\al \in A} \in \prod\limits_{\al \in A} Y_{\al}^{X_{\al}}$, i.e. for each $\al \in A$, $f_{\al}:X_{\al} \rightarrow Y_{\al}$. If $\# \{\al \in A: f_{\al} \text{ is not surjective}\} < \infty$ and for each $\al \in A$, $f_{\al}$ is open, then $\prod\limits_{\al \in A} f_{\al}$ is open.
	\end{ex}

	\begin{proof} Set $X \defeq \prod\limits_{\al \in A} X_{\al}$, $Y \defeq \prod\limits_{\al \in A}Y_{\al}$ and define $f: X \rightarrow Y$ by $f \defeq \prod\limits_{\al \in A} f_{\al}$. Denote the $\al$-th projection maps on $X$ and $Y$ by $\pi^X_{\al}$ and $\pi^Y_{\al}$ respectively. Set $\MT \defeq \bigotimes\limits_{\al \in A}\MT_{\al}$ and $\MS \defeq \bigotimes\limits_{\al \in A} \MS_{\al}$. Suppose that $\# \{\al \in A: f_{\al} \text{ is not surjective}\} < \infty$ and for each $\al \in A$, $f_{\al}$ is open. Set 
		$$\MB_X \defeq \bigg \{\prod_{\al \in A}U_{\al}: \text{ for each $\al \in A$,  $U_{\al} \in \MT_{\al}$ and $\# \{\al \in A: U_{\al} \neq X_{\al}\} < \infty$} \bigg\}$$
		$$\MB_Y \defeq \bigg \{\prod_{\al \in A}V_{\al}: \text{ for each $\al \in A$,  $V_{\al} \in \MS_{\al}$ and $\# \{\al \in A: V_{\al} \neq Y_{\al}\} < \infty$} \bigg\}$$
		A previous exercise implies that $\MB_X$ is a basis for $\MT$ and $\MB_Y$ is a basis for $\MS$. Let $U \in \MB_X$. Then for each $\al \in A$ there exist $U_{\al} \in \MT_{\al}$ such that $U = \prod\limits_{\al \in A} U_{\al}$. Define 
		\begin{itemize}
			\item $B_1 \defeq \{\al \in A: f_{\al} \text{ is not surjective}\}$ 
			\item $B_2 \defeq \{\al \in A: U_{\al} \neq X_{\al}\}$ 
			\item $B_3 \defeq \{\al \in A: f_{\al}(U_{\al}) \neq Y_{\al}\}$
		\end{itemize}
		Let $\al \in A$. Suppose that $\al \in B_1^c \cap B_2^c$. Then $f_{\al}$ is surjective and $U_{\al} = X_{\al}$. Thus $f_{\al}(U_{\al}) = Y_{\al}$ and $\al \in B_3^c$. Therefore if $\al \in B_3$, then $\al \in B_1 \cup B_2$. Since $\al \in A$ is arbitrary, $B_3 \subset B_1 \cup B_2$. By assumption, $\# B_1 < \infty$ and $\# B_2 < \infty$. Thus 
		\begin{align*}
			\# B_3
			& \leq \# (B_1 \cup B_2) \\
			& \leq \# B_1 + \# B_2 \\
			& < \infty
		\end{align*}
		Since for each $\al \in A$, $f_{\al}$ is open, we have that for each $\al \in A$, $f_{\al}(U_{\al}) \in \MS_{\al}$. Thus
		\begin{align*}
			f\bigg( \prod\limits_{\al \in A} U_{\al}\bigg) 
			& = \prod_{\al \in A} f_{\al}(U_{\al}) \\
			& \in \MB_{Y} \\
			& \subset \MS
		\end{align*}
		Since $U \in \MB_X$ is arbitrary, we have that for each $U \in \MB_X$, $f(U) \in \MS$. Since $\MB_X$ is a basis for $\MT$, an exercise about open maps in the section on continuous maps implies that $f$ is open.
	\end{proof}

	\begin{ex} \lex{ex:product_topology:0012}
		Let $(X_{\al}, \MT_{\al})_{\al \in A}$ and $(Y_{\al}, \MS_{\al})_{\al \in A}$ be collections of topological spaces and for each $\al \in A$, $U_{\al} \subset X_{\al}$, $V_{\al} \subset Y_{\al}$ and $f_{\al}:U_{\al} \rightarrow V_{\al}$. If for each $\al \in A$, $f_{\al}$ is $(\MT_{\al} \cap U_{\al})$-continuous, then $\prod\limits_{\al \in A} f_{\al}$ is $\bigg( \bigg[ \bigotimes\limits_{\al \in A} \MT_{\al} \bigg] \cap \bigg[ \prod\limits_{\al \in A} U_{\al} \bigg], \bigg[ \bigotimes\limits_{\al \in A} \MS_{\al} \bigg] \cap \bigg[ \prod\limits_{\al \in A} V_{\al} \bigg] \bigg)$-continuous.
	\end{ex}
	
	\begin{proof}
		Denote the $\al$-th projection maps on $X$ and $Y$ by $\pi^X_{\al}$ and $\pi^Y_{\al}$ respectively. Let $(x_{\gam})_{\gam \in \Gam} \subset \prod\limits_{\al \in A} U_{\al}$ and $x \in \prod\limits_{\al \in A}$. Suppose that $x_{\gam} \rightarrow x$ in $\bigg( \prod\limits_{\al \in A} U_{\al} , \bigg[ \bigotimes\limits_{\al \in A} \MT_{\al} \bigg] \cap \bigg[ \prod\limits_{\al \in A} U_{\al} \bigg] \bigg)$. \tcb{An exercise in the section on the subspace topology} implies that $x_{\gam} \rightarrow x$ in $\bigg( \prod\limits_{\al \in A} X_{\al}, \bigotimes\limits_{\al \in A} \MT_{\al}  \bigg)$. Let $\al \in A$. Since $\pi^X_{\al}$ is $\bigg(\bigotimes\limits_{\be \in A} \MT_{\be}, \MT_{\al} \bigg)$-continuous, we have that $\pi^X_{\al}(x_{\gam}) \rightarrow \pi^X_{\al}(x)$ in $(X_{\al}, \MT_{\al})$. \tcb{Another application of the same exercise} implies that $\pi^X_{\al}(x_{\gam}) \rightarrow \pi^X_{\al}(x)$ in $(U_{\al}, \MT_{\al} \cap U_{\al})$. Since $f_{\al}$ is $\bigg(\MT_{\al} \cap U_{\al}, \MS_{\al} \cap V_{\al} \bigg)$-continuous, 
		\begin{align*}
			\pi^Y_{\al} \circ \bigg[ \prod\limits_{\al \in A}f_{\al} \bigg] (x_{\gam})
			& = f_{\al} \circ \pi^X_{\al}(x_{\gam}) \\
			& \rightarrow f_{\al} \circ \pi^X_{\al}(x) \\
			& = \pi^Y_{\al} \circ \bigg[ \prod\limits_{\al \in A}f_{\al} \bigg] (x)  \text{ in $(V_{\al}, \MS_{\al} \cap V_{\al})$} 
		\end{align*}
		\tcb{Another application of the exercise} implies that $\pi^Y_{\al} \circ \bigg[ \prod\limits_{\al \in A}f_{\al} \bigg] (x_{\gam}) \rightarrow \pi^Y_{\al} \circ \bigg[ \prod\limits_{\al \in A}f_{\al} \bigg] (x) $ in $(Y_{\al}, \MS_{\al})$. Since $(x_{\gam})_{\gam \in \Gam} \subset \prod\limits_{\al \in A} U_{\al}$ and $x \in \prod\limits_{\al \in A}$ with $x_{\gam} \rightarrow x$ in $\bigg( \prod\limits_{\al \in A} U_{\al} , \bigg[ \bigotimes\limits_{\al \in A} \MT_{\al} \bigg] \cap \bigg[ \prod\limits_{\al \in A} U_{\al} \bigg] \bigg)$ is arbitary, we have that $\prod\limits_{\al \in A}f_{\al}$ is $\bigg( \bigg[ \bigotimes\limits_{\al \in A} T_{\al} \bigg]  \cap \bigg[ \prod\limits_{\al \in A} U_{\al} \bigg], \bigotimes\limits_{\al \in A} S_{\al} \bigg)$-continuous. \tcb{Another application of the exercise} implies that $\prod\limits_{\al \in A}f_{\al}$ is $\bigg( \bigg[ \bigotimes\limits_{\al \in A} T_{\al} \bigg]  \cap \bigg[ \prod\limits_{\al \in A} U_{\al} \bigg], \bigg[ \bigotimes\limits_{\al \in A} S_{\al} \bigg] \cap \bigg[ \prod\limits_{\al \in A} V_{\al} \bigg] \bigg)$-continuous.
	\end{proof}

	\begin{ex} \lex{ex:product_topology:0012.1}
		Let $(X, \MT_X)$ be a topological space, $(Y_{\al}, \MT_{Y_{\al}})_{\al \in A}$ a collection of topological spaces, $(f_{\al})_{\al \in A} \in \prod\limits_{\al \in A} Y_{\al}^X$. If for each $\al \in A$, $f_{\al}$ is $(\MT_X, \MT_{Y_{\al}})$-continuous, then $(f_{\al})_{\al \in A}$ is $(\MT_X, \bigotimes\limits_{\al \in A} \MT_{Y_{\al}})$-continuous.
	\end{ex}

	\begin{proof}
		Suppose that for each $\al \in A$, $f_{\al}$ is $(\MT_X, \MT_{Y_{\al}})$-continuous. Set $f \defeq (f_{\al})_{\al \in A}$. Since for each $\al \in A$, $\pi_{\al} \circ f = f_{\al}$, \rex{ex:product_topology:0007} implies that $f$ is $(\MT_X, \bigotimes\limits_{\al \in A} \MT_{Y_{\al}})$-continuous.
	\end{proof}




	\begin{defn}
		\tcr{define slice maps and then define the below sets in terms of them in the set theory section.}
	\end{defn}
	
	\begin{ex} \lex{ex:product_topology:0013}
		Let $X$ and $Y$ be topological spaces and $U \subset X \times Y$ open. Then for each $(x_0,  y_0) \in U$, $U^{x_0}$ and $U^{y_0}$ are open.
	\end{ex}
	
	\begin{proof}
		Let $(x_0, y_0) \in U$. Define $\phi: X \rightarrow X \times Y$ by $\phi(x) = (x, y_0)$. Since $\pi_X \circ \phi = \id_X$ and $\pi_Y \circ \phi$ is constant, $\pi_X \circ \phi$ and $\pi_Y \circ \phi$ are continous. Therefore, $\phi$ is continuous. Then $U^{y_0}$ is open since $U$ is open and $\phi^{-1}(U) = U^{y_0}$. Similarly, $U_{x_0}$ is open.
	\end{proof}

	\begin{ex} \lex{ex:product_topology:0014}
		Let $X$, $Y$ and $Z$ be topological spaces, $U \subset X \times Y$ open and $f: U \rightarrow Z$. Equip $U$ with the subspace topology. Suppose that $f$ is continuous. Let $(x_0, y_0) \in U$. Equip $U_{x_0}$ and $U^{y_0}$ with the subspace topology. Then $f_{x_0}:U_{x_0} \rightarrow Z$ and $f^{y_0}: U^{y_0} \rightarrow Z$ are continuous.
	\end{ex}

	\begin{proof}
		Let $(x_0, y_0) \in U$. Let $V \subset Z$. Suppose that $V$ is open. Continuity of $f$ implies that $f^{-1}(V)$ is open in $U$. Since $U$ is open in $X \times Y$, $f^{-1}(V)$ is open in $X \times Y$. A previous exercise in the section on product sets implies that $(f^{y_0})^{-1}(V) = (f^{-1}(V))^{y_0}$. The previous exercise implies that $(f^{-1}(V))^{y_0}$ is open in $X$. So $(f^{y_0})^{-1}(V)$ is open in $X$. Since $(f^{y_0})^{-1}(V) \subset U^{y_0}$, $(f^{y_0})^{-1}(V)$ is open in $U^{y^0}$. Thus $f^{y_0}: U^{y_0} \rightarrow Z$ is continuous. Similarly, $f_{x_0}: U_{x_0} \rightarrow Z$ is continuous.
	\end{proof}
	
	
	
	
	
	
	
	
	
	
	
	
	
	
	
	
	
	
	
	
	
	
	
	
	
	
	
	
	
	
	
	
	
	
	
	
	
	
	
	
	
	
	
	
	
	
	
	
	
	\newpage
	\section{Coproduct Topology}

	\begin{defn} \ld{def:topology:coproducts:0001}
			Let $(X_{\al}, \MT_{\al})_{\al \in A}$ be a collection of topological spaces. We define the \textbf{coproduct topology} on $\coprod\limits_{\al \in A}X_{\al}$, denoted $\bigoplus\limits_{\al \in A} \MA_{\al}$, by 
		$$\bigoplus\limits_{\al \in A} \MT_{\al} = \tau(\iota_{\al}: \al \in A)$$
		i.e. $\bigoplus\limits_{\al \in A} \MT_{\al}$ is the final topology on $\coprod\limits_{\al \in A} X_{\al}$ generated by the embedding maps $(\iota_{\al})_{\al \in A}$.
	\end{defn}

	\begin{ex} \lex{ex:topology:coproducts:0002}
		Let $(X_{\al}, \MT_{\al})_{\al \in A}$ be a collection of topological spaces. Then $$\bigoplus\limits_{\al \in A} \MT_{\al} = \bigg\{V \subset \coprod_{\al \in A}  X_{\al}: \text{ for each $\al \in A$, $\iota_{\al}^{-1}(V) \in \MT_{\al}$} \bigg\}$$
	\end{ex}
	
	\begin{proof}
			Set $X \defeq \coprod\limits_{\al \in A}  X_{\al}$ and $\MT \defeq \bigg \{V \subset \coprod\limits_{\al \in A}  X_{\al}: \text{ for each $\al \in A$, $\iota_{\al}^{-1}(V) \in \MT_{\al}$} \bigg\}$. 
		\begin{itemize}
			\item 
			\begin{enumerate}
				\item Clearly $\varnothing, X \in \MT$.
				\item Let $(U_{\gam})_{\gam \in \Gam} \subset \MT$. Then by definition, for each $\gam \in \Gam$ and $\al \in A$, $\iota_{\al}^{-1}(U_{\gam}) \in \MT_{\al}$. Hence
				\begin{align*}
					\iota_{\al}^{-1} \bigg( \bigcup_{\gam \in \Gam} U_{\gam} \bigg)
					& = \bigcup_{\gam \in \Gam} \iota_{\al}^{-1}(U_{\gam}) \\
					& \in \MT_{\al}
				\end{align*}
				\item Let $(U_j)_{j=1}^n \subset \MT$. Then by definition, for each $\al \in A$ and $j \in [n]$, $\iota_{\al}^{-1}(U_j) \in \MT_{\al}$. Hence 
				\begin{align*}
					\iota_{\al}^{-1}\bigg( \bigcap_{j=1}^n U_j \bigg)
					& = \bigcap_{j=1}^n \iota_{\al}^{-1}(U_j)\\
					& \in \MT_{\al}
				\end{align*}
			\end{enumerate}
			So $\MT$ is a topology on $X$.
			\item Since $\MT$ is a topology on $X$, we have that $\MT = \tau_X(\MT)$. By \rd{def:continuous_maps:0011}, 
			\begin{align*}
				\MT
				& = \tau_X(\MT) \\
				& = \tau_X(\iota_{\al}: \al \in A) \\
				& = \bigoplus\limits_{\al \in A} \MT_{\al}.
			\end{align*}
		\end{itemize}
	\end{proof}

	\begin{ex} \lex{ex:topology:coproducts:0003}
		Let $(X_{\al}, \MT_{\al})_{\al \in A}$ be a collection of measurable spaces. Then 
		$$\bigoplus\limits_{\al \in A} \MT_{\al} = \bigg \{\coprod_{\al \in A} B_{\al}: B_{\al} \in \MT_{\al} \bigg\} $$
	\end{ex}
	
	\begin{proof}
		Set  
		\begin{itemize}
			\item $\MF= \{V \subset \coprod_{\al \in A}  X_{\al}: \text{ for each $\al \in A$, $\iota_{\al}^{-1}(V) \in \MT_{\al}$}\}$ 
			\item $\MG = \bigg \{  \coprod\limits_{\al \in A}  B_{\al}: \text{ for each $\al \in A$, } B_{\al} \in \MT_{\al} \bigg\}$
		\end{itemize}
		Let $V \in \MG$. Then for each $\al \in A$, there exists $B_{\al} \in \MT_{\al}$ such that $V =  \coprod\limits_{\al \in A}  B_{\al}$. Therefore, for each $\al \in A$, 
		\begin{align*}
			\iota_{\al}^{-1}(V)
			& = \iota_{\al}^{-1} \bigg( \coprod\limits_{\al \in A}  B_{\al} \bigg) \\
			& = B_{\al} \\
			& \in \MT_{\al}
		\end{align*}
		Hence $V \in \MF$. Since $V \in \MG$ is arbitrary, $\MG \subset \MF$. \\
		Conversely, let $V \in \MF$. Then for each $\al \in A$, $\iota_{\al}^{-1}(V) \in \MT_{\al}$. For each $\al \in A$, define $B_{\al} \in \MT_{\al}$ by $B_{\al} = \iota_{\al}^{-1}(V)$. Then 
		\begin{align*}
			V 
			& = \coprod\limits_{\al \in A} B_{\al} \\
			& \in \MG 
		\end{align*}
		Since $V \in \MF$ is arbitrary, $\MF \subset \MG$. The previous exercise implies that 
		\begin{align*}
			\MG
			& = \MF \\
			& = \bigoplus\limits_{\al \in A} \MT_{\al}
		\end{align*}
	\end{proof}

	\begin{ex} \lex{ex:topology:coproducts:0003.1}
		Let $(X_{\al}, \MT_{\al})_{\al \in A}$ be a collection of topological spaces. Then for each $\al \in A$, $\iota_{\al}: X_{\al} \rightarrow \coprod\limits_{\al' \in A} X_{\al'}$ is $\bigg(\MT_{\al}, \bigoplus\limits_{\al' \in A} \MT_{\al'} \bigg)$-open.
	\end{ex}

	\begin{proof}
		Let $\al, \be \in A$ and $U \in \MT_{\al}$. Then  
		\begin{align*}
			\iota_{\be}^{-1}(\iota_{\al}(U))  
			& = \iota_{\be}^{-1}(\{\al\} \times U) \\
			& = 
			\begin{cases}
				U, & \be = \al \\
				\varnothing, & \be \neq \al
			\end{cases} \\
			& \in \MT_{\be}
		\end{align*}
		Since $\be \in A$ is arbitrary, we have that for each $\be \in A$, $\iota_{\be}^{-1}(\iota_{\al}(U)) \in \MT_{\be}$. \rex{ex:topology:coproducts:0002} implies that $\iota_{\al}(U) \in \bigoplus\limits_{\be \in A} \MT_{\be}$. Since $U \in \MT_{\al}$ is arbitrary, we have that $\bigg(\MT_{\al}, \bigoplus\limits_{\al' \in A} \MT_{\al'} \bigg)$-open.
	\end{proof}

	\begin{ex} \lex{ex:topology:coproducts:0003.11}
		Let $(X_{\al}, \MT_{\al})_{\al \in A}$ be a collection of topological spaces. Then for each $\al \in A$, $\iota_{\al}: X_{\al} \rightarrow \coprod\limits_{\al' \in A} X_{\al'}$ is $\bigg(\MT_{\al}, \bigoplus\limits_{\al' \in A} \MT_{\al'} \bigg)$-closed.
	\end{ex}
	
	\begin{proof}
		Let $\al_0 \in A$ and $C \subset X_{\al_0}$. Suppose that $C$ is closed in $(X_{\al_0}, \MT_{\al_0})$. Define $(B_{\al})_{\al \in A} \in \prod\limits_{\al \in A} \MT_{\al}$ by  
		\[
		B_{\al} \defeq 
		\begin{cases}
			C^c, & \al = \al_0 \\
			X_{\al}, & \al \neq \al_0.
		\end{cases}
		\]
		Then 
		\begin{align*}
			[\iota_{\al_0}(C)]^c
			& = [\{\al_0\} \times C]^c \\
			& = \coprod_{\al \in A} B_{\al} \\
			& \in \bigoplus_{\al \in A} \MT_{\al}.
		\end{align*}
		Hence $\iota_{\al_0}(C)$ is closed in $(\coprod\limits_{\al \in A} X_{\al}, \bigoplus\limits_{\al \in A} \MT_{\al})$. Since $C \subset X_{\al_0}$ with $C$ closed in $(X_{\al_0}, \MT_{\al_0})$ is arbitrary, we have that for each $C \subset X_{\al_0}$, $C$ closed in $(X_{\al_0}, \MT_{\al_0})$ implies that $\iota_{\al_0}(C)$ is closed in $(\coprod\limits_{\al \in A} X_{\al}, \bigoplus\limits_{\al \in A} \MT_{\al})$. Hence $\iota_{\al_0}$ is $\bigg(\MT_{\al_0}, \bigoplus\limits_{\al' \in A} \MT_{\al'} \bigg)$-closed. Since $\al_0 \in A$ is arbitrary, we have that for each $\al \in A$, $\iota_{\al}$ is $\bigg(\MT_{\al}, \bigoplus\limits_{\al' \in A} \MT_{\al'} \bigg)$-closed.
	\end{proof}

	\begin{ex} \lex{ex:topology:coproducts:0003.2}
		Let $(X_{\al}, \MT_{\al})_{\al \in A}$ be a collection of topological spaces and for each $\al \in A$, $\MB_{\al} \subset \MT_{\al}$. Suppose that for each $\al \in A$, $B_{\al}$ is a basis for $\MT_{\al}$. Then 
		$\{ \iota_{\al}(U) : \text{$\al \in A$ and $U \in \MB_{\al}$} \}$ is a basis for $\bigoplus\limits_{\al \in A} \MT_{\al}$.
	\end{ex}

	\begin{proof}
		Set $\MB \defeq \{ \iota_{\al}(U) : \text{$\al \in A$ and $U \in \MB_{\al}$} \}$ and $\MT \defeq \bigoplus\limits_{\al \in A} \MT_{\al}$. \rex{ex:topology:coproducts:0003.1} implies that $\MB \subset \MT$. Let $W \in \MT$ and $(\al_0, x_0) \in W$. Set $U \defeq \iota_{\al_0}^{-1}(W)$. By assumption, $x_0 \in U$. Since $\iota_{\al_0}$ is $(\MT_{\al_0}, \MT)$-continuous, we have that $U \in \MT_{\al_0}$. \rex{ex:topology:coproducts:0003.1} implies that  
		\begin{align*}
			\{\al_0 \} \times U
			& = \iota_{\al_0}(U) \\
			& \in \MT)
		\end{align*}
		By construction, 
		\begin{align*}
			(\al_0, x_0)
			& \in \{\al_0 \} \times U \\
			& = \iota_{\al_0}(U) \\
			& = \iota_{\al_0} (\iota_{\al_0}^{-1}(W)) \\
			& \subset W.
		\end{align*}
		Since $W \in \MT$ and $(\al_0, x_0) \in W$ are arbitrary, we have that for each $W \in \MT$ and $(\al_0, x_0) \in W$, there exists $B \in \MB$ such that $(\al_0, x_0) \in B \subset W$. Therefore $\MB$ is a basis for $\MT$.
	\end{proof}

	\begin{note}
		Let $\Gam$ be a directed set and $\gam_0 \in \Gam$. From \rd{def:set_theory:nets:0008.1} we recall the $\gam_0$-tail operator $L_{\gam_0}: X^{\Gam} \rightarrow X^{[\gam_0, \infty)}$. 
	\end{note}

	\begin{ex} \lex{ex:topology:coproducts:0003.3}
		Let $(X_{\al}, \MT_{\al})_{\al \in A}$ be a collection of topological spaces, $(\al_{\gam}, x_{\gam})_{\gam \in \Gam} \subset \coprod\limits_{\al \in A} X_{\al}$ a net and $(\al_0, x_0) \in \coprod\limits_{\al \in A} X_{\al}$. Then $(\al_{\gam}, x_{\gam}) \rightarrow (\al_0, x_0)$ in $\bigg( \coprod\limits_{\al \in A} X_{\al}, \bigoplus\limits_{\al \in A} \MT_{\al} \bigg)$ iff there exists $\gam_0 \in \Gam$ such that
		\begin{enumerate}
			\item for each $\gam \in \Gam$, $\gam \geq \gam_0$ implies that $\al_{\gam} = \al_0$ and $x_{\gam} \in X_{\al_0}$
			\item $[L_{\gam_0}(x)]_{\gam} \rightarrow x_0$ in $(X_{\al_0}, \MT_{\al_0})$.
		\end{enumerate}
	\end{ex}

	\begin{proof}
		Set $X \defeq \coprod\limits_{\al \in A} X_{\al}$ and $\MT \defeq \bigoplus\limits_{\al \in A} \MT_{\al}$.
		\begin{itemize}
			\item $(\implies):$ \\
			\begin{enumerate}
				\item Suppose that $(\al_{\gam}, x_{\gam}) \rightarrow (\al_0, x_0)$ in \rex{ex:topology:coproducts:0003.1} implies that $\{\al_0\} \times X_{\al_0} \in \MN(\al_0, x_0)$. Since $(\al_{\gam}, x_{\gam}) \rightarrow (\al_0, x_0)$, $(\al_{\gam}, x_{\gam})_{\gam \in \Gam}$ is eventually in $\{\al_0\} \times X_{\al_0}$. \rex{ex:nets:0009.1} implies that there exists $\gam_0 \in \Gam$ such that $(\al_{\gam}, x_{\gam})_{\gam \in [\gam_0, \infty)}  \subset \{\al_0\} \times X_{\al_0}$. Hence for each $\gam \in \Gam$, $\gam \geq \gam_0$ implies that $\al_{\gam} = \al_0$ and $(x_{\gam})_{\gam \in [\gam_0, \infty)} \subset X_{\al_0}$.
				\item Define $(y_{\gam})_{\gam \in [\gam_0, \infty)} \subset X_{\al_0}$ by $y \defeq L_{\gam_0}(x)$.   Let $U \in \MT_{\al_0}$. Suppose that $x_0 \in U$. \rex{ex:topology:coproducts:0003.1} implies that $\{\al_0\} \times U \in \MN(\al_0, x_0)$. Since $(\al_{\gam}, x_{\gam}) \rightarrow (\al_0, x_0)$ in $(X, \MT)$, $(\al_{\gam}, x_{\gam})_{\gam \in \Gam}$ is eventually in $\{\al_0\} \times U$. \rex{ex:nets:0009.1} implies that there exists $\gam_1 \in \Gam$ such that $(\al_{\gam}, x_{\gam})_{\gam \in [\gam_1, \infty)}  \subset \{\al_0\} \times U$. Since $A$ is a directed set, there exists $\gam_2 \in \Gam$ such that $\gam_0, \gam_1 \leq \gam_2$. Hence for each $\gam \in [\gam_2, \infty)$, $y_{\gam} \in U$. So $(y_{\gam})_{\gam \in [\gam_0, \infty)}$ is eventually in $U$. Since $U \in \MT_{\al_0}$ with $x_0 \in U$ is arbitrary, we have that $y_{\gam} \rightarrow x_0$. 
			\end{enumerate}
			\item $(\impliedby):$ \\
			Suppose that there exists $\gam_0 \in \Gam$ such that
			\begin{enumerate}
				\item for each $\gam \in \Gam$, $\gam \geq \gam_0$ implies that $\al_{\gam} = \al_0$ and $x_{\gam} \in X_{\al_0}$
				\item $[L_{\gam_0}(x)]_{\gam} \rightarrow x_0$ in $(X_{\al_0}, \MT_{\al_0})$.
			\end{enumerate}
			Let $W_0 \in \MN(\al_0, x_0)$. Set $W \defeq \Int W_0$. Then $W \in \MT$ and $(\al_0, x_0) \in W$. \rex{ex:topology:coproducts:0003.2} implies that there exists $U \in \MT_{\al_0}$ such that $(\al_0, x_0) \in \{\al_0\} \times U \subset W$. By assumption, there exists $\gam_1 \in \Gam$ such that for each $\gam \in \Gam$, $\gam \geq \gam_1$ implies that $\al_{\gam} = \al_0$ and $x_{\gam} \in X_{\al_0}$. Since $[L_{\gam_1}(x)]_{\gam} \rightarrow x_0$ in $(X_{\al_0}, \MT_{\al_0})$ and $U \in \MT_{\al_0}$, there exists $\gam_2 \in [\gam_1, \infty)$ such that for each $\gam \in [\gam_1, \infty)$, $\gam \geq \gam_2$ implies that $x_{\gam} \in U$. Since $\Gam$ is a directed set, there exists $\gam_0 \in \Gam$ such that $\gam \geq \gam_1, \gam_2$. Let $\gam \in \Gam$. Suppose that $\gam \geq \gam_0$. Then 
			\begin{align*}
				(\al_{\gam}, x_{\gam})
				& = (\al_0, x_{\gam}) \\
				& \in \{\al_0\} \times U \\
				& \subset W \\
				& \subset W_0
			\end{align*}
			Hence $(\al_{\gam}, x_{\gam})$ is eventually in $W_0$. Since $W_0 \in \MN(\al_0, x_0)$ is arbitrary, we have that for each $W_0 \in \MN(\al_0, x_0)$, $(\al_{\gam}, x_{\gam})$ is eventually in $W_0$. Thus $(\al_{\gam}, x_{\gam}) \rightarrow (\al_0, x_0)$ in $(X, \MT)$.  
		\end{itemize}
	\end{proof}
	
	\begin{ex} \lex{ex:topology:coproducts:0004}
		Let $(X_{\al}, \MT_{\al})_{\al \in A}$ be a collection of topological spaces, $(Y, \MS)$ a topological space and $f:\coprod\limits_{\al \in A} X_{\al} \rightarrow Y$. Then $f$ is $\bigg( \bigoplus\limits_{\al \in A} \MT_{\al}, \MS \bigg)$-continuous iff for each $\al \in A$, $f \circ \iota_{\al}$ is $(\MT_{\al}, \MS)$-continuous.
	\end{ex}
	
	\begin{proof}
		Clear by \rex{ex:continuous_maps:0012}.
		\tcr{add more details}
	\end{proof}
	
	\begin{ex} \lex{ex:topology:coproducts:0005}
		Let $(X_{\al}, \MT_{\al})_{\al \in A}$ and $(Y_{\al}, \MS_{\al})_{\al \in A}$ be collections of continuous spaces and $(f_{\al})_{\al \in A} \in \coprod\limits_{\al \in A} Y_{\al}^{X_{\al}}$, i.e. for each $\al \in A$, $f_{\al}:X_{\al} \rightarrow Y_{\al}$. If for each $\al \in A$, $f_{\al}$ is $(\MT_{\al}, \MS_{\al})$-continuous, then $\coprod\limits_{\al \in A} f_{\al}$ is $(\bigoplus\limits_{\al \in A} \MT_{\al}, \bigoplus\limits_{\al \in A} \MS_{\al})$-continuous.
	\end{ex}
	
	\begin{proof} 
		Set $X \defeq \coprod\limits_{\al \in A} X_{\al}$, $Y \defeq \coprod\limits_{\al \in A}Y_{\al}$, $\MT \defeq \bigoplus\limits_{\al \in A} \MT_{\al}$ and $\MS \defeq \bigoplus\limits_{\al \in A} \MS_{\al}$. Suppose that for each $\al \in A$, $f_{\al}$ is $(\MT_{\al}, \MS_{\al})$-continuous. Set $f = \coprod\limits_{\al \in A} f_{\al}$. Denote the $\al$-th embedding maps on $X$ and $Y$ by $\iota^X_{\al}$ and $\iota^Y_{\al}$ respectively. Let $\al \in A$. \rex{ex:set_theory:coproducts:0003} implies that $f \circ \iota^X_{\al} = \iota^Y_{\al} \circ f_{\al}$. Since $\iota^Y_{\al} \circ f_{\al}$ is $(\MT_{\al}, \MS)$-continuous, we have that $f \circ \iota^X_{\al}$ is $(\MT_{\al}, \MS)$-continuous. Since $\al \in A$ is arbitrary, we have that for each $\al \in A$, $f \circ \iota^X_{\al}$ is $(\MT_{\al}, \MS)$-continuous. \rex{ex:topology:coproducts:0004} implies that $f$ is $(\MT, \MS)$-continuous.
	\end{proof}
	
	\begin{ex} \lex{ex:topology:coproducts:0006}
		Let $(X, \MT)$ be a topological space and $(E_{\al})_{\al \in A} \subset \MT$. Define $\phi: \coprod\limits_{\al \in A} E_{\al} \rightarrow \bigcup\limits_{\al \in A} E_{\al}$ by $\phi(\al, x) \defeq x$. If $(E_{\al})_{\al \in A}$ is disjoint, then $\phi$ is a $\bigg( \bigoplus\limits_{\al \in A} [\MT \cap E_{\al}], \MT \cap \bigg[\bigcup\limits_{\al \in A} E_{\al} \bigg] \bigg)$-homeomorphism.
	\end{ex}
	
	\begin{proof} 
		Suppose that $(E_{\al})_{\al \in A}$ is disjoint. 
		\begin{itemize}
			\item \tbf{(bijectivity) :} \\
			\begin{itemize}
				\item \tbf{(injectivity) :} \\
				Let $(\al, x), (\be, y) \in \coprod\limits_{\al \in A} E_{\al}$. Suppose that $\phi(\al, x) =  \phi(\be, y)$. Then $x = y$. Thus $x \in E_{\al} \cap E_{\be}$ and therefore $E_{\al} \cap E_{\be} \neq \varnothing$. Since $(E_{\al'})_{\al' \in A}$ is disjoint, we have that $\al = \be$. Hence $(\al, x) = (\be, y)$. Since $(\al, x), (\be, y) \in \coprod\limits_{\al \in A} E_{\al}$ are arbitrary, we have that for each $(\al, x), (\be, y) \in \coprod\limits_{\al \in A} E_{\al}$, $\phi(\al, x) =  \phi(\be, y)$ implies that $(\al, x) = (\be, y)$. Thus $\phi$ is injective. 
				\item \tbf{(surjectivity) :} \\
				Let $x \in \bigcup\limits_{\al \in A} E_{\al}$. Then there exists $\al \in A$ such that $x \in E_{\al}$. Then $(\al, x) \in \coprod\limits_{\al \in A} E_{\al}$ and $\phi(\al, x) = x$. Since $x \in \bigcup\limits_{\al \in A} E_{\al}$ is arbitrary, we have that for each $x \in \bigcup\limits_{\al \in A} E_{\al}$, there exists $a \in \coprod\limits_{\al \in A} E_{\al}$ such that $\phi(a) = x$. Hence $\phi$ is surjective.
			\end{itemize}
			So $\phi$ is a bijection.
			\item \tbf{(continuity) :} \\
			For each $\al \in A$, define $\iota_{E_{\al}}: E_{\al} \rightarrow \bigcup\limits_{\al \in A} E_{\al}$ by $\iota_{E_{\al}}(x) = x$.
			\begin{itemize}
				\item Let $\al \in A$. Since $\phi \circ \iota_{\al} = \iota_{E_{\al}}$ and $\iota_{E_{\al}}$ is $\bigg( \MT \cap E_{\al}, \MT \cap \bigg[ \bigcup\limits_{\al \in A} E_{\al} \bigg] \bigg)$-continuous \tcr{(maybe give more details)}, we have that $\phi \circ \iota_{\al}$ is $\bigg( \MT \cap E_{\al}, \MT \cap \bigg[ \bigcup\limits_{\al \in A} E_{\al} \bigg] \bigg)$-continuous. Since $\al \in A$ is arbitrary, we have that for each $\al \in A$, $\phi \circ \iota_{\al}$ is $\bigg( \MT \cap E_{\al}, \MT \cap \bigg[ \bigcup\limits_{\al \in A} E_{\al} \bigg] \bigg)$-continuous. \rex{ex:topology:coproducts:0004} implies that $\phi$ is $\bigg( \bigoplus\limits_{\al \in A} (\MT \cap E_{\al}), \MT \cap \bigg[ \bigcup\limits_{\al \in A} E_{\al} \bigg] \bigg)$-continuous. 
				\item Let $B \in \bigoplus\limits_{\al \in A} (\MT \cap E_{\al})$. \rex{ex:topology:coproducts:0003} implies that for each $\al \in A$, there exist $B_{\al} \in \MT \cap E_{\al}$ such that $B = \coprod\limits_{\al \in A} B_{\al}$. Then for each $\al \in A$, there exists $C_{\al} \in \MT$ such that $B_{\al} = C_{\al} \cap E_{\al}$. Since $(E_{\al})_{\al \in A} \subset \MT$, we have that for each $\al \in A$, $C_{\al} \cap E_{\al} \in \MT$. Hence for each $\al \in A$, 
				\begin{align*}
					B_{\al}
					& = C_{\al} \cap E_{\al} \\
					& = (C_{\al} \cap E_{\al}) \cap \bigg[ \bigcup\limits_{\al' \in \N} E_{\al'} \bigg] \\
					& \in \MT \cap \bigg[ \bigcup\limits_{\al' \in \N} E_{\al'} \bigg].
				\end{align*}
				Therefore
				\begin{align*}
					\phi(B)
					& = \phi \bigg(\coprod\limits_{\al \in A} B_{\al}  \bigg) \\
					& = \phi \bigg( \bigcup_{\al \in A} \iota_{\al}(B_{\al}) \bigg) \\
					& = \bigcup_{\al \in A} \phi \circ \iota_{\al}(B_{\al}) \\
					& = \bigcup_{\al \in A} \iota_{E_{\al}}(B_{\al}) \\
					& = \bigcup_{\al \in A} B_{\al} \\
					& \in \MT \cap \bigg[ \bigcup\limits_{\al \in A} E_{\al} \bigg].
				\end{align*}
				Since $B \in \bigoplus\limits_{\al \in A} (\MT \cap E_{\al})$ is arbitrary, we have that for each $B \in \bigoplus\limits_{\al \in A} (\MT \cap E_{\al})$, $f(B) \in \MT$. 
			\end{itemize}
			Hence $\phi$ is a homeomorphism.
		\end{itemize}
	\end{proof}

	\begin{ex} \lex{ex:topology:coproducts:0007}
		Let $(X, \MT)$ be a topological space and $(E_{\al})_{\al \in A} \subset \MP(X)$. Define $\phi: \coprod\limits_{\al \in A} E_{\al} \rightarrow \bigcup\limits_{\al \in A} E_{\al}$ by $\phi(\al,  x) \defeq x$. Then $\phi$ is $(\bigoplus\limits_{\al \in A} (\MT \cap E_{\al}), \MT \cap (\bigcup\limits_{\al \in A} E_{\al}))$-continuous.
	\end{ex}

	\begin{proof}
		Set $E \defeq \bigcup\limits_{\al \in A} E_{\al}$. Let $\al \in A$ and $x \in E_{\al}$. Then 
		\begin{align*}
			\phi \circ \iota_{\al}(x)
			& = \phi(\al, x) \\
			& = x \\
			& = \id_{E_{\al}}(x).
		\end{align*} 
		Since $x \in E_{\al}$ is arbitrary, we have that for each $x \in E_{\al}$, $\phi \circ \iota_{\al}(x) = \id_{E_{\al}}(x)$. Hence $\phi \circ \iota_{\al} = \id_{E_{\al}}$. We note that $\phi \circ \iota_{\al}$ is $(\MT \cap E_{\al}, \MT \cap E_{\al})$-continuous. Since \rex{ex:topology:subspaces:0004} implies that $(\MT \cap E) \cap E_{\al} = \MT \cap E_{\al}$, \rex{ex:topology:subspaces:0005.0001} implies that $\id_{E_{\al}}$ is $(\MT \cap E_{\al}, \MT \cap E)$-continuous. Since $\al \in A$ is arbitrary, we have that for each $\al \in A$, $\phi \circ \iota_{\al}$ is $(\MT \cap E_{\al}, \MT \cap E)$-continuous. \rex{ex:topology:coproducts:0004} implies that $\phi$ is $(\bigoplus\limits_{\al \in A} (\MT \cap E), \MT \cap E)$-continuous.
	\end{proof}
	
	
	
	
	
	
	
	
	
	
	
	
	
	
	
	
	
	
	
	
	
	
	
	
	
	
	
	
	
	
	
	
	
	
	
	
	
	
	
	
	
	
	
	
	
	\newpage
	\section{Quotient Topology}
	
	\subsection{Introduction}
	
	\begin{defn} \ld{def:quotient_topology:0001}
	Let $(X, \MT_X)$, $(Y, \MT_Y)$ be topological spaces and $f:X \rightarrow Y$. Then $f$ is said to be an \tbf{$(\MT_X , \MT_Y)$-quotient map} if 
	\begin{enumerate}
	\item $f$ is surjective
	\item $\MT_Y = f_*\MT_X$
	\end{enumerate}
	\end{defn}
	
	\begin{note}
	We typically avoid specifying the topologies when they are clear from the context.
	\end{note}

	\begin{ex} \lex{ex:quotient_topology:0002}
		Let $(X, \MT_X)$, $(Y, \MT_Y)$ be topological spaces and $f:X \rightarrow Y$. Suppose that $f$ is surjective. Then 
		\begin{enumerate}
			\item $f$ is a quotient map iff 
			$$ \text{for each $V \subset Y$, $V \in \MT_Y$ iff $f^{-1}(V) \in \MT_X$} $$
			\item $f$ is a quotient map iff 
			$$ \text{for each $C \subset Y$, $C$ is closed in $Y$ iff $f^{-1}(C)$ is closed in $X$} $$
		\end{enumerate}
	\end{ex}
	
	\begin{proof}\
		\begin{enumerate}
			\item 
			\begin{itemize}
				\item ($\implies$) \\
				Suppose that $f$ is a quotient map.\\
				Let $V \subset Y$. Suppose that $V$ is open, then 
				\begin{align*}
					V 
					& \in \MT_Y \\
					& = f_*\MT_X \\
					& = \{V' \subset Y: f^{-1}(V') \in \MT_X \}
				\end{align*} 
				Thus $f^{-1}(V) \in \MT_X$. Hence $f^{-1}(V)$ is open. \\
				Conversely, suppose that $f^{-1}(V)$ is open. Then $f^{-1}(V) \in \MT_X$. Since
				\begin{align*}
					\MT_Y 
					& = f_*\MT_X \\
					& = \{V' \subset Y: f^{-1}(V') \in \MT_X \},
				\end{align*}
				we have that $V \in \MT_Y$. Hence $V$ is open. \\ 
				Thus $V$ is open iff $f^{-1}(V)$ is open. Since $V \subset Y$ is arbitrary, we have that for each $V \subset Y$, $V$ is open iff $f^{-1}(V)$ is open.
				\item ($\impliedby$) \\
				Suppose that for each $V \subset Y$, $V$ is open iff $f^{-1}(V)$ is open. Then 
				\begin{align*}
					\MT_Y
					& = \{V \subset Y: f^{-1}(V) \in \MT_X \} \\
					& = f_* \MT_X
				\end{align*}
				So $f$ is a quotient map.
			\end{itemize}
			\item 
			\begin{itemize}
				\item ($\implies$) \\
				Suppose that $f$ is a quotient map.\\
				Let $C \subset Y$. Suppose that $C$ is closed, then $C^c \in \MT_Y$. Continuity implies that
				\begin{align*}
					f^{-1}(C)^c 
					& = f^{-1}(C^c) \\
					& \in \MT_X 
				\end{align*}
				Thus $f^{-1}(C)$ is closed. \\
				Conversely, suppose that $f^{-1}(C)$ is closed. Then 
				\begin{align*}
					f^{-1}(C^c) 
					& = f^{-1}(C)^c \\
					& \in \MT_X 
				\end{align*}
				Since $f$ is a quotient map, $C^c \in \MT_Y$. Hence $C$ is closed. \\
				Thus $C$ is closed iff $f^{-1}(C)$ is closed. Since $C \subset Y$ is arbitrary, we have that for each $C \subset Y$, $C$ is closed iff $f^{-1}(C)$ is closed.
				\item ($\impliedby$) \\
				Suppose that for each $C \subset Y$, $C$ is closed iff $f^{-1}(C)$ is closed. \\
				Let $V \subset Y$. Suppose that $V$ is open. Then $V^c$ is closed. By assumption $f^{-1}(V^c)$ is closed and therefore
				\begin{align*}
					f^{-1}(V)
					& = f^{-1}(V^c)^c \\
					& \in \MT_X. 
				\end{align*}
				Thus $f^{-1}(V)$ is open. \\
				Conversely, suppose that $f^{-1}(V)$ is open. Then $f^{-1}(V)^c$ is closed. Since $f^{-1}(V^c) = f^{-1}(V)^c$, we have that $f^{-1}(V^c)$ is closed. By assumption, $V^c$ is closed. Thus $V$ is open. \\
				Thus $V$ is open iff $f^{-1}(V)$ is open. Since $V \subset Y$ is arbitrary, we have that for each $V \subset Y$, $V$ is open iff $f^{-1}(V)$ is open. Part $(1)$ implies that $f$ is a quotient map.
			\end{itemize}
		\end{enumerate}
	\end{proof}
	
	\begin{ex} \lex{ex:quotient_topology:0003}
	Let $(X, \MT_X)$, $(Y, \MT_Y)$ be topological spaces and $f:X \rightarrow Y$. If $f$ is a quotient map, then $f$ is continuous.
	\end{ex}
	
	\begin{proof}
	Suppose that $f$ is a quotient map. Let $V \subset Y$. Suppose that $V$ is open. \tcb{The previous exercise} implies that $f^{-1}(V)$ is open. Since $V \subset Y$ with $V$ open is arbitrary, we have that for each $V \subset Y$, $V$ is open implies that $f^{-1}(V)$ is open. Hence $f$ is continuous.  
	\end{proof}

	\begin{ex} \lex{ex:quotient_topology:0004}
		Let $(X, \MT_X)$, $(Y, \MT_Y), (Z, \MC)$ be topological spaces, $f:X \rightarrow Y$ and $g: Y \rightarrow Z$. If $f$ is a quotient map, then $g$ is continuous iff $g \circ f$ is continuous.
	\end{ex}
	
	\begin{proof}
		Suppose that $f$ is a quotient map. Then $\MT_Y = f_* \MT_X$. \tcb{An exercise in the section on continuous maps} implies that $g$ is continuous iff $g \circ f$ is continuous.
	\end{proof}

	\begin{ex} \lex{ex:quotient_topology:0005}
		Let $(X, \MT_X)$, $(Y_1, \MT_{Y_1})$, $(Y_2, \MT_{Y_2})$ be topological spaces, $f_1:X \rightarrow Y_1$, $f_2:X \rightarrow Y_2$ and $\phi: Y_1 \rightarrow Y_2$. Suppose that $f_1$ and $f_2$ are quotient maps and $\phi$ is a bijection. If $\phi \circ f_1 = f_2$, then $\phi$ is a homeomorphism. 
	\end{ex}
	
	\begin{proof}
		Since $f_1$ and $f_2$ are quotient maps, they are continous. Suppose that $\phi \circ f_1 = f_2$. Since $f_2$ is continuous, the previous exercise implies that $\phi$ is continuous. Since $\phi$ is a bijection, $f_1 = \phi^{-1} \circ f_2$. Similarly, since $f_1$ is continuous, the previous exercise implies that $\phi^{-1}$ is continuous. Hence $\phi$ is a homeomorphism.
	\end{proof}

	\begin{ex} \lex{ex:quotient_topology:0006}
		\tcb{Restate the last exercise categorically: Let $U: \Top \rightarrow \Set$ be the forgetful functor. If $\phi \in \Iso_{U(X)/U(\Top)}(U(f_1), U(f_2))$, then there exists $\phi' \in \Iso_{X/\Top}(f_1, f_2)$ such that $U(\phi') = \phi$, adjoint functor?...}
	\end{ex}

	\begin{proof}
		\tcr{FINISH!!!}
	\end{proof}
	
	\begin{ex} \lex{ex:quotient_topology:0007}
	Let $(X, \MT_X)$, $(Y, \MT_Y)$ be topological spaces and $f:X \rightarrow Y$. Suppose that $f$ is $(\MT_X, \MT_Y)$-continuous and surjective. If $f$ is $(\MT_X, \MT_Y)$-open or $f$ is $(\MT_X, \MT_Y)$-closed, then $f$ is a $(\MT_X, \MT_Y)$-quotient map. 
	\end{ex}
	
	\begin{proof}\
	\begin{itemize}	
	\item Suppose that $f$ is $(\MT_X, \MT_Y)$-open. Let $V \subset Y$. \\
	Suppose that $V \in \MT_Y$. Continuity implies that $f^{-1}(V) \in \MT_X$. Conversely, suppose that $f^{-1}(V) \in \MT_X$. Since $f$ is open $f(f^{-1}(V)) \in \MT_Y$. Surjectivity implies that $V = f(f^{-1}(V))$. So $V \in \MT_Y$. \rex{ex:quotient_topology:0002} then implies that $f$ is a quotient map.\\
	\item   
	Suppose that $f$ is closed. Then similarly to above, $f$ is a quotient map.
	\end{itemize}
	\end{proof}
	
	\begin{ex} \lex{ex:quotient_topology:0008}
	Let $(X, \MT_X)$, $(Y, \MT_Y)$ be topological spaces and $f:X \rightarrow Y$. Suppose that $f$ is a quotient map. Then 
	\begin{enumerate}
		\item $f$ is open iff 
		$$\text{for each $U \subset X$, $U$ is open implies that $f^{-1}(f(U))$ is open} $$
		\item $f$ is closed iff 
		$$\text{for each $C \subset X$, $C$ is closed implies that $f^{-1}(f(C))$ is closed} $$
	\end{enumerate}
	\end{ex}
	
	\begin{proof}\
	\begin{enumerate}
		\item 
		\begin{itemize}	
			\item ($\implies$) \\
			Suppose that $f$ is open.\\
			Let $U \subset X$. Suppose that $U$ is open. Since $f$ is open, $f(U)$ is open. Continuity implies that $f^{-1}(f(U))$ is open.\\ 
			\item ($\impliedby$) \\
			Suppose that for each $U \subset X$, $U$ is open implies that $f^{-1}(f(U))$ is open. \\
			Let $U \subset X$. Suppose that $U$ is open. By assumption, $f^{-1}(f(U))$ is open. Since $f$ is a quotient map, $f(U)$ is open. Since $U \subset X$ with $U$ open is arbitrary, we have that for each $U \subset X$, $U$ is open implies $f(U)$ is open. Thus $f$ is open.
		\end{itemize}
		\item 
		\begin{itemize}	
			\item ($\implies$) \\
			Suppose that $f$ is closed.\\
			Let $C \subset X$. Suppose that $C$ is closed. Since $f$ is closed, $f(C)$ is closed. Continuity implies that $f^{-1}(f(C))$ is closed.\\ 
			\item ($\impliedby$) \\
			Suppose that for each $X \subset X$, $C$ is closed implies that $f^{-1}(f(C))$ is closed. \\
			Let $C \subset X$. Suppose that $C$ is closed. By assumption, $f^{-1}(f(C))$ is closed. Since $f$ is a quotient map, $f(C)$ is closed. Since $C \subset X$ with $C$ closed is arbitrary, we have that for each $C \subset X$, $C$ is closed implies $f(C)$ is closed. Thus $f$ is closed.
		\end{itemize}
	\end{enumerate}
	\end{proof}
	
	\begin{ex} \lex{ex:quotient_topology:0010}
	Let $(X, \MT)$ be a topological space, $Y$ a set and $f:X \rightarrow Y$. Suppose that $f$ is surjective. Then $f: X \rightarrow Y$ is a $(\MT, f_*\MT)$ quotient map. 
	\end{ex}
	
	\begin{proof}
	Clear by definition.
	\end{proof}
	
	
	\begin{ex} \lex{ex:quotient_topology:0011}
	Let $(X, \MT)$ be a topological space, $\sim$ an eqivalence relation on $X$ and $\pi:X \rightarrow X/\sim$ the projection map given by $x \mapsto \bar{x}$. Then $\pi$ is a $(\MT, \pi_*\MT)$-quotient map. 
	\end{ex}
	
	\begin{proof}
	Since $\pi$ is surjective, the previous exercise implies that $\pi$ is a $(\MT, \pi_*\MT)$-quotient map. 
	\end{proof}

	\begin{defn} \ld{def:quotient_topology:0012}
		Let $(X, \MT)$ be a topological space, $\sim$ an eqivalence relation on $X$ and $\pi:X \rightarrow X/\sim$ the projection map given by $x \mapsto \bar{x}$. We define the \tbf{quotient topology on $X / \sim$} on $X/ \sim$, denoted $\MT_{X/ \sim}$, by $$\MT_{X/ \sim} = \pi_{*}\MT$$
	\end{defn}
	
	\begin{defn} \ld{def:quotient_topology:0014}
		Let $(X, \MT_X)$, $(Y, \MT_Y)$ be topological spaces, $\sim_X$ an equivalence relation on $X$, $\sim_Y$ and equivalence relation on $Y$ and $f : X \rightarrow Y $. Then $f$ is said to be \tbf{$(\sim_X, \sim_Y)$-invariant} if for each $x, y \in X$, $x \sim_X y$ implies that $f(x) \sim_Y f(y)$.
	\end{defn}
	
	\begin{ex} \lex{ex:quotient_topology:0015}
	Let $(X, \MT_X)$, $(Y, \MT_Y)$ be topological spaces, $\sim_X, \sim_Y$ eqivalence relations on $X$ and $Y$ respectively, $\pi_X:X \rightarrow X/\sim_X$, $\pi_Y:Y \rightarrow Y/\sim_Y$ the respective projection maps and $f:X \rightarrow Y$ continuous. If $f$ is $(\sim_X, \sim_Y)$-invariant, then there exists a unique $\bar{f}:X / {\sim}_X \rightarrow Y/ {\sim_Y}$ such that $\bar{f}$ is continuous and $\bar{f} \circ \pi_X = \pi_Y \circ f$, i.e. the following diagram commutes:
	\[ 
	\begin{tikzcd}
		X  \arrow[r, "f"]  \arrow[d, "\pi_X"']  & Y   \arrow[d, "\pi_Y"]\\
		X/ {\sim_X} \arrow[r, "\bar{f}"'] &  Y / {\sim_Y} \\
	\end{tikzcd}
	\]
	and $\bar{f}$ is $(\MT_{X/ \sim_X}, \MT_{Y/ \sim_Y})$-continuous.
	\end{ex}
	
	\begin{proof}\
	Suppose that $f$ is is $(\sim_X, \sim_Y)$-invariant. 
	\begin{itemize}
		\item \tbf{Existence:} \\
			Define $\bar{f}: X / {\sim_Y} \rightarrow Y/{\sim_Y}$ by $\bar{f}(\bar{x}) = \ol{f(x)}$. Let $a,b \in X$. Then 
		\begin{align*}
			\bar{a} = \bar{b}
			& \implies a \sim_X b \\
			& \implies f(a) \sim_Y f(b) \\
			& \implies \overline{f(a)} = \overline{f(b)} \\
			& \implies \bar{f}(\bar{a}) = \bar{f}(\bar{b}) 
		\end{align*}
		So $\bar{f}$ is well defined. By construction $\bar{f} \circ \pi_X = \pi_Y \circ f$.
		\item \tbf{Uniqueness:} \\
		Let $g: X / {\sim}_X \rightarrow Y/ {\sim_Y}$. Suppose that $g \circ \pi_X = \pi_Y \circ f$, i.e. the following diagram commutes:
		\[ 
		\begin{tikzcd}
			X  \arrow[r, "f"]  \arrow[d, "\pi_X"']  & Y   \arrow[d, "\pi_Y"]\\
			X/ {\sim_X} \arrow[r, "g"'] &  Y / {\sim_Y} \\
		\end{tikzcd}
		\]
		Then 
		\begin{align*}
			g \circ \pi_X 
			& = \pi_Y \circ f \\
			& = \bar{f} \circ \pi_X 
		\end{align*}
		i.e. the following diagram commites:
		\[ 
		\begin{tikzcd}
			X  \arrow[r, "\pi_X"]  \arrow[d, "\pi_X"']  & Y   \arrow[d, "\bar{f}"]\\
			X/ {\sim_X} \arrow[r, "g"'] &  Y / {\sim_Y} \\
		\end{tikzcd}
		\]
		Since $\pi_X$ is surjective, \rex{ex:set_theory:bijections:0003} implies that $\bar{f} = g$. 
		\item  \tbf{Continuity:} \\
		Let $V \in \MT_{Y/ \sim_Y}$. Continuity of $f$ and $\pi_Y$ implies that 
		\begin{align*}
			\pi_X^{-1}(\bar{f}^{-1}(V))
			& = (\bar{f} \circ \pi_X)^{-1}(V) \\
			& = (\pi_Y \circ f)^{-1}(V) \\
			& = f^{-1}(\pi_Y^{-1}(V)) \\
			& \in \MT_X
		\end{align*}
		By definition of the quotient topology, $\bar{f}^{-1}(V) \in \MT_{X/ \sim_X}$. So $\bar{f}$ is $(\MT_{X/ \sim_X}, \MT_{Y/ \sim_Y})$-continuous.
	\end{itemize}
	\end{proof}

	\begin{defn} \ld{def:quotient_topology:0015.1} 
		Let $(X, \MT_X)$, $(Y, \MT_Y)$ be topological spaces, $\sim_X$ an equivalence relation on $X$ and $f : X \rightarrow Y $. Then $f$ is said to be \tbf{${\sim}$-invariant} if $f$ is $({\sim}, {=_Y})$-invariant.
	\end{defn}

	\begin{ex} \lex{ex:quotient_topology:0015.2}
		Let $(X, \MT_X)$, $(Y, \MT_Y)$ be topological spaces, ${\sim} \subset X \times X$ an eqivalence relation on $X$, $\pi:X \rightarrow X/ {\sim}$ the projection map and $f:X \rightarrow Y$ continuous. If $f$ is ${\sim}$-invariant, then there exists a unique $\bar{f}:X / {\sim} \rightarrow Y$ such that $\bar{f}$ is continuous and $\bar{f} \circ \pi = f$, i.e. the following diagram commutes:
		\[ 
		\begin{tikzcd}
			X  \arrow[dr, "f"]  \arrow[d, "\pi"']  & \\
			X/ {\sim} \arrow[r, "\bar{f}"'] &  Y  \\
		\end{tikzcd}
		\]
		and $\bar{f}$ is $(\MT_{X/ \sim}, \MT_Y)$-continuous.
	\end{ex}

	\begin{proof}
		Set $\sim_X \defeq {\sim}$ and $\sim_Y \defeq {=_Y}$ and use \rex{ex:quotient_topology:0015}.
		\tcr{add details}
	\end{proof}

	\begin{ex} \lex{ex:quotient_topology:0013}
		Let $(X, \MT)$, $(Y, \MT_Y)$ be topological spaces, $\sim$ an eqivalence relation on $X$, $\pi:X \rightarrow X/\sim$ the projection map and $f:X \rightarrow Y$. Suppose $f$ is a quotient map. If for each $a, b \in X$, $a \sim b$ iff $f(a) = f(b)$, then there exists a unique $\bar{f}: X/{\sim} \rightarrow Y$ such that $\bar{f}$ is a  $(\MT_{X/{\sim}}, \MT_Y)$-homeomorphism and $\bar{f} \circ \pi = f$.
	\end{ex}
	
	\begin{proof}
		Suppose that for each $a, b \in X$, $a \sim b$ iff $f(a) = f(b)$. Then $f$ is ${\sim}$-invariant. Since $f$ is a quotient map, $f$ is continuous. \rex{ex:quotient_topology:0015.2} implies that there exists a unique $\bar{f}: X/{\sim} \rightarrow Y$ such that $\bar{f}$ is continuous and $\bar{f} \circ \pi = f$. 
		Let $y \in Y$. Since $f$ is a quotient map, $f$ is surjective. Therefore there exists $x \in X$ such that $f(x) = y$. Thus 
		\begin{align*}
			\bar{f}(\bar{x}) 
			& = f(x) \\
			& = y 
		\end{align*}
		Since $y \in Y$ is arbitrary, $\bar{f}$ is surjective. Let $\bar{a}, \bar{b} \in X/\sim$. Suppose that $\bar{f}(\bar{a}) = \bar{f}(\bar{b})$. Then 
		\begin{align*}
			f(a)
			& = \bar{f}(\bar{a}) \\
			& = \bar{f}(\bar{b}) \\
			& = f(b)
		\end{align*}
		By assumption, $a \sim b$. Hence $\bar{a} = \bar{b}$. Since $\bar{a}, \bar{b} \in X/\sim$ are arbitrary, $\bar{f}$ is injective. Therefore, $\bar{f}$ is a bijection. \rex{ex:quotient_topology:0005} implies that $\bar{f}$ is a homeomorphism.
	\end{proof}
	
	\begin{defn} \ld{def:quotient_topology:0013.1}
		Let $(X, \MT)$, $(Y, \MT_Y)$ be topological spaces and $f:X \rightarrow Y$. We define the relation ${\sim_f} \subset X \times X$ by $x \sim_f y$ iff $f(x) = f(y)$. 
	\end{defn}
	
	\begin{ex} \lex{ex:quotient_topology:0013.2}
		Let $(X, \MT)$, $(Y, \MT_Y)$ be topological spaces and $f:X \rightarrow Y$. Then 
		\begin{enumerate}
			\item $\sim_f$ is an equivlance relation on $X$,
			\item if $f$ is a quotient map, then there exists a unique $\bar{f}: X/{\sim_f} \rightarrow Y$ such that $\bar{f}$ is a  $(\MT_{X/{\sim_f}}, \MT_Y)$-homeomorphism and $\bar{f} \circ \pi = f$. 
		\end{enumerate}
	\end{ex}
	
	\begin{proof} \
		\begin{enumerate}
			\item Clear.
			\item Suppose that $f$ is a quotient map. \rex{ex:quotient_topology:0013} implies that then there exists a unique $\bar{f}: X/{\sim_f} \rightarrow Y$ such that $\bar{f}$ is a  $(\MT_{X/{\sim_f}}, \MT_Y)$-homeomorphism and $\bar{f} \circ \pi = f$. 
		\end{enumerate}
	\end{proof}



	\begin{ex} \lex{ex:quotient_topology:0013.3}
		Let $X$, $X_1$, $X_2$ be topological spaces and $f_1:X \rightarrow X_1$, $f_2:X \rightarrow X_2$ and $f_{1,2}:X_2 \rightarrow X_1$. Suppose that $f_1$ and $f_2$ are quotient maps. If $f_{1,2} \circ f_1 = f_2$, then there exists unique $\bar{f}_1:X/{\sim_{f_1}} \rightarrow X_1$, $\bar{f}_2:X/{\sim_{f_2}} \rightarrow X_2$, $\bar{f}_{1,2}: X/{\sim_{f_2}} \rightarrow X/{\sim_{f_1}}$ such that $\bar{f}_1$, $\bar{f}_2$ are homeomorphisms, $\bar{f}_{1,2}$ is continuous, $\bar{f}_1 \circ \pi_1  = f_1$, $\bar{f}_2 \circ \pi_2  = f_2$, $f_{1,2} \circ \bar{f}_1 = \bar{f}_2 \circ \bar{f}_{1,2}$ and $\bar{f}_{1,2} \circ \pi_1 = \pi_2$, i.e. the following diagram commutes:
		\[ 
		\begin{tikzcd}
			X_1  \arrow[rr, "f_{1,2}"]   & & X_2   \\
			X/ {\sim_{f_1}} \arrow[u, "\bar{f}_1"]   \arrow[rr, "\bar{f}_{1,2}"] & &  X / {\sim_{f_2}} \arrow[u, "\bar{f}_2"']  \\
			& X   \arrow[ul, "\pi_1"] \arrow[ur, "\pi_2"'] \arrow[bend left=90]{luu}{f_1} \arrow[bend right=90, swap]{ruu}{f_2}  & 
		\end{tikzcd}
		\]
	\end{ex}

	\begin{proof}\
		\begin{itemize}
			\item \tbf{Existence and Uniqueness of $\bar{f}_k$:} \\
			Since $f_1,f_2$ are quotient maps, \rex{ex:quotient_topology:0013.2} implies that there exist unique $\bar{f}_1: X/{\sim_{f_1}} \rightarrow X_1$, $\bar{f}_2:X/{\sim_{f_2}} \rightarrow X_2$ such that $\bar{f}_1$, $\bar{f}_2$ are homeomorphisms, $\bar{f}_1 \circ \pi_1 = f_1$ and $\bar{f}_2 \circ \pi_2 = f_2$.
			\item \tbf{Existence of $\bar{f}_{1,2}$:} \\
			Define $\bar{f}_{1,2}:X/{\sim_{f_1}} \rightarrow X/{\sim_{f_1}}$ by $\bar{f}_{1,2} \defeq \bar{f}_2^{-1} \circ f_{1,2} \circ \bar{f}_1$. Since $\bar{f}_2^{-1}$, $f_{1,2}$, $\bar{f}_1$ are continuous, $\bar{f}_{1,2}$ is continuous. By construction,
			\begin{align*}
				\bar{f}_2 \circ \bar{f}_{1,2} 
				& =  \bar{f}_2 \circ (\bar{f}_2^{-1} \circ f_{1,2} \circ \bar{f}_1) \\
				& = (\bar{f}_2 \circ \bar{f}_2^{-1}) \circ (f_{1,2} \circ \bar{f}_1) \\
				& = \id_{X_2} \circ (f_{1,2} \circ \bar{f}_1) \\
				& = f_{1,2} \circ \bar{f}_1. 
			\end{align*}
			and therefore
			\begin{align*}
				\bar{f}_{1,2} \circ \pi_1 
				& = (\bar{f}_2^{-1} \circ f_{1,2} \circ \bar{f}_1) \circ \pi_1 \\
				& = (\bar{f}_2^{-1} \circ f_{1,2}) \circ (\bar{f}_1 \circ \pi_1) \\
				& = (\bar{f}_2^{-1} \circ f_{1,2}) \circ f_1 \\
				& = \bar{f}_2^{-1} \circ (f_{1,2} \circ f_1) \\
				& = \bar{f}_2^{-1} \circ f_2 \\
				& = \pi_2 
			\end{align*}
			\item \tbf{Uniqueness of $\bar{f}_{1,2}$:} \\
			Let $g_{1,2}: X/{\sim_{f_1}} \rightarrow X/{\sim_{f_2}}$. Suppose that $\bar{f}_2 \circ g_{1,2} = f_{1,2} \circ \bar{f}_1$. Then 
			\begin{align*}
				g_{1,2}
				& = \bar{f}_2^{-1} \circ f_{1,2} \circ \bar{f}_1 \\
				& = \bar{f}_{1,2}.
			\end{align*}
			Thus $\bar{f}_{1,2}$ is unique.
		\end{itemize}
	\end{proof}

























	\subsection{Category of Topological Spaces with Equivalence Relations}

	\begin{defn} \ld{def:quotient_topology:0016} 
		We define the category $\TopEq$ 
		\begin{itemize}
			\item $\Obj(\TopEq) = \{(X, \sim):  \text{ $X$ is a topological space and $\sim$ is an equivalence relation on $X$}\}$
			\item $\Hom_{\TopEq}((X, \sim_X), (Y, \sim_Y)) = \{f: X \rightarrow Y: \text{$f$ is continuous and $f$ is $(\sim_X, \sim_Y)$-invariant}\}$. 
		\end{itemize}
	\end{defn}

	\begin{defn} \ld{def:quotient_topology:0017} 
		We define $F: \TopEq \rightarrow \Top$ by
		\begin{itemize}
			\item $F(X, \sim) = X/\sim$
			\item $F(f) = \bar{f}$
		\end{itemize}
	\end{defn}

	\begin{ex} \lex{ex:quotient_topology:0018} 
		We have that $F: \TopEq \rightarrow \Top$ is a functor
	\end{ex}

	\begin{proof}\
		\begin{enumerate}
			\item Let $(X, \sim_X)$, $(Y, \sim_Y) \in \TopEq$ and $f \in \Hom_{\TopEq}((X, \sim_X), (Y, \sim_Y))$. The previous exercise implies that $F(f) \in \Hom_{\Top}(X/{\sim_X}, Y/ {\sim_Y})$. 
			\item  Let $(X, \sim_X)$, $(Y, \sim_Y)$, $(Z, \sim_Z) \in \TopEq$,  $f \in \Hom_{\TopEq}((X, \sim_X), (Y, \sim_Y))$ and $g \in \Hom_{\TopEq}((Y, \sim_Y), (Z, \sim_Z))$. Then 
			\begin{align*}
				\pi_Z \circ (g \circ f)
				& = (\pi_Z \circ g) \circ f \\
				& = (\bar{g} \circ \pi_Y) \circ f \\
				& = \bar{g} \circ (\pi_Y \circ f) \\
				& = \bar{g} \circ (\bar{f} \circ \pi_X) \\
				& = (\bar{g} \circ \bar{f}) \circ \pi_X \\
			\end{align*}
			Uniqueness implies that $F(g \circ f) = F(g) \circ F(f)$. 
		\end{enumerate}
	\end{proof}

	\begin{ex} \lex{ex:quotient_topology:0019} 
		Let $(X, \sim_X)$, $(Y, \sim_Y)$, $(Z, =)  \in \Obj(\TopEq)$, $\iota \in \Hom_{\TopEq}((X, \sim_X), (Y, \sim_Y))$,  $ f \in \Hom_{\TopEq}((X, \sim_X), (Z, =))$ and  $ g \in \Hom_{\TopEq}((Y, \sim_Y), (Z, =))$. 
		Then $f = g \circ \iota $ iff $\bar{f} = \bar{g} \circ \bar{\iota}$, in which case the following diagram commutes: 
		\[ 
		\begin{tikzcd}
			& Z & \\
			X  \arrow[rr, "\iota"]  \arrow[d, "\pi_X"'] \arrow[ur, "f"] & & Y   \arrow[d, "\pi_Y"] \arrow[ul, "g"']\\
			X/ {\sim_X} \arrow[bend left=90]{uur}{\bar{f}} \arrow[rr, "\bar{\iota}"'] & &  Y / {\sim_Y} \arrow[bend right=90, swap]{uul}{\bar{g}} \\
		\end{tikzcd}
		\]
	\end{ex}

	\begin{proof}
		Suppose that $f = g \circ \iota$. Functoriality implies that 
		\begin{align*}
			\bar{f} 
			& = F(f) \\
			& = F(g \circ \iota) \\
			& = F(g) \circ F(\iota) \\
			&= \bar{g} \circ \bar{\iota}
		\end{align*}
		Conversely, suppose that $\bar{f} = \bar{g} \circ \bar{\iota}$. Then
		\begin{align*}
			f
			& = \bar{f} \circ \pi_X \\
			& = \bar{g} \circ \bar{\iota} \circ \pi_X \\
			& = \bar{g} \circ \pi_Y \circ \iota \\
			& = g \circ \iota 
		\end{align*}
	\end{proof}
	
	\begin{ex} \lex{ex:quotient_topology:0020} 
		Let $G$ be a group, $X$ a topological space and $\phi: G \times X \rightarrow X$ a group action. Suppose that for each $g \in G$, the map $\phi_g \in \Sym(X)$ defined by $\phi_g(x) = g \cdot x$ is continuous. Then $\pi: X \rightarrow X / G$ is open. 
		\tcr{move this before category theory stuff}
	\end{ex}

	\begin{proof}
		Suppose that for each $g \in G$, $\phi_g$ is continuous. Let $g \in G$. Since $(\phi_g)^{-1} = \phi_{g^{-1}}$, $\phi_g$ is a homeomorphism. Hence for each $g \in G$ and $U \subset X$, $U$ is open iff $g \cdot U$ is open. Let $U \subset X$. Suppose that $U$ is open. Then $\pi^{-1}(\pi(U)) = \bigcup_{g \in G} g \cdot U$ is open. Since $\pi$ is a quotient map, \rex{ex:quotient_topology:0008} implies that $\pi$ is open.
	\end{proof}














































	\newpage
	\section{Equalizers of Continuous Maps}
	
	\begin{note}
		Let $X, Y$ be sets and $f,g:X \rightarrow Y$. We denote the equalizer of $f$ and $g$ by $\Eq(f,g)$ as in \rd{def:set_theory:equalizers:0001}.
	\end{note}
	
	\begin{ex} \lex{ex:topology:equalizers:0001}
		Let $(X, \MT_X), (Y, \MT_Y)$ be topological spaces and $f,g:X \rightarrow Y$ $(\MT_X, \MT_Y)$-continuous. If $(Y, \MT_Y)$ is Hausdorff, then $\Eq(f,g)$ is closed in $(X, \MT_X)$.
	\end{ex}
	
	\begin{proof}
		Suppose that $(Y, \MT_Y)$ is Hausdorff. \rex{ex:topology:separation:0005} implies that $\Del_Y$ is closed in $(Y \times Y, \MT_Y \otimes \MT_Y)$. Since $f,g$ are $(\MT_X, \MT_Y)$-continuous, \rex{ex:product_topology:0012.1} implies that $(f,g): X \rightarrow Y \times Y$ is $(\MT_X, \MT_Y \otimes \MT_Y)$-continuous. Since $\Eq(f,g) = (f,g)^{-1}(\Del_Y)$, we have that $\Eq(f,g)$ is closed in $(X, \MT_X)$.
	\end{proof}


	
	
	
	
	
	
	
	
	
	
	
	
	
	
	
	
	
	
	
	
	
	
	
	
	
	
	
	
	
	
	
	
	
	
	
	
	
	
	
	
	
	\newpage
	\section{Projective Limits of Topological Spaces}
	
	\begin{note}
		Let $((X_j)_{j \in J}, (\pi_{j,k})_{(j,k) \in \leq})$ a projective system of sets.
		\begin{itemize}
			\item We denote the $j$-th projection map from $\prod\limits_{j \in J} X_j$ onto $X_j$ by $\prj_j: \prod\limits_{j \in J} X_j \rightarrow X_j$.
			\item We denote the $j$-th projection map from $\varprojlim\limits_{j \in J} X_j$ into $X_j$ by $\pi_j: \varprojlim\limits_{j \in J} X_j \rightarrow X_j$ as in \rd{def:set_theory:proj_limits:0002} 
		\end{itemize}
	\end{note}
	
	\tcr{define general inverse limits and show universal property and specific concrete examples}
	
	\begin{defn} \ld{def:topology:proj_limits:0001}
		Let $(J, {\leq})$ be a directed poset, $(X_j, \MT_j)_{j \in J}$ a collection of topological spaces and for each $(j,k) \in {\leq}$, $\pi_{j,k}:X_k \rightarrow X_j$ a $(\MT_k, \MT_j)$-continuous map. Suppose that for each $j,k,l \in J$, 
		\begin{enumerate}
			\item $\pi_{j,j} = \id_{X_j}$,
			\item $j \leq k$ and $k \leq l$ implies that $\pi_{j,k} \circ \pi_{k,l} = \pi_{j,l}$.
		\end{enumerate}
		Then $((X_j, \MT_j)_{j \in J}, (\pi_{j,k})_{(j,k) \in \leq})$ is said to be a \tbf{$\Top$-projective system}.
	\end{defn}
	
	\begin{defn} \ld{def:topology:proj_limits:0002}
		Let $(J, {\leq})$ be a directed poset, $((X_j, \MT_j)_{j \in J}, (\pi_{j,k})_{(j,k) \in \leq})$ a projective system of topological spaces. We define the \tbf{projective limit topology} on $\varprojlim\limits_{j \in J} X_j$, denoted $\varprojlim\limits_{j \in J} \MT_j$, by $$\varprojlim\limits_{j \in J} \MT_j \defeq \tau(\pi_j: j \in J).$$
	\end{defn}
	
	\begin{ex} \lex{ex:topology:proj_limits:0003}
		Let $(J, {\leq})$ be a directed poset and $((X_j, \MT_j)_{j \in J}, (\pi_{j,k})_{(j,k) \in \leq})$ a projective system of topological spaces. Then 
		$$\varprojlim\limits_{j \in J}  \MT_j = \bigg[ \bigotimes\limits_{j \in J} \MT_j \bigg] \cap \varprojlim\limits_{j \in J} X_j.$$
	\end{ex}

	\begin{proof} \
		\begin{itemize}
			\item Define $\ME \subset \varprojlim\limits_{j \in J}  \MT_j$ by 
			$$\ME \defeq \{\pi_j^{-1}(V): j \in J \text{ and } V \in \MT_j\}.$$ 
			By defintion, $\varprojlim\limits_{j \in J}  \MT_j = \tau(\ME)$. Let $U \in \ME$. Then there exists $j \in J$ and $V \in \MT_j$ such that $U = \pi_j^{-1}(V)$. Since $\pi_j = \prj_j|_{\varprojlim\limits_{j \in J} X_j}$ and $\prj_j^{-1}(V) \in \bigg[ \bigotimes\limits_{j \in J} \MT_j \bigg]$, we have that 
			\begin{align*}
				U
				& = \pi_j^{-1}(V) \\
				& = (\prj_j|_{\varprojlim\limits_{j \in J} X_j })^{-1}(V) \\
				& = \prj_j^{-1}(V) \cap \varprojlim\limits_{j \in J} X_j \\
				& \in \bigg[ \bigotimes\limits_{j \in J} \MT_j \bigg] \cap \varprojlim\limits_{j \in J} X_j.
			\end{align*} 
			Since $U \in \ME$ is arbitrary, we have that $\ME \subset \bigg[ \bigotimes\limits_{j \in J} \MT_j \bigg] \cap \varprojlim\limits_{j \in J} X_j$. Since by definition, $\varprojlim\limits_{j \in J}  \MT_j = \tau(\ME)$, we have that  
			\begin{align*}
				\varprojlim\limits_{j \in J}  \MT_j
				& = \tau(\ME) \\
				& \subset \tau \bigg( \bigg[ \bigotimes\limits_{j \in J} \MT_j \bigg] \cap \varprojlim\limits_{j \in J} X_j \bigg) \\
				& = \bigg[ \bigotimes\limits_{j \in J} \MT_j \bigg] \cap \varprojlim\limits_{j \in J} X_j.
			\end{align*}
			\item Define $\MF \subset \bigotimes\limits_{j \in J} \MT_j$ by $\MF \defeq \{\prj_j^{-1}(V): j \in J \text{ and } V \in \MT_j\}$. Let $F \in \MF \cap \varprojlim\limits_{j \in J} X_j$. Then there exists $F_0 \in \MF$ such that $F = F_0 \cap \varprojlim\limits_{j \in J} X_j$. Thus there exists $j \in J$ and $V \in \MT_j$ such that $F_0 = \prj_j^{-1}(V)$. Therefore
			\begin{align*}
				F
				& = F_0 \cap \varprojlim\limits_{j \in J} X_j \\
				& = \prj_j^{-1}(V) \cap \varprojlim\limits_{j \in J} X_j \\
				& = (\prj_j|_{\varprojlim\limits_{j \in J} X_j})^{-1}(V) \\
				& = \pi_j^{-1}(V) \\
				& \in \ME. 
			\end{align*}
			Since $F \in \MF \cap \varprojlim\limits_{j \in J} X_j$ is arbitrary, we have that $\MF \cap \varprojlim\limits_{j \in J} X_j \subset \ME$. \rex{ex:topology:subspaces:0003.1} implies that
			\begin{align*}
				\bigg [ \bigotimes_{j \in J} \MT_j \bigg] \cap \varprojlim\limits_{j \in J} X_j 
				& = \tau(\MF) \cap \varprojlim\limits_{j \in J} X_j \\
				& = \tau(\MF \cap \varprojlim\limits_{j \in J} X_j) \\
				& \subset \tau(\ME) \\
				& = \varprojlim\limits_{j \in J} \MT_j.
			\end{align*}
		\end{itemize}
		Since $\varprojlim\limits_{j \in J} \MT_j \subset 	\bigg[ \bigotimes\limits_{j \in J} \MT_j \bigg] \cap \varprojlim\limits_{j \in J} X_j$ and $	\bigg[ \bigotimes\limits_{j \in J} \MT_j \bigg] \cap \varprojlim\limits_{j \in J} X_j \subset \varprojlim\limits_{j \in J} \MT_j$, we have that 
		$$\varprojlim\limits_{j \in J}  \MT_j = \bigg[ \bigotimes\limits_{j \in J} \MT_j \bigg] \cap \varprojlim\limits_{j \in J} X_j.$$
	\end{proof}
	
	
	\begin{ex} \lex{ex:topology:proj_limits:0004}
		Let $(J, {\leq})$ be a directed poset, $((X_j, \MT_j)_{j \in J}, (\pi_{j,k})_{(j,k) \in \leq})$ a projective system of topological spaces. If for each $j \in J$, $(X_j, \MT_j)$ is Hausdorff, then $\varprojlim\limits_{j \in J} X_j$ is closed in $(\prod\limits_{j \in J} X_j, \bigotimes\limits_{j \in J} \MT_j)$.
	\end{ex}
	
	\begin{proof}
		Suppose that for each $j \in J$, $(X_j, \MT_j)$ is Hausdorff. Set $X \defeq \varprojlim\limits_{j \in J} X_j$ and $\MT \defeq \varprojlim\limits_{j \in J} \MT_j$. \rex{ex:set_theory:proj_limits:0004} implies that $X = \bigcap\limits_{(j,k) \in {\leq}} \Eq( \pi_{j,k} \circ \prj_k,  \prj_j)$. Let $(j,k) \in {\leq}$. Since $\pi_{j,k}$ is $(\MT_k, \MT_j)$-continuous and $\pi_k$ is $(\MT, \MT_k)$-continuous, we have that $\pi_{j,k} \circ \pi_k$ is $(\MT, \MT_j)$-continuous. Since $\pi_{j,k} \circ \pi_k$ and $\pi_j$ are $(\MT, \MT_j)$-continuous and $(X_j, \MT_j)$ is Hausdorff, \rex{ex:topology:equalizers:0001} implies that $\Eq( \pi_{j,k} \circ \prj_k,  \prj_j)$ is closed in $(X, \MT)$. Since $(j,k) \in {\leq}$ is arbitrary, we have that for each $(j,k) \in {\leq}$, $\Eq( \pi_{j,k} \circ \prj_k,  \prj_j)$ is closed in $(X, \MT)$. Since $X = \bigcap\limits_{(j,k) \in {\leq}} \Eq( \pi_{j,k} \circ \prj_k,  \prj_j)$, we have that $X$ is closed in $(\prod\limits_{j \in J} X_j, \bigotimes\limits_{j \in J} \MT_j)$.
	\end{proof}
	
	
	
	\begin{defn} \ld{def:topology:proj_limits:0007}
		Let $(J, {\leq})$ be a directed poset $((X_j)_{j \in J}, (\pi_{j,k})_{(j,k) \in \leq})$ be a projective system of topological spaces. Set $(X, (\pi_j)_{j \in J}) \defeq \varprojlim\limits_{j \in J} ((X_j)_{j \in J}, (\pi_{j,k})_{(j,k) \in \leq})$. 
		\begin{itemize}
			\item The projective limit $(X, (\pi_j)_{j \in J})$ is said to be \tbf{perfect} \tcr{(change this to a more descriptiv term)} if for each $j \in J$, $\pi_j$ is a quotient map.
			\item For each $j \in J$, we define
			\begin{itemize}
				\item the \tbf{equivalence relation on $X$ induced by $\pi_j$}, denoted ${\sim_j} \subset X \times X$, by $x \sim_j y$ iff $\pi_j(x) = \pi_j(y)$, 
				\item the \tbf{quotient of $X$ induced by $\pi_j$}, denoted $X^Q_j$, by $X^Q_j \defeq X/{\sim_j}$,
				\item the \tbf{quotient map induced by $\pi_j$}, denoted $\pi^Q_j:X \rightarrow X^Q_j$, by $\pi^Q_j(x) \defeq \{y \in X: y \sim_j x\}$.
				\item the \tbf{factor of $\pi_j$ by $\pi^Q_j$}, denoted $\bar{\pi}:X^Q_j \rightarrow X_j$, to be the unique map from $X^Q_j$ into $X_j$ such that $\bar{\pi}_j \circ \pi^Q_j = \pi_j$ as defined in \rex{ex:quotient_topology:0015.2}
			\end{itemize}
		\end{itemize}
	\end{defn}
	
	\begin{ex} \lex{ex:topology:proj_limits:0009}
		Let $(J, {\leq})$ be a directed poset $((X_j)_{j \in J}, (\pi_{j,k})_{(j,k) \in \leq})$ be a projective system of topological spaces. Set $(X, (\pi_j)_{j \in J}) \defeq \varprojlim\limits_{j \in J} ((X_j)_{j \in J}, (\pi_{j,k})_{(j,k) \in \leq})$. Suppose that $(X, (\pi_j)_{j \in J})$ is perfect. Then
		\begin{enumerate}
			\item for each $(j,k) \in {\leq}$, $\bar{\pi}_j$ is a homeomorphism and there exist unique $\pi^Q_{j, k}: X^Q_k \rightarrow X^Q_j$ such that $\pi^Q_{j, k}$ is continuous, $\bar{\pi}_j \circ \pi^Q_j = \pi_j$, $\pi_{j,k} \circ \bar{\pi}_k = \bar{\pi}_j \circ \pi^Q_{j, k}$ and $\pi^Q_{j, k} \circ \pi^Q_k = \pi^Q_j$, i.e. the following diagram commutes:
			\[ 
			\begin{tikzcd}
				X_k  \arrow[rr, "\pi_{j,k}"]   & & X_j   \\
				X^Q_k \arrow[u, "\bar{\pi}_k"]   \arrow[rr, "\pi^Q_{j, k}"] & &  X^Q_j \arrow[u, "\bar{\pi}_j"']  \\
				& X   \arrow[ul, "\pi^Q_k"] \arrow[ur, "\pi^Q_j"'] \arrow[bend left=90]{luu}{\pi_k} \arrow[bend right=90, swap]{ruu}{\pi_j}  & 
			\end{tikzcd}
			\]
			\item $((X^Q_j)_{j \in J}, (\pi_{j,k}^Q)_{(j,k) \in \leq})$ is a projective system of topological spaces. 
			\item $\varprojlim\limits_{j \in J} X^Q_j$ is homeomorphic to $X$. \tcr{or define inverse limit generally and show that $X$ is an inverse limit for the new system $(X^Q_j, (\pi^Q_{j,k})_{(j,k) \in {\leq}}$.}
		\end{enumerate}
	\end{ex}
	
	\begin{proof}\
		\begin{enumerate}
			\item Let $j \in J$. Since $(X, (\pi_{j'})_{j' \in J})$ is perfect, $\pi_j$ is a quotient map. Since $\pi_{j,k} \circ \pi_k = \pi_j$, \rex{ex:quotient_topology:0013.3} implies that there exist unique $\bar{\pi}_j:X^Q_j \rightarrow X_j$, $\pi^Q_{j, k}: X^Q_k\rightarrow X^Q_j$ such that $\bar{\pi}_j$ is a homeomorphism, $\pi^Q_{j, k}$ is continuous, $\bar{\pi}_j \circ \pi^Q_j = \pi_j$, $\pi_{j,k} \circ \bar{\pi}_k = \bar{\pi}_j \circ \pi^Q_{j, k}$ and $\pi^Q_{j, k} \circ \pi^Q_k = \pi^Q_j$.
			\item Let $j,k,l \in J$. Suppose that $j \leq k$ and $k \leq l$. Since $((X_j')_{j' \in J}, (\pi_{j',k'})_{(j',k') \in \leq})$ is a projective system of topological spaces, $\pi_{j,j} = \id_{X_j}$ and $\pi_{j,k} \circ \pi_{k,l} = \pi_{j,l}$. 
			\begin{enumerate}
				\item Since 
				\begin{align*}
					\pi_{j,j} \circ \bar{\pi}_j
					& = \id_{X_j} \circ \bar{\pi}_j \\
					& = \bar{\pi}_j \\
					& = \bar{\pi}_j \circ \id_{X^Q_j},
				\end{align*}
				uniqueness of $\pi^Q_{j,j}$ implies that $\pi^Q_{j,j} = \id_{X^Q_j}$.
				\item By construction
				\begin{align*}
					\pi^Q_{j,k} \circ \pi^Q_{k,l}
					& = (\bar{\pi}_j^{-1} \circ \pi_{j,k} \circ \bar{\pi}_k) \circ (\bar{\pi}_k^{-1} \circ \pi_{k,l} \circ \bar{\pi}_l) \\
					& = (\bar{\pi}_j^{-1} \circ \pi_{j,k}) \circ (\bar{\pi}_k \circ \bar{\pi}_k^{-1}) \circ (\pi_{k,l} \circ \bar{\pi}_l) \\
					& = (\bar{\pi}_j^{-1} \circ \pi_{j,k}) \circ \id_{X_k} \circ (\pi_{k,l} \circ \bar{\pi}_l) \\ 
					& = (\bar{\pi}_j^{-1} \circ \pi_{j,k}) \circ (\pi_{k,l} \circ \bar{\pi}_l) \\
					& = \bar{\pi}_j^{-1} \circ (\pi_{j,k} \circ \pi_{k,l}) \circ \bar{\pi}_l \\
					& = \bar{\pi}_j^{-1} \circ \pi_{j,l} \circ \bar{\pi}_l \\
					& = \pi^Q_{j,l}
				\end{align*}
				Therefore $((X / {\sim_j})_{j \in J}, (\pi_{j,k}^Q)_{(j,k) \in \leq})$ is a projective system of topological spaces.
				\item \tcr{FINISH!!!}
			\end{enumerate}
		\end{enumerate}
	\end{proof}
	
	\begin{note}
		\tcr{Here we started with a given projective system (diagram of objects with morphisms). These objects and morphisms induce new objects and morphisms which are tied via isomorphisms to the originals. These new objects and morphisms have the same structure as the original (projective system). what is this phenomenon in general?}
	\end{note}	
	
	\begin{ex}
		\tcr{move to compactness section}
		Let $(J, {\leq})$ be a directed poset $((X_j)_{j \in J}, (\pi_{j,k})_{(j,k) \in \leq})$ be a projective system of topological spaces. Set $(X, (\pi_j)_{j \in J}) \defeq \varprojlim\limits_{j \in J} ((X_j)_{j \in J}, (\pi_{j,k})_{(j,k) \in \leq})$. For each $j \in J$, define $C_j \subset C(X)$ by 
		$$C_j \defeq \{f \in C(X): \text{ for each $x,y \in X$, $\pi_j(x) = \pi_j(y)$ implies that $f(x) = f(y)$}\}$$
		and define $C \subset C(X)$ by $C \defeq \bigcup\limits_{j \in J} C_j$. Suppose that for each $j \in J$, $X_j$ is a compact Hausdorff space and $\pi_j$ is surjective. Then 
		\begin{enumerate}
			\item $C_j = \{f \in C(X): \text{$f$ is $\sim_j$-invariant}\}$
			\item $C$ is a subalgebra of $C(X)$,
			\item $C$ is dense in $C(X)$ 
			\item for each $f \in C_j$, there exists a unique $\bar{f}_j \in C(X^Q_j)$ such that $\bar{f}_j \circ \pi^Q_j = f$. 
		\end{enumerate}
	\end{ex}
	
	\begin{proof}\
		\begin{enumerate}
			\item Clear.
			\item Let $f_1 , f_2 \in C$ and $\lam \in \C$. Then there exist $j_1, j_2 \in J$ such that $f_1 \in C_{j_1}$ and $f_2 \in C_{j_2}$. Since $J$ is directed, there exists $j_0 \in J$ such that $j_0 \geq j_1, j_2$. Let $x,y \in X$. Suppose that $\pi_{j_0}(x) = \pi_{j_0}(y)$. Then 
			\begin{align*}
				\pi_{j_1}(x)
				& = \pi_{j_1, j_0} \circ \pi_{j_0}(x) \\
				& =  \pi_{j_1, j_0} \circ \pi_{j_0}(y) \\
				& = \pi_{j_1}(y)
			\end{align*}
			and similarly, $\pi_{j_2}(x) = \pi_{j_2}(y)$. Therefore $f_1(x) = f_1(y)$ and $f_{2}(x) = f_2(y)$. Hence 
			\begin{align*}
				(f_1 + \lam f_2)(x)
				& = f_1(x) + \lam f_2(x) \\
				& = f_1(y) + \lam f_2(y) \\
				& = (f_1 + \lam f_2)(y)
			\end{align*}
			and 
			\begin{align*}
				(f_1 \cdot f_2)(x)
				& = f_1(x)f_2(x) \\
				& = f_1(y)f_2(y) \\
				& = (f_1 \cdot f_2) (y).
			\end{align*}
			Since $x, y \in X$ with $\pi_{j_0}(x) = \pi_{j_0}(y)$ are arbitrary, we have that for each $x,y \in X$, $\pi_{j_0}(x) = \pi_{j_0}(y)$ implies that $(f_1 + \lam f_2)(x) = (f_1 + \lam f_2)(y)$ and $(f_1 \cdot f_2)(x) = (f_1 \cdot f_2)(y)$. Thus 
			\begin{align*}
				f_1 + \lam f_2 
				& \in C_{j_0} \\
				& \subset C
			\end{align*} 
			and 
			\begin{align*}
				f_1 \cdot f_2
				& \in C_{j_0} \\
				& \subset C
			\end{align*} 
			Since $f_1 , f_2 \in C$ and $\lam \in \C$ are arbitrary, for each $f_1 , f_2 \in C$ and $\lam \in \C$, $f_1 + \lam f_2 \in C$ and $f_1 \cdot f_2 \in C$. Hence $C$ is a subalgebra of $C(X)$. 
			\item Since for each $j \in J$, $X_j$ is a compact Hausdorff space, \rex{ex:topology:proj_limits:0006} implies that $X$ is a compact Hausdorff space and \rex{ex:topology:proj_limits:0008} implies that $(X, (\pi_j)_{j \in J})$ is perfect. Let $j \in J$. Since $(X, (\pi_{j'})_{j' \in J})$ is perfect, \rex{ex:topology:proj_limits:0009} implies that $\bar{\pi}:X/{\sim_j} \rightarrow X_j$ is a homeomorphism. Since $X_j$ is a compact Hausdorff space and $X_j$ is homeomorphic to $X/{\sim_j}$, we have that $X/{\sim_j}$ is a compact Hausdorff space. \tcr{\rex{ex:topology:separation:quotients:0003} implies that ${\sim_j}$ is closed in $X \times X$.} Let $x,y \in X$. Suppose that $x \neq y$. \rex{arg1} \tcr{(Make exercise in set theory section about $x = y$ iff for each $j$, $\pi_j(x) = \pi_j(y)$)} implies that there exists $j \in J$ such that $\pi_j(x) \neq \pi_j(y)$. Since $\bar{\pi}:X^Q_j \rightarrow X_j$ is a homeomorphism, 
			\begin{align*}
				\pi^Q_j(x)
				& = \bar{\pi}_j^{-1} \circ \pi_j (x) \\
				& \neq \bar{\pi}_j^{-1} \circ \pi_j (y) \\
				& = \pi^Q_j(y).
			\end{align*}
			Since $X/{\sim_j}$ is a compact Hausdorff space, \rex{arg1} \tcr{(make exercise in section on uryshon thoerem to get continuous map sepearting points on compact hausdorff space)} implies that there exists $\phi \in C(X/{\sim_j})$ such that $\phi(\pi^Q_j(x)) \neq \phi(\pi^Q_j(y))$. Define $f \in C(X)$ by $f \defeq \phi \circ \pi^Q_j$. By construction $f \in C_j \subset C$ and $f(x) \neq f(y)$. Since $x,y \in X$ with $x \neq y$ are arbitrary, we have that for each $x,y \in X$, $x \neq y$ implies that there exists $f \in C$ such that $f(x) \neq f(y)$. Thus $C$ separates the points of $X$.
			\tcr{finish off stone-weierstrass theorem assumptions to get denseness}
			\item Let $f \in C_j$. Since $f$ is continuous and ${\sim_j}$-invariant, \rex{ex:quotient_topology:0015.2} implies that there exists a unique $\bar{f}_j \in C(X^Q_j)$ such that $\bar{f}_j \circ \pi^Q_j = f$.   
		\end{enumerate}
	\end{proof}
















	
	
	
	
	
	
	
	
	
	
	
	
	
	
	
	
	
	
	
	
	
	
	
	
	
	
	
	\newpage
	\section{Separation Axioms}
	\tcr{Reformat using $(X, \MT)$ and writing $U \in \MT$ instead of saying $U$ is open.}
	
	\begin{defn} \ld{def:topology:separation:0001}
		Let $X$ be a topological space. Then $X$ is said to be 
		\begin{enumerate}
			\item $\mathbf{T_0}$ of \tbf{Kolmogorov} if for each $x,y \in X$, if $x \neq y$, then there exists $U \subset X$ such that 
			\begin{enumerate}
				\item $U$ is open 
				\item $(x, y) \in U \times U^c$ or $(x,y) \in U^c \times U$
			\end{enumerate}
			\item $\mathbf{T_1}$ if for each $x,y \in X$, if $x \neq y$, then there exists $U \in \MN(x)$ such that $U$ is open and $y \not \in U$.
			\item $\mathbf{T_2}$ or \tbf{Hausdorff} if for each $x,y \in X$, if $x \neq y$, then there exist $U \in \MN(x)$ and $V \in \MN(y)$ such that $U$ and $V$ are open and $U \cap V = \varnothing$.
			\item $\mathbf{T_3}$ or \tbf{regular} if $X$ is $T_1$ and for each $A \subset X$ and $x \in A^c$, if $A$ is closed, then there exists $U \in \MN(A)$ and $V \in \MN(x)$ such that $U$ and $V$ are open and $U \cap V = \varnothing$.
			\item $\mathbf{T_4}$ or \tbf{normal} if $X$ is $T_1$ and for each $A,B \subset X$, if $A$ and $B$ are closed and $A \cap B = \varnothing$, then there exists $U \in \MN(A)$ and $V \in \MN(B)$ such that $U$ and $V$ are open and $U \cap V = \varnothing$.
		\end{enumerate}
	\end{defn}

	\begin{note}
		Some authors do not require the $T_1$ assumption for regularity or normality. 
	\end{note}

	\begin{ex} \lex{ex:topology:separation:0002}
		Let $X$ be a topological space. Then the following are equivalent:
		\begin{enumerate}
			\item $X$ is $T_1$
			\item for each $x \in X$, $\{x\}$ is closed
			\item for each $A \subset X$, $A = \bigcap\limits_{U \in \MN(A)} U$
		\end{enumerate}
	\end{ex}
	
	\begin{proof}\
		\begin{itemize}
			\item $(1) \implies (2)$: \\
			Suppose that $X$ is $T_1$. Let $x \in X$. Since $X$ is $T_1$, for each $a \in \{x\}^c$, there exists $U_{a} \in \MN(a)$ such that $U_a$ is open and $U_{a} \subset \{x\}^c$. Therefore 
			$$\{x\}^c = \bigcup_{a \in \{x\}^c} U_a$$ 
			which is open. Hence 
			$$\{x\} = \bigcap_{a \in \{x\}^c} U_a^c$$
			which is closed. \\
			\item $(2) \implies (3)$: \\
			Suppose that for each $x \in X$, $\{x\}$ is closed. Clearly, $A \subset \bigcap\limits_{U \in \MN(A)} U$. Since for each $x \in A^c$, $\{x\}^c \in \MN(A)$, we have that 
			\begin{align*}
				\bigcap_{U \in \MN(A)} U
				& \subset \bigcap_{x \in A^c} \{x\}^c \\
				& = \bigg( \bigcup_{x \in A^c} \{x\} \bigg)^c \\
				& = (A^c)^c \\
				& = A
			\end{align*}
			\item $(3) \implies (1)$: \\
			Suppose that for each $A \subset X$, $A = \bigcap\limits_{U \in \MN(A)} U$. Let $x, y \in X$. Suppose that $x \neq y$. Since $\{x\} = \bigcap\limits_{V \in \MN(x)} V$, $y \not \in  \bigcap\limits_{V \in \MN(x)} V$. Thus there exists $V \in \MN(x)$ such that $y \not \in V$. Set $U = \Int V$. Then $U \in \MN(x)$, $U$ is open and $y \not \in U$. Since $x, y \in X$ are arbitrary, $X$ is $T_1$.
		\end{itemize}
	\end{proof}

	\begin{ex} \lex{ex:topology:separation:0002.1}
		Let $(X, \MT)$ be a topological space and $A \subset X$. Suppose that $(X, \MT)$ is $T_1$. If $A$ is countable, then $A$ is an $F_{\sig}$-set.
	\end{ex}

	\begin{proof}
		Suppose that $A$ is countable. \rex{ex:topology:separation:0002} implies that for each $x \in A$, $\{x\}$ is $\MT$-closed. Since $A = \bigcup\limits_{x \in A} \{x\}$ and $A$ is countable, we have that $A$ is an $F_{\sig}$-set.  
	\end{proof}

	\begin{ex} \lex{ex:topology:separation:0003}
		Let $X$ be a topological space. Then 
		\begin{enumerate}
			\item $X$ is $T_1$ implies that $X$ is $T_0$
			\item $X$ is $T_2$ implies that $X$ is $T_1$
			\item $X$ is $T_3$ implies that $X$ is $T_2$ 
			\item $X$ is $T_4$ implies that $X$ is $T_3$ 
		\end{enumerate}
	\end{ex}
	
	\begin{proof}
		Clear by definition and the previous exercise.
	\end{proof}

	\begin{note}
		Let $X$ be a set, we recall \rd{def:set_theory:products:00010} of $\Delta_X$ an $\Delta_{X^{\N}}$.
	\end{note}
	
	\begin{ex} \lex{ex:topology:separation:0005}
		Let $X$ be a topological space. Then the following are equivalent: 
		\begin{enumerate}
			\item $X$ is Hausdorff
			\item for each net $(x_{\al})_{\al \in A} \subset X$ and $x,y \in X$, if $x_{\al} \rightarrow x$ and $x_{\al} \rightarrow y$, then $x = y$.
			\item $\Del_X$ is closed in $X \times X$.
		\end{enumerate} 
	\end{ex}
	
	\begin{proof}\
		\begin{itemize}
			\item $(1) \implies (2)$: \\
			Suppose that $X$ is Hausdorff. Let $(x_{\al})_{\al \in A} \subset X$ be a net and $x,y \in X$. Suppose that $x_{\al} \rightarrow x$ and $x_{\al} \rightarrow y$. For the sake of contradiction, suppose that $x \neq y$. Then there exist $U \in \MN(x)$ and $V \in \MN(y)$ such that $U$ and $V$ are open and $U \cap V = \varnothing$. Since $x_{\al} \rightarrow x$, $(x_{\al})_{\al \in A}$ is eventually in $U$ and there exists $\be_x \in A$ such that for each $\al \in A$, $\al \geq \be_x$ implies that $x_{\al} \in U$. Since $x_{\al} \rightarrow y$, $(x_{\al})_{\al \in A}$ is eventually in $V$ and there exists $\be_y \in A$ such that for each $\al \in A$, $\al \geq \be_y$ implies that $x_{\al} \in V$. Since $A$ is directed, there exists $\be \in A$ such that $\be \geq \be_x, \be_y$. Hence 
			\begin{align*}
				x_{\be} 
				& \in U \cap V \\
				& = \varnothing
			\end{align*}
			which is a contradiction. So $x = y$.
			\item $(2) \implies (3)$: \\
			Let $(x_{\al}, y_{\al})_{\al \in A} \subset \Del_X$ be a net and $(x,y) \in X \times X$. Then for each $\al \in A$, $x_{\al} = y_{\al}$. Suppose that $(x_{\al}, y_{\al}) \rightarrow (x,y)$. So $x_{\al} \rightarrow x$ and $x_{\al} \rightarrow y$. Hence $x = y$ and $(x,y) \in \Del_X$. Thus $\Del_X$ is closed.
			\item $(3) \implies (1)$: \\
			Suppose that $\Del_X$ is closed. Let $x,y \in X$. Suppose that $x \neq y$. Then $(x,y) \in \Del_X^c$. Recall that $\MB = \{A \times B: A,B \subset X \text{ and $A$, $B$ are open}\}$ is a basis for the product topology on $X \times X$. Since $\Del_X^c$ is open and $(x,y) \in \Del_X^c$, there exist $A \times B \in \MB$ such that $(x,y) \in A \times B \subset \Del_X^c$. Suppose that $A \cap B \neq \varnothing$. Then there exists $z \in A \cap B$. Hence $(z,z) \in A \times B$. This is a contradiction since $A  \times B \subset \Del_X^c$. Thus $x \in A$, $y \in B$ and $A \cap B = \varnothing$ and $A$, $B$ are open. Since $x,y \in X$ are arbitrary, $X$ is Hausdorff. 
		\end{itemize}
	\end{proof}

	\begin{ex} \lex{ex:topology:separation:0006}
		Let $(X, \MT)$ be a topological space. If $X$ is Hausdorff, then $\Del_{X^{\N}}$ is closed in $(X^{\N}, \MT^{\otimes \N})$.
	\end{ex}

	\begin{proof}
		Suppose that $X$ is Hausdorff. Let $(x_{\al})_{\al \in A} \subset \Del_{X^{\N}}$ and $x \in X^{\N}$. Suppose that $x_{\al} \rightarrow x$. Then for each $n \in \N$, $\pi_n(x_{\al}) \conv{\al} \pi_n(x)$. Since $(x_{\al})_{\al \in A} \subset \Del_{X^{\N}}$, we have that for each $\al \in A$ and $m,n \in \N$, $\pi_m(x_{\al}) = \pi_m(x_{\al})$. Let $n \in \N$. Then
		\begin{align*}
			\pi_n(x_{\al}) 
			& = \pi_1(x_{\al}) \\
			& \conv{\al} \pi_1(x).
		\end{align*}
		Since $\pi_n(x_{\al}) \conv{\al} \pi_n(x)$, $\pi_n(x_{\al}) \conv{\al} \pi_1(x)$ and $X$ is Hausdorff, \rex{ex:topology:separation:0005} implies that $\pi_1(x) = \pi_n(x)$. Since $n \in \N$ is arbitrary, we have that for each $m,n \in \N$, 
		\begin{align*}
			\pi_m(x)
			& = \pi_1(x) \\
			& = \pi_n(x).
		\end{align*}
		Hence $x \in \Del_{X^{\N}}$. Since $(x_{\al})_{\al \in A} \subset \Del_{X^{\N}}$ and $x \in X^{\N}$ with $x_{\al} \rightarrow x$ are arbitrary, we have that for each $(x_{\al})_{\al \in A} \subset \Del_{X^{\N}}$ and $x \in X^{\N}$, if $x_{\al} \rightarrow x$, then $x \in \Del_{X^{\N}}$. Hence $\Del_{X^{\N}}$ is closed in $(X^{\N}, \MT^{\otimes \N})$. 
	\end{proof}

	\begin{ex} \lex{ex:topology:separation:0007}
		Let $X$ be a topological space. Suppose that $X$ is $T_1$. Then $X$ is regular iff for each $x \in X$ and $U \in \MN(x)$, $U$ is open implies that there exists $V \in \MN(x)$ such that $\cl V \subset U$. 
	\end{ex}

	\begin{proof}\
		\begin{itemize}
			\item $(\implies)$: \\
			Suppose that $X$ is regular. Let $x \in X$ and $U \in \MN(x)$. Suppose that $U$ is open. Then $U^c$ is closed. Since $x \not \in U^c$, there exists $V_x \in \MN(x)$ and $V_{U^c} \in \MN(U^c)$ such that $V_x$ and $V_{U^c}$ are open and $V_x \cap V_{U^c} = \varnothing$. Therefore, $V_{U^c}^c$ is closed and $V_x \subset V_{U^c}^c \subset U$. Hence
			\begin{align*}
				x 
				& \in V_x \\
				& \subset \cl V_x \\
				& \subset \cl V_{U^c}^c \\
				& = V_{U^c}^c \\
				& \subset U
			\end{align*} 
			\item $(\impliedby)$: \\
			Suppose that for each $x \in X$ and $U \in \MN(x)$, $U$ is open implies that there exists $V \in \MN(x)$ such that $\cl V \subset U$. Let $x \in X$ and $A \subset X$. Suppose that $A$ is closed and $x \not \in A$. Then $A^c$ is open and $x \in A^c$. By assumption, there exists $V \in \MN(x)$ such that $\cl V \subset A^c$. Set $U_x = \Int V$ and $U_A = \Int V^c$. Then 
			\begin{align*}
				A 
				& \subset (\cl V)^c \\
				& = \Int V^c \\
				& = U_A
			\end{align*} 
			so that $U_x$ and $U_A$ are open, $U_x \in \MN(x)$, $U_A \in \MN(A)$ and $U_x \cap U_A = \varnothing$. Hence $X$ is regular.
		\end{itemize}
	\end{proof}

	\begin{ex}
		lemma for Uryshohns lemma
	\end{ex}

	\begin{proof}
		\tcb{FINISH!!!}
	\end{proof}


	\begin{ex} \tbf{Urysohn's Lemma for Normal Spaces:} \\
		Let $X$ be a topological space. Suppose that $X$ is normal. Let $A,B \subset X$. Suppose that $A$ and $B$ are closed and $A \cap B = \varnothing$. Then there exists $f \in C(X, [0,1])$ such that $f|_A = 0$ and $f|_B = 1$. 
	\end{ex}

	\begin{proof}
		\tcb{FINISH!!!}
	\end{proof}

	\begin{ex} \tbf{Tietze Extension Theorem for Normal Spaces:} \\
		Let $X$ be a topological space. Suppose that $X$ is normal. Let $A,B \subset X$. Suppose that $A$ and $B$ are closed and $A \cap B = \varnothing$. Then there exists $f \in C(X, [0,1])$ such that $f|_A = 0$ and $f|_B = 1$. 
	\end{ex}
	
	\begin{proof}
		\tcb{FINISH!!!}
	\end{proof}






























	\subsection{Separation and Subspaces}
	
	\begin{ex} \lex{ex:topology:separation:subspaces:0001}
		Let $(X, \MT)$ be a topological space and $A \subset X$. If $(X, \MT)$ is $T_1$, then $(A, \MT \cap A)$ is $T_1$.
	\end{ex}
	
	\begin{proof}
		Suppose that $(X, \MT)$ is $T_1$. Let $x \in A$. Since $(X, \MT)$ is $T_1$, $\{x\}$ is closed in $X$. Thus $\{x\} = \{x\} \cap A$ is closed in $(A, \MT \cap A)$. Since $x \in A$ is arbitrary, $(A, \MT \cap A)$ is $T_1$. 
	\end{proof}
	
	\begin{ex} \lex{ex:topology:separation:subspaces:0002}
		Let $(X, \MT)$ be a topological space and $A \subset X$. If $(X, \MT)$ is Hausdorff, then $(A, \MT \cap A)$ is Hausdorff.
	\end{ex}
	
	\begin{proof}
		Suppose that $(X, \MT)$ is Hausdorff. Let $x, y \in A$. Since $(X, \MT)$ is Hausdorff, there exist $U' \in \MN(x)$, $V' \in \MN(y)$ such that $U,V \in \MT$ and $U' \cap V' = \varnothing$. Set $U = U' \cap A$ and $V = V' \cap A$. Then $U, V \in \MT \cap A$, $x \in U$, $y \in V$ and 
		\begin{align*}
			U \cap V
			& = (U' \cap A) \cap (V' \cap A) \\
			& = (U' \cap V') \cap A \\
			& = \varnothing
		\end{align*}
		Since $x,y \in A$ are arbitary, $(A, \MT \cap A)$ is Hausdorff.
	\end{proof}

	\begin{ex} \lex{ex:topology:separation:subspaces:0003}
		Let $(X, \MT)$ be a topological space and $A \subset X$. If $(X, \MT)$ is regular, then $(A, \MT \cap A)$ is regular.
	\end{ex}
	
	\begin{proof}
		Suppose that $(X, \MT)$ is regular. Let $x \in A$ and $U \in \MN(x)(\MT \cap A)$. Suppose that $U \in \MT \cap A$. Then there exists $U' \in \MT$ such that $U = U' \cap A$. Since $(X, \MT)$ is regular, there exist $V' \in \MN(x)(\MT)$ such that, $\cl_{\MT} V' \subset U'$. Set $V = V' \cap A$. Then $V \in \MN(x)(\MT \cap A)$   
		\begin{align*}
			\cl_{\MT \cap A} V
			& = \cl_{\MT} V' \cap A \\
			& \subset U' \cap A
		\end{align*}
		Since $x \in A$ are arbitary, $(A, \MT \cap A)$ is regular. \\
		\tcb{FINISH!!!}
	\end{proof}































	\subsection{Separation and Product Spaces}
	
	\begin{ex} \lex{ex:topology:separation:products:0001}
		Let $(X_{\al}, \MT_{\al})_{\al \in A}$ be a collection of topological spaces. Set $X = \prod\limits_{\al \in A} X_{\al}$ and denote the product topology on $X$ by $\MT_X$. If for each $\al \in A$, $(X_{\al}, \MT_{\al})$ is $T_1$, then $(X, \MT_X)$ is $T_1$.
	\end{ex}
	
	\begin{proof}
		Suppose that for each $\al \in A$, $(X_{\al}, \MT_{\al})$ is $T_1$. Let $(x_{\al})_{\al \in A}, (y_{\al})_{\al \in A} \in X$. Suppose that $(x_{\al})_{\al \in A} \neq (y_{\al})_{\al \in A}$. Then there exists $\al_0 \in A$ such that $x_{\al_0} \neq y_{\al_0}$. Then there exists $U_{\al_0} \in \MT_{\al_0}$ such that $x_{\al_0} \in U_{\al_0}$ and $y_{\al_0} \not \in U_{\al_0}$. Set $U = \pi_{\al_0}^{-1}(U_{\al_0})$. Then $U \in \MT_X$, $(x_{\al})_{\al \in A} \in U$ and $(y_{\al})_{\al \in A} \not \in U$. Since $(x_{\al})_{\al \in A}, (y_{\al})_{\al \in A} \in X$ are arbitrary, $(X, \MT_X)$ is $T_1$.
	\end{proof}
	
	\begin{ex} \lex{ex:topology:separation:products:0002}
		Let $(X_{\al}, \MT_{\al})_{\al \in A}$ be a collection of topological spaces. Set $X = \prod\limits_{\al \in A} X_{\al}$ and $\MT \defeq \bigotimes_{\al \in A} \MT_{\al}$. If for each $\al \in A$, $(X_{\al}, \MT_{\al})$ is Hausdorff, then $(X, \MT)$ is Hausdorff.
	\end{ex}
	
	\begin{proof}
		Suppose that for each $\al \in A$, $(X_{\al}, \MT_{\al})$ is Hausdorff. Let $(x_{\al})_{\al \in A}, (y_{\al})_{\al \in A} \in X$. Suppose that $(x_{\al})_{\al \in A} \neq (y_{\al})_{\al \in A}$. Then there exists $\al_0 \in A$ such that $x_{\al_0} \neq y_{\al_0}$. Then there exists $U_{\al_0}, V_{\al_0} \in \MT_{\al_0}$ such that $x_{\al_0} \in U_{\al_0}$, $y_{\al_0} \in V_{\al_0}$ and $U_{\al_0} \cap V_{\al_0} = \varnothing$. Set $U = \pi_{\al_0}^{-1}(U_{\al_0})$ and $V = \pi_{\al_0}^{-1}(V_{\al_0})$. Then $U, V \in \MT$, $(x_{\al})_{\al \in A} \in U$, $(y_{\al})_{\al \in A} \in V$ and 
		\begin{align*}
			U \cap V
			& = \pi_{\al_0}^{-1}(U_{\al_0}) \cap \pi_{\al_0}^{-1}(V_{\al_0}) \\
			& = \pi_{\al_0}^{-1}(U_{\al_0} \cap V_{\al_0}) \\
			& = \pi_{\al_0}^{-1}(\varnothing) \\
			& = \varnothing
		\end{align*}
		Since $(x_{\al})_{\al \in A}, (y_{\al})_{\al \in A} \in X$ are arbitrary, $(X, \MT)$ is Hausdorff.
	\end{proof}
	
	\begin{ex} \lex{ex:topology:separation:products:0003}
		Let $(X_{\al}, \MT_{\al})_{\al \in A}$ be a collection of topological spaces. Set $X = \prod\limits_{\al \in A} X_{\al}$ and denote the product topology on $X$ by $\MT_X$. If for each $\al \in A$, $(X_{\al}, \MT_{\al})$ is regular, then $(X, \MT_X)$ is regular. 
	\end{ex}

	\begin{proof}
		Let $x \in X$ and $U \in \MN(x)$. Suppose that $U$ is open. Set 
		$$\MB = \bigg \{\prod_{\al \in A}B_{\al}: \text{ for each $\al \in A$,  $B_{\al} \in \MT_{\al}$ and $\# \{\al \in A: B_{\al} \neq X_{\al}\} < \infty$} \bigg\}$$
		Then $\MB$ is a basis for $\MT_{X}$. So for each $\al \in A$, there exist $U_{\al} \in \MT_{\al}$ such that $\# \{\al \in A: B_{\al} \neq X_{\al}\} < \infty$ and $x \in \prod\limits_{\al \in A} U_{\al} \subset U$. Set $J = \{\al \in A: B_{\al} \neq X_{\al}\}$. Let $\al \in A$. Suppose that $\al \in J$. Then $x_{\al} \in U_{\al}$. Since $U_{\al} \in \MN(x)$ is an open neighborhood of $x_{\al}$ and $X_{\al}$ is regular, the previous exercise implies that there exists $V_{\al} \in \MN(x_{\al})$ such that $\cl V_{\al} \subset U_{\al}$. If $\al \in J^c$, set $V_{\al} = X_{\al}$. Define $V = \prod\limits_{\al \in A} V_{\al}$. Then $V \in \MN(x)$ and an exercise in the section on the product topology implies that 
		\begin{align*}
			\cl V
			& = \cl \prod_{\al \in A} V_{\al} \\
			& = \prod_{\al \in A} \cl V_{\al} \\
			& \subset \prod_{\al \in A} U_{\al} \\
			& \subset U
		\end{align*} 
	\end{proof}
	
	
	
	
	
	
	
	
	
	
	
	
	
	
	
	
	
	\subsection{Separation and Quotient Spaces}
	
	\begin{defn} \ld{def:topology:separation:quotients:0001}
		Let $(X, \MT)$ be a topological space. Define $\sim_0 \subset X \times X$ by
		$${\sim_0}  = \{(x,y) \in X \times X: \text{for each $U \in \MT$, $x \in U$ iff $y \in U$}\}.$$
	\end{defn}

	\begin{ex} \lex{ex:topology:separation:quotients:0002}
		Let $(X, \MT)$ be a topological space. Then
		\begin{enumerate}
			\item $\sim_0$ is an equivalence relation on $X$,
			\item for each $x,y \in X$, $x \not \sim_0 y$ iff there exists $U \in \MT$ such that $(x,y) \in U \times U^c$ or $(x,y) \in U^c \times U$,
			\item for each $x,y \in X$, $x \sim_0 y$ iff $\cl \{x\} = \cl \{y\}$,
			\item $X$ is $T_0$ iff for each $x,y \in X$, $x \neq y$ iff $x \not \sim_0 y$,
			\item 
			\begin{enumerate}
				\item for each $U \in \MT$, $\pi^{-1}(\pi(U)) = U$ 
				\item $\pi: X \rightarrow X / \sim_0$ is open,
			\end{enumerate}
			\item $X/ \sim_0$ is $T_0$,
			\item $X$ is $T_0$ iff $X/\sim_0 \cong X$.
		\end{enumerate}
	\end{ex}

	\begin{proof}\
		\begin{enumerate}
			\item Let $x,y,z \in X$.
			\begin{itemize}
				\item Clearly $x \sim_0 x$. 
				\item Clearly $x \sim_0 y$ implies that $y \sim_0 x$. 
				\item Suppose that $x \sim_0 y$ and $y \sim_0 z$. Let $U \in \MT$. 
				\begin{itemize}
					\item Suppose that $x \in U$. Since $x \sim_0 y$, we have that $y \in U$. Since $y \sim_0 z$, we have that $z \in U$. Thus $x \in U$ implies that $z \in U$.
					\item Similarly, $z \in U$ implies that $x \in U$. 
				\end{itemize}
				Thus $x \in U$ iff $z \in U$. Since $U \in \MT$ is arbitrary, we have that for each $U \in \MT$, $x \in U$ iff $z \in U$. Therefore $x \sim_0 z$.
			\end{itemize}
			Since $x,y,z \in X$ are arbitrary, we have that $\sim_0$ is an equivalence relation on $X$.
			\item Let $x, y \in X$. By definition, 
			\begin{align*}
				x \not \sim_0 y
				& \iff \neg [\forall U \in \MT, (x \in U \iff y \in U)] \\
				& \iff \exists U \in \MT \text{ s.t. } \neg [(x \in U \implies y \in U) \text{ and } (y \in U \implies x \in U)] \\
				& \iff \exists U \in \MT \text{ s.t. } \neg (x \in U \implies y \in U) \text{ or } \neg (y \in U \implies x \in U) \\
				& \iff \exists U \in \MT \text{ s.t. } (x \in U \wedge y \not \in U) \text{ or } (y \in U \wedge x \not \in U) \\
				& \iff \exists U \in \MT \text{ s.t. } (x,y) \in U \times U^c \text{ or } (x,y) \in U^c \times U
			\end{align*}
			\item Let $x,y \in X$. 
			\begin{itemize}
				\item $(\implies):$ \\
				Suppose that $x \sim_0 y$. Define $\MC_x \defeq \{C \subset X: \text{$C$ is closed and $x \in C$}\}$ and $\MC_y \defeq \{C \subset X: \text{$C$ is closed and $y \in C$}\}$. Let $C \in \MC_x$. For the sake of contradiction, suppose that $y \not \in C$. Then $y \in C^c$. Since $x \sim_0 y$, $C^c \in \MT$ and $y \in C^c$, we have that $x \in C^c$. This is a contradiction since $x \in C$. Thus $y \in C$ and therefore $C \in \MC_y$. Since $C \in \MC_x$ is arbitrary, we have that $\MC_x \subset \MC_y$. Similarly, $\MC_y \subset \MC_x$. Thus $\MC_y = \MC_x$. Therefore 
				\begin{align*}
					\cl \{x\} 
					& = \bigcap_{C \in \MC_x} C \\
					& = \bigcap_{C \in \MC_y} C \\
					& = \cl \{y\}
				\end{align*}
				\item $(\impliedby):$ \\
				Suppose that $x \not \sim_0 y$. Then part $(2)$ implies that there exists $U \in \MT$ such that $(x,y) \in U \times U^c$ or $(x,y) \in U^c \times U$. 
				\begin{itemize}
					\item Suppose that $(x,y) \in U \times U^c$. Then $U^c \in \MC_y$ and therefore 
					\begin{align*}
						\cl \{y\}
						& = \bigcap_{C \in \MC_y} C \\
						& \subset U^c 
					\end{align*}
					Since $x \in U$, $\{x\} \cap \cl \{y\} = \varnothing$. Since $x \in \cl\{x\}$, we have that $\cl \{x\} \not \subset \cl \{y\}$. Hence $\cl \{x\} \neq \cl \{y\}$.
					\item Similarly, if $(x,y) \in U^c \times U$, then $\cl \{x\} \neq \cl \{y\}$.
				\end{itemize}
			\end{itemize}
			\item 
			\begin{itemize}
				\item $(\implies):$ \\
				Suppose that $X$ is $T_0$. Let $x,y \in X$. Since $X$ is $T_0$, we have that
				\begin{align*}
					x \neq y
					& \iff \exists U \in \MT \text{ s.t. $(x,y) \in U \times U^c$ or $(x,y) \in U^c \times U$ } \\
					& \iff x \not \sim_0 y 
				\end{align*}
				\item $(\impliedby):$ \\
				Suppose that for each $x,y \in X$, $x \neq y$ iff $x \not \sim_0 y$. Let $x,y \in X$. Suppose that $x \neq y$. By assumption, $x \not \sim_0 y$. Thus there exists $U \in \MT$ such that $(x, y) \in U \times U^c$ or $(x, y) \in U^c \times U$. Since $x,y \in X$ are arbitrary, we have that $X$ is $T_0$.
			\end{itemize}
			\item 
			\begin{enumerate}
				\item Let $U \in \MT$ and $x \in \pi^{-1}(\pi(U))$. Then $\pi(x) \in \pi(U)$. Thus there exists $y \in U$ such that $\pi(x) = \pi(y)$. By definition of $\pi$, $x \sim_0 y$. By definition of $\sim_0$, $x \in U$. Since $x \in \pi^{-1}(\pi(U))$ is arbitrary, we have that $\pi^{-1}(\pi(U)) \subset U$. \rex{ex:set_theory:functions:0001} implies that $U \subset \pi^{-1}(\pi(U))$. Thus $\pi^{-1}(\pi(U)) = U$. Since $U \in \MT$ is arbitrary, we have that for each $U \in \MT$, $\pi^{-1}(\pi(U)) = U$. 
				\item In particular, for each $U \in \MT$, 
				\begin{align*}
					\pi^{-1}(\pi(U)) 
					& = U \\
					& \in \MT 
				\end{align*}
				and since $\pi$ is a quotient map, \rex{ex:quotient_topology:0008} implies that $\pi$ is open. 
			\end{enumerate}
		
		
%			\rex{ex:set_theory:functions:0001}
%			 Set $U' \defeq U$. Then $U' \in \MT$ and  
%			\begin{align*}
%				x
%				& \in U' \\
%				& \subset \pi^{-1}(\pi(U)) 
%			\end{align*}
%			Since $x \in \pi^{-1}(\pi(U))$ is arbitrary, we have that for each $x \in \pi^{-1}(\pi(U))$, there exists $U' \in \MT$ such that $x \in U'$ and $U' \subset \pi^{-1}(\pi(U))$. Hence $\pi^{-1}(\pi(U)) \in \MT$. Since $U \in \MT$ is arbitrary, we have that for each $U \in \MT$, $\pi^{-1}(\pi(U)) \in \MT$. Since $\pi: X \rightarrow X / \sim_0$ is a quotient map, 
			\item 	Let $a,b \in X / \sim_0$. For the sake of contradiction, suppose that $a \sim_0 b$ and $a \neq b$. Since $\pi$ is surjective, there exist $x,y \in X$ such that $\pi(x) = a$ and $\pi(y) = b$. Since $\pi(x) \neq \pi(y)$, $x \not \sim_0 y$. Therefore there exists $U \in \MT$ such that $(x,y) \in U \times U^c$ or $(x,y) \in U^c \times U$. 
			\begin{itemize}
				\item Suppose that $(x,y) \in U \times U^c$. Since $x \in U$ and $\pi$ is open, we have that
				\begin{align*}
					a
					& = \pi(x) \\
					& \in \pi(U)
				\end{align*}
				Since $a \sim_0 b$, $a \in \pi(U)$ and $\pi(U) \in \MT_{X/\sim_0}$, we have that $b \in \pi(U)$. Therefore 
				\begin{align*}
					y 
					& \in \pi^{-1}(\pi(U)) \\
					& = U
				\end{align*}
				This is a contradiction since $y \in U^c$. 
				\item Similarly, if $(x,y) \in U^c \times U$, then $x \in U$ which is a contradiction. 
			\end{itemize}
			Hence $a \sim_0 b$ implies that $a = b$. \\
			Conversely, since $\sim_0$ is an equivalence relation, $a = b$ implies that $a \sim_0 b$. \\
			Thus $a = b$ iff $a \sim_0 b$. Since $a, b \in X/\sim_0$ are arbitrary, we have that for each $a,b \in X/\sim_0$, $a = b$ iff $a \sim_0 b$. Part $(4)$ implies that $X / \sim_0$ is $T_0$.
			\item \tcr{FINISH!!!}
		\end{enumerate}
	\end{proof}
	
	\begin{ex} \lex{ex:topology:separation:quotients:0003}
		Let $X$ be a topological space and $\sim$ an equivalence relation on $X$. If $\pi: X \rightarrow X / \sim$ is open, then $X / \sim$ is Hausdorff iff $\sim$ is closed in $X \times X$.
	\end{ex}
	
	\begin{proof}
		Suppose that $\pi:X \rightarrow X / \sim$ is open. 
		\begin{itemize}
			\item $(\implies)$: \\
			Suppose that $X/\sim$ is Hausdorff. Let $(x_{\al}, y_{\al})_{\al \in A} \subset {\sim}$ be a net and $(x,y) \in X \times X$. Suppose that $x_{\al}, y_{\al} \rightarrow (x,y)$. Then $x_{\al} \rightarrow x$ and $y_{\al} \rightarrow y$. Since $\pi:X \rightarrow X / \sim$ is continuous, $\pi(x_{\al}) \rightarrow \pi(x)$ and $\pi(y_{\al}) \rightarrow \pi(y)$. Since for each $\al \in A$, $x_{\al} \sim y_{\al}$, we have that 
			\begin{align*}
				\pi(x_{\al}) 
				& = \pi(y_{\al})\\
				& \rightarrow \pi(y)
			\end{align*}
			Since $X/ \sim$ is Hausdorff, $\pi(x) = \pi(y)$. Hence $x \sim y$ and $(x,y) \in {\sim}$. Thus $\sim$ is closed in $X \times X$.\\
			\item $(\impliedby)$: \\
			Conversely, suppose that $\sim$ is closed in $X \times X$ is closed. Let $\bar{x}, \bar{y} \in X / \sim$. Suppose that $\bar{x} \neq \bar{y}$. Then $(x,y) \in {\sim^c}$. Recall that $\MB =\{A \times B: A,B \subset X \text{ and $A, B$ are open} \}$ is a basis for $X \times X$. Since  ${\sim^c}$ is open and $(x,y) \in {\sim^c}$, there exist $A,B \subset X$ such that $A,B$ are open and $(x,y) \in A \times B \subset  {\sim^c}$. Thus $x \in A$ and $y \in B$. Since $\pi$ is open, $\pi(A) = \bar{A}$ and $\pi(B) = \bar{B}$ are open. Suppose for the sake of contradiction that $\pi(A) \cap \pi(B) \neq \varnothing$. Then there exists $z \in X$ such that $\bar{z} \in \pi(A) \cap \pi(B)$. Therefore there exist $z_A \in A$ and $z_B \in B$ such that $z_A \sim z$ and $\sim z_B$. Then $(z_A, z_B) \in A \times B$ and $(z_A, z_B) \in {\sim}$. This is a contradiction since $A \times B \subset  {\sim^c}$. So $\pi(A) \cap \pi(B) = \varnothing$. Thus $\bar{x} \in \pi(A)$, $\bar{y} \in \pi(B)$, $\pi(A)$, $\pi(B)$ are open and $\pi(A) \cap \pi(B) = \varnothing$. Since $\bar{x}, \bar{y} \in X / {\sim}$ are arbitrary, $X / {\sim}$ is Hausdorff.
		\end{itemize}
	\end{proof}
	
	
	
	
	
	
	
	
	
	
	
	
	
	
	
	
	\newpage
	\section{Countability Axioms}
	
	\subsection{First-Countability}
	
	\begin{defn} \ld{def:topology:countability:0001}
		Let $(X, \MT)$ be a topological space. Then $(X, \MT)$ is said to be \tbf{first-countable} if for each $x \in X$, there exists $\MB_x \subset \MT$ such that 
		\begin{enumerate}
			\item $\MB_x$ is a local basis for $\MT$ at $x$
			\item $\MB_x$ is countable
		\end{enumerate}
	\end{defn}

	\begin{ex} \lex{ex:topology:countability:0002}
		Let $(X, \MT)$ be a topological space. Suppose that $(X, \MT)$ is first-countable. Then for each $x \in X$, there exists $(U_{x, n})_{n \in \N} \subset \MT$ such that 
		\begin{enumerate}
			\item $(U_{x, n})_{n \in \N}$ is a local basis for $\MT$ at $X$
			\item for each $n \in \N$, $U_{x, n+1} \subset U_{x, n}$
		\end{enumerate}
	\end{ex}

	\begin{proof}\
		\begin{enumerate}
			\item Let $x \in X$. Since $(X, \MT)$ is first-countable, there exists $(E_{x, j})_{j \in \N} \subset \MT$ such that $(E_{x, j})_{j \in \N}$ is a local basis for $\MT$ at $x$. Define $(U_{x,n})_{n \in \N} \subset \MT$ by 
			$$U_{x,n} = \bigcap_{j=1}^n E_{x,j}$$
			\begin{itemize}
				\item Since $(E_{x,j})_{j \in \N}$ is a local basis for $\MT$ at $x$, for each $j \in \N$, $x \in E_{x,j}$. Therefore for each $n \in \N$, 
				\begin{align*}
					x 
					& \in \bigcap_{j=1}^n E_{x, j} \\
					& = U_{x,n}
				\end{align*}
				\item Let $V \in \MT$. Suppose that $x \in V$. Since $(E_{x,j})_{j \in \N}$ is a local basis for $\MT$ at $x$, there exists $n \in \N$ such that $E_{x,n} \subset V$. Then 
				\begin{align*}
					U_{x,n}
					& = \bigcap_{j=1}^n E_{x, j} \\
					& \subset E_{x, n} \\
					& \subset V
				\end{align*}
			\end{itemize}
			Thus $(U_{x,n})_{n \in \N}$ is a local basis for $\MT$ at $x$. 
			\item By construction, for each $n \in \N$,
			\begin{align*}
				U_{x, n+1}
				& = \bigcap_{j=1}^{n+1} E_{x, j} \\
				& \subset \bigcap_{j=1}^n E_{x, j} \\
				& = U_{x, n}
			\end{align*}
		\end{enumerate}
	\end{proof}

	\begin{ex} \lex{ex:topology:countability:0003}
		Let $(X, \MT_X), (Y, \MT_Y)$ be topological spaces and $f: X \rightarrow Y$. Suppose that $(X, \MT)$ is first-countable. Then $f$ is continous iff for each sequence $(x_n)_{n \in \N} \subset X$ and $x \in X$, $x_n \rightarrow x$ implies that $f(x_n) \rightarrow x$.
	\end{ex}

	\begin{proof}\
		\begin{itemize}
			\item $(\implies)$: \\ 
			Suppose that $f$ is continuous. Let $(x_n)_{n \in \N} \subset X$ be a sequence and $x \in X$. Suppose that $x_n \rightarrow x$. Since $(x_n)_{n \in \N}$ is a net, a previous exercise implies that $f(x_n) \rightarrow x$. \\
			\item $(\impliedby)$: \\
			Conversely, suppose that for each sequence $(x_n)_{n \in \N} \subset X$ and $x \in X$, $x_n \rightarrow x$ implies that $f(x_n) \rightarrow x$. Since $(X, \MT)$ is first-countable, the previous exercise implies that there exists $(U_{x, n})_{n \in \N} \subset \MT$ such that 
			\begin{enumerate}
				\item $(U_{x, n})_{n \in \N}$ is a local basis for $\MT$ at $X$
				\item for each $n \in \N$, $U_{x, n+1} \subset U_{x, n}$
			\end{enumerate}
			For the sake of contradiction, suppose that $f$ is not continuous. Then there exists $x \in X$ such that $f$ is not continuous at $x$. Thus there exists $V \in \MN(f(x))$ such that for each $U \in \MN(x)$, $f(U) \not \subset V$.
			In particular, for each $n \in \N$, $f(U_{x,n}) \cap V^c \neq \varnothing$ and therefore $U_{x,n} \cap f^{-1}(V^c) \neq \varnothing$. The axiom of choice implies that there exists a sequence $(x_n)_{n \in \N} \subset X$ such that for each $n \in \N$, $x_n \in U_{x,n}$ and $f(x_n) \in V^c$. \\
			Let $U \in \MN(x)$. Since $(U_{x,n})_{n \in \N}$ is a local basis for $\MT$ at $x$, there exists $N \in \N$ such that $U_{x,N} \subset \Int U$. Then for each $n \in \N$, $n \geq N$ implies that 
			\begin{align*}
				x_n
				& \in U_{x,n} \\
				& \subset U_{x,N} \\
				& \subset \Int U \\
				& \subset U
			\end{align*}
			Hence $(x_n)_{n \in \N}$ is eventually in $U$. Since $U \in \MN(x)$ is arbitrary, $x_n \rightarrow x$. By assumption, $f(x_n) \rightarrow f(x)$. This is a contradiction since for each $n \in \N$, $f(x_n) \in V^c$ and therefore it is not the case that $(f(x_n))_{n \in \N}$ is eventually in $V$. Hence $f$ is continuous.  
		\end{itemize}
	\end{proof}
	
	
	\begin{ex} \lex{ex:topology:countability:0004}
		Let $(X_n, \MT_n)_{n \in \N} \subset \Obj(\Top)$. If for each $n \in \N$, $(X_n, \MT_n)$ is first-countable, then $\bigg( \prod\limits_{n \in \N} X_n, \bigotimes_{n \in \N} \MT_n \bigg)$ is first-countable. 
	\end{ex}

	\begin{proof}
		Set $X = \prod\limits_{n \in \N} X_n$ and $\MT = \bigotimes_{n \in \N} \MT_n$. Let $x \in X$. Since for each $n \in \N$, $X_n$ is first-countable, we have that for each $n \in \N$, there exists $\MB_{x_n} \subset \MT_n$ such that  
		\begin{enumerate}
			\item $\MB_{x_n}$ is a local basis for $\MT_n$ at $x_n$
			\item $\MB_{x_n}$ is countable
		\end{enumerate}
		Set 
		$$\MB_x = \bigg \{\prod_{n \in \N} U_n: \text{ [for each $n \in \N$, $U_n \in \MT_n$ and $U_n \neq X_n$ implies that $U_n \in \MB_{x_n}$] and $\# \{n \in \N: U_n \neq X_n\} < \infty$} \bigg\}$$
		Then $\MB_x$ is countable and an exercise in the section on the product topology implies that $\MB_x$ is a local basis for $\MT$ at $x$. Since $x \in X$ is arbitrary, we have that for each $x \in X$, there exists $\MB_x \subset \MT$ such that 
		\begin{enumerate}
			\item $\MB_x$ is a local basis for $\MT$ at $x$
			\item $\MB_x$ is countable
		\end{enumerate}
		Hence $(X, \MT)$ is first-countable.
	\end{proof}


	\begin{ex} \lex{ex:topology:countability:0005}
		Let $(X, \MT)$ be a topological space, $(x_n)_{n \in \N} \subset X$ and $x \in X$. Suppose that $(X, \MT)$ is first countable and $x$ is a cluster point of $(x_n)_{n \in \N}$. Then there exists $(x_{n_k})_{k \in \N} \subset (x_n)_{n \in \N}$ such that $x_{n_k} \rightarrow x$. 
	\end{ex}

	\begin{proof}\
		\tcr{redo or give alternative proof while citing \rex{ex:nets:0026}}
		\begin{itemize}
			\item Since $(X, \MT)$ is first-countable, there exists $(U_n)_{n \in \N} \subset \MT$ such that $(U_n)_{n \in \N}$ is a local basis for $\MT$ at $x$. Define $(V_n)_{n \in \N} \subset \MT$ by $V_n \defeq \bigcap\limits_{j=1}^n U_j$. Then for each $n \in \N$, $V_n \in \MN(x)$. 
			\item We define $(A_j)_{j,k \in \N} \subset \MP(\N)$ by $A_{k,j} \defeq \{n \in \N: n \geq j  \text{ and } x_n \in V_k\}$. Since $x$ is a cluster point of $(x_n)_{n \in \N}$, for each $j,k \in \N$, $A_{k,j} \neq \varnothing$.  
			\item We define $(n_k)_{k \in \N} \subset \N$ by 
			\[
			n_k \defeq 
			\begin{cases}
				\min A_{1, 1}, & k = 1 \\
				\min A_{k, n_{k-1} + 1}, & k \geq 2
			\end{cases} 
			\]
			By construction for each $k \in \N$, $n_k > n_{k-1}$. Thus $(x_{n_k})_{k \in \N} \subset (x_n)_{n \in \N}$. 
			\item Let $U \in \MN(x)$. Since $(U_n)_{n \in \N}$ is a local basis for $\MT$ at $x$, there exists $k_0 \in \N$ such that 
			\begin{align*}
				x 
				& \in V_{k_0} \\
				& \subset U_{k_0} \\
				& \subset \Int U \\
				& \subset U
			\end{align*}
			Let $k \in \N$. Suppose that $k \geq k_0$. Since 
			\begin{align*}
				n_k 
				& > n_{k-1} \\
				& \geq n_{k_0},
			\end{align*}
			we have that $V_{n_k} \subset V_{n_{k_0}}$ and
			\begin{align*}
				x_{n_k}
				& \in V_{n_k} \\
				& \subset V_{k_0} \\
				& \subset U.
			\end{align*}
			Hence there exists $k_0 \in \N$ such that for each $k \in \N$, $k \geq k_0$ implies that $x_{n_k} \in U$. Thus $(x_{n_k})$ is eventually in $U$. Since $U \in \MN(x)$ is arbitrary, we have that for each $U \in \MN(x)$, $(x_{n_k})$ is eventually in $U$. Therefore $x_{n_k} \rightarrow x$ in $(X, \MT)$. 
		\end{itemize}
	\end{proof}

	
	
	
	
	
	
	
	
	
	
	
	
	
	
	
	
	
	
	
	
	
	
	
	
	
	
	
	
	
	
	
	
	
	
	
	
	
	
	
	
	\subsection{Second-Countability}

	\begin{defn} \ld{def:topology:countability:0005}
		Let $(X, \MT)$ be a topological space. Then $(X, \MT)$ is said to be \tbf{second-countable} if there exists $\MB \subset \MT$ such that 
		\begin{enumerate}
			\item $\MB$ is a basis for $\MT$
			\item $\MB$ is countable
		\end{enumerate} 
	\end{defn}

	\begin{ex} \lex{ex:topology:countability:0007}
		Let $(X, \MT_X)$, $(Y, \MT_Y)$ be topological spaces and $f:X \rightarrow Y$. Suppose that $f$ is surjective, continuous and open. If $X$ is second countable, then $Y$ is second-countable. 
	\end{ex}

	\begin{ex} \lex{ex:topology:countability:0007.1}
		\tcr{second-countable implies first-countable}
	\end{ex}

	\begin{proof}
		\tcr{FINISH!!!}
	\end{proof}

	\begin{proof}
		Suppose that $X$ is second-countable. Then there exists $\MB_X \subset \MT_X$ such that $\MB_X$ is a basis for $\MT_X$ and $\MB_X$ is countable. Set $\MB_{Y} = \{f(A):A \in A \in \MT_X\}$. Since $\MB_X$ is countable, $\MB_Y$ is countable. Since $f$ is surjective, continuous and open, a previous exercise implies that $\MB_{Y}$ is a basis for $\MT_Y$. Hence $(Y, \MT_Y)$ is second countable.
	\end{proof}

	\begin{ex} \lex{ex:topology:countability:0008}
		Let $(X, \MT_X)$, $(Y, \MT_Y)$ be topological spaces and $f:X \rightarrow Y$. Suppose that $f$ is a homoemorphism. Then $(X, \MT_X)$ is second countable iff $(Y, \MT_Y)$ is second countable.
	\end{ex}

	\begin{proof}\
		\begin{itemize}
			\item $(\implies):$ \\
			Suppose that $(X, \MT_X)$ is second-countable. Since $f$ is surjective, continuous and open, the previous exercise implies that $(Y, \MT_Y)$ is second countable.
			\item $(\impliedby):$ \\
			Conversely, suppose that $(Y, \MT_Y)$ is second-countable. Since $f^{-1}: Y \rightarrow X$ is  surjective, continuous and open, the previous exercise implies that $(X, \MT_X)$ is second countable.
		\end{itemize}
	\end{proof}

	\begin{defn} \ld{def:topology:countability:0009}
		Let $(X, \MT)$ be a topological space. Then $(X, \MT)$ is said to be \tbf{separable} if there exists $S \subset X$ such that $S$ is dense in $X$ and $S$ is countable.
	\end{defn}

	\begin{ex} \lex{ex:topology:countability:00010}
		Let $(X, \MT)$ be a topological space. If $(X, \MT)$ is second-countable, then $(X, \MT)$ is separable. 
	\end{ex}

	\begin{proof}
		Suppose that $(X, \MT)$ is second-countable. Then there exists $\MB \subset \MT$ such that $\MB$ is a basis for $\MT$ and $\MB$ is countable. The axiom of choice implies that there exists $(x_U)_{U \in \MB} \subset X$ such that each $U \in \MB$, $x_U \in U$. Let $V \in \MT$. Suppose that $V \neq \varnothing$. Then there exists $x \in V$. Since $\MB$ is a basis for $\MT$, there exists $U \in \MB$ such that $x \in U \subset V$. Hence $x_U \in (x_U)_{U \in \MB} \cap V$ which implies that $(x_U)_{U \in \MB} \cap V \neq \varnothing$. Since $V \in \MT$ such that $V \neq \varnothing$ is arbitrary, we have that for each $V \in \MT$, $V \neq \varnothing$ implies that $(x_U)_{U \in \MB} \cap V \neq \varnothing$. \rex{31022} implies that $(x_U)_{U \in \MB}$ is dense in $X$. Since $\MB$ is countable, $(X, \MT)$ is separable.
	\end{proof}

	\begin{ex} \lex{ex:topology:countability:00010.1}
		Let $(X, \MT), (Y, \MT_Y)$ be a topological spaces and $f: X \rightarrow Y$ a $(\MT_X, \MT_Y)$-homeomorphism. Then $(X, \MT)$ is second-countable iff $(Y, \MT_Y)$ is second-countable.
	\end{ex}

	\begin{proof}\
		\begin{itemize}
			\item $(\implies): $ \\
			Suppose that $(X, \MT)$ is second-countable. Then there exists $\MB_X \subset \MT_X$ such that $\MB_X$ is a basis for $\MT_X$ and $\MB_X$ is countable. Define $\MB_Y \defeq \{f(U): U \in \MB_X\}$. Since $f$ is a $(\MT_X, \MT_Y)$-homeomorphism, we have that $f$ is open and $\MB_Y \subset \MT_Y$. Since $\MB_X$ is countable, $\MB_Y$ is countable. Let $V \in \MT_Y$ and $y \in V$. Set $U \defeq f^{-1}(V)$ and $x \defeq f^{-1}(y)$. Since $f$ is continuous, $U \in \MT_X$. By construction, $x \in U$. Since $\MB_X$ is a basis for $\MT_X$, there exists $B_X \in \MB_X$ such that $x \in B_X$ and $B_X \subset U$. Set $B_Y \defeq f(B_X)$. By definition, $B_Y \in \MB_Y$ and 
			\begin{align*}
				y
				& = f(x) \\
				& \in f(B_X) \\
				& \subset f(U) \\
				& = V
			\end{align*}
			and 
			\begin{align*}
				y
				& \in f(B_X) \\
				& = B_Y .
			\end{align*}
			Since $V \in \MT_Y$ and $y \in V$ are arbitrary, we have that for each $V \in \MT_Y$ and $y \in V$, there exists $B_Y \in \MB_Y$ such that $y \in B_Y \subset V$. Hence $\MB_Y$ is a basis for $\MT_Y$.
			\item $(\impliedby): $ \\
			Similar to $(\implies)$.
		\end{itemize}
	\end{proof}

	\begin{defn} \ld{def:topology:countability:0011}
		Let $X$ be a topological space. Then $X$ is said to be \tbf{Lindel\"{o}f} if for each open cover $\MU$ of $X$, there exists a subcover $\MU' \subset \MU$ of $X$ such that $\MU'$ is countable.\\
		\tcr{NEED TO DEFINE SUBCOVER}\\
		\tcr{FINISH!!!}
	\end{defn}

	\begin{ex} \lex{ex:topology:countability:0012}
		Let $(X, \MT)$ be a topological space. If $(X, \MT)$ is second countable, then $(X, \MT)$ is Lindel\"{o}f. 
	\end{ex}

	\begin{proof}
		Suppose that $X$ is second countable. Then there exists $\MB \subset \MT$ such that $\MB$ is a basis for $\MT$ and $\MB$ is countable. Let $\MU$ be an open cover of $X$. For $B \in \MB$, define $\MU_{B} \subset \MU$ by 
		$\MU_{B} = \{U \in \MU: B \subset U\}$. Set $\Gam = \{B \in \MB: \MU_B \neq \varnothing\}$. The axiom of choice implies that there exists $(V_B)_{B \in \Gam} \subset \MU$ such that for each $B \in \Gam$, $V_B \in \MU_B$. Set $\MU' = (V_B)_{B \in \Gam}$. Let $x \in X$. Since $\MU$ is an open cover of $X$, there exists $U \in \MU$ such that $x \in U$. Since $\MB$ is a basis for $\MT$, there exists $B \in \MB$ such that $x \in B \subset U$. Thus $B \in \Gam$ and $x \in V_B$. So $\MU'$ is an subcover of $\MU$. Since $\MB$ is countable, $\MU'$ is countable. Since $\MU$ is an arbitrary open cover of $X$, we have that for each open cover $\MU$ of $X$, there exists a countable subscover $\MU' \subset \MU$. Hence $(X, \MT)$ is Lindel\"{o}f.
	\end{proof}

	\begin{ex} \lex{ex:topology:countability:0012.02}
		Let $(X, \MT)$ be a topological space. Define $C \subset X$ by 
		$$C \defeq \{x \in X: \text{$x$ is a $\MT$-condensation point of $X$}\}.$$
		If $(X, \MT)$ is second-countable, then $C^c$ is countable. \\
		\tbf{Hint:} \rex{31027}
	\end{ex}
	
	\begin{proof} Suppose that $(X, \MT)$ is second-countable. Then there exists $\MB \subset \MT$ such that $\MB$ is a basis for $\MT$ and $\MB$ is countable. Let $x \in C^c$. \rex{31027} implies that there exists $U \in \MT$ such that $x \in U$, $U$ is countable and $U \subset C^c$. Since $\MB$ is a basis for $\MT$, there exists $B \in \MB$ such that $x \in B$ and $B \subset U$. Then $B$ and $B \subset C^c$ is countable. Since $x \in C^c$ is arbitrary, we have that for each $x \in X^x$, there exists $B \in \MB$ such that $x \in B$, $B$ is countable and $B \subset C^c$. The axiom of choice implies that there exists $(B_x)_{x \in C^c} \subset \MB$ such that for each $x \in C^c$, $x \in B_x$, $B_x$ is countable and $B_x \subset C^c$. Define $\MB_0 \subset \MB$ by $\MB_0 \defeq \{B \in \MB: \text{$B$ is countable}\}$. Then $\bigcup\limits_{B \in \MB_0} B $ is countable. Since
		\begin{align*}
			C^c
			& = \bigcup_{x \in C^c} B_x \\
			& \subset \bigcup_{B \in \MB_0} B, 
		\end{align*}
		we have that $C^c$ is countable.
	\end{proof}














	

	\subsubsection{Second-Countability and Subspaces}
	
	\begin{ex} \lex{ex:topology:countability:0013}
			Let $(X, \MT)$ be a topological space and $A \subset X$. If $(X, \MT)$ is second-countable, then $(A, \MT \cap A)$ is second-countable.
	\end{ex}

	\begin{proof}
		Suppose that $(X, \MT)$ is second-countable. Then there exists $\MB \subset \MT_d$ such that $\MB$ is a basis for $\MT_d$ and $\MB$ is countable. \rex{ex:topology:subspaces:0009} implies that $\MB \cap A$ is a basis for $\MT \cap A$. Since $\MT \cap A$ is countable, $(A, \MT \cap A)$ is second countable. 
	\end{proof}

	\begin{ex} \lex{ex:topology:countability:0013.1}
		Let $(X, \MT)$ be a topological space. Then $(X, \MT)$ is second-countable iff there exist $(U_n)_{n \in \N} \subset \MT$ such that 
		\begin{enumerate}
			\item $X = \bigcup\limits_{n \in \N}U_n$,
			\item for each $n \in \N$, $(U_n, \MT \cap U_n)$ is second-countable.
		\end{enumerate}
	\end{ex}
	
	\begin{proof}\
		\begin{itemize}
			\item $(\implies):$ \\
			Suppose that $(X, \MT)$ is second-countable. Define $(U_n)_{n \in \N} \subset \MT$ by
			\[
			U_n \defeq 
			\begin{cases}
				X, & n = 1 \\
				\varnothing, n > 1.
			\end{cases}
			\]
			Then
			\begin{enumerate}
				\item by construction, $X = \bigcup\limits_{n \in \N}U_n$,
				\item \rex{ex:topology:countability:0013} implies that for each $n \in \N$, $(U_n, \MT \cap U_n)$ is second-countable.
			\end{enumerate}
			\item $(\impliedby):$ \\
			Suppose that there exist $(U_n)_{n \in \N} \subset \MT$ such that 
			\begin{enumerate}
				\item $X = \bigcup\limits_{n \in \N}U_n$,
				\item for each $n \in \N$, $(U_n, \MT \cap U_n)$ is second-countable.
			\end{enumerate}
			Since for each $n \in \N$, $(U_n, \MT \cap U_n)$ is second-countable, we have that for each $n \in \N$, there exists $\MB_n \subset \MT\cap U_n$ such that $\MB_n$ is a basis for $\MT \cap U_n$ and $\MB_n$ is countable. Since $(U_n)_{n \in \N} \subset \MT$, we have that for each $n \in \N$,
			\begin{align*}
				\MB_n 
				& \subset \MT \cap U_n \\
				& \subset \MT 
			\end{align*}
			Define $\MB \subset \MT$ by $\MB = \bigcup\limits_{n \in \N} \MB_n$. Then $\MB$ is countable. 
			Let $V \in \MT$ and $n \in \N$. Then $V \cap U_n \in \MT \cap U_n$. Since $V \cap U_n \in \MT \cap U_n$, there exist $(B_{n, j})_{j \in \N} \subset \MB_n$ such that $V \cap U_n = \bigcup\limits_{j \in \N} B_{n,j}$.
			Then $(B_{n,j})_{n,j \in \N} \subset \MB$ and 
			\begin{align*}
				V
				& = V \cap X \\
				& = V \cap \bigg( \bigcup_{n \in \N} U_n \bigg) \\
				& = \bigcup_{n \in \N} V \cap U_n \\
				& = \bigcup_{n \in \N} \bigcup_{j \in \N} B_{n,j} 
			\end{align*}
			Since $V \in \MT$ is arbitrary, we have that for each $V \in \MT$, there exits $\MB' \subset \MB$ such that $V = \bigcup\limits_{B \in \MB'} B$. Hence $\MB$ is a basis for $\MT$. Since $\MB$ is countable, $(X, \MT)$ is second-countable.
		\end{itemize}
	\end{proof}
















	
	

	\subsubsection{Second-Countability and Product Spaces}
	
	\begin{ex} \lex{ex:topology:countability:0014}
		Let $(X_{\al}, \MT_{\al})_{\al \in A}$ be a collection of topological spaces. Set $X = \prod\limits_{\al \in A} X_{\al}$ and $\MT_X \defeq \bigotimes\limits_{\al \in A} \MT_{\al}$. Suppose that $A$ is countable. If for each $\al \in A$, $(X_{\al}, \MT_{\al})$ is second-countable, then $(X, \MT_X)$ is second-countable. 
	\end{ex}

	\begin{proof}
		Suppose that for each $\al \in A$, $(X_{\al}, \MT_{\al})$ is second-countable. Then for each  $\al \in A$, there exists $\MB_{\al} \subset \MT_{\al}$ such that $\MB_{\al}$ is a basis for $\MT_{\al}$ and $\MB_{\al}$ is countable. Set 
		\begin{align*}
			\MB = 
			& \bigg \{\prod_{\al \in A} U_{\al}: \text{there exists $J \subset A$ such that $ \# J < \infty$, } \\
			& \text{ for each $\al \in J$, $U_{\al} \in \MB_{\al}$ and for each $\al \in J^c$, $U_{\al} = X_{\al}$ } \bigg \}
		\end{align*}  
		\tcr{An exercise in the section on the product topology} implies that $\MB$ is a basis for $\MT_X$. Since $A$ is countable and for each $\al \in A$, $\MB_{\al}$ is countable, we have that $\MB$ is countable. Hence $\MT_X$ is second-countable. 
	\end{proof}




































	\subsubsection{Second-Countability and Coproduct Spaces}	
	
	\begin{ex} \lex{ex:topology:countability:0015}
			Let $(X_{\al}, \MT_{\al})_{\al \in A}$ be a collection of topological spaces. Set $X = \coprod\limits_{\al \in A} X_{\al}$ and $\MT \defeq \bigotimes\limits_{\al \in A} \MT_{\al}$. Suppose that $A$ is countable. If for each $\al \in A$, $(X_{\al}, \MT_{\al})$ is second-countable, then $(X, \MT)$ is second-countable. 
	\end{ex}

	\begin{proof}
		Suppose that for each $\al \in A$, $(X_{\al}, \MT_{\al})$ is second-countable. Then for each $\al \in A$, there exists $\MB_{\al} \subset \MT_{\al}$ such that $\MB_{\al}$ is a basis for $\MT_{\al}$ and $\MB_{\al}$ is countable. Define $\MB \defeq \bigcup\limits_{\al \in A} \MB_{\al}$. Since $A$ is countable and for each $\al \in A$, $\MB_{\al}$ is countable, we have that $\MB$ is countable. \rex{ex:topology:coproducts:0003.2} implies that $\MB$ is a basis for $\MT$. Hence $(X, \MT)$ is second-countable. 
	\end{proof}










































	\subsubsection{Second-Countability and Quotient Spaces}


























\subsubsection{Second-Countability and Projective Limits}

\tcr{make $\varprojlim\limits_{j \in J} X$ refer to set construction, always supress $\pi_j$ as in products, but $\pi_j$ always refer to $\prj|_X$, and define a projective limit of a projective system and show that $\varprojlim X_j$ is a projective limit in various categories when equipped with the corresponding structure}
	
\begin{ex} \lex{ex:topology:countability:0016}
	Let $(J, {\leq})$ be a directed set and $((X_j, \MT_j)_{j \in J}, (\pi_{i,j})_{(i,j) \in {\leq}})$ be a $\Top$-projective system. Set $(X, (\pi_j)_{j \in J}) \defeq \varprojlim\limits_{j \in J} ((X_j, \MT_j)_{j \in J}, (\pi_{i,j})_{(i,j) \in {\leq}})$. Suppose that $J$ is countable and for each $j \in J$, $(X_j, \MT_j)$ is second-countable. Then $(X, \MT)$ is second-countable.
\end{ex}

\begin{proof}
	Since $J$ is countable, \rex{ex:topology:countability:0014} implies that $(X, \MT)$ is second-countable. Since $X \subset \prod\limits_{j \in J} X_j$ and $\MT = \bigg( \bigotimes_{j \in J} \MT_j \bigg) \cap X$, \rex{ex:topology:countability:0013} implies that $(X, \MT)$ is second-countable. 
\end{proof}
	
	

	
	
	
	
	
	
	
	
	
	
	
	
	
	
	
	
	
	
	
	
	
	
	
	
	
	
	
	
	
	
	
	
	
	
	
	\newpage
	\section{Compactness}
	
	\subsection{Compactness and Subspaces}
	

	\begin{defn} \ld{def:topology:compactness:subspaces:00002}
		Let $(X, \MT)$ be a topological space. Then $(X, \MT)$ is said to be \tbf{compact} if for each $\MU \subset \MP(X)$, $\MU$ is an open cover of $X$ in $(X, \MT)$ implies that there exists $\MU_0 \subset \MU$ such that $\MU_0$ is an open cover of $X$ in $(X, \MT)$ and $\MU_0$ is finite.
	\end{defn}

	\begin{defn} \ld{def:topology:compactness:subspaces:00003}
		Let $(X, \MT)$ be topological space and $E \subset X$. Then $E$ is said to be compact in $(X, \MT)$ if $(E, \MT \cap E)$ is compact. 
	\end{defn}

	\begin{ex} \lex{ex:topology:compactness:subspaces:00004}
		Let $(X, \MT)$ be a topological space, $E \subset X$ and $A \subset E$. Then $A$ is compact in $(E, \MT \cap E)$ iff $A$ is compact in $(X, \MT)$.
	\end{ex}

	\begin{proof}
		We note that since $A \subset E$, $(\MT \cap E) \cap A = \MT \cap A$. Suppose that $A$ is compact in $(E, \MT \cap E)$. By definition, $(A, (\MT \cap E) \cap A)$ is compact. Since $(A, (\MT \cap E) \cap A) = (A, \MT \cap A)$, by definition, $A$ is compact in $(X, \MT)$.\\
		Coversely, suppose that $A$ is compact in $(X, \MT)$. By definition, $(A, \MT \cap A)$ is compact. Similarly, since $ (A, \MT \cap A) = (A, (\MT \cap E) \cap A)$, $A$ is compact in $(E, \MT \cap E)$.
	\end{proof}

	\begin{ex} \lex{ex:topology:compactness:subspaces:00005}
		Let $(X, \MT)$ be a topological space and $K \subset X$. Then $K$ is compact in $(X, \MT)$ iff for each $\MU \subset \MP(X)$, $\MU$ is an open cover of $K$ in $(X, \MT)$ implies that there exists $\MU_0 \subset \MU$ such that $\MU_0$ is an open cover of $K$ in $(X, \MT)$ and $\MU_0$ is finite.
	\end{ex}

	\begin{proof}\
		\begin{itemize}
			\item $(\implies)$: \\
				Suppose that $K$ is compact in $(X, \MT)$. Let $\MU \subset \MP(X)$. Suppose that $\MU$ is an open cover of $K$ in $(X, \MT)$. Then $\MU \subset \MT$ and $K \subset \bigcup\limits_{U \in \MU} U$. Therefore $\MU \cap K \subset \MT \cap K$ and  
			\begin{align*}
				K 
				& \subset \bigg[  \bigcup\limits_{U \in \MU} U \bigg] \cap K \\ 
				& = \bigcup\limits_{U \in \MU} U \cap K
			\end{align*}
			Hence $\MU \cap K$ is an open cover of $K$ in $(K, \MT \cap K)$. Since $K$ is compact in $(X, \MT)$, \rex{ex:topology:compactness:subspaces:00004} implies that $(K, \MT \cap K)$ is compact. By \rd{def:topology:compactness:subspaces:00002}, there exists $\MV_0 \subset \MU \cap K$ such that $\MV_0$ is an open cover of $K$ in $(K, \MT \cap K)$ and $\MV_0$ is finite. Since $\MV_0 \subset \MU \cap K$, for each $V \in \MV_0$, there exists $U_V \in \MU$ such that $V = U_V \cap K$. Set $\MU_0 = \{U_V: V \in \MV_0\}$. Then $\MU_0 \subset \MU$. 
			\begin{enumerate}
				\item Since $\MU \subset \MT$
				\begin{align*}
					\MU_0 
					& \subset \MU \\
					& \subset \MT
				\end{align*}
				\item Since $\MV_0$ is an open cover of $K$ in $(K, \MT \cap K)$, we have that 
				\begin{align*}
					K 
					& \subset \bigcup_{V \in \MV_0} V \\
					& = \bigcup_{U \in \MU_0} U \cap K \\
					& \subset \bigcup_{U \in \MU_0} U \\
				\end{align*}
			\end{enumerate} 
			By \rd{31029}, $\MU_0$ is an open cover of $K$ in $(X, \MT)$. By construction, $\MU_0$ is finite. 
			\item $(\impliedby)$: \\
			Suppose that for each $\MU \subset \MP(X)$, if $\MU$ is an open cover of $K$ in $(X, \MT)$, then there exists $\MU_0 \subset \MU$ such that $\MU_0$ is an open cover $K$ in $(X, \MT)$ and $\MU_0$ is finite. Let $\MV \subset \MP(K)$. Suppose that $\MV$ is an open cover of $K$ in $(K, \MT \cap K)$. Then $\MV \subset \MT \cap K$ and $K \subset \bigcup\limits_{V \in \MV} V$. By definition of $\MT \cap K$, for each $V \in \MV$, there exists $U \in \MT$ such that $V = U \cap K$. The axiom of choice implies that there exists $(U_V)_{V \in \MV} \subset \MT$ such that for each $V \in \MV$, $V = U_V \cap K$. Therefore
			\begin{align*}
				K 
				& \subset \bigcup_{V \in \MV} V \\
				& = \bigcup_{V \in \MV} U_V \cap K \\
				& \subset \bigcup_{V \in \MV} U_V \\
			\end{align*}
			Hence $(U_V)_{V \in \MV}$ is an open cover of $K$ in $(X, \MT)$. By assumption, there exists $\MV_0 \subset \MV$ such that $(U_V)_{V \in \MV_0}$ is an open cover of $K$ in $(X, \MT)$ and $\MV_0$ is finite. 
			\begin{enumerate}
				\item Since $\MV \subset \MT \cap K$, we have that 
				\begin{align*}
					\MV_0 
					& \subset V \\
					& \subset \MT \cap K
				\end{align*}
				\item Since $(U_V)_{V \in \MV_0}$ is an open cover of $K$ in $(X, \MT)$, we have that
				\begin{align*}
					K 
					& = K \cap K \\
					& \subset \bigg[ \bigcup_{V \in \MV_0} U_V \bigg] \cap K  \\
					& = \bigcup_{V \in \MV_0} U_V \cap K \\
					& = \bigcup_{V \in \MV_0} V \\
				\end{align*}
			\end{enumerate}
			Therefore $\MV_0$ is an open cover of $K$ in $(K, \MT \cap K)$. Since $\MV \subset \MP(K)$ such that $\MV$ is an open cover of $K$ in $(K, \MT \cap K)$ is arbitrary, we have that for each $\MV \subset \MP(K)$, if $\MV$ is an open cover of $K$ in $(K, \MT \cap K)$, then there exists $\MV_0 \subset \MV$ such that $\MV_0$ is an open cover of $K$ in $(K, \MT \cap K)$ and $\MV_0$ is finite. Hence $(K, \MT \cap K)$ is compact. By definition, $K$ is compact in $(X, \MT)$.
		\end{itemize}
	\end{proof}

	\begin{ex} \lex{ex:topology:compactness:subspaces:00005.1}
		Let $(X, \MT)$ be a topological space and $K, L \subset X$. If $K$ and $L$ are compact, then $K \cup L$ is compact. 
	\end{ex}
		
	\begin{proof}
		Suppose that $K$ and $L$ are compact. Let $\MU \subset \MP(X)$. Suppose that $\MU$ is an open cover of $K \cup L$ in $(X, \MT)$. Since $K,L \subset K \cup L$, $\MU$ is an open cover of $K$ and $L$. Since $K$, $L$ are compact, there exist $\MU_K, \MU_L \subset \MU$ such that $\MU_K$ is an open cover of $K$, $\MU_L$ is an open cover of $L$ and $\MU_K$, $\MU_L$ are finite. Define $\MU_0 \subset \MU$ by $\MU_0 \defeq \MU_K \cup \MU_L$. Then $\MU_0$ is an open cover of $K \cup L$ and $\MU_0$ is finite. Since $\MU \subset \MP(X)$ with $\MU$ an open cover of $K \cup L$ arbitrary, we have that for each $\MU \subset \MP(X)$, if $\MU$ is an open cover of $K \cup L$, there exists $\MU_0 \subset \MU$ such that $\MU_0$ is an open cover of $K \cup L$ and $\MU_0$ is finite. Thus $K \cup L$ is compact. 
  	\end{proof}
  
  	\begin{ex} \lex{ex:topology:compactness:subspaces:00005.2}
  		Let $(X, \MT)$ be a topological space, $U \in \MT$ and $x \in U$. Then $U \setminus \{x\}$ is not compact. 
  	\end{ex}
	
	\begin{ex} \lex{ex:topology:compactness:subspaces:00006}
		Let $(X, \MT)$ be a topological space and $K \subset X$. Suppose that $(X, \MT)$ is Hausdorff. If $K$ is compact in $X$, then $K$ is closed in $X$.
	\end{ex}

	\begin{proof}
		Suppose that $K$ is compact. Let $y \in K^c$. Since $(X, \MT)$ is Hausdorff, for each $x \in K$, there exists $U_x, V_x \in \MT$, such that $x \in U_x$, $y \in V_x$ and $U_x \cap V_x = \varnothing$. Thus $(U_x)_{x \in K}$ is an open cover of $K$ in $(X, \MT)$. Since $K$ is compact, there exist $x_1, \ldots, x_n \in K$ such that $(U_{x_j})_{j=1}^n$ is an open cover of $K$ in $(X, \MT)$. Set $V = \bigcap\limits_{j=1}^n V_{x_j}$. Then $V \in \MT$ and $y \in V$. Since for each $j \in \{1, \ldots, n\}$, $V \subset V_{x_j}$, we have that
		\begin{align*}
			V \cap K
			& \subset V \cap \bigg[ \bigcup_{j=1}^n U_{x_j} \bigg] \\
			& = \bigcup_{j=1}^n ( V \cap U_{x_j}) \\
			& \subset \bigcup_{j=1}^n (V_{x_j} \cap U_{x_j}) \\
			& = \bigcup_{j=1}^n \varnothing \\
			& = \varnothing
		\end{align*}
		Thus $V \subset K^c$. Since $y \in K^c$ is arbitrary, we have that for each $y \in K^c$, there exists $V \in \MT$ such that $y \in V$ and $V \subset K^c$. Hence $K^c$ is open. Thus $K$ is closed.
	\end{proof}
	
	\begin{ex} \lex{ex:topology:compactness:subspaces:00006.1}
		Let $(X, \MT)$ be a topological space and $E \subset X$. If $(X, \MT)$ is compact and $E$ is closed in $(X, \MT)$, then $E$ is compact in $(X, \MT)$.
	\end{ex}

	\begin{proof}
		Suppose that $(X, \MT)$ is compact and $E$ is closed in $(X, \MT)$. Let $\MU \subset \MP(X)$. Suppose that $\MU$ is an open cover of $E$ in $(X, \MT)$. Since $E$ is closed in $(X, \MT)$, $E^c \in \MT$. Set $\MU' = \MU \cup \{E^c\}$. Then $\MU' \subset \MT$ and $\MU'$ is an open cover of $X$ in $(X, \MT)$.  Since $(X, \MT)$ is compact, there exists $\MU'_0 \subset \MU'$ such that $\MU'_0$ is an open cover of $X$ in $(X, \MT)$ and $\MU'_0$ is finite. Set $\MU_0 = \MU'_0 \setminus \{E^c\}$. Then $\MU_0$ is an open cover for $E$ in $(X, \MT)$. Since $\MU$ such that $\MU$ is an open cover of $E$ in $(X, \MT)$ is arbitrary, \rex{ex:topology:compactness:subspaces:00005} implies that $E$ is compact in $(X, \MT)$.
		\tcb{GIVE MORE DETAILS}\\
		\tcb{FINISH!!!}
	\end{proof}
	
	
	
	
	\begin{defn} \ld{def:topology:compactness:subspaces:00007}
		Let $X$ be a topological space and $E \subset X$. Then $E$ is said to be \tbf{precompact} if $\cl E$ is compact.
	\end{defn}

	\begin{ex} \lex{ex:topology:compactness:subspaces:00009}
		Let $(X, \MT)$ be a topological space, $K \subset X$ and $x \in K^c$. Suppose that $(X, \MT)$ is Hausdorff. If $K$ is compact in $(X, \MT)$, then there exist $U, V \in \MT$ such that $U \cap V = \varnothing$, $x \in U$ and $K \subset V$.
	\end{ex}

	\begin{proof}
		Suppose that $K$ is compact in $(X, \MT)$. Since $(X, \MT)$ is Hausdorff, we have that for each $y \in K$, there exist $U_y, V_y \in \MT$ such that $U_y \cap V_y = \varnothing$, $x \in U_y$ and $y \in V_y$. Define $\MU, \MV \subset \MP(X)$ by $\MU \defeq \{U_y: y \in K\}$ and $\MV \defeq \{V_y: y \in K\}$. Then $\MV$ is an open cover of $K$ in $(X, \MT)$. Since $K$ is compact in $(X, \MT)$, there exists $\MV_0 \subset \MV$ such that $\MV_0$ is finite and $\MV_0$ is an open cover of $K$ in $(X, \MT)$. By definition of $\MV$, there exist $y_1, \ldots, y_n \in K$ such that $\MV_0 = \{V_{y_j}\}_{j=1}^n$. Define $\MU_0 \subset \MU$ by $\MU_0 = \{U_{y_j}\}_{j=1}^n$. Define $U,V \subset \MT$ by $U \defeq \bigcap\limits_{j=1}^n U_{y_j}$ and $V \defeq \bigcup\limits_{j=1}^n V_{y_j}$. Since $\MV_0$ is an open cover of $K$ in $(X, \MT)$, we have that $K \subset V$. Since for each $y \in K$, $x \in U_y$, we have that $x \in U$. Since for each $y \in K$, $U_y \cap V_y = \varnothing$, we have that 
		\begin{align*}
			U \cap V
			& = \bigg( \bigcap_{j=1}^n U_{y_j} \bigg) \cap  \bigg( \bigcup_{k=1}^n V_{y_k} \bigg) \\
			& = \bigcup_{k=1}^n \bigg[ \bigg(  \bigcap_{j=1}^n U_{y_j} \bigg) \cap V_{y_k}  \bigg] \\
			& \subset \bigcup_{k=1}^n U_{y_k} \cap V_{y_k} \\
			& = \bigcup_{k=1}^n \varnothing \\
			& = \varnothing
		\end{align*}
	\end{proof}
	
	\begin{ex} \lex{ex:topology:compactness:subspaces:00010}
		Let $(X, \MT)$ be a topological space and $U \in \MT$. Suppose that $(X, \MT)$ is Hausdorff. If $\cl U$ is compact, then for each $x \in U$, there exists $K \in \MN(x)$ such that $K \subset U$ and $K$ is compact.   
	\end{ex}

	\begin{proof}
		Suppose that $\cl U$ is compact. Since $U \in \MT$, $U \cap \p U = \varnothing$. Thus  
		\begin{align*}
			x 
			& \in U \\
			& \subset (\p U)^c 
		\end{align*}
		$x \not \in \p U$. Since $\cl U$ is compact, $\p U$ is closed and $\p U \subset \cl U$, we have that $\p U$ is compact. \tcb{The previous exercise} implies that there exist $V,W \in \MT$ such that $V \cap W = \varnothing$, $x \in V$ and $\p U \subset W$. Since $V \subset W^c$ and $W^c$ is closed, we have that $\cl V \subset W^c$. Since $\p U \subset W$, we have that $(\cl V) \cap \p U = \varnothing$. Hence $\cl V \subset U$. Set $K = \cl V$. Since $K$ is closed, $K \subset \cl U$ and $\cl U$ is compact, we have that $K$ is compact. By construction,  
		\begin{align*}
			x 
			& \in V \\
			& = \Int K 
		\end{align*}
		so $K \in \MN(x)$. 
	\end{proof}

	\begin{ex} \lex{ex:topology:compactness:subspaces:00011}
		Let $(X, \MT)$ be a topological space. If $(X, \MT)$ is compact and Hausdorff, then $(X, \MT)$ is normal.
	\end{ex}
	
	\begin{proof}
		Suppose that $(X, \MT)$ is compact and Hausdorff. Since $(X, \MT)$ is Hausdorff, $(X, \MT)$ is $\mathbf{T_1}$. Let $E, F \subset X$. Suppose that $E, F$ are closed and $E \cap F = \varnothing$. Since $(X, \MT)$ is compact and $E,F$ are closed, $E,F$ are compact. \tcb{A previous exercise} implies that for each $x \in E$, there exists $U_x, V_x \in \MT$ such that $U_x \cap V_x = \varnothing$, $x \in U_x$ and $F \subset V_x$. The axiom of choice implies that there exist $(U_x)_{x \in E}, (V_x)_{x \in E} \subset \MT$ such that for each $x \in X$, $U_x \cap V_x = \varnothing$, $x \in U_x$ and $F \subset V_x$. Then $(U_x)_{x \in E}$ is an open cover of $E$. Since $E$ is compact, there exist $x_1, \ldots, x_n \in E$ such that $(U_{x_j})_{j=1}^n$ is an open cover of $E$. Define $U, V \in \MT$ by $U \defeq \bigcup\limits_{j=1}^n U_{x_j}$ and $V \defeq \bigcap\limits_{k=1}^n V_{x_k}$. Then
		\begin{align*}
			U \cap V
			& = \bigg( \bigcup\limits_{j=1}^n U_{x_j} \bigg) \cap \bigg( \bigcap\limits_{k=1}^n V_{x_k} \bigg) \\
			& = \bigcup\limits_{j=1}^n \bigg[ U_{x_j} \cap \bigg( \bigcap\limits_{k=1}^n V_{x_k} \bigg) \bigg] \\
			& \subset \bigcup\limits_{j=1}^n ( U_{x_j} \cap V_{x_j} ) \\ 
			& = \bigcup\limits_{j=1}^n \varnothing \\
			& = \varnothing
		\end{align*}
		By construction,
		\begin{align*}
			E 
			& \subset \bigcup\limits_{j=1}^n U_{x_j} \\
			& = U
		\end{align*}
		and 
		\begin{align*}
			F 
			& \subset \bigcap\limits_{k=1}^n V_{x_j} \\
			& = V
		\end{align*}
		Since $E, F \subset X$ with $E,F$ closed and $E \cap F = \varnothing$ are arbitrary, we have that for each $E, F \subset X$, $E,F$ are closed and $E \cap F = \varnothing$ implies that there exist $U,V \in \MT$ such that $U \cap V = \varnothing$, $E \subset U$, $F \subset V$. So $(X, \MT)$ is normal.
	\end{proof}
		
	

	
	
	
	
	
	
	
	
	
	
	
	
	
	
	
	
		

	






















	\subsection{The Finite Intersection Property} 
	
	\begin{defn} \ld{def:topology:compactness:subspaces:00012}
		Let $(X, \MT)$ be a set and $\MA \subset \MP(X)$. Then $\MA$ is said to have the \tbf{finite intersection property} if for each $\MB \subset \MA$, $\MB$ is finite implies that $\bigcap\limits_{B \in \MB} B \neq \varnothing$. We define 
		$$\FIP(X, \MT) = \{\MA \subset \MP(X): \MA \text{ has the finite intersection property} \}$$ 
		and order $\FIP(X, \MT)$ by inclusion. 
	\end{defn}

	\begin{note}
		When the context is clear, we write $\FIP(X)$ in place of $\FIP(X, \MT)$.
	\end{note}

	\begin{ex} \lex{ex:topology:compactness:subspaces:00013}
		Let $X$ be a set. Then $\FIP(X)$ ordered by inclusion is a poset.
	\end{ex}

	\begin{proof}
		Clear.
	\end{proof}

	\begin{ex} \lex{ex:topology:compactness:subspaces:00014}
		Let $X$ be a set and $\MA_0 \in \FIP(X)$. Then there exists $\MA \in \FIP(X)$ such that $\MA$ is maximal in $[\MA_0, \infty)$. 
	\end{ex}

	\begin{proof}
		Let $\MC \subset [\MA_0, \infty)$. Suppose that $\MC$ is a chain. 
		\begin{itemize}
			\item Suppose that $\MC = \varnothing$. Set $S \defeq \MA_0$. Then $\MS \in [\MA_0, \infty)$ and it is vacuously true that for each $\ME \in \MC$, $\ME \subset \MS$. Since $\MC \subset [\MA_0, \infty)$ with $\MC$ a chain is arbitrary, we have that for each $\MC \subset [\MA_0, \infty)$, if $\MC$ is a chain, then there exists $\MS \in [\MA_0, \infty)$ such that $\MS$ is an upper bound for $\MC$.
			\item Suppose that $\MC \neq \varnothing$. Define $\MS \in \MP(X)$ by $\MS \defeq \bigcup\limits_{\ME \in \MC} \ME$. Since $\MC \neq \varnothing$, there exists $\ME_0 \in [\MA_0, \infty)$ such that $\ME_0 \in \MC$. Since $\MA_0 \subset \ME_0$, $\MA_0 \subset \MS$. Let $\MB \subset \MS$. Suppose that $\MB$ is finite. Since $\MB \subset \MS$ and $\MS = \bigcup\limits_{\ME \in \MC} \ME$, we have that for each $B \in \MB$, there exists $\ME_B \in \MC$ such that $B \in \ME_B$. Since $\MB$ is finite and $\MC$ is totally ordered, there exists $B_0 \in \MB$ such that $\ME_{B_0} = \max\limits_{B \in \MB} \ME_B$. Therefore for each $B \in \MB$, 
			\begin{align*}
				B 
				& \in \ME_B \\
				& \subset \ME_{B_0}  
			\end{align*}
			Hence $\MB \subset \ME_{B_0}$ which implies that 
			$$\bigcap\limits_{B \in \ME_{B_0}} B \subset \bigcap\limits_{B \in \MB} B. $$ 
			Since $\ME_{B_0} \in \MC$, and $\MC \subset \FIP(X)$, we have that $\ME_{B_0} \in \FIP(X)$. Since $\ME_{B_0} \in \FIP(X)$, $\MB \subset \ME_{B_0}$ and $\MB$ is finite, we have that $\bigcap\limits_{B \in \ME_{B_0}} B \neq \varnothing$. Thus there exists $x \in X$ such that $x \in \bigcap\limits_{B \in \ME_{B_0}} B$. Since $\bigcap\limits_{B \in \ME_{B_0}} B \subset \bigcap\limits_{B \in \MB} B$, we have that $x \in \bigcap\limits_{B \in \MB} B$. Hence $ \bigcap\limits_{B \in \MB} B \neq \varnothing$. Since $\MB \subset \MS$ with $\MB$ finite is arbitrary, we have that for each $\MB \subset \MS$, $\MB$ is finite implies that $ \bigcap\limits_{B \in \MB} B \neq \varnothing$. Thus $\MS \in \FIP(X)$. Since $\MA_0 \subset \MS$ and $\MS \in \FIP(X)$, we have that $\MS \in [\MA_0, \infty)$. By construction, for each $\ME \in \MC$, $\ME \subset \MS$ so that $\MS$ is an upper bound for $\MC$. Since $\MC \subset [\MA_0, \infty)$ with $\MC$ a chain is arbitrary, we have that for each $\MC \subset [\MA_0, \infty)$, if $\MC$ is a chain, then there exists $\MS \in [\MA_0, \infty)$ such that $\MS$ is an upper bound for $\MC$. Zorn's lemma implies that there exists $\MA \in [\MA_0, \infty]$ such that $\MA$ is maximal. 
		\end{itemize}
	\end{proof}

	\begin{ex} \lex{ex:topology:compactness:subspaces:00015}
		Let $X$ be a set and $\MA \in \FIP(X)$. Suppose that $\MA$ is maximal. Then 
		\begin{enumerate}
			\item for each $\MB \subset \MA$, if $\MB$ is finite, then $\bigcap\limits_{B \in \MB} B \in \MA$,
			\item for each $B \subset X$, if for each $A \in \MA$, $B \cap A \neq \varnothing$, then $B \in \MA$. \\
			\tbf{Hint:} use part $(1)$
		\end{enumerate}
	\end{ex}

	\begin{proof}\
		\begin{enumerate}
			\item Let $\MB \subset \MA$. Suppose that $\MB$ is finite. Set $B_0 \defeq \bigcap\limits_{B \in \MB} B$ and $\MA_0 = \MA \cup \{B_0\}$. Let $\MC \subset \MA_0$. Suppose that $\MC$ is finite. 
			\begin{itemize}
				\item Suppose that $B_0 \not \in \MC$. Since $\MA \in \FIP(X)$, $\MC \subset \MA$, $\MC$ is finite,  
				\begin{align*}
					\bigcap_{C \in \MC} C 
					& = \bigcap_{C \in \MC \cap \MA} C \\
					& \neq \varnothing
				\end{align*} 
				\item Now suppose that $B_0 \in \MC$. Since $\MC$ is finite and $\MB$ is finite, we have that $(\MC \cap \MA) \cup \MB$ is finite. Since $\MA \in \FIP(X)$ and $(\MC \cap \MA) \cup \MB \subset \MA$,  
				\begin{align*}
					\bigcap_{C \in \MC} C 
					& = \bigg( \bigcap_{C \in \MC \cap \MA} C \bigg) \cap B_0  \\
					& = \bigg( \bigcap_{C \in \MC \cap \MA} C \bigg) \cap \bigg( \bigcap\limits_{B \in \MB} B \bigg)  \\
					& = \bigcap_{C \in (\MC \cap \MA) \cup \MB} C \\
					& \neq \varnothing
				\end{align*} 
			\end{itemize}
			Therefore $\bigcap_{C \in \MC} C \neq \varnothing$. Since $\MC \subset \MA_0$ with $\MC$ finite is arbitrary, we have that for each $\MC \subset \MA_0$, $\MC$ is finite implies that $\bigcap_{C \in \MC} C \neq \varnothing$. Hence $\MA_0 \in \FIP(X)$. Since $\MA \in \FIP(X)$ is maximal and $\MA \subset \MA_0$, we have that $\MA = \MA_0$ and therefore $B_0 \in \MA$. 
			\item Let $B \subset X$. Suppose that for each $A \in \MA$, $B \cap A \neq \varnothing$. Define $\MA_0 = \MA \cup \{B\}$. Let $\MC \subset \MA_0$. Suppose that $\MC$ is finite. 
			\begin{itemize}
				\item If $B \not \in \MC$, then 
				\begin{align*}
					\bigcap_{C \in \MC} C
					& = \bigcap_{C \in \MC \cap \MA} C \\
					& \neq \varnothing
				\end{align*} 
				\item Suppose that $B \in \MC$. Since $\MC \cap \MA$ is finite, part $(1)$ implies that $\bigcap\limits_{C \in \MC \cap \MA} C \in \MA$. Then by assumption, $\bigg( \bigcap\limits_{C \in \MC \cap \MA} C \bigg) \cap B \neq \varnothing$.  Therefore 
				\begin{align*}
					\bigcap_{C \in \MC} C
					& = \bigcap_{C \in (\MC \cap \MA) \cup \{B\}} C \\
					& = \bigg( \bigcap_{C \in \MC \cap \MA} C \bigg) \cap B \\
					& \neq \varnothing
				\end{align*}  
			\end{itemize}
			Therefore $\bigcap_{C \in \MC} C \neq \varnothing$. Since $\MC \subset \MA_0$ with $\MC$ finite is arbitrary, we have that for each $\MC \subset \MA_0$, $\MC$ is finite implies that $\bigcap_{C \in \MC} C \neq \varnothing$. Hence $\MA_0 \in \FIP(X)$. Since $\MA \in \FIP(X)$ is maximal and $\MA \subset \MA_0$, we have that $\MA = \MA_0$ and therefore $B \in \MA$. 
		\end{enumerate}
	\end{proof}
	
	\begin{note}
		Recall the definition of $\MC_A(X, \MT)$ in \rd{31013}.
	\end{note}
	
	\begin{ex} \lex{ex:topology:compactness:subspaces:00016}
		Let $(X, \MT)$ be a topological space. Then $(X, \MT)$ is compact iff for each $\MC \subset \MC_\varnothing(X, \MT)$, $\MC \in \FIP(X, \MT)$ implies that $\bigcap\limits_{C \in \MC} C \neq \varnothing$. \\
		\tbf{Hint:} consider $\{C^c: C \in \MC\}$ and whether it is an open cover
	\end{ex}

	\begin{proof}\
		\begin{itemize}
			\item $(\implies)$: \\
			Suppose that $(X, \MT)$ is compact. Let $\MC \subset \MC_{\varnothing}$. Suppose that $\MC \in \FIP(X, \MT)$. For the sake of contradiction, suppose that $\bigcap\limits_{C \in \MC} C = \varnothing$. Define $\MU \subset \MT$ by $\MU \defeq \{C^c: C \in \MC\}$. Then 
			\begin{align*}
				\bigcup_{U \in \MU} U
				& = \bigcup_{C \in \MC} C^c \\
				& = \bigg( \bigcap_{C \in \MC} C \bigg)^c \\
				& = \varnothing^c \\
				& = X
			\end{align*}
			Thus $\MU$ is an open cover of $X$. Since $(X, \MT)$ is compact, there exists $\MU' \subset \MU$ such that $\MU'$ is finite and $\MU'$ is an open cover of $X$. Define $\MC' \subset \MC$ by $\MC' \defeq \{U^c: U \in \MU'\}$. Then 
			\begin{align*}
				\bigcap_{C \in \MC'} C 
				& = \bigcap_{U \in U'} U^c \\
				& = \bigg( \bigcup_{U \in U'} U \bigg)^c \\
				& = X^c \\
				& = \varnothing
			\end{align*}  
			However, since $\MC \in \FIP(X, \MT)$, $\MC' \subset \MC$ and $\MC'$ is finite, we have that $\bigcap\limits_{C \in \MC'} C \neq \varnothing$. This is a contradiction. Hence $\bigcap\limits_{C \in \MC} C \neq \varnothing$. Since $\MC \subset \MC_{\varnothing}$ such that $\MC \in \FIP(X, \MT)$ is arbitrary, we have have that for each $\MC \subset \MC_\varnothing$, $\MC \in \FIP(X, \MT)$ implies that $\bigcap\limits_{C \in \MC} C \neq \varnothing$. 
			\item $(\impliedby)$: \\
			Suppose that for each $\MC \subset \MC_\varnothing$, $\MC \in \FIP(X, \MT)$ implies that $\bigcap\limits_{C \in \MC} C \neq \varnothing$. Let $\MU \subset \MP(X)$. Suppose that $\MU$ is an open cover of $X$ in $(X, \MT)$. Then $\MU \subset \MT$ and $X = \bigcup\limits_{U \in \MU} U$. Define $\MC \subset \MC_{\varnothing}$ by $\MC = \{U^c: U \in \MU\}$. Then 
			\begin{align*}
				\bigcap_{C \in \MC} C
				& = \bigcap_{U \in \MU} U^c \\
				& = \bigg( \bigcup_{U \in \MU} U \bigg)^c \\
				& = X^c \\
				& = \varnothing
			\end{align*}  
			By assumption, $\MC \not \in \FIP(X, \MT)$. Thus there exists $\MC' \subset \MC$ such that $\MC'$ is finite and $\bigcap\limits_{C \in \MC'} C = \varnothing$. Define $\MU' \subset \MU$ by $\MU' \defeq \{C^c: C \in \MC' \}$. Then $\MU'$ is finite and 
			\begin{align*}
				X
				& = \varnothing^c \\
				& = \bigg( \bigcap_{C \in \MC'} C \bigg)^c \\
				& = \bigcup_{C \in \MC'} C^c \\
				& = \bigcup_{U \in \MU'} U \\
			\end{align*}
			Hence $\MU'$ is an open cover of $X$ in $(X, \MT)$. Since $\MU \subset \MP(X)$ such that $\MU$ is an open cover of $X$ in $(X, \MT)$, we have that for each $\MU \subset \MP(X)$, $\MU$ is an open cover of $X$ in $(X, \MT)$ implies that there exists $\MU' \subset \MU$ such that $\MU'$ is finite and $\MU'$ is an open cover of $X$ in $(X, \MT)$. Thus $(X, \MT)$ is compact. 
			\end{itemize}
	\end{proof}

	\begin{ex} \lex{ex:topology:compactness:subspaces:00017}
		Let $(X, \MT)$ be a topological space. Then the following are equivalent:
		\begin{enumerate}
			\item $(X, \MT)$ is compact
			\item for each net $(x_{\al})_{\al \in A} \subset X$, there exists $x \in X$ such that $x$ is a cluster point of $(x_{\al})_{\al \in A}$.
			\item for each net $(x_{\al})_{\al \in A} \subset X$, there exists a subnet $(x_{\al_{\be}})_{\be \in B}$ of $(x_{\al})_{\al \in A}$ and $x \in X$ such that $x_{\al_{\be}} \rightarrow x$. \\
		\end{enumerate} 
		\tbf{Hint:} 
		\begin{itemize}
			\item $(1) \implies (2)$: \\
			For $\al \in A$, set $E_{\al} \defeq \{x_{\al'}: \al' \geq \al \}$. Then $\{\cl E_{\al}: \al \in A\} \in \FIP(X, \MT)$.  
			\item $(3) \implies (1)$: \\
			If $(X, \MT)$ is not compact, choose open cover $\MU$ of $X$ such that for each $\MU_0 \subset \MU$, $\MU_0$ is finite implies that $\MU_0$ is not an open cover of $X$. Consider $\MF_{\MU} = \{\MU' \subset \MU: \MU' \text{ is finite}\}$ ordered by inclusion. Then there exists a net $(x_{\MU'})_{\MU' \in \MF_{\MU}} \subset X$ such that for each $x_{\MU'} \not \in  \bigcup\limits_{U \in \MU'} U$. 
		\end{itemize}
	\end{ex}
	
	\begin{proof}\
		\begin{itemize}
			\item $(1) \implies (2)$: \\
			Suppose that $(X, \MT)$ is compact. Let $(x_{\al})_{\al \in A} \subset X$ be a net. For $\al_0 \in A$, define $E_{\al} = \{x_{\al'} : \al' \geq \al\}$. Then for each $\al_1, \al_2 \in A$, $\al_1 \leq \al_2$ implies that $E_{\al_2} \subset E_{\al_1}$. Since $A$ is directed, for each $\al \in A$, $E_{\al} \neq \varnothing$ and for each $A_0 \subset A$, $A_0$ is finite implies that there exists $\al_0 \in A$ such that for each $\al \in A_0$, $\al_0 \geq \al$. 
			
			Define $\ME \subset \MP(X)$ by $\ME \defeq \{\cl E_{\al}: \al \in A\}$. Let $A_0 \subset A$. Suppose that $A_0$ is finite. Then there exists $\al_0 \in A$ such that for each $\al \in A_0$, $\al_0 \geq \al$. Then for each $\al \in A$, $E_{\al_0} \subset E_{\al}$. Thus 
			\begin{align*}
				\varnothing 
				& \neq E_{\al_0} \\
				& \subset \bigcap_{\al \in A_0} E_{\al} \\
				& \subset \bigcap_{\al \in A_0} \cl E_{\al}
			\end{align*}
			Since $A_0 \subset A$ with $A_0$ finite is arbitrary, we have that for each $A_0 \subset A$, $A_0$ is finite implies that $\bigcap_{\al \in A_0} \cl E_{\al} \neq \varnothing$. Thus $\ME \in \FIP(X, \MT)$. Since $(X, \MT)$ is compact, \tcb{the previous exercise} implies that $\bigcap\limits_{\al \in A} \cl E_{\al} \neq \varnothing$. Thus there exists $x \in X$ such that $x \in \bigcap\limits_{\al \in A} \cl E_{\al}$. \rex{ex:nets:0026} implies that $x$ is a cluster point of $(x_{\al})_{\al \in A}$.
			\item $(2) \implies (3)$: \\
			Immediate by \rex{ex:nets:0026}.
			\item $(3) \implies (1)$: \\
			Suppose that $(X, \MT)$ is not compact. Then there exists $\MU \subset \MP(X)$ such that $\MU$ is an open cover of $X$ in $(X, \MT)$ and for each $\MU_0 \subset \MU$, $\MU_0$ is finite implies that $\MU_0$ is not an open cover of $X$ in $(X, \MT)$. Define $\MF_{\MU} \subset \MP(X)$ by $\MF_{\MU} \defeq \{\MU_0 \subset \MU: \MU_0 \text{ is finite}\}$. We define $\leq \subset \MF_{\MU} \times \MF_{\MU}$ by inclusion so that $\MU_1 \leq \MU_2$ iff $\MU_1 \subset \MU_2$. Then $(\MF_U, \leq)$ is a directed set. By construction for each $\MU' \in \MF_{\MU}$, $\bigg( \bigcup\limits_{U \in \MU'} U \bigg)^c \neq \varnothing$. The axiom of choice implies that there exists a net $(x_{\MU'})_{\MU' \in \MF_{\MU}} \subset X$ such that for each $\MU' \in \MF_{\MU}$, $x_{\MU'} \in \bigg( \bigcup\limits_{U \in \MU'} U \bigg)^c$. 
			
			For the sake of contradiction suppose that there exists a subnet $(x_{\MU'_{\be}})_{\be \in B}$ of $(x_{\MU'})_{\MU' \in \MF_{\MU}}$ and $x \in X$ such that $x_{\MU'_{\be}} \rightarrow x$. Since $\MU$ is an open cover of $X$ in $(X, \MT)$, there exists $U_0 \in \MU$ such that $x \in U_0$. Since $x_{\MU'_{\be}} \rightarrow x$ and $U_0 \in \MN(x)$, there exists $\be_0 \in B$ such that for each $\be \in B$, $\be \geq \be_0$ implies that $x_{\MU'_{\be}} \in U_0$. Define $\MU_0 \in \MF_{U}$ by $\MU_0 \defeq \{U_0\}$. Since $(x_{\MU'_{\be}})_{\be \in B}$ is a subnet of $(x_{\MU'})_{\MU' \in \MF_{\MU}}$, there exists $\be_1 \in B$ such that for each $\be \in B$, $\be \geq \be_1$ implies that $\MU'_{\be} \geq \MU_0$. Since $B$ is a directed set, there exists $\be_2 \in B$ such that $\be_2 \geq \be_0, \be_1$. 
			
			Since $\be_2 \geq \be_0$, we have that $x_{\MU'_{\be_2}} \in U_0$. Since $\be_2 \geq \be_1$, we have that $\MU'_{\be_2} \geq \MU_0$. Hence $\MU_0 \subset \MU'_{\be_2}$ and therefore 
			\begin{align*}
				U_0 
				& = \bigcup\limits_{U \in \MU_0} U \\
				& \subset \bigcup\limits_{U \in \MU'_{\be_2}} U
			\end{align*}
			By construction,
			\begin{align*}
				x_{\MU'_{\be_2}} 
				& \in \bigg( \bigcup_{U \in \MU'_{\be_2}} U \bigg)^c \\
				& \subset \bigg( \bigcup_{U \in \MU_0} U \bigg)^c \\
				& = U_0^c 
			\end{align*}
			This is a contradiction. Thus for each subnet $(x_{\MU'_{\be}})_{\be \in B}$ of $(x_{\MU'})_{\MU' \in \MF_{\MU}}$ and $x \in X$, we have that $x_{\MU'_{\be}} \not \rightarrow x$. Therefore there exists a net $(x_{\al})_{\al \in A} \subset X$ such that for each subnet $(x_{\al_{\be}})_{\be \in B}$ of $(x_{\al})_{\al \in A}$ and $x \in X$, $x_{\al_{\be}} \not \rightarrow x$. By contrapositive, we have that $(3) \implies (1)$.
		\end{itemize}
	\end{proof}
	
	
 
 
 
 
 
 
 
 
 
 
 
 
 
 
 
 
 
 
 
 
 
 
 
 
 
 
 
 
 
 
 
 
 
 
 
 
 
 
 
 
 
 
 
 
 
 
 
 
 
 
 
 
 
 
 
 
 
 
 
 
 
 
 
 
 
 
 
 
 
 
 
 
 
 
 
 
 
 
 

 
 
 
 
 
 
 
 
 
 
 
 
 
 
 
 
 
 
 
 
 
 
 
 
 
 
 
 
 
 
 
 
 
 
 
 
 
 
 
 
 
 
 
 
 
 
 
 
 
 
 
 
 
 
 
 
 
 
 
 
 
 
 
 \subsection{Compactness and Product Spaces}
 
 \begin{ex} \lex{ex:topology:compactness_prod_spaces:00018} \tbf{Tychonoff's Theorem:} \\ 
 	Let $(X_{\al}, \MT_{\al})_{\al \in A}$ be a collection of topological spaces. Suppose that for each $\al \in A$, $(X_{\al} , \MT_{\al})$ is compact. Then $\bigg( \prod\limits_{\al \in A} X_{\al}, \bigotimes\limits_{\al \in A} \MT_{\al} \bigg)$ is compact. \\
 	\tbf{Hint:} 
 \end{ex}
 
 \begin{proof}
 	Set $X \defeq \prod\limits_{\al \in A} X_{\al}$ and $\MT \defeq \bigotimes\limits_{\al \in A} \MT_{\al}$. Let $\MC \subset \MC_{\varnothing}$. Suppose that $\MC \in \FIP(X, \MT)$. \tcb{A previous exercise} implies that there exists $\MD \in \FIP(X, \MT)$ such that $\MD$ is maximal in $[\MC, \infty)$. Let $\al \in A$. Set $\MD_{\al} \defeq \{\pi_{\al}(D):D \in \MD\}$. 
 	
 	Let $\MD_{\al, 0} \subset \MD_{\al}$. Suppose that $\MD_{\al,_0}$ is finite. Then there exist $\MD_0 \subset \MD$ such that $\MD_0$ is finite and $\MD_{\al, 0} = \{\pi_{\al}(D): D \in \MD_0\}$. Since $\MD \in \FIP(X, \MT)$, $\bigcap\limits_{D \in \MD_0} D \neq \varnothing$. Therefore 
 	\begin{align*}
 		\varnothing
 		& \neq \pi_{\al} \bigg( \bigcap_{D \in \MD_0} D \bigg) \\
 		& \subset \bigcap_{D \in \MD_0} \pi_{\al}(D) \\
 		& = \bigcap_{D \in \MD_{\al, 0}} D 
 	\end{align*}
 	Since $\MD_{\al, 0} \subset \MD_{\al}$ with $\MD_{\al, 0}$ finite is arbitrary, we have that for each $\MD_{\al, 0} \subset \MD_{\al}$, $\MD_{\al,0}$ is finite implies that $\bigcap\limits_{D \in \MD_{\al, 0}} D \neq \varnothing$. Hence $\MD_{\al} \in \FIP(X, \MT)$. 
 	
 	For the sake of contradiction, suppose that $\bigcap\limits_{D \in \MD} \cl \pi_{\al}(D) = \varnothing$. Then $X = \bigcup\limits_{D \in \MD} [ \cl \pi_{\al}(D) ]^c $. Since $\{[\cl \pi_{\al}(D)]^c:D \in \MD \} \subset \MT_{\al}$ and $(X_{\al}, \MT_{\al})$ is compact, there exists $\MD_0 \subset \MD$ such that $\MD_0$ is finite and $X = \bigcup\limits_{D \in \MD_0} [ \cl \pi_{\al}(D) ]^c$. Therefore $\bigcap\limits_{D \in \MD_0} \cl \pi_{\al}(D) = \varnothing$. Since $\MD_{\al} \in \FIP(X, \MT)$ and $\{ \cl \pi_{\al}(D): D \in \MD_0 \} \subset \MD_{\al}$ is finite, we have that $\bigcap\limits_{D \in \MD_0} \cl \pi_{\al}(D) \neq \varnothing$. This is a contradiction. Hence $\bigcap\limits_{D \in \MD} \cl \pi_{\al}(D) \neq \varnothing$. Since $\al \in A$ is arbitrary, we have that for each $\al \in A$, $\bigcap\limits_{D \in \MD} \cl \pi_{\al}(D) \neq \varnothing$.
 	
 	The axiom of choice implies that there exists $x \in X$ such that for each $\al \in A$, $x_{\al} \in \bigcap\limits_{D \in \MD} \cl \pi_{\al}(D)$. Set 
 	$$\ME_x \defeq \{\pi_{\al}^{-1}(E_{\al}): \al \in A, E_{\al} \in \MT_{\al}, x_{\al} \in E_{\al}\}$$ 
 	Let $\al \in A$ and $E_{\al} \in \MT_{\al}$. Suppose that $x \in \pi_{\al}^{-1}(E_{\al})$. Then $x_{\al} \in E_{\al}$. Let $D \in \MD$. Since $x_{\al} \in  E_{\al} \cap \cl \pi_{\al}(D)$, $ E_{\al} \cap \cl \pi_{\al}(D) \neq \varnothing$. \rex{ex:nets:0018} implies that $ E_{\al} \cap \pi_{\al}(D) \neq \varnothing$. Therefore $\pi_{\al}^{-1}(E_{\al}) \cap D \neq \varnothing$. Since $D \in \MD$ is arbitrary, we have that for each $D \in \MD$, $\pi_{\al}^{-1}(E_{\al}) \cap D \neq \varnothing$. Since $\MD \in \FIP(X, \MT)$, \tcb{a previous exercise} implies that $\pi_{\al}^{-1}(E_{\al}) \in \MD$. Since $\al \in A$ and $E_{\al} \in \MT_{\al}$ are arbitrary, we have that $\ME_x \subset \MD$. Set 
 	$$\MB_x \defeq \bigg \{\bigcap_{j=1}^n V_j:(V_j)_{j=1}^n \subset \ME_x \bigg \}$$ 
 	Then \tcb{an exercise in the section on the product topology} implies that $\MB_x \subset \MT_X $ and $\MB_x$ is a local basis for $\MT$ at $x$. Since $\MD \in \FIP(X, \MT)$, \tcb{a previous exercise} implies that $\MB_x \subset \MD$. 
 	
 	Let $D \in \MD$ and $E \in B_x$. Since $\MD \in \FIP(X, \MT)$ and $D, E \in \MD$, we have that $D \cap E \neq \varnothing$. Since $E \in \MB_x$ is arbitary we have that for each $E \in \MB_x$, $D \cap E \neq \varnothing$. \tcb{An exercise in the introduction to topology section} implies that $x \in \cl D$. Since $D \in \MD$ is arbitrary, we have that for each $D \in \MD$, $x \in \cl D$. Thus $x \in \bigcap\limits_{D \in \MD} \cl D $ and therefore $\bigcap\limits_{D \in \MD} \cl D \neq \varnothing$. Since $\MC \subset \MC_{\varnothing}$, for each $C \in \MC$, $C = \cl C$. By construction, $\MC \subset \MD$, which implies that 
 	\begin{align*}
 		\varnothing
 		& \neq \bigcap_{D \in \MD} \cl D \\
 		& \subset \bigcap_{C \in \MC} \cl C \\
 		& = \bigcap_{C \in \MC} C 
 	\end{align*}
 	Since $\MC \subset \MC_{\varnothing}$ with $\MC \in \FIP(X, \MT)$ is arbitrary, we have that for each $\MC \subset \MC_{\varnothing}$, $\MC \in \FIP(X, \MT)$ implies that $\bigcap_{C \in \MC} C \neq \varnothing$. \tcb{The previous exercise} implies that $(X, \MT)$ is compact.
 \end{proof}
 
 \begin{ex} \lex{ex:topology:compactness_prod_spaces:00019}
 	Let $(X_{\al}, \MT_{\al})_{\al \in A}$ be a collection of topological spaces. Set $X \defeq \prod\limits_{\al \in A} X_{\al}$ and $\MT_X \defeq \bigotimes\limits_{\al \in A} \MT_{\al}$. Let $\al_0 \in A$. Set $A' \defeq A \setminus \{\al_0\}$,  $Y \defeq \prod\limits_{\al \in A'}X_{\al}$ and $\MT_Y \defeq \bigotimes\limits_{\al \in A'} \MT_{\al}$. Let $E \subset X$ and $a \in \pi_{\al_0}(E)^c$. Suppose that for each $\al \in A$, $(X_{\al} , \MT_{\al})$ is compact and $E$ is closed in $(X, \MT_X)$.
 	\begin{enumerate}
 		\item For each $y \in Y$, there exists $x^y \in X$ such that $\pi_{\al_0}(x^y) = a$, for each $\al \in A'$, $\pi_{\al}(x^y) = y_{\al}$ and $x^y \in E^c$.
 		\item For each $y \in Y$, there exists $U^y_{\al_0} \in \MT_{\al_0}$, $V^y \in \MT_Y$ and $W^y \in \MT_X$ such that $U^y_{\al_0} = \pi_{\al_0}(W^y)$, $V^y = \prod\limits_{\al \in A'} \pi_{\al}(W^y)$, $a \in U^y_{\al_0}$, $y \in V^y$ and $x^y \in W^y \subset E^c$. \\
 		\tbf{Hint}: $E$ is closed
 		\item There exist $U_{\al_0} \in \MT_{\al_0}$ such that 
 		\begin{enumerate}
 			\item $a \in U_{\al_0}$, \\
 			\tbf{Hint:} consider the intersection of a finite open cover of $Y$.
 			\item $U_{\al_0} \subset \pi_{\al_0}(E)^c$ \\
 			\tbf{Hint:} proof by contradiction
 		\end{enumerate}
 	\end{enumerate}
 \end{ex}
 
 \begin{proof}\
 	\begin{enumerate}
 		\item Let $y \in Y$. Define $x^y \in X$ by $x^y_{\al} \defeq 
 		\begin{cases}
 			a, & \al = \al_0 \\
 			y_{\al}, & \al \neq \al_0. 
 		\end{cases}
 		$
 		For the sake of contradiction, suppose that $x^y \in E$. Then 
 		\begin{align*}
 			a
 			& = x^y_{\al_0} \\
 			& = \pi_{\al_0}(x^y) \\
 			& \in \pi_{\al_0}(E).
 		\end{align*}
 		This is a contradiction since $a \in (\pi_{\al_0}(E))^c$. Hence $x^y \in E^c$. 
 		\item Since $E$ is closed in $(X, \MT_X)$, $E^c \in \MT_X$. \rex{ex:product_topology:0002} then implies that there exist $(U^y_{\al})_{\al \in A} \in \prod\limits_{\al \in A} \MT_{\al}$ such that $\prod\limits_{\al \in A} U^y_{\al} \in \MT_X$, $x^y \in \prod\limits_{\al \in A} U^y_{\al} \subset E^c$ and $\# \{\al \in A: U^y_{\al} \neq X_{\al}\} < \infty$. Set $W^y \defeq \prod\limits_{\al \in A} U^y_{\al}$ and $V^y \defeq \prod\limits_{\al \in A'}U^y_{\al}$. Since $\# \{\al \in A: U^y_{\al} \neq X_{\al}\} < \infty$, we have that $\# \{\al \in A': U^y_{\al} \neq X_{\al}\} < \infty$ and therefore $V^y \in \MT_Y$. Then 
 		\begin{align*}
 			a
 			& = x^y_{\al_0} \\
 			& \in U^y_{\al_0} 
 		\end{align*}
 		and $y \in V^y$. By construction, $U^y_{\al_0} = \pi_{\al_0}(W^y)$ and $V^y = \prod\limits_{\al \in A'} \pi_{\al}(W^y)$. Since $y \in Y$ is arbitrary, we have that for each $y \in Y$, there exists $U^y_{\al_0} \in \MT_{\al_0}$, $V^y \in \MT_Y$ and $W^y \in \MT_X$ such that $U^y_{\al_0} = \pi_{\al_0}(W^y)$, $V^y = \prod\limits_{\al \in A'} \pi_{\al}(W^y)$, $a \in U^y_{\al_0}$, $y \in V^y$ and $x^y \in W^y \subset E^c$.
 		\item 
 		\begin{enumerate}
 			\item \rex{ex:topology:compactness_prod_spaces:00018} implies that $(Y, \MT_Y)$ is compact. Since $Y \subset \bigcup\limits_{y \in Y} V^y$ and $(Y, \MT_Y)$ is compact, there exists $y_1, \ldots, y_n \in Y$ such that $Y \subset \bigcup\limits_{j=1}^n V^{y_j}$. Set $U_{\al_0} \defeq \bigcap\limits_{j=1}^n U^{y_j}_{\al_0}$. Then $U_{\al_0} \in \MT_{\al_0}$ and $a \in U_{\al_0}$. 
 			\item For the sake of contradiction, suppose that $U_{\al_0} \not \subset \pi_{\al_0}(E)^c$. Then there exists $b \in U_{\al_0}$ such that $b \in \pi_{\al_0}(E)$.  Since $b \in \pi_{\al_0}(E)$, there exists $z \in E$ such that $\pi_{\al_0}(z) = b$. Since $Y \subset \bigcup\limits_{j=1}^n V^{y_j}$, there exists $j_0 \in [n]$ such that $(z_{\al})_{\al \in A'} \in V^{y_{j_0}}$. Since 
 			\begin{align*}
 				z_{\al_0}
 				& = b \\
 				& \in U_{\al_0} \\
 				& = \bigcap\limits_{j=1}^n U_{\al_0}^{y_j} \\
 				& \subset U_{\al_0}^{y_{j_0}},
 			\end{align*}
 			\begin{align*}
 				(z_{\al})_{\al \in A'}
 				& \in V^{y_{j_0}} \\
 				& = \prod\limits_{\al \in A'} U_{\al}^{y_{j_0}}
 			\end{align*}
 			and $W^{y_{j_0}} = \prod\limits_{\al \in A}U_{\al}^{y_{j_0}}$, we have that $z \in W^{y_{j_0}}$. Since $z \in E$, 
 			\begin{align*}
 				z 
 				& \in W^{y_{j_0}} \cap E \\
 				& = \varnothing,
 			\end{align*}
 			which is a contradiction. Hence $U_{\al_0} \subset \pi_{\al_0}(E)^c$.
 		\end{enumerate}
 	\end{enumerate}
 \end{proof}
 
 
 
 \begin{ex} \lex{ex:topology:compactness_prod_spaces:00020}
 	Let $(X_{\al}, \MT_{\al})_{\al \in A}$ be a collection of topological spaces. Suppose that for each $\al \in A$, $(X_{\al} , \MT_{\al})$ is compact. Then for each $\al \in A$, $\pi_{\al}: \prod\limits_{\al \in A} X_{\al} \rightarrow X_{\al}$ is closed. \\
 	\tbf{Hint:} \rex{ex:topology:compactness_prod_spaces:00019}
 \end{ex}
 
 \begin{proof}
 	Set $X \defeq \prod\limits_{\al \in A} X_{\al}$ and $\MT_X \defeq \bigotimes_{\al \in A} \MT_{\al}$. Let $\al_0 \in A$ and $E \subset X$. Suppose that $E$ is closed in $(X, \MT_X)$. Let $a \in \pi_{\al_0}(E)^c$. \rex{ex:topology:compactness_prod_spaces:00019} implies that there exists $U_{\al_0} \in \MT_{\al_0}$ such that $a \in U_{\al_0}$ and $U_{\al_0} \subset \pi_{\al_0}(E)^c$. Since $a \in \pi_{\al_0}(E)^c$ is arbitrary, we have that for each $a \in \pi_{\al_0}(E)^c$, there exists $U_{\al_0} \in \MT_{\al_0}$ such that $a \in U_{\al_0}$ and $U_{\al_0} \subset \pi_{\al_0}(E)^c$. Therefore $\pi_{\al_0}(E)^c \in \MT_{\al_0}$ and $\pi_{\al_0}(E)$ is closed in $(X_{\al_0}, \MT_{\al_0})$. Since $E \subset X$ with $E$ closed in $(X, \MT_X)$ is arbitrary, we have that for each $E \subset X$, $E$ closed in $(X, \MT_X)$ implies that $\pi_{\al_0}(E)$ is closed in $(X_{\al_0}, \MT_{\al_0})$. Hence $\pi_{\al_0}$ is closed. Since $\al_0 \in A$ is arbitrary, we have that for each $\al \in A$, $\pi_{\al} : \prod\limits_{\al \in A} X_{\al} \rightarrow X_{\al}$ is closed.
 \end{proof}
 
 
 
 
 
 
 
 
 
 
 
 
 
 
 
 
 
 
 
 
 
 
 
 
 
 
 
 
 
 
 
 
 
 
 
 
 
 
 
 
 
 
 \subsection{Compactness and Coproduct Spaces}
 
 
 
 
 
 
 
 
 
 
 
 
 
 
 
 
 
 
 
 
 
 
 
 
 
 
 
 
 
 
 
 
 
 
 
 
 
 
 
 
 
 
 
 
 
 \subsection{Compactness and Quotient Spaces}
 
 \begin{ex} \lex{ex:topology:compactness:quotient_spaces:00011.1}
 	Let $(X, \MT)$ be a topological space and ${\sim} \subset X \times X$ an equivalence relation. If $(X, \MT)$ is compact, then $(X/{\sim}, \MT/{\sim})$ is compact. 
 \end{ex}
 
 \begin{proof}
 	Suppose that $(X, \MT)$ is compact. Let $\MV \subset \MP(X/{\sim})$. Suppose that $\MV$ is an open cover of $X/{\sim}$ in $(X, \MT/{\sim})$. Then $\MV \subset \MT/{\sim}$ and $X/{\sim} \subset \bigcup\limits_{V \in \MV} V$. Define $\MU \subset \MP(X)$ by $\MU \defeq \{\pi^{-1}(V): V \in \MV\}$. Since $\pi$ is $(\MT. \MT/{\sim})$-continuous and $\MV \subset \MT/{\sim}$, we have that 
 	\begin{align*}
 		\MU 
 		& = \{\pi^{-1}(V): V \in \MV\} \\
 		& \in \MT.
 	\end{align*}
 	Since $X/{\sim} \subset \bigcup\limits_{V \in \MV} V$, we have that 
 	\begin{align*}
 		X
 		& = \pi^{-1}(X/{\sim}) \\
 		& \subset \pi^{-1}(\bigcup\limits_{V \in \MV} V) \\
 		& = \bigcup\limits_{V \in \MV} \pi^{-1}(V) \\
 		& = \bigcup\limits_{U \in \MU} U \\
 	\end{align*}
 	Thus $\MU$ is an open cover of $X$ in $(X, \MT)$. Since $(X, \MT)$ is compact, there exists $\MU_0 \subset \MU$ such that $\MU_0$ is an open cover of $X$ in $(X, \MT)$ and $\MU_0$ is finite. For each $U \in \MU_0$, define $A_U \subset \MV$ by $A_U \defeq \{V \in \MV: \pi^{-1}(V) = U\}$. By construction, for each $U \in \MU_0$, $A_U \neq \varnothing$. Since $\MU_0$ is finite, $\prod\limits_{U \in \MU_0} A_U \neq \varnothing$. Thus there exists $(V_U)_{U \in \MU_0} \in \prod\limits_{U \in \MU_0} A_U$. Define $\MV_0 \subset \MP(X/{\sim})$ by $\MV_0 \defeq \{V_U:U \in \MU_0\}$. Then $\MV_0$ is finite. By construction,  
 	\begin{align*}
 		\MV_0 
 		& \subset \MV \\
 		& \subset \MT/{\sim}
 	\end{align*}
 	and for each $U \in \MU_0$, $\pi^{-1}(V_U) = U$. Since $\pi$ is surjective, we have that for each $U \in \MU_0$, 
 	\begin{align*}
 		V_U
 		& = V_U \cap X/{\sim} \\
 		& = \pi(\pi^{-1}(V_U)) \\
 		& = \pi(U).
 	\end{align*}
 	Therefore 
 	\begin{align*}
 		X/{\sim}
 		& = \pi(X) \\
 		& \subset \pi(\bigcup\limits_{U \in \MU_0} U) \\
 		& = \bigcup\limits_{U \in \MU_0} \pi(U) \\
 		& = \bigcup\limits_{U \in \MU_0} V_U \\
 		& = \bigcup\limits_{V \in \MV_0} V.
 	\end{align*} 
 	Hence $\MV_0$ is an open cover of $X/{\sim}$. Since $\MV \subset \MP(X)$ such that $\MV$ is an open cover of $X/{\sim}$ is arbitrary, we have that for each $\MV \subset \MP(X)$, if $\MV$ is an open cover of $X/{\sim}$ in $(X/{\sim}, \MT/{\sim})$, then there exists $\MV_0 \subset \MV$ such that $\MV_0$ is an open cover of $X/{\sim}$ in $(X/{\sim}, \MT/{\sim})$ and $\MV_0$ is finite. Thus $(X/{\sim}, \MT/{\sim})$ is compact.
 \end{proof}
 
 
 
 
 
 
 
 
 
 
 
 
 
 
 
 
 
 
 
 
 
 
 
 
 
 
 
 
 
 
 
 
 
 
 
 
 
 
 
 
 
 
 
 
 
 
 \subsection{Compactness and Projective Systems}
 
 \begin{ex} \lex{ex:topology:topology:compactness_proj_limits:0005} 
 	Let $(J, {\leq})$ be a directed poset, $((X_j, \MT_j)_{j \in J}, (\pi_{j,k})_{(j,k) \in \leq})$ a projective system of topological spaces. Set $X \defeq \varprojlim\limits_{j \in J} X_j$ and $\MT \defeq \varprojlim\limits_{j \in J} \MT_j$. If for each $j \in J$, $(X_j, \MT_j)$ is compact Hausdorff, then for each $j \in J$, $\pi_j: X \rightarrow X_j$ is $(\MT, \MT_j)$-closed.
 \end{ex}
 
 \begin{proof}
 	Suppose that for each $j \in J$, $(X_j, \MT_j)$ is compact Hausdorff. Let $j_0 \in J$. Set $X_0 \defeq \prod\limits_{j \in J} X_j$ and $\MT_0 \defeq \bigotimes\limits_{j \in J} T_j$. Since for each $j \in J$, $(X_j, \MT_j)$ is compact, \rex{ex:topology:compactness_prod_spaces:00020} implies that for each $j \in J$, $\prj_j: X_0 \rightarrow X_j$ is $(\MT_0, \MT_j)$-closed. Since for each $j \in J$, $(X_j, \MT_j)$ is Hausdorff, \rex{ex:topology:proj_limits:0004} implies that $X$ is closed in $(X_0, \MT_0)$. Since $\pi_{j_0} = \prj_{j_0}|_{X}$, \rex{ex:topology:subspaces:0010.01} then implies that $\pi_{j_0}:X \rightarrow X_j$ is $(\MT, \MT_j)$-closed. Since $j_0 \in J$ is arbitrary, we have that for each $j \in J$, $\pi_j:X \rightarrow X_j$ is $(\MT, \MT_j)$-closed.
 \end{proof}
 
 \begin{ex} \lex{ex:topology:topology:compactness_proj_limits:0006}
 	Let $(J, {\leq})$ be a directed poset, $((X_j, \MT_j)_{j \in J}, (\pi_{j,k})_{(j,k) \in \leq})$ a projective system of topological spaces. Set $((X, \MT), (\pi_j)_{j \in J}) \defeq \varprojlim\limits_{j \in J} ((X_j, \MT_j)_{j \in J}, (\pi_{j,k})_{(j,k) \in \leq})$. If for each $j \in J$, $(X_j, \MT_j)$ is a compact Hausdorff space, then $(X, \MT)$ is a compact Hausdorff space. 
 \end{ex}
 
 \begin{proof}
 	Suppose that for each $j \in J$, $(X_j, \MT_j)$ is compact Hausdorff. Set $X_0 \defeq \prod\limits_{j \in J} X_j$ and $\MT_0 \defeq \bigotimes\limits_{j \in J} \MT_j$. Since for each $j \in J$, $(X_j, \MT_j)$ is Hausdorff, \rex{ex:topology:separation:products:0001} implies that $(X_0, \MT_0)$ is Hausdorff. Since $\MT = \MT_0 \cap X$, \rex{ex:topology:separation:subspaces:0002} implies that $(X, \MT)$ is Hausdorff. Since for each $j \in J$, $(X_j, \MT_j)$ is compact, \rex{ex:topology:compactness_prod_spaces:00018} implies that $(X_0, \MT_0)$ is compact. \rex{ex:topology:proj_limits:0004} implies that $X$ is closed in $(X_0, \MT_0)$. Since $X$ is closed in $(X_0, \MT_0)$, \rex{ex:topology:compactness:subspaces:00006.1} implies that $(X, \MT)$ is compact. 
 \end{proof}
 
 \begin{ex} \lex{ex:topology:topology:compactness_proj_limits:0008}
 	Let $(J, {\leq})$ be a directed poset $((X_j)_{j \in J}, (\pi_{j,k})_{(j,k) \in \leq})$ be a projective system of topological spaces. Set $(X, (\pi_j)_{j \in J}) \defeq \varprojlim\limits_{j \in J} ((X_j)_{j \in J}, (\pi_{j,k})_{(j,k) \in \leq})$. If for each $j \in J$, $X_j$ is a compact Hausdorff space and $\pi_j$ is surjective, then $(X, (\pi_j)_{j \in J})$ is perfect. 
 \end{ex}
 
 \begin{proof}
 	Suppose that for each $j \in J$, $X_j$ is a compact Hausdorff space and $\pi_j$ is surjective. Let $j \in J$. Since for each $j' \in J$, $X_{j'}$ is compact Hausdorff, \rex{ex:topology:topology:compactness_proj_limits:0005} implies that $\pi_j$ is closed. Since $\pi_j$ is continuous, closed and surjective, \rex{ex:quotient_topology:0007} implies that $\pi_j$ is a quotient map. Since $j \in J$ is arbitrary, we have that for each $j \in J$, $\pi_j$ is a quotient map. Thus $X$ is perfect.
 \end{proof}













































\subsection{Compactness and Continuity} 

\begin{ex} \lex{ex:topology:compactness:continuity:0001}
	Let $(X, \MT_X)$ and $(Y, \MT_Y)$ be topological spaces and $f: X \rightarrow Y$. Suppose that $f$ is $(\MT_X, \MT_Y)$-continuous. Then for each $K \subset X$, if $K$ is compact in $(X, \MT_X)$, then $f(K)$ is compact in $(Y, \MT_Y)$. 
\end{ex}

\begin{proof}
	Let $K \subset X$. Suppose that $K$ is compact in $(X, \MT_X)$. Let $\MV \subset \MP(Y)$. Suppose that $\MV$ is an open cover of $f(K)$ in $(Y, \MT_Y)$. By definition, $\MV \subset \MT_Y$ and $f(K) \subset \bigcup\limits_{V \in \MV} V$. Define $\MU \subset \MP(X)$ by $\MU = \{f^{-1}(V):V \in \MV\}$. Since $f$ is continuous and $\MV \subset \MT_Y$, $\MU \subset \MT_X$ and by construction
	\begin{align*}
		K
		& \subset f^{-1}(f(K)) \\
		& \subset f^{-1}\bigg( \bigcup\limits_{V \in \MV} V \bigg) \\
		& = \bigcup\limits_{V \in \MV} f^{-1}(V) \\
		& = \bigcup\limits_{U \in \MU} U \\
	\end{align*}
	Hence $\MU$ is an open cover of $K$ in $(X, \MT)$. Since $K$ is compact, there exists $\MU_0 \subset \MU$ such that $\MU_0$ is an open cover of $K$ in $(X, \MT_X)$ and $\MU_0$ is finite. For $U \in \MU_0$, set $A_U = \{V \in \MV: f^{-1}(V) = U \}$. By construction, for each $U \in \MU_0$, $A_U \neq \varnothing$. Since $\MU_0$ is finite, $\prod\limits_{U \in \MU_0} A_U \neq \varnothing$. Thus there exists $V \in \prod\limits_{U \in \MU_0} A_U$. Set $\MV_0 = \{V_U: U \in \MU_0\}$. By construction $\MV_0 \subset \MV$ and for each $U \in \MU_0$, $f^{-1}(V_U) = U$. Therefore 
	\begin{align*}
		f(K)
		& \subset f \bigg( \bigcup\limits_{U \in \MU_0} U \bigg) \\
		& = \bigcup\limits_{U \in \MU_0} f(U) \\
		& = \bigcup\limits_{U \in \MU_0} f (f^{-1}(V_U)) \\
		& = \bigcup\limits_{U \in \MU_0} V_U \cap f(X) \\
		& \subset \bigcup\limits_{U \in \MU_0} V_U \\
		& = \bigcup\limits_{V \in \MV_0} V \\
	\end{align*}
	Hence $\MV_0$ is an open cover of $f(K)$. Since $\MV \subset \MP(X)$ with $\MV$ an open cover of $f(K)$ is arbitrary, we have that $f(K)$ is compact.
\end{proof}


\begin{ex} \lex{ex:topology:compactness:continuity:0002}  \tbf{Closed Map Lemma:} \\
	Let $(X, \MT_X), (Y, \MT_Y)$ be topological spaces and $f \in \Hom_{\Top}((X, \MT_X), (Y, \MT_Y))$. Suppose that $(X, \MT_X)$ is compact, $(Y, \MT_Y)$ is Hausdorff. Then 
	\begin{enumerate}
		\item $f$ is closed,
		\item $f$ is surjective implies that $f$ is a quotient map,
		\item $f$ is injective implies that $f$ is a $\Top$-embedding,
		\item $f$ is bijective implies that $f \in \Iso_{\Top}((X, \MT_X), (Y, \MT_Y))$.
	\end{enumerate}
\end{ex}

\begin{proof}\
	\begin{enumerate}
		\item Let $C \subset X$. Suppose that $C$ is closed in $(X, \MT_X)$. Since $(X, \MT_X)$ is compact, \rex{ex:topology:compactness:subspaces:00006.1} implies that $C$ is compact in $(X, \MT_X)$. Since $f$ is continuous and $C$ is compact in $(X, \MT_X)$, \rex{ex:topology:compactness:subspaces:00017} implies that $f(C)$ is compact in $(Y, \MT_Y)$. Since $(Y, \MT_Y)$ is Hausdorff, \rex{ex:topology:compactness:subspaces:00006} implies that $f(C)$ is closed in $(Y, \MT_Y)$. Since $C \subset X$ with $C$ closed in $(X, \MT_X)$ is arbitrary, we have that for each $C \subset X$, $C$ is closed in $(X, \MT_X)$ implies that $f(C)$ is closed in $(Y, \MT_Y)$.
		\item Suppose that $f$ is surjective. \rex{ex:quotient_topology:0007} implies that $f$ is a quotient map. 
		\item Suppose that $f$ is injective. \rex{ex:topology:subspaces:0017} implies that $f$ is a $\Top$-embedding.
		\item Clear by previous parts.
	\end{enumerate}
\end{proof}



\begin{defn} \ld{def:topology:compactness:continuity:0003} 
	Let $(X, \MT_X), (Y, \MT_Y)$ be topological spaces and $f: X \rightarrow Y$. Then $f$ is said to be \tbf{proper} if for each $K \subset X$, $K$ is compact in $(Y, \MT_Y)$ implies that $f^{-1}(K)$ is compact in $(X, \MT)$. 
\end{defn}

 
 
 
 
 
 
 
 
 
 
 
 
 
 
 
 
 
 
 
 
 
 
 
 
 
 
 
 
 
 
 
 
 
 
 
 
 
 
 
 
 
 
 
 
 
 \subsection{Sequential Compactness}
 
 \begin{defn} \ld{def:topology:compactness:sequent_compact:0001} 
 	Let $(X, \MT)$ be a topological space. Then $(X, \MT)$ is said to be \tbf{sequentially compact} if for each $(x_n)_{n \in \N} \subset X$, there exists $(x_{n_k})_{k \in \N} \subset (x_n)_{n \in \N}$ and $x \in X$ such that $x_{n_k} \rightarrow x$.   
 \end{defn}
 
 \begin{ex} \lex{ex:topology:compactness:sequent_compact:0002}
 	Let $(X, \MT)$ be a topological space. Suppose that $(X, \MT)$ is first-countable and $(X, \MT)$ is not sequentially compact. Then
 	\begin{enumerate}
 		\item there exists $(x_n)_{n \in \N} \subset X$ such that for each $x \in X$, there exists $U_x \in \MT$ and $n_x \in \N$ such that $x \in U_x$ and for each $n \in \N$, $n \geq n_x$ implies that $x_n \in U^c$, \\
 		\tbf{Hint:} \rex{ex:topology:countability:0005}
 		\item $(X, \MT)$ is not compact
 	\end{enumerate}
 \end{ex}
 
 \begin{proof}\
 	\begin{enumerate}
 		\item Since $(X, \MT)$ is not sequentially compact, there exists $(x_n)_{n \in \N} \subset X$ such that for each $(x_{n_k})_{k \in \N} \subset (x_n)_{n \in \N}$ and $x \in X$, $x_{n_k} \not \rightarrow x$. \rex{ex:topology:countability:0005} implies that for each $x \in X$, $x$ is not a cluster point of $(x_n)_{n \in \N}$. Thus for each $x \in X$, there exists $U_x \in \MT$ and $n_x \in \N$ such that $x \in U_x$ and for each $n \in \N$, $n \geq n_x$ implies that $x_n \in U_x^c$. 
 		\item For the sake of contradiction suppose that $(X, \MT)$ is compact. Define $\MU \subset \MT$ by $\MU \defeq \{ U_x: x \in X\}$. Then $\MU$ is an open cover of $X$ in $(X, \MT)$. Since $(X, \MT)$ is compact, there exists $J \in \N$ and $(a_j)_{j = 1}^J \subset X$ such that $(U_{a_j})_{j=1}^J$ is an open cover of $X$ in $(X, \MT)$. Set $n_0 \defeq \max(n_{a_j}: j \in [J])$. Let $n \in \N$. Suppose that $n \geq n_0$. Then for each $j \in [J]$, 
 		\begin{align*}
 			n 
 			& \geq n_0 \\
 			& \geq n_{a_j}.
 		\end{align*}
 		Thus for each $j \in [J]$, $x_n \in U_{a_j}^c$. Hence 
 		\begin{align*}
 			x_n 
 			& \in \bigcap\limits_{j=1}^J U_{a_j}^c \\
 			& = \bigg( \bigcup\limits_{j=1}^J U_{a_j} \bigg)^c \\
 			& = X^c \\
 			& = \varnothing.
 		\end{align*}
 		This is a contradiction. Therefore $(X, \MT)$ is not compact.
 	\end{enumerate}
 \end{proof}
 
 \begin{ex} \lex{ex:topology:compactness:sequent_compact:0003}
 	Let $(X, \MT)$ be a topological space and $(x_n)_{n \in \N} \subset X$. Suppose that $(X, \MT)$ is first-countable. If $(X, \MT)$ is compact, then $(X, \MT)$ is sequentially compact.
 \end{ex}
 
 \begin{proof}
 	Clear by the previous exercise. \tcr{(add detail?)}
 \end{proof}
 
 
 
 
 
 
 
 
 
 
 
 
 
 
 
 
 
 
 
 
 
 
 
 
 
 

 
 
 
 
 
 
 
 
 
 
 
 
 
 
 
 
 
 
 
 
 
 
 
 
 
 
 
 
 
 
 
 
 
 
 
 
 
 
 
 	\newpage
 \section{Locally Compact Hausdorff Spaces}
 
 \subsection{Introduction}
 
 \begin{defn}
 	Let $X$ be a topological space. Then 
 	\begin{itemize}
 		\item $X$ is said to be \tbf{locally compact} if for each $x \in X$, there exists $K \in \MN(x)$ such that $K$ is compact
 		\item $X$ is said to be a locally compact Hausdorff (LCH) space if $X$ is locally compact and $X$ is Hausdorff. 
 	\end{itemize} 
 \end{defn}

\begin{ex}
	Let $X$ be a LCH space and $U \subset X$. Suppose that $U$ is open. Then for each $x \in U$, there exists $K \in \MN(x)$ such that $K \subset U$ and $K$ is compact.
\end{ex}

\begin{proof}
	Let $x \in U$. Since $X$ is locally compact, there exists $K_0 \in \MN(x)$ such that $K_0$ is compact. Set $U_0 = (\Int K_0) \cap U$. Since $\cl U_0 \subset K_0$, $\cl U_0$ is closed and $K_0$ is compact, we have that $\cl U_0$ is compact. Since $U_0$ is open and $x \in U_0$, \tcb{an exercise in the section on compactness} implies that there exists $K \in \MN(x)$ such that $K \subset U$ and $K$ is compact.
\end{proof}

\begin{ex}
	Let $X$ be a LCH space and $U \subset X$ and $K \subset U$. If $U$ is open and $K$ is compact, then there exists $V \subset X$ such that $V$ is open, $K \subset V$, $\cl V \subset U$ and $V$ is precompact. 
\end{ex}

\begin{proof}
	Suppose that $U$ is open and $K$ is compact. \tcb{The previous exercise} implies that for each $x \in K$, there exist $N \in \MN(x)$ such that $N \subset U$ and $N$ is compact. The axiom of choice implies that there exists $(N_x)_{x \in K} \subset \MP(X)$ such that for each $x \in K$, $N_x \in \MN(x)$, $N_x \subset U$ and $N_x$ is compact. Then $(\Int N_x)_{x \in K}$ is an open cover of $K$. Since $K$ is compact, there exist $x_1, \ldots, x_n \in K$ such that $(\Int N_{x_j})_{j=1}^n$ is an open cover of $K$. Set $V = \bigcup\limits_{j=1}^n \Int N_{x_j}$. Then $V$ is open and since $(\Int N_{x_j})_{j=1}^n$ is an open cover of $K$, we have that 
	\begin{align*}
		K
		& \subset \bigcup_{j=1}^n \Int N_{x_j} \\
		& = V \\
	\end{align*}
	By construction, $\cl V = \bigcap\limits_{j=1}^n N_{x_j}$ which is compact, so $V$ is precompact. Finally
	\begin{align*}
		\cl V
		& = \bigcap_{j=1}^n N_{x_j} \\
		& \subset U
	\end{align*}
\end{proof}

\begin{ex} \tbf{Urysohn's Lemma for LCH Spaces:} \\
	Let $X$ be a LCH space, $U \subset X$ and $K \subset U$. If $U$ is open and $K$ is compact, then there exists $f \in C_c(X, [0,1])$ such that $f|_K = 1$ and $\supp f \subset U$.
\end{ex}

\begin{proof}
	Suppose that $U$ is open and $K$ is compact. \tcb{The previous exercise} implies that there exists $V \subset X$ such that $V$ is open, $K \subset V$, $\cl V \subset U$ and $V$ is precompact. 
\end{proof}

\begin{defn}
	Let $X$ be a LCH space and $f \in C(X)$. Then $f$ is said to \tbf{vanish at infinity} if for each $\ep >0$, $|f|^{-1}([\ep, \infty))$ is compact.
\end{defn}

\begin{ex} 
	Let $X$ be a LCH space. Then $\cl C_c(X) = C_0(X)$. 
\end{ex}

\begin{proof}
	\tcb{FINISH!!!}
\end{proof}

	
\begin{ex}
	Let $X, Y$ be topological spaces and $f: X \rightarrow Y$. Suppose that $Y$ is a LCH space and $f$ is continuous. If $f$ is proper, then $f$ is closed. \\
	\tbf{Hint:} Let $(y_{\al})_{\al \in A} \subset f(C)$ be a net and $y \in Y$. Suppose that $y_{\al} \rightarrow y$ and consider $K_y \in \MN(X)$, then $f(C \cap f^{-1}(K_y)) = f(C) \cap K_{y}$.
\end{ex}

\begin{proof}
	Suppose that $f$ is proper. Let $C \subset X$. Suppose that $C$ is closed in $X$. Let $(y_{\al})_{\al \in A} \subset f(C)$ be a net and $y \in Y$. Suppose that $y_{\al} \rightarrow y$. 
	
	Since $Y$ is LCH, there exists $K_y \in \MN(y)$ such that $K_y$ is compact. Since $Y$ is Hausdorff, \rex{ex:topology:compactness:subspaces:00006} implies that $K_y$ is closed. Since $f$ is continuous, $f^{-1}(K_y)$ is closed. Hence $C \cap f^{-1}(K_y)$ is closed. Since $f$ is proper, $f^{-1}(K_y)$ is compact. \rex{ex:topology:compactness:subspaces:00006.1} implies that $C \cap f^{-1}(K_y)$ is compact. Since $f$ is continous, $f(C \cap f^{-1}(K_y))$ is compact. \rex{ex:set_theory:functions:0004} implies that $f(C \cap f^{-1}(K_y)) = f(C) \cap K_y$. Since $f(C) \cap K_y$ is compact and $Y$ is Hausdorff, \rex{ex:topology:compactness:subspaces:00006} implies that $f(C) \cap K_y$ is closed in $Y$.
	
	Since $K_y \in \MN(y)$ and $y_{\al} \rightarrow y$, there exists $\al_0 \in A$ such that for each $\al \in A$, $\al \geq \al_0$ implies that $y_{\al} \in K_y$. Hence for each $\al \in A$, $\al \geq \al_0$ implies that $y_{\al} \in f(C) \cap K_y$. Since $f(C) \cap K_y$ is closed in $Y$, we have that
	\begin{align*}
		y 
		& \in f(C) \cap K_y \\
		& \subset f(C).
	\end{align*}
	Since $(y_{\al})_{\al \in A} \subset f(C)$ a net and $y \in Y$ such that $y_{\al} \rightarrow y$ are arbitrary, we have that for each net $(y_{\al})_{\al \in A} \subset f(C)$ and $y \in Y$,  $y_{\al} \rightarrow y$ implies that $y \in f(C)$. Hence $f(C)$ is closed in $Y$. Since $C \subset X$ such that $C$ is closed in $X$ is arbitrary, we have that for each $C \subset X$, $C$ is closed in $X$ implies that $f(C)$ is closed in $Y$. Therefore $f$ is closed.  
\end{proof}
	
	
	
	
	
	
	
	
	
	
	
	
	
	
	
	
	
	
	
	
	
	
	
	
	
	
	
	
	
\subsection{Arzela-Ascoli Theorem}



































\subsection{Stone-Weierstrass Theorem}
	
\begin{defn}
	Let $X$ be a compact Hausdorff space and $\MA \subset C(X, \C)$ a subalgebra. Then $\MA$ is said to \tbf{separate the points of $X$} if for each $x,y \in X$, $x \neq y$ implies that there exists $f \in \MA$ such that $f(x) \neq f(y)$. 
\end{defn}

\begin{ex}
	Equip $\R^2$ with pointwise addition and multiplication. 
	\begin{enumerate}
		\item For each $\MA \in \{ \{(0,0)\}, \spn \{(1, 0)\}, \spn \{(0,1)\}, \spn\{ (1,1), \R^2 \} \}$, $\MA$ is a subalgebra of $\R^2$. 
		\item For each $\MA \subset \R^2$, $\MA$ is a subalgebra implies that $\MA \in \{ \{(0,0)\}, \spn \{(1, 0)\}, \spn \{(0,1)\}, \spn\{ (1,1), \R^2 \} \}$. \\
		\tbf{Hint:} If $(s,t) \in \MA$, then $(s^2, t^2) \in \MA$. 
	\end{enumerate}
\end{ex}

\begin{proof}\
	\begin{enumerate}
		\item Clear.
		\item Since $\dim \R^2 = 2$, we have that $\dim \MA \in \{0, 1, 2\}$. If $\dim \MA = 0$, then $\MA = \{(0,0)\}$. If $\dim \MA = 2$, then $\MA = \R^2$. Suppose that $\dim \MA = 1$. Then there exists $(s,t) \in \MA$ such that $\MA = \spn \{(s,t)\}$. Thus $(s,t) \neq (0,0)$. Since $(s,t) \in \MA$, $(s^2, t^2) \in \MA$. Since $(s,t) \neq (0,0)$, $(s^2, t^2) \neq (0,0)$. Since $\MA = \spn \{(s,t)\}$, there exists $\lam \in \R$ such that $(s^2, t^2) = \lam (s,t)$. Thus $s(s - \lam) = 0$ and $t(t - \lam) = 0$. Since $(s^2, t^2) \neq (0,0)$, $\lam \neq 0$. Hence $(s,t) \in \{(\lam, 0), (0, \lam), (\lam, \lam)\}$. Therefore 
		\begin{align*}
			\MA
			& = \spn \{(s,t)\} \\
			& \in \{\spn \{(1,0)\}, \spn \{(0, 1)\}, \spn \{(1,1)\} \}.
		\end{align*}
	\end{enumerate}
\end{proof}

\begin{ex}
	For each $\ep > 0$, there exists $p \in \R[x]$ such that $p(0) = 0$ and for each $x \in [-1,1]$, $\bigg| |x| - p(x) \bigg| < \ep$. \\ 
	\tbf{Hint:} Define $f:(-1,1) \rightarrow \R$ by $f(t) \defeq (1 - t)^{1/2}$. Consider the Taylor series for $f$ at $t = 0$.  
\end{ex}

\begin{proof}
	\tcr{FINISH!!!}
\end{proof}

\begin{ex}
	Let $X$ be a compact Hausdorff space and $\MA \subset C(X, \C)$ a closed subalgebra.
\end{ex}


\begin{ex} \tbf{Real Stone-Weierstrass Theorem}
	Let $X$ be a compact Hausdorff space and $\MA \subset C(X, \C)$ a closed subalgebra. Then $\MA = C(X, \R)$ or there exists $x_0 \in X$ such that $\MA = \{f \in C(X, \R): f(x_0) = 0\}$.
\end{ex}

\begin{proof}
	content...
\end{proof}

\tcr{FINISH!!!}
	
	
	
	
	
	
	
	
	
	
	
	
	
	
	
	
	
	
	
	
	
	
	
	
	
	
	
	
	
	
	
	
	
	
	
	\newpage
	\section{Tychonoff Spaces}
	
	
	
	
	
	
	
	
	
	
	
	
	
	
	
	
	
	
	
	
	
	
	
	
	
	
	
	
	
	
	
	
	
	
	
	
	
	
	
	
	
	
	\newpage
	\section{Compactification}
	
	\begin{defn}
		Let $X$ and $Y$ be topological spaces and $\phi: X \rightarrow Y$. Then $(Y, \phi)$ is said to be a \tbf{compactification of $X$} if 
		\begin{enumerate}
			\item $Y$ is compact
			\item $\phi(X)$ is dense in $Y$
			\item $\phi: X \rightarrow \phi(X)$ is a homeomorphism
		\end{enumerate}
	\end{defn}
	
	\begin{defn}
		Let $X, X^* \in \Obj(\Top)$ and $\iota_X \in \Hom_{\Top}(X, Y)$. Then $(X', \iota_X)$ is said to be a \tbf{Stone-\v{C}ech compactification of $X$} if 
		\begin{enumerate}
			\item $(X', \iota_X)$ is a compactification of $X$
			\item for each compactification $(Y, \phi)$ of $X$, there exists a unique $\phi' \in \Hom_{\Top}(X', Y)$ such that $\phi' \circ \iota_X = \phi$, i.e. the following diagram commutes:
			\[ \begin{tikzcd}
				X \arrow[r, "\iota_X"] \arrow[dr, "\phi"'] & X' \arrow[d, "\phi'"]  \\
				                                           & Y
			\end{tikzcd}
			\]
		\end{enumerate}
	\end{defn}

















































	
	
	
	
	
	
	





	
	
	
	
	
	
	
	
	
	
	
	
	
	
	
	
	
	
	
	\newpage
	\section{Zero-Dimensional Spaces} 
	
	
	\begin{defn} \ld{def:topology:zero_dim:0001}
		Let $(X, \MT)$ be a topological space. Then $(X, \MT)$ is said to be \tbf{zero-dimensional} if there exists $\MB \subset \MT$ such that $\MB$ is a basis for $\MT$ and for each $B \in \MB$, $B$ is closed.
	\end{defn}
	
	\begin{ex} \lex{ex:topology:zero_dim:0001.1}
		Let $(X_1, \MT_1)$ and $(X_2, \MT_2)$ be topological spaces and $\phi: X_1 \rightarrow X_2$. If $\phi$ is a homeomporphism, then $(X_1, \MT_1)$ is zero-dimensional iff $(X_2, \MT_2)$ is zero-dimensional. 
	\end{ex}

	\begin{proof}
		\tcr{FINISH!!!}
	\end{proof}
	
	
	
	
	
	
	
	
	
	
	
	
	
	
	
	
	
	
	
	
	
	
	
	
	
	
	
	
	
	
	
	
	\subsection{Zero-Dimensional Subspaces}
	
	\begin{ex} \lex{ex:topology:zero_dim:0002}
		Let $(X, \MT)$ be a topological space and $A \subset X$. If $(X, \MT)$ is a zero-dimensional space, then $(A, \MT \cap A)$ is zero-dimensional. 
	\end{ex}
	
	\begin{proof}
		Suppose that $(X, \MT)$ is zero-dimensional. Then there exists $\MB_0 \subset \MT$ such that $\MB_0$ is a basis for $\MT$ and for each $B_0 \in \MB_0$, $B_0$ is open and closed. \rex{ex:topology:subspaces:0009} implies that $\MB_0 \cap A$ is a basis for $\MT \cap A$. Let $B \in \MB \cap A$. By definition, there exists $B_0 \in \MB$ such that $B = B_0 \cap A$. Since $B_0$ is open and closed in $(X, \MT)$, \rex{ex:topology:subspaces:0007} implies that $B$ is closed in $(A, \MT \cap A)$. Since $B \in \MB$ is arbitrary, we have that for each $B \in \MB$, $B$ is closed. Thus there exists $\MB \subset \MT \cap A$ such that $\MB$ is a basis for $\MT \cap A$ and for each $B \in \MB$, $B$ is closed. Hence $(A, \MT \cap A)$ is zero-dimensional.
	\end{proof}
	
	
	
	
	
	
	
	
	
	
	
	
	
	
	
	
	
	
	
	
	
	
	
	
	
	
	
	
	
	
	
	
	
	
	
	\subsection{Zero-Dimensional Product Spaces}
	
	\tcr{(change $n \in \N$ to $\al \in A$ or something, doesn't need to be countable.)}
	
	\begin{ex} \lex{ex:topology:zero_dim:0003}
		Let $(X_n, \MT_n)_{n \in \N}$ be a collection of topological spaces. If for each $n \in \N$, $(X_n, \MT_n)$ is zero-dimensional, then $(\prod\limits_{n \in \N} X_n, \bigotimes\limits_{n \in \N} \MT_n)$ is zero-dimensional. 
	\end{ex}
	
	\begin{proof}
		Suppose that for each $n \in \N$, $(X_n, \MT_n)$ is zero-dimensional. Then for each $n \in \N$, there exists $\MB_n \subset \MT_n$ such that $\MB_n$ is a basis for $\MT_n$ and for each $n \in \N$, $B \in \MB_n$, $B$ is open and closed in $(X_n, \MT_n)$. Set
		\begin{align*}
			\MB \defeq 
			& \bigg \{\prod_{n \in \N} B_n: \text{there exists $J \subset \N$ such that $ \# J < \infty$, } \\
			& \text{ for each $n \in J$, $B_n \in \MB_n$ and for each $n \in J^c$, $B_n = X_n$ } \bigg \}.
		\end{align*}
		\rex{ex:product_topology:0005} implies that $\MB$ is a basis for $\bigotimes\limits_{n \in \N} \MT_n$. Let $B \in \MB$. By definition of $\MB$, there exists $J \subset \N$ and $(B_n)_{n \in \N} \in \prod\limits_{n \in \N} \MT_n$ such that $ \# J < \infty$, for each $n \in J$, $B_n \in \MB_n$ and for each $n \in J^c$, $B_n = X_n$. Let $n \in \N$. If $n \in J$, then $B_n \in \MB_n$ and therefore $B_n$ is closed in $(X_n, \MT_n)$. If $n \in J^c$, then $B_n = X_n$ and therefore $B_n$ is closed in $(X_n, \MT_n)$. Therefore for each $n \in \N$, $B_n$ is closed in $(X_n, \MT_n)$. Hence for each $n \in \N$, $\pi_n^{-1}(B_n)$ is closed in $(\prod\limits_{n \in \N} X_n, \bigotimes\limits_{n \in \N} \MT_n)$. Since $J$ is finite and
		\begin{align*}
			B
			& = \prod_{n \in \N} B_n \\
			& = \bigcap_{n \in J} \pi_n^{-1}(B_n),
		\end{align*}  
		we have that $B$ is closed in $(\prod\limits_{n \in \N} X_n, \bigotimes\limits_{n \in \N} \MT_n)$. Since $B \in \MB$ is arbitrary, we have that for each $B \in \MB$, $B$ is closed in $(\prod\limits_{n \in \N} X_n, \bigotimes\limits_{n \in \N} \MT_n)$. Hence there exists $\MB \subset \bigotimes\limits_{n \in \N} \MT_n$ such that $\MB$ is a basis for $\bigotimes\limits_{n \in \N} \MT_n$ and for each $B \in \MB$, $B$ is closed in $(\prod\limits_{n \in \N} X_n, \bigotimes\limits_{n \in \N} \MT_n)$. Thus $(\prod\limits_{n \in \N} X_n, \bigotimes\limits_{n \in \N} \MT_n)$ is zero-dimensional. 
	\end{proof}
	
	
	
	
	
	
	
	
	
	
	
	
	
	
	
	
	
	
	
	
	
	
	
	
	
	
	
	
	
	
	
	
	
	
	
	\subsection{Zero-Dimensional Coproduct Spaces}
	
	\tcr{(change $n \in \N$ to $\al \in A$ or something, doesn't need to be countable.)}
	
	\begin{ex} \lex{ex:topology:zero_dim:0004}
		Let $(X_n, \MT_n)_{n \in \N}$ be a collection of topological spaces. If for each $n \in \N$, $(X_n, \MT_n)$ is zero-dimensional, then $(\coprod\limits_{n \in \N} X_n, \bigoplus\limits_{n \in \N} \MT_n)$ is zero-dimensional. 
	\end{ex}
	
	\begin{proof}
		Suppose that for each $n \in \N$, $(X_n, \MT_n)$ is zero-dimensional. Then for each $n \in \N$, there exists $\MB_n \subset \MT_n$ such that $\MB_n$ is a basis for $\MT_n$ and for each $B \in \MB_n$, $B$ is closed. Define $\MB \subset \bigoplus\limits_{n \in \N} \MT_n$ by $\MB \defeq \{ \iota_n(U) : \text{$n \in \N$ and $U \in \MB_n$} \}$. \rex{ex:topology:coproducts:0003.2} implies that $\MB$ is a basis for $\bigoplus\limits_{n \in \N} \MT_n$. \rex{ex:topology:coproducts:0003.11} implies that for each $n \in \N$, $\iota_n$ is closed. Hence for each $B \in \MB$, $B$ is closed. Thus $(\coprod\limits_{n \in \N} X_n, \bigoplus\limits_{n \in \N} \MT_n)$ is zero-dimensional. 
	\end{proof}
	
	\begin{ex} \lex{ex:topology:zero_dim:0005}
		Let $(X, \MT)$ be a topological space and $(A_n)_{n \in \N} \subset \MP(X)$. Suppose that $(A_n)_{n \in \N}$ is disjoint. If $(X, \MT)$ is zero dimensional, then $(\bigcup\limits_{n \in \N} A_n, \MT \cap [\bigcup\limits_{n \in \N} A_n])$ is zero-dimensional. 
	\end{ex}
	
	\begin{proof}
		Suppose that Since for each $n \in \N$, $A_n \subset X$ and $(X, \MT)$ is zero-dimensional, \rex{ex:topology:zero_dim:0002} implies that for each $n \in \N$, $(A_n, \MT \cap A_n)$ is zero-dimensional. \rex{ex:topology:zero_dim:0004} then implies that $(\coprod\limits_{n \in \N} A_n, \bigoplus\limits_{n \in \N} (\MT \cap A_n))$ is zero-dimensional. Define $\phi: \coprod\limits_{\al \in A} E_{\al} \rightarrow \bigcup\limits_{\al \in A} E_{\al}$ by $\phi(\al, x) \defeq x$. Since$(A_n)_{n \in \N}$ is disjoint, \rex{ex:topology:coproducts:0006} implies that $\phi$ is a $\bigg( \bigoplus\limits_{n \in \N} [\MT \cap A_n], \MT \cap \bigg[\bigcup\limits_{n \in \N} A_n \bigg] \bigg)$-homeomorphism. \rex{ex:topology:zero_dim:0001.1} then implies that $(\bigcup\limits_{n \in \N} A_n, \MT \cap [\bigcup\limits_{n \in \N} A_n])$ is zero-dimensional.
	\end{proof}
	
	
	
	
	
	
	
	
	
	
	
	
	
	
	
	
	
	
	
	
	
	
	
	
	
	
	
	
	
	
	
	
	
	
	
	
	
	
	
	
	
	
	\newpage
	\section{Semi-continuity}
	
	\begin{defn} \ld{}
	Let $X$ be a topological space, $f: X \rightarrow \Ru$ and $x_0 \in X$. Then $f$ is said to be \tbf{lower semicontinuous at $x_0$} if $$\liminf_{x \rightarrow x_0}f(x) \geq f(x_0)$$ and $f$ is said to be \tbf{lower semicontinuous} if for each $x_0 \in X$, $f$ is lower semicontinuous at $x_0$. 
	\end{defn}
	
	\begin{ex} \lex{}
	Let $X$ be a topological space and $f: X \rightarrow \Ru$. Then $f$ is \lsc\ iff for each $\al \in \R$, $f^{-1}((\al, \infty])$ is open. 
	\end{ex}
	
	\begin{proof}\
	\begin{itemize}
		\item $(\implies):$ \\
		Suppose that $f$ is \lsc. Let $\al \in \R$ and $x_0 \in f^{-1}(\al, \infty]$. 
		\begin{itemize}
			\item Suppose that $f(x_0) = \infty$. Since $f$ is lower semicontinuous, 
			\begin{align*}
				\sup_{V \in \MN(x_0)} \bigg[ \inf_{x \in V\setminus \{x_0\} } f(x) \bigg] 
				& = \liminf_{x \rightarrow x_0} f(x) \\
				& \geq f(x_0) \\
				& = \infty
			\end{align*}
			Thus there exists $V_{\al} \in \MN(x)$ such that 
			\begin{align*}
				\inf\limits_{x \in V_{\al} \setminus \{x_0\} } f(x) 
				& \geq \al + 1 \\
				& > \al 
			\end{align*}
			Hence for each $x \in V_{\al} \setminus \{x_0\}$,
			\begin{align*}
				f(x)
				& \geq \inf\limits_{t \in V_{\al} \setminus \{x_0\} } f(t) \\
				& > \al 
			\end{align*}
			Since $f(x_0) = \infty$, $f(x_0) > \al$. Hence  
			\begin{align*}
				\Int V_{\al} 
				& \subset V_{\al} \\
				& \subset f^{-1}((\al, \infty])
			\end{align*} 
			Thus there exists $V \in \MN(x_0)$ such that $V$ is open and $V \subset f^{-1}((\al, \infty])$.
			\item Suppose that $f(x_0) < \infty$. Set $\ep \defeq f(x_0) - \al$. By definition,  
			\begin{align*}
				\sup_{V \in \MN(x_0)} \bigg[ \inf_{x \in V \setminus \{x_0\}} f(x)  \bigg]
				& \geq f(x_0) \\
				& > f(x_0) - \ep 
			\end{align*}
			Choose $V_{\ep} \in \MN(x_0)$ such that 
			\begin{align*}
				\inf_{x \in V_{\ep} \setminus \{x_0\}} f(x)  
				&> f(x_0) - \ep \\
				&= \al
			\end{align*}
			Then
			\begin{align*}
				\Int V_{\ep} 
				& \subset V_{\ep} \\
				&\subset f^{-1}((\al, \infty])
			\end{align*} 
			Thus there exists $V \in \MN(x_0)$ such that $V$ is open and $V \subset f^{-1}((\al, \infty])$.
		\end{itemize}
		Since $x_0 \in f^{-1}((\al, \infty])$ is arbitrary, we have that for each $x_0 \in f^{-1}((\al, \infty])$, there exists $V \subset f^{-1}((\al, \infty])$ such that $V$ is open and $x_0 \in V$. Hence $f^{-1}((\al, \infty])$ is open. Since $\al \in \R$ is arbitrary, we have that for each $\al \in \R$, $f^{-1}((\al, \infty])$ is open.
		\item $(\impliedby):$ \\ 
		Suppose that for each $\al \in \R$, $f^{-1}((\al, \infty])$ is open. Let $x_0 \in X$. 
		\begin{itemize}
			\item Suppose that $f(x_0) = \infty$.  For $M > 0$, define $V_M = f^{-1}((M, \infty]) $. Then for each $M > 0$, $V_M \in \MN(x_0)$ and 
			\begin{align*}
				\liminf_{x \rightarrow x_0} f(x) 
				&= \sup_{V \in \MN(x_0)} \bigg[ \inf_{x \in V \setminus \{x_0\}} f(x) \bigg] \\
				& \geq \sup_{M > 0} \bigg[ \inf_{x \in V_M \setminus \{x_0\}} f(x) \bigg] \\
				& \geq \sup_{n \in \N} M \\
				& = \infty \\
				&= f(x_0) 
			\end{align*}
			\item Suppose that $f(x_0) < \infty$. Set $\al \defeq f(x_0)$. For $n \in \N$, define $V_n = f^{-1}((f(x_0)-1/n, \infty]) $. Then for each $n \in \N$, $V_n \in \MN(x_0)$ and 
			\begin{align*}
				\liminf_{x \rightarrow x_0} f(x) 
				&= \sup_{V \in \MN(x_0)} \bigg[ \inf_{x \in V \setminus \{x_0\}} f(x) \bigg] \\
				& \geq \sup_{n \in \N} \bigg[ \inf_{x \in V_n \setminus \{x_0\}} f(x) \bigg] \\
				& \geq \sup_{n \in \N} f(x_0)-1/n \\
				&= f(x_0) 
			\end{align*}
		\end{itemize}
		Hence So $f$ is lower semicontinuous at $x_0$. Since $x_0 \in X$ is arbitrary, we have that for each $x_0 \in X$, $f$ is lower semicontinuous at $x_0$. Therefore $f$ is lower semicontinuous.
	\end{itemize}
	\end{proof}

	\begin{defn}
		Let $X$ be a topological space and $f: X \rightarrow \R$. We define the \tbf{epigraph of $f$}, denoted $\epi f$, by 
		$$\epi f = \{(x, y) \in X \times \R: f(x) \leq y\}$$
	\end{defn}

	\begin{ex}
		Let $X$ be a topological space and $f: X \rightarrow \R$. Then $f$ is lower semicontinuous iff $\epi f$ is closed.
	\end{ex}

	\begin{proof}
		Suppose that $f$ is lower semicontinuous. Let $(x_{\al}, y_{\al})_{\al \in A} \subset \epi f$ be a net and $(x, y) \in X \times \R$. Then for each $\al \in A$, $f(x_{\al}) \leq y_{\al}$. Suppose that $(x_{\al}, y_{\al}) \rightarrow (x, y)$. Then $x_{\al} \rightarrow x$ and $y_{\al} \rightarrow y$. Therefore 
		\begin{align*}
			f(x) 
			& \leq \liminf_{t \rightarrow x} f(t) \\
			& \leq \liminf_{\al \in A} f(x_{\al}) \\
			& \leq \liminf_{\al \in A} y_{\al} \\
			&= y 
		\end{align*}
	So $(x, y) \in \epi f$ and $\epi f$ is closed. \\
	Conversely, suppose that $\epi f$ is closed. 
	\end{proof}

	\begin{ex}
		Let $X$ be a topological space and $ (f_{\lam})_{\lam \in \Lam} \subset (-\infty, \infty]^X$. Suppose that for each $\lam \in \Lam$, $f_\lam$ is \lsc. Set $f = \sup\limits_{\lam \in \Lam} f_{\lam}$. Then $f$ is \lsc.
	\end{ex}

	\begin{proof}
		Let $\al \in \R$ and $x \in X$. Then 
		\begin{align*}
			x \in f^{-1}((\al, \infty])
			& \iff \sup_{\lam \in \Lam} f_{\lam}(x) > \al \\
			& \iff \text{there exists $\lam \in \Lam$ such that } f_{\lam}(x) > \al \\
			& \iff \text{there exists $\lam \in \Lam$ such that } x \in f_{\lam}^{-1}((\al, \infty]) \\
			& \iff x \in \bigcup_{\lam \in \Lam} f_{\lam}^{-1}((\al, \infty])
		\end{align*}
		Since for each $\lam \in \Lam$, $f_{\lam}^{-1}((\al, \infty])$ is open, $f^{-1}((\al, \infty]) = \bigcup\limits_{\lam \in \Lam} f_{\lam}^{-1}((\al, \infty])$ is open. So $f$ is \lsc.
	\end{proof}
	
	
	
	
	
	
	
	
	
	
	
	
	
	
	
	
	
	
	
	
	
	
	
	
	
	
	
	
	
	
	
	
	
	
	
	
	
	
	
	
	
	
	
	
	
	
	
	
	
	
	
	
	
	\newpage
	\chapter{Locales}
	
	\begin{note}
		This chapter assumes familiarity with lattice theory. See \tcb{algebra notes}
	\end{note}
	
	\section{Sober Spaces}
	
	\begin{ex}
		Let $(X, \MT)$ be a topological space. Then 
		\begin{enumerate}
			\item $(\MT, \subset)$ is an ordered lattice
			\item $\vee_{\subset} = \cup$ and $\wedge_{\subset} = \cap$ 
			\item $(\MT, \vee_{\subset}, \wedge_{\subset})$ is a distributive lattice
		\end{enumerate}
	\end{ex}
	
	\begin{proof}
		Clear. \tcr{Fill out!!!}
	\end{proof}
	
	\begin{ex}
		Let $(X, \MT)$ be a topological space. Then for each $x \in X$, $\{x\}^c$ is meet-irreducible.
	\end{ex}
	
	\begin{proof}
		Let $x \in X$. Let $U,V \in \MT$. Suppose that $\{x\}^c = U \cap V$.  
	\end{proof}
	
	\begin{defn}
		We define the \tbf{}
	\end{defn}
	

































































\newpage
\chapter{Metric Spaces}

\section{Introduction}

\subsection{Metrics}

\begin{defn} \ld{def:metric_spaces:introduction:0001}
	Let $X$ be a set and $d: X \times X \rightarrow \R$. Then $d$ is said to be a \tbf{metric on $X$} if for each $x,y,z \in X$, 
	\begin{enumerate}
		\item $d(x,y) = d(y,x)$
		\item $d(x,y) = 0$ iff $x = y$
		\item $d(x, y) \leq d(x, z) + d(z, y)$
	\end{enumerate}	 
\end{defn}	

\begin{ex} \lex{ex:metric_spaces:introduction:0002}
	Let $X$ be a set and $d: X \times X \rightarrow \R$ a metric on $X$. Then for each $x,y \in X$, $d(x,y) \geq 0$. 
\end{ex}

\begin{proof}
	Let $x, y, z \in X$. Then $d(x,z) \leq d(x, y) + d(y,z)$. This implies that $d(x,z) - d(x, y) \leq d(y, z)$. Since $z$ is arbitrary, taking $z=x$, we obtain 
	\begin{align*}
		d(x,x) - d(x, y) \leq d(y, x)
		& \implies - d(x, y) \leq d(x, y) \\
		& \implies 0 \leq 2 d(x,y) \\
		& \implies d(x,y) \geq 0
	\end{align*}
\end{proof}	

\begin{defn} \ld{def:metric_spaces:introduction:0003}
	Let $X$ be a set and $d: X \times X \rightarrow \Rg$ a metric. Then $(X, d)$ is called a \tbf{metric space}.
\end{defn}	

\begin{note}
	We usually suppress the metric and write $X$ in place of $(X, d)$.
\end{note}	

\begin{ex} \lex{ex:metric_spaces:introduction:0003.1} \tbf{Reverse Triangle Inequality:} \\
	Let $(X, d)$ be a metric space. Then for each $x,y, z \in X$, $|d(x, y) - d(y,z)| \leq d(x,z)$.
\end{ex}

\begin{proof}
	Let $x,y, z \in X$. The triangle inequality implies that $d(x,y) \leq d(x, z) + d(z,y)$. Hence $d(x,y) - d(z,y) \leq d(x,z)$. Similarly, the triangle inequality implies that $d(y,z) \leq d(y, x) + d(x,z)$. Then $-d(y,z) \geq -d(y, x) - d(x,z)$. Therefore 
	$$- d(x,z) \leq d(x,y) - d(z,y) \leq d(x,z)$$ 
	and $|d(x,y) - d(z,y)| \leq d(x, z)$. 
\end{proof}

\begin{defn} \ld{def:metric_spaces:introduction:0005}
	Let $(X, d)$ be a metric space, $A \subset X$ and $x \in X$. We define \tbf{the distance between $A$ and $x$}, denoted $d(x, A)$, by 
	$$d(x, A) = \inf \{d(x, a): a \in A\}$$
\end{defn}

\begin{ex} \lex{ex:metric_spaces:introduction:0006}
	Let $(X, d)$ be a metric space, $A \subset X$. Then for each $x,y \in X$, $|d(x, A) - d(y, A)| \leq d(x, y)$.
\end{ex}

\begin{proof}
	Let $x,y \in X$. Let $\ep > 0$. Then there exists $a \in A$ such that $d(y, a) < d(y, A) + \ep$. The triangle inequality implies that
	\begin{align*}
		d(x, A) 
		& \leq d(x, a) \\
		& \leq d(x, y) + d(y, a) \\
		& < d(x, y) + d(y, A) + \ep. 
	\end{align*}
	Since $\ep > 0$ is arbitrary, we have that $d(x, A) \leq d(x, y) + d(y, A)$. Hence $d(x, A) - d(y, A) \leq d(x, y)$. \\
	Similarly, we have that $d(y, A) - d(x, A) \leq d(x, y)$. Therefore $|d(y, A) - d(x, A)| \leq d(x,y)$. 
\end{proof}

\begin{defn} \ld{ex:metric_spaces:introduction:0007}
	Let $(X,d)$ be a metric space and $A,B \subset X$. We define the \tbf{distance between $A$ and $B$}, denoted $d(A,B)$, by $$d(A,B) = \inf_{\substack{a \in A \\ b \in B}} d(a,b)$$
\end{defn}

\begin{ex} \lex{ex:metric_spaces:introduction:0008}
	Let $(X,d)$ be a metric space. Then for each $A,B \subset X$ and $c \in X$, $$d(A,B) \leq d(A,c) + d(c, B)$$
\end{ex}

\begin{proof}
	Let $A,B \subset X$, $c \in X$ and $\ep>0$. Choose $a \in A$ and $b \in B$ such that $d(a,c) < d(A,c)+ \ep/2$ and  $d(c,b) < d(c,B)+ \ep/2$. Then 
	\begin{align*}
		d(A,B) 
		&\leq d(a,b) \\
		&\leq d(a,c) + d(c,b) \\
		&< d(A,c) + \frac{\ep}{2} + d(c,B) + \frac{\ep}{2} \\
		&= d(A,c) + d(c,B) + \ep
	\end{align*}
	Since $\ep >0$ is arbitrary, $d(A,B) \leq d(A,c) + d(c,B)$.
\end{proof}































\subsection{Metric Topology}

\begin{defn} \ld{def:metric_spaces:introduction:0009}
	Let $(X, d)$ be a metric space, $x \in X$ and $r > 0$. We define the 
	\begin{itemize}
		\item \tbf{open ball of radius $r$ centered at $x$ in $(X, d)$}, denoted $B_d(x, r)$, by $$B_d(x, r) = \{y \in X: d(x,y) < r\}$$
		\item \tbf{closed ball of radius $r$ centered at $x$ in $(X, d)$}, denoted $\bar{B}_d(x, r)$, by $$\bar{B}_d(x, r) = \{y \in X: d(x,y) \leq r\}$$
		\item \tbf{circle of radius $r$ centered at $x$ in $(X, d)$}, denoted $T_d(x, r)$, by $$T_d(x, r) = \{y \in X: d(x,y) = r\}$$
	\end{itemize}
\end{defn}

\begin{note}
	When the context is clear we write $B(x, r)$, $\bar{B}(x, r)$ and $T(x, r)$ in place of $B_d(x, r)$, $\bar{B}_d(x, r)$ and $T_d(x, r)$ respectively.
\end{note}

\begin{defn} \ld{def:metric_spaces:introduction:0010}
	Let $(X, d)$ be a metric space. We define the \tbf{metric topology on X}, denoted $\MT_d$, by $$\MT_d = \{U \subset X: \text{ for each $x \in U$, there exists $\del > 0$ such that $B(x, \del) \subset U$}\}$$
\end{defn}

\begin{ex} \lex{ex:metric_spaces:introduction:0011}
	Let $(X,d)$ be a metric space. Then 
	\begin{enumerate}
		\item $\MT_d$ is a topology on $X$,
		\item $\{B(x, \del): x \in X, \del >0\}$ is a basis for $\MT_d$.
	\end{enumerate}
\end{ex}

\begin{proof}\
	\begin{enumerate}
		\item 
		\begin{enumerate}
			\item Clearly $X \in \MT_d$ and it is vacuously true that $\varnothing \in \MT_d$.
			\item Let $(U_{\al})_{\al \in A} \subset \MT_d$. Let $x \in \bigcup\limits_{\al \in A} U_{\al}$. Then there exists $\al_0 \in A$ such that $x \in U_{\al_0}$. By definition, there exists $\del > 0$ such that $B(x, \del) \subset U_{\al_0}$. Then 
			\begin{align*}
				B(x, \del) 
				& \subset U_{\al_0} \\
				& \subset \bigcup_{\al \in A} U_{\al} 
			\end{align*}
			Hence $\bigcup\limits_{\al \in A} U_{\al} \in \MT_d$.
			\item Let $(U_j)_{j=1}^n \subset \MT_d$ and $x \in \bigcap\limits_{j =1}^n U_j$. Then for each $j \in [n]$, $x \in U_j$. By definition, for each $j \in [n]$, there exists $\del_j > 0$ such that $B(x, \del_j) \subset U_j$. Set $\del \defeq \min(\del_1, \ldots, \del_n)$. Then for each $j \in [n]$, 
			\begin{align*}
				B(x, \del)
				& \subset B(x, \del_j) \\
				& \subset U_j.
			\end{align*}
			Therefore $B(x, \del) \subset \bigcap\limits_{j=1}^n U_j$. Hence $\bigcap\limits_{j=1}^n U_j \in \MT_d$.
		\end{enumerate}
		Thus $\MT_d$ is a topology on $X$.
		\item Set $\MB \defeq \{B(x, \del): x \in X, \del >0\}$. Let $U \in \MT_d$. Let $x \in U$. By definition, there exists $\del >0$ such that $B(x, \del) \subset U$. Set $B \defeq B(x, \del)$. Then $B \in \MB$ and $x \in B \subset U$. Since $x \in U$ is arbitrary, we have that for each $x \in U$, there exists $B \in \MB$ such that $x \in B \subset U$. Therefore $\MB$ is a basis for $\MT_d$.
	\end{enumerate}
\end{proof}

\begin{ex} \lex{ex:metric_spaces:introduction:0012}
	Let $X$ be a metric space and $U \subset X$. Then $U \in \MT_d$ iff for each $x \in U$, there exists $\del > 0$ such that $B(x, \del) \subset U$.
\end{ex}

\begin{proof}\
	\begin{itemize}
		\item $(\implies):$ \\
		Suppose that $U \in \MT_d$. Let $x \in U$. \rex{ex:metric_spaces:introduction:0011} implies that there exists $x_0 \in X$ and $\del_0 > 0$ such that $x \in B(x_0, \del_0)$ and $B(x_0, \del_0) \subset U$. Define $\del > 0$ by $\del \defeq \del_0 - d(x_0, x)$. Let $y \in B(x, \del)$. Then 
		\begin{align*}
			d(x_0, y)
			& \leq d(x_0, x) + d(x, y) \\
			& < d(x_0, x) + \del \\
			& = d(x_0, x) + (\del_1 - d(x_0, x)) \\
			& = \del_0.
		\end{align*}
		Thus $y \in B(x_0, \del_0)$. Since $y \in B(x, \del)$ is arbitrary, we have that for each $y \in B(x, \del)$, $y \in B(x_0, \del_0)$. Hence
		\begin{align*}
			B(x, \del) 
			& \subset B(x_0, \del_0) \\
			& \subset U. 
		\end{align*} 
		Since $x \in U$ is arbitrary, we have that for each $x \in U$, there exists $\del >0$ such that $B(x, \del) \subset U$.
		\item $(\impliedby):$ \\
		Suppose that for each $x \in U$, there exists $\del > 0$ such that $B(x, \del) \subset U$. The axiom of choice implies that there exists $(\del_x)_{x \in U} \subset (0, \infty)$ such that for each $x \in U$, $B(x, \del_x) \subset U$. Since for each $x \in U$, $B(x, \del_x) \in \MT_d$, we have that 
		\begin{align*}
			U
			& = \bigcup\limits_{x \in U} B(x, \del_x) \\
			& \in \MT_d. 
		\end{align*}
	\end{itemize}
\end{proof}

\begin{ex} \lex{ex:metric_spaces:introduction:0012.1}
	Let $(X, d)$ be a metric space, $\MB$ a basis for $\MT_d$ and $\ep > 0$. Define $\MB_{\ep} \subset \MT_d$ by $\MB_{\ep} \defeq \{B \in \MB: \diam(B) \leq \ep\}$. Then $\MB_{\ep}$ is a basis for $\MT_d$. \tcr{DEFINE $\diam: \MP(X) \rightarrow \RG$. !!!}.
\end{ex}

\begin{proof}
	Let $U \in \MT_d$ and $x \in U$. \rex{ex:metric_spaces:introduction:0012} implies that there exists $\del_0 > 0$ such that $B(x, \del_0) \subset U$. Set $\del \defeq \min(\del_0, \ep/2)$. Since $B(x, \del) \in \MT_d$ and $\MB$ is a basis for $\MT_d$, there exists $B \in \MB$ such that $x \in B$ and $B \subset B(x, \del)$. We note that
	\begin{align*}
		B
		& \subset B(x, \del) \\
		& \subset B(x, \del_0) \\
		& \subset U
	\end{align*} 
	and 
	\begin{align*}
		\diam(B)
		& \leq \diam(B(x, \del)) \\
		& \leq 2 \del \\
		& \leq \ep.
	\end{align*}
	Since $\diam(B) \leq \ep$, $B \in \MB_{\ep}$. Since $U \in \MT_d$ and $x \in U$ are arbitrary, we have that for each $U \in \MT_d$ and $x \in U$, there exists $B \in \MB_{\ep}$ such that $x \in B$ and $B \subset U$. Thus $\MB_{\ep}$ is a basis for $\MT_d$. 
\end{proof}

\begin{ex} \lex{ex:metric_spaces:introduction:0013}
	Let $(X,d)$ be a metric space and $x \in X$. Set $\MB_x = \{B(x, \del): \del > 0\}$. Then $\MB_x$ is a local basis for $\MT_d$ at $x$. \\
\end{ex}	

\begin{proof}\
	\begin{enumerate}
		\item Clearly for each $U \in \MB_x$, $x \in U$. 
		\item Let $V \in \MT_d$. Suppose that $x \in V$. By definition of $\MT_d$, there exists $\del > 0$ such that $B(x, \del) \in V$. Set $U \defeq B(x, \del)$. Then $U \in \MB_x$ and $U \subset V$.
	\end{enumerate}
	Hence $\MB_x$ is a local basis for $\MT_d$ at $x$.
\end{proof}

\begin{ex} \lex{ex:metric_spaces:introduction:0010.1}
	Let $(X, d)$ be a metric space. Then for each $x \in X$ and $r \geq 0$,
	\begin{enumerate}
		\item $\bar{B}(x, r)$ is $\MT_d$-closed,
		\item $T(x, r)$ is $\MT_d$-closed,
	\end{enumerate}
\end{ex}

\begin{proof}
	Let $x \in X$ and $r \geq 0$.
	\begin{enumerate}
		\item Let $y \in \bar{B}(x, r)^c$. Then $d(x, y) > r$. Define $\ep > 0$ by $\ep \defeq d(x,y) - r$. Let $z \in B(y, \ep)$. Since $d(x,y) \leq d(x, z) + d(z,y)$, we have that
		\begin{align*}
			d(x, z) 
			& \geq d(x,y) - d(z,y) \\
			& > d(x,y) - \ep \\
			& = r.
		\end{align*} 
		Thus $z \in \bar{B}(x, r)^c$. Since $z \in B(y, \ep)$ is arbitrary, we have that for each $z \in B(y, \ep)$, $z \in \bar{B}(x, r)^c$. Hence $B(y, \ep) \subset \bar{B}(x, r)^c$. Since $y \in \bar{B}(x, r)^c$ is arbitrary, we have that for each $y \in \bar{B}(x, r)^c$, there exists $\ep > 0$ such that $B(y, \ep) \subset \bar{B}(x, r)^c$. \rex{ex:metric_spaces:introduction:0012} then implies that $\bar{B}(x, r)^c \in \MT_d$. Thus $\bar{B}(x, r)$ is $\MT_d$-closed.
		\item We note that  
		\begin{align*}
			T(x, r) 
			& = \bar{B}(x, r) \setminus B(x, r) \\
			& = \bar{B}(x, r) \cap B(x, r)^c.
		\end{align*}
		Since $\bar{B}(x, r)$ is $\MT_d$-closed and $B(x, r)^c$ is $\MT_d$-closed, we have that $T(x, r)$ is $\MT_d$-closed.
	\end{enumerate}
\end{proof}

\begin{ex} \lex{ex:metric_spaces:introduction:0014}
	Let $(X, d_X)$, $(Y, d_Y)$ be a metric spaces, $f: X \rightarrow Y$ and $x_0 \in X$. Then $f$ is continuous at $x_0$ iff for each $\ep > 0$, there exists $\del > 0$ such that for each $x \in X$, $d_X(x_0, x) < \del$ implies that $d_Y(f(x_0, f(x))) < \ep$. 
\end{ex}

\begin{proof}\
	\begin{itemize}
		\item $(\implies):$ \\
		Suppose that $f$ is continuous at $x_0$. Let $\ep > 0$. Set $V \defeq B_Y(f(x_0), \ep)$. Since $V \in \MN(f(x_0))$, continuity at $x_0$ and \rex{ex:continuous_maps:0003} imply that $f^{-1}(V) \in \MN(x_0)$. \rex{ex:metric_spaces:introduction:0012} implies that there exists $\del > 0$ such that $x \in B_X(x_0, \del) \subset f^{-1}(V)$. Set $U \defeq B_X(x_0, \del)$. Let $x \in X$. Suppose that $d_X(x_0, x) < \del$. Then $x \in U$. Since $U \subset f^{-1}(V)$, we have that 
		\begin{align*}
			f(x) 
			& \in f(U) \\
			& \subset V \\
			& = B_Y(f(x_0), \ep)
		\end{align*}
		and $d_Y(f(x_0), f(x)) < \ep$. Since $\ep > 0$ is arbitrary, we have that for each $\ep >0$, there exists $\del > 0$ such that for each $x \in X$, $d_X(x_0, x) < \del$ implies that $d_Y(f(x_0), f(x)) < \ep$. 
		\item $(\impliedby):$ \\
		Suppose that for each $\ep > 0$, there exists $\del > 0$ such that for each $x \in X$, $d_X(x_0, x) < \del$ implies that $d_Y(f(x_0, f(x))) < \ep$. \tcr{FINISH!!!}
	\end{itemize}
\end{proof}

\begin{ex} \lex{ex:metric_spaces:introduction:0015}
	Let $(X, d)$ be a metric space, $(x_{\gam})_{\gam \in \Gam}$ a net and $x \in X$. Then $x_{\gam} \rightarrow x$ in $(X, \MT_d)$ iff $d(x_{\gam}, x) \rightarrow 0$ in $(\R, \MT_{\R})$. 
\end{ex}

\begin{proof}\
	\begin{itemize}
		\item $(\implies):$ \\
		Suppose that $x_{\gam} \rightarrow x$ in $(X, \MT_d)$. Let $U \in \MN_{\R}(0)$. Then there exists $\ep > 0$ such that $(-\ep, \ep) \subset U$. Since $B(x, \ep) \in \MN(x)$ and $x_{\gam} \rightarrow x$ in $(X, \MT_d)$, we have that $(x_{\gam})_{\gam \in \Gam}$ is eventually in $B(x, \ep)$. Thus there exists $\gam_0 \in \Gam$ such that for each $\gam \in \Gam$, $\gam \geq \gam_0$ implies that $x_{\gam} \in B(x, \ep)$. Let $\gam \in \Gam$. Suppose that $\gam \geq \gam_0$. Then 
		\begin{align*}
			d(x_{\gam}, x) 
			& \in (-\ep, \ep) \\
			& \subset U.
		\end{align*}
		Thus $(d(x_{\gam}, x))_{\gam \in \Gam}$ is eventually in $U$. Since $U \in \MN_{\R}(0)$ is arbitrary, we have that for each $U \in \MN_{\R}(0)$, $(d(x_{\gam}, x))_{\gam \in \Gam}$ is eventually in $U$. Hence $d(x_{\gam}, x) \rightarrow 0$.
		\item $(\impliedby):$ \\
		Conversely, suppose that $d(x_{\gam}, x) \rightarrow 0$ in $(\R, \MT_{\R})$. Let $U \in \MN_X(x)$. Then there exists $\ep > 0$ such that $B(x, \ep) \subset U$. Since $(-\ep, \ep) \in \MN_{\R}(0)$ and $d(x_{\gam}, x) \rightarrow 0$ in $(\R, \MT_{\R})$, we have that $(d(x_{\gam}, x))_{\gam \in \Gam}$ is eventually in $(-\ep, \ep)$. Thus there exists $\gam_0 \in \Gam$ such that for each $\gam \in \Gam$, $\gam \geq \gam_0$ implies that $d(x_{\gam}, x) \in (-\ep, \ep)$. Let $\gam \in \Gam$. Suppose that $\gam \geq \gam_0$. Then $d(x_{\gam}, x) < \ep$ and 
		\begin{align*}
			x_{\gam} 
			& \in B(x, \ep) \\
			& \subset U. 
		\end{align*} 
		Thus $(x_{\gam})$ is eventually in $U$. Since $U \in \MN(x)$ is arbitrary, we have that for each $U \in \MN(x)$, $(x_{\gam})_{\gam \in \Gam}$ is eventually in $U$. Therefore $x_{\gam} \rightarrow x$.  
	\end{itemize}
\end{proof}

\begin{defn} \ld{def:metric_spaces:introduction:0015.1}
	Let $(X, d_X)$ and $(Y, d_Y)$ be metric spaces and $f:X \rightarrow Y$. Then $f$ is said to be \tbf{uniformly continuous} if for each $\ep > 0$, there exists $\del > 0$ such that for each $x_1, x_2 \in X$, $d_X(x_1, x_2) < \del$ implies that $d_Y(f(x_1), f(x_2)) < \ep$.
\end{defn}

\begin{ex} \lex{ex:metric_spaces:introduction:0016}
	Let $(X, d)$ be a metric space, $A \subset X$. Then $d(\cdot, A)$ is uniformly continuous.
\end{ex}

\begin{proof}
	Let $\ep > 0$. Choose $\del = \ep$. Let $x, y \in X$. Suppose that $d(x,y) < \del$. Then 
	\begin{align*}
		|d(y, A) - d(x, A)|
		& \leq d(x,y) \\
		& < \del \\
		& = \ep 
	\end{align*}
	Since $\ep > 0$ is arbitrary, we have that for each $\ep > 0$, there exists $\del > 0$ such that for each $x,y \in X$, $d(x,y) < \del$ implies that $|d(y, A) - d(x, A)| < \ep$. Hence $d(\cdot, A)$ is uniformly continuous.
\end{proof}

\begin{defn} \ld{def:metric_spaces:introduction:0016.1}
	Let $(X, d_X)$ and $(Y, d_Y)$ be metric spaces and $f:X \rightarrow Y$. Then $f$ is said to be \tbf{Lipchitz} if there exists $K > 0$ such that for each $x_1, x_2 \in X$, $d_Y(f(x_1), f(x_2)) \leq K d_X(x_1, x_2)$. 
\end{defn}

\begin{ex} \lex{ex:metric_spaces:introduction:0016.11}
	Define $\phi: \Rg \rightarrow [0,1)$ by $\phi(t) \defeq t/(1+ t)$. Then $\phi$ is Lipschitz. 
\end{ex}

\begin{proof}
	Let $t_1, t_2 \in \Rg$. Then 
	\begin{align*}
		\phi(t_1) - \phi(t_2)
		& = \frac{t_1}{1 + t_1} - \frac{t_2}{1 + t_2} \\
		& = \frac{t_1(1+ t_2) - t_2(1+ t_1)}{(1 + t_1)(1 + t_2)} \\
		& = \frac{t_1 - t_2}{(1 + t_1)(1 + t_2)}.
	\end{align*}
	Therefore 
	\begin{align*}
		|\phi(t_1) - \phi(t_2)| 
		& = \bigg| \frac{t_1 - t_2}{(1 + t_1)(1 + t_2)} \bigg| \\
		& = \frac{|t_1 - t_2|}{|1 + t_1||1 + t_2|} \\
		& \leq |t_1 - t_2|.
	\end{align*}
	Since $t_1, t_2 \in \Rg$ is arbitrary, we have that $\phi$ is Lipschitz.
\end{proof}

\begin{ex} \lex{ex:metric_spaces:introduction:0016.2}
	Let $(X, d_X)$ and $(Y, d_Y)$ be metric spaces and $f:X \rightarrow Y$. If $f$ is Lipschitz, then $f$ is uniformly continuous.
\end{ex}

\begin{proof}
	Suppose that $f$ is Lipschitz. Let $\ep > 0$. Since $f$ is Lipschitz, there exists $K > 0$ such that for each $x_1, x_2 \in X$, $d_Y(f(x_1), f(x_2)) \leq K d_X(x_1, x_2)$. Set $\del \defeq \ep/K$. Then $\del > 0$. Let $x_1, x_2 \in X$. Suppose that $d_X(x_1, x_2) < \del$. Then
	\begin{align*}
		d_Y(f(x_1), f(x_2))
		& \leq K d_X(x_1, x_2) \\
		& < K \del \\
		& = \frac{K \ep}{K} \\
		& = \ep 
	\end{align*}
	Since $\ep > 0$ is arbitrary, we have that for each $\ep > 0$, there exists $\del > 0$ such that for each $x_1, x_2 \in X$, $d_X(x_1, x_2) < \del$ implies that $d_Y(f(x_1), f(x_2)) < \ep$. Hence $f$ is uniformly continuous.
\end{proof}

\begin{ex} \lex{ex:metric_spaces:introduction:0017}
	Let $(X, d)$ be a metric space, $C \subset X$ and $x \in X$. Suppose that $C$ is closed. Then $d(x, C) = 0$ iff $x \in C$.
\end{ex}

\begin{proof}
	Suppose that $d(x, C) = 0$. Then for each $n \in \N$, there exists $c_n \in C$ such that $d(x, c_n) < 1/n$. Then $c_n \rightarrow x$. Since $C$ is closed, $x \in C$. \\
	Conversely, suppose that $x \in C$. Then 
	\begin{align*}
		d(x, C)
		& = \inf \{d(x, c): c \in C\} \\
		& \leq d(x, x) \\
		& = 0
	\end{align*}
	Hence $d(x, C) = 0$.
\end{proof}

\begin{defn} \ld{ex:metric_spaces:introduction:0018}
	Let $(X, d)$ be a metric space, $A \subset X$ and $\ep >0$. We define the \tbf{$\ep$-enlargement of $A$}, denoted $A_{\ep}$, by 
	$$A_{\ep} = \{x \in X: d(x, A) < \ep\}$$ 
\end{defn}

\begin{ex} \lex{ex:metric_spaces:introduction:0019}
	Let $(X, d)$ be a metric space, $A \subset X$ and $\ep >0$. Then $A_{\ep}$ is open. 
\end{ex}

\begin{proof}
	Let $x \in A_{\ep}$. By definition, $d(x, A) < \ep$. Set $\del = (\ep - d(x,A)) / 2$. Then $\del > 0$ and thus there exists $a \in A$ such that $d(x,a) < d(x, A) + \del$. Let $y \in B(x, \del)$. Therefore
	\begin{align*}
		d(y, A)
		& = \inf \{d(y, b): b \in A\} \\
		& \leq d(y, a) \\
		& \leq d(y, x) + d(x, a) \\
		& < \del + d(x, A) + \del \\
		& = d(x, A) + 2\del \\
		& = d(x, A) + \ep - d(x, A) \\
		& = \ep 
	\end{align*}
	Hence $y \in A_{\ep}$. Since $y \in B(x, \del)$ is arbitrary, $B(x, \del) \subset A_{\ep}$. Since $x \in A_{\ep}$ is arbitrary, we have that for each $x \in A_{\ep}$, there exists $\del > 0$ such that $B(x, \del) \subset A_{\ep}$. Hence $A_{\ep}$ is open.
\end{proof}

\begin{ex} \lex{ex:metric_spaces:introduction:0020}
	Let $(X, d)$ be a metric space and $C, U \subset X$. 
	\begin{enumerate}
		\item If $C$ is closed, then $C = \bigcap\limits_{n \in \N} C_{1/n}$.
		\item If $U$ is open, then $U = \bigcup\limits_{n \in \N} [(U^c)_{1/n}]^c$.
	\end{enumerate} 
\end{ex}

\begin{proof}\
	\begin{enumerate}
		\item Suppose that $C$ is closed. Since for each $n \in \N$, $C \subset C_{1/n}$, we have that $C \subset \bigcap\limits_{n \in \N} C_{1/n}$. \\
		For the sake of contradiction, suppose that $\bigcap\limits_{n \in \N} C_{1/n} \not \subset C$. Then there exists $x \in \bigcap\limits_{n \in \N} C_{1/n}$ such that $x \not \in C$. Since $C$ is closed, a previous exercise implies that $d(x, C) > 0$. Set $\ep = d(x, C)$. Since $\ep >0$, there exists $N \in \N$ such that $1/N < \ep$. Since $x \in \bigcap\limits_{n \in \N}$, $x \in C_{1/N}$. Thus 
		\begin{align*}
			d(x, C)
			& < 1/N \\
			& < \ep \\
			& = d(x, C)
		\end{align*}
		which is a contradiction. Hence $\bigcap\limits_{n \in \N} C_{1/n} \subset C$. Thus $C = \bigcap\limits_{n \in \N} C_{1/n}$.
		\item Suppose that $U$ is open. Then $U^c$ is closed. The previous part implies that 
		\begin{align*}
			U
			& = (U^c)^c \\
			& = (\bigcap\limits_{n \in \N} (U^c)_{1/n})^c \\
			& = \bigcup_{n \in \N} [(U^c)_{1/n}]^c 
		\end{align*}
	\end{enumerate}
\end{proof}

\begin{ex} \lex{ex:metric_spaces:introduction:0022}
	Let $(X, d)$ be a metric space. Then 
	\begin{enumerate}
		\item for each $U \subset X$, if $U$ is open, then $U$ is an $F_{\sig}$ set,
		\item for each $C \subset X$, if $C$ is closed, then $C$ is a $G_{\del}$ set.
	\end{enumerate}
\end{ex}

\begin{proof}
	Clear by \tcb{the previous exercise}.
\end{proof}

\begin{ex} \lex{ex:metric_spaces:introduction:0023}
	Let $(X, d)$ be a metric space, $(x_n)_{n \in \N} \subset X$, $x \in X$ and $r > 0$. Suppose that $x_n \rightarrow x$. Then 
	\begin{enumerate}
		\item for each $\ep > 0$ there exists $N \in \N$ such that for each $n \in \N$, $n \geq N$ implies that $B(x_n, r) \subset B(x, r+\ep)$.
		\item for each $\ep > 0$ there exists $N \in \N$ such that for each $n \in \N$, $n \geq N$ implies that $\bar{B}(x_n, r) \subset \bar{B}(x, r+\ep)$.
	\end{enumerate} 
\end{ex}

\begin{proof}\
	\begin{enumerate}
		\item Let $\ep >0$. Since $x_n \rightarrow x$, there exists $N \in \N$ such that for each $n \in \N$, $n \geq N$ implies that $d(x_n, x) < \ep$. Let $n \in \N$. Suppose that $n \geq N$. Let $y \in B(x_n, r)$. Then 
		\begin{align*}
			d(y, x)
			& \leq d(y, x_n) + d(x_n, x) \\
			& < r + \ep \\
		\end{align*}
		Thus $y \in B(x, r+ \ep)$. Since $y \in B(x_n, r)$ is arbitrary, $B(x_n, r) \subset  B(x, r+ \ep)$.
		\item Similar to $(1)$. 
	\end{enumerate}
\end{proof}


\begin{ex} \lex{ex:metric_spaces:introduction:0024}
	Let $(X, d)$ be a metric space and $E \subset X$. Then $E$ is dense in $(X, \MT)$ iff for each $x \in X$ and $\ep > 0$, there exists $x' \in E$ such that $d(x, x')< \ep$. 
\end{ex}

\begin{proof}\
	\begin{itemize}
		\item $(\implies):$ \\
		Suppose that $E$ is dense in $(X, \MT)$. Let $x \in X$ and $\ep > 0$. Since $B(x, \ep) \in \MT_d$ and $B(x, \ep) \neq \varnothing$, \rex{31022} implies that there exists $x' \in E$ such that $x'\in B(x, \ep)$. Then $d(x, x')< \ep$. Since $x \in X$ and $\ep > 0$ are arbitrary, we have that for each $x \in X$ and $\ep > 0$, there exists $x' \in E$ such that $d(x, x')< \ep$. 
		\item $(\impliedby):$ \\
		Conversely, suppose that for each $x \in X$ and $\ep > 0$, there exists $x' \in E$ such that $d(x, x')< \ep$. Let $U \in \MT_d$. Suppose that $U \neq \varnothing$. Then there exists $x \in X$ such that $x \in U$. \rex{ex:metric_spaces:introduction:0012} implies that there exists $\del > 0$ such that
	\end{itemize}
\end{proof}


\begin{ex} \lex{ex:metric_spaces:introduction:0024.1}
	Let $(X, d)$ be a complete metric space and $U \subset X$. Suppose that $(X, \MT_d)$ is separable and $U$ is open. 
	Set $C \defeq \{(x, d(x, U^c)^{-1}): x \in U\}$ and define $f:U \rightarrow C$ by $f(x) \defeq (x, d(\cdot, U^c)^{-1})$. Then 
	\begin{enumerate}
		\item $f$ is a homeomorphism,
		\item $C$ is closed in $X \times \R$.
	\end{enumerate}
\end{ex}

\begin{proof}\
	\begin{enumerate}
		\item 
		\begin{itemize}
			\item We note that $\prj_1|_C \circ f = \id_X$ and $f \circ \prj_1|_C = \id_{C}$. Thus $f$ is a bijection with $f^{-1} = \prj_1|_C$. 
			\item \rex{ex:metric_spaces:introduction:0016} and \rex{ex:metric_spaces:introduction:0017} imply that $f$ is continous. Since $\prj_1:X \times \R \rightarrow X$ is continuous, $f^{-1} = \prj_1|_C$ is continuous. Hence $f$ is a homeomorphism. 
		\end{itemize}
		\item Let $(y_n)_{n \in \N} \subset C$ and $y \in X \times \R$. Suppose that $y_n \rightarrow y$. Since $C = f(U)$, there exist $(x_n)_{n \in \N} \subset U$ such that for each $n \in \N$, $y_n = f(x_n)$. Since $y \in X \times \R$, there exists $x \in X$ and $a \in \R$ such that $y = (x, a)$. Since $y_n \rightarrow y$, we have that
		\begin{align*}
			(x_n, d(x_n, U^c)^{-1}) 
			& = y_n \\
			& \rightarrow y \\
			& = (x, a).  
		\end{align*}  
		Hence $x_n \rightarrow x$ and $d(x_n, U^c)^{-1} \rightarrow a$. For the sake of contradiction, suppose that $x \in U^c$. Since $x_n \rightarrow x$, 
		\begin{align*}
			d(x_n, U^c) 
			& \leq d(x_n, x) \\
			& \rightarrow 0.
		\end{align*}
		Therefore 
		\begin{align*}
			d(x_n, U^c)^{-1} 
			& \rightarrow \infty \\
			& \neq a. 
		\end{align*}
		This is a contradiction. Hence $x \in U$. By continuity,
		\begin{align*}
			(x_n, d(x_n, U^c)^{-1}) 
			& = f(x_n) \\
			& \rightarrow f(x) \\
			& = (x, d(x, U^c)^{-1}).
		\end{align*}
		In particular, $d(x_n, U^c)^{-1} \rightarrow d(x, U^c)^{-1}$. Since $d(x_n, U^c)^{-1} \rightarrow a$, we have that $a = d(x, U^c)^{-1}$. Therefore
		\begin{align*}
			y
			& = (x, a) \\
			& = (x, d(x, U^c)^{-1}) \\
			& = f(x) \\
			& \in f(U) \\
			& = C.
		\end{align*}
		Since $(y_n)_{n \in \N} \subset C$ and $y \in X \times \R$ with $y_n \rightarrow y$ are arbitrary, we have that for each $(y_n)_{n \in \N} \subset C$ and $y \in X \times \R$, $y_n \rightarrow y$ implies that $y \in C$. Thus $C$ is closed in $X \times \R$. 
	\end{enumerate}
\end{proof}


\begin{ex} \lex{ex:metric_spaces:introduction:0024.2}
	Let $(X, d)$ be a complete metric space and $(U_n)_{n \in \N} \subset \MT_{d}$. Suppose that $(X, \MT_d)$ is separable. Set $E \defeq \bigcap\limits_{n \in \N} U_n$ and define $C \subset X \times \R^{\N}$ and $f:E \rightarrow C$ by 
	$$C \defeq \{(x, d(x, U_1^c)^{-1}, d(x, U_2^c)^{-1}, \ldots): x \in X\} \quad \text{ and } \quad f(x) \defeq (x, d(x, U_1^c)^{-1}, d(x, U_2^c)^{-1}, \ldots).$$ 
	Then 
	\begin{enumerate}
		\item $f$ is a homeomorphism,
		\item $C$ is closed in $X \times \R^{\N}$.
	\end{enumerate}
\end{ex}

\begin{proof}\
	\begin{enumerate}
		\item 
		\begin{itemize}
			\item We note that $\prj_1|_C \circ f = \id_X$ and $f \circ \prj_1|_C = \id_{C}$. Thus $f$ is a bijection with $f^{-1} = \prj_1|_C$. 
			\item \rex{ex:metric_spaces:introduction:0016} and \rex{ex:metric_spaces:introduction:0017} imply that $f$ is continous. Since $\prj_1:X \times \R \rightarrow X$ is continuous, $f^{-1} = \prj_1|_C$ is continuous. Hence $f$ is a homeomorphism. 
		\end{itemize}
		\item Let $(y_n)_{n \in \N} \subset C$ and $y \in X \times \R^{\N}$. Suppose that $y_n \rightarrow y$. Since $C = f(U)$, there exist $(x_n)_{n \in \N} \subset U$ such that for each $n \in \N$, $y_n = f(x_n)$. Since $y \in X \times \R^{\N}$, there exists $x \in X$ and $a \in \R^{\N}$ such that $y = (x, a)$. Since $y_n \rightarrow y$, we have that
		\begin{align*}
			(x_n, d(x_n, U^c_1)^{-1}, d(x_n, U^c_2)^{-1}, \ldots) 
			& = y_n \\
			& \rightarrow y \\
			& = (x, a_1, a_2, \ldots).  
		\end{align*}  
		Hence $x_n \rightarrow x$ and for each $k \in \N$ $d(x_n, U^c_k)^{-1} \conv{n} a_k$. For the sake of contradiction, suppose that $x \in E^c$. Since  
		\begin{align*}
			E^c 
			& = \bigg( \bigcap_{k \in \N} U_k \bigg)^c \\
			& = \bigcup_{k \in \N} U_k^c 
		\end{align*}
		there exists $k_0 \in \N$ such that $x \in U_{k_0}^c$.
		Since $x_n \rightarrow x$, 
		\begin{align*}
			d(x_n, U^c_{k_0}) 
			& \leq d(x_n, x) \\
			& \rightarrow 0.
		\end{align*}
		Therefore 
		\begin{align*}
			d(x_n, U_{k_0}^c)^{-1} 
			& \rightarrow \infty \\
			& \neq a_{k_0}. 
		\end{align*}
		This is a contradiction. Hence $x \in E$. By continuity,
		\begin{align*}
			(x_n, d(x_n, U^c_1)^{-1}, d(x_n, U^c_2)^{-1}, \ldots) 
			& = f(x_n) \\
			& \rightarrow f(x) \\
			& = (x, d(x, U^c_1)^{-1}, d(x, U^c_2)^{-1}, \ldots).
		\end{align*}
		In particular, for each $k \in \N$, $d(x_n, U^c_k)^{-1} \conv{n} d(x, U^c_k)^{-1}$. Since $d(x_n, U^c_k)^{-1} \conv{n} a_k$, we have that $a = d(x, U^c)^{-1}$. Therefore
		\begin{align*}
			y
			& = (x, a_1, a_2, \ldots) \\
			& = (x, d(x, U^c_1)^{-1}, d(x, U^c_2)^{-1}, \ldots) \\
			& = f(x) \\
			& \in f(U) \\
			& = C.
		\end{align*}
		Since $(y_n)_{n \in \N} \subset C$ and $y \in X \times \R^{\N}$ with $y_n \rightarrow y$ are arbitrary, we have that for each $(y_n)_{n \in \N} \subset C$ and $y \in X \times \R$, $y_n \rightarrow y$ implies that $y \in C$. Thus $C$ is closed in $X \times \R$. 
	\end{enumerate}
\end{proof}




























\subsection{Isometries}

\begin{defn} \ld{def:metric_spaces:introduction:0008.1}
	Let $(X, d_X)$ and $(Y, d_Y)$ be metric spaces and $f:X \rightarrow Y$. Then $f$ is said to be a \tbf{$(d_X, d_Y)$-isometry} if for each $x_1, x_2 \in X$, $d_Y(f(x_1), f(x_2)) = d_X(x_1, x_2)$. 
\end{defn}

\begin{ex} \lex{ex:metric_spaces:introduction:0008.2}
	Let $(X, d_X)$ and $(Y, d_Y)$ be metric spaces and $f:X \rightarrow Y$. If $f$ is a $(d_X, d_Y)$-isometry, then $f$ is injective.
\end{ex}

\begin{proof}
	Suppose that $f$ is an $(d_X, d_Y)$-isometry. Let $x_1. x_2 \in X$. Suppose that $f(x_1) = f(x_2)$. Then 
	\begin{align*}
		d_X(x_1, x_2)
		& = d_Y(f(x_1), f(x_2)) \\
		& = 0.
	\end{align*}
	Hence $x_1 = x_2$. Since $x_1. x_2 \in X$ are arbitrary, we have that for each $x_1. x_2 \in X$, $f(x_1) = f(x_2)$ implies that $x_1 = x_2$. Thus $f$ is injective. 
\end{proof}

\begin{defn} \ld{def:metric_spaces:introduction:0024.2}
	Let $X$ a set, $(Y, d)$ be a metric space and $f:X \rightarrow Y$ an injection. We define the \tbf{pullback of $d$ by $f$}, denoted $f^*d: X \times X \rightarrow \Rg$, by 
	$$f^*d(x_1, x_2) \defeq d(f(x_1), f(x_2))$$ 
\end{defn}

\begin{ex} \lex{ex:metric_spaces:introduction:0024.3}
	Let $X$ a set, $(Y, d)$ be a metric space and $f:X \rightarrow Y$ an injection. Then $f^*d$ is a metric on $X$.
\end{ex}

\begin{proof} 
	Let $x_1,x_2,x_3 \in X$. Then
	\begin{enumerate}
		\item
		\begin{align*}
			f^*d(x_1, x_2)
			& = d(f(x_1), f(x_2)) \\
			& = d(f(x_2), f(x_1)) \\
			& = f^*d(x_2, x_1)
		\end{align*}
		\item 
		\begin{align*}
			f^*d(x_1, x_2) = 0
			& \iff  d(f(x_1), f(x_2)) = 0 \\
			& \iff f(x_1) = f(x_2) \\
			& \iff x_1 = x_2
		\end{align*}
		\item 
		\begin{align*}
			f^*d(x_1, x_3) \\
			& = d(f(x_1), f(x_3)) \\
			& \leq d(f(x_1), f(x_2)) + d(f(x_2), f(x_3)) \\
			& = f^*d(x_1, x_2) + f^*d(x_2, x_3)
		\end{align*}
		So $f^*d$ is a metric on $X$.
	\end{enumerate}
\end{proof}

\begin{ex} \lex{ex:metric_spaces:introduction:0024.4}
	Let $X$ a set, $(Y, d)$ be a metric space and $f:X \rightarrow Y$ an injection. Then $f$ is an $(f^*d, d)$-isometry.
\end{ex}

\begin{proof}
	Let $x_1, x_2 \in X$. Then 
	\begin{align*}
		d(f(x_1), f(x_2))
		& = f^*d(x_1, x_2).
	\end{align*}
	Since $x_1, x_2 \in X$ are arbitrary, we have that $f$ is $(f^*d, d)$-isometry. 
\end{proof}

\begin{ex} \lex{ex:metric_spaces:introduction:0024.4}
	Let $(X, d_X), (Y, d_Y)$ be metric spaces and $f:X \rightarrow Y$. Then $f$ is a $(d_X, d_Y)$-isometry iff $f$ is injective and $d_X = f^*d_Y$.
\end{ex}

\begin{proof}\
	\begin{itemize}
		\item $(\implies):$ \\
		Suppose that $f$ is a $(d_X, d_Y)$-isometry. \rex{ex:metric_spaces:introduction:0008.2} implies that $f$ is injective. Then for each $x_1, x_2 \in X$, we have that
		\begin{align*}
			d_X(x_1, x_2)
			& = d_Y(f(x_1), f(x_2)) \\
			& = f^*d_Y(x_1, x_2)
		\end{align*}
		Since $x_1, x_2 \in X$ are arbitrary, we have that $d_X = f^*d_Y$. 
		\item $(\impliedby):$ \\
		Conversely, suppose that  $f$ is injective and $d_X = f^*d_Y$. \rex{ex:metric_spaces:introduction:0024.4} implies that $f$ is a $(f^*d_Y, d_Y)$-isometry. Since $d_X = f^*d_Y$, we have that $f$ is a $(d_X, d_Y)$-isometry.
	\end{itemize}
\end{proof}

\begin{ex} \lex{ex:metric_spaces:introduction:0024.5}
	Let $(X, d_X), (Y, d_Y)$ be metric spaces and $f:X \rightarrow Y$. Suppose that $f$ is a $(d_X, d_Y)$-isometry and $f$ is a bijection. Then $f^{-1}$ is a $(d_Y, d_X)$-isometry.
\end{ex}

\begin{proof}
	Since $f^{-1}$ is a bijection, $f^{-1}$ is injective. Let $y_1, y_2 \in Y$. \rex{ex:metric_spaces:introduction:0024.4} implies that $f^*d_Y = d_X$. Set $x_1 \defeq f^{-1}(y_1)$ and $x_2 \defeq f^{-1}(y_2)$. Then 
	\begin{align*}
		(f^{-1})^*d_X(y_1, y_2)
		& = d_X(f^{-1}(y_1), f^{-1}(y_2)) \\
		& = d_X(x_1, x_2) \\
		& = f^*d_Y(x_1, x_2) \\
		& = d_Y(f(x_1), f(x_2)) \\
		& = d_Y(y_1, y_2).
	\end{align*}
	Since $y_1, y_2 \in Y$ are arbitrary, we have that for each $y_1, y_2 \in Y$, $(f^{-1})^*d_X(y_1, y_2) = d_Y(y_1, y_2)$. Hence $d_Y = (f^{-1})^*d_X$. \rex{ex:metric_spaces:introduction:0024.4} then implies that $f^{-1}$ is a $(d_Y, d_X)$-isometry.
\end{proof}

\begin{ex} \lex{ex:metric_spaces:introduction:0024.6}
	Let $(X, \MT_X)$ and $(Y, \MT_Y)$ be topological spaces, $d_X: X \times X \rightarrow \Rg$ a metric on $X$ and $f:X \rightarrow Y$ a $(\MT_X, \MT_Y)$-homeomorphism. Then $\MT_Y = \MT_{(f^{-1})^*d_X}$.
\end{ex}

\begin{proof}\
		Set $d_Y \defeq (f^{-1})^*d_X$. Let $(y_n)_{n \in \N} \subset Y$ and $y \in Y$. Define $(x_n)_{n \in \N} \subset X$ and $x \in X$ by $x_n \defeq f^{-1}(y_n)$ and $x \defeq f^{-1}(y)$. 
		\begin{itemize}
			\item Suppose that $y_n \rightarrow y$ in $(Y, \MT_Y)$. Since $f^{-1}$ is a $(\MT_Y, \MT_X)$-homeomorphism, 
			\begin{align*}
				x_n
				& = f^{-1}(y_n) \\
				& \rightarrow f^{-1}(y) \\
				& = x
			\end{align*}
			in $(X, \MT_X)$. Since $\MT_X = \MT_{d_X}$, we have that
			\begin{align*}
				d_Y(y_n, y)
				& = (f^{-1})^*d_X(y_n, y) \\
				& = d_X(f^{-1}(y_n), f^{-1}(y)) \\
				& = d_X(x_n, x) \\
				& \rightarrow 0.
			\end{align*} 
			Thus $y_n \rightarrow y$ in $(X, \MT_{d_Y})$. 
			\item Conversely, suppose that $y_n \rightarrow y$ in $(Y, \MT_{d_Y})$. Then 
			\begin{align*}
				d_X(x_n, x)
				& = d_X(f^{-1}(y_n), f^{-1}(y)) \\
				& = (f^{-1})^*d_X(y_n, y) \\
				&= d_Y(y_n, y) \\
				& \rightarrow 0.
			\end{align*}
			Hence $x_n \rightarrow x$ in $(X, \MT_{d_X})$. Since $\MT_{d_X} = \MT_X$ and $f$ is a $(\MT_X, \MT_Y)$-homeomorphism, we have that 
			\begin{align*}
				y_n
				& = f(x_n) \\
				& \rightarrow f(x) \\
				& = y
			\end{align*}
			in $(Y, \MT_Y)$. 
	\end{itemize}
	Since $(y_n)_{n \in \N}$ and $y \in Y$ are arbitrary, we have that for each $(y_n)_{n \in \N}$ and $y \in Y$, $y_n \rightarrow y$ in $(Y, \MT_Y)$ iff $y_n \rightarrow y$ in $(Y, \MT_{d_Y})$. \rex{ex:nets:0021} implies that $\MT_Y = \MT_{d_Y}$. 
\end{proof}

































\vspace{3cm}
\subsection{Countability}


\begin{ex} \lex{ex:metric_spaces:introduction:0025}
	Let $(X, d)$ be a metric space. Then $(X, \MT_d)$ is second countable iff $(X, \MT_d)$ is separable.
\end{ex}

\begin{proof}\
	\begin{itemize}
		\item $(\implies): $ \\
		Suppose that $(X, \MT_d)$ is second countable. \rex{ex:topology:countability:00010} implies that $(X, \MT_d)$ is separable.
		\item $(\impliedby): $ \\
		Conversely, suppose that $(X, \MT_d)$ is separable. Then there exists $(x_n)_{n \in \N} \subset X$ such that $(x_n)_{n \in \N}$ is dense in $X$. Define $\MB \defeq \{B(x_n, q): (n, q) \in \N \times \Q \cap (0,1)\}$. Then $\MB$ is countable. 
		
		Let $U \in \MT_d$ and $x \in U$. Since $U$ is open, \rex{ex:metric_spaces:introduction:0012} implies that there exists $\del_0 > 0$ such that $B(x, \del_0) \subset U$. Set $\del \defeq \del_0/2$. Since $B(x, \del)$ is open, $B(x, \del) \neq \varnothing$ and $(x_n)_{n \in \N}$ is dense in $X$, \rex{31022} implies that there exists $n \in \N$ such that $x_n \in B(x, \del)$. Set $B \defeq B(x_n, \del)$. By construction $B \in \MB$ and $x \in B$. Let $y \in B$. Then 
		\begin{align*}
			d(x, y)
			& \leq d(x, x_n) + d(x_n, y) \\
			& < \del + \del \\
			& = 2 \del \\
			& = \del_0 
		\end{align*}
		Thus 
		\begin{align*}
			y 
			& \in B(x, \del_0) \\
			& \subset U
		\end{align*} 
		Since $y \in B$ is arbitrary, we have that $B \subset U$. Therefore, there exists $B \in \MB$ such that 
		\begin{align*}
			x 
			& \in B \\
			& \subset U.
		\end{align*}
		Hence $\MB$ is a basis for $\MT_d$. Since $\MB$ is countable, $(X, \MT_d)$ is second-countable.
	\end{itemize}
\end{proof}

\begin{ex} \lex{ex:metric_spaces:introduction:0026}
	Let $(X, d_X)$ and $(Y, d_Y)$ be metric spaces and $f:X \rightarrow Y$ a $(\MT_{d_X}, \MT_{d_Y})$-homeomorphism. Then $(X, \MT_{d_X})$ is separable iff $(Y, \MT_{d_Y})$ is separable. 
\end{ex}

\begin{proof}\
	\begin{itemize}
		\item $(\implies): $ \\
		Suppose that $(X, \MT_{d_X})$ is separable. \rex{ex:metric_spaces:introduction:0025} implies that $(X, \MT_{d_X})$ is second-countable. \rex{ex:topology:countability:00010.1} implies that $(Y, \MT_{d_Y})$ is second-countable. Another application of \rex{ex:metric_spaces:introduction:0025} implies that $(Y, \MT_{d_Y})$ is separable.
		\item $(\impliedby): $ \\
		Similar to $(\implies)$.
	\end{itemize}
\end{proof}

\begin{defn} \ld{def:metric_spaces:introduction:0027}\
	\begin{itemize}
		\item for $n \in \N$ and $\gam \in \N^n$, we define the \tbf{length} of $\gam$, denoted $|\gam|$, by $|\gam| \defeq n$,
		\item for $n \in \N$, we define the \tbf{projection of $\N^{[n+1]}$ onto $\N^n$}, denoted $\pi^{[n+1]}_{[n]}: \N^{n+1} \rightarrow \N^n$, by $\pi^{[n+1]}_{[n]}(\gam_1, \ldots, \gam_{n+1}) \defeq (\gam_1, \ldots, \gam_{n})$.
	\end{itemize}
\end{defn}

\begin{ex} \lex{ex:metric_spaces:introduction:0028}
	Let $(X,d)$ be a metric space. Suppose that $(X, \MT_d)$ is separable and $X \neq \varnothing$. Define $\Gam \defeq \bigcup\limits_{n \in \N} \N^n$. Then there exists $(C_{\gam})_{\gam \in \Gam} \subset \MP(X)$ such that 
	\begin{enumerate}
		\item for each $\gam \in \Gam$,
		\begin{enumerate}
			\item $C_{\gam}$ is closed and $C_{\gam} \neq \varnothing$,
			\item $\diam C_{\gam} \leq |\gam|^{-1}$,
			\item for each $n \in \N$, $|\gam| = n$ implies that $C_{\gam} = \bigcup\limits_{\gam' \in (\pi^{[n+1]}_{[n]})^{-1}(\{\gam\})} C_{\gam'}$
		\end{enumerate}
		\item $X = \bigcup\limits_{|\gam| = 1} C_{\gam}$
	\end{enumerate} 
\end{ex}

\begin{proof}
	Since $X$ is separable, there exists $(x^0_j)_{j \in \N} \subset X$ such that $(x^0_j)_{j \in \N}$ is dense in $X$. We define $(C_{\gam})_{\gam \in \Gam}$ inductively as follows:
	\begin{itemize}
		\item For $\gam \in \N$, we define $C_{\gam} \defeq \bar{B}(x^0_{\gam}, 2^{-1})$. 
		\item Let $n \geq 2$ and $\gam \in \N^n$. Define $\tl{\gam} \in \N^{n-1}$ by $\tl{\gam} \defeq \pi^{[n]}_{[n-1]}(\gam)$. Since $(x^0_j)_{j \in \N}$ is dense in $X$ and $C_{\tl{\gam}} \neq \varnothing$, we have that $\# \{j \in \N: \bar{B}(x^0_j, (2n)^{-1}) \cap C_{\tl{\gam}} \neq \varnothing\} = \infty$. Define $(x^{\tl{\gam}}_j)_{j \in \N} \subset X$ to be an enumeration of $\{x^0_j: j \in \N \text{ and } \bar{B}(x^0_j, (2n)^{-1}) \cap C_{\tl{\gam}} \neq \varnothing\}$. Define $C_{\gam} \subset C_{\tl{\gam}}$ by $C_{\gam} \defeq \bar{B}(x^{\tl{\gam}}_{\gam_n}, (2n)^{-1}) \cap C_{\tl{\gam}}$.
	\end{itemize}
	\begin{enumerate}
		\item Let $\gam \in \Gam$. 
		\begin{enumerate}
			\item By construction, $C_{\gam}$ is closed and $C_{\gam} \neq \varnothing$.
			\item Set $n \defeq |\gam|$.
			\begin{itemize}
				\item If $n = 1$, then 
				\begin{align*}
					\diam C_{\gam}
					& = \diam \bar{B}(x^0_{\gam_1}, 2^{-1}) \\
					& \leq 1 \\
					& = |\gam|^{-1}.
				\end{align*}
				\item Suppose that $n \geq 2$. Set $\tl{\gam} \defeq \pi^{[n]}_{[n-1]}(\gam)$. Then
				\begin{align*}
					\diam C_{\gam}
					& = \diam \bar{B}(x^{\tl{\gam}}_{\gam_n}, (2n)^{-1}) \cap C_{\tl{\gam}} \\
					& \leq \diam \bar{B}(x^{\tl{\gam}}_{\gam_n}, (2n)^{-1}) \\
					& \leq n^{-1} \\
					& =|\gam|^{-1}.
				\end{align*}
			\end{itemize}
			\item Let $n \in \N$. Suppose that $|\gam| = n$.
			\begin{itemize}
				\item By construction, for each $\gam' \in (\pi^{[n+1]}_{[n]})^{-1}(\{\gam\})$, $C_{\gam'} \subset C_{\gam}$. Hence  
				\begin{align*}
					\bigcup\limits_{\gam' \in (\pi^{[n+1]}_{[n]})^{-1}(\{\gam\})} C_{\gam'}
					& \subset C_{\gam}.
				\end{align*}
				Let $x \in C_{\gam}$. Since $(x^0_j)_{j \in \N}$ is dense in $X$, there exists $k \in \N$ such that $d(x_k^0, x) \leq [2(n+1)]^{-1}$. Thus  
				\begin{align*}
					\varnothing
					& \neq \{x\} \\
					& \subset \bar{B}(x^0_k, [2(n+1)]^{-1}) \cap C_{\gam}
				\end{align*}
				and by construction, there exists $j \in \N$ such that $x^{\gam}_j = x^0_k$. Define $\gam' \in (\pi^{[n+1]}_{[n]})^{-1}(\{\gam\})$ by $\gam' \defeq (\gam_1, \ldots, \gam_n, j)$. Then 
				\begin{align*}
					x
					& \in \bar{B}(x^{\gam}_j, [2(n+1)]^{-1}) \cap C_{\gam} \\
					& = \bar{B}(x^{\gam}_{\gam'_{n+1}}, [2(n+1)]^{-1}) \cap C_{\gam} \\
					& = C_{\gam'} \\
					& \subset \bigcup\limits_{\gam' \in (\pi^{[n+1]}_{[n]})^{-1}(\{\gam\})} C_{\gam'}
				\end{align*}
				Since $x \in C_{\gam}$ is arbitrary, we have that for each $x \in C_{\gam}$, $x \in \bigcup\limits_{\gam' \in (\pi^{[n+1]}_{[n]})^{-1}(\{\gam\})} C_{\gam'}$. Hence $C_{\gam} \subset \bigcup\limits_{\gam' \in (\pi^{[n+1]}_{[n]})^{-1}(\{\gam\})} C_{\gam'}$. Since $\bigcup\limits_{\gam' \in (\pi^{[n+1]}_{[n]})^{-1}(\{\gam\})} C_{\gam'}
				\subset C_{\gam}$ and $C_{\gam} \subset \bigcup\limits_{\gam' \in (\pi^{[n+1]}_{[n]})^{-1}(\{\gam\})} C_{\gam'}$, we have that $C_{\gam} = \bigcup\limits_{\gam' \in (\pi^{[n+1]}_{[n]})^{-1}(\{\gam\})} C_{\gam'}$.
			\end{itemize}
			\end{enumerate}
			\item By construction, since $(x^0_j)_{j \in \N}$ is dense in $X$, 
			\begin{align*}
				X 
				& = \bigcup\limits_{j \in \N} \bar{B}(x^0_j, 2^{-1}) \\
				& = \bigcup\limits_{|\gam| = 1} \bar{B}(x^0_{\gam_1}, 2^{-1}) \\
				& = \bigcup\limits_{|\gam| = 1} C_{\gam}.
			\end{align*}
	\end{enumerate}
\end{proof}


































































\newpage
\section{$\Top$-Equivalent Metrics}

\begin{defn} \ld{def:metric_spaces:topologically_equivalent_metrics:0001}
	Let $X$ be a set, $d_1, d_2: X \times X \rightarrow \Rg$ metrics on $X$. Then $d_1$ and $d_2$ are said to be  \tbf{$\Top$-equivalent}, denoted $d_1 \sim_{\Top} d_2$, if $\MT_{d_1} = \MT_{d_2}$.
\end{defn}	

\begin{ex} \lex{ex:metric_spaces:topologically_equivalent_metrics:0002}
	Let $X$ be a set, $d_1, d_2: X \times X \rightarrow \Rg$ metrics on $X$. Then $d_1$ and $d_2$ are $\Top$-equivalent iff for each $(x_n)_{n \in \N} \subset X$ and $x \in X$, $x_n \conv{d_1} x$ iff $x_n \conv{d_2} x$.	
\end{ex}

\begin{proof}
	\tcr{FINISH!!!}
\end{proof}

\begin{defn} \ld{def:metric_spaces:topologically_equivalent_metrics:0003}
	Let $\phi: \Rg \rightarrow \Rg$. Then $\phi$ is said to be \tbf{$\Top$ metric-preserving} if for each set $X$ and metric $d$ on $X$, 
	\begin{enumerate}
		\item $\phi \circ d$ is a metric on $X$
		\item $\phi \circ d \sim_{\Top} d$
	\end{enumerate} 
\end{defn}

\begin{defn} \ld{def:metric_spaces:topologically_equivalent_metrics:0004}
	Let $(X, d)$ be a metric space and $\phi: \Rg \rightarrow \Rg$. Suppose that $\phi$ is said to be $\Top$ metric-preserving. We define the \tbf{$\phi$-iterate of $d$}, denoted $d^{\phi}$, by $d^{\phi} = \phi \circ d$.  
\end{defn}

\begin{ex} \lex{ex:metric_spaces:topologically_equivalent_metrics:0005}
	Let $\phi: \Rg \rightarrow \Rg$. Suppose that  
	\begin{enumerate}
		\item $\phi$ is continuous
		\item $\phi$ is increasing
		\item $\phi^{-1}(\{0\}) = \{0\}$ 
	\end{enumerate}
	Then for each $(s_n)_{n \in \N} \subset \Rg$, $s_n \rightarrow 0$ iff $\phi(s_n) \rightarrow 0$. 
\end{ex}

\begin{proof}
	Let $(s_n)_{n \in \N} \subset \Rg$. Suppose that $s_n \rightarrow 0$. Since $\phi$ is continuous, 
	\begin{align*}
		\phi(s_n) 
		& \rightarrow \phi(0) \\
		& = 0
	\end{align*}
	Conversely, suppose that $\phi(s_n) \rightarrow 0$. For the sake of contradiction, suppose that $s_n \not \rightarrow 0$. Then there exists $\ep >0$ and a subsequence $(s_{n_k})_{k \in \N} \subset (s_n)_{n \in \N}$ such that $(s_{n_k})_{k \in \N} \subset B(0, \ep)^c$. Since $\phi^{-1}(\{0\}) = \{0\}$, for each $k \in \N$, $\phi(s_{n_k}) >0$. Since $\phi(s_{n_k}) \rightarrow 0$, there exists a subsequence $(s_{n_{k_j}})_{j \in \N} \subset (s_{n_k})_{k \in \N}$ such that for each $j \in \N$, $\phi(s_{n_{k_{j+1}}}) < \phi(s_{n_{k_j}})$. Define $(t_j)_{j \in \N} \subset B(0,\ep)^c$ by $t_j = s_{n_{k_j}}$. For the sake of contradiction, suppose that there exists $j \in \N$ such that $t_j \leq t_{j+1}$. Since $\phi$ is increasing, $\phi(t_j) \leq \phi(t_{j+1})$. This is a contradiction since by construction, $\phi(t_{j+1}) < \phi(t_j)$. Therefore for each $j \in \N$, $t_{j+1} < t_j$. Hence $(t_j)_{j \in \N}$ is decreasing and $t_j \rightarrow \inf_{j \in \N} t_j$. Set $t = \inf_{j \in \N} t_j$. Since $(t_j)_{j \in \N} \subset B(0, \ep)^c$, $t \in B(0, \ep)^c$. Since $t \neq 0$ and $\phi^{-1}(\{0\}) = \{0\}$, we have that $\phi(t) \neq 0$. Since $\phi$ is continuous, $\phi(t_j) \rightarrow \phi(t)$. By construction $\phi(t_j) \rightarrow 0$. Hence $\phi(t) = 0$. This is a contradiction. Hence $s_n \rightarrow 0$.
\end{proof}

\begin{ex} \lex{ex:metric_spaces:topologically_equivalent_metrics:0006}
	Let $\phi: \Rg \rightarrow \Rg$. Suppose that
	\begin{enumerate}
		\item $\phi$ is continuous
		\item $\phi$ is increasing
		\item $\phi$ is subadditive
		\item $\phi^{-1}(\{0\}) = \{0\}$
	\end{enumerate}
	Then $\phi$ is $\Top$ metric-preserving.
\end{ex}

\begin{proof}
	Let $(X, d)$ be a metric space.
	\begin{enumerate}
		\item
		\begin{enumerate}
			\item Let $x, y \in X$. Suppose that $x = y$. Then $d(x,y) = 0$. Since $0 \in \phi^{-1}(\{0\})$, we have that
			\begin{align*}
				d^{\phi} (x,y)
				& = \phi(d(x,y)) \\
				& = \phi(0) \\
				& = 0
			\end{align*}
			Conversely, suppose that $ d^{\phi}(x,y) = 0$. Then $\phi(d(x,y)) = 0$ and therefore
			\begin{align*}
				d(x,y)
				& \in \phi^{-1}(\{0\}) \\
				& = \{0\}
			\end{align*}
			Thus $d(x,y) = 0$. Since $d$ is a metric on $X$, $x = y$. Hence $d^{\phi}(x,y) = 0$ iff $x = y$.
			\item Let $x,y,z \in X$. Since $\phi$ is increasing and subadditive, we have that
			\begin{align*}
				d_{\phi(x,z)}
				& = \phi(d(x,z)) \\
				& \leq \phi(d(x,y) + d(y,z)) \\
				& \leq \phi(d(x,y)) + \phi(d(y,z)) \\
				& = d^{\phi}(x,y) + d^{\phi}(y,z)
			\end{align*}
		\end{enumerate}
		Therefore $d^{\phi}$ is a metric on $X$. 
		\item Let $(x_n)_{n \in \N} \subset X$ and $x \in X$. Suppose that $x_n \conv{d} x$. Then $d(x_n, x) \rightarrow 0$. Since $\phi$ is continuous, 
		\begin{align*}
			d^{\phi}(x_n, x)
			& = \phi(d(x_n,x)) \\
			& \rightarrow 0 
		\end{align*}
		So $x_n \conv{d^{\phi}} x$. \\
		Conversely, suppose that $x_n \conv{d^{\phi}} x$. Then 
		\begin{align*}
			\phi(d(x_n,x))
			& = d^{\phi}(x_n, x) \\
			& \rightarrow 0
		\end{align*}
		\tcb{The previous exercise} implies that $d(x_n,x) \rightarrow 0$. Hence $x_n \conv{d} x$. Since $(x_n)_{n \in \N} \subset X$ and $x \in X$ are arbitrary, we have that $(x_n)_{n \in \N} \subset X$ and $x \in X$, $x_n \conv{d} x$ iff $x_n \conv{d^{\phi}} x$. Therefore $d^{\phi} \sim_{\Top} d$. 
	\end{enumerate}
	Since $(X, d)$ is arbitrary, $\phi$ is $\Top$ metric-preserving.
\end{proof}

\begin{ex} \lex{ex:metric_spaces:topologically_equivalent_metrics:0007}
	Define $\phi:\Rg \rightarrow [0,1)$ by 
	$$\phi(t) = \frac{t}{1+t}$$
	Then $\phi$ is $\Top$ metric-preserving.
\end{ex}

\begin{proof}\
	\begin{enumerate}
		\item We note that $\phi \in C^{\infty}(\Rg)$ and for each $t \in \Rg$, 
		$$\phi'(t) = \frac{1}{(1+t)^2} \quad \text{and} \quad \phi''(t) = -\frac{2}{(1+t)^3}$$
		In particular, $\phi$ is continuous.
		\item Since $\phi' > 0$, $\phi$ is strictly increasing. 
		\item Since $\phi'' < 0$, $\phi$ is strictly concave. Since $\phi(0) = 0$, an exercise in the section on convex functions implies that $\phi$ is subadditive. \tcb{reference section on convex functions}
		\item Clearly $\phi^{-1}(\{0\}) = \{0\}$.
	\end{enumerate}	
	So $\phi$ is $\Top$ metric-preserving.
\end{proof}

\begin{ex} \lex{ex:metric_spaces:topologically_equivalent_metrics:0008}
	Let $a \in (0, \infty)$. Define $\phi_a:\Rg \rightarrow \Rg$ by 
	$$\phi_a(t) = t \wedge a$$
	Then $\phi_a$ is $\Top$ metric-preserving.
\end{ex}

\begin{proof}\
	\begin{enumerate}
		\item Clearly $\phi$ is continuous.
		\item Clearly, $\phi$ is increasing. 
		\item Since the minimum of two concave functions is concave, $\phi$ is concave. Since $\phi(0) = 0$, an exercise in the section on convex functions implies that $\phi$ is subadditive. \tcb{reference section on convex functions}
		\item Clearly $\phi^{-1}(\{0\}) = \{0\}$.	
	\end{enumerate}	
	So $\phi_a$ is $\Top$ metric-preserving.
\end{proof}
















































\newpage
\section{Subspaces}

\subsection{Introduction}

\begin{ex} \lex{ex:metric_spaces:subspaces:0001}
	Let $(X, d)$ be a metric space. Then $\MT_{d|_{E \times E}} = \MT_d \cap E$. 
\end{ex}

\begin{proof}Set $d_E \defeq d|_{E \times E}$. Let $(x_n)_{n \in \N} \subset E$ and $x \in E$.
	\begin{itemize}
		\item Suppose that $x_n \rightarrow x$ in $(E, \MT_{d_E})$. Then $d_E(x_n, x) \rightarrow 0$. Hence  
		\begin{align*}
			d(x_n, x)
			& = d_E(x_n, x) \\
			& \rightarrow 0. 
		\end{align*}
		Therefore $x_n \rightarrow x$ in $(X, \MT_d)$. \rex{ex:topology:subspaces:0005} implies that $x_n \rightarrow x$ in $(E, \MT_d \cap E)$. 
		\item Conversely, suppose that $x_n \rightarrow x$ in $(E, \MT_d \cap E)$. \rex{ex:topology:subspaces:0005} implies that $x_n \rightarrow x$ in $(X, \MT_d)$. Then $d(x_n, x) \rightarrow 0$. Hence  
		\begin{align*}
			d_E(x_n, x)
			& = d(x_n, x) \\
			& \rightarrow 0. 
		\end{align*}
		Therefore $x_n \rightarrow x$ in $(E, \MT_{d_E})$. 
	\end{itemize}
	Hence $x_n \rightarrow x$ in $(E, \MT_{d_E})$ iff  $x_n \rightarrow x$ in $(E, \MT_d \cap E)$. \rex{ex:nets:0021} implies that $\MT_d \cap E = \MT_{d_E}$.
\end{proof}

\begin{ex} \lex{ex:metric_spaces:subspaces:0001.1}
	Let $(X, d)$ be a metric space and $E \subset X$. Set $d_E \defeq d|_{E^2}$. Then for each $x \in E$ and $r > 0$,
	\begin{enumerate}
		\item $B_{d_E}(x, r) = B_d(x, r) \cap E$,
		\item $\bar{B}_{d_E}(x, r) = \bar{B}_d(x, r) \cap E$.
	\end{enumerate}
\end{ex}

\begin{proof}
	Let $x \in E$ and $r >0$.
	\begin{enumerate}
		\item 
		\begin{itemize}
			\item Let $y \in B_{d_E}(x, r)$. Then $y \in E$ and 
			\begin{align*}
				d(x, y)
				& = d_E(x, y) \\
				& < r.
			\end{align*}
			Thus $y \in B_d(x, r)$. Therefore $y \in B_d(x, r) \cap E$. Since $y \in B_{d_E}(x, r)$ is arbitrary, we have that for each $y \in B_{d_E}(x, r)$, $y \in B_d(x, r) \cap E$. Hence $B_{d_E}(x, r) \subset B_d(x, r) \cap E$. 
			\item Let $y \in B_d(x, r) \cap E$. Then 
			\begin{align*}
				d_E(x, y)
				& = d(x, y) \\
				& < r.
			\end{align*}
			Hence $y \in B_{d_E}(x, r)$. Since $y \in B_d(x, r) \cap E$ is arbitrary, we have that for each $y \in B_d(x, r) \cap E$, $y \in B_{d_E}(x, r)$. Hence $B_d(x, r) \cap E \subset B_{d_E}(x, r)$. 
		\end{itemize} 
		\item Similar to $(1)$.
	\end{enumerate}
\end{proof}


\begin{ex} \lex{ex:metric_spaces:subspaces:0002}
	Let $(X, d)$ be a metric space and $E \subset X$. If $(X, \MT_d)$ is separable, then $(E, \MT_d \cap E)$ is separable.  
\end{ex}

\begin{proof}
	Suppose that $(X, \MT_d)$ is separable. \rex{ex:metric_spaces:introduction:0025} implies that $(X, \MT_d)$ is second countable. \rex{ex:topology:countability:0013} implies that $(E, \MT_d \cap E)$ is second-countable. Another application of \rex{ex:metric_spaces:introduction:0025} implies that $(E, \MT_d \cap E)$ is separable. 
\end{proof}















\vspace{4cm}

\subsection{Discrete Subsets}

\begin{defn}
	Let $(X,d)$ be a metric space and $S \subset X$. Then $S$ is said to be \tbf{discrete} if for each $x \in X$, there exists $r > 0$ such that $B(x, r) \cap X = \{x\}$. 
\end{defn}

\begin{ex}
	Let $(X, d)$ be a metric space and $S \subset X$. Then $S$ is discrete iff $\MT_d \cap S = \MT_{dscrt(X)}$.   
\end{ex}

\begin{proof}
	\tcb{FINISH!!!}
\end{proof}

\begin{ex} \tcr{compare with discrete sets in topology section}
	Let $(X, d)$ be a metric space, $A \subset X$ and $x \in A$. Then $x$ is an isolated point of $A$ iff there exists $r > 0$ such that $B(x, r) \cap A = \{x\}$. 
\end{ex}

\begin{proof}\
	\begin{itemize}
		\item $(\implies):$ \\
		Suppose that $x$ is an isolated point of $A$. Then there exists $U \subset X$ such that $U$ is open in $X$ and $U \cap A = \{x\}$. Since $U$ is open, $x \in U$ and $\{B(x, r): r > 0\}$ is a local basis for the topology on $X$ at $x$, there exists $r > 0$ such that $B(x, r) \subset U$. Hence 
		\begin{align*}
			B(x, r) \cap A 
			& \subset U \cap A \\
			& = \{x\}
		\end{align*}
		Since $x \in B(x, r) \cap A$, we have that $\{x\} \subset B(x, r) \cap A$. Hence $B(x, r) \cap A = \{x\}$. \\
		\item $(\impliedby):$ \\
		Conversely, suppose that there exists $r > 0$ such that $B(x, r) \cap A = \{x\}$. Since $B(x, r)$ is open in $X$, $x$ is an isolated point of $A$. 
	\end{itemize}
\end{proof}

\begin{ex}
	Let $(X, d)$ be a metric space and $A \subset X$. Suppose that $A$ is discrete. If $X$ is separable, then $A$ is countable. \\
	\tbf{Hint:} If $E \subset X$ is countable and dense in $X$, then for each $x \in A$, there exists $y \in E$ and $q \in \Q \cap \Rgp$ such that $x \in B(y, q)$.
\end{ex}

\begin{proof}
	Suppose that $X$ is separable. Let $x \in A$. Since $X$ is separable, there exists $(x_n)_{n \in \N} \subset X$ such that $(x_n)_{n \in \N}$ is dense in $X$. Since $A$ is discrete, $x$ is an isolated point of $A$ and \tcb{the previous exercise} implies that there exists $r > 0$ such that $B(x, r) \cap A = \{x\}$. Choose $q \in \Q \cap (0,r)$. Set $\ep = \min(r -q, q)$. Then $\ep > 0$. Since $(x_n)_{n \in \N}$ is dense in $X$, there exists $N \in \N$ such that $d(x_N, x) < \ep$. Let $y \in B(x_N, q)$. Then
	\begin{align*}
		d(y, x)
		& \leq d(y, x_N) + d(x_N, x) \\
		& < q  + \ep \\
		& \leq q + (r - q) \\
		& = r
	\end{align*}
	Thus $y \in B(x, r)$. Since $y \in B(x_N, q)$ is arbitrary, we have that $B(x_N, q) \subset B(x, r)$. In addition, 
	\begin{align*}
		d(x_N, x)
		& < \ep \\
		& \leq q 
	\end{align*}
	which implies that $x \in B(x_N, q)$ and therefore $\{x\} \subset B(x_N, q) \cap A$. Conversely, 
	\begin{align*}
		B(x_N, q) \cap A 
		& \subset B(x, r) \cap A \\
		& = \{x\}
	\end{align*}
	Hence $B(x_N, q) \cap A = \{x\}$. Since $x \in A$ is arbitrary, we have that for each $x \in A$, 
	$$\{B(x_n, q):  (n, q) \in \N \times (\Q \cap \Rgp) \text{ and } B(x_n, q) \cap A = \{x\}\} \neq \varnothing$$ 
	For each $x \in A$, define 
	$$V(x) \defeq \{B(x_n, q):  (n, q) \in \N \times (\Q \cap \Rgp) \text{ and } B(x_n, q) \cap A = \{x\}\}$$ 
	Define 
	$$V \defeq \{B(x_n, q):  (n, q) \in \N \times (\Q \cap \Rgp) \}$$
	Since $\bigcup\limits_{x \in A} V(x) \subset V$ and $V$ is countable, we have that $\bigcup\limits_{x \in A} V(x)$ is countable. The axiom of choice implies that there exists $\phi: A \rightarrow \bigcup\limits_{x \in A} V(x)$ such that for each $x \in A$, $\phi(x) \in V(x)$. Let $x,y \in A$. Suppose that $\phi(x) = \phi(y)$. By construction,
	\begin{align*}
		\{x\}
		& = \phi(x) \cap A \\
		& = \phi(y) \cap A \\
		& = \{y\}
	\end{align*}
	Hence $x = y$. Since $x,y \in A$ are arbitrary, $\phi$ is injective. Since $\phi: A \rightarrow \bigcup\limits_{x \in A} V(x)$ is injective, and $\bigcup\limits_{x \in A} V(x)$ is countable, we have that $A$ is countable. 
\end{proof}





















































\newpage
\section{Product Spaces}

\begin{defn} \ld{def:metric_spaces:product_spaces:0001}
	Let $(X_n, d_n)_{n \in \N}$ be a collection of metric spaces. Set $X \defeq \prod\limits_{n \in \N} X_n$. Define $\phi:\Rg \rightarrow [0,1)$ by $\phi(t) \defeq t/(1 + t)$. We define the \tbf{product metric on $X$}, denoted $d_X: X \times X \rightarrow \Rg$ by 
	$$d_X((x_n)_{n \in \N}, (y_n)_{n \in \N}) = \sum_{n \in \N} 2^{-n} d_n^{\phi}(x_n, y_n)$$ 
\end{defn}

\begin{ex} \lex{ex:metric_spaces:product_spaces:0002}
	Let $(X_n, d_n)_{n \in \N}$ be a collection of metric spaces. Set $X \defeq \prod\limits_{n \in \N} X_n$ and $\MT \defeq \bigotimes\limits_{n \in \N} \MT_{d_n}$.
	Then 
	\begin{enumerate}
		\item $d_X$ is a metric on $X$,
		\item $\MT_{d_X} = \MT$.
	\end{enumerate} 
\end{ex}

\begin{proof}\
	\begin{enumerate}
		\item Let $(x_n)_{n \in \N}, (y_n)_{n \in \N}, (z_n)_{n \in \N} \in X$. 
		\begin{enumerate}
			\item 
			\begin{align*}
				d_X((x_n)_{n \in \N}, (y_n)_{n \in \N})
				& = \sum_{n \in \N} 2^{-n} d_n^{\phi}(x_n, y_n) \\
				& = \sum_{n \in \N} 2^{-n} d_n^{\phi}(y_n, x_n) \\
				& = d_X((y_n)_{n \in \N}, (x_n)_{n \in \N})
			\end{align*}
			\item 
			\begin{itemize}
				\item Suppose that $(x_n)_{n \in \N} = (y_n)_{n \in \N}$. Then for each $n \in \N$, $x_n = y_n$. Thus for each $n \in \N$, $d_n^{\phi}(x_n, y_n) = 0$. Hence
				\begin{align*}
					d_X((x_n)_{n \in \N}, (y_n)_{n \in \N})
					& = \sum_{n \in \N} 2^{-n} d_n^{\phi}(x_n, y_n) \\
					& = 0
				\end{align*}
				\item Suppose that $d_X((x_n)_{n \in \N}, (y_n)_{n \in \N}) = 0$. Then for each $n \in \N$, $d_n^{\phi}(x_n, y_n) = 0$. Therefore for each $n \in \N$, $x_n = y_n$. Hence $(x_n)_{n \in \N} = (y_n)_{n \in \N}$.
			\end{itemize}
			Therefore $d_X((x_n)_{n \in \N}, (y_n)_{n \in \N}) = 0$ iff $(x_n)_{n \in \N} = (y_n)_{n \in \N}$.
			\item 
			\begin{align*}
				d_X((x_n)_{n \in \N}, (y_n)_{n \in \N})
				& = \sum_{n \in \N} 2^{-n} d_n^{\phi}(x_n, y_n) \\
				& \leq \sum_{n \in \N} 2^{-n} [d_n^{\phi}(x_n, z_n) + d_n^{\phi}(z_n, y_n)] \\
				& = \sum_{n \in \N} 2^{-n} d_n^{\phi}(x_n, z_n) + \sum_{n \in \N} 2^{-n} d_n^{\phi}(z_n, y_n) \\
				& = d_X((x_n)_{n \in \N}, (z_n)_{n \in \N}) + d_X((z_n)_{n \in \N}, (y_n)_{n \in \N})
			\end{align*}
		\end{enumerate}
		So $d_X$ is a metric. 
		\item Let $(a_m)_{m \in \N} \subset X$ and $a \in X$. Then for each $m \in \N$, there exist $(x_{m,n})_{n \in \N} \in X$ such that $a_m = (x_{m,n})$ and there exists $(x_n)_{n \in \N} \in X$ such that $a = (x_n)_{n \in \N}$.
		\begin{itemize}
			\item Suppose that $a_m \rightarrow a$ in $(X, \MT)$. Let $\ep > 0$. Choose $N_0 \in \N$ such that $\sum\limits_{n \geq N_0 + 1} 2^{-n} < \ep/2$. Since $a_m \rightarrow a$ in $(X, \MT)$, we have that for each $n \in \N$, $x_{m, n} \conv{m} x_n$ in $(X_n, \MT_{d_n})$. Hence for each $n \in \N$, $d_n(x_{m,n}, x_n) \conv{m} 0$. \rex{ex:metric_spaces:topologically_equivalent_metrics:0007} implies that for each $n \in \N$, $d_n^{\phi}(x_{m,n}, x_n) \conv{m} 0$. Let $n \in [N_0]$. Since $d_n^{\phi}(x_{m,n}, x_n) \conv{m} 0$, there exists $M_n \in \N$ such that for each $m \in \N$, $m \geq M_n$ implies that $d_n^{\phi}(x_{m,n}, x_n) < \ep/2$. Set $M \defeq \max(M_1, \ldots, M_{N_0})$. Let $m \in \N$. Suppose that $m \geq M$. Then 
			\begin{align*}
				d_X(a_m, a)
				& = \sum_{n \in \N} 2^{-n} d_n^{\phi}(x_{m,n}, x_n) \\
				& = \sum_{n = 1}^{N_0} 2^{-n} d_n^{\phi}(x_{m,n}, x_n) + \sum_{n \geq N_0 + 1} 2^{-n} d_n^{\phi}(x_{m,n}, x_n) \\
				& < \sum_{n = 1}^{N_0} 2^{-n} \bigg( \frac{\ep}{2} \bigg) + \sum_{n \geq N_0 + 1} 2^{-n} \\
				& = \frac{\ep}{2} \sum_{n = 1}^{N_0} 2^{-n}  + \sum_{n \geq N_0 + 1} 2^{-n} \\
				& < \frac{\ep}{2} + \frac{\ep}{2} \\
				& = \ep 
			\end{align*}
			Since $\ep > 0$ is arbitrary, we have that for each $\ep > 0$, there exists $M \in \N$ such that for each $m \in \N$, $m \geq M$ implies that $d_X(a_m, a) < \ep$. Hence $d_X(a_m, a) \rightarrow 0$ and therefore $a_m \rightarrow a$ in $(X, \MT_{d_X})$.  
			\item Conversely, suppose that $a_m \rightarrow a$ in $(X, \MT_{d_X})$. Then $d_X(a_m, a) \rightarrow 0$. Let $n \in \N$. Then 
			\begin{align*}
				2^{-n} d_n^{\phi}(x_{m,n}, x_n)
				& \leq d_X(a_m, a) \\
				& \rightarrow 0
			\end{align*}
			Hence $d_n^{\phi}(x_{m,n}, x_n) \rightarrow 0$. Thus $x_{m,n} \conv{m} x_n$ in $(X, \MT_{d_n^{\phi}})$. Since $\MT_{d_n^{\phi}} = \MT_{d_n}$, we have that $x_{m,n} \conv{m} x_n$ in $(X, \MT_{d_n})$. Since $n \in \N$ is arbitrary, we have that $a_m \rightarrow a$ in $(X, \MT)$. 
		\end{itemize} 
		Since $(a_m)_{m \in \N} \subset X$ and $a \in X$ are arbitrary, we have that for each $(a_m)_{m \in \N} \subset X$ and $a \in X$, $a_m \rightarrow a$ in $(X, \MT)$ iff $a_m \rightarrow a$ in $(X, \MT_{d_X})$. \rex{ex:nets:0021} implies that $\MT_{d_X} = \MT$.
	\end{enumerate}
\end{proof}


\begin{note}
	\tcr{Thus the product topolgy is basically the topology of waning importance.} More importance is given to earlier entries than later entries in a point of the product space. \tcr{should have another example of a metric compatible with the product, maybe the supremum/n one} 
\end{note}
















































\newpage
\section{Coproduct Spaces}

\begin{defn} \ld{def:metric_spaces:coproducts:0001}
	Let $(X_{\al}, d_{\al})_{\al \in A}$ be a collection of metric space. Set $X \defeq \coprod\limits_{n \in \N} X_n$. Define $\phi:\Rg \rightarrow [0,1)$ by $\phi(t) \defeq t/(1 + t)$. We define the \tbf{coproduct metric on $X$}, denoted $d_X: X \times X \rightarrow \Rg$, by  
	\[
	d_X((\al, x), (\be, y)) \defeq
	\begin{cases}
		d_{\al}^{\phi}(x, y), & \al = \be \\
		1, & \al \neq \be 
	\end{cases}
	\]
\end{defn}

\begin{ex} \lex{ex:metric_spaces:coproducts:0002}
	Let $(X_{\al}, d_{\al})_{\al \in A}$ be a collection of metric space. Set $X \defeq \coprod\limits_{\al \in A} X_{\al}$ and $\MT \defeq \bigotimes\limits_{\al \in A} \MT_{d_{\al}}$. Then 
	\begin{enumerate}
		\item $d_X$ is a metric on $X$,
		\item $\MT_{d_X} = \MT$. 
	\end{enumerate}
\end{ex}

\begin{proof}\
	\begin{enumerate}
		\item Let $(\al_1, x_1), (\al_2, x_2), (\al_3, x_3) \in X$. 
		\begin{enumerate}
			\item 
			\begin{itemize}
				\item Suppose that $\al_1 = \al_2$. Then 
				\begin{align*}
					d_X((\al_1, x_1), (\al_2, x_2))
					& = d_{\al_1}^{\phi}(x_1, x_2) \\
					& = d_{\al_1}^{\phi}(x_2, x_1) \\
					& = d_X((\al_2, x_2), (\al_1, x_1)).
				\end{align*}
				\item Suppose that $\al_1 \neq \al_2$. Then 
				\begin{align*}
					d_X((\al_1, x_1), (\al_2, x_2))
					& = 1 \\
					& = d_X((\al_2, x_2), (\al_1, x_1)).
				\end{align*}
			\end{itemize}
			\item 
			\begin{itemize}
				\item If $(\al_1, x_1) = (\al_2, x_2)$, then $\al_1 = \al_2$ and $x_1 = x_2$. Thus 
				\begin{align*}
					d_X((\al_1, x_1), (\al_2, x_2))
					& = d_{\al_1}^{\phi}(x_1, x_1) \\
					& = 0.
				\end{align*}
				\item Suppose that $d_X((\al_1, x_1), (\al_2, x_2)) = 0$. 
				For the sake of contradiction, suppose that $\al_1 \neq \al_2$. Then
				\begin{align*}
					0
					& = d_X((\al_1, x_1), (\al_2, x_2)) \\
					& = 1
				\end{align*} 
				which is a contradiction. Thus $\al_1 = \al_2$. Then
				\begin{align*}
					0
					& = d_X((\al_1, x_1), (\al_2, x_2)) \\
					& = d_{\al_1}^{\phi}(x_1, x_2).
				\end{align*} 
				Hence $x_1 = x_2$ and $(\al_1, x_1) = (\al_2, x_2)$.
			\end{itemize}
			\item 
			\begin{itemize}
				\item Suppose that $\al_1 = \al_2$ and $\al_1 = \al_3$. Then $\al_2 = \al_3$ and 
				\begin{align*}
					d_X((\al_1, x_1), (\al_2, x_2))
					& = d_{\al_1}^{\phi}(x_1, x_2) \\
					& \leq d_{\al_1}^{\phi}(x_1, x_3) + d_{\al_1}^{\phi}(x_3, x_2) \\
					& = d_X((\al_1, x_1), (\al_3, x_3)) + d_X((\al_3, x_3), (\al_2, x_2)).
				\end{align*}
				\item Suppose that $\al_1 = \al_2$ and $\al_1 \neq \al_3$. Then $\al_2 \neq \al_3$ and 
				\begin{align*}
					d_X((\al_1, x_1), (\al_2, x_2))
					& = d_{\al_1}^{\phi}(x_1, x_2) \\
					& \leq 2 \\
					& = d_X((\al_1, x_1), (\al_3, x_3)) + d_X((\al_3, x_3), (\al_2, x_2)).
				\end{align*}
				\item Suppose that $\al_1 \neq \al_2$ and $\al_1 = \al_3$. Then $\al_2 \neq \al_3$ and 
				\begin{align*}
					d_X((\al_1, x_1), (\al_2, x_2))
					& = 1 \\
					& \leq d_X((\al_1, x_1), (\al_3, x_3)) + 1  \\
					& = d_X((\al_1, x_1), (\al_3, x_3)) + d_X((\al_3, x_3), (\al_2, x_2)).
				\end{align*}
				\item Suppose that $\al_1 \neq \al_2$ and $\al_1 \neq \al_3$. Then 
				\begin{align*}
					d_X((\al_1, x_1), (\al_2, x_2))
					& = 1 \\
					& \leq 1 + d_X((\al_3, x_3), (\al_2, x_2)) \\
					& = d_X((\al_1, x_1), (\al_3, x_3)) + d_X((\al_3, x_3), (\al_2, x_2)).
				\end{align*}
			\end{itemize}
		\end{enumerate}
		\item Let $(\al_j, x_j)_{j \in \N} \subset X$ and $(\al_0, x_0) \in X$. 
		\begin{itemize}
			\item Suppose that $(\al_j, x_j) \rightarrow (\al_0, x_0)$ in $(X, \MT)$. \rex{ex:topology:coproducts:0003.3} implies that there exists $j_0 \in \N$ such that 
			\begin{enumerate}
				\item for each $j \in \N$, $j \geq j_0$ implies that $\al_j = \al_0$, 
				\item $[L_{j_0}(x)]_j \rightarrow x_0 \in (X_{\al_0}, \MT_{d_{\al_0}})$.
			\end{enumerate}
			Hence $d_{\al_0}(x_{j+j_0}, x_0) \rightarrow 0$. Therefore 
			\begin{align*}
				d_X((\al_{j+j_0}, x_{j+j_0}), (\al_0, x_0))
				& = d^{\phi}(x_j, x_0) \\
				& \rightarrow 0.
			\end{align*}
			Hence $(\al_j, x_j) \rightarrow (\al_0, x_0)$ in $(X, \MT_{d_X})$.
			\item Conversely, suppose that $(\al_j, x_j) \rightarrow (\al_0, x_0)$ in $(X, \MT_{d_X})$. Then $d_X((\al_j, x_j), (\al_0, x_0)) \rightarrow 0$. Therefore there exists $j_0 \in \N$ such that for each $j \in \N$, $j \geq j_0$ implies that $\al_j = \al_0$ and $d_{\al_0}^{\phi}(x_{j+j_0}, x_0) \rightarrow 0$. Therefore $d_{\al_0}(L_{j_0}(x), x_0) \rightarrow 0$. \rex{ex:topology:coproducts:0003.3} implies that $(\al_j, x_j) \rightarrow (\al_0, x_0)$ in $(X, \MT)$. 
		\end{itemize}
		Since $(\al_j, x_j) \rightarrow (\al_0, x_0)$ in $(X, \MT_{d_X})$ iff $(\al_j, x_j) \rightarrow (\al_0, x_0)$ in $(X, \MT)$, \rex{ex:nets:0021} implies that $\MT = \MT_{d_X}$. 
	\end{enumerate}
\end{proof}

\begin{ex} \lex{ex:metric_spaces:coproducts:0003}
	Let $(X_{\al}, d_{\al})_{\al \in A}$ be a collection of metric space. Set $X \defeq \coprod\limits_{\al \in A} X_{\al}$ and $\MT \defeq \bigotimes\limits_{\al \in A} \MT_{d_{\al}}$. Then for each $\al \in A$, $\iota_{\al}:X_{\al} \rightarrow X$ is Lipschitz.
\end{ex}

\begin{proof}
	Define $\phi:\Rg \rightarrow [0,1)$ by $\phi(t) \defeq t/(1 + t)$. \rex{ex:metric_spaces:introduction:0016.11} implies that $\phi$ is Lipschitz. Hence there exists $K > 0$ such that for each $a, b \in \Rg$, $|\phi(a) - \phi(b)| \leq K |a - b|$. Let $\al \in A$ and $x_1, x_2 \in X_{\al}$. Then 
	\begin{align*}
		d_X(\iota_{\al}(x_1) \iota_{\al}(x_2))
		& = d_X((\al, x_1), (\al, x_2)) \\
		& = d_{\al}^{\phi}(x_1, x_2) \\
		& = \phi(d_{\al}(x_1, x_2)) \\
		& = \phi(|d_{\al}(x_1, x_2) - 0|)
		& \leq K|d_{\al}(x_1, x_2) - \phi(0)| \\
		& = K d_{\al}(x_1, x_2) 
	\end{align*}
	Since $x_1, x_2 \in X_{\al}$ are arbitrary, we have that $\iota_{\al}$ is Lipschitz. Since $\al \in A$ is arbitrary, we have that for each $\al \in A$, $\iota_{\al}$ is Lipschitz.
\end{proof}



































































\newpage
\section{Completeness}

\begin{defn} \ld{def:metric_spaces:completeness:0001}
	Let $(X, d)$ be a metric space and $(a_n)_{n \in \N} \subset X$. Then $(a_n)_{n \in \N}$ is said to be \tbf{Cauchy} in $(X, d)$ if for each $\ep > 0$, there exists $N \in \N$ such that for each $m,n \in \N$, $m,n \geq N$ implies that $d(a_m, a_n) < \ep$.
\end{defn}

\begin{ex} \lex{ex:metric_spaces:introduction:0001.1}
	Let $(X, d)$ be a metric space and $(x_n)_{n \in \N} \subset X$ and $x \in X$. If $x_n \rightarrow x$, then $(x_n)_{n \in \N}$ is Cauchy.  
\end{ex}

\begin{proof}
	Suppose that $x_n \rightarrow x$. Let $\ep > 0$. Then there exists $N \in \N$ such that for each $n \in \N$, $n \geq N$ implies that $d(x_n, x) < \ep/2$. Then for each $m,n \in \N$, $m,n \geq N$ implies that 
	\begin{align*}
		d(x_m, x_n)
		& \leq d(x_m, x) + d(x, x_n) \\
		& < \frac{\ep}{2} + \frac{\ep}{2} \\
		& = \ep.
	\end{align*}
	Hence $(x_n)_{n \in \N}$ is Cauchy.
\end{proof}

\begin{ex} \lex{ex:metric_spaces:introduction:0001.2}
	Let $(X, d)$ be a metric space and $(x_n)_{n \in \N} \subset X$. Suppose that $(x_n)_{n \in \N}$ is Cauchy. If there exists $(x_{n_k})_{k \in \N} \subset (x_n)_{n \in \N}$ and $x \in X$ such that $x_{n_k} \rightarrow x$, then $x_n \rightarrow x$. \tcr{NEED TO DEFINE SUBSEQUENCE (specifically how $n_k$ is an increasing sequence from $\N \rightarrow \N$ and therefore $n_k \geq k$) in topology net section!!!}
\end{ex}

\begin{proof}
	Suppose that there exists $(x_{n_k})_{k \in \N} \subset (x_n)_{n \in \N}$ and $x \in X$ such that $x_{n_k} \rightarrow x$. Let $\ep >0$. Since $x_{n_k} \rightarrow x$, there exists $K_0 \in \N$ such that for each $k \in \N$, $k \geq K_0$ implies that $d(x_{n_k}, x) < \ep/2$. Since $(x_n)_{n \in \N}$ is Cauchy, there exists $N_0 \in \N$ such that for each $m,n \in \N$, $m,n \geq N$ implies that $d(x_m, x_n) < \ep/2$. Set $K \defeq \max(N_0, K_0)$. Then $K \geq K_0$ and since $(n_k)_{k \in \N}$ is strictly increasing, we have that 
	\begin{align*}
		n_K 
		& \geq K \\
		& \geq N_0.
	\end{align*}
	Therefore for each $n \in \N$, $n \geq N_K$ implies that 
	\begin{align*}
		d(x_n, x)
		& \leq d(x_n, x_{n_K}) + d(x_{n_K}, x) \\
		& < \frac{\ep}{2} + \frac{\ep}{2} \\
		& = \ep.
	\end{align*} 
	Since $\ep > 0$ is arbitrary, we have that for each $\ep > 0$, there exists $N \in \N$ such that for each $n \in \N$, $n \geq N$ implies that $d(x_n, x) < \ep$. Hence $x_n \rightarrow x$. 
\end{proof}

\begin{ex} \lex{ex:metric_spaces:introduction:0001.3}
	Let $(X, d_X)$ and $(Y, d_Y)$ be metric spaces and $f:X \rightarrow Y$. If $f$ is uniformly continuous, then for each $(a_n)_{n \in \N} \subset X$, $(a_n)_{n \in \N}$ is Cauchy in $(X, d_X)$ implies that $(f(a_n))_{n \in \N}$ is Cauchy in $(Y, d_Y)$.
\end{ex}

\begin{proof}
	Suppose that $f$ is uniformly continuous. Let $(a_n)_{n \in \N} \subset X$. Suppose that $(a_n)_{n \in \N}$ is Cauchy in $(X, d_X)$. Let $\ep > 0$. Since $f$ is uniformly continuous, there exists $\del > 0$ such that for each $x_1, x_2 \in X$, $d_X(x_1, x_2) < \del$ implies that $d_Y(f(x_1), f(x_2)) < \ep$. Since $(a_n)_{n \in \N}$ is Cauchy in $(X, d_X)$, there exists $N \in \N$ such that for each $m,n \in \N$, $m, n \geq N$ implies that $d_X(a_m, a_n) < \del$. Let $m,n \in \N$. Suppose that $m,n \geq N$. Then $d_Y(f(a_m), f(a_n)) < \ep$. Since $\ep > 0$ is arbitrary, we have that for each $\ep > 0$, there exists $N \in \N$ such that for each $m,n \in \N$, $m,n \geq N$ implies that $d_Y(f(a_m), f(a_n)) < \ep$. Hence $(f(a_n))_{n \in \N}$ is Cauchy in $(Y, d_Y)$.
\end{proof}

\begin{ex} \lex{ex:metric_spaces:completeness:0002}
	Let $(X,d)$ be a metric space and $\phi:\Rg \rightarrow \Rg$. Suppose that $\phi$ is $\Top$ metric-preserving and
	\begin{enumerate}
		\item $\phi$ is homeomorphism, \tcr{(relax this to homeo near $0$)}
		\item $\phi(0) = 0$.
	\end{enumerate}
	Then for each $(x_n)_{n \in \N} \subset X$, $(x_n)_{n \in \N}$ is Cauchy in $(X, d)$ iff $(x_n)_{n \in \N}$ is Cauchy in $(X, d^{\phi})$.
	\tcr{(try showing that a homoemorphism with $\phi(0) = 0$ is strictly increasing)}
\end{ex}

\begin{proof}
	Let $(x_n)_{n \in \N} \subset X$. Since \tcr{$\phi$ is a homoemorphism and $\phi(0) = 0$, we have that $\phi$ is injective and strictly increasing}. 
	\begin{itemize}
		\item $(\implies):$ \\
		Suppose that $(x_n)_{n \in \N}$ is Cauchy in $(X, d)$. Let $\ep >0$. Since $\phi$ is injective and $\phi(0 = 0)$, $\phi^{-1}(\ep) > 0$ and there exists $N \in \N$ such that for each $m,n \in \N$, $m,n \geq N$ implies that $d(x_m, x_n) < \phi^{-1}(\ep)$. Let $m,n \in \N$. Suppose that $m,n \geq N$. Since $\phi$ is increasing, we have that 
		\begin{align*}
			d^{\phi}(x_m, x_n)
			& = \phi \circ d(x_m, x_n) \\
			& \leq \phi(\phi^{-1}(\ep)) \\
			& = \ep 
		\end{align*}
		Thus $(x_n)_{n \in \N}$ is Cauchy in $(X, d^{\phi})$.  
		\item $(\impliedby):$ \\
		Conversely, suppose that $(x_n)_{n \in \N}$ is Cauchy in $(X, d^{\phi})$. Let $\ep >0$. Since $\phi$ is injective and $\phi(0 = 0)$, $\phi(\ep) > 0$ and there exists $N \in \N$ such that for each $m,n \in \N$, $m,n \geq N$ implies that $d^{\phi}(x_m, x_n) < \phi(\ep)$. Let $m,n \in \N$. Suppose that $m,n \geq N$. Since $\phi^{-1}$ is increasing, we have that 
		\begin{align*}
			d(x_m, x_n)
			& = \phi^{-1} \circ \phi \circ d(x_m, x_n) \\
			& = \phi^{-1} \circ d^{\phi}(x_m, x_n) \\
			& \leq \phi^{-1}(\phi(\ep)) \\
			& = \ep 
		\end{align*}
		Thus $(x_n)_{n \in \N}$ is Cauchy in $(X, d^{\phi})$.  
	\end{itemize}
\end{proof}

\begin{defn} \ld{def:metric_spaces:completeness:0002.1}
	Let $(X, d)$ be a metric space. Then $(X, d)$ is said to be \tbf{complete} if for each $(a_n)_{n \in \N} \subset X$, $(a_n)_{n \in \N}$ is Cauchy in $(X,d)$ implies that there exists $a \in X$ such that $a_n \rightarrow a$ in $(X, \MT_d)$.
\end{defn}

\begin{ex} \lex{ex:metric_spaces:completeness:0002.2}
	Let $(X, d_X), (Y, d_Y)$ be metric spaces, $f:X \rightarrow Y$ a $(d_X, d_Y)$-isometry and $(x_n)_{n \in \N}$. Then $(x_n)_{n \in \N}$ is Cauchy in $(X, d_X)$ iff $(f(x_n))_{n \in \N}$ is Cauchy in $(Y, d_Y)$. 
\end{ex}

\begin{proof} 
	Since $f$ is a $(d_X, d_Y)$-isometry, \rex{ex:metric_spaces:introduction:0024.4} implies that $f$ is injective and $d_X = f^*d_Y$.
	\begin{itemize}
		\item $(\implies):$ \\
			Suppose that $(x_n)_{n \in \N}$ is Cauchy in $(X, d_X)$. Let $\ep > 0$. Since $(x_n)_{n \in \N}$ is Cauchy in $(X, d_X)$, there exists $N \in \N$ such that for each $m, n \in \N$, $m, n \geq N$ implies that $d_X(x_m, x_n) < \ep$. Let $m, n \in \N$.  Suppose that $m,n \geq N$. Then 
		\begin{align*}
			d_Y(f(x_m), f(x_n))
			& = f^*d_Y(x_m, x_n) \\
			& = d_X(x_m, x_n) \\
			& < \ep.
		\end{align*}  
		Since $\ep > 0$ is arbitrary, we have that for each $\ep > 0$, there exists $N \in \N$ such that for each $m, n \in \N$, $m, n \geq N$ implies that $d_Y(f(x_m), f(x_n)) < \ep$. Hence $(f(x_n))_{n \in \N}$ is Cauchy in $(Y, d_Y)$.
		\item $(\impliedby):$ \\
		Conversely, suppose that $(f(x_n))_{n \in \N}$ is Cauchy in $(Y, d_Y)$. Let $\ep > 0$. Since $(f(x_n))_{n \in \N}$ is Cauchy in $(Y, d_Y)$, there exists $N \in \N$ such that for each $m, n \in \N$, $m, n \geq N$ implies that $d_Y(f(x_m), f(x_n)) < \ep$. Let $m, n \in \N$. Suppose that $m,n \geq N$. Then 
		\begin{align*}
			d_X(x_m, x_n)
			& = f^*d_Y(x_m, x_n) \\
			& = d_Y(f(x_m), f(x_n)) \\
			& < \ep. 
		\end{align*}  
		Since $\ep > 0$ is arbitrary, we have that for each $\ep > 0$, there exists $N \in \N$ such that for each $m, n \in \N$, $m, n \geq N$ implies that $d_X(x_m, x_n) < \ep$. Hence $(x_n)_{n \in \N}$ is Cauchy in $(X, d_X)$.
	\end{itemize}
\end{proof}

\begin{ex} \lex{ex:metric_spaces:completeness:0002.3}
	Let $(X, d_X), (Y, d_Y)$ be metric spaces and $f:X \rightarrow Y$ a surjective $(d_X, d_Y)$-isometry. Then $(X, d_X)$ is complete iff $(Y, d)$ is complete.
\end{ex}

\begin{proof}
	\rex{ex:metric_spaces:introduction:0024.4} implies that $f$ is injective and $d_X = f^*d_Y$. Since $f$ is surjective, $f$ is a bijection. \rex{ex:metric_spaces:introduction:0024.5} implies that $f^{-1}$ is a $(d_Y, d_X)$-isometry and \rex{ex:metric_spaces:introduction:0024.4} implies that $d_Y = (f^{-1})^*d_X$.
	\begin{itemize}
		\item $(\implies):$ \\
		Suppose that $(X, d_X)$ is complete. Let $(y_n)_{n \in \N} \subset Y$. Suppose that $(y_n)_{n \in \N}$ is Cauchy in $(Y, d_Y)$. Define $(x_n)_{n \in \N} \subset X$ by $x_n \defeq f^{-1}(y_n)$. Since $f^{-1}$ is a $(d_Y, d_X)$-isometry, \rex{ex:metric_spaces:completeness:0002.2} then implies that $(x_n)_{n \in \N}$ is Cauchy in $(X, d_X)$. Since $(X, d_X)$ is complete, there exists $x \in X$ such that $x_n \rightarrow x$ in $(X, d_X)$. Set $y \defeq f(x)$. Then 
		\begin{align*}
			d_Y(y_n, y)
			& = (f^{-1})^*d_X(y_n, y) \\
			& = d_X(f^{-1}(y_n), f^{-1}(y)) \\
			& = d_X(x_n, x) \\
			& \rightarrow 0 
		\end{align*}
		\rex{ex:metric_spaces:introduction:0015} implies that $y_n \rightarrow y$. Since $(y_n)_{n \in \N} \subset Y$ with $(y_n)_{n \in \N}$ Cauchy in $(Y, d_Y)$ is arbitrary, we have that for each $(y_n)_{n \in \N} \subset Y$, $(y_n)_{n \in \N}$ is Cauchy in $(Y, d_Y)$ implies that there exists $y \in Y$ such that $y_n \rightarrow y$. Hence $(Y, d_Y)$ is complete.
		\item $(\impliedby):$ \\
		Similar to $(\implies)$.
	\end{itemize}
\end{proof}

\begin{ex} \lex{ex:metric_spaces:completeness:0002.3.1}
	Let $(X, d_X), (Y, d_Y)$ be metric spaces, $S \subset X$ and $f_0:S \rightarrow Y$. Suppose that $(Y, d_Y)$ is complete, $S$ is dense in $X$ and $f_0$ is uniformly continuous. Then there exists a unique $f:X \rightarrow Y$ such that $f$ is uniformly continuous and $f|_S = f_0$.
\end{ex}

\begin{proof}\
	\begin{itemize}
		\item For each $x \in X$, define $E_x \subset S^{\N}$ by $E_x \defeq \{(x_n)_{n \in \N} \subset S: x_n \rightarrow x \text{ and $f_0(x_n)$ converges}\}$. Let $x \in X$. Since $S$ is dense in $X$, there exists $(x_n)_{n \in \N} \subset S$ such that $x_n \rightarrow x$. \rex{ex:metric_spaces:introduction:0001.1} then implies that $(x_n)_{n \in \N}$ is Cauchy in $(X, d_X)$. Since $f_0$ is uniformly continuous, \rex{ex:metric_spaces:introduction:0001.3} implies that $(f_0(x_n))_{n \in \N}$ is Cauchy in $(Y, d_Y)$. Since $(Y, d_Y)$ is complete, there exists $y_x \in Y$ such that $f_0(x_n) \rightarrow y_x$. Thus $E_x \neq \varnothing$. Since $x \in X$ is arbitrary, we have that for each $x \in X$, $E_x \neq \varnothing$. The axiom of choice implies that there exists $\phi \in \prod\limits_{x \in X} E_x$. Define $f: X \rightarrow Y$ by $f(x) \defeq \limn f_0(\phi(x)_n)$. 
		\item Let $\ep > 0$. Since $f_0$ is uniformly continuous, there exists $\del_0 > 0$ such that for each $a,b \in S$, $d_X(a,b) < \del_0$ implies $d_Y(f_0(a), f_0(b)) < \ep/3$. Define $\del > 0$ by $\del \defeq \del_0/3$. Let $a,b \in X$. Suppose that $d_X(a,b) < \del$. Since $\phi(a)_n \rightarrow a$, $\phi(b)_n \rightarrow b$, there exist $M_a, M_b \in \N$ such that for each $n \in \N$, $n \geq M_a$ implies that $d_X(\phi(a)_n, a) < \del$ and $n \geq M_b$ implies that $d_X(\phi(b)_n, b) < \del$. Define $M \in \N$ by $M \defeq \max \{M_a, M_b\}$. Let $n \in \N$. Suppose that $n \geq M$. Since $d_X(a,b) < \del$, we have that
		\begin{align*}
			d_X(\phi(a)_n, \phi(b)_n) 
			& \leq d_X(\phi(a)_n, a) + d_X(a, b) + d_X(b, \phi(b)_n) \\
			& < \del + \del + \del \\
			& = \del_0. 
		\end{align*}
		Therefore $d_Y(f_0(\phi(a)_n), f_0(\phi(b)_n)) < \ep/3$. Since $n \in \N$ with $n \geq M$ is arbitrary, we have that for each $n \in \N$, $n \geq M$ implies that $d_Y(f_0(\phi(a)_n), f_0(\phi(b)_n)) < \ep/3$. Since $f_0(\phi(a)_n) \rightarrow f(a)$ and $f_0(\phi(b)_n) \rightarrow f(b)$, there exist $N_a, N_b \in \N$ such that for each $n \in \N$, $n \geq N_a$ implies that $d_Y(f_0(\phi(a)_n), f(a)) < \ep/3$ and $n \geq N_b$ implies that $d_Y(f_0(\phi(b)_n), f(b)) < \ep/3$. Define $N \in \N$ by $N \defeq \max \{M, N_a, N_b\}$. Then 
		\begin{align*}
			d_Y(f(a), f(b))
			& \leq d_Y(f(a), f_0(\phi(a)_N)) + d_Y(f_0(\phi(a)_N), f_0(\phi(b)_N)) + d_Y(f_0(\phi(b)_N), f(b)) \\
			& < \frac{\ep}{3} + \frac{\ep}{3} + \frac{\ep}{3} \\
			& = \ep. 
		\end{align*}
		Since $\ep>0$ is arbitrary, we have that for each $\ep >0$, there  exists $\del >0$ such that for each $a,b \in X$, $d_X(a,b) < \del$ implies that $d_Y(f(a), f(b)) < \ep$. Hence $f$ is uniformly continuous.
		\item 
		\begin{itemize}
			\item Let $x \in S$,  $(a_n)_{n \in \N}$ and $(b_n)_{n \in \N} \in E_x$. By definition, $(a_n)_{n \in \N}, (b_n)_{n \in \N} \subset S$, $a_n, b_n \rightarrow x$. As in the previous part, there exists $y_a, y_b \in Y$ such that $f_0(a_n) \rightarrow y_a$ and $f_0(b_n) \rightarrow y_b$. Let $\ep > 0$. Since $f_0$ is uniformly continuous, there exists $\del > 0$ such that for each $a,b \in S$, $d_X(a,b) < \del$ implies $d_Y(f_0(a), f_0(b)) < \ep/3$. Since $f_0(a_n) \rightarrow y_a$ and $f_0(b_n) \rightarrow y_b$, there exist $N_a, N_b \in \N$ such that for each $n \in \N$, $n \geq N_a$ implies that $d_Y(f_0(a_n), y_a) < \ep/3$ and $n \geq N_b$ implies that $d_Y(f_0(b_n), y_b) < \ep/3$. Since $a_n \rightarrow x$ and $b_n \rightarrow x$, we have that 
			\begin{align*}
				d_X(a_n, b_n) 
				& \leq d_X(a_n, x) + d_X(x, b_n) \\
				& \rightarrow 0.
			\end{align*}
			Thus there exists $N_0 \in \N$ such that for each $n \in \N$, $n \geq N_0$ implies that $d_X(a_n, b_n) < \del$. Set $N \defeq \max \{N_0, N_a, N_b\}$. Then
			\begin{align*}
				d_Y(y_a, y_b)
				& \leq d_Y(y_a, f_0(a_N)) + d_Y(f_0(a_N), f_0(b_N)) + d_Y(f_0(b_N), y_b) \\
				& < \frac{\ep}{3} + \frac{\ep}{3} + \frac{\ep}{3} \\
				& = \ep. 
			\end{align*}
			Since $\ep > 0$ is arbitrary, we have that for each $\ep >0$, $d_Y(y_a, y_b) < \ep$. Hence $d_Y(y_a, y_b) = 0$ and 
			\begin{align*}
				\limn f_0(a_n)
				& = y_a \\
				& = y_b \\
				& = \limn f_0(b_n).
			\end{align*}
			Since $x \in S$ and $(a_n)_{n \in \N}, (b_n)_{n \in \N} \in E_x$ are arbitrary, we have that for each $x \in S$ and $(a_n)_{n \in \N}, (b_n)_{n \in \N} \in E_x$, $\limn f_0(a_n) = \limn f_0(b_n)$.
			\item Let $x \in S$. Define $(x_n)_{n \in \N} \in E_x$ by $x_n \defeq x$. Then from above, we have that
			\begin{align*}
				f(x)
				& = \limn f_0(\phi(x)_n) \\
				& = \limn f_0(x_n) \\
				& = f_0(x).
			\end{align*}
			Since $x \in S$ is arbitrary, we have that for each $x \in S$, $f(x) = f_0(x)$. Hence $f|_S = f_0$.
		\end{itemize}
		\item Let $g:X \rightarrow Y$. Suppose that $g$ is uniformly continuous and $g|_S = f_0$. Let $x \in X$. Since $E_x \neq \varnothing$, there exists $(x_n)_{n \in \N} \in E_x$. Then from above, we have that
		\begin{align*}
			f(x)
			& = \limn f_0(\phi(x)_n) \\
			& = \limn f_0(x_n) \\
			& = \limn g(x_n) \\
			& = g(x).
		\end{align*} 
		Since $x \in X$ is arbitrary, we have that for each $x \in X$, $f(x) = g(x)$. Thus $g= f$. Hence $f$ is unique.
	\end{itemize}
\end{proof}


\begin{ex} \lex{ex:metric_spaces:completeness:0002.4} \tbf{Cantor’s Nested Set Theorem:} \\ 
	Let $(X, d)$ be a complete metric space and $(C_n)_{n \in \N} \subset \MP(X)$. Suppose that 
	\begin{itemize}
		\item for each $n \in \N$, $C_n$ is closed, $C_n \neq \varnothing$ and $C_{n+1} \subset C_n$,
		\item $\diam C_n \rightarrow 0$.
	\end{itemize}  
	Then there exists $x \in X$ such that $\bigcap\limits_{n \in \N} C_n = \{x\}$. \\
	\tbf{Hint:} Use $(C_n)_{n \in \N}$ to obtain a Cauchy sequence.
\end{ex}

\begin{proof}\
	\begin{itemize}
		\item Since for each $n \in \N$, $C_n \neq \varnothing$, the axiom of choice implies that there exists $(x_n)_{n \in \N}$ such that for each $n \in \N$, $x_n \in C_n$. Let $\ep >0$. Since $\diam C_n \rightarrow 0$, there exists $N \in \N$ such that for each $n \in \N$, $n \geq N$ implies that $\diam C_n < \ep$. Let $m,n \in \N$. Suppose that $m,n \geq N$. 
		\begin{itemize}
			\item Suppose that $m \geq n$. Then $x_n \in C_n$ and 
			\begin{align*}
				x_m 
				& \in C_m \\
				& \subset C_n.
			\end{align*} 
			Hence
			\begin{align*}
				d(x_m, x_n)
				& \leq \diam C_n \\
				& < \ep. 
			\end{align*} 
			\item Similarly, if $n > m$, then $d(x_m, x_n) < \ep$.
		\end{itemize}
		Since $m,n \in \N$ such that $m,n \geq N$ are arbitrary, we have that for each $m,n \in \N$, $m, n \geq N$ implies that $d(x_m, x_n) < \ep$. Since $\ep > 0$ is arbitrary, we have that for each $\ep >0$, there exists $N \in \N$ such that for each $m,n \in \N$, $m,n \geq N$ implies that $d(x_m,x_n) < \ep$. Thus $(x_n)_{n \in \N}$ is Cauchy. Since $(X, d)$ is complete, there exists $x \in X$ such that $x_n \rightarrow x$. 
		\item 
		\begin{itemize}
			\item For the sake of contradiction, suppose that $x \not \in \bigcap\limits_{n \in \N} C_n$. Then $x \in \bigcup \limits_{n \in \N} C_n^c$. Hence there exists $N_1 \in \N$ such that $x \in C_{N_1}^c$. Since for each $n \in \N$, $C_{n+1} \subset C_{n}$, we have that for each $n \in \N$, $n \geq N_1$ implies that 
			\begin{align*}
				x
				& \in C_{N_1}^c \\
				& \subset C_{n}^c.
			\end{align*}
			Since $C_{N_1}$ is closed, \rex{ex:metric_spaces:introduction:0017} implies that $d(x, C_{N_1}) > 0$. Set $\ep \defeq d(x, C_{N_1})$. We note that for each $n \in \N$, $n \geq N_1$ implies that $\{d(x, y): y \in C_n\} \subset \{d(x, y): y \in C_{N_1}\}$ and therefore
			\begin{align*}
				d(x, C_n)
				& = \inf \{d(x, y): y \in C_n\} \\
				& \geq \inf \{d(x, y): y \in C_{N_1} \} \\
				& = d(x, C_{N_1}) \\
				& = \ep.
			\end{align*}
			Since $x_n \rightarrow x$, there exists $N_2 \in \N$ such that for each $n \in \N$, $n \geq N_2$ implies that $d(x_n, x) < \ep$. Set $N \defeq \max(N_1, N_2)$. Then 
			\begin{align*}
				\ep 
				& = d(x, C_{N_1}) \\
				& \leq d(x, C_N) \\
				& \leq d(x, x_N) \\
				& < \ep,
			\end{align*}
			which is a contradiction. Hence $\{x\} \subset \bigcap\limits_{n \in \N} C_n$. 
			\item For the sake of contradiction, suppose that $\bigcap\limits_{n \in \N} C_n \not \subset \{x\}$. Then there exists $x_0 \in \bigcap\limits_{n \in \N} C_n $ such that $x_0 \neq x$. Set $\ep \defeq d(x, x_0)$. Then $\ep > 0$. Since $\diam C_n \rightarrow 0$, there exists $N \in \N$ such that for each $n \in \N$, $n \geq N$ implies that $\diam C_n < \ep$. Since $x, x_0 \in \bigcap\limits_{n \in \N} C_n$, we have that $x, x_0 \in C_N$ and therefore
			\begin{align*}
				\ep 
				& = d(x, x_0) \\
				& \leq \diam C_N \\
				& < \ep,
			\end{align*}
			which is a contradiction. Hence $\bigcap\limits_{n \in \N} C_n \subset \{x\}$.
		\end{itemize}
		Since $\{x\} \subset \bigcap\limits_{n \in \N} C_n$ and $\bigcap\limits_{n \in \N} C_n \subset \{x\}$, we have that $\bigcap\limits_{n \in \N} C_n = \{x\}$.
		\end{itemize}
\end{proof}



































\subsection{Completeness and Subspaces}

\begin{ex} \lex{ex:metric_spaces:completeness:0003}
	Let $(X, d)$ be a metric space, $E \subset X$ and $(a_n)_{n \in \N} \subset E$. Then $(a_n)_{n \in \N}$ is Cauchy in $(E, d|_{E \times E})$ iff $(a_n)_{n \in \N}$ is Cauchy in $(X, d)$.  
\end{ex}

\begin{proof}
	Set $d_E \defeq d|_{E \times E}$.
	\begin{itemize}
		\item $(\implies):$ \\
		Suppose that $(a_n)_{n \in \N}$ is Cauchy in $(E, d|_{E \times E})$. Let $\ep > 0$. Since $(a_n)_{n \in \N}$ is Cauchy in $(E, d_E)$, there exists $N \in \N$ such that for each $m,n \in \N$, $m,n \geq N$ implies that $d_E(a_m, a_n) < \ep$. Thus for each $m,n \in \N$, $m,n \geq N$ implies that
		\begin{align*}
			d(a_m, a_n)
			& = d_E(a_m, a_n) \\
			& < \ep 
		\end{align*}
		Since $\ep > 0 $ is arbitrary, we have that for each $\ep > 0$, there exists $N \in \N$ such that for each $m,n \in \N$, $m,n \geq N$ implies that $d(a_m, a_n) < \ep$. Hence $(a_n)_{n \in \N}$ is Cauchy in $(X, d)$.
		\item $(\impliedby):$ \\
		Suppose that $(a_n)_{n \in \N}$ is Cauchy in $(X, d)$. Let $\ep > 0$. Since $(a_n)_{n \in \N}$ is Cauchy in $(X, d)$, there exists $N \in \N$ such that for each $m,n \in \N$, $m,n \geq N$ implies that $d(a_m, a_n) < \ep$. Thus for each $m,n \in \N$, $m,n \geq N$ implies that
		\begin{align*}
			d_E(a_m, a_n)
			& = d(a_m, a_n) \\
			& < \ep 
		\end{align*}
		Since $\ep > 0 $ is arbitrary, we have that for each $\ep > 0$, there exists $N \in \N$ such that for each $m,n \in \N$, $m,n \geq N$ implies that $d_E(a_m, a_n) < \ep$. Hence $(a_n)_{n \in \N}$ is Cauchy in $(E, d_E)$.
	\end{itemize}
\end{proof}

\begin{ex} \lex{ex:metric_spaces:completeness:0004}
	Let $(X, d)$ be a metric space and $C \subset X$. Suppose that $(X, d)$ is complete. If $C$ is closed, then $(C, d|_{C \times C})$ is complete. 
\end{ex}

\begin{proof}
	Suppose that $C$ is closed. Set $d_C \defeq d|_{C \times C}$. Let $(a_n)_{n \in \N} \subset C$. Suppose that $(a_n)_{n \in \N}$ is Cauchy in $(C, d_C)$. \rex{ex:metric_spaces:completeness:0003} implies that $(a_n)_{n \in \N}$ is Cauchy in $(X, d)$. Since $(X, d)$ is complete, there exists $a \in X$ such that $a_n \rightarrow a$ in $(X, \MT_X)$. Since $C$ is closed, $a \in C$. \rex{ex:topology:subspaces:0005} implies that $a_n \rightarrow a$ in $(C, \MT_X \cap C)$. Since $(a_n)_{n \in \N} \subset C$ is arbitrary, we have that for each $(a_n)_{n \in \N} \subset C$, $(a_n)_{n \in \N}$ is Cauchy in $(C, d_C)$ implies that there exists $a \in C$ such that $a_n \rightarrow a$ in $(C, \MT_X \cap C)$. Hence $(C, d_C)$ is complete. 
\end{proof}


















\subsection{Completeness and Product Spaces}

\begin{ex} \lex{ex:metric_spaces:completeness:0005}
	Let $(X_n, d_n)_{n \in \N}$ be a collection of metric spaces. Set $X \defeq \prod\limits_{n \in \N} X_n$ and define $d_X: X \times X \rightarrow \Rg$ as in \rd{def:metric_spaces:product_spaces:0001}. Then for each $(a_m)_{m \in \N} \subset X$, $(a_m)_{m \in \N}$ is Cauchy in $(X, d_X)$ iff for each $n \in \N$, $(\pi_n(a_m))_{m \in \N}$ is Cauchy in $(X_n, d_n)$.
\end{ex}

\begin{proof}
	Define $\phi: \Rg \rightarrow \Rg$ as in \rd{def:metric_spaces:product_spaces:0001}. Let $(a_m)_{m \in \N} \subset X$. 
	\begin{itemize}
		\item Suppose that $(a_m)_{m \in \N}$ is Cauchy in $(X, d_X)$. Let $n \in \N$ and $\ep >0$. Since $(a_m)_{m \in \N}$ is Cauchy in $(X, d_X)$, there exists $M \in \N$ such that for each $m_1, m_2 \in \N$, $m_1, m_2 \geq M$ implies that $d_X(a_{m_1}, a_{m_2}) < 2^{-n} \ep$. Let $m_1, m_2 \in \N$. Suppose that $m_1, m_2 \geq M$. Then 
		\begin{align*}
			d^{\phi}_n(\pi_n(a_{m_1}), \pi_n(a_{m_2}))
			& \leq 2^{n} d_X(a_{m_1}, a_{m_2}) \\
			& < \frac{ 2^n \ep}{2^n} \\
			& = \ep.
		\end{align*}
		Thus $(\pi_n(a_m))_{m \in \N}$ is Cauchy in $(X_n, d_n^{\phi})$. \rex{ex:metric_spaces:completeness:0002} and \rex{ex:metric_spaces:topologically_equivalent_metrics:0007} imply that $(\pi_n(a_m))_{m \in \N}$ is Cauchy in $(X_n, d_n)$.
		\item Conversely, suppose that for each $n \in \N$, $(\pi_n(a_m))_{m \in \N}$ is Cauchy in $(X_n, d_n)$. \rex{ex:metric_spaces:completeness:0002} and \rex{ex:metric_spaces:topologically_equivalent_metrics:0007} imply that for each $n \in \N$, $(\pi_n(a_m))_{m \in \N}$ is Cauchy in $(X_n, d_n^{\phi})$. Let $\ep >0$. Choose $N_0 \in \N$ such that $\sum\limits_{n \geq N_0+1} 2^{-n} < \ep/2$. Since for each $n \in \N$, $(\pi_n(a_m))_{m \in \N}$ is Cauchy in $(X_n, d_n^{\phi})$, we have that for each $n \in [N_0]$, there exists $M_n \in \N$ such that for each $m_1, m_2 \in \N$, $m_1, m_2 \geq M_n$ implies that $d_n^{\phi}(\pi_n(a_{m_1}), \pi_n(a_{m_2})) < \ep / 2$. Set $M \defeq \max(M_1, \ldots, M_{N_0})$. Let $m_1, m_2 \in \N$. Suppose that $m_1, m_2 \geq M$. Then 
		\begin{align*}
			d_X(a_{m_1}, a_{m_1})
			& = \sum_{n \in \N} 2^{-n} d_n^{\phi}(\pi_n(a_{m_1}), \pi_n(a_{m_2})) \\
			& = \sum_{n =1}^{N_0} 2^{-n} d_n^{\phi}(\pi_n(a_{m_1}), \pi_n(a_{m_2})) + \sum_{n \geq N_0+1} 2^{-n} d_n^{\phi}(\pi_n(a_{m_1}), \pi_n(a_{m_2})) \\
			& < \sum_{n =1}^{N_0} 2^{-n} \bigg( \frac{\ep}{2} \bigg) + \sum_{n \geq N_0+1} 2^{-n} \\
			& = \frac{\ep}{2}\sum_{n =1}^{N_0} 2^{-n} + \sum_{n \geq N_0+1} 2^{-n} \\
			& < \frac{\ep}{2} + \frac{\ep}{2} \\
			& = \ep.
		\end{align*}
		Since $\ep > 0$ is arbitrary, we have that for each $\ep >0$, there exists $M \in \N$ such that for each $m_1, m_2 \in \N$, $m_1, m_2 \geq M$ implies that $d_X(a_{m_1}, a_{m_1}) < \ep$. Hence $(a_m)_{m \in \N}$ is Cauchy in $(X, d_X)$.
	\end{itemize}
	Thus $(a_m)$ is Cauchy in $(X< d_X)$ iff for each $n \in \N$, $(\pi_n(a_m))_{m \in \N}$ is Cauchy in $(X_n, d_n)$. Since $(a_m)_{m \in \N} \subset X$ is arbitrary, we have that for each $(a_m)_{m \in \N} \subset X$, $(a_m)_{m \in \N}$ is Cauchy in $(X, d_X)$ iff for each $n \in \N$, $(\pi_n(a_m))_{m \in \N}$ is Cauchy in $(X_n, d_n)$.
\end{proof}

\begin{ex} \lex{ex:metric_spaces:completeness:0006}
	Let $(X_n, d_n)_{n \in \N}$ be a collection of metric spaces. Set $X \defeq \prod\limits_{n \in \N} X_n$ and define $d_X: X \times X \rightarrow \Rg$ as in \rd{def:metric_spaces:product_spaces:0001}. Then $(X, d_X)$ is complete iff for each $n \in \N$, $(X_n, d_n)$ is complete. 
\end{ex}

\begin{proof}
	Define $\phi:\Rg \rightarrow [0, 1)$ and $\MT$ as in \rd{def:metric_spaces:product_spaces:0001}. 
	\begin{itemize}
		\item Suppose that $(X, d_X)$ is complete. Let $n \in \N$ and $(x_m)_{m \in \N} \subset X_n$. Suppose that $(x_m)_{m \in \N}$ is Cauchy in $(X_n, d_n)$. Choose $a_0 \in X$. Define $(a_m)_{m \in \N} \subset X$ by  
		\[
		\pi_k(a_m) =
		\begin{cases}
			\pi_k(a_0), & k \neq n \\
			x_m, & k = n
		\end{cases}
		\]
		Let $\ep > 0$. Since $(x_m)_{m \in \N}$ is Cauchy in $(X_n, d_n)$, \rex{ex:metric_spaces:completeness:0002} implies that $(x_m)_{m \in \N}$ is Cauchy in $(X_n, d_n^{\phi})$. Thus there exists $M \in \N$ such that for each $m_1, m_2 \in \N$, $m_1, m_2 \geq M$ implies that $d_n^{\phi}(x_{m_1}, x_{m_2}) < 2^n \ep$. Let $m_1, m_2 \in \N$. Suppose that $m_1, m_2 \geq M$. Then 
		\begin{align*}
			d_X(a_{m_1}, a_{m_2})
			& = \sum_{k \in \N} 2^{-k} d_k^{\phi}(\pi_k(a_{m_1}), \pi_k(a_{m_2})) \\
			& = 2^{-n} d_n^{\phi}(\pi_n(a_{m_1}), \pi_n(a_{m_2})) + \sum_{k \neq n} 2^{-k} d_k^{\phi}(\pi_k(a_{m_1}), \pi_k(a_{m_2})) \\
			& = 2^{-n} d_n^{\phi}(x_{m_1}, x_{m_2}) + \sum_{k \neq n} 2^{-k} d_k^{\phi}(\pi_k(a_0), \pi_k(a_0)) \\
			& = 2^{-n} d_n^{\phi}(x_{m_1}, x_{m_2}) \\
			& < 2^{-n} (2^n \ep) \\
			& = \ep .
		\end{align*}
		Since $\ep > 0$ is arbitrary, we have that for each $\ep > 0$, there exists $M \in \N$ such that for each $m_1, m_2 \in \N$, $m_1, m_2 \geq M$ implies that $d_X(a_{m_1}, a_{m_2}) < \ep$. Therefore $(a_m)_{m \in \N}$ is Cauchy in $(X, d_X)$. Since $(X, d_X)$ is complete, there exists $a \in X$ such that $a_m \rightarrow a$ in $(X, \MT_{d_X})$. \rex{ex:metric_spaces:product_spaces:0002} implies that $a_m \rightarrow a$ in $(X, \MT)$. Define $x \in X_n$ by $x \defeq \pi_n(a)$. \rex{ex:product_topology:0006.1} implies that 
		\begin{align*}
			x_m
			& = \pi_n(a_m) \\
			& \rightarrow \pi_n(a) \\
			& = x
		\end{align*}
		in $(X_n, \MT_{d_n})$. Since $(x_m)_{m \in \N} \subset X_n$ with $(x_m)_{m \in \N}$ Cauchy is arbitrary, we have that for each $(x_m)_{m \in \N} \subset X_n$, if $(x_m)_{m \in \N}$ Cauchy, then there exists $x \in X_n$ such that $x_m \rightarrow x$ in $(X, \MT_{d_n})$. Hence $(X_n, d_n)$ is Complete. Since $n \in \N$ is arbitrary, we have that for each $n \in \N$, $(X_n, d_n)$ is complete.
		\item Converesly, suppose that for each $n \in \N$, $(X_n, d_n)$ is complete. Let $(a_m)_{m \in \N} \subset X$. Suppose that $(a_m)_{m \in \N}$ is Cauchy in $(X, d_X)$. \rex{ex:metric_spaces:completeness:0005} implies that for each $n \in \N$, $(\pi_n(a_m))_{m \in \N}$ is Cauchy in $(X_n, d_n)$. Since for each $n \in \N$, $(X_n, d_n)$ is complete, we have that for each $n \in \N$, there exists $x_n \in X_n$ such that $\pi_n(a_m) \conv{m} x_n$ in $(X_n, \MT_{d_n})$. Define $a \in X$ by $a \defeq (x_n)_{n \in \N}$. \rex{ex:product_topology:0006.1} implies that $a_m \rightarrow a$ in $(X, \MT)$. \rex{ex:metric_spaces:product_spaces:0002} implies that $a_m \rightarrow a$ in $(X, \MT_{d_X})$. Since $(a_m)_{m \in \N} \subset X$ is arbitrary, we have that for each $(a_m)_{m \in \N} \subset X$, if $(a_m)_{m \in \N}$ is Cauchy in $(X, d_X)$, then there exists $a \in X$ such that $a_m \rightarrow a$ in $(X, \MT_{d_X})$. Hence $(X, d_X)$ is complete.
	\end{itemize}
\end{proof}

%\begin{ex} \lex{ex:metric_spaces:completeness:0007}
%	Let $(X_n, d_n)_{n \in \N}$ be a collection of metric spaces. Suppose that for each $n \in \N$, $(X_n, d_n)$ is complete. Set $X \defeq \prod\limits_{n \in \N} X_n$ and define $d_X: (\prod\limits_{n \in \N} X_n)^2 \rightarrow \Rg$ as in \rd{def:metric_spaces:product_spaces:0001}. Then $(X, d_X)$ is complete.  
%\end{ex}
%
%\begin{proof}
%	Set $X \defeq \prod\limits_{n \in \N} X_n$ and $\MT \defeq \bigotimes\limits_{n \in \N} \MT_n$. Let $(a_m)_{m \in \N} \subset X$. Then for each $m \in \N$, there exists $(x_{m,n})_{n \in \N} \in X$ such that $a_m = (x_{m,n})_{n \in \N}$. Suppose that $(a_m)_{m \in \N}$ is Cauchy. Let $n \in \N$ and $\ep > 0$. Then $2^{-n} \ep > 0$. Since $(a_m)_{m \in \N}$ is Cauchy, there exists $N \in \N$ such that for each $m_1, m_2 \in \N$, $m_1, m_2 \geq N$ implies that $d(a_{m_1}, a_{m_2}) < 2^{-n} \ep$. Let $m_1, m_2 \in \N$. Suppose that $m_1, m_2 \geq N$. Then
%	\begin{align*}
%		2^{-n} d_n(x_{m_1, n}, x_{m_2, n})
%		& \leq d(a_{m_1}, a_{m_2}) \\
%		& < 2^{-n} \ep  
%	\end{align*}
%	and thus $d_n(x_{m_1, n}, x_{m_2, n}) < \ep$. Since $\ep > 0$ is arbitrary, we have that for each $\ep >0$, there exists $N \in \N$ such that for each $m_1, m_2 \in \N$, $m_1, m_2 \geq N$ implies that $d_n(x_{m_1, n}, x_{m_2, n}) < \ep$. Therefore $(x_{m,n})_{m \in \N}$ is Cauchy. Since $(X_n, d_n)$ is complete, there exists $x_n \in X_n$ such that $x_{m,n} \conv{m} x_n$. Since $n \in \N$ is arbitrary, we have that for each $n \in \N$, there exists $x_n \in X_n$ such that $\pi_n(a_m) \conv{m} x_n$. Define $a \in X$ by $a_n \defeq x_n$. \tcb{An exercise or definition in product topology section} implies that $a_m \rightarrow a$. \tcr{(maybe clean up and cite product metric space exercise)}. Since $(a_m)_{m \in \N} \subset X$ with $(a_m)_{m \in \N}$ Cauchy is arbitrary, we have that for each $(a_m)_{m \in \N} \subset X$, if $(a_m)_{m \in \N}$ Cauchy, then there exists $a \in X$ such that $a_m \rightarrow a$. Thus $(X, d)$ is complete.
%\end{proof}













































\subsection{Completeness and Coproduct Spaces}

\begin{ex} \lex{ex:metric_spaces:completeness:0008}
	Let $(X_{\al}, d_{\al})_{\al \in A}$ be a collection of metric spaces. Set $X \defeq \coprod\limits_{\al \in A} X_{\al}$ and define $\phi:\Rg \rightarrow [0,1)$ and $d_X: X \times X \rightarrow \Rg$ as in \rd{def:metric_spaces:coproducts:0001}. Then for each $(\al_j, x_j)_{j \in \N} \subset X$, $(\al_j, x_j)_{j \in \N}$ is Cauchy in $(X, d_X)$ iff there exists $\al_0 \in A$ and $j_0 \in \N$ such that 
	\begin{enumerate}
		\item for each $j \in \N$, $j \geq j_0$ implies that $\al_j = \al_0$ and $x_j \in X_{\al_0}$,
		\item $(x_{j'+j_0})_{j' \in \N}$ is Cauchy in $(X_{\al_0}, d_{\al_0})$. 
	\end{enumerate}
\end{ex}

\begin{proof} Let $(\al_j, x_j)_{j \in \N} \subset X$. 
	\begin{itemize}
		\item $(\implies):$ \\
		Suppose that $(\al_j, x_j)_{j \in \N}$ is Cauchy in $(X, d_X)$. Set $\ep_0 \defeq 1/2$. Since $(\al_j, x_j)_{j \in \N}$ is Cauchy in $(X, d_X)$, there exists $j_0 \in \N$ such that for each $j,k \in \N$, $j,k \geq j_0$ implies that $d_X((\al_j, x_j), (\al_k, x_k)) < \ep_0$. Set $\al_0 \defeq \al_{j_0}$. 
		\begin{enumerate}
			\item Since $\ep_0 < 1$, we have that for each $j \in \N$, $j \geq j_0$ implies that $\al_j = \al_0$ and therefore $x_j \in X_{\al_0}$.
			\item Define $(y_j)_{j \in \N} \subset X_{\al_0}$ by $y_j \defeq x_{j + j_0}$. Let $\ep > 0$. Since $(\al_j, x_j)_{j \in \N}$ is Cauchy in $(X, \MT_X)$, there exists $N_0 \in \N$ such that for each $j,k \in \N$, $j,k \geq N_0$ implies that $d_X((\al_j, x_j), (\al_k, x_k)) < \ep$. Set $N \defeq \max(j_0 + 1, N_0) - j_0$. Then for each $j,k \in \N$, $j,k \geq N$ implies that $j + j_0, k+j_0 \geq N_0$ and 
			\begin{align*}
				d^{\phi}_{\al_0}(y_j, y_k)
				& = d^{\phi}_{\al_0}(x_{j+j_0}, y_{k+j_0}) \\
				& = d_X((\al_j, x_{j+j_0}), (\al_k, y_{k+j_0})) \\
				& < \ep .
			\end{align*}
			Thus $(y_j)_{j \in \N}$ is Cauchy in $(X_{\al_0}, d^{\phi}_{\al_0})$.  \rex{ex:metric_spaces:completeness:0002} implies that $(y_j)_{j \in \N}$ is Cauchy in $(X_{\al_0}, d_{\al_0})$.  
		\end{enumerate}
		\item $(\impliedby):$ \\
		Conversely, suppose that there exists $\al_0 \in A$ and $j_0 \in \N$ such that 
		\begin{enumerate}
			\item for each $j \in \N$, $j \geq j_0$ implies that $\al_j = \al_0$ and $x_j \in X_{\al_0}$,
			\item $(x_{j'+j_0})_{j' \in \N}$ is Cauchy in $(X_{\al_0}, d_{\al_0})$. 
		\end{enumerate}
		Define $(y_j)_{j \in \N} \subset X_{\al_0}$ by $y_j \defeq x_{j + j_0}$. Since $(y_j)_{j \in \N}$ is Cauchy in $(X_{\al_0}, d_{\al_0})$, we have that $(y_j)_{j \in \N}$ is Cauchy in $(X_{\al_0}, d^{\phi}_{\al_0})$. Let $\ep > 0$. Since $(y_j)_{j \in \N}$ is Cauchy in $(X_{\al_0}, d^{\phi}_{\al_0})$, there exists $N_0 \in \N$ such that for each $j,k \in \N$, $j,k \geq N_0$ implies that $d^{\phi}(y_j, j_k) < \ep$. Set $N \defeq \max(j_0, N_0) + j_0$. Then for each $j,k \in \N$, $j,k \geq N$ implies that $j-j_0, k-j_0 \geq N_0, j_0$. Thus 
		\begin{align*}
			d_X((\al_j, x_j), (\al_k, x_k))
			& = d^{\phi}(x_j, x_k) \\
			& = d^{\phi}(y_{j- j_0}, y_{k-j_0}) \\
			& < \ep. 
		\end{align*} 
		Hence $(\al_j, x_j)_{j \in \N}$ is Cauchy in $(X, d_X)$. 
	\end{itemize}
\end{proof}

\begin{ex} \lex{ex:metric_spaces:completeness:0009}
	Let $(X_{\al}, d_{\al})_{\al \in A}$ be a collection of metric spaces. Set $X \defeq \coprod\limits_{\al \in A} X_{\al}$ and define $\phi:\Rg \rightarrow [0,1)$ and $d_X: X \times X \rightarrow \Rg$ as in \rd{def:metric_spaces:coproducts:0001}. Then $(X, d_X)$ is complete iff for each $\al \in A$, $(X_{\al}, d_{\al})$ is complete.
\end{ex}

\begin{proof}\
	\begin{itemize}
		\item $(\implies):$ \\
		Suppose that $(X, d_X)$ is complete. Let $\al \in A$ and $(x_n)_{n \in \N} \subset X_{\al}$. Suppose that $(x_n)_{n \in \N}$ is Cauchy in $(X_{\al}, d_{\al})$. \rex{ex:metric_spaces:coproducts:0003} implies that $\iota_{\al}: X_{\al} \rightarrow X$ is Lipschitz. Therefore $\iota_{\al}$ is uniformly continuous and \rex{ex:metric_spaces:introduction:0001.3} implies that $(\iota_{\al}(x_n))_{n \in \N}$ is Cauchy in $(X, d_X)$. Since $(X, d_X)$ is complete, there exists $(\al_0, x_0) \in X$ such that $(\iota_{\al}(x_n)) \rightarrow (\al_0, x_0)$ in $(X, d_X)$. \rex{ex:metric_spaces:coproducts:0002} and \rex{ex:topology:coproducts:0003.3} imply that $x_n \rightarrow x_0$. Since $(x_n)_{n \in \N} \subset X_{\al}$ with $(x_n)_{n \in \N}$ Cauchy in $(X_{\al}, d_{\al})$ is arbitrary, we have that for each $(x_n)_{n \in \N} \subset X_{\al}$, if $(x_n)_{n \in \N}$ is Cauchy, then there exists $x_0 \in X_{\al}$ such that $x_n \rightarrow x_0$. Hence $(X_{\al}, d_{\al})$ is complete. Since $\al \in A$ is arbitrary, we have that for each $\al \in A$, $(X_{\al}, d_{\al})$ is complete.
		\item $(\impliedby):$ \\
		Conversely, suppose that for each $\al \in A$, $(X_{\al}, d_{\al})$ is complete. Let $(\al_n, x_n)_{n \in \N} \subset X$. Suppose that $(\al_n, x_n)_{n \in \N}$ is Cauchy in $(X, d_X)$. \rex{ex:metric_spaces:completeness:0008} implies that there exists $\al_0 \in A$ and $j_0 \in \N$ such that 
		\begin{enumerate}
			\item for each $j \in \N$, $j \geq j_0$ implies that $\al_j = \al_0$ and $x_j \in X_{\al_0}$,
			\item $(x_{j'+j_0})_{j' \in \N}$ is Cauchy in $(X_{\al_0}, d_{\al_0})$. 
		\end{enumerate}
		Define $(y_j)_{j \in \N} \subset X_{\al_0}$ by $y_j \defeq x_{j + j_0}$. Since $(X_{\al_0}, d_{\al_0})$ is complete, and $(y_j)_{j \in \N}$ is Cauchy in $(X_{\al_0}, d_{\al_0})$, there exists $x_0 \in X_{\al_0}$ such that $y_n \rightarrow x_0$ in $(X, \MT_{d_{\al_0}})$. Thus $y_n \rightarrow x_0$ in $(X, \MT_{d^{\phi}_{\al_0}})$. 
		Let $\ep > 0$. Since $y_n \rightarrow x_0$ in $(X, \MT_{d^{\phi}_{\al_0}})$, there exists $N_0 \in \N$ such that for each $n \in \N$, $n \geq N_0$ implies that $d^{\phi}_{\al_0}(y_n, x_0) < \ep$. Set $N \defeq \max(j_0, N_0) + j_0$. Let $n \in \N$. Suppose that $n \geq N$. Then $n-j_0 \geq j_0, N_0$ and
		\begin{align*}
			d_X((\al_n, x_n), (\al_0, x_0)) 
			& = d_X((\al_0, x_n), (\al_0, x_0)) \\
			& = d^{\phi}(x_n, x_0) \\
			& = d^{\phi}(y_{n-j_0}, x_0) \\
			& < \ep. 
		\end{align*}
		Hence $(\al_n, x_n) \rightarrow (\al_0, x_0)$ in $(X, \MT_{d_X})$. Since $(\al_n, x_n)_{n \in \N} \subset X$ with $(\al_n, x_n)_{n \in \N}$ Cauchy is arbitrary, we have that for each $(\al_n, x_n)_{n \in \N} \subset X$, $(\al_n, x_n)_{n \in \N}$ is Cauchy implies that there exists $(\al_0, x_0) \in X$ such that $(\al_n, x_n) \rightarrow (\al_0, x_0)$ in $(X, \MT_{d_X})$. Thus $(X, \MT_{d_X})$ is complete.
	\end{itemize}
\end{proof}



















































\section{Compactness}


\begin{defn} \ld{def:metric_spaces:compactness:0001}
	Let $(X, d)$ be a metric space. Then $(X, d)$ is said to be \tbf{totally bounded} if for each $\ep >0$, there exists $\MU \subset \MT_d$ such that $\MU$ is an open cover of $X$, $\MU$ is finite and for each $U \in \MU$, $\diam(U) \leq \ep$. 
\end{defn}

\begin{ex} \lex{ex:metric_spaces:compactness:0002}
	Let $(X, d)$ be a metric space. Suppose that $X \neq \varnothing$. Then $X$ is is totally bounded iff for each $\ep >0$, there exists $n \in \N$ and $(x_j)_{j=1}^n \subset X$ such that $X \subset \bigcup\limits_{j=1}^n B(x_j, \ep)$. 
\end{ex}

\begin{proof}\
	\begin{itemize}
		\item $(\implies):$ \\
		Suppose that $X$ is totally bounded in $(X, d)$. Let $\ep > 0$. Define $\del >0$ by $\del \defeq \ep/2$. Then, there exists $\MU' \subset \MT_d$ such that $\MU'$ is an open cover of $X$, $\MU'$ is finite and for each $U \in \MU'$, $\diam(U) \leq \del$. Define $\MU \subset \MU'$ by $\MU \defeq \{U \in \MU': U \neq \varnothing\}$. Since $X \neq \varnothing$, $\MU \neq \varnothing$. Define $n \in \N$ by $n \defeq \# \MU$. Write $\MU = (U_j)_{j=1}^n$. Since for each $j \in [n]$, $U_j \neq \varnothing$, there exists $(x_j)_{j=1}^n \subset X$ such that for each $j \in [n]$, $x_j \in U_j$. Let $j \in [n]$ and $y \in U_j$. Then
		\begin{align*}
			d(x_j, y)
			& \leq \sup_{a,b \in U_j} d(a,b) \\
			& = \diam(U_j) \\
			& \leq \del \\
			& = \frac{\ep}{2} \\
			& < \ep. 
		\end{align*}
		Hence $y \in B(x_j, \ep)$. Since $y \in U_j$ is arbitrary, we have that for each $y \in U_j$, $y \in B(x_j, \ep)$. Thus $U_j \subset B(x_j, \ep)$. Since $j \in [n]$ is arbitrary, we have that for each $j \in [n]$, $U_j \subset B(x_j, \ep)$. Then 
		\begin{align*}
			X
			& \subset \bigcup\limits_{j=1}^n U_j \\
			& \subset \bigcup\limits_{j=1}^n B(x_j, \ep).
		\end{align*}
		\item $(\impliedby):$ \\
		Suppose that for each $\ep >0$, there exists $n \in \N$ and $(x_j)_{j=1}^n \subset X$ such that $X \subset \bigcup\limits_{j=1}^n B(x_j, \ep)$. Let $\ep >0$. Define $\del >0$ by $\del \defeq \ep/2$. By assumption, there exists $n \in \N$ and $(x_j)_{j=1}^n \subset X$ such that $X \subset \bigcup\limits_{j=1}^n B(x_j, \del)$. We note that for each $j \in [n]$,
		\begin{align*}
			\diam B(x_j, \del)
			& \leq 2 \del \\
			& = \ep. 
		\end{align*}
		Define $\MU \subset \MT_d$ by $\MU \defeq \{B(x_j, \del): j \in [n]\}$. Then $\MU$ is an open cover of $X$, $\MU$ is finite and for each $U \in \MU$, $\diam(U) \leq \ep$. 
	\end{itemize}
\end{proof}


\begin{defn} \ld{def:metric_spaces:compactness:0003}
	Let $(X, d)$ be a metric space and $A \subset X$. Then $A$ is said to be \tbf{totally bounded} in $(X, d)$ if $(A, d|_{A^2})$ is totally bounded.
\end{defn}

\begin{ex} \lex{ex:metric_spaces:compactness:0004}
	Let $(X, d)$ be a metric space and $A \subset X$. Suppose that $A \neq \varnothing$. Then $A$ is is totally bounded in $(X, d)$ iff for each $\ep >0$, there exists $n \in \N$ and $(x_j)_{j=1}^n \subset A$ such that $A \subset \bigcup\limits_{j=1}^n B_d(x_j, \ep)$. 
\end{ex}

\begin{proof}\
	Define $d_A: A \times A \rightarrow \Rg$ by $d_A \defeq d|_{A^2}$.
	\begin{itemize}
		\item $(\implies):$ \\
		Suppose that $A$ is totally bounded in $(X, d)$. By definition, $(A, d_A)$ is totally bounded. Let $\ep >0$. Since $A \neq \varnothing$, \rex{ex:metric_spaces:compactness:0002} implies that there exists $n \in \N$ and $(x_j)_{j=1}^n \subset A$ such that $A \subset \bigcup\limits_{j=1}^n B_{d_A}(x_j, \ep)$. \rex{ex:metric_spaces:subspaces:0001.1} then implies that
		\begin{align*}
			A 
			& \subset \bigcup\limits_{j=1}^n B_{d_A}(x_j, \ep) \\
			& = \bigcup\limits_{j=1}^n B_d(x_j, \ep) \cap A \\
			& \subset \bigcup\limits_{j=1}^n B_d(x_j, \ep).
		\end{align*}
		Since $\ep > 0$ is arbitrary, we have that for each $\ep >0$, there exists $n \in \N$ and $(x_j)_{j=1}^n \subset A$ such that $A \subset \bigcup\limits_{j=1}^n B_d(x_j, \ep)$
		\item $(\impliedby):$ \\
		Suppose that for each $\ep >0$, there exists $n \in \N$ and $(x_j)_{j=1}^n \subset A$ such that $A \subset \bigcup\limits_{j=1}^n B_d(x_j, \ep)$. Let $\ep >0$. Define $\del >0$ by $\del \defeq \ep/2$. By assumption, there exists $n \in \N$ and $(x_j)_{j=1}^n \subset A$ such that $A \subset \bigcup\limits_{j=1}^n B_d(x_j, \del)$. Define $\MU \subset \MT_{d_A}$ by $\MU \defeq \{B_d(x_j, \del) \cap A: j \in [n]\}$. Then 
		\begin{align*}
			A
			& \subset \bigg[ \bigcup\limits_{j=1}^n B_d(x_j, \del) \bigg] \cap A \\
			& = \bigcup\limits_{j=1}^n [B_d(x_j, \del) \cap A] \\
			& = \bigcup\limits_{U \in \MU} U.
		\end{align*}
		Let $U \in \MU$. Then there exists $j \in [n]$ such that $U = B_d(x_j, \del) \cap A$. Thus
		\begin{align*}
			\diam_{d_A} U 
			& = \sup_{a,b \in U} d_A(a,b) \\
			& = \sup_{a,b \in B_d(x, \del) \cap A} d(a,b) \\
			& \leq \sup_{a,b \in B_d(x, \del)} d(a,b) \\
			& = \diam_d B_d(x_j, \del) \\
			& \leq 2 \del \\
			& = \ep. 
		\end{align*}
		\tcr{make and cite exercise about diameter and subspaces mentioned in \rex{ex:metric_spaces:compactness:0005}}
		Hence $\MU$ is an open cover of $A$, $\MU$ is finite and for each $U \in \MU$, $\diam_{d_A} U \leq \ep$. By definition, $(A, d_A)$ is totally bounded and therefore $A$ is totally bounded in $(X, d)$. 
	\end{itemize}
\end{proof}

\begin{ex} \lex{ex:metric_spaces:compactness:0005}
	Let $(X, d)$ be a metric space and $A \subset X$. If $(X, d)$ is totally bounded, then $A$ is totally bounded in $(X, d)$. 
\end{ex}

\begin{proof}
	Suppose that $(X, d)$ is totally bounded. Define $d_A: A \times A \rightarrow \Rg$ by $d_A \defeq d|_{A^2}$. Let $\ep >0$. Then there exists $\MU \subset \MT_d$ such that $\MU$ is an open cover of $X$, $\MU$ is finite and for each $U \in \MU$, $\diam(U) \leq \ep$. 
	\begin{itemize}
		\item We note that \rex{ex:metric_spaces:subspaces:0001} implies that $\MT_d \cap A = \MT_{d_A}$ and \rex{ex:topology:subspaces:0012} then implies that $\MU \cap A$ is an open cover of $A$ in $(A, \MT_{d_A})$.
		\item Since $\MU$ is finite, $\MU \cap A$ is finite. 
		\item Since for each $U \in \MU$, $\diam_d(U) \leq \ep$, we have that for each $U \in \MU \cap A$, 
		\begin{align*}
			\diam_{d_A}(U)
			& = \sup_{a, b \in U} d_A(a,b) \\
			& = \sup_{a, b \in U} d(a,b) \\
			& = \diam_d(U).
		\end{align*}
		\tcr{make exercise in metric subspace section about how in general for $U \in \MP(X)$, $\diam_{d|_A} U \cap A \leq \diam_d U$ and for each $U \in \MP(A)$, $\diam_{d|_A}(U) = \diam_d(U)$, then reference in this proof.}
	\end{itemize} 
	Therefore $(A, d_A)$ is totally bounded and by definition, $A$ is totally bounded in $(X, d)$.
\end{proof}

\begin{ex} \lex{ex:metric_spaces:compactness:0006}
	Let $(X, d)$ be a metric space and $\MU \subset \MT_d$. Suppose that $(X, d)$ is totally bounded, $X \neq \varnothing$ and $\MU$ is an open cover for $X$. If for each $\MU_0 \subset \MU$, $\MU_0$ is finite implies that $\MU_0$ is not an open cover of $X$, then there exists $(x_n)_{n \in \N} \subset X$, $(V_n)_{n \in \N} \subset \MT_d$ and $(\MU_n)_{n \in \N} \in \MP(\MT_d)^{\N}$ such that for each $n \in \N$,
	\begin{itemize}
		\item $x_n \in V_n$, $V_{n+1} \subset V_n$ and $\diam V_n \leq 1/n$,
		\item $\MU_n = \MU \cap V_n$,
		\item for each $\MU_0 \subset \MU_n$, $\MU_0$ is finite implies that $\MU_0$ is not an open cover of $V_n$. 
	\end{itemize}
\end{ex}

\begin{proof}
	Suppose that for each $\MU_0 \subset \MU$, $\MU_0$ is finite implies that $\MU_0$ is not an open cover of $X$. We define $(x_n)_{n \in \N} \subset X$, $(V_n)_{n \in \N} \subset \MT_d$ and $(\MU_n)_{n \in \N} \in \MP(\MT_d)^{\N}$ inductively as follows:
	\begin{itemize}
		\item \tbf{Base Case:} \\ 
		Set $r_1 \defeq 2^{-1}$. Since $(X, d)$ is totally bounded, \rex{ex:metric_spaces:compactness:0004} implies that there exists $k_1 \in \N$ and $(x^{(1)}_j)_{j=1}^{k_1} \subset X$ such that $X \subset \bigcup\limits_{j=1}^{k_1} B(x^{(1)}_j, r_1)$. \rex{ex:set_theory:sets:0002} then implies that there exists $j_1 \in [k_1]$ such that for each $\MU_0 \subset \MU$, $\MU_0$ is finite implies that $\MU_0$ is not an open cover of $B(x^{(1)}_{j_1}, r_1)$. Define $x_1 \in X$, $V_1 \in \MT_d$ and $\MU_1 \subset \MT_d$ by $x_1 \defeq x^{(1)}_{j_1}$, $V_1 \defeq B(x_1, r_1)$ and $\MU_1 \defeq \MU \cap V_1$. We note the following:
		\begin{itemize}
			\item By construction, $V_1 \in \MT_d$ and 
			\begin{align*}
				\diam V_1
				& \leq 2 r_1 \\
				& = 1.
			\end{align*}
			\item Since $(X, d)$ is totally bounded, \rex{ex:metric_spaces:compactness:0005} implies that $V_1$ is totally bounded in $(X, d)$.
			\item By construction, $x_1 \in V_n$ and therefore $V_1 \neq \varnothing$.
			\item By construction $\MU_1 = \MU \cap V_1$. 
			\item Let $\MU_0 \subset \MU_1$. Suppose that $\MU_0$ is finite. For the sake of contradiction, suppose that $\MU_0$ is an open cover of $V_1$. Since $\MU_1 = \MU \cap V_1$, there exists $\MU_0^* \subset \MU$ such that $\MU_0^*$ is finite and $\MU_0 = \MU_0^* \cap V_1$. Then 
			\begin{align*}
				V_1 
				& \subset \bigcup\limits_{U \in \MU_0} U \\
				& = \bigcup\limits_{U \in \MU_0^*} U \cap V_1 \\
				& \subset \bigcup\limits_{U \in \MU_0^*} U.
			\end{align*}
			Hence $\MU_0^*$ is an open cover of $V_1$ which is a contradiction. Thus $\MU_0$ is not an open cover of $V_1$. Since $\MU_0 \subset \MU_1$ such that $\MU_0$ is finite is arbitrary, we have that for each $\MU_0 \subset \MU_1$, $\MU_0$ is finite implies that $\MU_0$ is not an open cover of $V_1$.
		\end{itemize} 
		\item \tbf{Induction Step:} \\  Let $n \geq 2$. Set $r_n \defeq (2n)^{-1}$. Suppose that  
		\begin{itemize}
			\item $V_{n-1} \in \MT_d$,
			\item $V_{n-1}$ is totally bounded in $(X,d)$,
			\item $V_{n-1} \neq \varnothing$,
			\item $\MU_{n-1} = \MU \cap V_{n-1}$,
			\item for each $\MU_0 \subset \MU_{n-1}$, $\MU_0$ is finite implies that $\MU_0$ is not an open cover of $V_{n-1}$.
		\end{itemize}
		Since $V_{n-1}$ is totally bounded in $(X,d)$, \rex{ex:metric_spaces:compactness:0004} implies that there exists $k_n \in \N$ and $(x^{(n)}_j)_{j=1}^{k_n} \subset V_{n-1}$ such that $V_{n-1} \subset \bigcup\limits_{j=1}^{k_n} B(x^{(n)}_j, r_n)$. Since $\MU_{n-1} = \MU \cap V_{n-1}$, $V_{n-1} \in \MT_d$ and $\MU$ is an open cover of $X$, we have that $\MU_{n-1}$ is an open cover of $V_{n-1}$. \rex{ex:set_theory:sets:0002} then implies that there exists $j_n \in [k_n]$ such that for each $\MU_0 \subset \MU_{n-1}$, $\MU_0$ is finite implies that $\MU_0$ is not an open cover of $B(x^{(n)}_{j_n}, r_n)$. Define $x_n \in X$, $V_n \in \MT_d$ and $\MU_n \subset \MT_d$ by $x_n \defeq x^{(n)}_{j_n}$, $V_n \defeq B(x_n, r_n) \cap V_{n-1}$ and $\MU_n \defeq \MU \cap V_n$. We note the following: 
		\begin{itemize}
			\item By construction $V_n \in \MT_d$, $V_n \subset V_{n-1}$ and 
			\begin{align*}
				\diam V_n
				& \leq 2 r_n \\
				& = n^{-1}.
			\end{align*} 
			\item Since $(X, d)$ is totally bounded, $V_n$ is totally bounded in $(X,d)$.
			\item By construction, $x_n \in V_n$ and therefore $V_n \neq \varnothing$. 
			\item By construction, $\MU_n = \MU \cap V_n$.
			\item Similarly to the base case, for each $\MU_0 \subset \MU_n$, $\MU_0$ is finite implies that $\MU_0$ is not an open cover of $V_n$.  \tcr{maybe write out details}
		\end{itemize} 
	\end{itemize}
\end{proof}


\begin{ex} \lex{ex:metric_spaces:compactness:0007}
	Let $(X, d)$ be a metric space.
	\begin{enumerate}
		\item $(X, \MT_d)$ is compact
		\item $(X, \MT_d)$ is sequentially compact
		\item $(X, d)$ is complete and totally bounded
	\end{enumerate}
	\tcr{rework with \rex{ex:metric_spaces:compactness:0004} in mind}
\end{ex}

\begin{proof}\
	\begin{enumerate}
		\item $(1) \implies (2):$ \\
		Immediate by \rex{ex:topology:compactness:sequent_compact:0003} 
		\item $(2) \implies (3)$: \\
		Suppose that $(X, \MT_d)$ is sequentially compact. 
		\begin{itemize}
			\item \tbf{(completeness):} \\
			Let $(x_n)_{n \in \N} \subset X$. Suppose that $(x_n)_{n \in \N}$ is Cauchy. Since $(X, \MT_d)$ is sequentially compact, there exists $(x_{n_k})_{k \in \N} \subset (x_n)_{n \in \N}$ and $x \in X$ such that $x_{n_k} \rightarrow x$. \rex{ex:metric_spaces:introduction:0001.2} implies that $x_n \rightarrow x$. Since $(x_n)_{n \in \N} \subset X$ with $(x_n)_{n \in \N}$ Cauchy is arbitrary, we have that for each $(x_n)_{n \in \N} \subset X$, $(x_n)_{n \in \N}$ is Cauchy implies that there exists $x \in X$ such that $x_n \rightarrow x$. Thus $(X, d)$ is complete. 
			\item \tbf{(total boundedness):} \\
			If $X = \varnothing$, then $X$ is totally bounded. Suppose that $X \neq \varnothing$. For the sake of contradiction, suppose that $(X, d)$ is not totally bounded. \rex{ex:metric_spaces:compactness:0004} then implies that there exists $\ep > 0$ such that for each $n \in \N$ and $(x_j)_{j=1}^n \subset X$, $X \not \subset \bigcup\limits_{j=1}^n B(x_j, \ep)$. Define $(x_n)_{n \in \N} \subset X$ inductively as follows:
			\begin{itemize}
				\item Since $X \neq \varnothing$, there exists $a \in X$. Define $x_1 \in X$ by $x_1 \defeq a$. 
				\item Let $n \geq 2$. By assumption, $X \not \subset \bigcup\limits_{j =1}^{n-1} B(x_j, \ep)$. Thus there exists $b \in X$ such that $x \in \bigcap\limits_{j=1}^{n-1} B(x_j, \ep)^c$. Define $x_n \in X$ by $x_n \defeq b$. 
			\end{itemize}
			By construction, for each $m, n$, $m \neq n$ implies that $d(x_m, x_n) \geq \ep$. Since $(X, d)$ is sequentially compact, there exists $(x_{n_k})_{k \in \N} \subset (x_n)_{n \in \N}$ and $x \in X$ such that $x_{n_k} \rightarrow x$. \rex{ex:metric_spaces:introduction:0001.1} implies that $(x_{n_k})_{k \in \N}$ is Cauchy. Thus there exists $K \in \N$ such that for each $k,l \in \N$, $k,l \geq K$ implies that $d(x_{n_k}, x_{n_l}) < \ep$. In particular,
			\begin{align*}
				\ep 
				& \leq d(x_{n_K}, x_{n_{K+1}}) \\
				& < \ep,
			\end{align*}
			which is a contradiction. Hence $(X, d)$ is totally bounded.
		\end{itemize}
		\item $(3) \implies (1)$: \\
		\tcr{REWORK and make content about obtaining $V_n, x_n$ a spearate exercise to make proof shorter!!!}
		Suppose that $(X, d)$ is complete and totally bounded. If $X = \varnothing$, then $(X, \MT)$ is compact. Suppose that $X \neq \varnothing$. For the sake of contradiction, suppose that $(X, \MT_d)$ is not compact. Then there exists $\MU \subset \MT_d$ such that $\MU$ is an open cover of $X$ and for each $\MU_0 \subset \MU$, $\MU_0$ is finite implies that $\MU_0$ is not an open cover of $X$. \rex{ex:metric_spaces:compactness:0006} then implies that there exist $(x_n)_{n \in \N} \subset X$, $(V_n)_{n \in \N} \subset \MT_d$ and $(\MU_n)_{n \in \N} \in \MP(\MT_d)^{\N}$ such that for each $n \in \N$,
		\begin{itemize}
			\item $x_n \in V_n$, $V_{n+1} \subset V_n$ and $\diam V_n \leq 1/n$,
			\item $\MU_n = \MU \cap V_n$,
			\item for each $\MU_0 \subset \MU_n$, $\MU_0$ is finite implies that $\MU_0$ is not an open cover of $V_n$. 
		\end{itemize}
		Let $\ep > 0$. Choose $N \in \N$ such that $2/N < \ep$. Let $m,n \in \N$. Suppose that $m,n \geq N$. Then $x_m, x_n, x_N \in V_N$ and therefore
		\begin{align*}
			d(x_m, x_n)
			& \leq d(x_m, x_N) + d(x_N, x_n) \\
			& \leq \diam(V_N) + \diam(V_N) \\
			& \leq 1/N + 1/N \\
			& = \frac{2}{N} \\
			& < \ep. 
		\end{align*}
		Since $\ep > 0$ is arbitrary, we have that for each $\ep >0$, there exists $N \in \N$ such that for each $m,n \in \N$, $m,n \geq N$ implies that $d(x_m, x_n) < \ep$. Thus $(x_n)_{n \in \N}$ is Cauchy. Since $(X, d)$ is complete, there exists $x \in X$ such that $x_n \rightarrow x$. Since for each $n \in \N$, $x_n \in V_n$ and $V_{n+1} \subset V_n$, \rex{}\tcr{make exercise in topology first countablitiy section} implies that $x \in \bigcap\limits_{n \in \N} V_n$. Since $\MU$ is an open cover of $X$, there exists $U \in \MU$ such that $x \in U$. Since $U \in \MT_d$, there exists $\ep >0$ such that $B(x, \ep) \subset U$. Choose $N \in \N$ such that $2/N < \ep$. In particular $x \in V_N$ and for each $y \in V_N$, 
		\begin{align*}
			d(y, x)
			& \leq d(y, x_N) + d(x_N, x) \\
			& < \frac{1}{N} + \frac{1}{N} \\
			& = \frac{2}{N} \\
			& < \ep. 
		\end{align*}
		Hence 
		\begin{align*}
			V_N 
			& \subset B(x, \ep) \\
			& \subset U. 
		\end{align*} 
		and therefore $V_N \subset U \cap V_N$. Define $\MU_0 \subset \MU_N$ by $\MU_0 \defeq \{U \cap V_N\}$. Then $\MU_0$ is finite and $\MU_0$ is an open cover of $V_N$ which is a contradiction. Hence $(X, \MT_d)$ is compact.
	\end{enumerate}   
\end{proof}


\begin{ex} \lex{ex:metric_spaces:compactness:0008}
	Let $(X, d)$ be a metric space, $U \in \MT_d$ and $x \in U$. 
	\begin{enumerate}
		\item If $x \in U'$, then $U \setminus \{x\}$ is not compact.
		\item If $x$ is a condensation point of $U$, then $U \setminus \{x\}$ is not compact.
	\end{enumerate} 
\end{ex}

\begin{proof}\
	\begin{enumerate}
		\item Suppose that $x \in U'$. Set $U_0 \defeq U \setminus \{x\}$. For the sake of contradiction, suppose that $U_0$ is compact. \rex{ex:metric_spaces:compact:0001} implies that $U_0$ is sequentially compact. Since $x \in U'$, \rex{ex:nets:0013} implies that there exists $(x_n)_{n \in \N} \subset U_0$ such that $x_n \rightarrow x$. \tcr{(\rex{ex:nets:0013} is for nets, need exercise to show existence of a subnet that is a sequence from a net)} Since $U_0$ is sequentially compact, there exists $(x_{n_k})_{k \in \N} \subset (x_n)_n$ and $x_0 \in U_0$ such that $x_{n_k} \rightarrow x_0$. Since $x_n \rightarrow x$, we have that $x_{n_k} \rightarrow x$. Since $X$ is Hausdorff,
		\begin{align*}
			x 
			& = x_0 \\
			& \in U_0,
		\end{align*}
		which is a contradiction. Thus $U_0$ is not compact. 
		\item Suppose that $x$ is a condensation point of $U$. \rex{31026.1} implies that $x \in U'$. The previous part then implies that $U \setminus \{x\}$ is not compact.
	\end{enumerate}
	
\end{proof}



























































\newpage
\section{The Baire Category Theorem}

\begin{ex} \lex{43001}
	Let $X$ be a complete metric space and $(U_n)_{n \in \N} \subset \MP(X)$. Suppose that for each $n \in \N$, $U_n$ is open and dense in $X$. Then $\bigcap\limits_{n \in \N} U_n$ is dense in X. \\
	\tbf{Hint:} Let $W \subset X$. Suppose that $W$ is open. Since $U_1$ is open and dense in $X$, \rex{31022} implies that $U_1 \cap W$ is open and nonempty. Hence there exists $x_1 \in U_1 \cap W$ and $r_1 \in (0, 2^{-1})$ such that $\bar{B}(x_1, r_1) \subset U_1 \cap W$. Inductively define $(x_n)_{n \in \N} \subset X$ and $(r_n)_{n \in \N} \subset (0,1)$.
\end{ex}

\begin{proof}
	Set $U \defeq \bigcap\limits_{n \in \N} U_n$. Let $W \subset X$. Suppose that $W$ is open and nonempty. Since $U_1$ is open and dense in $X$, \rex{31022} implies that $U_1 \cap W$ is open and nonempty. Hence there exists $x_1 \in U_1 \cap W$ and $r_1 \in (0, 2^{-1})$ such that $\bar{B}(x_1, r_1) \subset U_1 \cap W$. For $n \geq 2$, \rex{31022} implies that $U_n \cap B(x_{n-1}, r_{n-1})$ is open and nonempy. Hence there exists $x_n \in U_n \cap B(x_{n-1}, r_{n-1})$ and $r_n \in (0, 2^{-n})$ such that $\bar{B}(x_n, r_n) \subset U_n \cap B(x_{n-1}, r_{n-1})$. Note that for each $N,n \in \N$, if $n \geq N$, then by definition,
	\begin{align*}
		x_n 
		& \in B(x_n, r_n) \\
		& \subset U_n \cap B(x_{n-1}, r_{n-1}) \\
		& \subset \bigg( \bigcap_{j=N+1}^{n} U_j \bigg) \cap B(x_N, r_N)
	\end{align*}
	Let $\ep > 0$. Choose $N \in \N$ such that $2^{1-N} < \ep$. Let $n, m \in \N$. Suppose that $n, m \geq N$. Then 
	\begin{align*}
		d(x_n, x_m)
		& \leq d(x_n, x_N) + d(x_N, x_m) \\
		& \leq 2^{-N} + 2^{-N} \\
		& = 2^{1-N} \\
		& < \ep 
	\end{align*}
	Thus $(x_n)_{n \in \N}$ is Cauchy. Since $X$ is complete, there exists $x \in X$ such that $x_n \rightarrow x$. Let $n \in \N$. Since $(x_n)_{n \geq N} \subset \bar{B}(x_1, r_1)$, we have that 
	\begin{align*}
		x 
		& \in \bar{B}(x_1, r_1) \\
		& \subset W
	\end{align*}
	Similarly, for each $n \in \N$,
	\begin{align*}
		x_n 
		& \in U_n \cap \bar{B}(x_n, r_n) \\
		& \subset U_n
	\end{align*}
	which implies that $x \in U$. Hence $\bigcap\limits_{n \in \N} U_n \cap W \neq \varnothing$. Since $W$ is an arbitrary open nonempty subset of $X$, we have that for each $W \subset X$, if $W$ is open and nonempty, then $ U \cap W \neq \varnothing$. By definition, $W$ is dense in $X$.
\end{proof}

\begin{ex} \lex{43002}
	Let $X$ be a complete metric space and $(A_n)_{n \in \N} \subset \MP(X)$. If for each $n \in \N$, $A_n$ is nowhere dense, then $X \neq \bigcup\limits_{n \in \N} A_n$.   
\end{ex}

\begin{proof}
	Suppose that for each $n \in \N$, $A_n$ is nowhere dense. \rex{31024} and \rex{31025} imply that for each $n \in \N$, $(\cl A_n)^c$ is dense and open. For the sake of contradiction, suppose that $X = \bigcup\limits_{n \in \N} A_n$. Then $X = \bigcup\limits_{n \in \N} \cl A_n$. \rex{43001} implies that  $\varnothing = \bigcap\limits_{n \in \N} (\cl A_n)^c$ is dense in $X$. This is a contradiction. Hence $X \neq \bigcup\limits_{n \in \N} A_n$. 
\end{proof}

\begin{defn} \lex{43004}
	Let $X$ be a topological space. Set $\MD_{\MO}(X) = \{U \subset X: U \text{ is open and dense in $X$}\}$. Then $X$ is said to be a \tbf{Baire space} if for each $(U_n)_{n \in \N} \subset \MD_{\MO}(X)$, $\bigcap\limits_{n \in \N} U_n$ is dense in $X$. \\
\end{defn}

\begin{defn} \lex{43003}
	Let $X$ be a topological space. Set $\MD_{\MN}(X) = \{U \subset X: U \text{ is nowhere dense in $X$}\}$. Let $E \subset X$. Then $E$ is said to be \tbf{meager} in $X$ if there exist $(A_n)_{n \in \N} \subset \MD_{\MN}(X)$ such that $E = \bigcup\limits_{n \in \N} A_n$.
\end{defn}

\begin{thm} \tbf{Baire Category Theorem:} \\
	Let $X$ be a complete metric space. Then 
	\begin{enumerate}
		\item $X$ is a Baire space
		\item $X$ is not meager
	\end{enumerate} 
\end{thm}

\begin{proof}
	Immediate by \rex{43001} and \rex{43002}.
\end{proof}

\begin{defn}
	content...
\end{defn}









































































\newpage
\section{Metrizable Spaces}

\subsection{Introduction}

\begin{defn} \ld{def:metric_spaces:metrizable_spaces:0001}
	Let $(X, \MT)$ be a topological space. Then $(X, \MT)$ is said to be \tbf{metrizable} if there exists a metric $d:X \times X \rightarrow \Rg$ such that $\MT = \MT_d$.
\end{defn}

\begin{defn} \ld{def:metric_spaces:metrizable_spaces:0001.0001}
	We define the \tbf{Hilbert cube}, denoted $\H$, by $\H \defeq [0,1]^{\N}$.
\end{defn}

\begin{ex} \lex{ex:metric_spaces:metrizable_spaces:0001.001}
	Let $(X, \MT)$ be a topological space. Suppose that $(X, \MT)$ is separable and metrizable. Then there exists $B \subset \H$ and $f: X \rightarrow B$ such that $f$ is a homeomorphism. \\
	\tbf{Hint:} Choose $(a_n)_{n \in \N} \subset X$ and $d:X \rightarrow [0,1]$ such that $(a_n)_{n \in \N}$ is dense in $(X, \MT)$ and $\MT_d = \MT$. Define $f:X \rightarrow \H$ by $f(x) \defeq (d(x, a_j))_{j \in \N}$.
\end{ex}

	\begin{proof}
		Since $(X, \MT)$ is metrizable, \rex{} \tcr{(reference exercise here)} implies that there exists a metric $d:X \times X \rightarrow [0,1]$ such that $\MT = \MT_d$. Since $(X, \MT)$ is separable, there exists $(a_j)_{j \in \N}$ such that $(a_j)_{j \in \N}$ is dense in $(X, \MT)$. Define $B \subset \H$ and $f: X \rightarrow B$ by $B \defeq \{(d(x, a_j))_{j \in \N}: x \in X\}$ and $f(x) \defeq (d(x, a_j))_{j \in \N}$. 
		\begin{itemize}
			\item Let $x,y \in X$. Suppose that $f(x) = f(y)$. Then $(d(x, a_j))_{j \in \N} = (d(y, a_j))_{j \in \N}$. Since $(a_j)_{j \in \N}$ is dense in $(X, \MT)$, there exists a subsequence $(a_{j_k})_{k \in \N} \subset (a_j)_{j \in \N}$ such that $a_{j_k} \rightarrow x$. Thus 
			\begin{align*}
				d(y, a_{j_k}) 
				& = d(x, a_{j,k}) \\
				& \rightarrow 0.
			\end{align*}
			Thus $a_{j_k} \rightarrow y$. Since $(X, \MT)$ is Hausdorff, $a_{j_k} \rightarrow x$ and $a_{j_k} \rightarrow y$, we have that $x = y$. Thus $f(x) = f(y)$ implies that $x = y$. Since $x,y \in X$ are arbitrary, we have that for each $x,y \in X$, $f(x) = f(y)$ implies that $x = y$. Hence $f$ is injective.
			\item By construction $f$ is surjective. 
		\end{itemize}
		Hence $f$ is a bijection.
		\begin{itemize}
			\item Let $(x_n)_{n \in \N} \subset X$ and $x \in X$. Suppose that $x_n \rightarrow x$. Since $d: X \times X \rightarrow [0,1]$ is continuous, we have that for each $j \in \N$, $d(x_n, a_j) \rightarrow d(x, a_j)$ in $[0,1]$. Thus  
			\begin{align*}
				f(x_n)
				& = (d(x_n, a_j))_{j \in \N} \\
				& \rightarrow (d(x, a_j))_{j \in \N} \\
				& = f(y)
			\end{align*}
			in $\H$. Thus $x_n \rightarrow x$ implies that $f(x_n) \rightarrow f(x)$. Since $(x_n)_{n \in \N} \subset X$ and $x \in X$ are arbitrary, we have that for each $(x_n)_{n \in \N} \subset X$ and $x \in X$, $x_n \rightarrow x$ implies that $f(x_n) \rightarrow f(x)$. Hence $f$ is continuous.
			\item Let $(r_n)_{n \in \N} \subset B$ and $r \in B$. Suppose that $r_n \rightarrow r$. Define $(x_n)_{n \in \N} \subset X$ and $x \in X$ by $x_n \defeq f^{-1}(r_n)$ and $x \defeq f^{-1}(r)$. Since 
			\begin{align*}
				(d(x_n, a_j))_{j \in \N}
				& = r_n \\
				& \rightarrow r \\
				& = (d(x, a_j))_{j \in \N},
			\end{align*}
			we have that for each $j \in \N$, 
			\begin{align*}
				d(x_n, a_j)
				& = \pi_j(r_n) \\
				& \rightarrow \pi_j(r) \\
				& = d(x, a_j).
			\end{align*} 
			Since $(a_j)_{j \in \N}$ is dense in $(X, \MT)$, there exists a subsequence $(a_{j_k})_{k \in \N} \subset (a_j)_{j \in \N}$ such that $a_{j_k} \rightarrow x$. Let $\ep > 0$. Then $\ep/3 > 0$. Since $a_{j_k} \rightarrow x$, there exists $K \in \N$ such that for each $k \in \N$, $k \geq K$ implies that $d(a_{j_k}, x) < \ep/3$. Since $d(x_n, a_{j_K}) \rightarrow d(x, a_{j_K})$, there exists $N \in \N$ such that for each $n \in \N$, $n \geq N$ implies that $|d(x_n, a_j) - d(x, a_j)| < \ep/3$. Let $n \in \N$. Suppose that $n \geq N$. Then 
			\begin{align*}
				d(f^{-1}(r_n), f^{-1}(r))
				& = d(x_n, x) \\
				& \leq d(x_n, a_{j_K}) + d(a_{j_K}, x) \\
				& < \bigg[ d(x, a_{j_K}) + \frac{\ep}{3} \bigg] + d(a_{j_K}, x) \\
				& = 2 d(x, a_{j_K}) + \frac{\ep}{3} \\
				& < \frac{2 \ep}{3} + \frac{\ep}{3} \\
				& = \ep.
			\end{align*}
			Since $\ep > 0$ is arbitrary, we have that for each $\ep > 0$, there exists $N \in \N$ such that for each $n \in \N$, $n \geq N$ implies that $d(f^{-1}(r_n), f^{-1}(r)) < \ep$. Hence $f^{-1}(r_n) \rightarrow f^{-1}(r)$. Since $(r_n)_{n \in \N} \subset B$ and $r \in B$ are arbitrary, we have that for each $(r_n)_{n \in \N} \subset B$ and $r \in B$, $r_n \rightarrow r$ implies that $f^{-1}(r_n) \rightarrow f^{-1}(r)$. Hence $f^{-1}$ is continuous. 
		\end{itemize} 
		Thus $f$ is a homeomorphism. 
	\end{proof}

\begin{defn} \ld{def:metric_spaces:metrizable_spaces:0002}
	Let $(X, \MT)$ be a topological space. Then $(X, \MT)$ is said to be \tbf{completely metrizable} if there exists a metric $d:X \times X \rightarrow \Rg$ such that $\MT = \MT_d$ and $(X, d)$ is complete.
\end{defn}

\begin{ex} \lex{ex:metric_spaces:metrizable_spaces:0003}
	Let $(X, \MT)$ be a topological space and $C \subset X$. Suppose that $(X, \MT)$ is completely metrizable. If $C$ is closed, then $(C, \MT \cap C)$ is completely metrizable. Then $(X, \MT)$ is said to be \tbf{completely metrizable} if there exists a metric $d:X \times X \rightarrow \Rg$ such that $\MT = \MT_d$ and $(X, d)$ is complete.
\end{ex}

\begin{proof}
	Suppose that $C$ is closed. Since $(X, \MT)$ is completely metrizable, there exists a metric $d:X \times X \rightarrow \Rg$ such that $\MT = \MT_d$ and $(X, d)$ is complete. \rex{ex:metric_spaces:completeness:0004} implies that $(C, d|_{C \times C})$ is complete. \rex{ex:metric_spaces:subspaces:0001} implies that $\MT \cap C = \MT_{d|_{C \times C}}$. Thus $(C, \MT \cap C)$ is completely metrizable.
\end{proof}

\begin{ex} \lex{ex:metric_spaces:metrizable_spaces:0003.1}
	Let $(X, \MT_X)$ and $(Y, \MT_Y)$ be topological spaces and $f:X \rightarrow Y$ a  $(\MT_X, \MT_Y)$-homeomorphism. Then $(X, \MT_X)$ is completely metrizable iff $(Y, \MT_Y)$ is completely metrizable.
\end{ex}

\begin{proof}\
	\begin{itemize}
		\item $(\implies):$ \\
		Suppose that $(X, \MT_X)$ is completely metrizable. Then there exists a metric $d_X: X \times X \rightarrow \Rg$ such that $\MT_X = \MT_{d_X}$ and $(X, d_X)$ is complete. Set $d_Y \defeq (f^{-1})^*d_X$.
		\begin{itemize}
			\item \rex{ex:metric_spaces:introduction:0024.6} implies that $\MT_Y = \MT_{d_Y}$. So $(Y, \MT_Y)$ is metrizable.
			\item \rex{ex:metric_spaces:introduction:0024.4} implies that $f^{-1}$ is a $(d_Y, d_X)$-isometry. Since $(X, d_X)$ is complete, \rex{ex:metric_spaces:completeness:0002.3} implies that $(Y, d_Y)$ is complete.
		\end{itemize}
		Hence $(Y, \MT_Y)$ is completely metrizable.
		\item $(\impliedby):$ \\
		Similar to $(\implies)$.
	\end{itemize}
\end{proof}


























\subsection{Metrizability of Subspaces}

\begin{ex} \lex{ex:metric_spaces:metrizable_spaces:0004}
	Let $(X, \MT)$ be a topological space and $C \subset X$. Suppose that $(X, \MT)$ is completely metrizable. If $C$ is closed in $(X, \MT)$, then $(C, \MT \cap C)$ is completely metrizable. 
\end{ex}

\begin{proof}
	Suppose that $C$ is closed in $(X, \MT)$. Since $(X, \MT)$ is completely metrizable, there exists a metric $d: X \times X \rightarrow	\Rg$ such that $\MT = \MT_d$ and $(X, d)$ is complete. \rex{ex:metric_spaces:completeness:0004} implies that $(C, d|_{C \times C})$ is complete. \rex{ex:metric_spaces:subspaces:0001} implies that $\MT_{d|_{C \times C}} = \MT_d \cap C$. Hence $(C, \MT_d \cap C)$ is completely metrizable. 
\end{proof}


























\subsection{Metrizability of Product Spaces}

\begin{ex}
	
\end{ex}

\begin{ex} \lex{ex:metric_spaces:metrizable_spaces:0005}
	Let $(X_n, \MT_n)_{n \in \N}$ be a collection of topological spaces. Suppose that for each $n \in \N$, $(X_n, \MT_n)$ is completely metrizable. Then $(\prod\limits_{n \in \N} X_n, \bigotimes\limits_{n \in \N} \MT_n)$ is completely metrizable.  
\end{ex}

\begin{proof}
	Define $\phi: \Rg \rightarrow \Rg$, $X$ and $\MT$ as in \rd{def:metric_spaces:product_spaces:0001}. Since for each $n \in \N$, $(X_n, \MT_n)$ is completely metrizable, for each $n \in \N$, there exists a metric $d_n:X_n \times X_n \rightarrow \Rg$ such that $(X_n, d_n)$ is a complete metric space. \rex{ex:metric_spaces:completeness:0006} implies that $(X, d_X)$ is complete. Since \rex{ex:metric_spaces:product_spaces:0002} implies that $\MT = \MT_{d_X}$, we have that $(X, \MT)$ is completely metrizable. 
\end{proof}



























\subsection{Metrizability of Coproduct Spaces}


\begin{ex} \lex{ex:metric_spaces:metrizable_spaces:0006}
	Let $(X_{\al}, \MT_{\al})_{\al \in A}$ be a collection of topological spaces. Suppose that for each ${\al} \in A$, $(X_{\al}, \MT_{\al})$ is completely metrizable. Then $(\prod\limits_{\al \in A} X_{\al}, \bigoplus\limits_{\al \in A} \MT_\al)$ is completely metrizable.  
\end{ex}

\begin{proof}
	Define $\phi: \Rg \rightarrow \Rg$, $X$ and $\MT$ as in \rd{def:metric_spaces:coproducts:0001}. Since for each $\al \in A$, $(X_{\al}, \MT_{\al})$ is completely metrizable, for each $\al \in A$, there exists a metric $d_{\al}:X_{\al} \times X_{\al} \rightarrow \Rg$ such that $(X_{\al}, d_{\al})$ is a complete metric space. Define $d_X:X \times X \rightarrow \Rg$ as in \rd{def:metric_spaces:coproducts:0001}. \rex{ex:metric_spaces:completeness:0009} implies that $(X, d_X)$ is complete. Since \rex{ex:metric_spaces:coproducts:0002} implies that $\MT = \MT_{d_X}$, we have that $(X, \MT)$ is completely metrizable. 
\end{proof}























\subsection{Metrizability of Projective Limits}

\begin{ex} \lex{ex:metric_spaces:metrizable_spaces:0007}
	Let $(J, {\leq})$ be a directed set and $((X_j, \MT_j)_{j \in J}, (\pi_{i,j})_{(i,j) \in {\leq}})$ be a $\Top$-projective system. Set $(X, (\pi_j)_{j \in J}) \defeq \varprojlim\limits_{j \in J} ((X_j, \MT_j)_{j \in J}, (\pi_{i,j})_{(i,j) \in {\leq}})$. Suppose that $J$ is countable and for each $j \in J$, $(X_j, \MT_j)$ is completely metrizable. Then $(X, \MT)$ is completely metrizable.
\end{ex}

\begin{proof}
	Since $J$ is countable, \rex{ex:metric_spaces:metrizable_spaces:0005} implies that $(\prod\limits_{j \in J} X_j, \bigotimes\limits_{j \in J} \MT_j)$ is completely metrizable. Since for each $j \in J$, $(X_j, \MT_j)$ is metrizable, for each $j \in J$, $(X_j, \MT_j)$ is Hasudorff. \rex{ex:topology:proj_limits:0004} implies that $X$ is closed in $(\prod\limits_{j \in J} X_j, \bigotimes\limits_{j \in J} \MT_j)$. Since $\MT = \bigg( \bigotimes\limits_{j \in J} \MT_j \bigg) \cap X$, \rex{ex:metric_spaces:metrizable_spaces:0004} implies that $(X, \MT)$ is completely metrizable.
\end{proof}







































\newpage
\section{Polish Spaces}

\subsection{Introduction}

\begin{defn} \ld{def:metric_spaces:polish_spaces:0001}
	Let $(X, \MT)$ be a topological space. Then $(X, \MT)$ is said to be a \tbf{Polish} space if 
	\begin{enumerate}
		\item $(X, \MT)$ is completely metrizable
		\item $(X, \MT)$ is separable
	\end{enumerate}
\end{defn}

\begin{ex} \lex{ex:metric_spaces:polish_spaces:0002}
	Let $(X, \MT)$ be a Polish space and $C \subset X$. If $C$ is closed in $(X, \MT)$, then $(C, \MT \cap C)$ is a Polish space.
\end{ex}

\begin{proof}
	Suppose that $C$ is closed. 
	\begin{enumerate}
		\item Since $(X, \MT)$ is completely metrizable and $C$ is closed, \rex{ex:metric_spaces:metrizable_spaces:0004} implies that $(C, \MT \cap C)$ is completely metriable.  
		\item Since $(X, \MT)$ is separable, \rex{ex:metric_spaces:subspaces:0002} implies that $(C, \MT \cap C)$ is separable.
	\end{enumerate}
	Therefore $(C, \MT \cap C)$ is a Polish space.
\end{proof}

\begin{ex} \lex{ex:metric_spaces:polish_spaces:0003}
	Let $(X_n, \MT_n)_{n \in \N}$ be a collection of Polish spaces. Then $(\prod\limits_{n \in \N} X_n, \bigotimes\limits_{n \in \N} \MT_n)$ is a Polish space.  
\end{ex}

\begin{proof} Since for each $n \in \N$, $(X_n, \MT_n)$ is a Polish space, we have that for each $n \in \N$, $(X_n, \MT_n)$ is completely metrizable and $(X_n, \MT_n)$ is separable. Set $X \defeq \prod\limits_{n \in \N} X_n$ and $\MT \defeq \bigotimes\limits_{n \in \N} \MT_{n}$. 
	\begin{enumerate}
		\item \rex{ex:metric_spaces:metrizable_spaces:0005} implies that $(X, \MT)$ is completely metrizable. 
		\item Since for each $n \in \N$, $(X_n, \MT_n)$ is separable, \rex{ex:metric_spaces:introduction:0025} implies that for each $n \in \N$, $(X_n, \MT_n)$ is second-countable. \rex{ex:topology:countability:0014} then implies that $(X, \MT)$ is second-countable. Another application of \rex{ex:metric_spaces:introduction:0025} implies that $(X, \MT)$ is separable. 
	\end{enumerate}
	Thus $(X, \MT)$ is a Polish space.
\end{proof}

\begin{ex} \lex{ex:metric_spaces:polish_spaces:0003.1}
	Let $(X_n, \MT_n)_{n \in \N}$ be a collection of Polish spaces. Then $(\coprod\limits_{n \in \N} X_n, \bigoplus\limits_{n \in \N} \MT_n)$ is Polish.  
\end{ex}

\begin{proof} 
	Set $X \defeq \coprod\limits_{n \in \N} X_n$ and $\MT \defeq \bigoplus\limits_{n \in \N} \MT_n$. Since for each $n \in \N$, $(X_n, \MT_n)$ is a Polish space, for each $n \in \N$, $(X_n, \MT_n)$ is completely metrizable and second-countable. \rex{ex:metric_spaces:metrizable_spaces:0006} implies that $(X, \MT)$ is completely metrizable and \rex{ex:topology:countability:0015} implies that $(X, \MT)$ is second-countable. Hence $(X, \MT)$ is a Polish space.
\end{proof}

\begin{ex} \lex{ex:metric_spaces:polish_spaces:0003.2}
	Let $(J, {\leq})$ be a directed set and $((X_j, \MT_j)_{j \in J}, (\pi_{i,j})_{(i,j) \in {\leq}})$ be a $\Top$-projective system. Set $(X, (\pi_j)_{j \in J}) \defeq \varprojlim\limits_{j \in J} ((X_j, \MT_j)_{j \in J}, (\pi_{i,j})_{(i,j) \in {\leq}})$. Suppose that $J$ is countable and for each $j \in J$, $(X_j, \MT_j)$ is a Polish space. Then $(X, \MT)$ is a Polish space.
\end{ex}

\begin{proof}
	Since for each $j \in J$, $(X_j, \MT_j)$ is a polish space, for each $j \in J$, $(X_j, \MT_j)$ is completely metrizable. \rex{ex:metric_spaces:metrizable_spaces:0007} that $(X, \MT)$ is completely metrizable. Since for each $j \in J$, $(X_j, \MT_j)$ is separable, \rex{ex:metric_spaces:introduction:0025} implies that for each $j \in J$, $(X_j, \MT_j)$ is second-countable. Since $J$ is countable, \rex{ex:topology:countability:0016} implies that $(X, \MT)$ is second-countable. Another application of \rex{ex:metric_spaces:introduction:0025} implies that $(X, \MT)$ is separable. Hence $(X, \MT)$ is a Polish space.
\end{proof}



\begin{ex} \lex{ex:metric_spaces:polish_spaces:0004}
	Let $(X, \MT_X)$ and $(Y, \MT_Y)$ be topological spaces and $f:X \rightarrow Y$ a $(\MT_X, \MT_Y)$-homeomorphism. Then $(X, \MT_X)$ is a Polish space iff $(Y, \MT_Y)$ is a Polish space.
\end{ex}

\begin{proof}\
	\begin{itemize}
		\item $(\implies):$ \\
		Suppose that $(X, \MT_X)$ is a Polish space. Then $(X, \MT_X)$ is completely metrizable and $(X, \MT_X)$ is separable. 
		\begin{enumerate}
			\item \rex{ex:metric_spaces:metrizable_spaces:0003.1} implies that $(Y, \MT_Y)$ is completely metrizable.
			\item Since $(X,\MT_X)$ and $(Y, \MT_Y)$ are metrizable, the exist metrics $d_X: X \times X \rightarrow \Rg$ and $d_Y: Y \times Y \rightarrow \Rg$ on $X$ and $Y$ respectively such that $\MT_X = \MT_{d_X}$ and $\MT_Y = \MT_{d_Y}$. Since $(X, \MT_X)$ is separable and $f$ is a $(\MT_X, \MT_Y)$-homeomorphism, we have that \rex{ex:metric_spaces:introduction:0026} implies that $(Y, \MT_Y)$ is separable.
		\end{enumerate}
		Thus $(Y, \MT_Y)$ is a Polish space.
		\item $(\impliedby):$ \\
		Similar to $(\implies)$.
	\end{itemize}
\end{proof}

\begin{ex} \lex{ex:metric_spaces:polish_spaces:0005}
	Let $(X, \MT)$ be a Polish space and $U \in \MT$. Then $(U, \MT \cap U)$ is a Polish space. \\
	\tbf{Hint:} \rex{ex:metric_spaces:introduction:0024.1}
\end{ex}

\begin{proof}
	Define $C \subset X \times \R$ and $f:U \rightarrow C$ as in \rex{ex:metric_spaces:introduction:0024.1}. \rex{ex:metric_spaces:introduction:0024.1} implies that $f$ is a $(\MT \cap U, (\MT \otimes \MT_{\R}) \cap C)$-homeomorphism and $C$ is closed. \rex{ex:metric_spaces:polish_spaces:0003} implies that $(X \times \R, \MT \otimes \MT_{\R})$ is a Polish space. \rex{ex:metric_spaces:polish_spaces:0004} implies that $(C, (\MT \otimes \MT_{\R}) \cap C)$ is a Polish space.
\end{proof}

\begin{ex} \lex{ex:metric_spaces:polish_spaces:0006}
	Let $(X, \MT)$ be a Polish space and $E \subset X$. Then $E$ is a $G_{\del}$-set iff $(E, \MT \cap E)$ is a Polish space. \\
	\tbf{Hint:} \rex{ex:metric_spaces:introduction:0024.2}
\end{ex}

\begin{proof}\
	\begin{itemize}
		\item $(\implies):$ \\
		Suppose that $E$ is a $G_{\del}$-set. Set $\MS \defeq \MT \otimes (\MT_{\R})^{\otimes \N}$. Since $E$ is a $G_{\del}$-set, there exists $(U_n)_{n \in \N} \subset \MT$ such that $E = \bigcap\limits_{n \in \N} U_n$. \rex{ex:metric_spaces:introduction:0024.2} implies that there exists $C \subset X \times \R^{\N}$ and $f:E \rightarrow C$ such that $f$ is a $(\MT \cap E, \MS \cap C)$-homeomorphism and $C$ is closed in $X \times \R^{\N}$. \rex{ex:metric_spaces:polish_spaces:0003} implies that $X \times \R^{\N}$ is a Polish space. Since $C$ is closed in $X \times \R^{\N}$, \rex{ex:metric_spaces:polish_spaces:0002} implies that $(C, \MS \cap C)$ is a Polish space. \rex{ex:metric_spaces:polish_spaces:0004} implies that $(E, \MT \cap E)$ is a Polish space.
		\item $(\impliedby):$ \\
		Conversely, suppose that $(E, \MT \cap E)$ is a Polish space. \tcr{FINISH!!!}
	\end{itemize}
\end{proof}

\begin{ex} \lex{ex:metric_spaces:polish_spaces:0007}
	Let $(X, \MT)$ be a Polish space. Set $\Del_{X^{\N}} \defeq \{x \in X: \text{ for each $m,n \in \N$, $\pi_m(x) = \pi_n(x)$}\}$. Then $(\Del_{X^{\N}}, \MT^{\otimes \N} \cap \Del_{X^{\N}})$ is a Polish space.
	\tbf{Hint:} \rex{ex:topology:separation:0006}
\end{ex}

\begin{proof}
	Since $(X, \MT)$ is Polish, $(X, \MT)$ is Hausdorff. \rex{ex:metric_spaces:polish_spaces:0003} implies that $(X^{\N}, \MT^{\otimes \N})$ is a Polish space. and \rex{ex:topology:separation:0006} implies that $\Del_{X^{\N}}$ is closed in $(X^{\N}, \MT^{\otimes \N})$. \rex{ex:metric_spaces:polish_spaces:0002} then implies that $(\Del_{X^{\N}}, \MT^{\otimes \N} \cap \Del_{X^{\N}})$ is a Polish space. 
\end{proof}

\begin{ex} \lex{ex:metric_spaces:polish_spaces:0007.1}
	We have that $\H$ is a Polish space. 
\end{ex}

\begin{proof}
	Since $[0,1]$ is a Polish space, \rex{ex:metric_spaces:polish_spaces:0003} implies that $\H$ is a Polish space. 
\end{proof}

\begin{ex} \lex{ex:metric_spaces:polish_spaces:0008}
	Let $(X, \MT)$ be a Polish space. Then there exists $B \subset \H$ and $f: X \rightarrow B$ such that $B$ is a $G_{\del}$ set and $f$ is a homeomorphism. \\
	\tbf{Hint:} \rex{ex:metric_spaces:metrizable_spaces:0001.001} 
\end{ex}

\begin{proof}
	Define $B \subset \H$ and $f: X \rightarrow B$ as in \rex{ex:metric_spaces:metrizable_spaces:0001.001}. Then $f$ is a homeomorphism. Since $X$ is a polish space, \rex{ex:metric_spaces:polish_spaces:0004} implies that $B$ is a Polish space. \rex{ex:metric_spaces:polish_spaces:0006} implies that $B$ is a $G_{\del}$-set. 
\end{proof}























































\newpage
\section{Ultrametric Spaces}
\subsection{Introduction}

Ultrametric spaces are given by sequences of partitions of $X$, $(\MP_j)_{j \in \N}$, where for each $j$ and $E \in \MP_j$, there exists $\ME \subset \MP_{j+1}$ such that $E = \bigcup_{F \in \MR} F$. Then set $d(x, y) = 2^{-n}$ if $n = \max(j \in \N: \text{there exists $E \in \MP_j:$ such that $x,y \in E$})$.

\begin{defn} \ld{def:metric_spaces:ultra_metric:0001}
	Let $X$ be a set and $d:X \times X \rightarrow \Rg$. Then $d$ is said to be and \tbf{ultrametric on $X$} if for each $x, y, z \in X$,
	\begin{enumerate}
		\item \tbf{(symmetry):} $d(x,y) = d(y,x)$
		\item \tbf{(definiteness):} $d(x,y) = 0$ iff $x= y$
		\item \tbf{(strong triangle inequality):} $d(x,z) \leq \max(d(x,y), d(y,z))$ 
	\end{enumerate} 
\end{defn}

\begin{ex} \lex{ex:metric_spaces:ultra_metric:0002} 
	Let $X$ be a set and $d$ an ultrametric on $X$. Then $d$ is a metric on $X$.
\end{ex}

\begin{proof}
	Let $x,y,z \in X$. Since $(d(x,y) , d(y,z) \geq 0$, we have that $d(x,y), d(y,z) \leq  d(x,y) + d(y,z)$. Therefore
	\begin{align*}
		d(x,y)
		& \leq \max(d(x,y) , d(y,z)) \\
		& \leq  d(x,y) + d(y,z)
	\end{align*}
\end{proof}

\begin{defn} \ld{def:metric_spaces:ultra_metric:0003}
	Let $X$ be a set and $d$ an ultrametric on $X$. Then $(X, d)$ is said to be an \tbf{ultrametric space}.
\end{defn}

\begin{ex} \lex{ex:metric_spaces:ultra_metric:0004} \tbf{Isosceles Triangle Property:} \\
	Let $(X,d)$ be an ultrametric space and $x,y, z \in X$. 
	\begin{enumerate}
		\item If $d(x,y) < d(y,z)$, then 
		\begin{align*}
			\max(d(x,y), d(x,z))
			& = d(y,z) \\
			& = d(x,z) \\
			& = \max(d(x,y), d(y,z)).
		\end{align*}
		\item If $d(x,y) \neq d(y,z)$, then $d(x,z) = \max(d(x,y), d(y,z))$.
	\end{enumerate} 
\end{ex}

\begin{proof}\
	\begin{enumerate}
		\item Suppose that $d(x,y) < d(y,z)$. Then 
		\begin{align*}
			d(x,z)
			& \leq \max(d(x,y), d(y,z)) \\
			& = d(y,z)
		\end{align*}
		For the sake of contradiction, suppose that $d(x,z) \leq d(x,y)$. Since $d(x,y) < d(y,z)$, we have that 
		\begin{align*}
			d(y,z) 
			& \leq \max(d(x,y), d(x,z)) \\
			& = d(x, y) \\
			& < d(y, z)
		\end{align*}
		which is a contradictiton. Hence $d(x, y) < d(x,z)$. Therefore, we have that 
		\begin{align*}
			d(x,z)
			& \leq \max(d(x,y), d(y,z)) \\
			& = d(y,z) \\
			& \leq \max(d(y, x), d(x, z)) \\
			& = d(x, z) \\
		\end{align*}
		Hence
		\begin{align*}
			\max(d(x,y), d(x,z))
			& = d(y,z) \\
			& = d(x,z) \\
			& = \max(d(x,y), d(y,z))
		\end{align*}
		and we have the following isosceles triangle: 
		\[ 
		\begin{tikzcd}
			& z \arrow[dddr, dash, "{d(x, z)}"] \arrow[dddl, dash, "{d(y,z)}"']& \\
			& & \\
			& & \\
			y  \arrow[rr, dash, "{d(x,y)}"']  & & x 
		\end{tikzcd}
		\] 
		\item Suppose that $d(x,y) \neq d(y,z)$. \\
		If $d(x,y) < d(y,z)$, then part $(1)$ implies that $d(x,z) = \max(d(x,y), d(y,z))$. \\
		Suppose that $d(y,z) < d(x,y)$. Then $d(z,y) < d(y, x)$. If we permute $x$ and $z$ in part $(1)$, we see that
		\begin{align*}
			d(x,z) 
			& = \max(d(z,y), d(y,x)) \\
			& = \max(d(x,y), d(y,z)).
		\end{align*}
	\end{enumerate}
\end{proof}

\begin{defn} \ld{def:metric_spaces:ultra_metric:0005}
	Let $(X,d)$ be an ultrametric space and $r > 0$. We define the 
	\begin{itemize}
		\item \tbf{open $r$-ball relation} on $X$, denoted $\sim_r \subset X \times X$ by $x \sim_r y$ iff $d(x,y) < r$
		\item \tbf{closed $r$-ball relation} on $X$, denoted $\simeq_r \subset X \times X$ by $x \simeq_r y$ iff $d(x,y) \leq r$
	\end{itemize}
\end{defn}

\begin{ex} \lex{ex:metric_spaces:ultra_metric:0006}
	Let $(X,d)$ be an ultrametric space and $r > 0$. Then
	\begin{enumerate}
		\item ${\sim_r}$ is an equivalence relation on $X$
		\item ${\simeq_r}$ is an equivalence relation on $X$. 
	\end{enumerate}
\end{ex}

\begin{proof}\
	\begin{enumerate}
		\item 
		\begin{enumerate}
			\item  Let $x \in X$. Since 
			\begin{align*}
				d(x,x) 
				& = 0 \\
				& < r
			\end{align*}
			we have that $x \sim_r x$. 
			\item Let $x,y \in X$. Suppose that $x \sim_r y$. Then $d(x,y) < r$. This implies that
			\begin{align*}
				d(y,x) 
				& = d(x,y) \\
				& < r
			\end{align*}
			So $y \sim_r x$.
			\item Let $x,y,z \in X$. Suppose that $x \sim_r y$ and $y \sim_r z$. Then $d(x,y) < r$ and $d(y,z) < r$. The strong triangle inequality implies that
			\begin{align*}
				d(x,z) 
				& \leq \max(d(x,y), d(y,z)) \\
				& < r
			\end{align*}
			Hence $x \sim_r z$.
		\end{enumerate}
		\item 
		\begin{enumerate}
			\item  Let $x \in X$. Since 
			\begin{align*}
				d(x,x) 
				& = 0 \\
				& \leq r
			\end{align*}
			we have that $x \simeq_r x$. 
			\item Let $x,y \in X$. Suppose that $x \simeq_r y$. Then $d(x,y) \leq r$. This implies that
			\begin{align*}
				d(y,x) 
				& = d(x,y) \\
				& \leq r
			\end{align*}
			So $y \simeq_r x$.
			\item Let $x,y,z \in X$. Suppose that $x \simeq_r y$ and $y \simeq_r z$. Then $d(x,y) \leq r$ and $d(y,z) \leq r$. The strong triangle inequality implies that
			\begin{align*}
				d(x,z) 
				& \leq \max(d(x,y), d(y,z)) \\
				& \leq r
			\end{align*}
			Hence $x \simeq_r z$.
		\end{enumerate}
	\end{enumerate}
\end{proof}

\begin{defn} \ld{def:metric_spaces:ultra_metric:0007}
	Let $(X, d)$ be an ultrametric space and $r > 0$. We denote
	\begin{itemize}
		\item the projection of $X$ onto $X/ {\sim_r}$ by $\pi^d_r: X \rightarrow X/ {\sim_r}$
		\item the projection of $X$ onto $X/ {\simeq_r}$ by $\bar{\pi}^d_r: X \rightarrow X/ {\simeq_r}$
	\end{itemize}
\end{defn}

\begin{ex} \lex{ex:metric_spaces:ultra_metric:0008}
	Let $(X, d)$ be an ultrametric space, $x \in X$ and $r >0$. Then 
	\begin{enumerate}
		\item $\pi^d_r(x) = B(x, r)$
		\item $\bar{\pi}^d_r(x) = \bar{B}(x, r)$.
	\end{enumerate} 
\end{ex}

\begin{proof} \
	\begin{enumerate}
		\item For each $y \in X$,
		\begin{align*}
			y \sim_r x
			& \iff d(x,y) < r \\
			& \iff y \in B(x, r)
		\end{align*}
		so that $\pi^d_r(x) = \bar{B}(x,r)$.
		\item For each $y \in X$,
		\begin{align*}
			y \simeq_r x
			& \iff d(x,y) \leq r \\
			& \iff y \in \bar{B}(x, r)
		\end{align*}
		so that $\bar{\pi}^d_r(x) = \bar{B}(x,r)$.
	\end{enumerate}
\end{proof}

\begin{ex} \lex{ex:metric_spaces:ultra_metric:0009}
	Let $(X, d)$ be an ultrametric space, $x,y \in X$ and $s \in \Rg$. If $y \in \bar{B}(x, s)$, then for each $r \in \Rg$, $r \leq s$ implies that $\bar{B}(y, r) \subset \bar{B}(x, s)$. 
\end{ex}

\begin{proof}
	Suppose that $y \in \bar{B}(x, s)$. Let $r \in \Rg$. Suppose that $r \leq s$. Let  
	\begin{align*}
		z 
		& \in \bar{B}(y, r) \\
		& \subset \bar{B}(y, s)
	\end{align*}
	Then $z \simeq_s y$. Since $y \simeq_s x$, the previous exercise implies that $z \simeq_s x$. Hence $z \in \bar{B}(x, s)$. Since $z \in \bar{B}(y, r)$ is arbitrary, $\bar{B}(y, r) \subset \bar{B}(x, s)$. 
\end{proof}

\begin{ex} \lex{ex:metric_spaces:ultra_metric:0010}
	Let $(X, d)$ be an ultrametric space, $x \in X$ and $r > 0$. Then 
	\begin{enumerate}
		\item $B(x, r)$ is closed and open,
		\item $T(x, r)$ is closed and open,
		\item $\bar{B}(x, r)$ is closed and open.
	\end{enumerate}
\end{ex}

\begin{proof}\
	\begin{enumerate}
		\item By definition, $B(x,r)$ is open. Since $\sim_r$ is an equivalence relation, we have that 
		$$B(x,r)^c = \bigcup\limits_{y \in B(x,r)^c} B(y, r)$$
		which is open. Hence $B(x,r)$ is closed.
		\item \rex{ex:metric_spaces:introduction:0010.1} implies that $T(x, r)$ is closed. Let $y \in T(x, r)$. By definition, $d(x, y) = r$. Let $z \in B(y, r)$. Then  
		\begin{align*}
			d(y, z) 
			& < r \\
			& = d(x, y)
		\end{align*}
		\rex{ex:metric_spaces:ultra_metric:0004} then implies that
		\begin{align*}
			d(x, z)
			& = \max(d(x, y), d(y,z)) \\
			& = d(x,y) \\
			& = r
		\end{align*}
		Hence $z \in T(x, r)$. Since $z \in B(y, r)$ is arbitrary, we have that for each $z \in B(y, r)$, $z \in T(x, r)$. Thus $B(y, r) \subset T(x, r)$. Since $y \in T(x, r)$ is arbitrary, we have that for each $y \in T(x, r)$, there exists $r > 0$ such that $B(y, r) \subset T(x, r)$. Thus $T(x, r)$ is open. 
		\item \rex{ex:metric_spaces:introduction:0010.1} implies that $\bar{B}(x,r)$ is closed. Since $B(x, r)$ and $T(x, r)$ are open and
		\begin{align*}
			\bar{B}(x, r)
			& = B(x,r) \cup T(x, r),
		\end{align*} 
		we have that $\bar{B}(x, r)$ is open.
	\end{enumerate}
\end{proof}








































\subsection{Partitions}

\begin{defn} \ld{def:metric_spaces:ultra_metric:0011}
	Let $X$ be a set. We define the \tbf{collection of partitions of $X$}, denoted $\Part(X)$, by 
	$$\Part(X) = \{\MP \subset \MP(X): \MP \text{ is a partition of $X$}\}$$ 
	Let $\Gam$ be a preordered set and $\MP: \Gam \rightarrow \Part(X)$. 
	\begin{itemize}
		\item For each $r \in \Gam$, we define the \tbf{$r$-th partition relation} on $X$, denoted $\sim_{\MP_r}$, by $x \sim_{\MP_r} y$ iff there exists $E \in \MP_r$ such that $x \in E$ and $y \in E$.
		\item For $r \in \Gam$, we denote the projection of $X$ onto $X/{\sim_{\MP_r}}$ by $\pi^{\MP}_r: X \rightarrow X/ {\sim_{\MP_r}}$ so that 
		$$\MP_r = \{\pi^{\MP}_r(x): x \in X\}$$ 
	\end{itemize}
	Then 
	\begin{itemize}
		\item $\MP$ is said to \tbf{separate points} if for each $x,y \in X$, $x \neq y$ implies that there exists $r \in \Gam$ such that $\pi^{\MP}_r(x) \neq \pi^{\MP}_r(y)$
		\item $\MP$ is said to \tbf{collect points} if for each $x,y \in X$, there exists $r \in \Gam$ such that $\pi^{\MP}_{r}(x) = \pi^{\MP}_{r}(y)$.
		\item $\MP$ is said to be 
		\tbf{decreasing} if for each $r,s \in \Gam$, $r \leq s$ implies that for each $x \in X$, there exists $\MF_x \subset X$ such that 
		$$\pi^{\MP}_r(x) = \bigcup\limits_{y \in \MF_x} \pi^{\MP}_s(y)$$
	\end{itemize}
\end{defn}

\begin{ex} \lex{ex:metric_spaces:ultra_metric:0012}
	Let $X$ be a set, $\Gam$ a preordered set and $\MP: \Gam \rightarrow \Part(X)$. Suppose that $\MP$ is decreasing. Let $x\in X$ and $r, s \in \Gam$. If $r \leq s$, then $\pi^{\MP}_s(x) \subset \pi^{\MP}_r(x)$.
\end{ex}

\begin{proof}
	Suppose that $r \leq s$. Since $\MP$ is decreasing, there exists $\MF_x \subset X$ such that 
	$$\pi^{\MP}_r(x) = \bigcup\limits_{y \in \MF_x} \pi^{\MP}_s(y)$$ 
	Since $x \in \pi^{\MP}_r(x)$, there exists $y_0 \in \MF_x$ such that $x \in \pi^{\MP}_s(y_0)$. Then $\pi^{\MP}_s(y_0) = \pi^{\MP}_s(x)$ and 
	\begin{align*}
		\pi^{\MP}_s(x)
		& = \pi^{\MP}_s(y_0) \\
		& \subset \bigcup\limits_{y \in \MF_x} \pi^{\MP}_s(y) \\
		& = \pi^{\MP}_r(x)
	\end{align*}
\end{proof}

\begin{ex} \lex{ex:metric_spaces:ultra_metric:0013}
	Let $X$ be a set, $\Gam$ a preordered set and $\MP: \Gam \rightarrow \Part(X)$. Suppose that $\MP$ is decreasing. Then for each $x,y \in X$ and $s \in \Gam$, if $\pi^{\MP}_s(x) = \pi^{\MP}_s(y)$, then for each $r \in \Gam$, $r \leq s$ implies that $\pi^{\MP}_r(x) = \pi^{\MP}_r(y)$.
\end{ex}

\begin{proof}
	Let $x,y \in X$ and $s \in \Gam$. Suppose that $\pi^{\MP}_s(x) = \pi^{\MP}_s(y)$. Let $r \in \Gam$. Suppose that $r \leq s$. Since $\MP$ is decreasing, there exists $\MF_x \subset X$ such that $\pi^{\MP}_r(x) = \bigcup\limits_{z \in \MF_x} \pi^{\MP}_s(z)$. Since $\MP_s$ is a partition of $X$,  there exists $x' \in \MF_x$ such that $\pi^{\MP}_s(x') = \pi^{\MP}_s(x)$. Since $\pi^{\MP}_s(x) = \pi^{\MP}_s(y)$, we have that
	\begin{align*}
		y
		& \in \bigcup\limits_{z \in \MF_x} \pi^{\MP}_s(z) \\
		& = \pi^{\MP}_r(x)
	\end{align*}
	Since $\MP_r$ is a partition of $X$, $\pi^{\MP}_r(y) = \pi^{\MP}_r(x)$.
\end{proof}

\begin{ex} \lex{ex:metric_spaces:ultra_metric:0014}
	Let $X$ be a set, $\Gam$ a preordered set and $\MP: \Gam \rightarrow \Part(X)$. Suppose that $\MP$ is decreasing. Let $x, y \in X$ and $r,s \in \Gam$. Suppose $r \leq s$. If $\pi^{\MP}_r(x) \cap \pi^{\MP}_s(y) \neq \varnothing$, then $\pi^{\MP}_s(y) \subset \pi^{\MP}_r(x)$.  
\end{ex}

\begin{proof}
	Suppose that $\pi^{\MP}_r(x) \cap \pi^{\MP}_s(y) \neq \varnothing$. Then there exists $z \in X$ such that $z \in \pi^{\MP}_r(x) \cap \pi^{\MP}_s(y)$. Therefore $\pi^{\MP}_r(z) = \pi^{\MP}_r(x)$ and $\pi^{\MP}_s(z) = \pi^{\MP}_s(y)$. Since $r \leq s$, \rex{ex:metric_spaces:ultra_metric:0013} implies that $\pi^{\MP}_r(z) = \pi^{\MP}_r(y)$. Since $\MP$ is decreasing, we have that
	\begin{align*}
		\pi^{\MP}_s(y)
		& \subset \pi^{\MP}_r(y) \\
		& = \pi^{\MP}_r(z) \\
		& = \pi^{\MP}_r(x) 
	\end{align*}
\end{proof}

\begin{ex} \lex{ex:metric_spaces:ultra_metric:0015}
	Let $X$ be a set, $\Gam$ a preordered set and $\MP: \Gam \rightarrow \Part(X)$. Suppose that $\MP$ is decreasing. Let $x,y \in X$. Suppose that there exists $r \in \Gam$ such that $\pi^{\MP}_r(x) \neq \pi^{\MP}_r(y)$. Then for each $s \geq r$, $\pi^{\MP}_s(x) \neq \pi^{\MP}_s(y)$.
\end{ex}

\begin{proof}
	Let $x,y \in X$. Let $s \geq r$. Since $\MP$ is decreasing, there exist $\MF_x, \MF_y \subset X$ such that $\pi^{\MP}_r(x) = \bigcup\limits_{z \in \MF_x} \pi^{\MP}_s(z)$ and $\pi^{\MP}_r(y) = \bigcup\limits_{w \in \MF_y} \pi^{\MP}_s(w)$. Since $\pi^{\MP}_r(x) \cap \pi^{\MP}_r(y) = \varnothing$, we have that for each $z \in \MF_x$ and $w \in \MF_y$, $\pi^{\MP}_s(z) \cap \pi^{\MP}_s(w) = \varnothing$. Since $\MP_s$ is a partition of $X$, there exist $x' \in \MF_x$ and $y' \in \MF_y$ such that $\pi^{\MP}_s(x') = \pi^{\MP}_s(x)$ and $\pi^{\MP}_s(y') = \pi^{\MP}_s(y)$. Therefore $\pi^{\MP}_s(x) \cap \pi^{\MP}_s(y) = \varnothing$ and thus $\pi^{\MP}_s(x) \neq \pi^{\MP}_s(y)$. 
\end{proof}

\begin{defn} \ld{def:metric_spaces:ultra_metric:0016}
	Let $X$ be a set, $\Gam$ a poset with the LUB property, $\MP: \Gam \rightarrow \Part(X)$ and $x,y \in X$. 
	\begin{itemize}
		\item We define $A^{\MP}(x,y) \subset \Gam$ by $A^{\MP}(x,y) \defeq \{r \in \Gam: \pi^{\MP}_r(x) = \pi^{\MP}_r(y)\}$. 
		\item If $A^{\MP}(x,y) \neq \varnothing$ and $A^{\MP}(x,y)$ is bounded above, we define $\al^{\MP}(x,y) \in \Gam$ by $\al^{\MP}(x,y) \defeq \sup A^{\MP}(x,y)$.
	\end{itemize}
	Then $\MP$ is said to be \tbf{left-continuous} if for each $x,y \in X$, if $A^{\MP}(x,y) \neq \varnothing$ and $A^{\MP}(x,y)$ is bounded above, then $\al^{\MP}(x,y) \in A^{\MP}(x,y)$
\end{defn}

\begin{defn} \ld{def:metric_spaces:ultra_metric:0017}
	Let $X$ be a set and $\MP: \N \rightarrow \Part(X)$. For $x,y \in X$, we define the \tbf{left-continuous extension of $\MP$} denoted $\bar{\MP}: \Gam \rightarrow \Part(X)$, by 
	$$\bar{\MP}_r = \MP_{\lceil r \rceil}$$
\end{defn}

\begin{ex} \lex{ex:metric_spaces:ultra_metric:0018}
	Let $X$ be a set and $\MP: \N \rightarrow \Part(X)$. Then $\bar{\MP}$ is left-continuous.
\end{ex}

\begin{proof}
	Let $x,y \in X$. Suppose that $A^{\bar{\MP}}(x,y) \neq \varnothing$ and $\al^{\bar{\MP}}(x,y) < \infty$. Set $s = \al^{\bar{\MP}}(x,y)$. 
	\begin{itemize}
		\item For the sake of contradiction, suppose that $s \neq {\lceil s \rceil}$. Then ${\lfloor s \rfloor} < s <  {\lceil s \rceil}$. Set $\ep = 2^{-1} \min(s - {\lfloor s \rfloor}, {\lceil s \rceil} - s) $. Then $\ep > 0$, $s - \ep \in ({\lfloor s \rfloor}, s)$ and $s + \ep \in (s, {\lceil s \rceil})$. Since $s - \ep \in ({\lfloor s \rfloor}, s)$, there exists $r \in A^{\bar{\MP}}(x,y)$ such that $r \in (s - \ep, s]$.
		Set $t = s + \ep$. Since $r, t \in ({\lfloor s \rfloor}, {\lceil s \rceil})$, we have that 
		$${\lceil r \rceil}, {\lceil t \rceil} = {\lceil s \rceil}$$
		Therefore, the definition of $\bar{\MP}$ implies that 
		\begin{align*}
			\pi_t^{\bar{\MP}}(x) 
			& = \pi_{\lceil t \rceil }^{\MP}(x) \\
			& = \pi_{\lceil s \rceil }^{\MP}(x)
		\end{align*}
		Similarly, $\pi_t^{\bar{\MP}}(y) = \pi_{\lceil s \rceil }^{\MP}(y)$, $\pi_r^{\bar{\MP}}(x) = \pi_{\lceil s \rceil }^{\MP}(x)$ and $\pi_r^{\bar{\MP}}(y) = \pi_{\lceil s \rceil }^{\MP}(y)$. \\
		Since $r \in A^{\bar{\MP}}(x,y)$, we have that $\pi^{\bar{\MP}}_r(x) = \pi^{\bar{\MP}}_r(y)$. By definition of $\bar{\MP}$, we have that 
		\begin{align*}
			\pi_t^{\bar{\MP}}(x) 
			& = \pi_{\lceil s \rceil }^{\MP}(x) \\
			& = \pi_r^{\bar{\MP}}(x) \\
			& = \pi_r^{\bar{\MP}}(y) \\
			& = \pi_{\lceil s \rceil }^{\MP}(y) \\
			& = \pi_t^{\bar{\MP}}(y) 
		\end{align*}
		Hence $t \in A^{\bar{\MP}}(x,y)$. This is a contradiction since 
		\begin{align*}
			\sup A^{\bar{\MP}}(x,y)
			& = s \\
			& < t \\
			& \leq \sup A^{\bar{\MP}}(x,y)
		\end{align*}
		Hence $s = {\lceil s \rceil}$ and $s \in \N$.
		\item Choose $r \in A^{\bar{\MP}}(x,y)$ such that $r \in (s-1, s]$. Then ${\lceil r \rceil } = s$, and 
		\begin{align*}
			\pi_s^{\bar{\MP}}(x) 
			& = \pi_{\lceil s \rceil }^{\MP}(x) \\
			& = \pi_{s }^{\MP}(x) \\
			& = \pi_{\lceil r \rceil }^{\MP}(x) \\
			& = \pi_r^{\bar{\MP}}(x)
		\end{align*}
		Similarly, $\pi_s^{\bar{\MP}}(y) = \pi_r^{\bar{\MP}}(y)$. Since $r \in A^{\bar{\MP}}(x,y)$, $\pi_r^{\bar{\MP}}(x) = \pi_r^{\bar{\MP}}(y)$. Hence
		\begin{align*}
			\pi_s^{\bar{\MP}}(x) 
			& = \pi_r^{\bar{\MP}}(x) \\
			& = \pi_r^{\bar{\MP}}(y) \\
			& = \pi_s^{\bar{\MP}}(y) 
		\end{align*}
		Hence 
		\begin{align*}
			\al^{\bar{\MP}}(x,y)
			& = s \\
			& \in A^{\bar{\MP}}(x,y)
		\end{align*}
	\end{itemize} 
	Since $x,y \in X$ with $A^{\bar{\MP}}(x,y) \neq \varnothing$ and $\al^{\bar{\MP}}(x,y) < \infty$ are arbitrary, $\bar{\MP}$ is left-continuous.
\end{proof}

\begin{defn} \ld{def:metric_spaces:ultra_metric:0019}
	Let $X$ be a set, $\Gam$ a poset with the LUB property and $\MP: \Gam \rightarrow \Part(X)$. Then $\MP$ is said to be \tbf{ultrametric-equivalent} if 
	\begin{enumerate}
		\item $\MP$ separates points
		\item $\MP$ collects points
		\item $\MP$ is decreasing 
		\item $\MP$ is left-continuous
	\end{enumerate}
\end{defn}

\begin{ex} \lex{ex:metric_spaces:ultra_metric:0020}
	Let $X$ be a set and $\MP: \N \rightarrow \Part(X)$. Suppose that $\MP$ separates points, collects points and is decreasing. Then $\bar{\MP}$ is ultrametric-equivalent.
\end{ex}

\begin{proof}\
	\begin{enumerate}
		\item Let $x,y \in X$. Suppose that $x \neq y$. Since $\MP$ separates points, there exists $n \in \N$ such that 
		\begin{align*}
			\pi_n^{\bar{\MP}}(x)
			& = \pi_n^{\MP}(x) \\
			& \neq \pi_n^{\MP}(y) \\
			& = \pi_n^{\bar{\MP}}(y)
		\end{align*} 
		Since $x,y \in X$ with $x \neq y$ are arbitrary, $\bar{\MP}$ separates points.
		\item Let $x, y \in X$. Since $\MP$ collects points, there exists $n \in \N$ such that 
		\begin{align*}
			\pi_n^{\bar{\MP}}(x)
			& = \pi_n^{\MP}(x) \\
			& = \pi_n^{\MP}(y) \\
			& = \pi_n^{\bar{\MP}}(y)
		\end{align*} 
		Since $x,y \in X$ are arbitrary, $\bar{\MP}$ collects points.
		\item Let $r,s \in \Rgp$. Suppose that $r \leq s$. Let $x \in X$. Since $r \leq s$, we have that ${\lceil r \rceil} \leq {\lceil s \rceil}$. Since $\MP$ is decreasing, there exists $\MF_x \subset X$ such that 
		\begin{align*}
			\pi^{\bar{\MP}}_{r}(x)
			& = \pi^{\MP}_{\lceil r \rceil}(x) \\
			& = \bigcup\limits_{y \in \MF_x} \pi^{\MP}_{\lceil s \rceil}(y) \\
			& = \bigcup\limits_{y \in \MF_x} \pi^{\bar{\MP}}_{s}(y) \\
		\end{align*}
		Since $r,s \in \Rgp$ with $r \leq s$ and $x \in X$ are arbitrary, we have that $\bar{\MP}$ is decreasing.
		\item \rex{ex:metric_spaces:ultra_metric:0018} implies that $\bar{\MP}$ is left continuous.
	\end{enumerate}
	Since $\bar{\MP}$ separates points, collects points, is decreasing and is left continuous, $\bar{\MP}$ is ultrametric-equivalent.
\end{proof}

\begin{ex} \lex{ex:metric_spaces:ultra_metric:0021}
	Let $X$ be a set, $\Gam$ a poset with the LUB property and $\MP: \Gam \rightarrow \Part(X)$. Suppose that $\MP$ is ultrametric-equivalent. Then for each $x,y \in X$, if $x \neq y$, then $\al^{\MP}(x,y)$ exists and $A^{\MP}(x,y) = (-\infty, \al^{\MP}(x,y)]$
\end{ex}

\begin{proof}
	Let $x,y \in X$. Suppose that $x \neq y$. Since $\MP$ collects points, there exists $s > 0$ such that $\pi^{\MP}_s(x) = \pi^{\MP}_s(y)$. Hence $A^{\MP}(x,y) \neq \varnothing$. Since $\MP$ separates points, there exists $r \in \Gam$ such that $r \not \in A^{\MP}(x,y)$. If there exists $s \in A^{\MP}(x,y)$ such that $r \leq s$, then \rex{ex:metric_spaces:ultra_metric:0013} implies that $r \in A^{\MP}(x,y)$ which is a contradiction. Thus for each $s \in A^{\MP}(x,y)$, $s < r$. Hence $r$ is an upper bound of $A^{\MP}(x,y)$ and $A^{\MP}(x,y)$ is bounded above. Since $A^{\MP}(x,y)$ is nonempty and bounded above and $\Gam$ has the LUB property, $\al^{\MP}(x,y)$ exists. By definition of the supremum, $A^{\MP}(x,y) \subset (-\infty, \al^{\MP}(x,y)]$. Since $\MP$ is left-continuous, $\al^{\MP}(x,y) \in A^{\MP}(x,y)$. \rex{ex:metric_spaces:ultra_metric:0013} then implies that $(-\infty, \al^{\MP}(x,y)] \subset A^{\MP}(x,y)$. Hence $A^{\MP}(x,y) = (-\infty, \al^{\MP}(x,y)]$.
\end{proof}

\begin{ex} \lex{ex:metric_spaces:ultra_metric:0022} \tbf{Fundamental Example:} \\ 
	Let $X$ be a set, $\Gam$ a poset with the LUB property and $\MP: \Gam \rightarrow \Part(X)$ and $(\ep_{\gam})_{\gam \in \Gam} \subset \Rg$. Suppose that $(\ep_{\gam})_{\gam \in \Gam}$ is strictly decreasing and $\MP$ is ultrametric-equivalent. Define $d^{\MP}: X \times X \rightarrow \Rgp$ by 
	\[
	d^{\MP}(x,y) =
	\begin{cases}
		\exp(-\ep_{\al^{\MP}(x,y)}) & x \neq y \\
		0 &  x = y
	\end{cases}
	\]
	Then $d^{\MP}$ is an ultrametric on $X$.
\end{ex}

\begin{proof}
	Let $x, y \in X$. 
	\begin{enumerate}
		\item Suppose that $x \neq y$. Since ${\sim_{\MP_r}}$ is symmetric,  
		\begin{align*}
			\al^{\MP}(x,y) 
			& = \sup A^{\MP}(x,y) \\ 
			& = \sup A^{\MP}(y,x) \\
			& = \al^{\MP}(y,x).
		\end{align*}
		Hence $d^{\MP}(x,y) = d^{\MP}(y,x)$. If $x = y$, then
		\begin{align*}
			d^{\MP}(x,y) 
			& = 0 \\
			& = d^{\MP}(y,x).
		\end{align*} 
		\item By definition, $d^{\MP}(x,y) = 0$ iff $x = y$.  
		\item If $x = z$, $x = y$ or $y=z$, then $d^{\MP}(x,z) \leq \max(d^{\MP}(x,y), d^{\MP}(y,z))$. Suppose that $x \neq z$, $x \neq y$ and $y \neq z$. Then $d^{\MP}(x,z) \neq 0$, $d^{\MP}(x,y) \neq 0$ and $d^{\MP}(y,z) \neq 0$. \\
		Suppose that $d^{\MP}(x,y) \leq d^{\MP}(y,z)$. Then $\al^{\MP}(y,z) \leq \al^{\MP}(x,y)$. \rex{ex:metric_spaces:ultra_metric:0021} implies that $\al^{\MP}(y,z) \in A^{\MP}(x,y) \cap A^{\MP}(y,z)$. Hence
		\begin{align*}
			\pi^{\MP}_{\al^{\MP}(y,z)}(x) 
			& = \pi^{\MP}_{\al^{\MP}(y,z)}(y) \\
			& = \pi^{\MP}_{\al^{\MP}(y,z)}(z)
		\end{align*}
		Hence $\al^{\MP}(y,z) \in A^{\MP}(x, z)$. Therefore $\al^{\MP}(y,z) \leq \al^{\MP}(x,z)$ which implies that 
		\begin{align*}
			d^{\MP}(x,z) 
			& \leq d^{\MP}(y,z) \\
			& = \max(d^{\MP}(x,y), d^{\MP}(y,z))
		\end{align*}
		Simlilarly, if $d^{\MP}(y,z) \leq d^{\MP}(x,y)$, then 
		\begin{align*}
			d^{\MP}(x,z) 
			& \leq d^{\MP}(x,y) \\
			& = \max(d^{\MP}(x,y), d^{\MP}(y,z))
		\end{align*}
	\end{enumerate}
	Hence $d^{\MP}$ satisfies the strong triangle inequality. Therefore $d^{\MP}$ is an ultrametric on $X$.  
\end{proof}

\begin{defn} \ld{def:metric_spaces:ultra_metric:0023}
	Let $(X, d)$ be an ultrametric space and $\Gam$ a poset with the LUB property. Write $d(X \times X) \setminus \{0\} = (\ep_{\gam})_{\gam \in \Gam}$. Suppose that $(\ep_{\gam})_{\gam \in \Gam}$ is strictly decreasing and $\ep_{\gam} \rightarrow 0$. We define $\MP^d: \Gam \rightarrow \Part(X)$ by $$\MP^d_{\gam} = \{\bar{\pi}^d_{\ep_{\gam}}(x): x \in X\}$$
\end{defn}

\begin{ex} \lex{ex:metric_spaces:ultra_metric:0024}
	Let $(X, d)$ be an ultrametric space and $\Gam$ a poset with the LUB property. Write $d(X \times X) \setminus \{0\} = (\ep_{\gam})_{\gam \in \Gam}$. Suppose that $\# X \geq 2$, $(\ep_{\gam})_{{\gam} \in \Gam}$ is strictly decreasing and $\ep_{\gam} \rightarrow 0$. Then $\MP^d$ is ultrametric-equivalent.
\end{ex}

\begin{proof}\
	\begin{enumerate}
		\item Let $x, y \in X$. Suppose that $x \neq y$. Then $d(x,y) > 0$. Thus there exists $s \in \Gam$ such that $d(x,y) = \ep_s$. Since $(\ep_{\gam})_{\gam \in \Gam}$ is strictly decreasing and $\ep_{\gam} \rightarrow 0$, there exists $r \in \Gam$ such that $s < r$ and $\ep_r < \ep_s$. Then 
		\begin{align*}
			d(x,y) 
			& = \ep_s \\
			& > \ep_r.
		\end{align*}
		Hence $x \not \simeq_{\ep_r} y$ and therefore 
		\begin{align*}
			\pi^{\MP^d}_r(x) \cap \pi^{\MP^d}_r(y)
			& = \bar{B}(x, \ep_r) \cap \bar{B}(y, \ep_r) \\
			& = \varnothing.
		\end{align*}
		Since $x,y \in X$ with $x \neq y$ are arbitrary, we have that for each $x,y \in X$, there exists $r \in \Gam$ such that $\pi^{\MP^d}_r(x) \neq \pi^{\MP^d}_r(y)$. Hence $\MP^d$ separates points.
		\item Let $x,y \in X$. 
		\begin{itemize}
			\item Suppose that $x = y$. Since $\# X \geq 2$, we have that $\# d(X \times X) \setminus \{0\} \geq 1$. Thus $\Gam \neq \varnothing$ and there exists $r \in \Gam$. Then $\pi^{\MP^d}_r(x) = \pi^{\MP^d}_r(y)$.
			\item Suppose that $x \neq y$. Then $d(x, y) > 0$. Thus there exists $r \in \Gam$ such that $\ep_r = d(x,y)$. Then $x \simeq_{\ep_r} y$ and 
			\begin{align*}
				\pi^{\MP^d}_{r}(x) 
				& = \bar{B}(x, \ep_r) \\
				& = \bar{B}(y, \ep_r) \\
				& = \pi^{\MP^d}_{r}(y).
			\end{align*}
		\end{itemize}
		Since $x,y \in X$ are arbitrary, we have that for each $x,y \in X$, there exists $r \in \Gam$ such that $\pi^{\MP^d}_{r}(x) = \pi^{\MP^d}_{r}(y)$. Hence $\MP^d$ collects points.
		\item Let $r,s \in \Gam$. Suppose that $r \leq s$. Since $(\ep_{\gam})_{\gam \in \Gam}$ is strictly decreasing, $\ep_s \leq \ep_r$. Let $x \in X$. Choose $\MF_x = \pi^{\MP^d}_r(x)$. Let $y \in \bar{B}(x, \ep_r)$. By definition of $\MP^d$ and \rex{ex:metric_spaces:ultra_metric:0008}, we have that  
		\begin{align*}
			\pi^{\MP^d}_r(x) 
			& = \bar{\pi}^d_{\ep_r}(x) \\
			& = \bar{B}(x, \ep_r)
		\end{align*}
		and 
		\begin{align*}
			\pi^{\MP^d}_s(y) 
			& = \bar{\pi}^d_{\ep_s}(y) \\
			& = \bar{B}(y, \ep_s).
		\end{align*}
		Since $\ep_s \leq \ep_r$ and $y \in \bar{B}(x, \ep_r)$, \rex{ex:metric_spaces:ultra_metric:0009} implies that
		\begin{align*}
			\bar{B}(y, \ep_s)
			& \subset \bar{B}(x, \ep_r) \\
			& = \pi^{\MP^d}_r(x) 
		\end{align*} 
		Since $y \in \bar{B}(x, \ep_r)$ is arbitrary, we have that for each $y \in \bar{B}(x, \ep_r)$, $\bar{B}(y, \ep_s) \subset \bar{B}(x, \ep_r)$. Therefore
		\begin{align*}
			\pi^{\MP^d}_r(x)
			& = \bar{B}(x, \ep_r) \\
			& = \bigcup\limits_{y \in \bar{B}(x, \ep_r)} \bar{B}(y, \ep_s) \\
			& = \bigcup\limits_{y \in \MF_x} \pi^{\MP^d}_s(y) 
		\end{align*}
		Since $x \in X$ is arbitrary, we have that for each $x \in X$, there exists $\MF_x \subset X$ such that $\pi^{\MP^d}_r(x) = \bigcup\limits_{y \in \MF_x} \pi^{\MP^d}_s(y)$. Since $r, s \in \Gam$ with $r \leq s$ are arbitrary, we have that for each $r,s \in \Gam$, $r \leq s$ implies that for each $x \in X$, there exists $\MF_x \subset X$ such that $\pi^{\MP^d}_r(x) = \bigcup\limits_{y \in \MF_x} \pi^{\MP^d}_s(y)$. Therefore $\MP^d$ is decreasing. 
		\item \tcr{FIX!!!} Let $x,y \in X$. Suppose that $A^{\MP^d}(x,y) \neq \varnothing$ and $A^{\MP^d}(x,y)$ is bounded above. Let $\al \in A^{\MP}(x,y)$. 
		
		
		
		For the sake of contradiction, suppose that $\al^{\MP^d}(x,y) \not \in A^{\MP^d}(x,y)$. Let $\al \in A^{\MP}(x,y)$. Then $\al < \al^{\MP^d}(x,y)$. Since $(\ep_{\gam})_{\gam \in \Gam}$ is strictly decreasing, $\ep_{\al} > \ep_{\al^{\MP^d}(x,y)}$. 
		
		\begin{align*}
			d(x,y)
			& \leq \ep_{\al} \\
			& \tcr{\text{need $(\ep_{\gam})_{\gam \in \Gam}$ to be left-continuous}}
		\end{align*}
		
		Then there exists $(\al_n)_{n \in \N} \subset A^{\MP^d}(x,y)$ such that for each $n \in \N$, $\al_n < \al^{\MP^d}(x,y)$ and $\al^{\MP^d}(x,y) = \sup\limits_{n \in \N} \al_n$. Let $n \in \N$. Since $\al_n \in A^{\MP^d}(x,y)$, 
		\begin{align*}
			\pi^{\MP^d}_{\al_n}(x) = \pi^{\MP^d}_{\al_n}(y)
			& \implies \bar{B}(x, \al_n^{-1}) = \bar{B}(y, \al_n^{-1}) \\
			& \implies d(x,y) \leq \al_n^{-1}
		\end{align*}
		Since $n \in \N$ is arbitrary, 
		\begin{align*}
			d(x,y)
			& \leq \inf_{n \in \N} \al_n^{-1} \\
			& = (\sup_{n \in \N} \al_n)^{-1} \\
			& = \al^{\MP^d}(x,y)^{-1} 
		\end{align*}
		Hence 
		\begin{align*}
			\pi^{\MP^d}_{\al^{\MP^d}(x,y)}(x)
			& = \bar{B}(x, \al^{\MP^d}(x,y)^{-1}) \\
			& = \bar{B}(y, \al^{\MP^d}(x,y)^{-1}) \\
			& = \pi^{\MP^d}_{\al^{\MP^d}(x,y)}(y)
		\end{align*}
		Therefore $\al^{\MP^d}(x,y) \in A^{\MP^d}(x,y)$ and $\MP^d$ is left-continuous
	\end{enumerate}
	Hence $\MP^d$ is ultrametric-equivalent.
\end{proof}

\begin{ex} \lex{ex:metric_spaces:ultra_metric:0025}
	Let $(X, d)$ be an ultrametric space. Then $d^{\MP^d} \sim_{\Top} d$. \\
	\tcb{FINISH!!!}
\end{ex}

\begin{ex} \lex{ex:metric_spaces:ultra_metric:0026}
	Let $(X, d)$ be an ultrametric space. If $(X, \MT_d)$ is compact, then for each $(a_n)_{n \in \N} \subset d(X \times X)$, $(a_n)_{n \in \N}$ is injective implies that $a_n \rightarrow 0$. \\
	\tbf{Hint:} \rex{ex:metric_spaces:ultra_metric:0004}
\end{ex}

\begin{proof}
	Suppose that $(X, \MT_d)$ is compact. For the sake of contradiction, suppose that there exists $(a_n)_{n \in \N} \subset d(X \times X)$ such that $(a_n)_{n \in \N}$ is injective and $a_n \not \rightarrow 0$. Then there exists $\ep >0$ and $(a_{n_j})_{j \in \N} \subset (a_n)_{n \in \N}$ such that such that for each $n \in \N$, $a_{n_j} \geq \ep$. Define $(b_j)_{j \in \N} \subset d(X \times X)$ by $b_j \defeq a_{n_j}$. Since $(b_n)_{n \in \N} \subset d(X \times X)$, there exists $(x_n, y_n) \in X \times X$ such that for each $n \in \N$, $d(x_n, y_n) = b_n$. Since $(X, \MT_d)$ is compact, $(X \times X, \MT_d \otimes \MT_d)$ is compact and there exists a subsequence $(x_{n_j}, y_{n_j})_{j \in \N} \subset (x_n, y_n)_{n \in \N}$ and $(x, y) \in X \times X$ such that $(x_{n_j}, y_{n_j}) \rightarrow (x,y)$. Since for each $j \in \N$, $b_{n_j} \geq \ep$, we have that 
	\begin{align*}
		d(x,y)
		& = \lim_{j \rightarrow \infty} d(x_{n_j}, y_{n_j}) \\
		& = \lim_{j \rightarrow \infty} b_{n_j} \\
		& \geq \ep.
	\end{align*}
	Since $(x_{n_j}, y_{n_j}) \rightarrow (x,y)$, we have that $x_{n_j} \rightarrow x$ and $y_{n_j} \rightarrow y$. Then there exists $J_1, J_2 \in \N$ such that for each $j \in \N$, $j \geq J_1$ implies that $d(x_{n_j}, x) < \ep$ and $j \geq J_2$ implies that $d(y_{n_j}, y) < \ep$. Define $J \in \N$ by $J \defeq \max \{J_1, J_2\}$. Let $j \in \N$. Suppose that $j \geq J$. Since 
	\begin{align*}
		d(x_{n_j}, x), d(y_{n_j}, y)
		& < \ep \\
		& \leq d(x,y),
	\end{align*}  
	\rex{ex:metric_spaces:ultra_metric:0004} implies that 
	\begin{align*}
		d(x, y_{n_j}) 
		& = \max \{ d(x, y) , d(y, y_{n_j}) \} \\
		& = d(x, y) \\
		& \geq \ep \\
		& > d(x_{n_j}, x).
	\end{align*}
	Another application of \rex{ex:metric_spaces:ultra_metric:0004} implies that 
	\begin{align*}
		d(x_{n_j}, y_{n_j})
		& = \max \{ d(x_{n_j}, x), d(x, y_{n_j}) \} \\
		& = d(x, y_{n_j}).
	\end{align*}
	Therefore
	\begin{align*}
		b_{n_j}
		& = d(x_{n_j}, y_{n_j}) \\
		& = d(x, y_{n_j}) \\
		& = d(x, y).
	\end{align*}
	Since $j \in \N$ with $j \geq J$ is arbitrary, we have that for each $j \in \N$, $j \geq J$ implies that $b_{n_j} = d(x,y)$. This is a contradiction since $(b_{n_j})_{j \in \N}$ is injective. Thus for each $(a_n)_{n \in \N} \subset d(X \times X)$, $(a_n)_{n \in \N}$ is injective implies that $a_n \rightarrow 0$. 
\end{proof}

\begin{ex} \lex{ex:metric_spaces:ultra_metric:0026.1}
	Let $(X, d)$ be an ultrametric space. If $(X, \MT_d)$ is compact, then $d(X \times X) \setminus \{0\}$ is discrete. 
\end{ex}

\begin{proof}
	Set $A \defeq d(X \times X) \setminus \{0\}$. For the sake of contradiction, suppose that $A$ is not discrete. Then there exists $a \in A$ and $(a_n)_{n \in \N} \subset A \setminus \{a\}$ such that $(a_n)_{n \in \N}$ is injective and $a_n \rightarrow a$. \rex{ex:metric_spaces:ultra_metric:0026} implies that $a_n \rightarrow 0$. Thus $a = 0$, which is a contradiction. Hence $A$ is discrete. 
\end{proof}

\begin{ex} \lex{ex:metric_spaces:ultra_metric:0027} \tcb{Conjecture:} \\
	Let $(X, d)$ be an ultrametric space. If $(X, \MT_d)$ is compact, then there exists $\MP: \N \rightarrow \Part(X)$ such that $\MP$ separates points, collects points, is decreasing and $d^{\bar{\MP}} \sim_{\Top} d$.
\end{ex}

\tcr{make exercise in earlier section that if $d(X \times X)$ is finite, then $\MT_d = \MP(X)$ i.e. $d$ is basically the discrete metric.}

\begin{proof} Suppose that $(X, \MT_d)$ is compact.
	\begin{itemize}
		\item Suppose that $d(X \times X)$ is finite. Set $n \defeq \# d(X \times X)$ and write $d(X \times X) = (a_j)_{j \in [n]}$ where $a_1 \geq \ldots \geq a_n$. Define $\MP: \N \rightarrow \Part(X)$ by $\MP_j \defeq \{\bar{B}(x, a_n): x \in X\}$.
		\tcr{FINISH!!!}
		\item Suppose that $d(X \times X)$ is infinite. Since $(X, \MT_d)$ is compact, there exists $a \in d(X \times X)$ such that $a = \max d(X \times X)$. Then there exists $(a_n)_{n \in \N} \subset d(X \times X)$ such that $a_1 = a$ and $(a_n)_{n \in \N}$ is strictly decreasing. \rex{ex:metric_spaces:ultra_metric:0026} then implies that $a_n \rightarrow 0$. Define $\MP: \N \rightarrow \Part(X)$ by $\MP_n \defeq \{\bar{B}(x, a_n): x \in X \}$.
		\begin{itemize}
			\item Let $x,y \in X$. Suppose that $x \neq y$. Define $\ep > 0$ by $\ep \defeq d(x,y)$. Since $a_n \rightarrow 0$, there exists $N \in \N$ such that for each $n \in \N$, $n \geq N$ implies that $a_n < \ep$. If $y \in \bar{B}(x, a_N)$, then 
			\begin{align*}
				d(x,y)
				& \leq a_N \\
				& < \ep \\
				& = d(x,y),
			\end{align*}
			which is a contradiction. Hence $y \not \in \bar{B}(x, a_N)$ and therefore
			\begin{align*}
				\pi^{\MP}_n(x) 
				& = \bar{B}(x, a_N) \\
				& \neq \bar{B}(y, a_N) \\
				& = \pi^{\MP}_1(y). 
			\end{align*}
			Since $x,y \in X$ with $x \neq y$ are arbitrary, we have that for each $x,y \in X$, $x \neq y$ implies that there exists $N \in \N$ such that $\pi^{\MP}_n(x) \neq \pi^{\MP}_n(y)$. Hence $\MP$ separates points. 
			\item Let $x, y \in X$. Since $a_1 = a$, 
			\begin{align*}
				\pi^{\MP}_1(x) 
				& = \bar{B}(x, a) \\
				& = X \\
				& = \bar{B}(y, a) \\
				& = \pi^{\MP}_1(y). 
			\end{align*}
			Since $x,y \in X$ are arbitrary, we have that for each $x,y \in X$, $\pi^{\MP}_1(x) = \pi^{\MP}_1(y)$. Thus $\MP$ collects points. 
			\item Let $m, n \in \N$. Suppose that $m \leq n$. Let $x \in X$. Define $\MF_x \defeq \bar{B}(x, a_m)$. Since $m \leq n$, $a_n \leq a_m$. \rex{ex:metric_spaces:ultra_metric:0009} then implies that 
			\begin{align*}
				\pi^{\MP}_m(x)
				& = \bar{B}(x, a_m) \\
				& = \bigcup\limits_{y \in \MF_x} \bar{B}(x, a_n) \\
				& = \bigcup\limits_{y \in \MF_x} \pi^{\MP}_n(y).
			\end{align*}
			Since $x \in X$ is arbitrary, we have that for each $x \in X$, there exists $\MF_x \subset X$ such that $\pi^{\MP}_m(x) = \bigcup\limits_{y \in \MF_x} \pi^{\MP}_n(y)$. Since $m,n \in \N$ with $m \leq n$ are arbitrary, we have that for each $m,n \in \N$, $m \leq n$ implies that for each $x \in X$, there exists $\MF_x \subset X$ such that $\pi^{\MP}_m(x) = \bigcup\limits_{y \in \MF_x} \pi^{\MP}_n(y)$. Thus $\MP$ is decreasing.
		\end{itemize}
		\rex{ex:metric_spaces:ultra_metric:0020} then implies that $\bar{\MP}$ is ultrametric-equivalent and \rex{ex:metric_spaces:ultra_metric:0022} implies that $d^{\MP}$ is an ultrametric on $X$. 
		\tcr{need to show $d^{\bar{\MP}} \sim_{\Top} d$.}
	\end{itemize}
\end{proof}

\tcb{want to characterize when a discrete valued metric is basically a tree. }























































\newpage
\section{Zero-dimensional Metric Spaces}



\subsection{Second-Countability}

\begin{note}
	Zero dimensional metric spaces are generalizations of countable sets. The following exercise captures some of the behavior of such spaces. \tcr{rewrite or write more here}
\end{note}

\begin{ex} \lex{ex:metric_spaces_zero_dim:0001}
	Let $(X, d)$ be a metric space, $U \in \MT_d$ and $\ep > 0$. Suppose that $(X, \MT_d)$ is separable and zero-dimensional and $U$ is not compact. Then there exists $(V_n)_{n \in \N} \subset \MT_d$ such that 
	\begin{itemize}
		\item for each $n \in \N$, $V_n \neq \varnothing$, $V_n$ is closed and $\diam V_n \leq \ep$,
		\item $(V_n)_{n \in \N}$ is disjoint,
		\item $U = \bigcup\limits_{n \in \N} V_n$
	\end{itemize} 
	\tcr{space out with steps and give hints!!!}
\end{ex}

\begin{proof}
	\rex{ex:metric_spaces:introduction:0025} implies that $(X, \MT_d)$ is second-countable. Then \rex{ex:topology:countability:0013} implies that $(U, \MT_d \cap U)$ is second-countable. \rex{ex:topology:countability:0012} then implies that $(U, \MT_d \cap U)$ is Lindel\"{o}f. Since $U \in \MT_d$, we have that $\MT_d \cap U \subset \MT_d$. Since $U$ is not compact, there exists $\MV \subset \MT_d \cap U$ such that $\MV$ is an open cover of $U$ and for each $\MV' \subset \MV$, $\MV'$ is finite implies that $\MV'$ is not an open cover of $U$. Since $(X, \MT_d)$ is zero-dimensional, there exists $\MB \subset \MT_d$ such that $\MB$ is a basis for $\MT_d$ and for each $B \in \MB$, $B$ is closed. Define $\MB_{\ep} \subset \MT_d$ by $\MB_{\ep} \defeq \{B \in \MB: \diam(B) \leq \ep\}$. \rex{ex:metric_spaces:introduction:0012.1} implies that  $\MB_{\ep}$ is a basis for $\MT_d$.  Define $\MW \subset \MT_d \cap U$ by 
	$$\MW \defeq \{W \in \MT_d \cap U: \text{$W \neq \varnothing$, $W$ is $\MT_d$-closed, $\diam W \leq \ep$ and there exists $V \in \MV$ such that $W \subset V$}\}.$$
	Let $x \in U$. Since $\MV \subset \MT_d \cap U$ is a $\MT_d \cap U$-open cover of $U$, there exists $V \in \MV$ such that $x \in V$. Since $V \in \MT_d$ and $\MB_{\ep}$ is a basis for $\MT_d$, there exists $B \in \MB_{\ep}$ such that $x \in B$ and $B \subset V$. Thus $B \neq \varnothing$, $B \subset V$ and $\diam B \leq \ep$. Hence $B \in \MW$. Since $x \in U$ is arbitrary, we have that for each $x \in U$, there exists $B \in \MW$ such that $x \in B$. Hence $\MW$ is a $\MT_d \cap U$-open cover of $U$. Since $(U, \MT_d \cap U)$ is Lindel\"{o}f, there exists $(W'_n)_{n \in \N} \subset \MW$ such that $(W'_n)_{n \in \N}$ is an open cover of $U$. Define $(W_n)_{n \in \N} \subset \MT_d \cap U$ by 
	\[
	W_n \defeq 
	\begin{cases}
		W_1', & n = 1 \\
		W_n' \setminus \bigcup\limits_{j=1}^{n-1} W_j', & n \geq 2.
	\end{cases}
	\]
	Then for each $n \in \N$, $W_n$ is closed, 
	\begin{align*}
		\diam W_n 
		& \leq \diam W_n' \\
		& \leq \ep,
	\end{align*}  
	and 
	\begin{align*}
		U 
		& = \bigcup_{n \in \N} W_n' \\
		& = \bigcup_{n \in \N} W_n.
	\end{align*}  
	Define $J \subset \N$ by $J \defeq \{n \in \N: W_n \neq \varnothing\}$. For the sake of contradiction, suppose that $\# J < \infty$. Then there exists $N \in \N$ such that for each $n \in \N$, $n \geq N$ implies that $W_n = \varnothing$. Then 
	\begin{align*}
		U
		& = \bigcup_{n \in \N} W_n \\
		& = \bigcup_{n =1}^N W_n \\
		& \subset \bigcup_{n =1}^N W_n'.
	\end{align*} 
	Since for each $n \in \N$, $W_n' \subset U$, we have that $U = \bigcup_{n =1}^N W_n'$. By construction, for each $n \in \N$, there exists $V_n \in \MV$ such that $W_n' \subset V_n$. Then 
	\begin{align*}
		U
		& = \bigcup_{n =1}^N W_n' \\
		& \subset \bigcup_{n =1}^N V_n.
	\end{align*} 
	Therefore $(V_n)_{n=1}^N$ is an open cover of $U$. This is a contradiction. Hence $\# J = \infty$. We define $(j_n)_{n \in \N} \subset \N$ and $(V_n)_{n \in \N} \subset \MW$ by 
	\[
	j_n \defeq 
	\begin{cases}
		1, & n = 1 \\
		\min \{j \in [j_{n-1}]^c: W_j \neq \varnothing\}, & n \geq 2
	\end{cases}
	\]
	and $V_n \defeq W_{j_n}$. 
	Then for each $n \in \N$, $V_n \neq \varnothing$, $V_n$ is closed, $\diam V_n \leq \ep$, $(V_n)_{n \in \N}$ is disjoint, and $U = \bigcup\limits_{n \in \N} V_n$. 
\end{proof}

\begin{note}
	We recall \rd{def:metric_spaces:introduction:0027} \tcr{(finish statement)}. 
\end{note}

\begin{ex} \lex{ex:metric_spaces_zero_dim:0002}
	Let $(X, d)$ be a metric space. Suppose that $(X, \MT_d)$ is separable and zero-dimensional, $X \neq \varnothing$ and for each $x \in X$, $x$ is a $\MT_d$-condensation point of $X$. Then there exists $(C_{\gam})_{\gam \in \Gam} \subset \MT_d$ and $(x_{\gam})_{\gam \in \Gam} \in \prod\limits_{\gam \in \Gam} C_{\gam}$ such that 
	\begin{enumerate}
		\item 
		\begin{itemize}
			\item there exists $x_0 \in X$ such that $X \setminus \{x_0\} = \bigcup\limits_{|\gam| = 1} C_{\gam}$,
			\item $(C_{\gam})_{\gam \in \N}$ is disjoint
		\end{itemize}
		\item for each $\gam \in \Gam$,
		\begin{enumerate}
			\item $C_{\gam}$ is $\MT_d$-closed and $C_{\gam} \neq \varnothing$,
			\item $\diam C_{\gam} \leq |\gam|^{-1}$,
			\item for each $n \in \N$, $|\gam| = n$ implies that 
			\begin{itemize}
				\item 	$C_{\gam} \setminus \{x_{\gam}\} = \bigcup\limits_{\gam' \in (\pi^{[n+1]}_{[n]})^{-1}(\{\gam\})} C_{\gam'}$
				\item $(C_{\gam'})_{\gam' \in (\pi^{[n+1]}_{[n]})^{-1}(\{\gam\})}$ is disjoint
			\end{itemize}
			\item for each $x \in C_{\gam}$, $x$ is a $\MT_d \cap C_{\gam}$-condensation point of $C_{\gam}$. 
		\end{enumerate} 
	\tcr{possibly reformat to say the same more efficiently, maybe index by $\Gam$ including length 0 indices?}
	\end{enumerate} 
\end{ex}

\begin{proof}
	We define $(C_{\gam})_{\gam \in \Gam}$ and $(x_{\gam})_{\gam \in \Gam} \in \prod\limits_{\gam \in \Gam} C_{\gam}$ inductively as follows:
	\begin{itemize}
		\item \tbf{Base Case:} \\
		Let $\gam \in \N^{1}$. Since $X \neq \varnothing$, there exists $x_0 \in X$. Since $(X, \MT_d)$ is Hausdorff, $\{x_0\}$ is closed. Define $X_0 \in \MT_d$ by $X_0 \defeq X \setminus \{x_0\}$. Since $x_0$ is a $\MT_d$-condensation point of $X$, \rex{ex:metric_spaces:compactness:0008} implies that $X_0$ is not compact. \rex{ex:metric_spaces_zero_dim:0001} then implies that there exists $(V_n)_{n \in \N} \subset \MT_d$ such that
		\begin{itemize}
			\item for each $n \in \N$, $V_n \neq \varnothing$, $V_n$ is $\MT_d$-closed and $\diam V_n \leq 1$,
			\item $(V_n)_{n \in \N}$ is disjoint,
			\item $X_0 = \bigcup\limits_{n \in \N} V_n$
		\end{itemize} 
		We define $C_{\gam} \defeq V_{\gam_1}$. Let $x \in C_{\gam}$. Since $C_{\gam} \in \MT_d$, we have that $C_{\gam} \in \MN_{\MT_d}(x)$. By assumption, $x$ is a $\MT_d$-condensation point of $X$. \rex{31028} then implies that $x$ is a $\MT_d$-condensation point of $C_{\gam}$ and \rex{ex:topology:subspaces:0004.2} implies that $x$ is a $\MT_d \cap C_{\gam}$-condensation point of $C_{\gam}$. Since $x \in C_{\gam}$ is arbitrary, we have that for each $x \in C_{\gam}$, $x$ is a $\MT_d \cap C_{\gam}$-condensation point of $C_{\gam}$.
		\item \tbf{Induction Step:} \\
		Let $n \geq 2$ and $\gam \in \N^{n}$. Define $\tl{\gam} \in \N^{n-1}$ by $\tl{\gam} \defeq \pi^{[n]}_{[n-1]}(\gam)$. Suppose that 
		\begin{itemize}
			\item $C_{\tl{\gam}} \in \MT_d$, $C_{\tl{\gam}}$ is $\MT_d$-closed and $C_{\tl{\gam}} \neq \varnothing$,
			\item for each $x \in C_{\tl{\gam}}$, $x$ is a $\MT_d \cap C_{\tl{\gam}}$-condensation point of $C_{\tl{\gam}}$
		\end{itemize}
		Define $d_{\tl{\gam}}: C_{\tl{\gam}}^2 \rightarrow \Rg$ by $d_{\tl{\gam}} \defeq d|_{C_{\tl{\gam}}}$. Since $(X, d)$ is separable, \rex{} \tcr{cite ex here} implies that $(C_{\tl{\gam}}, d_{\tl{\gam}})$ is separable. Since $(X, \MT_d)$ is zero-dimensional, \rex{ex:topology:zero_dim:0002} implies that $(C_{\tl{\gam}}, \MT_d \cap C_{\tl{\gam}})$ is zero-dimensional. Since $C_{\tl{\gam}} \neq \varnothing$, there exists $x_{\tl{\gam}} \in C_{\tl{\gam}}$. Since $(C_{\tl{\gam}}, d_{\tl{\gam}})$ is Hausdorff, $\{x_{\tl{\gam}}\}$ is $\MT_d \cap C_{\tl{\gam}}$-closed. Define $C_{\tl{\gam}, 0} \in \MT_d \cap C_{\tl{\gam}}$ by $C_{\tl{\gam}, 0} \defeq C_{\tl{\gam}} \setminus \{x_{\tl{\gam}}\}$. By assumption, $x_{\tl{\gam}}$ is a $\MT_d \cap C_{\tl{\gam}}$-condensation point of $C_{\tl{\gam}}$. \rex{ex:metric_spaces:compactness:0008} then implies that $C_{\tl{\gam}, 0}$ is not compact. \rex{ex:metric_spaces_zero_dim:0001} implies that there exists $(V^{\tl{\gam}}_n)_{n \in \N} \subset \MT_d \cap C_{\tl{\gam}}$ such that
		\begin{itemize}
			\item for each $n \in \N$, $V^{\tl{\gam}}_n \neq \varnothing$, $V^{\tl{\gam}}_n$ is $\MT_d \cap C_{\tl{\gam}}$-closed and $\diam V^{\tl{\gam}}_n \leq 1/n$,
			\item $(V^{\tl{\gam}}_n)_{n \in \N}$ is disjoint,
			\item $C_{\tl{\gam}, 0} = \bigcup\limits_{n \in \N} V^{\tl{\gam}}_n$.
		\end{itemize} 
		We define $C_{\gam} \defeq V^{\tl{\gam}}_{\gam_n}$. Since $C_{\gam} \in \MT_d \cap C_{\tl{\gam}}$ and $C_{\tl{\gam}} \in \MT_d$, we have that $C_{\gam} \in \MT_d$. Since $C_{\gam}$ is $\MT_d \cap C_{\tl{\gam}}$-closed and $C_{\tl{\gam}}$ is $\MT_d$-closed, \rex{} \tcr{cite ex} implies that $C_{\gam}$ is $\MT_d$-closed. Let $x \in C_{\gam}$. Since $C_{\gam} \in \MT_d$, we have that $C_{\gam} \in \MN_{\MT_d}(x)$. By assumption, $x$ is a $\MT_d$-condensation point of $X$. \rex{31028} then implies that $x$ is a $\MT_d$-condensation point of $C_{\gam}$ and \rex{ex:topology:subspaces:0004.2} implies that $x$ is a $\MT_d \cap C_{\gam}$-condensation point of $C_{\gam}$. Since $x \in C_{\gam}$ is arbitrary, we have that for each $x \in C_{\gam}$, $x$ is a $\MT_d \cap C_{\gam}$-condensation point of $C_{\gam}$. 
		\end{itemize}
		Thus there exist $(C_{\gam})_{\gam \in \Gam} \subset \MT_d$ and $(x_{\gam})_{\gam \in \Gam} \in \prod\limits_{\gam \in \Gam} C_{\gam}$ such that
		\begin{enumerate}
			\item 
			\begin{itemize}
				\item there exists $x_0 \in X$ such that $X \setminus \{x_0\} = \bigcup\limits_{|\gam| = 1} C_{\gam}$,
				\item $(C_{\gam})_{\gam \in \N}$ is disjoint
			\end{itemize}
			\item for each $\gam \in \Gam$,
			\begin{enumerate}
				\item $C_{\gam}$ is $\MT_d$-closed and $C_{\gam} \neq \varnothing$,
				\item $\diam C_{\gam} \leq |\gam|^{-1}$,
				\item 
				\begin{itemize}
					\item 	$C_{\gam} \setminus \{x_{\gam}\} = \bigcup\limits_{\gam' \in (\pi^{[n+1]}_{[n]})^{-1}(\{\gam\})} C_{\gam'}$
					\item $(C_{\gam'})_{\gam' \in (\pi^{[n+1]}_{[n]})^{-1}(\{\gam\})}$ is disjoint
				\end{itemize}
				\item for each $x \in C_{\gam}$, $x$ is a $\MT_d \cap C_{\gam}$-condensation point of $C_{\gam}$. 
			\end{enumerate} 
		\end{enumerate} 
\end{proof}










































\subsection{Polish Spaces}


\begin{ex} \lex{ex:metric_spaces_zero_dim:0003}
	Let $X$ be a Polish space. Suppose that $X$ is zero-dimensional. Define $X_0 \subset X$ by 
	$$X_0 \defeq \{x \in X: \text{$x$ is a condensation point of $X$}\}.$$
	Then $X_0$ is Polish, $X_0$ is zero dimensional and for each $x \in X_0$, $x$ is a condensation point of $X_0$. \\
	\tbf{Hint:} \rex{ex:topology:countability:0012.02} and \rex{31027}
\end{ex}


\begin{proof}
	Since $X$ is Polish, $X$ is second-countable. \rex{ex:topology:countability:0012.02} and \rex{31027} imply that $X_0$ is closed and $X_0^c$ is countable. Since $X_0$ is closed and $X$ is Polish, \rex{ex:metric_spaces:polish_spaces:0002} implies that $X_0$ is Polish. Since $X_0 \subset X$ and $X$ is zero-dimensional, \rex{ex:topology:zero_dim:0002} implies that $X_0$ is zero-dimensional. Let $x \in X_0$ and $U \in \MN(x)$. By definition, $x$ is a condensation point of $X$. Thus $U \cap X$ is uncountable. For the sake of contradiction, suppose that $U \cap X_0$ is countable. We note that 
	\begin{align*}
		U \cap X
		& = [(U \cap X) \cap X_0] \cup [(U \cap X) \cap X_0^c] \\
		& = (U \cap X_0) \cup (U \cap X \cap X_0^c). 
	\end{align*}
	Since $U \cap X \cap X_0^c \subset X_0^c$ and $X_0^c$ is countable, we have that $U \cap X \cap X_0^c$ is countable. Since $U \cap X_0$ is countable and $U \cap X = (U \cap X_0) \cup (U \cap X \cap X_0^c)$, we have that $U \cap X$ is countable, which is a contradiction. Hence $U \cap X_0$ is uncountable. Since $U \in \MN(x)$ is arbitrary, we have that for each $U \in \MN(x)$, $U \cap X_0$ is uncountable. Hence $x$ is a condensation point of $X_0$. Since $x \in X_0$ is arbitrary, we have that for each $x \in X_0$, $x$ is a condensation point of $X_0$. 
\end{proof}








































\subsection{Ultrametric Spaces}



\begin{ex} \lex{ex:metric_spaces_zero_dim:0004}
	Let $(X, d)$ be an ultrametric space. Then $(X, \MT_d)$ is zero-dimensional.
\end{ex}

\begin{proof}
	Define $\MB \subset \MT_d$ by $\MB \defeq \{B(x, r): x \in X \text{ and } r>0\}$. \rex{ex:metric_spaces:introduction:0011} implies that $\MB$ is a basis for $\MT_d$. \tcr{An exercise in the section on ultrametric spaces} implies that for each $B \in \MB$, $B$ is closed. Hence $(X, \MT_d)$ is zero-dimensional. 
\end{proof}














































\subsection{Cantor Space}

\begin{defn} \ld{ex:metric_spaces_zero_dim:0005}
	We define the \tbf{Cantor space}, denoted $\MC$, by $\MC \defeq \{0,1\}^{\N}$.
\end{defn}

\begin{ex} \lex{ex:metric_spaces_zero_dim:0005.1}
	We have that $\MC$ is a zero-dimensional Polish space.
\end{ex}

\begin{proof}
	Clear by \tcr{make finite set examples in sections on polish space section and discrete topology in zero-dimensional space section}
\end{proof}

\begin{ex} \lex{ex:metric_spaces_zero_dim:0006} 
	For each $x \in [0, 1]$, there exists $(x_n)_{n \in \N} \in \MC$ such that $x = \sum\limits_{j \in \N} x_j2^{-j}$. \\
	\textbf{Hint:} Set $x_1 = 
	\begin{cases}
		0, & x < 1/2 \\
		1, & x \geq 1/2 
	\end{cases}$
	and proceed inductively. 
\end{ex}

\begin{proof}
	Let $x \in [0,1]$. Set 
	$$x_1 = 
	\begin{cases}
		0, & x < 1/2 \\
		1, & x \geq 1/2 
	\end{cases}$$
	and for $j \geq 2$, set 
	$$x_j = \begin{cases}
		0, & x - \sum\limits_{k = 1}^{j-1}x_k2^{-k} < 2^{-j} \\
		1, & x - \sum\limits_{k = 1}^{j-1}x_k2^{-k} \geq 2^{-j} 
	\end{cases}$$
	Note that for each $j \in \N$, $x - \sum\limits_{k = 1}^{j}x_k2^{-k} \in [0, 2^{-j}]$. Hence $x = \sum\limits_{j \in \N} x_j2^{-j}$
\end{proof}

\begin{ex} \lex{ex:metric_spaces_zero_dim:0007} 
	Let $(x_n)_{n \in \N}, (y_n)_{n \in \N} \subset \MC$. Suppose that $(x_n)_{n \in \N} \neq (y_n)_{n \in \N}$. Set $N = \min\{j \in \N: x_j \neq y_j\}$. Suppose that $x_N = 0$ and $y_N = 1$. Then
	$$\sum\limits_{j \in \N} x_j 2^{-j} = \sum\limits_{j \in \N} y_j 2^{-j}$$ 
	iff for each $j \in \N$, $j > N$ implies that $x_j = 1$ and $y_j = 0$. 
\end{ex}

\begin{proof}
	Suppose that 
	$$\sum\limits_{j \in \N} x_j 2^{-j} = \sum\limits_{j \in \N} y_j 2^{-j}$$
	By definition of $N$, for each $j \in \N$, $j < N$ implies that $x_j = y_j$. Hence 
	$$\sum\limits_{j = N} x_j 2^{-j} = \sum\limits_{j = N} y_j 2^{-j}$$ 
	Since $x_N = 0$ and $y_N = 1$, we have that 
	$$\sum\limits_{j = N +1} x_j 2^{-j} = 2^{-N} + \sum\limits_{j = N+1} y_j 2^{-j}$$ 
	Thus $2^{-N} = \sum\limits_{j = N+1} (x_j - y_j) 2^{-j}$. For the sake of contradiction, suppose that there exists $m > N$ and $x_m \neq 1$ or $y_m \neq 0$. Then
	\begin{align*}
		2^{-N} 
		&= \sum\limits_{j = N+1} (x_j - y_j) 2^{-j} \\
		& < \sum\limits_{j = N+1} (1 - 0) 2^{-j} \\
		& = 2^{-N}
	\end{align*}
	which is a contradiction. Hence for each $m \in \N$, $m > N$ implies that $x_m = 1$ and $y_m = 0$. \vspace{.2cm}\\
	Conversely, suppose that for each $j \in \N$, $j > N$ implies that $x_j = 1$ and $y_j = 0$. Then 
	\begin{align*}
		\sum\limits_{j \in \N} x_j 2^{-j}
		& = \sum_{j=1}^{N-1} x_j + \sum_{j \geq N+1} 2^{-j} \\
		& = \sum_{j=1}^{N-1} x_j + 2^{-N} \sum_{j \in \N} 2^{-j} \\
		& = \sum_{j=1}^{N-1} y_j + 2^{-N} \\
		& = \sum_{j=1}^N y_j2^{-j} \\
		& = \sum\limits_{j \in \N} y_j 2^{-j}
	\end{align*}
\end{proof}

\begin{defn} \ld{def:metric_spaces_zero_dim:0008} \
	\begin{itemize}
		\item We define $Z \subset \MC$ by 
		$$Z = \bigg \{(x_n)_{n \in \N} \in \MC: \# \{n \in \N: x_n = 0\} = \infty  \bigg \} \cup \{(1, 1, 1, \ldots)\}$$ 
		\item We define $\phi: Z \rightarrow [0,1]$ by 
		$$\phi(x) = \sum\limits_{n \in \N} x_n2^{-n}$$
	\end{itemize}
\end{defn}

\begin{ex} \lex{ex:metric_spaces_zero_dim:0009}
	We have that $Z^c$ is an $F_{\sig}$-set, $Z$ is a $G_{\del}$-set and $Z$ is a zero-dimensional Polish space. \\
	\tbf{Hint:} $Z^c$ is countable
\end{ex}

\begin{proof}
	Since $\MC$ is zero-dimensional and $Z \subset \MC$, \rex{ex:topology:zero_dim:0002} implies that $Z$ is zero-dimensional. Since $Z^c \subset \bigg \{(x_n)_{n \in \N} \in \MC: \# \{n \in \N: x_n = 0\} < \infty  \bigg \}$, we have that $Z^c$ is countable. \rex{ex:topology:separation:0002.1} then implies that $Z^c$ is an $F_{\sig}$-set. Thus $Z$ is a $G_{\del}$-set. \rex{ex:metric_spaces:polish_spaces:0006} then implies that $Z$ is a Polish space.
\end{proof}

\begin{ex} \lex{ex:metric_spaces_zero_dim:00010}
	We have that $\phi:Z \rightarrow [0,1]$ is a continuous bijection. \\
	\tbf{Hint:} Weierstrass M-test
\end{ex}

\begin{proof}\
	\begin{itemize}
		\item Let $x \in [0,1]$. Then \rex{ex:metric_spaces_zero_dim:0006} implies that there exists $(x_n)_{n \in \N} \in \MC$ such that $x = \sum\limits_{n \in \N} x_n2^{-n}$. If for each $n \in \N$, $x_n = 1$, then $(x_n)_{n \in \N} \in Z$. Suppose that there exists $n \in \N$ such that $x_n = 0$. If $\# \{n \in \N: x_n = 0\} = \infty$, then $(x_n)_{n \in \N} \in Z$. Suppose that $ \# \{n \in \N: x_n = 0\} < \infty$. Set $N = \max \{ n \in \N: x_n = 0\}$. Define $(y_n)_{n \in \N} \in Z$ by 
		\[
		y_n = 
		\begin{cases}
			x_n, & n \in \{1, \ldots, N-1\} \\
			1, & n = N \\
			0, & n > N
		\end{cases}
		\]
		Then \rex{ex:metric_spaces_zero_dim:0007} implies that $\phi((y_n)_{n \in \N}) = x$. Since $x \in [0,1]$ is arbitrary, $\phi$ is surjective. \vspace{.2cm}\\  Let $(x_n)_{n \in \N}, (y_n)_{n \in \N} \in Z$. Suppose that $(x_n)_{n \in \N} \neq (y_n)_{n \in \N}$. If  $\phi((x_n)_{n \in \N}) = \phi((y_n)_{n \in \N})$, then \rex{ex:metric_spaces_zero_dim:0007} implies that $(x_n)_{n \in \N} \not \in Z$ or $(y_n)_{n \in \N} \not \in Z$, which is a contradiction. Hence $\phi((x_n)_{n \in \N}) \neq \phi((y_n)_{n \in \N})$. Since $(x_n)_{n \in \N}, (y_n)_{n \in \N} \in Z$ are arbitrary, $\phi$ is injective. So $\phi$ is a bijection. \\
		\item For $n \in \N$, define $\phi_n: Z \rightarrow [0, 1]$ by $\phi_n \defeq 2^{-n}\pi_n|_{Z}$. Then $\phi = \sum\limits_{n \in \N} \phi_n$ and $\|\phi_n\|_{\infty} = 2^{-n}$. The Weierstrass M-test implies that $\phi$ is continuous.
	\end{itemize}
\end{proof}

\begin{ex} \lex{ex:metric_spaces_zero_dim:00011}
	There exists $Z \subset \MC$ and $\phi: Z \rightarrow [0,1]$ such that $Z$ is a zero-dimensional Polish space and $\phi$ is a continuous bijection.
\end{ex}

\begin{proof}
	Clear by \rex{ex:metric_spaces_zero_dim:0009} and \rex{ex:metric_spaces_zero_dim:00010}. 
\end{proof}

\begin{defn} \ld{def:metric_spaces_zero_dim:0012}
	We define $a: \N^2 \rightarrow \N$ by $a(k,n) = 2^{k-1}(2n-1)$. 
\end{defn}

\begin{ex} \lex{ex:metric_spaces_zero_dim:0013}
	We have that $a: \N^2 \rightarrow \N$ is a bijection.
\end{ex}

\begin{proof}\
	\begin{itemize}
		\item \textbf{Injectivity} \\ 
		Let $(k,n), (k', n') \in \N^2$. Suppose that $a(k,n) = a(k',n')$. Then $2^{k-1}(2n-1) = 2^{k'-1}(2n'-1)$. Set $l = 2n-1$ and $l' = 2n'-1$. Then $l = 2^{k'-k}l'$. Since $l,l' \in \N$ and $l' \equiv 1 \pmod{2}$, we have that $k'-k \geq 0$. Since $l \equiv 1 \pmod{2}$, we have that $k'-k = 0$. Therefore $l = l'$ which implies that $n = n'$. Therefore $(k,n) = (k', n')$. Since  $(k,n), (k', n') \in \N^2$ are arbitrary, $a$ is injective.
		\item \textbf{Surjectivity:} \\
		Let $j \in \N$. Define $k_0,l_0 \in \N$ by $k_0 = \max \{k \in \N: \gcd(j, 2^{k-1}) = 2^{k-1}\}$ and $l_0 = j/2^{k_0-1}$. Since $l_0 \equiv 1 \pmod{2}$ there exists $n_0 \in \N$ such that $l_0 = 2n_0-1$. Thus, $a(k_0, n_0) = j$. Since $j \in \N$ is arbitrary, $a$ is surjective.
	\end{itemize}
\end{proof}

\begin{defn} \ld{def:metric_spaces_zero_dim:0014} \
	We define $\eta_0, \eta_1: \MC \rightarrow \MC$ and $H: \MC \rightarrow \MC^{\N}$ by
	\begin{itemize}
		\item  $\eta_0(x) = (x_{2n})_{n \in \N}$
		\item $\eta_1(x) = (x_{2n-1})_{n \in \N}$ 
		\item for $k \in \N$, $[H(x)]_k = \eta_1 \circ (\eta_0)^{k-1}(x)$. 
	\end{itemize}
\end{defn}

\begin{ex} \lex{ex:metric_spaces_zero_dim:0015} \
	\begin{enumerate}
		\item For each $x \in \MC$, $[H(x)]_k = (x_{2^{k-1}(2n-1)})_{n \in \N}$
		\item $H: \MC \rightarrow \MC^{\N}$ is a bijection
		\item $H: \MC \rightarrow \MC^{\N}$ is a homeomorphism
	\end{enumerate}
\end{ex}

\begin{proof}\
	\begin{enumerate}
		\item Let $x \in \MC$ and $k \in \N$. Define $y \in \MC$ by $ y = (\eta_0)^{k-1}(x)$. Clearly $y = (x_{2^{k-1}n})_{n \in \N}$. Then 
		\begin{align*}
			[H(x)]_k
			& = \eta_1 \circ (\eta_0)^{k-1}(x) \\
			& = \eta_1(y) \\
			& = (y_{2n-1})_{n \in \N} \\
			& =  (x_{2^{k-1}(2n-1)})_{n \in \N}
		\end{align*}
		\item  
		\begin{itemize}
			\item \textbf{Injectivity:} \\
			Let $x, y \in \MC$. Suppose that $H(x) = H(y)$. Let $j \in \N$. Define $(k,n) \in \N^2$ by $(k,n) = a^{-1}(j)$. Then $j = 2^{k-1}(2n-1)$ and
			\begin{align*}
				x_j
				& = x_{2^{k-1}(2n-1)} \\
				& = ([H(x)]_k)_n \\
				& = ([H(y)]_k)_n \\
				& = y_{2^{k-1}(2n-1)} \\
				& = y_j
			\end{align*}
			Since $j \in \N$ is arbitrary, $x = y$. Since $x,y \in \MC$ are arbitrary, $H$ is injective.
			\item \textbf{Surjectivity:} \\
			Let $X \in \MC^{\N}$. Define $(k_j,n_j)_{j \in \N} \in (\N^2)^{\N}$ and $x \in \MC$ by $(k_j,n_j) = a^{-1}(j)$ and $x_j = (X_{k_j})_{n_j}$. Let $(k,n) \in \N$. Define $j \in \N$ by $j = a(k,n)$. Then
			\begin{align*}
				([H(x)]_{k})_{n} 
				& = x_{2^{k-1}(2n-1)} \\
				& = x_{a(k, n)} \\
				& = x_j \\
				& = (X_{k})_{n}
			\end{align*}
			Hence $X = H(x)$. Since $X \in \MC^{\N}$ is arbitrary, $H$ is surjective.
		\end{itemize}
		Therefore $H$ is a bijection.
		\item Let $(x_{\al})_{\al \in A} \subset \MC$ be a net and $x \in \MC$. Suppose that $x_{\al} \rightarrow x$. Then for each $n \in \N$, 
		\begin{align*}
			(x_{\al})_n
			& = \pi_n(x_{\al}) \\
			& \rightarrow \pi_n(x) \\
			& = x_n 
		\end{align*}
		Let $k \in \N$. Then for each $n \in \N$,
		\begin{align*}
			\pi_{n}(\pi_k(H(x_{\al})))
			& = (x_{\al})_{ 2^{k-1}(2n -1)} \\
			& \rightarrow x_{2^{k-1}(2n -1)} \\
			& = \pi_n(\pi_k(H(x)))
		\end{align*}
		Thus 
		$$\pi_k(H(x_{\al})) \rightarrow \pi_k(H(x))$$
		Since $k \in \N$ is arbitrary, we have that 
		$$H(x_{\al}) \rightarrow H(x)$$
		Hence $H$ is continuous. \\
		Conversely, let $(X_{\al})_{\al \in A} \subset \MC^{\N}$ be a net and $X \in \MC^{\N}$. Suppose that $X_{\al} \rightarrow X$. Then for each $k, n \in \N$, 
		\begin{align*}
			([X_{\al}]_k)_n  
			& = \pi_n(\pi_k(X_{\al})) \\
			& \rightarrow \pi_n(\pi_k(X)) \\ 
			& = (X_k)_n
		\end{align*}
		Let $j \in \N$. Define $(k,n) \in \N^2$ by $(k,n) = a^{-1}(j)$. Then 
		\begin{align*}
			\pi_j(H^{-1}(X_{\al}))
			& = [H^{-1}(X_{\al})]_j \\
			& = ([X_{\al}]_k)_n \\
			& \rightarrow (X_k)_n \\
			& = [H^{-1}(X)]_j \\
			& = \pi_j(H^{-1}(X))
		\end{align*}
		Since $j \in \N$ is arbitrary, 
		$$H^{-1}(X_{\al}) \rightarrow H^{-1}(X)$$
		Hence $H^{-1}$ is continuous. Thus $H$ is a homeomorphism.
	\end{enumerate}
\end{proof}












































	














































































\newpage
\chapter{Analysis on $(C(X), \|\cdot\|_{\infty})$}
































































\newpage
\section{LEFTOVERS}
\begin{defn}
	Let $X$ be a metric space. Then $X$ is said to be \tbf{separable} if there exists $D \subset X$ such that $D$ is countable and for each $x \in X$ and $\ep >0$, there exists $y \in D$ such that $d(x, y) < \ep$.
\end{defn}

\begin{ex}
	Let $X$ be a metric space. If $X$ is separable, then for each $A \subset X$, if $A$ is open, then 
	\begin{enumerate}
		\item there there exist $(x_n)_{n \in \N} \subset X$ and $(r_n)_{n \in \N} \subset (0, \infty)$ such that $$A = \bigcup\limits_{n \in \N} B(x_n, r_n)$$
		i.e. $A$ is a countable union of open balls
		\item there there exist $(x_n)_{n \in \N} \subset X$ and $(r_n)_{n \in \N} \subset (0, \infty)$ such that $$A = \bigcup\limits_{n \in \N} \bar{B}(x_n, r_n)$$
		i.e. $A$ is a countable union of closed balls.
	\end{enumerate}
\end{ex}

\begin{proof}
	Suppose that $X$ is separable. Then there exists $(x_n)_{n \in \N} \subset X$ such that for each $x \in X$ and $\ep >0$, there exists $N \in \N$ such that $d(x, x_N) < \ep$. Let $A \subset X$. Suppose that $A$ is open.
	\begin{enumerate}
		\item  Set 
		$$\MB = \{B(x_n, r): r \in \Q \text{ and } B(x_n, r) \subset A\}$$ 
		Note that $\MB$ is countable. Let $x \in A$. Since $A$ is open, there exists $s \in \R$ such that $B(x, s) \subset A$. Then there exists  $r \in \Q \cap (0, r)$. Choose $N \in \N$ such that $d(x, x_N) < r/2$. Let $y \in B(x_N, r/2)$, then 
		\begin{align*}
			d(x, y) 
			& \leq d(x, x_N) + d(x_N, y) \\
			& < r/2 + r/2 \\
			& = r
		\end{align*}
		Therefore 
		\begin{align*}
			x 
			& \in B(x_N, r/2) \\
			& \subset B(x, r) \\
			& \subset A
		\end{align*}
		Hence $B(x_N, r/2) \in \MB$ and $x \in \bigcup\limits_{B \in \MB}B$. Since $x \in A$ is arbitrary, $A \subset \bigcup\limits_{B \in \MB}B$.
		\item Similar, but take $r/4$ instead of $r/2$.
	\end{enumerate}
\end{proof}

\begin{defn} \ld{}
	Let $X$ be a set, $d_1, d_2: X \times X \rightarrow \Rg$ metrics on $X$. Then $d_1$ and $d_2$ are said to be \tbf{equivalent} if there exist $A, B > 0$ such that $$A d_1 \leq d_2 \leq B d_1$$		
\end{defn}	

\begin{defn} \ld{}
	Let $(X, d_X)$ and $(Y, d_Y)$ be metric spaces and $f: X \rightarrow Y$. Then $f$ is said to be \tbf{Lipchitz} if there exists $K \geq 0$ such that for each $a, b \in X$, $$d_Y(f(a), f(b)) \leq Kd_X(a,b)$$
\end{defn}	

\begin{ex} \lex{}
	Let $(X, d_X)$ and $(Y, d_Y)$ be metric spaces and $f: X \rightarrow Y$. If $f$ is Lipchitz, then $f$ is uniformly continuous.	
\end{ex}

\begin{proof}
	By definition, there exists $K \geq 0$ such that for each $a, b \in X$, $$d_Y(f(a), f(b)) \leq Kd_X(a,b)$$ Let $\ep >0$. Choose $\del = \ep / (K+1)$. Let $a, b \in X$. Suppose that $d_X(a,b) < \del$. Then 
	\begin{align*}
		d_Y(f(a), f(b)) 
		& \leq Kd_X(a,b) \\
		& < K \del \\
		&= K \frac{\ep}{K+1} \\
		&< \ep  
	\end{align*}
\end{proof}

\begin{defn} \ld{}
	Let $(X, d_X)$ and $(Y, d_Y)$ be metric spaces and $f: X \rightarrow Y$ and $x_0 \in X$. Then $f$ is said to be \tbf{locally Lipchitz at $x_0$} if there exists $U \in \MN(x_0)$ such that $f$ is Lipschitz on $U$.
\end{defn}

\begin{defn} \ld{}
	Let $(X, d_X)$ and $(Y, d_Y)$ be metric spaces and $f: X \rightarrow Y$. Then $f$ is said to be \tbf{locally Lipchitz} if for each $x_0 \in X$, $f$ is locally Lipschitz at $x_0$.
\end{defn}


\begin{defn} \ld{}
	Let $X, Y$ be metric spaces and $T : X \rightarrow Y$. Then $T$ is said to be an \tbf{isometry} if for each $x_1, x_2 \in X$, $d( Tx_1, Tx_2) = d(x_1,x_2) $.
\end{defn}

\begin{ex} \lex{}
	Let $X,Y$ be metric spaces and $T:X \rightarrow Y$ and isometry. Then $T$ is injective.
\end{ex}

\begin{proof}
	Let $x_1, x_2 \in X$. Suppose that $Tx_1=Tx_2$. Then $0= d( Tx_1, Tx_2) = d(x_1,x_2)$. So $x_1 = x_2$. Hence $T$ is injective.
\end{proof}

\begin{note}
	Let $X,Y$ be metric spaces and $T:X \rightarrow Y$ an isometry. Then $T$ is clearly continuous. If $T$ is surjective, then $T^{-1}$ is an isometry and therefore continuous. Hence $T$ is a homeomorphism.
\end{note}

\begin{defn} \ld{}
	Let $(X,d)$ be a metric space. Then $(X,d)$ is said to be a \tbf{Polish space} if $(X,d)$ is complete and separable. 
\end{defn}



\begin{ex} \lex{}
	Let $(X, d)$ be a compact metric space, $E \subset X$ closed, $U \subset X$ open. Suppose that $E \subset U$. Then there exists $\del >0$ such that for each $x \in E$, $B(x, \del) \subset U$.
\end{ex}	

\begin{proof}
	Since $X$ is compact, $E$ and $U^c$ are compact. Then there exist $x_0 \in E$ and $y_0 \in U^c$ such that $d(E, U^c) = d(x_0,y_0)$. Since $E \cap U^c = \varnothing$, $x_0 \neq y_0$ and $d(E, U^c) >0$. Put $\ep = d(E, U^c)$ and $\del = \frac{\ep}{2}$.  Let $x \in E$, $w \in B(x, \del)$ and $y \in U^c$. Then 
	\begin{align*}
		d(y, w) 
		&\geq d(y, x) - d(x, w) \\
		&> \ep - \del \\
		&= \ep - \frac{\ep}{2} \\
		&= \frac{\ep}{2} \\
		&> 0
	\end{align*}	  
	So $y \neq w$. Since and $y \in U^c$ and $w \in B(x, \del)$ are arbitrary, $B(x, \del) \subset U$.
\end{proof}

\begin{defn} \ld{}
	Let $S$ be a set, $(X, d)$ a metric space and $B(S, X) = \{f: S \rightarrow X: f \text{ is bounded} \}$. We define the \tbf{supremum metric}, denoted $d_u:B(S,X) \times B(S,X) \rightarrow \Rg$, by $$d_u(f, g) = \sup_{x \in X}d(f(x), g(x)) $$ 
\end{defn}

\begin{ex} \lex{211111111}
	Let $X$ be a set, $(Y, d_Y)$, $(Z, d_Z)$ metric spaces, $(f_n)_{n \in \N} \subset B(X, Y)$, $f \in B(X, Y)$ and $g \in C(Y, Z)$. Suppose that $g$ is uniformly continuous. If $f_n \convt{u} f$, then $g \circ f_n \convt{u} g \circ f$. 
\end{ex}

\begin{proof}
	Suppose that $f_n \convt{u} f$. Let $\ep >0$. Uniform continuity of $g$ implies that there exists $\del >0$ such that for each $y_1, y_2 \in Y$, $d_Y(y_1, y_2) < \del$ implies that $d_Z(g(y_1), g(y_2)) < \ep/2$.  Uniform convergence implies that there exists $N \in \N$ such that for each $n \in \N$, $n \geq \N$ implies that $d_u(f_n, f) < \del/2$. Let $n \in \N$. Suppose that $n \geq N$. Let $x \in X$. Then $d_Y(f_n(x), f(x)) < \del$. This implies that $d_Z(g(f_n(x)), g(f(x))) < \ep/2$. Hence $\sup\limits_{x \in X} d_Z(g \circ f_n(x), g \circ f(x)) \leq \ep/2$. Thus $d_u(g \circ f_n , g \circ f) < \ep$. So $g \circ f_n \convt{u} g \circ f$.
\end{proof}

\begin{defn} \ld{}
	Let $(X, d)$ be a metric space. Define
	\begin{enumerate}
		\item $\Aut(X) = \{\sig:X \rightarrow X: \sig \text{ is a homeomorphism}\}$
		\item $\Aut(X, d) = \{\sig:X \rightarrow X: \sig \text{ is an isometric isomorphism}\}$
	\end{enumerate}
\end{defn}

\begin{ex} \lex{}
	Let $(X, d)$ be a compact metric space, $E \subset X$ closed, $U \subset X$ open. Suppose that $E \subset U$. Let $(f_n)_{n \in \N} \in \Aut(X)$, $f \in \Aut(X)$.  Suppose that $f_n \convt{u} f$. Then there exists $N \in \N$ such that for each $n \geq N$, $f(E) \subset f_n(U)$.
\end{ex}

\begin{proof}
	Since $f$ is a homeomorphism, $E$ is closed and $U$ is open, $f(E)$ is compact and $f(U)$ is open and $f(E) \subset f(U)$. Then $d(f(E), f(U^c)) >0$. Put $\ep = d(f(E), f(U^c))$. Choose $\del = \ep/2$. Then there exists $N \in \N$ such that for each $n \in \N$, $n \geq N$ implies that $\sup\limits_{z \in X} d(f(z), f_n(z)) < \del$. Let $n \geq N$, $x \in E$ and $w \in B(f(x), \del)$. For the sake of contradiction, suppose that $w \in f_n(U^c)$. Then there exist $p \in U^c$ such that $w = f_n(p)$. Put $z = f(p) \in f(U^c)$. Then 
	\begin{align*}
		\ep 
		&\leq d(f(x), z) \\ 
		&\leq d(f(x), w) + d(w, z) \\
		& = d(f(x), w) + d(f_n(p), f(p))  \\
		& < \del + \del \\
		& = \ep
	\end{align*}
	which is a contradiction. So $w \in f_n(U)$. Hence $B(f(x), \del) \subset f_n(U)$
\end{proof}
































































	\newpage
	\chapter{Topological Vector Spaces}
	
	\section{Introduction}
	
	\subsection{Topology}
	
	\begin{defn} \ld{def:top_vec:intro:0001}
		Let $X$ be a vector space and $\MT$ a topology on $X$. Then $X$ is said to be a \tbf{topological vector space} if
		\begin{enumerate}
			\item  addition $X \times X \rightarrow X$ is continuous  \item scalar multiplication $\C \times X \rightarrow X$ is continuous
		\end{enumerate}
	\end{defn}
	
	\begin{note}
		We usually suppress the topology $\MT$.
	\end{note}

	\begin{ex} \lex{ex:top_vec:intro:0002}
		Let $X$ be a vector space and $\MT$ a topology on $X$. Then $(X, \MT)$ is a topological vector space iff for each $(x_{\al})_{\al \in A}$, $(y_{\al})_{\al \in A} \subset X$, $(\lam_{\al})_{\al \in A} \subset \K$ nets, $x, y \in X$ and $\lam \in \K$, $x_{\al} + \lam_{\al}y_{\al} \rightarrow x + \lam y$.
	\end{ex}

	\begin{proof}
		Clear by \rex{ex:nets:0019}.
%		\begin{itemize}
%			\item $(\implies):$ \\
%			Suppose that $(X, \MT)$ is a topological vector space. Then addition $X \times X \rightarrow X$ and scalar multiplication $\K \times X \rightarrow X$ are continuous. \rex{ex:nets:0019} implies that for each $(x_{\al})_{\al \in A}$, $(y_{\al})_{\al \in A} \subset X$, $(\lam_{\al})_{\al \in A} \subset \K$ nets, $x, y \in X$ and $\lam \in \K$, $x_{\al} + \lam_{\al}y_{\al} \rightarrow x + \lam y$.
%			\item $(\impliedby):$ \\
%		\end{itemize}
	\end{proof}
	
	\begin{ex} \lex{ex:top_vec:intro:0003}
		Let $X$ be a topological vector space, $y \in X$ and $\lam \in \C^{\times}$. Define $f,g: X \rightarrow X$ by $f(x) = x + y$ and $g(x) = \lam x$. Then $f$ and $g$ are homeomorphisms. 
	\end{ex}
	
	\begin{proof}
		Since $X$ is a topological vector space, $f$ and $g$ are continuous. Clearly $f$ and $g$ are bijections with $f^{-1}(x) = x - y$ and $g^{-1}(x) = \lam^{-1}x$. Again, since $X$ is a topological vector space, $f^{-1}$ and $g^{-1}$ are continuous.
	\end{proof}

	\begin{ex} \lex{ex:top_vec:intro:0004}
		Let $X$ be a topological vector space. Then $X$ is Hausdorff iff $\{0\}$ is closed.
	\end{ex}

	\begin{proof}
		An exercise in a previous section implies that $X$ is Hausdorff iff for each $x \in X$, $\{x\}$ is closed. Thus, if $X$ is Hausdorff, then $\{0\}$. Conversely, if $\{0\}$ is closed, then the previous exercise implies that for each $x \in X$, $\{x\}$ is closed. Hence $X$ is Hausdorff.
	\end{proof}
	
	\begin{ex} \lex{ex:top_vec:intro:0005}
		We have that $(\C, \MT_{\C})$ is a topological vector space.  
	\end{ex}
	
	\begin{ex}  \lex{ex:top_vec:intro:0006}
		Let $X$ be a topological vector space, $x,y \in X$ and $U \in \MN(x)$. If $U$ is open, then there exists $r >0$ such that for each $t \in \R$, $|t| \leq r$ implies that $x+ ty \in U$.
	\end{ex}

	\begin{proof}
		Suppose that $U$ is open. For the sake of contradiction, suppose that for each $r > 0$, there exists $t \in \R$ such that $t \leq r$ and $x+ ty \not \in U$. Then for each $n \in \N$, there exists $t_n \in \R$ such that $|t_n| \leq 1/n$ and $x + t_ny \in U^c$. Since $t_n \rightarrow 0$, 
		\begin{align*}
			x + t_ny 
			& \rightarrow x + 0y \\
			&= x
		\end{align*}
		Since $U^c$ is closed, $x \in U^c$. This is a contradiction. Hence there exists $r >0$ such that for each $t \in \R$, $|t| \leq r$ implies that $x+ ty \in U$.
	\end{proof}

	\begin{ex} \lex{ex:top_vec:intro:0007}
		Let $X$ be a topological vector space and $A$, $B \subset X$. If $A$ is open, then $A + B$ is open.
	\end{ex}
	
	\begin{proof} \
		Suppose that $A$ is open. Then for each $b \in B$, $A + b$ is open. Since 
		$$A + B = \bigcup_{b \in B} A + b$$
		we have that $A + B$ is open.
	\end{proof}

	\begin{ex} \lex{ex:top_vec:intro:0008}
		Let $X$ be a topological vector space and $A,B \subset X$. Suppose that $A$ is compact, $B$ is closed and $A \cap B = \varnothing$. Then there exists $U \in \MN(0)$ such that $U$ is open and $(A + U) \cap B = \varnothing$. 
	\end{ex}
	
	\begin{proof}
		Set $\Gam = \{U \in \MN(0): U \text{ is open}\}$ and order $\Gam$ by reverse inclusion, so that $\Gam$ is a directed set. For the sake of contradiction, suppose that for each $U \in \Gam$, $(A + U) \cap B \neq \varnothing$. Then for each $\gam \in \Gam$, there exist $a_{\gam} \in A$ and $u_{\gam} \in \gam$ such that $a_{\gam} + u_{\gam} \in B$. Let $V \in \MN(0)$. Since $\Int V \in \Gam$
		\begin{align*}
			u_{\Int V} 
			& \in \Int V \\
			& \subset V
		\end{align*}
		Since $V \in \MN(0)$ is arbitrary, $u_{\gam} \rightarrow 0$.  Since $A$ is compact, there exists $a \in A$ and a subnet $(a_{\gam_{\ze}})_{\ze \in Z}$ of $(a_{\gam})_{\gam \in \Gam}$ such that $a_{\gam_{\ze}} \rightarrow a$. Then $a_{\gam_{\ze}} + u_{\gam_{\ze}} \rightarrow a$. Since $(a_{\gam_{\ze}} + u_{\gam_{\ze}})_{\ze \in Z} \subset B$ and $B$ is closed, we have that $a \in B$. This is a contradiction since $A \cap B = \varnothing$. So there exists $U \in \MN(0)$ such that $U$ is open and $(A+U) \cap B = \varnothing$.
	\end{proof}

	\begin{ex} \lex{ex:top_vec:intro:0009}
		Let $X$ be a topological vector space and $U \in \MN(0)$. If $U$ is open, then there exists $V \in \MN(0)$ such that $V$ is open and $V+V \subset U$.
	\end{ex}

	\begin{proof}
		Suppose that $U$ is open. Set $\Gam = \{V \in \MN(0): V \text{ is open}\}$ and order $\Gam$ by reverse inclusion, so that $\Gam$ is a directed set. For the sake of contradiction, suppose that for each $V \in \MN(0)$, if $V$ is open, then $V + V \not \subset U$. Then for each $\gam \in \Gam$, there exists $x_{\gam}, y_{\gam} \in \gam$ such that $x_{\gam}+y_{\gam} \in U^c$. Let $W \in \MN(0)$. Set $\be = \Int V$. Then $\be \in \Gam$. Then for each $\gam \geq \be$, 
		\begin{align*}
			x_{\gam},y_{\gam} 
			& \in \gam \\
			& \subset \be \\
			& \subset W
		\end{align*}
		So that $(x_{\gam})_{\gam \in \Gam}$ and $(y_{\gam})_{\gam \in \Gam}$ are eventually in $W$. Since $W \in \MN(0)$ is arbitrary, $x_{\gam} \rightarrow 0$ and	$y_{\gam} \rightarrow 0$. Therefore $x_{\gam} + y_{\gam} \rightarrow 0$. Since for each $\gam \in \Gam$, $x_{\gam} + y_{\gam} \in U^c$ and $U^c$ is closed, $0 \in U^c$. This is a contradiction, so there exists $V \in \MN(0)$ such that $V$ is open and $V+V \subset U$.
		\end{proof}






































	\subsection{Continuous Linear Functionals}

	\begin{defn} \ld{def:top_vec:intro:0010}
		Let $X,Y$ be topological vector spaces over $\K$. We define $L(X;Y) \defeq \{T:X \rightarrow Y: \text{ $T$ is $\K$-linear and continuous}\}$.
	\end{defn}

	\begin{ex} \lex{ex:top_vec:intro:0011}
		Let $X,Y$ be topological vector spaces. Then $L(X;Y)$ is a vector space. 
	\end{ex}
	
	\begin{proof}
		\tcr{FINISH!!!}
	\end{proof}
	
	\begin{defn} \ld{def:top_vec:intro:0012}\
		Let $X$ be a topological vector space over $\K$. We define the \tbf{topological dual space of} $X$, denoted $X^*$, by $X^* \defeq L(X; \K)$.
	\end{defn}

	\begin{ex} \lex{ex:top_vec:intro:0013}
		Let $X$, $Y$ be topological vector spaces and $\phi:X \rightarrow Y$. Suppose that $\phi$ is linear. Then $\phi$ is continuous iff $\phi$ is continuous at $0$.
	\end{ex}
	
	\begin{proof}
		If $\phi$ is continuous, then $\phi$ is continuous at $0$.\\
		Conversely, suppose that $\phi$ is continous at $0$. Let $(x_{\al})_{\al \in A} \subset X$ be a net and $x \in X$. Suppose that $x_{\al} \rightarrow x$. Then $x_{\al} - x \rightarrow 0$. Hence 
		\begin{align*}
			\phi(x_{\al}) - \phi(x) 
			&= \phi(x_{\al} - x) \\
			&\rightarrow \phi(0) \\
			&= 0
		\end{align*}
		Therefore $\phi(x_{\al}) \rightarrow \phi(x)$ and $\phi$ is continuous at $x$. Since $x \in X$ is arbitrary, $\phi$ is continuous. 
	\end{proof}

\begin{ex} \lex{ex:top_vec:intro:0014}
	Let $X$ be a topological vector space and $\phi :X \rightarrow \C$ linear. Then $\phi \in X^*$ iff $|\phi|$ is continuous. 
\end{ex}

\begin{proof}
	Suppose that $\phi$ is continuous. Since  $|\cdot|:\C \rightarrow \Rg$ is continuous, $|\phi|$ is continuous. \\
	Conversely, suppose that $|\phi|$ is continuous. Let $(x_{\al})_{\al \in A} \subset X$ be a net and $x \in X$. Suppose that $x_{\al} \rightarrow x$. Then $x_{\al} - x \rightarrow 0$. Therefore 
	\begin{align*}
		|\phi(x_{\al}) - \phi(x)| 
		&= |\phi(x_{\al} - x)| \\
		& \rightarrow |\phi(0)| \\
		&= 0
	\end{align*} 
	So $\phi(x_{\al}) \rightarrow \phi(x)$ and $\phi$ is continuous.
\end{proof}

	\begin{ex} \lex{ex:top_vec:intro:0015}
		Let $X$ be a real topological vector space and $\phi \in X^*$. If $\phi$ is not constant, then $\phi$ is open. \\
		\tbf{Hint:} There exists $x_* \in X$  such that $\phi(x_*) = 1$ and for each $U \subset X$ open and $x \in U$, there exists $r >0$ such that for each $t \in \R$, $|t| \leq r$ implies that $x + tx_* \in U$. 
	\end{ex}
	
	\begin{proof}
		Suppose that $\phi$ is not constant. Then there exists $x_0 \in X$ such that $\phi(x_0) \neq 0$. Set $x_* = \phi(x_0)^{-1} x_0$. Then
		\begin{align*}
			\phi(x_*)
			&= \phi(\phi(x_0)^{-1}x_0) \\
			&= \phi(x_0)^{-1}\phi(x_0) \\
			&= 1
		\end{align*}
		Let $U \subset X$ be open and $y \in \phi(U)$. Then there exists $x \in U$ such that $\phi(x) = y$. Sine $U$ is open, a previous exercise implies that there exists $r > 0$ such that for each $t \in \R$, $\|t\| \leq r$ implies that $x + tx_* \in U$. Let $t \in (-r, r)$. Then $\phi(x+ tx_*) \in \phi(U)$. Since 
		\begin{align*}
			\phi(x+ tx_*) 
			&= \phi(x) + t\phi(x_*) \\
			&= y + t
		\end{align*}
	we have that $(y-r, y+r) \subset \phi(U)$. Since $y \in U$ is arbitrary, $\phi(U)$ is open thus $\phi$ is open. 
	\end{proof}

	\begin{defn} \ld{def:top_vec:intro:0016}
		Let $X$ be a vector space and $\phi: X \rightarrow \C$. Then $\phi$ is said to be \tbf{real-linear} if for each $x,y \in X$ and $\lam \in \R$, $\phi(x+ \lam y) = \phi(x) + \lam \phi(y)$.
	\end{defn}

	\begin{ex} \lex{ex:top_vec:intro:0017}
		Let $X$ be a topological vector space and $\phi \in X^*$. Then $\Re \phi$ is continuous and real-linear. 
	\end{ex}

	\begin{proof}
		Clear.
	\end{proof}

	\begin{ex} \lex{ex:top_vec:intro:0018}
		Let $X$ be a topological vector space and $f:X \rightarrow \R$. If $f$ is continuous and real-linear, then there exists a unique $\phi \in X^*$ such that $\Re \phi = f$. \\
		\tbf{Hint:} For each $z \in \C$, $z = \pi_{\R}(z) - i \pi_{\R}(iz)$
	\end{ex}
	
	\begin{proof}
		Suppose that $f$ is continuous and real-linear. Define $\phi:X \rightarrow \C$ by $\phi(x) = f(x) -if(ix)$. Then $\phi$ is continuous. Let $x,y \in X$ and $\lam \in C$. Write $\lam = a + bi$. Then 
		\begin{align*}
			\phi(x + \lam y)
			&= f(x + \lam y) -if(i(x + \lam y)) \\
			&= f(x + ay + i by) -if(ix + iay - by) \\
			&= f(x) + af(y) + bf(iy) - if(ix) - iaf(iy) +ibf(y) \\
			&= [f(x) - if(ix)] + a[f(y) - if(iy)] + ib[f(y) -if(iy)] \\
			&= \phi(x) + a\phi(y) +ibf(y) \\
			&= \phi(x) + \lam \phi(y) 
		\end{align*}
	So $\phi$ is linear and $\phi \in X^*$. Let $\psi \in X^*$. Suppose that $f = \Re \psi$. 
	Then for each $x \in X$,
	\begin{align*}
		\phi(x) 
		&= f(x) - if(ix) \\
		&= \Re[ \psi(x)] -i \Re[ \psi(ix)] \\
		&= \Re [\psi(x)] - \Re [i\psi(x)] \\
		&= \Re [\psi(x)] + \Im[ \psi(x)] \\
		&= \psi(x)
	\end{align*} 
	So $\psi = \phi$ and $\phi$ is unique.
	\end{proof}

	\begin{ex} \lex{ex:top_vec:intro:0019}
		Let $X$ be a vector space, $(Y_{\al}, \MT_{\al})_{\al \in A}$ a collection of topological vector spaces and $(\phi_{\al})_{\al \in A} \in \prod\limits_{\al \in A} \ML(X, Y_{\al})$. Set $\MT \defeq \sig_X(\phi_{\al}: \al \in A)$. Then $(X, \MT)$ is a topological vector space. 
	\end{ex}

	\begin{proof}
		Let $(x_{1, \gam})_{\gam \in \Gam}, (x_{2, \gam})_{\gam \in \Gam} \subset X$, $(\lam_{\gam})_{\gam \in \Gam} \subset \K$ be nets, $x_1, x_2 \in X$ and $\lam \in \K$. Suppose that $x_{1,\gam} \rightarrow x_1$ and $x_{2,\gam} \rightarrow x_2$ in $(X, \MT)$ and $\lam_{\gam} \rightarrow \lam$. Let $\al \in A$. Since $\MT = \sig_X(\phi_{\al}: \al \in A)$, \rex{ex:nets:0020} implies that $\phi_{\al}(x_{1, \gam}) \rightarrow \phi_{\al}(x_1)$ and $\phi_{\al}(x_{2, \gam}) \rightarrow \phi_{\al}(x_2)$. Since $(Y_{\al}, \MT_{\al})$ is a topological vector space, we have that 
		\begin{align*}
			\phi_{\al}(x_{1,\gam} + \lam_{\gam} x_{2,\gam})
			& = \phi_{\al}(x_{1,\gam}) + \lam_{\gam} \phi_{\al}(x_{2,\gam}) \\
			& \rightarrow \phi_{\al}(x_1) + \lam \phi_{\al}(x_1) \\
			& = \phi_{\al}(x_1 + \lam x_1).
		\end{align*} 
		Since $\al \in A$ is arbitrary, we have that for each $\al \in A$, $\phi_{\al}(x_{1,\gam} + \lam_{\gam} x_{2,\gam}) \rightarrow \phi_{\al}(x_1 + \lam x_1)$. \rex{ex:nets:0020} implies that $x_{1,\gam} + \lam_{\gam} x_{2,\gam} \rightarrow x_1 + \lam x_2$ in $(X, \MT)$. Since $(x_{1, \gam})_{\gam \in \Gam}, (x_{2, \gam})_{\gam \in \Gam} \subset X$, $(\lam_{\gam})_{\gam \in \Gam} \subset \K$ nets, $x_1, x_2 \in X$ and $\lam \in \K$ are arbitrary, we have that for each $(x_{1, \gam})_{\gam \in \Gam}, (x_{2, \gam})_{\gam \in \Gam} \subset X$, $(\lam_{\gam})_{\gam \in \Gam} \subset \K$ nets, $x_1, x_2 \in X$ and $\lam \in \K$, $x_{1,\gam} \rightarrow x_1$, $x_{2,\gam} \rightarrow x_2$ in $(X, \MT)$ and $\lam_{\gam} \rightarrow \lam$ implies that $x_{1,\gam} + \lam_{\gam} x_{2,\gam} \rightarrow x_1 + \lam x_2$ in $(X, \MT)$. \rex{ex:top_vec:intro:0002} implies that $(X, \MT)$ is a topological vector space.
	\end{proof}

	
	
	
	
	
	
	
	
	
	
	
	
	
	
	
	
	
	
	
	
	
	
	
	
	
	
	
	
	
	
	
	
	
	
	
	
	
	
	
	
	
	
	

	\subsection{Products}
	
	\begin{ex} \lex{ex:top_vec:intro:0020}
		Let $(X_{\al}, \MT_{\al})_{\al \in A}$ a collection of topological vector spaces. Then $\bigg( \prod\limits_{\al \in A} X_{\al}, \bigotimes_{\al \in A} \MT_{\al} \bigg)$ is a topological vector space.  
	\end{ex}

	\begin{proof}
		Clear by \rex{ex:top_vec:intro:0019}. 
	\end{proof}
	
	\begin{ex} \lex{ex:top_vec:intro:0021}
		We have that $(\R^{\N}, \MT_{\R}^{\otimes \N})$ is a topological vector space. 
	\end{ex}
	
	\begin{proof}
		Clear by \rex{ex:top_vec:intro:0020}.
	\end{proof}
	
	\begin{ex} \lex{ex:top_vec:intro:0022} 
		Define $(e_n)_{n \in \N} \subset \R^{\N}$ by $\pi_k(e_n) \defeq \del_{n,k}$.
		\begin{enumerate}
			\item We have that $\spn (e_n:n \in \N)$ is dense in $(\R^{\N}, \MT_{\R}^{\otimes \N})$.
			\item Let $\phi \in (\R^{\N})^*$. Then $\# \{n \in \N: \phi(e_n) \neq 0  \} < \infty$. \\
			\tbf{Hint:} \rex{ex:product_topology:0006.2}
			\item We have that $(\R^{\N})^* = \spn \{\pi_j:j \in \N\}$. \\
			\tbf{Hint:} \rex{ex:metric_spaces:completeness:0002.3.1}
		\end{enumerate}
	\end{ex}

	\begin{proof}\
		\begin{enumerate}
			\item Define $E \subset \R^{\N}$ by $E \defeq \spn \{e_j:j \in \N\}$. Let $x \in \R^{\N}$. Define $(x_n)_{n \in \N} \subset E$ by $x_n \defeq \sum\limits_{j =1}^n \pi_j(x)e_j$. Let $k \in \N$. Then for each $n \in \N$, $n \geq k$ implies that 
			\begin{align*}
				\pi_k(x_n)
				& = \sum\limits_{j =1}^n \pi_j(x)\pi_k(e_j) \\
				& = \sum\limits_{j =1}^n \pi_j(x) \del_{k, j} \\
				& = \pi_k(x).
			\end{align*}
			Therefore $\pi_k(x_n) \rightarrow \pi_k(x)$. Since $k \in \N$ is arbitrary, we have that for each $k \in \N$, $\pi_k(x_n) \rightarrow \pi_k(x)$. Thus $x_n \rightarrow x$ in $(\R^{\N}, \MT_{\R}^{\otimes \N})$. Hence $x \in \cl E$. Since $x \in \R^{\N}$ is arbitrary, we have that $\R^{\N} \subset \cl E$. Since clearly, $\cl E \subset \R^{\N}$, we have that $\R^{\N} = \cl E$. Thus $E$ is dense in $\R^{\N}$.
			\item Define $a \in \R^{\N}$ by $a_n \defeq \phi(e_n)$. Set $J \defeq \{n \in \N: a_n \neq 0\}$. For the sake of contradiction, suppose that $\# J = \infty$. Then there exists a subsequence $(a_{n_j})_{j \in \N} \subset (a_n)_{n \in \N}$ such that for each $j \in \N$, $a_{n_j} \neq 0$. Define $(x_j)_{j \in \N} \subset \R^{\N}$ by $x_j \defeq a_{n_j}^{-1}e_{n_j}$. We note that for each $j \in \N$, 
			\begin{align*}
				\phi(x_j)
				& = \phi(a_{n_j}^{-1}e_{n_j}) \\
				& = a_{n_j}^{-1} \phi(e_{n_j}) \\
				& = a_{n_j}^{-1} a_{n_j} \\
				& = 1.
			\end{align*}
			Thus $\phi(x_j) \rightarrow 1$. Since for each $k \in \N$, 
			\begin{align*}
				\pi_k(x_j)
				& = a_{n_j}^{-1} \pi_k(e_{n_j}) \\
				& = a_{n_j}^{-1} \del_{k, n_j} \\
				& \conv{j} 0.
			\end{align*}
			\rex{ex:product_topology:0006.1} implies that $x_j \rightarrow 0$ in $(\R^{\N}, \MT_{\R}^{\otimes \N})$. Since $\phi$ is continuous, $\phi(x_j) \rightarrow 0$. This is a contradiction. Hence $\# J < \infty$ 
			\item Define $F \subset (\R^{\N})^*$ by $F \defeq \spn\{\pi_j:j \in \N\}$. Let $\phi \in (\R^{\N})^*$. The previous part implies that there exists $N \in \N$ such that for each $n \in \N$, $n > N$ implies that $\phi(e_n) = 0$. Define $\lam \in (\R^{\N})^*$ by $\lam \defeq \phi - \sum_{n =1}^N \phi(e_n) \pi_n$. Then for each $l \in \N$, $l \leq N$ implies that
			\begin{align*}
				\lam(e_l)
				& = \phi(e_l) - \sum_{n =1}^N \phi(e_n) \pi_n(e_l) \\
				& = \phi(e_l) - \sum_{n =1}^N \phi(e_n) \del_{n,l} \\
				& = \phi(e_l) - \phi(e_l) \\
				& = 0.
			\end{align*}
			and $l > N$ implies that 
			\begin{align*}
				\lam(e_l)
				& = \phi(e_l) - \sum_{n =1}^N \phi(e_n) \pi_n(e_l) \\
				& = \phi(e_l) - \sum_{n =1}^N \phi(e_n) \del_{n,l} \\
				& = 0 - 0 \\
				& = 0.
			\end{align*}
			Thus $\lam|_{E} = 0$. Since $E$ is dense in $\R^{\N}$ and $\lam$ is continuous. We recall that \rex{ex:metric_spaces:product_spaces:0002} implies that $\R^{\N}$ is a metric space. \rex{ex:metric_spaces:completeness:0002.3.1} then implies that $\lam = 0$. Thus 
			\begin{align*}
				\phi 
				& = \sum_{n=1}^N \phi(e_n)\pi_n \\
				& \in F.
			\end{align*}
			Since $\phi \in (\R^{\N})^*$ is arbitrary, we have that $(\R^{\N})^* \subset F$. Clearly $F \subset (\R^{\N})^*$. Hence $(\R^{\N})^* = F$. 
		\end{enumerate}
	\end{proof}





















	
	\newpage	
	\section{Sublinear Functionals}
	
	\begin{defn} \ld{55003}
		Let $X$ be a real vector space and $p:X \rightarrow \R$. Then $p$ is said to be a \tbf{sublinear functional} if for each $x,y \in X$, $\lam \geq 0$, 
		\begin{enumerate}
			\item $p(x+y) \leq p(x) + p(y)$
			\item $p(\lam x ) = \lam p(x)$
		\end{enumerate}  
	\end{defn}
	
	\begin{ex} \lex{55004}
		Let $X$ be a vector space and $p: X \rightarrow \R$ be a sublinear functional. Then $p(0) = 0$.
	\end{ex}
	
	\begin{proof} Set $\lam = 0$. Then 
		\begin{align*}
			0
			&= \lam p(0) \\
			&= p(\lam 0) \\
			&= p(0)
		\end{align*}
	\end{proof}
	
	\begin{proof}
		Clear
	\end{proof}
	
	\begin{ex} \lex{55008}
		Let $X$ be a vector space and $p:X \rightarrow \R$ a sublinear functional. Then for each $x, y \in X$
		\begin{enumerate}
			\item $-p(-x) \leq p(x)$
			\item $- p(y-x) \leq p(x) - p(y) \leq p(x-y)$
		\end{enumerate}
	\end{ex}
	
	\begin{proof}
		Let $x, y \in X$.
		\begin{enumerate}
			\item We have
			\begin{align*}
				0
				&= p(0) \\ 
				&= p(x - x) \\
				& \leq p(x) + p(-x)
			\end{align*}
			So $-p(-x) \leq p(x)$.
			\item We have
			\begin{align*}
				p(x)
				&= p(x -y + y) \\
				& \leq p(x-y) + p(y)
			\end{align*}
			So $p(x) - p(y) \leq p(x-y)$. Switching $x$ and $y$ gives us $p(y) - p(x) \leq p(y-x)$ and multiplying both sides by $-1$ yields $-p(y-x) \leq p(x) - p(y)$ \\ 
			Putting these two together, we see that $$-p(y-x) \leq p(x) - p(y) \leq p(x-y)$$
		\end{enumerate}
	\end{proof}
	
	
	
	\begin{thm}\tbf{Hahn-Banach Theorem for Sublinear Functionals}\\
		Let $X$ be a vector space, $p:X \rightarrow \R$ a sublinear functional, $M \subset X$ a subspace and $f:M \rightarrow \R$ a linear functional. If for each $x \in M$, $ f(x)  \leq p(x)$, then there exists a linear functional $F:X \rightarrow \R$ such that for each $x \in X$, $F(x) \leq p(x)$ and $F|_{M}=f$.
	\end{thm}
	
	\begin{ex} \lex{55011}
		Let $X$ be a vector space and $p:X \rightarrow \R$ a sublinear functional. Then there exists a linear functional $F: X \rightarrow \R$ such that for each $x \in X$, $F(x) \leq p(x)$.
	\end{ex}
	
	\begin{proof}
		Take $M = \{0\}$ and $f \equiv 0$ and apply the Hahn-Banach theorem.
	\end{proof}	
	
	\begin{ex} \lex{55012} \tbf{Equivalency of linearity (General Case)}
		Let $X$ be a vector space and $p:X \rightarrow \R$ a sublinear functional. Then the following are equivalent:
		\begin{enumerate}
			\item there exists a unique $F \in X^*$ such that $F \leq p$
			\item for each $x \in X$, $-p(-x) = p(x)$
			\item $p$ is linear
		\end{enumerate}	
		\tbf{Hint:} If there exists $x \in X$ such that $-p(-x) \neq p(x)$, define $f_1,f_2 :\spn(x) \rightarrow \R$ by $f_1(tx) = t p(x)$ and $f_2(tx) = -tp(-x)$
	\end{ex}	
	
	\begin{proof} \
		\begin{itemize}
			\item $(1) \implies (2)$: \\ 
			Suppose that there exists a unique $F \in X^*$ such that $F \leq p$. For the sake of contradiction, suppose that there exists $x \in X$ such that $-p(-x) \neq p(x)$. Define $f_1,f_2: \spn(x) \rightarrow \R$ by $$f_1(tx) = t p(x)$$ and $$f_2(tx) = -tp(-x)$$ Let $y \in \spn(x)$. Then there exists $t \in \R$ such that $y = tx$. Then for each $k \in \R$,
			\begin{align*}
				f_1(ky)
				&= f_1(ktx) \\
				&= ktp(x) \\
				&= k f_1(tx) \\
				&= k f_1(y)
			\end{align*}
			Similarly, $f_2(ky) = kf_2(y)$ and so $f_1, f_2 \in \spn(x)^*$. 
			If $t \geq 0$, then 
			\begin{align*}
				f_1(y) 
				&= f_1(tx) \\
				&= tp(x) \\
				&= p(tx) \\
				&= p(y) 
			\end{align*}
			If $t <0$, then 	
			\begin{align*}
				f_1(y) 
				&= f_1(tx) \\
				&= tp(x) \\
				&= -|t|p(x) \\
				&= -p(|t|x) \\
				&= -p(-tx) \\
				& \leq p(tx) \\
				&= p(y)  
			\end{align*}
			So $f_1 \leq p$ on $\spn(x)$. Similarly, $f_2 \leq p$ on $\spn(x)$. The Hahn-Banach theorem implies that there exist $F_1, F_2 \in X^*$ such that $F_1, F_2 \leq p$ and $F_1 = f_1, F_2 = f_2$ on $\spn(x)$. By the assumption of uniqueness, $F_1 = F_2$. This is a contradiction since 
			\begin{align*}
				F_1(x) 
				&= p(x) \\
				& \neq -p(-x) \\
				& = F_2(x) 
			\end{align*}		
			So for each $x \in X$, $-p(-x) = p(x)$. 
			\item $(2) \Rightarrow (3)$: \\
			Suppose that for each $x \in X$, $-p(-x) = p(x)$. The previous exercise implies that there exists $F \in X^*$ such that $F \leq p$. Let $x \in X$. Then 
			\begin{align*}
				-F(x) 
				&= F(-x) \\
				& \leq p(-x) \\
				&= -p(x)
			\end{align*}	
			So $p(x) \leq F(x)$ and $p \leq F$. Therefore $p = F$ and $p$ is linear.  
			\item $(3) \implies (1)$: \\ 
			Suppose that $p$ is linear. Let $F \in X^*$. Suppose that $F \leq p$. Let $x \in X$. Then as in the case for $(2) \implies (3)$, we have that
			\begin{align*}
				-F(x) 
				&= F(-x) \\
				& \leq p(-x) \\
				&= -p(x)
			\end{align*}	 
			which implies that $p = F$. So $p$ is the unique linear function $F \in X^*$ such that $F \leq p$.
		\end{itemize}
	\end{proof}
	






















	\newpage
	\section{Seminorms}
	\begin{defn} \ld{def:top_vec:seminorm:0001}
		Let $X$ be a vector space and $p:X \rightarrow \R$. Then $ p$ is said to be a \tbf{seminorm} if for each $x,y \in X$, $\lam \in \R$, 
		\begin{enumerate}
			\item $p(x+y) \leq p(x) + p(y)$
			\item $p(\lam x) = |\lam| p(x)$
		\end{enumerate}  
	\end{defn}
	
	\begin{ex} \lex{ex:top_vec:seminorm:0002}
		Let $X$ be a vector space and $p: X \rightarrow \R$ be a seminorm, then $p$ is a sublinear functional.
	\end{ex}
	
	\begin{proof}
		Clear
	\end{proof}

	\begin{ex} \lex{ex:top_vec:seminorm:0003}
		Let $X$ be a vector space and $\phi \in X^*$. Then $|\phi|$ is a seminorm on $X$.
	\end{ex}
	
	\begin{proof}
		Clear.
	\end{proof}

	\begin{ex} \lex{ex:top_vec:seminorm:0004}
		Let $X,Y$ be a vector spaces, $T \in L(X; Y)$ and $p$ a seminorm on $Y$. Then $p \circ T$ is a seminorm on $X$.
	\end{ex}
	
	\begin{proof}
		Clear.
	\end{proof}
	
	\begin{ex} \lex{ex:top_vec:seminorm:0005}
		Let $X$ be a vector space and $p: X \rightarrow \R$ be a seminorm. Then $p \geq 0$. 
	\end{ex}
	
	\begin{proof}
		Let $x \in X$. Then 
		\begin{align*}
			0 
			&= p(0) \\ 
			&= p(x - x) \\
			&\leq  p(x) + p(-x) \\
			&= p(x) + p(x) \\
			&= 2p(x)
		\end{align*}
		So $p(x) \geq 0$.
	\end{proof}

	\begin{ex} \lex{ex:top_vec:seminorm:0006} \tbf{Reverse Triangle Inequality:} \\
		Let $X$ be a vector space and $p:X \rightarrow \Rg$ be a seminorm on $X$. Then for each $x ,y \in X$, $|p(x) - p(y)| \leq p(x - y)$.  
	\end{ex}
	
	\begin{proof}
		Let $x, y \in X$. Then 
		\begin{align*}
			p(x)
			&= p(x -y + y) \\
			&\leq p(x - y) + p(y) 
		\end{align*}
		So $p(x) - p(y) \leq p(x - y)$. 
		Similarly, $p(y) \leq p(y - x) + p(y)$ and so $p(x) - p(y) \leq p(x - y)$. Therefore $|p(x) - p(y)| \leq p(x - y)$.
	\end{proof}
	
	\begin{ex} \lex{ex:top_vec:seminorm:0007}
		Let $X$ be a vector space, $p: X \rightarrow \Rg$ a seminorm and $\phi \in X^*$. Then $\phi \leq p$ iff $|\phi| \leq p$. 
	\end{ex}
	
	\begin{proof}
		Suppose that $\phi \leq p$. Let $x \in X$. Then 
		\begin{align*}
			-\phi(x)
			& = \phi(-x) \\
			& \leq p(-x) \\
			& = p(x) \\
		\end{align*}
		So $-p(x) \leq \phi(x)$. Hence $-p \leq \phi \leq p$. Thus $|\phi| \leq p$. \\
		Conversely, if $|\phi| \leq p$, then clearly $\phi \leq p$.
	\end{proof}
	
	\begin{defn} \ld{def:top_vec:seminorm:0008}
		Let $X$ be a vector space and $p:X \rightarrow \Rg$ be a seminorm on $X$. We define the \tbf{kernel of $p$}, denoted $\ker p$, by $\ker p = p^{-1}(\{0\})$.
	\end{defn}
	
	\begin{ex} \lex{ex:top_vec:seminorm:0009}
		Let $X$ be a vector space and $p: X \rightarrow \Rg$ a seminorm. Then $\ker p$ is a subspace of $X$.
	\end{ex}
	
	\begin{proof}
		Let $x, y \in \ker p$ and $\lam \in \C$. Then $p(x) = p(y) = 0$. Thus 
		\begin{align*}
			p(x + \lam y) 
			&\leq p(x) + p(\lam y) \\
			&= p(x) + |\lam|p(y) \\
			&= 0
		\end{align*} 
		So $x + \lam y \in N$ and $N$ is a subspace.
	\end{proof}
	
	
	\begin{defn} \ld{def:top_vec:seminorm:0010}
		Let $X$ be a vector space and $p:X \rightarrow [0, \infty)$ a seminorm on $X$. We define the \tbf{norm induced by $p$}, denoted $\bar{p} : X / \ker p \rightarrow  \Rg$, by $$\bar{p}(\bar{x}) = p(x)$$
	\end{defn}
	
	\begin{ex} \lex{ex:top_vec:seminorm:0011}
		Let $X$ be a vector space and $p:X \rightarrow [0, \infty)$ a seminorm on $X$. Then $\bar{p} : X / \ker p \rightarrow  \Rg$ is well defined and a norm. 
	\end{ex}
	
	\begin{proof}
		Let $x, y \in X$. Suppose that $\bar{x} = \bar{y}$. Then there exists $n \in \ker p$ such that $x = y + n$. Therefore, 
		\begin{align*}
			\bar{p}(\bar{x}) 
			&= p(x) \\
			&= p(y + n) \\
			&\leq p(y) + p(n) \\
			&= p(y) \\
			&= \bar{p}(\bar{y})
		\end{align*}
		and 
		\begin{align*}
			\bar{p}(\bar{y}) 
			&= p(y) \\
			&= p(x - n) \\
			&\leq p(x) + p(n) \\
			&= p(x) \\
			&= \bar{p}(\bar{x})
		\end{align*}
		So $\bar{p}(\bar{x}) = \bar{p}(\bar{y})$ and $\bar{p}: X / \ker p \rightarrow \Rg$ is well defined. Let $x \in X$. Suppose that $\bar{x} = \bar{0}$. Then there exists $n \in \ker p$ such that $x = n$. Therefore 
		\begin{align*}
			\bar{p}(\bar{x}) 
			&= p(x) \\
			&= p(n) \\
			&= 0
		\end{align*}
		So $\bar{p}$ is a norm.
	\end{proof}

	
	\begin{defn} \ld{def:top_vec:seminorm:0012}
		Let $X$ be a vector space, $p:X \rightarrow \Rg$ a seminorm on $X$, $x \in X$ and $r >0$. We define the 
		\begin{itemize}
			\item \tbf{open semiball of $p$ at $x$ of radius $r$}, denoted $B_p(x, r)$, by $$B_p(x, r) = \{y \in X: p(x - y) < r\}$$
			\item \tbf{closed semiball of $p$ at $x$ of radius $r$}, denoted $\bar{B}_p(x, r)$, by $$\bar{B}_p(x, r) = \{y \in X: p(x - y) \leq r\}$$
		\end{itemize}
	\end{defn}

	\begin{ex} \lex{ex:top_vec:seminorm:0013}
		Let $X$ be a vector space, $p:X \rightarrow \Rg$ a seminorm on $X$, $x \in X$ and $r >0$. Then $B_p(x, r) =  x + rB_p(0, 1)$. 
	\end{ex}

	\begin{proof}
		Let $y \in B_p(x, r)$. Then  
		\begin{align*}
			p(r^{-1}(y - x)) 
			&= r^{-1}p(y - x) \\
			&< r^{-1} r \\
			&= 1
		\end{align*}
		So $r^{-1}(y - x) \in B_p(0, 1)$. By definition, there exists $u \in B_p(0,1)$ such that $r^{-1}(y - x) = u$, which implies that 
		\begin{align*}
			y 
			&=  x + ru \\
			&\in x + rB_p(0, 1)
		\end{align*} 
		Conversely, let $y \in x + rB_p(0, 1)$. By definition, there exists $u \in B_p(0,1)$ such that $y = x + ru$. Then 
		\begin{align*}
			p(y - x) 
			&= p(ru) \\
			&= rp(u) \\
			&< r
		\end{align*}
	So $y \in B_p(x, r)$ 
	\end{proof}

	\begin{ex} \lex{ex:top_vec:seminorm:0014}
		Let $X$ be a vector space and $p,q:X \rightarrow [0, \infty)$ seminorms on $X$. Then $p \leq q$ iff $B_q(0,1) \subset B_p(0,1)$.  
	\end{ex}

	\begin{proof}
		Suppose that $p \leq q$. Let $x \in B_q(0,1)$. Then 
		\begin{align*}
			p(x) 
			& \leq q(x) \\
			& < 1
		\end{align*}
		So $x \in B_p(0,1)$. \\
		Conversely, suppose that $B_q(0,1) \subset B_p(0,1)$. Let $x \in X$. If $p(x) = 0$, then $p(x) \leq q(x)$. Suppose that $p(x) > 0$. For the sake of contradiction, suppose that $p(x) > q(x)$. Then 
		\begin{align*}
			q \bigg( \frac{x}{p(x)}\bigg) 
			&= \frac{q(x)}{p(x)} \\
			& < 1
		\end{align*}
		Therefore, $x/p(x) \in B_q(0,1) \subset B_p(0,1)$. By definition,
		\begin{align*}
			\frac{p(x)}{p(x)}
			& =  p\bigg( \frac{x}{p(x)}\bigg)\\
			&< 1
		\end{align*} 
		which is a contradiction. So $p(x) \leq q(x)$. Since $x \in X$ is arbitrary, $p \leq q$.
	\end{proof}

	\begin{ex} \lex{ex:top_vec:seminorm:0015}
		Let $X$ be a topological vector space and $p:X \rightarrow \Rg$ a continuous seminorm. Then 
		\begin{enumerate}
			\item $B_p(0,1)$ is open 
			\item $\bar{B}_p(0,1)$ is closed
		\end{enumerate}
	\end{ex}

	\begin{proof}\
		\begin{enumerate}
			\item Let $(x_{\al})_{\al \in A}$ be a net in $B_p(0,1)^c$ and $x \in X$. Suppose that $x_{\al} \rightarrow x$. Then $p(x_{\al}) \rightarrow p(x)$. Since for each $\al \in A$, $p(x_{\al}) \geq 1$, $p(x) \geq 1$. Hence $x \in B_p(0,1)^c$. So $B_p(0,1)^c$ is closed which implies that $B_p(0,1)$ is open.
			\item Let $(x_{\al})_{\al \in A}$ be a net in $\bar{B}_p(0,1)$ and $x \in X$. Suppose that $x_{\al} \rightarrow x$. Then $p(x_{\al}) \rightarrow p(x)$. Since for each $\al \in A$, $p(x_{\al}) \leq 1$, $p(x) \leq 1$. Hence $x \in \bar{B}_p(0,1)$. So $\bar{B}_p(0,1)$ is closed.
		\end{enumerate}
	\end{proof}

	\begin{ex} \lex{ex:top_vec:seminorm:0016}
		Let $X$ be a topological vector space and $p:X \rightarrow \Rg$ a seminorm. Then the following are quivalent:
		\begin{enumerate}
			\item $p$ is continuous
			\item $B_p(0,1)$ is open
			\item $\bar{B}_p(0,1) \in \MN(0)$ 
			\item $p$ is continuous at $0$. 
		\end{enumerate}
	\end{ex}

	\begin{proof}\
		\begin{itemize}
			\item $(1) \implies (2)$: \\
			Clear from previous exercise. \\
			\item $(2) \implies (3)$: \\
			Clear since $B_p(0,1) \subset \bar{B}_p(0,1)$. \\
			\item $(3) \implies (4)$: \\
			Let $(x_{\al})_{\al \in A} \subset X$ be a net. Suppose that $x_{\al} \rightarrow 0$. Let $U \subset \R$. Suppose that $U \in \MN(0)$. Then there exists $\ep > 0$ such that $\bar{B}(0, \ep) \subset U$. Since the map $f_{\ep}: X \rightarrow X$ defined by $f_{\ep}(x) = \ep x$ is a homeomorphism, $\bar{B}_p(0,\ep) = \ep \bar{B}_p(0,1) \in \MN(0)$. Hence there exists $\beta \in A$ such that for each $\al \geq \beta$, $x_{\al} \in \bar{B}_p(0, \ep)$. Let $\al \in A$. Suppose that $\al \geq \beta$. By definition, $p(x_{\al}) \leq \ep$. So $p(x_{\al}) \in \bar{B}(0,\ep) \subset U$. Hence $(p(x_{\al}))_{\al \in A}$ is eventually in $U$. Since $U \in \MN(0)$ is arbitrary, $p(x_{\al}) \rightarrow 0$. So $p$ is continuous at $0$. \\
			\item $(4) \implies (1)$: \\
			Let $(x_{\al})_{\al \in A} \subset X$ be a net and $x \in X$. Suppose that $x_{\al} \rightarrow x$. Then $x_{\al} - x \rightarrow 0$. Therefore $p(x_{\al} - x) \rightarrow 0$. The reverse triangle inequality implies that $p(x_{\al}) \rightarrow p(x)$. So $p$ is continuous.
		\end{itemize}
	\end{proof}

	\begin{ex} \lex{ex:top_vec:seminorm:0017}
		Let $X$ be a topological vector space and $p:X \rightarrow \Rg$ a seminorm. Then $p$ is continuous iff there exists a continuous seminorm $q: X \rightarrow \Rg$ such that $p \leq q$. 
	\end{ex}

	\begin{proof}
		Suppose that $p$ is continuous. Set $q = p$. Then $q$ is a continuous and $p \leq q$\\
		Conversely, suppose that there exists a continuous seminorm $q:X \rightarrow \Rg$ such that $p \leq q$. Then $\bar{B}_q(0,1) \subset \bar{B}_p(0,1)$. The previous exercise tells us that 
		\begin{align*}
			q \text{ is continuous} 
			& \iff \bar{B}_q(0,1) \in \MN(0) \\
			& \implies \bar{B}_p(0,1) \in \MN(0) \\
			& \iff p \text{ is continuous} 
		\end{align*}
	\end{proof}

	\begin{thm} \lex{ex:top_vec:seminorm:0018} \tbf{Hahn-Banach Theorem for Seminorms}\\
		Let $X$ be a vector space, $p:X \rightarrow \R$ a seminorm, $M \subset X$ a subspace and $f:M \rightarrow \C$ a linear functional. If for each $x \in M$, $\vert f(x) \vert \leq p(x)$, then there exists a linear functional $F:X \rightarrow \C$ such that for each $x \in X$, $\vert F(x) \vert \leq p(x)$ and $F|_{M}=f$.
	\end{thm}	






















	
	\newpage
	\section{Minkowski Functionals}

	\begin{defn}
		Let $X$ be a vector space and $A \subset X$. Then $A$ is said to be \tbf{convex} if for each $x, y \in A$, $t \in [0,1]$, $tx +(1-t)y \in A$.
	\end{defn}

	\begin{ex}
		Let $X$ be a vector space and $\MA \subset \MP(X)$, Suppose that for each $A \in \MA$, $A$ is convex. Then $$\bigcap_{A \in \MA} A$$ is convex.
	\end{ex}

	\begin{proof}
		Let $x,y \in \bigcap\limits_{A \in \MA} A$ and $t \in [0,1]$. Then for each $A \in \MA$, $x,y \in A$. Let $A \in \MA$. Since $A$ is convex, $tx + (1-t)y \in A$. Since $A \in \MA$ is arbitrary, $tx + (1-t)y \in \bigcap\limits_{A \in \MA} A$. So $\bigcap\limits_{A \in \MA} A $ is convex.
	\end{proof}

	\begin{defn}
		Let $X$ be a vector space and $A \subset X$. Set 
		$$\MS = \{S \subset X: S \text{ is convex and } A \subset S\}$$ 
		We define the \tbf{convex hull of $A$}, denoted $\cnv A$, by $$\cnv A = \bigcap_{S \in \MS}S$$
	\end{defn}

	\begin{note}
		We may think of $\cnv A$ as the smallest convex set containing $A$. 
	\end{note}

	\begin{defn}
		Let $X$ be a vector space, $A \subset X$ and $x \in X$. Then $x$ is said to be a \tbf{convex combinations of elements of $A$} if there exist $(a_j)_{j=1}^n \subset A$ and $(t_j)_{j=1}^n \subset [0,1]$ such that $x = \sum\limits_{j=1}^n t_j a_j$ and $\sum\limits_{j=1}^n t_j = 1$. We define $C_A \subset X$ by 
		
		$$C_A = \{x \in X:\text{$x$ is a convex combination of elements of $A$}\}$$
	\end{defn}

	\begin{ex}
		Let $X$ be a vector space and $A \subset X$. 
 		Then 
 		\begin{enumerate}
 			\item $A \subset C_A$
 			\item $C_A$ is convex 
 		\end{enumerate}
	\end{ex}

	\begin{proof}\
		\begin{enumerate}
			\item Let $x \in A$, then 
			\begin{align*}
				x 
				&= 1x \\
				& \in C_A 
			\end{align*}
			So $A \subset C_A$.\\
			\item Let $x, y \in C_A$. and $\lam \in [0,1]$. Then there exist $(a_i)_{i=1}^n$, $(b_j)_{j=1}^m \subset A$ and $(s_i)_{i=1}^n$, $(t_j)_{j=1}^m \subset [0,1]$ such that $x = \sum\limits_{i=1}^n s_i a_i$ and $y = \sum\limits_{j=1}^m t_j b_j$. Then
			\begin{align*}
				\lam x + (1-\lam)y 
				&= \lam [\sum\limits_{i=1}^n s_i a_i] + (1-\lam)[\sum\limits_{j=1}^m t_j b_j] \\
				&= \sum\limits_{i=1}^n \lam s_i a_i + \sum\limits_{j=1}^m (1-\lam) t_j b_j
			\end{align*}
			Since 
			\begin{enumerate}
				\item for each $i \in \{1, \ldots, n\}$ and $j \in \{1, \ldots, m\}$, we have that $\lam s_i \in [0,1]$ and $(1-\lam)t_j \in [0,1]$
				\item 
				\begin{align*}
					\sum\limits_{i=1}^n \lam s_i + \sum\limits_{j=1}^m (1-\lam) t_j
					&= \lam \sum\limits_{i=1}^n s_i +  (1-\lam) \sum\limits_{j=1}^m  t_j \\
					&= \lam + (1-\lam) \\
					&= 1
				\end{align*}
			\end{enumerate}
			we have that $\lam x+(1-\lam) y \in C_A$. So $C_A$ is convex.
		\end{enumerate}
	\end{proof}

	\begin{ex}
		Let $X$ be a vector space and $A \subset X$. Let $(a_j)_{j=1}^n \subset A$ and $(t_j)_{j=1}^n \subset [0,1]$. Suppose that $\sum\limits_{j=1}^n t_j = 1$. If $A$ is convex, then $\sum\limits_{j=1}^n t_ja_j \in A$.\\
		\tbf{Hint:} proceed by induction on $n$
	\end{ex}
	

	\begin{proof}
		Suppose that $A$ is convex. If $n = 2$, then by definition, $\sum\limits_{j=1}^n t_ja_j \in A$. \\
		Suppose that the claim is true for $n - 1$. Since $\sum\limits_{j=1}^n t_j = 1$, then there $k \in \{1, \ldots, n\}$ such that $t_k > 0$. Choose Choose $l \in \{1, \ldots, n\}$ such that $l \neq k$. Set $S = \{1, \ldots, n\} \setminus \{t_l\}$. Then $1 - t_l >0$ and  
		\begin{align*}
			x
			&= \sum\limits_{j=1}^n t_j a_j \\
			&= t_l a_l + \sum_{j \in S} t_ja_j \\
			&= t_la_l + (1-t_l) \sum_{j \in S} \frac{t_j}{1 - t_l}a_j
		\end{align*}
		Since 
		\begin{align*}
			\sum_{j \in S} \frac{t_j}{1-t_l} 
			&= \frac{1-t_l}{1-t_l} \\
			&= 1
		\end{align*}
		our induction hypothesis implies that 
		$$\sum\limits_{j \in S} \frac{t_j}{1-t_l} a_j \in A$$ 
		Since $A$ is convex, by definition we have that 
		\begin{align*}
			x 
			&= t_la_l + (1-t_l) \bigg[ \sum_{j \in S} \frac{t_j}{1 - t_l}a_j \bigg] \\
			& \in A
		\end{align*}
	\end{proof}

	\begin{ex}
		Let $X$ be a vector space and $A \subset X$. Then 
		$$\cnv A = C_A$$
	\end{ex}

	\begin{proof}
	Since $A \subset C_A$ and $C_A$ is convex, $\cnv A \subset C_A$. \\
	Conversely, Let $x \in C_A$. Then there exist $(a_j)_{j=1}^n \subset A$ and $(t_j)_{j=1}^n \subset [0,1]$ such that $x = \sum\limits_{j=1}^n t_j a_j$ and $\sum\limits_{j=1}^n t_j = 1$. Since $A \subset \cnv A$ and $\cnv A$ is convex, the previous exercise implies that $x \in \cnv A$. So $C_A \subset \cnv A$. Hence $\cnv A = C_A$.
	\end{proof}

	\begin{ex}
		Let $X$ be a vector space and $A$, $B \subset X$ convex and $\lam \in \C$. Then 
		\begin{enumerate}
			\item $A + B$ is convex
			\item $\lam A$ is convex
		\end{enumerate}
	\end{ex}
	
	\begin{proof}\
		\begin{enumerate}
			\item Let $x,y \in A + B$ and $t \in [0,1]$. Then there exist $a_x, a_y \in A$, $b_x, b_y \in B$ such that $x = a_x + b_x$ and $y = a_y + b_y$. Since $A$ and $B$ are convex, $ta_x + (1-t)a_y \in A$ and $tb_x + (1-t)b_y \in B$. Hence 
			\begin{align*}
				tx + (1-t)y
				&= ta_x + tb_x + (1-t)a_y + (1-t)b_y \\
				&= [ta_x + (1-t)a_y] + [tb_x + (1-t)b_y] \\
				& \in A + B
			\end{align*}
			So $A + B$ is convex.
			\item Let $x, y \in \lam A$ and $t \in [0,1]$. Then there exist $a_x, a_y \in A$ such that $x = \lam a_x$ and $y = \lam a_y$. Since $A$ is convex, $t a_x + (1-t) a_y \in A$. Therefore 
			\begin{align*}
				tx + (1-t)y 
				&= t \lam a_x + (1-t) \lam a_y \\
				&= \lam [t a_x + (1-t) a_y ] \\
				& \in \lam A
			\end{align*}
			So $\lam A$ is convex.
		\end{enumerate}
	\end{proof}

		\begin{defn}
		Let $X$ be a vector space and $A \subset X$. Then $A$ is said to be \tbf{balanced} if for each $x \in A$, $c \in \C$, $|c| \leq 1$ implies that $cx \in A$.
	\end{defn}

		\begin{ex}
		Let $X$ be a vector space and $\MA \subset \MP(X)$, Suppose that for each $A \in \MA$, $A$ is balanced. Then $$\bigcup_{A \in \MA} A$$ is balanced.
	\end{ex}
	
	\begin{proof}
		Let $x \in \bigcap\limits_{A \in \MA} A$ and $r \in \C$. Suppose that $|r| \leq 1$. Then there exists $B \in \MA$ such that $x \in B$. Since $A$ is balanced, 
		\begin{align*}
			rx 
			&\in B \\
			& \subset \bigcap\limits_{A \in \MA} A
		\end{align*}
		So $\bigcap\limits_{A \in \MA} A$ is balanced.
	\end{proof}

	\begin{defn}
		Let $X$ be a vector space and $A \subset X$. We define the \tbf{balanced hull of $A$}, denoted $\bal A$, by $$\bal A = \bigcup_{\substack{r \in \C \\ |r| \leq 1}} rA$$
	\end{defn}

	\begin{ex}
		Let $X$ be a vector space and $A \subset X$. Then $\bal A$ is balanced. 
	\end{ex}

	\begin{proof}
		Let $x \in \bal A$ and $r \in \C$. Suppose that $|r| \leq 1$. By definition, there exists $s \in \C$ and $a \in A$ such that $|s| \leq 1$ and $x = sa$. Then 
		\begin{align*}
			|rs| 
			&= |r||s| \\
			&\leq 1
		\end{align*}
		which implies that
		\begin{align*}
			rx 
			&= rsa \\
			& \in rsA \\
			& \subset \bigcup_{\substack{q \in \C \\ |q| \leq 1}} qA \\
			&= \bal A
		\end{align*}
		So $\bal A$ is balanced.
	\end{proof}
	
	\begin{note}
		We may think of $\bal A$ as the smallest balanced set containing $A$. 
	\end{note}
	
	\begin{ex}
		Let $X$ be a vector space and $A \subset X$. Suppose that $A \neq \varnothing$. If $A$ is balanced, then $0 \in A$.
	\end{ex}
	
	\begin{proof}
		Clear by definition.
	\end{proof}

	\begin{ex}
		Let $X$ be a vector space, $A \subset X$, $x \in X$ and $\lam \in \C$. Suppose that $A$ is balanced. Then $\lam x \in A$ iff $|\lam| x \in A$.  
	\end{ex}
	
	\begin{proof}
		If $\lam = 0$, then the claim is clearly true. Suppose that $\lam \neq 0$. Set $s = \sgn(\lam)$. Suppose that $\lam x \in A$. Since $A$ is balanced and $|s| = |s^{-1}| = 1$, 
		\begin{align*}
			|\lam|x 
			&= s^{-1} \lam x \\
			& \in A
		\end{align*}
		Conversely, suppose that $|\lam| x \in A$. Then 
		\begin{align*}
			\lam x 
			&= s |\lam| x \\
			& \in A
		\end{align*}
	\end{proof}

	\begin{ex}
		Let $X$ be a vector space and $A \subset X$. If $A$ is balanced, then $\cnv A$ is balanced.
	\end{ex}

	\begin{proof}
		Suppose that $A$ is balanced. Let $x \in \cnv A$ and $r \in \C$. Suppose that $|r| \leq 1$. Then there exist $(a_j)_{j=1}^n \subset A$ and $(t_j)_{j=1}^n \subset [0,1]$ such that $x = \sum\limits_{j=1}^n t_j a_j$ and $\sum\limits_{j=1}^n t_j = 1$. Since $A$ is balanced, for each $j \in \{1, \ldots, n\}$, 
		\begin{align*}
			ra_j 
			&\in A \\
			&\subset \cnv A
		\end{align*}
		Since $\cnv A$ is convex, we have that 
		\begin{align*}
			rx 
			&= r\sum\limits_{j=1}^n t_j a_j\\
			&= \sum\limits_{j=1}^n t_j ra_j\\
			&\in \cnv A
		\end{align*}
		Hence $\cnv A$ is balanced..
	\end{proof}



	\begin{defn}
		Let $X$ be a vector space and $A \subset X$. Then $A$ is said to be \tbf{absorbing} if for each $x \in X$, there exists $r > 0$ such that for each $c \in \R$, $|c| \geq r$ implies that $x \in cA$.
	\end{defn}

	\begin{ex}
		Let $X$ be a topological vector space and $A \in \MN(0)$. Then $A$ is absorbing. 
	\end{ex}

	\begin{proof}
		Let $x \in A$. For the sake of contradiction, suppose that for each $r > 0$, there exists $c \in \R$ such that $|c| \geq r$ and $c^{-1}x \in A^c$. Then there exists a sequence $(c_{n})_{n \in \N} \subset \R$ such that for each $n \in \N$, $c_n \geq n$ and $c_n^{-1}x \in A^c$. Since $c_n^{-1} \rightarrow 0$, $c_n^{-1}x \rightarrow 0$. Since $A \in \MN(0)$, $(c_n^{-1}x)_{n \in \N}$ is eventually in $A$. This is a contradiction. So there exists $r > 0$ such that for each $c \in \R$, $|c| \geq r$ implies that $x \in cA$. Hence $A$ is absorbing.
	\end{proof}

	\begin{ex}
		
	\end{ex}
	
	\begin{proof}
		
	\end{proof}

	\begin{defn}
		Let $X$ be a vector space and $A \subset X$. For $x \in X$, set $$T^A_x = \{t > 0: x \in tA\}$$ We define the \tbf{Minkowski functional}, denoted $p_A: X \rightarrow \RG$, by $$p_A(x) = \inf T^A_x$$ 
	\end{defn}

	\begin{ex}
		Let $X$ be a vector space and $A \subset X$. Suppose that $A$ is convex, absorbing and $0 \in A$. Then 
		\begin{enumerate}
			\item $p_A:X \rightarrow \Rg$ 
			\item $p(0) = 0$
			\item $p_A$ is a sublinear functional on $X$
		\end{enumerate}
	\end{ex}

	\begin{proof}\
		\begin{enumerate}
			\item Since $A$ is absorbing, there exists $r >0$ such that for each $c \in \R$, $|c| \geq r$ implies that $x \in cA$. Therefore $p_A(x) \leq |c|$ and $p_A: X \rightarrow \Rg$.
			\item Since $0 \in A$, 
			\begin{align*}
				p_A(0) 
				&= \inf T^A_0 \\
				&= 0
			\end{align*}
			\item \begin{itemize}
				\item Let $\ep >0$. Choose $t_x \in T^A_x$ and $t_y \in T^A_y$ such that $t_x < p_A(x) + \ep/2$ and $t_y < p_A(y) + \ep/2$. By definition, $t_x^{-1}x$, $t_y^{-1}y \in A$. Set $\theta = t_x(t_x +t_y)^{-1} \in (0, 1)$. Since $A$ is convex, 
				\begin{align*}
					(t_x +t_y)^{-1}(x+y) 
					&= (t_x +t_y)^{-1}x + (t_x +t_y)^{-1}y \\
					&= \theta t_x^{-1}x + (1 - \theta)t_y^{-1}y \\
					& \in A
				\end{align*}
				Therefore, $t_x + t_y \in T^A_{x+y}$ and 
				\begin{align*}
					p_A(x+y) 
					& \leq t_x + t_y \\
					& < p_A(x) + \frac{\ep}{2} + p_A(y) + \frac{\ep}{2} \\
					& = p_A(x) + p_A(y) + \ep.
				\end{align*}
				Since $\ep >0$ is arbitrary, $p_A(x+y) \leq p_A(x) + p_A(y)$.
				\item If $\lam =0$, then 
				\begin{align*}
					p_A(\lam x) 
					&= p_A(0) \\
					&= 0 \\
					&= |\lam| p_A(x)
				\end{align*}
				Suppose that $\lam > 0$. Let $t >0$. Then
				\begin{align*}
					p_A(\lam x) 
					&= \inf \{t > 0: \lam x \in tA\} \\
					&= \inf \{t > 0: x \in \lam^{-1} tA\} \\
					&= \inf \{\lam s > 0: x \in sA\} \\
					&= \lam \inf \{ s > 0: x \in sA\} \\
					&= \lam p_A(x)
				\end{align*}
			\end{itemize}
		\end{enumerate}
		So $p$ is a sublinear functional on $X$. 
	\end{proof}

	\begin{ex}
		Let $X$ be a vector space and $A \subset X$. Suppose that $A$ is convex, absorbing and $0 \in A$. Then $p_A^{-1}[0, 1) \subset A$. 
	\end{ex}

	\begin{proof}
		Let $x \in p_A^{-1}[0, 1)$. Then $p_A(x) < 1$. By definition, there exists $t \in (0,1)$ such that $x \in tA$. Thus $t^{-1}x \in A$. Since $0 \in A$ and $A$ is convex, we have that 
		\begin{align*}
			x
			&= t (t^{-1}x) + (1-t)0 \\
			& \in A
		\end{align*}
	 	Since $x \in p_A^{-1}[0, 1)$ is arbitrary, $p_A^{-1}[0, 1) \subset A$.
	\end{proof}

		\begin{ex}
		Let $X$ be a topological vector space and $A \subset X$. Suppose that $A$ is open, convex, and $0 \in A$. Then $p_A^{-1}[0, 1) = A$.\\
		\tbf{Hint:} for $x \in A$, consider the sequence $(1 + 1/n)x$
	\end{ex}
	
	\begin{proof}
		Since $A$ is open and $0 \in A$, $A \in \MN(0)$ which implies that $A$ is absorbing. The previous exercise implies that $p_A^{-1}[0, 1) \subset A$. \\
		Conversely, let $x \in A$. Since $A$ is open, $A \in \MN(x)$. Since $1 + 1/n \rightarrow 1$, $(1 + 1/n)x \rightarrow x$. Therefore, there exits $N \in \N$ such that for each $n \in \N$, $n \geq N$ implies that $(1 + 1/n)x \in A$. In particular, $x \in (1 + 1/N)^{-1}A$. Hence $(1 + 1/N)^{-1} \in T_x^A$ and
		\begin{align*}
			p_A(x)
			&= \leq (1 + 1/N)^{-1} \\
			& < 1
		\end{align*}
		So $x \in p_A^{-1}[0, 1)$ and $A \subset B_{p_A}(0,1)$. \\
	\end{proof}

	\begin{ex}
		Let $X$ be a topological vector space, $A \subset X$ and $x_0 \in A^c$. Suppose that $A$ is convex, $A \in \MN(0)$ and $A$ is open. Then there exists $F \in X^*$ such that $\Re F(x_0) =1$ and $\Re F|_A < 1$. \\
		\tbf{Hint:} Assume $X$ is real.
		\begin{enumerate}
			\item \tbf{Existence:} Consider a special $f \in (\R x_0)^*$ and use $p_A$ to apply the Hahn-Banach theorem.
			\item \tbf{Continuity:} for $\ep > 0$, consider the neighborhood $U_{\ep} = \ep A \cap - \ep A$
		\end{enumerate} 
	\end{ex}

	\begin{proof} Assume that $X$ is real.
		\begin{enumerate}
			\item Define $f \in (\R x_0)^*$ by $f(tx_0) = t$. Then $f(x_0) = 1$. Since $A \in \MN(0)$, $0 \in A$ and a previous exercise implies that $A$ is absorbing. Since $A$ is convex, absorbing and $0 \in A$, $p_A:X \rightarrow [0, \infty)$ is a sublinear functional on $X$. Since $x_0 \in A^c$, the previous exercise implies that $1 \leq p_A(x_0)$. Let $x \in \R x_0$. Then there exists $t \in \R$ such that $x = tx_0$. 
			\begin{itemize}
				\item If $t \geq 0$, then 
				\begin{align*}
					f(x)
					&= t \\
					&\leq tp_A(x_0) \\
					&= p_A(tx_0) \\
					&= p_A(x)
				\end{align*}
				\item If $t<0$, then $-t >0$ and an exercise from the section on sublinear functionals implies that
				\begin{align*}
					f(x)
					&= t \\
					&= < 0 \\
					& \leq p_A(x)
				\end{align*}
			\end{itemize}
				So $f \leq p_A$ on $\R x_0$. The Hahn-Banach theorem implies that there exists $F:X \rightarrow \R$ such that $F$ is linear, $F|_{\R x_0} = f$ and $F \leq p_A$. The previous exercise implies that  $p_A|_A < 1$. Hence $F|_A < 1$. 
		\item Let $V \in \MN(0_{\R})$. Choose $\ep >0$ such that $B(0,\ep) \subset V$. Set $U_{\ep} = \ep A \cap -\ep A$. Then $U_{\ep} \in \MN(0)$. Let $u \in U_{\ep}$. Then $\ep^{-1}u, -\ep^{-1}u \in A$. A previous exercise implies that $p_A^{-1}([0,1)) = A$. Hence 
		\begin{align*}
			\ep^{-1}F(u)
			&= F(\ep^{-1}u) \\
			&\leq p_A(\ep^{-1} u) \\
			&< 1
		\end{align*}
		So $F(u) < \ep$. Similarly, $F(-u) < \ep$. So $-\ep < F(u) < \ep$ and
		\begin{align*}
			F(U_{\ep}) 
			&\subset B(0, \ep) \\
			& \subset V 
		\end{align*}
		Since $V \in \MN(0_{\R})$ is arbitrary, $F$ is continuous at $0$. Since $F$ is linear and $F$ is continuous at $0$, $F$ is continuous. Hence $F \in X^*$.
		\end{enumerate}
	If $X$ is complex, then the previous part implies that there exists $G:X \rightarrow \R$ such that $G$ is continuous, real-linear, $G(x_0) = 1$ and $G|_A < 1$. A previous exercise implies that there exists a unique $F \in X^*$ such that $\Re F = G$.
	\end{proof}

	\begin{ex} \tbf{Hahn-Banach Separation Theorem 1:} \\
		Let $X$ be a topological vector space and $A$, $B \subset X$. Suppose that $A$, $B$ are nonempty, convex and disjoint. If $A$ is open, then there exists $\phi \in X^*$ and $c \in \R$ such that for each $x \in A$, $y \in B$, $$\Re \phi(x) < c \leq \Re \phi(y)$$
		\tbf{Hint:} Assume $X$ is real.
		\begin{enumerate}
			\item Choose $a_0 \in A$ and $b_0 \in B$ and set $x_0 = b_0 - a_0$ and $C = A - B + x_0$. Then there exists $\phi \in X^*$ such that $\phi(x_0) = 1$ and $\phi|_C < 1$.
			\item For each $a \in A$, $b \in B$, $\phi(a) < \phi(b)$. Set $c = \sup\limits_{a \in A}\phi(a)$. Since $\phi$ is not constant, $\phi$ is open.
		\end{enumerate}
	\end{ex}
	
	\begin{proof}\
		Assume $X$ is real.
		\begin{enumerate}
			\item Since $A, B$ are nonempty, there exist $a_0 \in A$ and $b_0 \in B$. Set $x_0 = b_0 - a_0$. Previous exercises imply that $A - B$ is open and convex. Set $C = A - B + x_0$. Then $C$ is open and convex. 
			Since  
			\begin{align*}
				0 
				&= a_0 - b_0 + x_0 \\
				&\in C
			\end{align*}
			$C \in \MN(0)$. For the sake of contradiction, suppose that $x_0 \in C$. Then there exist $a \in A$, $b \in B$ such that $x_0 = a - b + x_0$. This implies that $a = b$. This is a contradiction since $A \cap B = \varnothing$. Hence $x_0 \not \in C$. The previous exercise implies that there exists a $\phi \in X^*$ such that $\phi(x_0) = 1$ and  $\phi|_C < 1$. 
			\item Let $x \in A$ and $y \in B$. Then 
			\begin{align*}
				\phi(a) - \phi(b) + 1
				&= \phi(a) - \phi(b) + \phi(x_0) \\
				&= \phi(a - b + x_0) \\
				& < 1
			\end{align*}
			So $\phi(a) < \phi(b)$. Set $c = \sup\limits_{a \in A}\phi(a)$. Since $A$ is open and $\phi \in X^*$ is open. Thus for each $x \in A$, $y \in B$, 
			$$\phi(x) < c \leq \phi(y)$$
		\end{enumerate}
	If $X$ is complex, then the previous part implies that there exists $f:X \rightarrow \R$ and $c \in \R$ such that $f$ is continuous, real-linear and for each $x \in A$ and $y \in B$, 
	$$f(x) < c \leq f(y)$$ 
	A previous exercise implies that there exists a unique $\phi \in X^*$ such that $\Re \phi = f$.
	\end{proof}
	
		\begin{defn}
		Let $X$ be a vector space and $A \subset X$. Then $A$ is said to be an \tbf{absorbing disk} if $A$ is convex, absorbing and balanced.
	\end{defn}
	
	\begin{ex}
		Let $X$ be a vector space, $p :X \rightarrow \Rg$ a seminorm on $X$ and $r >0$. Then $B_p(0, r)$ is an absorbing disk.
	\end{ex}
	
	\begin{proof}\
		\begin{enumerate}
			\item Let $a, b \in B_p(0, r)$ and $t \in [0,1]$. Then $p(a - x) < r$ and $p(b) < r$. So 
			\begin{align*}
				p([ta + (1 - t)b]) 
				&\leq p(ta + p((1-t)b) \\
				&= tp(a) + (1-t)p(b) \\
				&< tr + (1-t)r \\
				&= r
			\end{align*}
			So $ta + (1 - t)b \in B_p(0, r)$ and $B_p(0, r)$ is convex.
			\item Let $a \in X$. Set $s = (p(a) + 1)/ r$. Then for each $t \geq s$, $tr \geq p(a)+1$ so that 
			\begin{align*}
				a 
				& \in B_p(0, p(a)+ 1) \\
				& \subset B_p(0, tr) \\
				& = tB_p(0, r) 
			\end{align*} 
			So $B_p(0,r)$ is absorbing.
			\item Let $a \in B_p(0, r)$ and $u \in \C$. Uppose that $|u| \leq 1$. Then
			\begin{align*}
				p(ua)
				&= |u|p(a) \\
				&< |u|r \\
				&\leq r
			\end{align*}
			So $ua \in B_p(0, r)$ and $B_p(0, r)$ is balanced. 
		\end{enumerate}
		Since $B_p(0, r)$ is convex, absorbing and balanced, it is an absorbing disk. 
	\end{proof}
	
	
	\begin{ex}
		Let $X$ be a vector space and $A \subset X$. Suppose that $A$ is an absorbing disk. Then $p_A:X \rightarrow \Rg$ is a seminorm on $X$.
	\end{ex}
	
	\begin{proof} Since $A$ is an absorbing disk, $A$ is convex, absorbing and balanced. So $0 \in A$ and the previous exercise tells us that $p$ is a sublinear functional on $X$. Let $x\in X$ and $\lam \in \C$. \\
		If $\lam =0$, then 
				\begin{align*}
					p_A(\lam x) 
					&= p_A(0) \\
					&= 0 \\
					&= |\lam| p_A(x)
				\end{align*}
		Suppose that $\lam \neq 0$. Since $A$ is balanced, for $t >0$, $\lam t^{-1} x \in A$ iff $|\lam| t^{-1} x \in A$. So
				\begin{align*}
					p_A(\lam x) 
					&= \inf \{t > 0: \lam x \in tA\} \\
					&= \inf \{t > 0: x \in |\lam|^{-1} tA\} \\
					&= \inf \{|\lam| s > 0: x \in sA\} \\
					&= |\lam|\inf \{ s > 0: x \in sA\} \\
					&= |\lam | p_A(x)
				\end{align*}
	So $p$ is a seminorm on $X$.
	\end{proof}
	
	\begin{ex}
		Let $X$ be a topological vector space and $A \subset X$. Suppose that $A$ is an absorbing disk and $A$ is open. Then $B_{p_A}(0, 1) = A$.
	\end{ex}

	\begin{proof}
		Clear by previous exercise.
	\end{proof}

	\begin{ex}
		Let $X$ be a topological vector space and $A \subset X$. Suppose that $A$ is an absorbing disk. Then $p_A:X \rightarrow \Rg$ is continuous iff $A$ is open. 
	\end{ex}

	\begin{proof}
		If $A$ is open, then 
		\begin{align*}
			A 
			&= B_{p_A}(0,1) \\
			& \subset \bar{B}_{p_A}(0,1) \\
		\end{align*}
		which implies that $\bar{B}_{p_A}(0,1) \in \MN(0)$. An exercise in the previous section implies that $p_A$ is continuous. \\
		Conversely, if $p_A$ is continuous, then an exercise in the previous section implies that $B_{p_A}(0,1)$ is open. 
	\end{proof}
	

	

	






















	
	
	
	
	
	
	
	

	
	\newpage
	\section{Locally Convex Spaces}
	
	\begin{defn} \ld{def:top_vec:locally_convex:0001}
		Let $X$ be a vector space and $p:X \rightarrow [0, \infty)$ a seminorm on $X$. We equip $X / \ker p$ with the topology induced by the norm $\bar{p}:X / \ker p \rightarrow \Rg$. We define the projection $\pi_p: X \rightarrow X / \ker p$ by $\pi_p(x) = \bar{x} = x + \ker p$.
	\end{defn}
	
	\begin{defn} \ld{def:top_vec:locally_convex:0002}
		Let $X$ be a vector space and $\MP$ a family of seminorms on $X$. Then $\MP$ is said to \tbf{separate points of $X$} if for each $x \in X$, if $x \neq 0$, then there exists $p \in \MP$ such that $p(x) \neq 0$.
	\end{defn}
	
	\begin{defn} \ld{def:top_vec:locally_convex:0003}
		Let $X$ be a vector space, $\MT$ a topology on $X$ and $\MP$ a family of seminorms on $X$. Then $(X, \MT)$ is said to be a \tbf{locally convex space with associated family of seminorms $\MP$} if 
		\begin{itemize}
			\item $\MP$ separates points of $X$
			\item $\MT = \tau_X(\pi_p : p \in \MP)$
		\end{itemize} 
	\tcr{maybe consider changing defintion to $(X, \MT)$ is locally convex if there exists a collection of seminorms $\MP$ on $X$ such that $\MT = \tau_X(\pi_p:p \in P)$. Would need to change all exercise proofs to say that since $(X, \MT)$ is locally convex, there exists $\MP$ such that $\MT = \tau_X(\pi_p:p \in \MP)$.}
	\end{defn}
	
	\begin{note}
		We will generally suppress the family $\MP$ of seminorms and the induced topology $\MT$.
	\end{note}
	
	\begin{ex} \lex{ex:top_vec:locally_convex:0004}
		Let $X$ be a locally convex space and $(x_{\al})_{\al \in A} \subset X$ a net and $x \in X$. Then $x_{\al} \rightarrow x$ iff for each $p \in \MP$, $p(x_{\al} - x) \rightarrow 0$.
	\end{ex}
	
	\begin{proof}
		Suppose that $x_{\al} \rightarrow x$. Let $p \in \MP$. 
		By assumption,  
		\begin{align*}
			\bar{x}_{\al} 
			&= \pi_p(x_{\al}) \\
			&\rightarrow \pi_p(x) \\
			&= \bar{x}
		\end{align*}   
		So 
		\begin{align*}
			p(x_{\al} - x) 
			&= \bar{p}(\bar{x}_{\al} - \bar{x}) \\
			& \rightarrow 0
		\end{align*}
		Conversely, suppose that for each $p \in \MP$, $p(x_{\al} - x) \rightarrow 0$. Let $p \in \MP$. Then 
		\begin{align*}
			\bar{p}(\bar{x}_{\al} - \bar{x}) 
			&= p(x_{\al} - x) \\
			& \rightarrow 0
		\end{align*} 
		So $\pi_p(x_{\al}) \rightarrow \pi_p(x)$. Since $p \in \MP$ is arbitrary, $x_{\al} \rightarrow x$. 
	\end{proof}
	
	\begin{ex} \lex{ex:top_vec:locally_convex:0005}
		Let $X$ be a locally convex space. Then for each $p \in \MP$, $p$ is continuous. 
	\end{ex}
	
	\begin{proof}
		Let $(x_{\al})_{\al \in A} \subset X$ be a net and $x \in X$. Suppose that $x_{\al} \rightarrow x$. Let $p \in \MP$. Then $p(x_{\al} - x) \rightarrow 0$. The reverse triangle inequality implies that 
		\begin{align*}
			|p(x_{\al}) - p(x)| 
			& \leq p(x_{\al} - x) \\
			& \rightarrow 0
		\end{align*}
		So $p(x_{\al}) \rightarrow p(x)$ and $p$ is continuous. 
	\end{proof}
	
	\begin{ex} \lex{ex:top_vec:locally_convex:0006}
		Let $X$ be a locally convex space. Then X is a Hausdorff topological vector space.
	\end{ex}
	
	\begin{proof}\
		\begin{enumerate}
			\item Let $(x_{\al})_{\al \in A}$, $(x_{\al})_{\al \in A} \subset X$ and $(\lam_{\al})_{\al \in A} \subset \C$ be nets and $x$,$y \in X$, $\lam \in \C$. Suppose that $x_{\al} \rightarrow x$, $y_{\al} \rightarrow y$ and $\lam_{\al} \rightarrow \lam$. Let $P \in \MP$. Then 
			\begin{align*}
				p([x_{\al} + y_{\al}] - [x + y]) 
				&= p([x_{\al} - x] + [y_{\al} - y]) \\
				&\leq p(x_{\al} - x) + p(y_{\al} - y) \\
				& \rightarrow 0
			\end{align*}
			Since $p \in \MP$ is arbitrary, $x_{\al} + y_{\al} \rightarrow x + y$ and addition $X \times X \rightarrow X$ is continuous. \\
			
			\item Similiarly, 
			\begin{align*}
				p(\lam_{\al} x_{\al} - \lam x) 
				&= p([\lam_{\al} x_{\al} - \lam x_{\al}] + [\lam x_{\al} - \lam x]) \\
				& \leq p(\lam_{\al} x_{\al} - \lam x_{\al}) + p(\lam x_{\al} - \lam x) \\
				&= p([\lam_{\al} - \lam] x_{\al}) + p(\lam [x_{\al} - x]) \\
				&= |\lam_{\al} - \lam|p(x_{\al}) + |\lam|p(x_{\al} - x) \\
				&\rightarrow 0 
			\end{align*}
			So scalar multiplication $ \C \times X \rightarrow X$ is  continuous. \\

			\item Let $x, y \in X$. Suppose that $x \neq y$. Since $\MP$ separates points of $X$, there exists $p \in \MP$ such that $p(x - y) \neq 0$. Thus $\bar{p}(\bar{x} - \bar{y}) \neq 0$. Thus $\bar{x} \neq \bar{y}$. Since $X / \ker p$ is Hausdorff, there exists $U' \in \MN(\bar{x})$ and $V' \in \MN(\bar{y})$ such that $U' \cap V' = \varnothing$. Set $U = \pi_p^{-1}(U')$ and $V =  \pi_p^{-1}(V')$. Then $U \in \MN(x)$, $V \in \MN(y)$ and  
			\begin{align*}
				U \cap V 
				&= \pi_p^{-1}(U') \cap  \pi_p^{-1}(V') \\
				&=  \pi_p^{-1}(U' \cap V') \\
				&=  \pi_p^{-1}(\varnothing) \\
				&= \varnothing
			\end{align*}
			So $X$ is Hausdorff.
		\end{enumerate}
	\end{proof}

	
	\begin{ex} \lex{ex:top_vec:locally_convex:0007}
		Let $X$ be a locally convex space and $U \in \MN(0)$ open. Then there exist $p \in \MP$ and $r >0$ such that $B_p(0,r) \subset U$.
	\end{ex}

	\begin{proof}
		For the sake of contradiction, suppose that for each $p \in \MP$ and $r >0$, $B_p(0,r) \not \subset U$. Then there exists a sequence $(x_n)_{n \in \N} \subset U^c$ such that for each $p \in \MP$ and $n \in \N$, $p(x_n) < 1/n$. So $x_n \rightarrow 0$. Since $U^c$ is closed, $0 \in U^c$ which is a contradiction. Hence there exist $p \in \MP$ and $r >0$ such that $B_p(0,r) \subset U$.
	\end{proof}

	\begin{ex} \lex{ex:top_vec:locally_convex:0008}
		Let $X$ be a locally convex space. Then for each $U \in \MN(0)$, if $U$ is open, then there exists $V \subset U$ such that $V$ is an open absorbing disk.
	\end{ex}

	\begin{proof}
		Let $U \in \MN(0)$. Suppose that $U$ is open. The previous exercise implies that there exists $p \in \MP$ and $r > 0$ such that $B_p(0,1) \subset U$. A previous exercise tells us that $B_p(0,1)$ is an open absorbing disk.
	\end{proof}

	\begin{ex} \lex{ex:top_vec:locally_convex:0009}
		Let $(X, \MT)$ be a locally convex space with associated  family of seminorms $\MP$ and $M \subset X$ a subspace. Define $\MP_M = \{p|_M: p \in \MP\}$. Then $(M, \MT \cap M)$ is a locally convex space with associated family of seminorms $\MP_M$. 
	\end{ex}

	\begin{proof}
		Let $(x_{\al})_{\al \in A} \subset M$ be a net and $x \in M$. Suppose that $x_{\al} \rightarrow x$ in $\MT \cap M$. Then an exercise in the section on the subspace topology implies that $x_{\al} \rightarrow x$ in $\MT$. Let $q \in \MP_M$. Then there exists $p \in \MP$ such that $q = p|_M$. Therefore
		\begin{align*}
			q(x_{\al} - x) 
			&= p|_M(x_{\al} - x) \\ 
			&= p(x_{\al} - x) \\
			& \rightarrow 0
		\end{align*}  
		Hence $x_{\al} \rightarrow x$ in $\tau_X(\pi_q:q \in \MP_M)$. \\
		Conversely, suppose that $x_{\al} \rightarrow x$ in $\tau_X(\pi_q:q \in \MP_M)$. Let $p \in \MP$. Then 
		\begin{align*}
			p(x_{\al} - x) 
			&= p|_M(x_{\al} - x) \\
			& \rightarrow 0
		\end{align*}
		Hence $x_{\al} \rightarrow x$ in $\MT$. So $x_{\al} \rightarrow x$ in $\MT \cap M$. Therefore $\MT \cap M = \tau_X(\pi_q: q \in \MP_M)$. 
	\end{proof}
	
	\begin{ex} \lex{ex:top_vec:locally_convex:0010}
		Let $X$ be a locally convex space, $M \subset X$ a subspace and $f \in M^*$. Then there exists $F \in X^*$ such that $F|_M = f$.  
	\end{ex}

	\begin{proof}
		Set $p_f = |f|$. Since $p_f$ is a continuous seminorm, $B_{p_f}(0,1)$ is open in $M$. Therefore, there exists $U \subset X$ open such that $B_{p_f}(0,1) = U \cap M$. A previous exercise implies that there exists $p \in \MP$ and $r >0$ such that $B_p(0, r) \subset U$. Set $A = B_p(0, r)$. Since $A$ is open, $p_A:X \rightarrow \Rg$ is continuous and $A = B_{p_A}(0,1)$. Hence 
		\begin{align*}
			B_{p_A|_M}(0,1) 
			&= A \cap M \subset U \cap M \\
			&= B_{p_f}(0,1) 
		\end{align*}  
		Therefore $p_f \leq p_A|_M$ and $|f| \leq p_A$ on $M$. The Hahn-Banach theorem implies that there exists $F:X \rightarrow \C$ such that $F$ is linear, $F|_M = f$ and $|F| \leq p_A$. Since $p_A$ is continuous, $|F|$ is continuous, which implies that $F$ is continuous. So $F \in X^*$.  
	\end{proof}
	
	\begin{ex} \lex{ex:top_vec:locally_convex:0011} \tbf{Hahn-Banach Separation Theorem 2:}\\
		Let $X$ be a locally convex space and $A$, $B \subset X$. Suppose that $A$, $B$ are nonempty, convex and disjoint. If $A$ is compact and $B$ is closed, then there exists $\phi \in X^*$ and $c_1, c_2 \in \R$ such that for each $x \in A$, $y \in B$, $$\Re \phi(x) < c_1 < c_2 \leq \Re \phi(y)$$
		\tbf{Hint:} Assume $X$ is real. Since $X$ is locally convex, there exists $V \subset U$ such that $V$ is an open absorbing disk and $(A + V) \cap B = \varnothing$. Then apply the first Hahn-Banach separation theorem to $A+V$ and $B$.
	\end{ex}
	
	\begin{proof}
		Assume $X$ is real. Suppose that $A$ is compact and $B$ is closed. A previous exercise implies that there exists $U \in \MN(0)$ such that $U$ is open and $(A + U) \cap B = \varnothing$. Since $X$ is locally convex, there exists $V \subset U$ such that $V$ is an open  absorbing disk. Then $(A + V)$ is open and convex. By the first Hahn-Banach separation theorem, there exist $\phi \in X^*$ and $c_2 \in \R$ such that for each $x \in A + V$, $y \in B$, $$\phi(x) < c_2 \leq \phi(y)$$
		Specifically, $c_2 = \sup\limits_{x \in A + V} \phi(x)$. Since $\phi \in X^*$ is not constant, $\phi$ is open and thus $\phi(A + V)$ is open. Continuity of $\phi$ implies that $\phi(A)$ is compact. Therefore, $\sup \phi(A) < \sup \phi(A + V)$. So there exists $c_1 \in \phi(A + V)$ such that $\sup \phi(A) < c_1$. Hence there exists $x_1 \in A + V$ such that $\phi(x_1) = c_1$. Then for each $x \in A$ and $y \in B$, 
		\begin{align*}
			\phi(x) 
			&\leq \sup \phi(A) \\
			&< c_1 \\
			&= \phi(x_1) \\
			&< c_2 \\
			& \leq \phi(y)
		\end{align*}
	If $X$ is complex, then the previous part implies that there exists $f:X \rightarrow \R$ and $c_1, c_2 \in \R$ such that $f$ is continuous, real-linear and for each $x \in A$ and $y \in B$, 
	$$f(x) < c_1 < c_2 \leq f(y)$$ 
	A previous exercise implies that there exists a unique $\phi \in X^*$ such that $\Re \phi = f$.
	\end{proof}


	\begin{ex} \lex{ex:top_vec:locally_convex:0012}
		Let $X$ be a locally convex space and $M \subset X$ a closed subspace. If $M \neq X$, then there exists $\phi \in X^*$ such that $\phi \neq 0$ and $\phi|_M = 0$. 
	\end{ex}

	\begin{proof}
		Assume that $X$ is real. Suppose that $M \neq X$. Then there exists $x_0 \in X$ such that $x_0 \not \in M$. Since $\{x_0\}$ is compact and convex, $M$ is closed and convex and $\{x_0\} \cap M = \varnothing$, the second Hahn-Banach separation theorem implies that there exists $\phi \in X^*$ such that for each $x \in M$, $$\phi(x_0) < \phi(x)$$
		Since $0 \in M$, 
		\begin{align*}
			\phi(x_0)
			&< \phi(0) \\
			&= 0
		\end{align*}
	so that $\phi \neq 0$. For the sake of contradiction, suppose that $\phi|_M \neq 0$. Then there exists $x_1 \in M$ such that $\phi(x_1) \neq 0$. Then for each $t \in \R$, 
	\begin{align*}
		\phi(x_0) 
		&< \phi(tx_1) \\
		&= t \phi(x_1)
	\end{align*}
	Set $t = \frac{\phi(x_0)}{\phi(x_1)}$. Then 
	\begin{align*}
		\phi(x_0) 
		&< t \phi(x_1) \\
		&= \phi(x_0)
	\end{align*}
	which is a contradiction. So $\phi|_M = 0$.
	\end{proof}

	\begin{ex} \lex{ex:top_vec:locally_convex:0013}
		Let $X$ be a locally convex space. Then $X^*$ separates the points of $X$. 
	\end{ex}

	\begin{proof}
		Let $x, y \in X$. The second Hahn-Banach separation theorem implies that there exists $\phi \in X^*$ such that $\phi(x) \neq \phi(y)$. 
	\end{proof}

	\begin{ex} \lex{ex:top_vec:locally_convex:0014}
		Let $X$ be a vector space, $(Y_{\al}, \MT_{\al})_{\al \in A}$ a collection of locally convex spaces and $(\phi_{\al})_{\al \in A} \in \prod\limits_{\al \in A} \ML(X, Y_{\al})$. Set $\MT \defeq \sig_X(\phi_{\al}: \al \in A)$. Then $(X, \MT)$ is a locally convex space. 
	\end{ex}

	\begin{proof}
		Since for each $\al \in A$, $(Y_{\al}, \MT_{\al})$ is locally convex, we have that for each $\al \in A$, there exists a collection of seminorms $\MP_{\al}$ on $Y_{\al}$ such that $\MT_{\al} = \tau_{Y_{\al}}(\pi_q:q \in \MP_{\al})$. Define $\MP \defeq \{q \circ \phi_{\al}: \al \in A \text{ and } q \in \MP_{\al}\}$. \rex{ex:top_vec:seminorm:0004} implies that $\MP$ is a collection of seminorms on $X$. Define $\MT' \subset \MP(X)$ by $\MT' \defeq \tau_X(\pi_p: p \in \MP)$. Let $(x_{\gam})_{\gam \in \Gam} \subset X$ be a net and $x \in X$. 
		\begin{itemize}
			\item Suppose that $x_{\gam} \rightarrow x$ in $(X, \MT)$. \tcr{need exercise showing $(X, \MT)$ is a topological vector space.} 
			
			
			Since $(X, \MT)$ is a topological vector space, $x_{\gam} - x \rightarrow 0$ in $(X, \MT)$. Let $p \in \MP$. Then there exists $\al \in A$ and $q \in \MP_{\al}$ such that $p = q \circ \phi_{\al}$. Since $\MT = \sig_X(\phi_{\al}: \al \in A)$, $\phi_{\al}$ is $(\MT, \MT_{\al})$-continuous. Since $\MT_{\al} = \tau_{Y_{\al}}(\pi_q:q \in \MP_{\al})$, \rex{ex:top_vec:locally_convex:0005} implies that for each $q \in \MP_{\al}$, $q$ is $(\MT_{\al}, \MT_{\R})$-continuous. Thus $q \circ \phi_{\al}$ is $(\MT, \MT_{\R})$-continuous. Hence 
			\begin{align*}
				p(x_{\gam} - x)
				& = q \circ \phi_{\al}(x_{\gam} - x) \\
				& \rightarrow q \circ \phi_{\al}(0) \\
				& = q(0) \\
				& = 0.
			\end{align*}
			Since $p \in \MP$ is arbitrary, we have that for each $p \in \MP$, $p(x_{\gam} - x) \rightarrow 0$. \rex{ex:top_vec:locally_convex:0004} implies that $x_{\gam} \rightarrow x$ in $(X, \MT')$.
			\item Suppose that $x_{\gam} \rightarrow x$ in $(X, \MT')$. Let $\al \in A$ and $q \in \MP_{\al}$. Since $q \circ \phi_{\al} \in \MP$, \rex{ex:top_vec:locally_convex:0004} implies that  
			\begin{align*}
				q(\phi_{\al}(x_{\gam}) - \phi_{\al}(x))
				& = q \circ \phi_{\al}(x_{\al} - x) \\
				& \rightarrow 0.
			\end{align*}
			Since $q \in \MP_{\al}$ is arbitrary, we have that for each $q \in \MP_{\al}$, $q(\phi_{\al}(x_{\gam}) - \phi_{\al}(x)) \rightarrow 0$. \rex{ex:top_vec:locally_convex:0004} then implies that $\phi_{\al}(x_{\gam}) \rightarrow \phi_{\al}(x)$ in $(Y_{\al}, \MT_{\al})$. Since $\al \in A$ is arbitrary, we have that for each $\al \in A$, $\phi_{\al}(x_{\gam}) \rightarrow \phi_{\al}(x)$ in $(Y_{\al}, \MT_{\al})$. \rex{ex:nets:0020} implies that $x_{\gam} \rightarrow x$ in $(X, \MT)$. 
		\end{itemize}
		Thus $x_{\gam} \rightarrow x$ in $(X, \MT')$ iff $x_{\gam} \rightarrow x$ in $(X, \MT)$. \rex{ex:nets:0021} then implies that $\MT = \MT'$.
	\end{proof}


































	\newpage
	\section{Subspaces}
	
	\tcr{previous section has results about subspaces. consider moving those exercises to this section, including hahn-banach theorems}



























\newpage
\section{Products}








	
	
	
	
	
	
	
	
	
	
	
	
	
	
	
	
	\newpage
	\section{Direct Sums}

	























	\newpage
	\section{Quotient Spaces}
	
	\begin{ex}
		Let $X$ be a topological vector space and $M \subset X$ a subspace. Then $\pi: X \rightarrow X / M$ is open. 
	\end{ex}

	\begin{proof}
		Define the action $\phi: M \times X \rightarrow X$ by $m \cdot x = x + m$. Then $M \cdot x = x+ M$. Since for each $m \in M$, the map $x \mapsto x +m$ is continuous, \rex{ex:quotient_topology:0020} implies that $\pi: X \rightarrow X/M$ is open.  
	\end{proof}
	
	\begin{ex}
		Let $(X, \MT)$ be a topological vector space and $M \subset X$ a subspace. Then $(X/M, \MT_{X/M})$ is a topological vector space.
	\end{ex}

	\begin{proof}
		Denote addition on $X$ and $X /M$ by $A: X^2 \rightarrow X$ and $\bar{A}:(X /M)^2 \rightarrow X/ M$ respectively. Similarly, denote scalar multiplication on $X$ and $X /M$ by $\Lam: \C \times X \rightarrow X$ and $\bar{\Lam}:\C \times (X /M) \rightarrow X/ M$ respectively. 
		\begin{itemize}
			\item Let $\bar{x}, \bar{y} \in X /M$. Let $U \in \MN(\bar{x} + \bar{y})$. Since $\pi: X \rightarrow X / M$ is continuous, we have that $\pi^{-1}(U) \in \MN(x+y)$. Since addition $A: X^2 \rightarrow X$ is continuous,
			\begin{align*}
				(\pi \circ A)^{-1}(U) 
				& = A^{-1}(\pi^{-1}(U)) \\
				&\in \MN(x,y)
			\end{align*}
			Since $\MB = \{A \times B: A,B \subset X \text{ and $A,B$ are open}\}$ is a basis for the product topology on $X^2$, there exist $V_x \times V_y \in \MB$ such that $(x,y) \in V_x \times V_y \subset (\pi \circ A)^{-1}(U)$. Thus $V_x \in \MN(x), V_y \in \MN(y)$ and $V_x \times V_y \in \MN(x,y)$. Recall that $\pi \times \pi: X^2 \rightarrow (X /M)^2$ is defined by $\pi \times \pi (x,y) = (\pi(x), \pi(y))$. For $x,y \in X$, we have that 
			\begin{align*}
				\bar{A}\circ (\pi \times \pi)(x,y)
				& = \bar{A}(\bar{x}, \bar{y}) \\
				& = \bar{x} + \bar{y} \\
				&= \pi(x) + \pi(y) \\
				&= \pi (x + y) \\
				&= \pi \circ A (x,y) 
			\end{align*}
			So $\bar{A}\circ (\pi \times \pi) = \pi \circ A$.  Since $\pi$ is open, an exercise in the section on the product topology implies that $\pi \times \pi$ is open and therefore $\pi \times \pi(V_x \times V_y) \in \MN(\bar{x}, \bar{y})$. Hence
			\begin{align*}
				\bar{A} \circ (\pi \times \pi) (V_x \times V_y) 
				& \subset \bar{A} \circ (\pi \times \pi)((\pi \circ A)^{-1}(U)) \\
				& = \bar{A} \circ (\pi \times \pi)((\bar{A} \circ (\pi \times \pi))^{-1}(U)) \\
				& \subset U
			\end{align*} 
			So for each $U \in \MN(\bar{x} + \bar{y})$, there exists $\pi \times \pi(V_x \times V_y) \in \MN(\bar{x}, \bar{y})$ such that $\bar{A}(\pi \times \pi(V_x \times V_y)) \subset U$. Hence $\bar{A}$ is continuous at $(\bar{x}, \bar{y})$. Since $\bar{x}, \bar{y} \in X/ M$ are arbitrary, $\bar{A}$ is continuous. 
			\item Let $\lam \in \C$ and $\bar{x} \in X / M$. Let $U \in \MN(\lam \bar{x})$. Since $\pi$ is continuous, $\pi^{-1}(U) \in \MN(\lam x)$. Since scalar multiplication $\Lam: \C \times X  \rightarrow X$ is continuous, 
			\begin{align*}
				\Lam^{-1}(\pi^{-1}(U)) 
				& = (\pi \circ \Lam)^{-1}( U) \\
				& \in \MN(\lam, x)
			\end{align*} 
			Since $\MB = \{A \times B: \text{$A \subset \C$, $B \subset X$ and $A,B$ are open}\}$ is a basis for the product topology on $\C \times X$, there exist $V_{\lam} \times V_x \in \MB$ such that $(\lam,x) \in V_x \times V_y \subset  (\pi \circ \Lam)^{-1}(U)$. Thus $V_{\lam} \in \MN(\lam), V_x \in \MN(x)$ and $V_{\lam} \times V_x \in \MN(\lam, x)$. As in the previous part, $\pi \circ \Lam = \bar{\Lam} \circ (\id_{\C} \times \pi)$ and $\id_{\C}$ is open. Hence $\id_{\C} \times \pi$ is open and $\id_{\C} \times \pi (V_{\lam} \times V_x) \in \MN(\lam, \bar{x})$. As in the previous part we have that 
			\begin{align*}
				\bar{\Lam} \circ (\id_{\C} \times \pi) (V_{\lam} \times V_x) 
				& \subset \bar{\Lam} \circ (\id_{\C} \times \pi)((\pi \circ \Lam)^{-1}(U)) \\
				& = \bar{\Lam} \circ (\id_{\C} \times \pi)((\bar{\Lam} \circ (\id_{\C} \times \pi))^{-1}(U)) \\
				& \subset U
			\end{align*} 
			So for each $U \in \MN(\lam \bar{x})$, there exists $\id_{\C} \times \pi(V_{\lam} \times V_x) \in \MN(\lam, \bar{x})$ such that $\bar{\Lam}(\id_{\C} \times \pi(V_{\lam} \times V_x)) \subset U$. Hence $\bar{\Lam}$ is continuous at $(\lam, \bar{x})$. Since $\lam \in \C$ and $\bar{x} \in X/ M$ are arbitrary, $\bar{\Lam}$ is continuous. 
		\end{itemize}
	\end{proof}

	\begin{ex}
		Let $X$ be a topological vector space and $M \subset X$ a subspace. Then $M$ is closed iff $X / M$ is Hausdorff. 
	\end{ex}
	
	\begin{proof}
		Suppose that $M$ is closed. Define the action $\phi: M \times X \rightarrow X$ by $m \cdot x = m  + x$. Denote by $\sim$, the equivalence relation induced by $\phi$ (i.e. $x \sim y$ iff $x-y \in M$). A previous exercise implies that $\pi: X \rightarrow X /M$ is open. Let $(x_{\al},y_{\al})_{\al \in A} \subset  {\sim}$ be a net and $(x,y) \in X \times X$. Suppose that $(x_{\al},y_{\al}) \rightarrow (x,y)$. Then $x_{\al} \rightarrow x$ and $y_{\al} \rightarrow y$. Therefore $x_{\al} - y_{\al} \rightarrow x -y$. Since for each $\al \in A$, $x_{\al} - y_{\al} \in M$ and $M$ is closed, we have that $x -y \in M$. Hence $(x,y) \in  {\sim}$ and $\sim$ is closed. Since $\pi$ is open, a previous exercise in the section on separation and countability implies that $X / M$ is Hausdorff. \\
		Conversely, suppose that $X/ M$ is Hausdorff. Then $\{0 + M\}$ is closed in $X /M$. Since $\pi: X \rightarrow X/M$ is continuous, we have that $M =  \pi^{-1}(0 + M)$ is closed in $X$.
	\end{proof}

	\begin{ex}
		Let $X$ be a topological vector spaces and $\phi : X \rightarrow \C$ linear. Then $\ker \phi$ is closed iff $\phi$ is continuous. \\
		\tbf{Note: need to show that if $T:X \rightarrow Y$ is linear, then $T$ is continuous iff $\bar{T}: X / \ker T \rightarrow Y$ is continuous}
	\end{ex}
	
	\begin{proof}
		Suppose that $\phi$ is continuous. Since $\{0\} \subset \C$ is closed, $\ker \phi = \phi^{-1}(\{0\})$ is closed. \\ Conversely, suppose that $\ker \phi$ is closed. Then $X/ \ker \phi$ is Hausdorff. Hence 
		\tcb{FINISH!!!}
	\end{proof}
	
	\begin{ex}
		Let $X$ be a topological vector space and $\phi,\psi \in X^*$. If $\ker  \phi \subset \ker \psi$, then there exists $\lam \in \C$ such that $\psi = \lam \phi$.\\
		\tbf{Hint:} This is just a fact about vector spaces. The isomorphism theorems imply that there exists $g: \Img \phi \rightarrow \Img \psi$ such that $\psi = g \circ \phi$. 
	\end{ex}

	\begin{proof}
		Suppose that $\ker  \phi \subset \ker \psi$. If $\phi = 0$, then 
		\begin{align*}
			X 
			& = \ker \phi\\
			& \subset \ker \psi 
		\end{align*}
		So 
		\begin{align*}
			\psi 
			& = 0 \\
			& = \phi
		\end{align*}
		Suppose that $\phi \neq 0$. Then $\Img \phi = \C$. Let $\pi_{\phi}: X \rightarrow X / \ker \phi$ and $\pi_{\psi}: X \rightarrow X / \ker` \psi$ be the canonical projection maps and let $\tilde{\phi}: X / \ker \phi \rightarrow \Img \phi$ and $\tilde{\psi}:X / \ker \psi \rightarrow \Img \psi$ be the unique maps such that $\tilde{\phi} \circ \pi_{\phi} = \phi$ and $\tilde{\psi} \circ \pi_{\psi} = \psi$. Note that $\tilde{\phi}$ and $\tilde{\psi}$ are vector space isomorphisms. Define the linear map $\iota: X /\ker \phi \rightarrow X / \ker \psi$ by $\iota(x + \ker \phi) = x + \ker \psi$. Let $x,y \in X$. If $x + \ker \phi = y + \ker \phi$, then 
		\begin{align*}
			x -y 
			& \in \ker \phi \\
			& \subset \ker \psi 
		\end{align*}
		So 
		\begin{align*}
			\iota(x) 
			& = x + \ker \psi \\
			& = y + \ker \psi \\
			& = \iota(y)
		\end{align*}
		and $\iota$ is well defined. Define $g: \Img \phi \rightarrow \Img \psi$ by $g(y) = \tilde{\psi} \circ \iota \circ \tilde{\phi}^{-1}$. Set $\lam = g(1)$. Since $g: \C \rightarrow \C$ is linear, $g = \lam \id_{\C}$. Thus the following diagram commutates: 
		\[ 
		\begin{tikzcd}
			& X_{\al} \arrow[dr, "\pi_{\psi}"] \arrow[dl, "\pi_{\phi}"'] &  \\
			X / \ker \phi \arrow[rr, "\iota"] \arrow[d, "\tilde{\psi}"] & & X / \ker \psi  \arrow[d, "\tilde{\psi}"] \\
			\Img \phi \arrow[rr, "g = \lam \id_{\C}"'] & & \Img \psi
		\end{tikzcd}
		\]
		Hence 
		\begin{align*}
			\psi
			& = g \circ \phi \\
			& = \lam \id_{\C} \circ \phi \\
			& = \lam \phi
		\end{align*} 
	\end{proof}


	
	
	
	
	
	
	
	
	
	
	
	
	
	
	
	
	
	
	
	
	
	
	
	
	
	\newpage
	\section{Duality}
	
	\subsection{Weak Topology}
	
	\begin{defn} \ld{def:top_vec_space:duality:0001}
		Let $X,Y$ and $Z$ be topological vector spaces (over the same field) and $b \in \ML^2(X,Y;Z)$. Then $(X,Y,Z, b)$ is said to be a \tbf{pairing}. 
	\end{defn}

	\begin{defn}  \ld{def:top_vec_space:duality:0002}
		Let $X,Y$ and $Z$ be topological vector spaces and $b \in \ML^2(X,Y;Z)$. We define the \tbf{dual pairing} of $(X,Y,Z, b)$, denoted $(Y,X,Z, b^*)$, by $b^*(y,x) \defeq b(x,y)$. 
	\end{defn}

	\begin{ex}  \lex{ex:top_vec_space:duality:0003}
		Let $X,Y$ and $Z$ be topological vector spaces and $b \in \ML^2(X,Y;Z)$. Then $(Y,X,Z, b^*)$ is a pairing.
	\end{ex}

	\begin{proof}
		Clear.
	\end{proof}

	\begin{defn}  \ld{def:top_vec_space:duality:0004}
		Let $X,Y$ and $Z$ be topological vector spaces and $b \in \ML^2(X,Y;Z)$. We define the \tbf{weak topology on $X$ induced by $b$}, denoted $\sig_b(X, Y)$ by 
		$$\sig_b(X,Y) \defeq \tau_{X}(b(\cdot, y): y \in Y)$$ 
		We define the \tbf{weak topology on $Y$ induced by $b$}, denoted $\sig_b(Y, X)$, by $\sig_b(Y,X) \defeq \sig_{b^*}(Y, X)$.
	\end{defn}

	\begin{ex} \lex{ex:top_vec_space:duality:0005}
		Let $X,Y$ and $Z$ be topological vector spaces, $b \in \ML^2(X,Y;Z)$, $(x_{\al})_{\al \in A} \subset X$ a net, $(y_{\al})_{\al \in A} \subset Y$ a net and $x \in X$, $y \in Y$. Then  
		\begin{enumerate}
			\item $x_{\al} \rightarrow x$ in $(X, \sig_b(X,Y))$ iff for each $y \in Y$, $b(x_{\al}, y) \rightarrow b(x, y)$
			\item $y_{\al} \rightarrow y$ in $(Y, \sig_b(Y,X))$ iff for each $x \in X$, $b(x, y_{\al}) \rightarrow b(x, y)$.
		\end{enumerate}
	\end{ex}
	
	\begin{proof}
		Immediate by \rex{ex:nets:0020}.
	\end{proof}

	\begin{ex} \lex{ex:top_vec_space:duality:0006}
		Let $X,Y$ and $Z$ be topological vector spaces and $b \in \ML^2(X,Y;Z)$. Then 
		\begin{enumerate}
			\item $(X, \sig_b(X, Y))$ is a topological vector space
			\item $(Y, \sig_b(Y, X))$ is a topological vector space
		\end{enumerate}
	\end{ex}

	\begin{proof}\
		\begin{enumerate}
			\item Let $(u_{\al})_{\al \in A}$, $(v_{\al})_{\al \in A} \subset X$ and $(\lam_{\al})_{\al \in A} \subset \C$ be nets and $u,v \in X$ and $\lam \in \C$. Suppose that $u_{\al} \rightarrow u$ and $v_{\al} \rightarrow v$ in $(X, \sig_b(X,Y))$ and $\lam_{\al} \rightarrow \lam$ in $(\K, \MT_{\K})$. Let $y \in Y$. Then $b(u_{\al}, y) \rightarrow b(u, y)$ and $b(v_{\al}, y) \rightarrow b(v,y)$. Since $Z$ is a topological vector space,
			\begin{align*}
				b(u_{\al} + v_{\al}, y) 
				& = b(u_{\al}, y) + b( v_{\al}, y) \\
				& \rightarrow b(u, y) + b(v, y) \\
				& = b(u+v, y)
			\end{align*}
		and 
			\begin{align*}
				b(\lam_{\al} u_{\al} , y) 
				& = \lam_{\al} b(u_{\al}, y)\\
				& \rightarrow \lam b(u, y) \\
				& = b(\lam u, y)
			\end{align*}
			Since $y \in Y$ is arbitrary, $u_{\al} + v_{\al} \rightarrow u+v$ and $\lam_{\al}u_{\al} \rightarrow \lam u$. Hence addition $X \times X \rightarrow X$ and scalar multiplication $\C \times X$ $\rightarrow X$ are continuous.
			\item Since $\sig_b(X,Y) = \sig_{b^*}(Y,X)$, $(1)$ implies $(2)$. 
		\end{enumerate}
	\end{proof}

	\begin{defn} \ld{def:top_vec_space:duality:0007}
		Let $X,Y$ and $Z$ be topological vector spaces and $b \in \ML^2(X,Y;Z)$. Then 
		\begin{itemize}
			\item $Y$ is said to \tbf{separate the points of $X$ via $b$} if
			for each $x \in X$, $x \neq 0$ implies that there exists $y \in Y$ such that $b(x,y) \neq 0$
			\item $X$ is said to \tbf{separate the points of $Y$ via $b$} if $X$ separates the points of $Y$ via $b^*$
		\end{itemize}
	\end{defn}

	\begin{ex} \lex{ex:top_vec_space:duality:0008}
		Let $X,Y$ and $Z$ be topological vector spaces and $b \in \ML^2(X,Y;Z)$. Suppose that $Z$ is Hausdorff.   
		\begin{enumerate}
			\item if $Y$ separates the points of $X$ via $b$, then $(X, \sig_b(X,Y))$ is Hausdorff
			\item 
		\end{enumerate}
	\end{ex}

	\begin{proof} \
		\begin{enumerate}
			\item Suppose that $Y$ separates the points of $X$ via $b$. Let $x_1, x_2 \in X$. Suppose that $x_1 \neq x_2$. Then $x_1 - x_2 \neq 0$. Hence there exists $y \in Y$ such that 
			\begin{align*}
				b(x_1, y) - b(x_2,y)
				& = b(x_1 - x_2,y) \\
				& \neq 0
			\end{align*}
			 Since $Z$ is Hausdorff, there exist $V_1 \in \MN(b(x_1, y)), V_2 \in \MN(b(x_2, y))$ such that $V_1$ and $V_2$ are open and $V_1 \cap V_2 = \varnothing$. Set $U_1 = b(\cdot, y)^{-1}(V_1)$ and $U_2 = b(\cdot, y)^{-1}(V_2)$. By definition of $\sig_b(X,Y)$, $b(\cdot, y): X \rightarrow Z$ is continuous. Thus $U_1$, $U_2 \in \sig_b(X,Y)$, $x_1 \in U_1$, $x_2 \in U_2$ and 
			 \begin{align*}
			 	U_1 \cap U_2
			 	& = b(\cdot, y)^{-1}(V_1) \cap b(\cdot, y)^{-1}(V_2) \\
			 	& = b(\cdot, y)^{-1}(V_1 \cap V_2) \\
			 	& = b(\cdot, y)^{-1}(\varnothing) \\
			 	& = \varnothing
			 \end{align*}
		 	Therefore $(X, \sig_b(X,Y))$ is Hausdorff. \\
			\item 
		\end{enumerate}
	\end{proof}

	\begin{defn}
		Let $X$ be a topological vector space. 
		\begin{itemize}
			\item We define the \tbf{canonical pairing of $X$ with $X^*$}, denoted $\l \cdot, \cdot \r: X \times X^* \rightarrow \C$, by $\l x, \phi \r \defeq \phi(x)$. 
			\item For each $x \in X$, we define $\hat{x}:X^* \rightarrow \C$ by $\hat{x} \defeq \l x, \cdot \r$.
			\item We define $\hat{X} \subset \C^{X^*}$ by $\hat{X} = \{\hat{x}: x \in X\}$.
		\end{itemize}
	\end{defn}

	\begin{defn}
		Let $X$ be a topological vector space. We define the \tbf{weak topology on $X$}, denoted $\MT_w$, by $\MT_w \defeq \sig_{\l \cdot, \cdot \r} (X, X^*)$.
	\end{defn}

	\begin{ex}
		Let $X$ be a topological vector space. Then $\MT_w = \tau_X(X^*)$.
	\end{ex}

	\begin{proof}
		Clear.
	\end{proof}

	\begin{defn}
		Let $X$ be a topological vector space, $(x_{\al})_{\al \in A} \subset X$ and $x \in X$. Then $(x_{\al})_{\al \in A}$ is said to \tbf{converge weakly to $x$}, denoted $x_{\al} \conv{w} x$, if $x_{\al} \rightarrow x$ in $(X, \MT_w)$.
	\end{defn}
	
	\begin{ex} \lex{}
		Let $X$ be a topological vector, $(x_{\al})_{\al \in A} \subset X$ a net and $x \in X$. Then $x_{\al} \conv{w} x$ iff for each $\lam \in X^*$, $ \lam (x_{\al}) \rightarrow \lam(x)$. 
	\end{ex}
	
	\begin{proof}
		Immediate by \rex{ex:top_vec_space:duality:0005}.
	\end{proof}
	
	\begin{defn}
		Let $X$ be a topological vector space. We define the \tbf{weak-* topology on $X^*$}, denoted $\MT_{w*}$, by $\MT_{w*} \defeq \sig_{\l \cdot, \cdot \r} (X^*, X)$. 
	\end{defn}

	\begin{ex}
		Let $X$ be a topological vector space. Then $\MT_{w*} = \tau_{X^*}(\hat{X})$.
	\end{ex}
	
	\begin{proof}
		Clear.
	\end{proof}
	
	\begin{defn}
		Let $X$ be a topological vector space, $(\lam_{\al})_{\al \in A} \subset X^*$ and $\lam \in X^*$. Then $(\lam_{\al})_{\al \in A}$ is said to \tbf{converge in weak-* to $\lam$}, denoted $\lam_{\al} \conv{w^*} \lam$, if $\lam_{\al} \rightarrow \lam$ in $(X^*, \MT_{w*})$.
	\end{defn}
	
	\begin{ex} \lex{}
		Let $X$ be a topological vector, $(\lam_{\al})_{\al \in A} \subset X^*$ a net and $\lam \in X^*$. Then $\lam_{\al} \conv{w^*} \lam$ iff for each $x \in X$, $ \lam_{\al} (x) \rightarrow \lam(x)$. 
	\end{ex}
	
	\begin{proof}
		Immediate by \rex{ex:top_vec_space:duality:0005}.
	\end{proof}
	
	\begin{ex}
		Let $X$ be a topological vector space. 
		\begin{enumerate}
			\item If $X^*$ separates the points of $X$, then  $(X, \MT_w)$ is a locally convex space 
			\item $(X^*, \MT_{w^*})$ is a locally convex space 
		\end{enumerate}
	\end{ex}
	
	\begin{proof}\
		\begin{enumerate}
			\item Suppose that $X^*$ separates the points of $X$. For $\lam \in X^*$, define $p_{\lam}:X \rightarrow \Rg$ by $p_{\lam} = |\lam|$. Set $\MP_{w} = \{p_{\lam}: \lam \in X^*\}$. Then $\MP_{w}$ separates the points of $X$. Let $(x_{\al})_{\al \in A} \subset X$ be a net and $x \in X$. Suppose that $x_{\al} \conv{w} x$. Let $\lam \in X^*$. Then 
			\begin{align*}
				p_{\lam}(x_{\al} - x) 
				&= |\lam(x_{\al} - x)| \\
				&= |\lam(x_{\al}) - \lam(x)| \\
				& \rightarrow 0
			\end{align*}
			So $x_{\al} \rightarrow x$ in $\tau_X(\pi_p: p \in \MP_w)$. \\
			Conversely, suppose that $x_{\al} \rightarrow x$ in $\tau_X(\pi_p: p \in \MP_w)$. Then for each $x \in X$,
			\begin{align*}
				|\lam(x_{\al}) - \lam(x)|
				&= p_{\lam}(x_{\al} - x) \\
				& \rightarrow 0
			\end{align*}
			So that $\lam(x_{\al}) \rightarrow \lam(x)$ and $x_{\al} \conv{w} x$. Hence $\MT_{w} = \tau_{X}(\pi_p: p \in \MP_{w})$ and $(X, \MT_{w})$ is a locally convex space.  \\
			\item For $x \in X$, define $p_x:X^* \rightarrow \Rg$ by $p_x = |\hat{x}|$. Set $\MP_{w^*} = \{p_x:x \in X\}$. Let $\phi \in X^*$. Suppose that $\phi \neq 0$. Then there exists $x \in X$ such that 
			\begin{align*}
				\hat{x}(\phi)
				& = \phi(x)  \\
				& \neq 0
			\end{align*}
			So $\MP_{w^*}$ separates the points of $X^*$. Let $(\lam_{\al})_{\al \in A} \subset X^*$ be a net and $\lam \in X^*$. Suppose that $\lam_{\al} \conv{w^*} \lam$. Let $x \in X$. Then 
			\begin{align*}
				p_x(\lam_{\al} - \lam) 
				&= |\hat{x}(\lam_{\al} - \lam)| \\
				&= |\hat{x}(\lam_{\al}) - \hat{x}(\lam)| \\
				& \rightarrow 0
			\end{align*}
			So $\lam_{\al} \rightarrow \lam$ in $\tau_{X^*}(\pi_p: p \in \MP_{w^*})$. \\
			Conversely, suppose that $\lam_{\al} \rightarrow \lam $ in $\tau_{X^*}(\pi_p: p \in \MP_{w^*})$. Then for each $x \in X$,
			\begin{align*}
				|\hat{x}(\lam_{\al}) - \hat{x}(\lam)|
				&= p_x(\lam_{\al} - \lam) \\
				& \rightarrow 0
			\end{align*}
			So that $\hat{x}(\lam_{\al}) \rightarrow \hat{x}(\lam)$ and $\lam_{\al} \conv{w^*} \lam$. Hence $\MT_{w^*} = \tau_{X^*}(\pi_p: p \in \MP_{w^*})$ and $(X^*, \MT_{w^*})$ is a locally convex space.  
		\end{enumerate}
	\end{proof}

	\begin{note}
		Let $X$ be a topological vector space. When we equip $X^*$ with the weak-$*$ topology, we write $X^{**}$ in place of $(X^*)^*$.
	\end{note}
	
	
	\begin{ex} \lex{}
		Let $X$ be a topological vector space. Then $X^{**} = \hat{X}$. \\
		\tbf{Hint:} Hahn-Banach theorem
	\end{ex}
	
	\begin{proof}
		Let $f \in X^{**}$. Define $p_{f} = |f|$. Then $p_f$ is a continuous seminorm on $X^*$. Therefore $B_{p_f}(0,1)$ is open. \tcb{A previous exercise} implies that there exists $p \in \MP_{w^*}$ and $r >0$ such that 
		\begin{align*}
			B_{r^{-1}p}(0,1)
			& = B_{p}(0,r) \\
			& \subset B_{p_f}(0,1)
		\end{align*}
		\tcb{A previous exercise} implies that $p_f \leq r^{-1}p$. By definition of $\MP_{w^*}$, there exists $x \in X$ such that $p = |\hat{x}|$. Thus
		\begin{align*}
			p_f
			& = |f| \\
			& \leq r^{-1}p \\
			& = |r^{-1}\hat{x}| \\
		\end{align*} 
	Therefore $\ker \hat{x} \subset \ker f$. \tcb{An exercise in the section on quotient spaces of locally convex spaces} implies that there exists $\lam \in \C$ such that 
	\begin{align*}
		f 
		& = \lam r^{-1}\hat{x} \\
		& \in \hat{X}
	\end{align*}
	So $X^{**} = \hat{X}$.
	\end{proof}














































\subsection{Annihilators}

\begin{defn}
	Let $X,Y$ and $Z$ be topological vector spaces, $b \in \ML^2(X,Y;Z)$ and $A \subset X$. We define the \tbf{annihilator of $A$ with respect to $b$}, denoted $A^{\perp_b}$, by $A^{\perp_b} \defeq \{y \in Y: \text{ for each $x \in A$, $b(x, y) = 0$}\}$.
\end{defn}

\begin{note}
	Let $A \subset X$ and $B \subset Y$. When the context is clear, we write $A^{\perp}$ and $B^{\perp}$ in place of $A^{\perp_b}$ and $B^{\perp_{b^*}}$ respectively. 
\end{note}

\begin{ex}
	Let $X,Y$ and $Z$ be topological vector spaces, $b \in \ML^2(X,Y;Z)$, $A \subset X$ and $B \subset Y$. Then 
	\begin{enumerate}
		\item $A^{\perp}$ is a subspace of $Y$.
		\item 
		\item 
		\item 
		\item 
		\item 
	\end{enumerate}
\end{ex}

\begin{proof}
	\begin{enumerate}
		\item Let $y_1, y_2 \in A^{\perp}$, $\lam \in \K$ and $x \in A$. Then 
		\begin{align*}
			b()
		\end{align*}
		\item 
		\item 
		\item 
		\item 
		\item 
	\end{enumerate}
\end{proof}












































\subsection{Adjoints}

\begin{defn}
	
\end{defn}
































\newpage
\section{Continous Linear Maps}

\tcr{redo in terms of "boundedness", need to define bounded subsets, then continuous maps should send bounded sets to bounded sets}

\begin{defn}
	Let $X,Y$ be topological vector spaces. We define 
	$$L(X; Y) = \{T:X \rightarrow Y: T \text{ is linear and continuous}\}$$
\end{defn}

\begin{defn}
	Let $X, Y$ be locally convex spaces with respective associated families of seminorms $\MP$ and $\MQ$ and $p \in \MP$, $q \in \MQ$. We define $\|\cdot\|_{p,q}: L(X; Y) \rightarrow [0, \infty)$ by 
	$$\|T\|_{p,q} = \inf \{C \geq 0: \text{ for each $x \in X$, $q(Tx) \leq Cp(x)$} \}$$
\end{defn}

\begin{ex}
	Let $X, Y$ be locally convex spaces with respective associated families of seminorms $\MP$ and $\MQ$, $p \in \MP$, $q \in \MQ$ and $T \in L(X; Y)$. Then for each $x \in X$, $q(Tx) \leq \|T\|_{p,q}p(x)$. 
\end{ex}

\begin{proof}
	Set $A = \{C \geq 0: \text{ for each $x \in X$, $q(Tx) \leq Cp(x)$} \}$. Let $C \in A$ and $x \in X$. Let $\ep >0$. Then $\ep / [1 + p(x)] > 0$. Hence there exists $C \in A$ such that $$C < \|T\|_{p,q} + \frac{\ep}{1 + p(x)}$$
	Therefore, 
	\begin{align*}
		q(Tx) 
		& \leq Cp(x) \\
		& \leq \bigg[\|T\|_{p,q} + \frac{\ep}{1 + p(x)}\bigg]p(x) \\
		& < \|T\|_{p,q}p(x) + \ep
	\end{align*}
	Since $\ep >0$ is arbitrary, $q(Tx) \leq \|T\|_{p,q}p(x) $. Since $x \in X$ is arbitrary, $\|T\|_{p,q} \in A$. 
\end{proof}

	\begin{ex}
		Let $X, Y$ be locally convex spaces with respective associated families of seminorms $\MP$ and $\MQ$, $p \in \MP$, $q \in \MQ$ and $T \in L(X; Y)$. Then 
		$$\|T\|_{p,q} = \sup \{q(Tx): p(x) = 1\}$$
	\end{ex}

	\begin{proof}
		Let 
	\end{proof}

\begin{ex}
	Let $X, Y$ be locally convex spaces with respective associated families of seminorms $\MP$ and $\MQ$ and $p \in \MP$, $q \in \MQ$. Then $\|\cdot\|_{p,q}$ is a seminorm  on $L(X; Y)$. 
\end{ex}

\begin{proof}
	Let $S, T \in L(X; Y)$ and $\lam \in \C$. 
	\begin{enumerate}
		\item Let $x \in X$. Then 
		\begin{align*}
			q((S+T)(x)) 
			& = q(Sx + Tx) \\
			& \leq q(Sx) + q(Tx) \\
			& \leq \|S\|_{p,q}p(x) + \|T\|_{p,q}p(x) \\
			& =  (\|S\|_{p,q} + \|T\|_{p,q}) p(x)
		\end{align*}
		Since $x \in X$ is arbitrary, $\|S+T\|_{p,q} \leq $
		$\|S + T\|_{p,q} $
		\item Let $x \in X$. Then 
		\begin{align*}
			q((\lam T)(x)) 
			& = q(\lam Tx) \\
			& = |\lam| q(Tx) \\
			& \leq |\lam |\|T\|_{p,q}p(x) \\
		\end{align*}
		Since $x \in X$ is arbitrary, $\|\lam T\|_{p,q} \leq $
	\end{enumerate}
\end{proof}

 

	
	
	
	
	
	
	
	




	
	
	
	
	
	
	
	
	
	
	
	
	
	
	\newpage
	\chapter{Banach Spaces}

	\section{Introduction}
	
	\begin{note}
		In the following, we will consider vector spaces over $\C$. There are analogous results for real vector spaces as well, just replace every $\C$ with $\R$.
	\end{note}
	
	\begin{defn} \ld{}
		Let $X$ be a normed vector space. Then $X$ is said to be a \tbf{Banach space} if $X$ is complete.  
	\end{defn}
	
	\begin{defn} \ld{}
		Let $X$ be a normed vector space and $(x_i)_{i=1}^n \subset X$. The series $\sum_{i =1}^{\infty}x_i$ is said to \tbf{converge} if the sequence $s_n := \sum_{i=1}^n x_i$ converges. The series $\sum_{i =1}^{\infty}x_i$ is said to \tbf{converge absolutely} if $\sum_{i\in \N}\|x_i \|< \infty$.
	\end{defn}
	
	\begin{ex} \lex{}
		Let $X$ be a normed vector space. Then $X$ is complete iff for each $\seq{x}{i} \subset X$, $\sum_{i =1}^{\infty}x_i$ converges absolutely implies that $\sum_{i=1}^{\infty}x_i$ converges. \\
		\tbf{Hint:} Given a Cauchy sequence $(x_n)_{n \in \N}$, obtain a subsequence $(x_{n_j})_{j \in \N} \subset (x_n)_{n \in \N}$ such that for each $j \in \N$, $\|x_{n_{j+1}} - x_{n_{j}}\| < 2^{-j}$. Define a new sequence $(y_j)_{j \in \N} \subset X$ by 
		\[
		y_j = 
		\begin{cases}
		x_{n_1} & j =1 \\  
		x_{n_j} - x_{n_{j-1}} & j \geq 2	
		\end{cases}
		\] 
	\end{ex}
	
	\begin{proof}
		Suppose that $X$ is complete. Let $\seq{x}{i} \subset X$. Suppose that $\sum_{i=1}^{\infty}x_i$ converges absolutely. Let $\ep >0$. Choose $N \in \N$ such that for each $m,n \in \N$, if $m, n \geq N$ and $m< n$, then $\sum_{m+1}^n \|x_i \|< \ep$. Let $m, n \in \N$. Suppose that $m<n$. Then 
		\begin{align*}
			\|s_n-s_m \|
			&= \bigg \|\sum_{i=1}^n x_i -\sum_{i=1}^m x_i\bigg \|\\
			&= \bigg\|\sum_{i=m+1}^{n} x_i \bigg \| \\
			& \leq \sum_{i=m+1}^n \|x_i \|\\
			& < \ep
		\end{align*}
		
		Thus $(s_n)_{n \in N}$ is Cauchy. Since $X$ is complete, $\sum_{i=1}^{\infty}x_i$ converges. \\
		Conversely, Suppose that for each $\seq{x}{i} \subset X$, $\sum_{i =1}^{\infty}x_i$ converges absolutely implies that $\sum_{i=1}^{\infty}x_i$ converges. Let $\seq{x}{i} \subset X$ be Cauchy. Proceed inductively to create a strictly increasing sequence $(n_i)_{i \in \N} \subset \N$ such that for each $m, n \in \N$, if $m,n \geq n_i$, then $ \|x_m-x_n \|< 2^{-i}$. Define $(y_i)_{i \in \N} \subset X$ by 
		\[ y_i = \begin{cases}
			x_{n_1} & i=1 \\
			x_{n_i} - x_{n_{i-1}} & i \geq 2\\
		\end{cases}\]
		
		Then $\sum_{i=1}^k y_i = x_{n_k}$ and 
		\begin{align*}
			\sum_{i \in \N} \|y_i \|
			&= \|x_{n_1} \|+ \sum_{i \in \N} \|x_{n_i}-x_{n_{i-1}} \|\\
			& \leq \|x_{n_1} \|+ 2\sum_{i \in \N}2^{-i}\\
			& = \|x_{n_1} \|+2
		\end{align*}
		Hence $(x_{n_k})_{k \in \N} = (\sum_{i=1}^k y_i)_{i\in \N}$ converges. Since $(x_i)_{i \in \N}$ is cauchy and has a convergent subsequence, it converges. So $X$ is complete.
	\end{proof}
	
	\begin{ex} \lex{}
		Let $X$ be a normed vector space. Then addition $X \times X \rightarrow X$ and scalar multiplication $\C \times X \rightarrow X$ are continuous and $\|\cdot \|:X \rightarrow \Rg$ is continuous.
	\end{ex}
	
	\begin{proof}
		Let $\ep > 0$. Choose $\del = \frac{\ep}{2}$. Let $(x_1,y_1), (x_2,y_2) \in X \times X$. Suppose that 
		$$\max\{\|x_1-x_2 \|, \|y_1 - y_2 \|\} < \del$$
		Then 
		\begin{align*}
			\|(x_1 + y_1) - (x_2+y_2) \|
			&= \|(x_1-x_2) + (y_1-y_2) \|\\
			& \leq \| x_1-x_2 \|+ \|y_1-y_2 \|\\
			& < 2\del \\
			&= \ep
		\end{align*} 
		Hence addition is uniformly continuous. \vspace{1cm}\\ Let $(\lam_1,x_1) \in \C \times X$ and $\ep >0$. Choose $\del = \min\{\frac{\ep}{2(\vert \lam_1 \vert + \|x_1 \|+ 1)}, \frac{\sqrt{\ep}}{\sqrt{2}}\}$. Let $(\lam_2, x_2) \in \C \times X$. Suppose that $$ \max\{\vert \lam_1-\lam_2 \vert , \|x_1 - x_2 \|\} < \del$$ 
		Then 
		\begin{align*}
			\|\lam_1x_1 - \lam_2x_2 \|
			&= \|\lam_1x_1 - \lam_1x_2 + \lam_1x_2- \lam_2x_2 \|\\
			&= \|\lam_1(x_1-x_2) + (\lam_1-\lam_2)x_2 \|\\
			& \leq \vert \lam_1 \vert \| x_1-x_2 \|+ \vert \lam_1-\lam_2 \vert \|x_2\|\\
			& \leq \vert \lam_1 \vert  \| x_1-x_2 \|+ \vert \lam_1-\lam_2 \vert (\|x_1 -x_2\|+ \|x_1\|)\\
			& < \vert \lam_1 \vert \del  +  \del( \del + \|x_1 \|)\\
			&= (\vert \lam_1 \vert + \|x_1 \|) \del + \del^2 \\
			&< \frac{\ep}{2}+ \frac{\ep}{2}\\
			&= \ep
		\end{align*}
		Since $(\lam_1, x_1) \in \C \times X$ is arbitrary, scalar multiplication is continuous. \vspace{1cm} \\ Let $\ep > 0$. Choose $\del = \ep$. Let $x,y \in X$. Suppose that $\|x-y \|< \del$. Then 
		\begin{align*}
			\big \vert \|x \|- \|y \|\big  \vert
			& \leq \|x - y \|\\
			&< \del\\
			&=\ep
		\end{align*}  
		So $\|\cdot \|: X \rightarrow \Rg$ is uniformly continuous.
	\end{proof}
	
	
	
	
	
	
	
	
	
	
	
	
	
	
	\newpage
	\section{Bounded Operators}
	
	\begin{defn} \ld{42001} 
		Let $X,Y$ be a normed vector spaces and $T:X \rightarrow Y$ linear. Then $T$ is said to be \tbf{bounded} if $T(\bar{B}(0,1))$ is bounded. We define $$L(X; Y) = \{T:X \rightarrow Y: T \text{ is linear and bounded}\}$$
		When $X=Y$, we write $L(X)$.
	\end{defn}
	
	\begin{ex} \ld{42001.1} 
		Let $X,Y$ be normed vector spaces and $T:X \rightarrow Y$ linear. Then $T$ is bounded iff there exists $C \geq 0$ such that for each $x \in X$, $$\|Tx \|\leq C \|x \|$$ 
	\end{ex}
	
	\begin{proof}
	Suppose that $T$ is bounded. If $T = 0$, choose $C = 0$. Suppose that $T \neq 0$. Set $ A = \{\|Tx\|: \|x\| =1\}$. Since $T \neq 0$, there exists $x_0 \in X$ such that $\|x_0\| = 1$ so that $A \neq \varnothing$.  Boundedness of $T$ implies that $A$ is bounded. Set $C = \sup A$. Let $x \in X$. If $x = 0$, then $Tx = 0$ and $\|Tx\| \leq C \|x\|$. Suppose that $x \neq 0$. Then $Tx = \|x\| T(\|x\|^{-1} x)$. Since $\|\|x\|^{-1} x\| = 1$, we have that
	\begin{align*}
	\|Tx\|
	&= \|T(\|x\|^{-1} x)\| \|x\|  \\
	& \leq C\|x\| 
\end{align*}	
Conversely, suppose that there exists $C \geq 0$ such that for each $x \in X$, $\|Tx \|\leq C \|x \|$. Let $x \in \bar{B}(0,1)$. Then 
	\begin{align*}
	\|Tx\| 
	&\leq C \|x\| \\
	&\leq C
	\end{align*}
So that $T(\bar{B}(0,1))$ is bounded. 
	\end{proof}


	\begin{ex} \lex{ex:banach:bounded_ops:0003}
		Let $X,Y$ be normed vector spaces and $T \in L(X; Y)$. Then $T$ is an isometry iff for each $x \in X$, 
		$$\|Tx\| = \|x\|$$
	\end{ex}
	
	\begin{proof}\
		\begin{itemize}
			\item $(\implies):$ \\
			Suppose that $T$ is an isometry. Then for each $x \in X$, 
			\begin{align*}
				\|Tx\|
				& = d_Y(Tx, 0) \\
				& = d_Y(Tx, T0) \\
				& = d_X(x, 0) \\
				& = \|x\|.
			\end{align*}
			\item $(\impliedby):$ \\
			Suppose that for each $x \in X$, $\|Tx\| = \|x\|$. Then for each $x_1, x_2 \in X$,
			\begin{align*}
				d_Y(Tx_1, Tx_2) 
				& = \|Tx_1 - Tx_2\| \\
				& = \|T(x_1 - x_2)\| \\
				& = \|x_1 - x_2\| \\
				& = d_X(x_1, x_2).
			\end{align*}
			Thus $T$ is an isometry.
		\end{itemize}
	\end{proof}

	
	\begin{ex} \lex{42002}
	Set $X = C^{1}([0,1])$ and $Y = C([0,1])$. Equip both $X$ and $Y$ with the sup norm. Define $T:X \rightarrow Y$ by $Tf = f'$. Then $T$ is not bounded.
	\end{ex}
	
	\begin{proof}
	For the sake of contradiction, suppose that $T$ is bounded. Then there exists $C \geq 0$ such that for each $f \in X$, $\|Tf\| \leq C \|f\|$. Choose $n \in \N$ such that $n > C$. Define $f \in X$ by $f(x) = x^n$. Then
	\begin{align*}
	n
	&= \|Tf\| \\
	&\leq C \|f\| \\
	&= C
\end{align*}		
	which is a contradiction. Hence $T$ is not bounded.
	\end{proof}
	
	\begin{ex} \lex{42003}
		Let $X,Y$ be a normed vector spaces and $T:X \rightarrow Y$ a linear map. Then $T$ is bounded iff there exists $r,s>0$ such that $T(B(0,r)) \subset B(0,s)$
	\end{ex}
	
	\begin{proof}
		Suppose that $T$ is bounded. Then there exists $C \geq 0$ such that for each $x \in X$, $\|Tx \|\leq C \|x \|$. Thus $T(B(0,1)) \subset B(0,C+1)$. Conversely. Suppose that there exists $r,s >0$ such that $T(B(0,r)) \subset B(0,s)$. Define $C = \frac{2s}{r}$. Let $x \in X$. Put $\al = \frac{r}{2\|x \|}$ Then $\al x \in B(0,r)$. So $T(\al x ) = \al T(x) \in B(0,s)$. Hence 
		\begin{align*}
			\|T(\al x) \|
			&= \|\al T(x) \|\\
			&= \vert \al \vert \|T(x) \|\\
			& = \frac{r}{2 \|x \|}  \|T(x) \|\\
			& < s.
		\end{align*}
		Thus $$\|Tx \|< \frac{2 s}{r} \|x \|= C \|x \|$$ So $T$ is bounded. 
	\end{proof}
	
	\begin{ex} \lex{42003.1}
	Let $X, Y$ be normed vector spaces and $T:X \rightarrow Y$. Suppose that $T$ is linear. Then there exists $x_0 \in X$ such that $T$ is continuous at $x_0$ iff $T$ is continuous at 0.
	\end{ex}
	
	\begin{proof}
	Suppose that there exists $x_0 \in X$ such that $T$ is continuous at $x_0$. Since $T$ is linear, $T(0) = 0$. Let $(x_n)_{n \in \N} \subset X$. Suppose that $x_n \rightarrow 0$. Then $x_n + x_0 \rightarrow x_0$. Hence 
	\begin{align*}
	T(x_n) + T(x_0)
	&= T(x_n + x_0) \\
	& \rightarrow T(x_0)
	\end{align*}	  
	This implies that 
	\begin{align*}
	T(x_n) 
	&\rightarrow 0 \\
	& = T(0)
	\end{align*}	 
	Therefore $T$ is continuous at $0$. \\
	Conversely, if $T$ is continuous at $0$, then trivially, there exists $x_0 \in X$ such that $T$ is continuous at $x_0$.
	\end{proof}
	
	\begin{ex} \lex{42004}
		Let $X,Y$ be normed vector spaces and $T:X \rightarrow Y$ a linear map. Then the following are equivalent:
		\begin{enumerate}
			\item $T$ is continuous
			\item $T$ is continuous at $x=0$
			\item $T$ is bounded
		\end{enumerate}
	\end{ex}
	
	\begin{proof}\
		\begin{itemize}
		\item $(1) \implies (2)$:\\
		Trivial
		\item $(2) \implies (3)$:\\
		Suppose that $T$ is continuous at $x=0$. Then there exists $\del>0$ such that for each $x \in X$, if $\|x \|< \del$, then $\|Tx \|< 1$. Choose $C = \frac{2}{\del}$. If $x=0$, then $\|Tx \|\leq C \|x \|$. Suppose that $\|x \|\neq 0$. Define $y = \frac{\del}{2 \|x \|}x$. Then $\|y \|< \del$. So 
		\begin{align*}
		1 
		&> \|Ty \|\\
		&= \frac{\del}{2 \|x \|} \|Tx \|
		\end{align*}
		Thus 
		\begin{align*}
			\|Tx \|
			&< \frac{2}{\del} \|x \| \\
			&=C \|x \|
		\end{align*}
		
		Hence $T$ is bounded.
		\item $(3) \implies (1)$\\
		Suppose that $T$ is bounded. Then there exists $C \geq 0$ such that for each $x \in X$, $\|Tx \|\leq C\|x \|$. Let $\ep >0$. Choose $\del = \frac{\ep}{C+1}$. Let $x,y \in X$ Suppose that $\|x-y \|< \del$. Then 
		\begin{align*}
			\|Tx-Ty \|
			& = \|T(x-y) \| \\
			& \leq C \|x-y \|\\
			&< (C+1) \del\\ 
			&= \ep
		\end{align*}
		
		So $T$ is continuous.
		\end{itemize}
	\end{proof}
	
	\begin{defn} \ld{42005}
		Let $X,Y$ be normed vector spaces. Define $\|\cdot\|: L(X; Y)\rightarrow \Rg$ by $$\|T\| = \inf \{C \geq 0: \text{for each }x \in X\text{, } \|Tx \|\leq C\|x\|\}$$ We call $\|\cdot \|$ the \tbf{operator norm on $L(X; Y)$}
	\end{defn}
	
	\begin{ex} \lex{42006}
		Let $X,Y$ be normed vector spaces. If $X\neq \{0\}$, then the operater norm on $L(X; Y)$ is given by: 
		\begin{enumerate}
			\item $\|T\| = \sup\limits_{\|x\|=1}\|Tx\|$
			\item $\|T\| = \sup\limits_{x \neq 0}\|x\|^{-1} \|Tx\|$
			\item $\|T\| = \inf \{C \geq 0: \text{for each }x \in X\text{, } \|Tx \|\leq C\|x\|\}$
		\end{enumerate}
	\end{ex}
	
	\begin{proof} Since $X \neq \{0\}$, the supremums in (1) and (2) are well defined. Let $T \in L(X; Y)$. By linearity of $T$, the sets over which the supremums are taken in (1) and (2) are the same. So (1) and (2) are equal.\\
		Now, set $M = \sup\limits_{\|x \|=1} \|Tx \|$ and $m = \inf \{C \geq 0: \text{ for each }x \in X\text{, } \|Tx \|\leq C \|x \|\}$. Let $x \in X$. If $\|x \|=0$, then $\|Tx \|\leq M \|x \|$. Suppose that $\|x \|\neq 0$. Then 
		\begin{align*}
			\|Tx \|
			&= \bigg(\big\|T(|x\|^{-1}x)\big\|\bigg)\|x \|\\
			& \leq M \|x\|
		\end{align*}
		Hence $M \in \{C \geq 0: \text{ for each }x \in X\text{, } \|Tx \|\leq C \|x \|\}$ and $m \leq M$.
		Let $C \in \{C \geq 0: \text{ for each }x \in X\text{, } \|Tx \|\leq C \|x\|\}$. Suppose that $\|x \|=1$. Then $\Vert Tx\Vert \leq C \|x \|= C$. So $M \leq C$. Therefore $M \leq m$. So $M=m$ and the supremum in (1) is the same as the infimum in (3). 
	\end{proof}
	
	\begin{note}
		From here on, unless stated otherwise, we assume $X \neq 0$.
	\end{note}
	
	\begin{ex} \lex{42007}
		Let $X,Y$ be normed vector spaces and $T \in L(X; Y)$. Then for each $x \in X$, $\|Tx \| \leq \|T\|\|x \|$
	\end{ex}
	
	\begin{proof}
		Let $x \in X$. If $x = 0$, then $\|Tx \|\leq \|T \|\|x \|$. Suppose that $x \neq 0$. The previous exercise implies that 
		\begin{align*}
			\|Tx \|
			& = \| T(|x\|^{-1} x)\| \|x\| \\
			& \leq \bigg( \sup\limits_{\|x\|=1}\|Tx\| \bigg) \|x\| \\
			& =  \|T\| \|x\|
		\end{align*}
	\end{proof}
	
	\begin{ex} \lex{42008}
		Let $X, Y$ be normed vector spaces. Then $L(X; Y)$ is a vector space and the operator norm is a norm on $L(X; Y)$.
	\end{ex}
	
	\begin{proof}
		Let $S,T \in L(X;Y)$ and $\al \in \C$. 
		\begin{itemize}
			\item It is clear that $S+T$ is linear. For each $x \in X$,
			\begin{align*}
				\|(S+T)x \|
				&= \|Sx+Tx \|\\
				& \leq \|Sx \|+ \|Tx \|\\
				&\leq \|S \|\|x \|+ \|T \|\|x \|\\
				&= \big(\|S \|+ \|T \|\big) \|x \|
			\end{align*}
			So $S+T \in L(X;Y)$ and $\|S+T\|\leq \|S \|+ \|T \|$
			\item It is clear that $\al T$ is linear. For each $x \in X$,
			\begin{align*}
				\|\al T \|
				&= \sup_{\|x \|=1} \|(\al T)x \|\\
				&= \sup_{\|x \|=1} \vert \al \vert \|Tx \|\\
				&=\vert \al \vert \sup_{\|x \|=1} \|Tx \|\\
				&=\vert \al \vert \|T \|
			\end{align*} 
			So $\al T \in L(X;Y)$ and $\|\al T\| \leq |\al| \|T\|$.
			\item Suppose that $\|T\| = 0$.  Let $x \in X$. Then  
			\begin{align*}
				\|T x\|
				& \leq \|T \|\|x \| \\
				& = 0
			\end{align*}
			So $Tx=0$. Since $x \in X$ is arbitrary, we have that $T=0$. 
		\end{itemize}
		Therefore $L(X;Y)$ is a vector space and $\|\cdot\|: L(X;Y) \rightarrow \Rg$ is a norm.
	\end{proof}
	
	\begin{ex} \lex{42009}
		Let $X,Y,Z$ be normed vector spaces, $T \in L(X; Y)$ and $S \in L(Y,Z)$. Define $ST:X \rightarrow Z$ by $STx = S(Tx)$. Then $ST \in L(X,Z)$ and $\|ST \|\leq \|S \|\|T \|$. 
	\end{ex}
	
	\begin{proof}
		Clearly $ST$ is linear. Let $x \in X$. Then 
		\begin{align*}
			\|ST x \|
			& = \|S(Tx) \|\\
			& \leq \|S \|\|Tx \|\\
			& \leq \|S \|\|T \|\|x \|
		\end{align*}
		
		So $\|ST \|\leq \|S \|\|T \|$.
	\end{proof}
	
	\begin{defn} \ld{42010}
		Let $X,Y$ be a normed vector spaces and $T \in L(X; Y)$. Then $T$ is said to be \tbf{invertible} or an \tbf{isomorphism} if $T$ is a bijection and $T^{-1} \in L(Y,X)$.
	\end{defn}
	
	\begin{defn} \ld{42011}
		Let $X$ be a normed vector space. Define $GL(X) := \{T \in L(X,X): T \text{ is invertible}\}$.
	\end{defn}
	
	\begin{ex} \lex{42013}
		Let $X,Y$ be normed vector spaces. If $Y$ is complete, then $L(X; Y)$ is complete.
	\end{ex}
	
	\begin{proof}
		Suppose that $Y$ is complete. Let $(T_n)_{n \in \N} \subset L(X; Y)$. Suppose that $(T_n)_{n \in \N}$ is Cauchy.
		\begin{itemize}
			\item Since for each $m,n \in \N$, $\big\vert \|T_m \|- \|T_n \|\big\vert \leq \|T_m -T_n \|$, we have that $(\|T_n \|)_{n \in \N} \subset \Rg$ is Cauchy. Hence $\lim\limits_{n \rightarrow \infty}\|T_n \|$ exists. \vspace{1cm} \\ Let $x \in X$ and $m,n \in \N$. Then 
			\begin{align*}
				\|T_m x - T_n x \|
				&= \|(T_m-T_n) x \|\\
				&\leq \|T_m-T_n \|\|x \|
			\end{align*}
			So $(T_nx)_{n \in \N} \subset Y$ is Cauchy and hence converges. Define $T:X \rightarrow Y$ by $Tx = \lim\limits_{n \rightarrow \infty} T_nx$. Since addition and scalar multiplication are continuous, for each $x_1, x_2 \in X$ and $\lam \in \K$, 
			\begin{align*}
				T(x_1+\lam x_2) 
				& = \lim_{n \rightarrow \infty} T_n(x_1 + \lam x_2) \\
				& =  \lim_{n \rightarrow \infty} [T_n(x_1) + \lam T_n(x_2) ]\\
				& = \lim_{n \rightarrow \infty} T_n(x_1) + \lam \lim_{n \rightarrow \infty} T_n(x_2) \\
				& = T(x_1) + \lam T(x_2).
			\end{align*}
			Hence $T$ is linear. Let $x \in X$ and $\ep>0$. Choose $N \in \N$ such that for each $n \in N$, if $n \geq N$, then $\|Tx - T_n x\|< \ep$. Then for each $n \in \N$, if $n \geq N$ we have that 
			\begin{align*}
				\|Tx\|
				&\leq \|Tx-T_nx \|+ \|T_nx \|\\
				&< \ep + \|T_nx \|\\
				&\leq \ep + \|T_n \|\|x \|
			\end{align*}  
			Thus $\|Tx \|\leq \ep +(\lim\limits_{n \rightarrow \infty} \|T_n \|) \|x \|$. Since $\ep >0$ is arbitrary, $\|Tx \|\leq (\lim\limits_{n \rightarrow \infty} \|T_n \|) \|x \|$. Since $x \in X$ is arbitrary, $T \in L(X; Y)$ and $\|T \|\leq \limn \|T_n \|$.
			\item Let $\ep >0 $. Since $(T_n)_{n \in \N}$ is Cauchy, there exists $N \in \N$ such that for each $m, n \in \N$, $n,m \geq N$ implies that $\|T_n - T_m\| < \ep$. Let $n \in \N$. Suppose that $n \geq N$. Since addition $A:X \times X \rightarrow X$, scalar multiplication $M:\K \times X \rightarrow X$ and the norm $\|\cdot\|:X \rightarrow \Rg$ are continuous, we have that for each $x \in X$, 
			\begin{align*}
				\|(T_n - T)(x)\|
				& \|T_n(x) - T(x)\| \\
				& \|T_n(x) - \lim_{m \rightarrow \infty} T_m(x)\| \\
				& = \lim_{m \rightarrow \infty} \|T_n(x) - T_m(x)\| \\
				& = \liminf_{m \rightarrow \infty} \|T_n(x) - T_m(x)\| \\
				& \leq \liminf_{m \rightarrow \infty} \|T_n - T_m\| \|x\| \\
				& \leq \ep \|x\|.
			\end{align*}
			Hence $\|T_n -T\| \leq \ep$. Since $n \in \N$ with $n \geq N$ is arbitrary, we have that for each $n \in \N$, $n \geq N$ implies that $\|T_n -T\| \leq \ep$. Therefore $T_n \rightarrow T$. 
		\end{itemize}
	 Since $(T_n)_{n \in \N} \subset L(X; Y)$ with $(T_n)_{n \in \N}$ Cauchy is arbitrary, we have that for each $(T_n)_{n \in \N} \subset L(X; Y)$, $(T_n)_{n \in \N}$ is Cauchy implies that there exists $T \in L(X;Y)$ such that $T_n \rightarrow T$. Hence $L(X;Y)$ is complete.
	\end{proof}
	
	
	
	
	
	
	
	
	
	
	
	
	
	
	
	
	
	
	
	
	
	
	
	
	
	
	
	
	
	
	
	\newpage
	\section{Direct Sums}
	
	\begin{defn} \ld{}
	Let $X, Y$ be normed vector spaces and $p \in [1, \infty]$. Let $\| \cdot \|'_p: \R^2 \rightarrow [0, \infty)$ denote the usual $l^p$ norm. We define $\| \cdot \|_p : X \oplus Y \rightarrow \Rg$ by $$\|(x, y) \|_p = \|( \|x\|, \| y \|) \|'_ p$$
	\end{defn}
	
	\begin{ex} \lex{}	
	Let $X, Y$  be normed vector spaces. Then 
	\begin{enumerate}
	\item for each $p \in [1, \infty]$, $\|\cdot\|_p: X \oplus Y \rightarrow \Rg$ is a norm on $X \oplus Y$
	\item  $\{\|\cdot \|_p:  p \in [1, \infty]\}$ are equivalent. 
	\end{enumerate}
	\end{ex}
	
	\begin{proof}\
	\begin{enumerate}
	\item Let $p \in [1, \infty]$, $(x_1,y_1), (x_2,y_2) \in X \oplus Y$ and $\lam \in \C$.
	\begin{itemize}
	\item Clearly if $(x_1, y_1) = (0,0)$, then $\|S\|_p = 0$. Conversely, suppose that $\|(x_1, y_1)\|_p = 0$. Then $\|x_1\| = 0$ and $\|y_1\| = 0$. So $x_1 = 0$ and $y_1 = 0$. Therefore $S = 0$. 
	\item 
	\begin{align*}
	\|\lam (x_1, y_1)\|_p
	&= \|(\|\lam x_1\|, \|\lam y_1\|)\|_p' \\
	&= \|(|\lam|\| x_1\|, |\lam|\|y_1 \|)\|_p' \\
	&= \||\lam| (\| x_1\|,\|y_1\| )\|_p' \\
	&= |\lam| \|(\| x_1\|,\|y_1\|)\|_p' \\
	&= |\lam| \| (x_1, y_1)\|_p
	\end{align*}
	\item 
	\begin{align*}
	\|(x_1, y_1) + (x_2, y_2)\|_p
	&= \|(\|x_1 + x_2\|, \|y_1 + y_2\|)\|_p' \\
	&\leq \|(\|x_1\| + \|x_2\|, \|y_1\| + \|y_2\|)\|_p' \\
	&= \|(\|x_1\|, \|y_1\|) + (\|x_2\|, \|y_2\|)\|_p' \\
	&\leq \|(\|x_1\|, \|y_1\|)\|_p' + \|(\|x_2\|, \|y_2\|)\|_p' \\
	&= \|(x_1, y_1)\|_p + \|(x_2, y_2)\|_p \\ 
	\end{align*}
	\end{itemize}
	\item All norms on $\R^2$ are equivalent.
	\end{enumerate}
\end{proof}		

\begin{ex} \lex{}	
	Let $X, Y$ be Banach spaces. Then $X \oplus Y$ equipped with $\|\cdot \|_p:X \oplus Y \rightarrow [0, \infty)$ is a Banach space. 
	\end{ex}
	
	\begin{proof}
	
	\end{proof}
	
	\begin{ex}
	Let $X, Y$ and $Z$ be Banach spaces and $p \in [0, \infty]$. Equip $Y \oplus Z$ with $\|\cdot\|_p$. Let $T \in L(X, Y \oplus Z)$ with $T = (T_Y, T_Z)$. Then $T_Y \in L(X; Y)$ and $T_Z \in L(X, Z)$.
	\end{ex}
	
	\begin{proof}
	Let $x \in X$. Then $\|T_Y(x)\|, \|T_Z(x)\| \leq $
	\\ \tbf{FINISH!!!}
	\end{proof}

	\begin{defn}
	Let $X, Y$ and $Z$ be Banach spaces and $p \in [0, \infty]$. Let $\| \cdot \|'_p: \R^2 \rightarrow [0, \infty)$ denote the usual $l^p$ norm. Equip $Y \oplus Z$ with $\|\cdot\|_p$. Let $T \in L(X, Y \oplus Z)$ with $T = (T_Y, T_Z)$. Define $\|\cdot \|_p: L(X, Y \oplus Z) \rightarrow \Rg$ by $$\|T\|_p = \|(\|T_Y\|, \|T_Z\|)\|_p'$$
	\end{defn}
	
	\begin{ex}
	Let $X, Y$ and $Z$ be Banach spaces and $p \in [0, \infty]$. Then $\|\cdot \|_p: L(X, Y \oplus Z) \rightarrow \Rg$ is a norm on $L(X, Y \oplus Z)$. 
	\end{ex}
	
	\begin{proof}
	Let $\lam \in \C$ and $S, T \in L(X, Y \oplus Z)$ with $S= (S_Y, S_Z)$ and $T = (T_Y, T_Z)$. 
	\begin{itemize}
	\item Clearly if $S = 0$, then $\|S\|_p = 0$. Conversely, suppose that $\|S\|_p = 0$. Then $\|S_Y\| = 0$ and $\|S_Z\| = 0$. So $S_Y = 0$ and $S_Z = 0$. Therefore $S = 0$. 
	\item 
	\begin{align*}
	\|\lam S\|_p
	&= \|(\|\lam S_Y\|, \|\lam S_Z\|)\|_p' \\
	&= \|(|\lam|\| S_Y\|, |\lam|\|S_Z \|)\|_p' \\
	&= \||\lam| (\| S_Y\|,\|S_Z\| )\|_p' \\
	&= |\lam| \|(\| S_Y\|,\|S_Z\|)\|_p' \\
	&= |\lam| \| S\|_p
	\end{align*}
	\item 
	\begin{align*}
	\|S + T\|_p
	&= \|(\|S_Y + T_Y\|, \|S_Z + T_Z\|)\|_p' \\
	&\leq \|(\|S_Y\| + \|T_Y\|, \|S_Z\| + \|T_Z\|)\|_p' \\
	&= \|(\|S_Y\|, \|S_Z\|) + (\|T_Y\|, \|T_Z\|)\|_p' \\
	&\leq \|(\|S_Y\|, \|S_Z\|)\|_p' + \|(\|T_Y\|, \|T_Y\|)\|_p' \\
	&= \|S\|_p + \|T\|_p \\ 
	\end{align*}
	\end{itemize}
	So $\|\cdot \|_p: L(X, Y \oplus Z) \rightarrow \Rg$ is a norm on $L(X, Y \oplus Z)$. 
	\end{proof}
	
	
	\begin{ex}
	Let $X, Y$ and $Z$ be Banach spaces and $p \in [0, \infty]$. Equip $Y \oplus Z$ with $\|\cdot\|_p$. Let $T \in L(X, Y \oplus Z)$ with $T = (T_Y, T_Z)$. Then $\|T\|\leq 2^{1/p}\|T\|_p$.
	\end{ex}
	
	\begin{proof}
	Let $x \in X$. If $p < \infty$, then
	\begin{align*}
	\|T(x)\|_p
	&= \|(T_Y(x), T_Z(x))\|_p \\
	& \|( \|T_Y(x)\|, \|T_Z(x)\|)\|_p' \\
	&=  \bigg(\|T_Y(x) \|^p +  \|T_Z(x) \|^p \bigg)^{1/p} \\
	& \leq \bigg(\|T_Y\|^p\|x\|^p +  \|T_Z\|^p\|x\|^p \bigg)^{1/p} \\
	& \leq \bigg[ (\|T_Y\|^p+ \|T_Z\|^p)\|x\|^p +  (\|T_Y\|^p + \|T_Z\|^p)\|x\|^p \bigg ]^{1/p} \\
	&= \bigg[ 2(\|T_Y\|^p + \|T_Z\|^p)\|x\|^p \bigg ]^{1/p} \\
	&= 2^{1/p}\|T\|_p\|x\| \\
	\end{align*}
	Hence $\|T\| \leq 2^{1/p}\|T\|_p$
	If $p = \infty$, then 
	\begin{align*}
	\|T(x)\|_{\infty} 
	&= \max(\|T_Y(x) \|, \|T_Z(x)\|) \\
	& \leq \max(\|T_Y\|\|x \|, \|T_Z\|\|x\|) \\
	& \leq \max \bigg[ \max (\|T_Y\|, \|T_Z\|)\|x \|, \max(\|T_Y\| ,\|T_Z\|)\|x\| \bigg] \\
	&= \max(\|T_Y\| ,\|T_Z\|) \|x\|\\
	&= \|T\|_{\infty} \|x\|
	\end{align*}
	Hence 
	\begin{align*}
	\|T\| 
	& \leq \|T\|_{\infty} \\
	&= 2^{1/\infty}\|T\|_{\infty}
	\end{align*}
	\end{proof}
	
	\begin{ex}
	Let $X$ and $X_1, \cdots, X_n$ be Banach spaces and $p \in [0, \infty]$. Equip $\bigoplus\limits_{j=1}^n X_j$ with $\|\cdot\|_p$. Let $T \in L(X, \bigoplus\limits_{j=1}^n X_j)$. Then $\|T\|\leq n^{1/p}\|T\|_p$.
	\end{ex}
	
	\begin{proof}
	Similar to the previous exercise.
	\end{proof}
	
	
	
	
	
	
	
	
	
	
	
	
	
	
	
	
	
	
	
	
	
	
	\newpage
	\section{Quotient Spaces}	
	
	\begin{defn} \ld{def:banach_space:quotient_space:0001}
		Let $X$ be a normed vector space and $M \subset X$ a closed subspace. Define $\|\cdot\|:X/M \rightarrow \Rg$ by $$\|x+M\|_{X/M} := \inf_{y \in M}\|x+y\|$$
		
		We call $\|\cdot\|_{X/M}$ the \tbf{subspace norm on $X/M$}
	\end{defn}

	\begin{note}
		When the context is clear, we write $\|\cdot\|$ in place of $\| \cdot \|_{X/M}$.
	\end{note}
	
	\begin{ex} \lex{ex:banach_space:quotient_space:0002}
		Let $X$ be a normed vector space and $M \subsetneq X$ a proper, closed subspace of $M$. 
		Then 
		\begin{enumerate}
			\item The previously defined subspace norm on $X/M$ is well defined and is a norm. 
			\item For each $\ep > 0$, there exists $x \in X$ such that $\|x\|=1$ and $\|x+M\| \geq 1-\ep$.
			\item The projection map $\pi:X \rightarrow X/M$ defined by $\pi(x) = x+M$ is continuous and $\|\pi\|=1$. 
			\item If $X$ is complete, then $X/M$ is complete. 
		\end{enumerate} 
	\end{ex}
	
	\begin{proof}\
		\begin{enumerate}
			\item  Let $x, y \in X$ and $\al \in \C$. Suppose that $x+M =y+M$. Then there exists $m \in M$ such that $x=y+m$. Since $M$ is a subspace, the map $T:M \rightarrow M$ given by $Tx = x+m$ is a bijection. So $$\inf_{z \in M} \|y+m+z \|= \inf_{z \in M} \|y+z \|$$ which implies that 
			\begin{align*}
				\|x +M \|
				&= \inf_{z \in M} \|x+z \|\\
				&= \inf_{z \in M} \|y+m+z \|\\
				&= \inf_{z \in M} \|y+z \|\\
				&= \|y+M \|
			\end{align*} 
			So $\|\cdot \|: X/M \rightarrow \Rg$ is well defined.\vspace{.5cm}\\
			We observe that for each $z,w \in M$, $$\|x+y+z \|\leq \|x+w \|+ \|y+w+z \|$$
			Taking infimums over $M$ with respect to $z$ in this inequality implies that for each $w \in M$,
			\begin{align*}
				\inf_{z \in M}\|x+y+z \|
				&\leq \inf_{z \in M} \bigg( \|x+w \|+ \|y+w+z \|\bigg) \\
				&= \|x+w \|+\inf_{z \in M}\|y+w+z \|
			\end{align*}
			Again we use the fact that for each $w \in M$, $$\inf_{z \in M}\|y+w+z \|= \inf_{z \in M}\|y+z \|$$
			This implies that for each $w \in M$, $$\inf_{z \in M}\|x+y+z \|\leq \|x+w \|+ \inf_{z \in M}\|y+z \|$$
			
			Therefore, taking infimums over $M$ with respect to $w$ in this inequality yields
			\begin{align*}
				\|x+y+M \|
				&= \inf_{z \in M} \|x+y +z \|\\
				& \leq \inf_{w \in M} \bigg(\|x+w \|+ \inf_{z \in M}\|y+z \|\bigg)\\
				&= \inf_{w \in M} \|x+w \|+ \inf_{z \in M}\|y+z \|\\
				&= \|x+M \|+ \|y+M \|
			\end{align*}
			\vspace{.5cm}\\
			If $\al =0$, then $\al x = 0$. Choosing $z = 0 \in M$ gives $\|\al x+M \|=0 = \vert \al \vert \|x+M \|$. Suppose that $\al \neq 0$. Then the map $T:M \rightarrow M$ given by $Tx = \al ^{-1}x$ is a bijection and thus $\inf\limits_{z \in M} \|x+\al^{-1}z \|= \inf\limits_{z \in M} \|x+z \|$. Hence we have that
			\begin{align*}
				\|\al x+M \|
				&= \inf_{z \in M} \|\al x +z \|\\
				&= \inf_{z \in M} \vert \al \vert \|x +\al^{-1}z \|\\
				&= \vert \al \vert \inf_{z \in M}\|x +\al^{-1}z \|\\
				&= \vert \al \vert \inf_{z \in M}\|x +z \|\\
				&= \vert \al \vert \|x+M \|
			\end{align*} 
			
			Suppose that $\|x \|=0$. Choose a sequence $(z_n)_{n \in N} \subset M$ such that 
			\begin{align*}
				\lim\limits_{n \rightarrow \infty} \|x - z_n \|
				& = \inf_{z \in M} \|x+ z \|\\
				& = 0
			\end{align*} 
			Then $\limn z_n =x$. Since $M$ is closed, $x \in M$. Hence $x+M=0+M$.
			\item Since $M$ is a proper subspace, there exists $v \in X$ such that $v \not \in M$. Then $\|v +M \|\neq 0$. Let $\ep >0$. Then 
			\begin{align*}
				(1-\ep)^{-1}\|v+M \|
				& > \|v+M \| \\
				& = \inf_{y \in M} \|x + y\|
			\end{align*}
			So there exists $z \in M$ such that $\|v+z \| < (1-\ep)^{-1} \|v+M \|$. Since $v+M \neq 0 + M$, we have that $v+z \neq 0$. Choose $x = \|v+z \|^{-1}(v+z)$. Then $\|x \|=1$ and 
			\begin{align*}
				\|x+M \|
				&= \|v+z \|^{-1} \|v+z +M \|\\
				&= \|v+z \|^{-1} \|v +M \|\\
				&> 1-\ep
			\end{align*}
			\item Let $x \in X$. Taking $z=0$, we we see that $\|\pi(x) \|=\|x+M \|\leq \|x+z \|= \|x \|$. So $\pi$ is bounded and in particular, $$\sup_{\|x \|=1} \|\pi(x) \|\leq 1$$ 
			From (2) we see that $$\sup_{\|x \|=1} \|\pi(x) \|\geq 1$$
			Hence $\|\pi\|= 1$. \vspace{.5cm}\\
			\item Suppose that $X$ is complete. Let $(x_i+M)_{i\in \N} \subset X/M$. Suppose that $\sum\limits_{i\in \N} \|x_i+M \|< \infty$. Let $\ep>0$. Then for each $i \in \N$, there exists $z_i \in M$ such that $\|x_i +z_i \|< \|x_i +M \|+ \ep2^{-i}$. Define the sequence $(a_i)_{i\in \N} \subset X$ by $a_i = x_i +z_i$. Then we have 
			\begin{align*}
				\sum_{i\in \N} \|a_i \|
				&= \sum_{i \in N} \|x_i + z_i \|\\
				&\leq \sum_{i \in N} \bigg (\|x_i +M \|+ \ep2^{-i} \bigg)\\
				&= \sum_{i\in \N} \|x_i+M\|+ \ep
			\end{align*}
			Since $\ep>0$ is arbitrary, it follows that $$\sum_{i\in \N} \|a_i \|\leq \sum_{i\in \N} \|x_i+M\|< \infty$$
			Since $X$ is complete, $\sum\limits_{i=1}^{\infty}a_i$ converges in $X$. Define $(s_n)_{n \in \N} \subset X$ and $s \in X$ by $s_n = \sum\limits_{i =1}^n a_i$ and $s = \sum\limits_{i=1}^\infty a_i $. Since $\limn s_n = s$, and $\pi: X \rightarrow X/M$ is continuous, it follows that $\limn \pi(s_n) = \pi(s)$. Since 
			\begin{align*}
				\pi(s_n) 
				&= \sum_{i=1}^n a_i +M\\
				&= \sum_{i=1}^n x_i +M
			\end{align*} 
			We have that $\sum\limits_{i=1}^{\infty}x_i +M$ converges which implies that $X/M$ is complete.
		\end{enumerate}
	\end{proof}
	
	\begin{ex} \lex{}
		Let $X,Y$ be normed vector spaces and $T \in L(X; Y)$. Then
		\begin{enumerate}
			\item $\ker T$ is closed
			\item there exists a unique map $S :X/ \ker T \rightarrow T(X)$ such that $T = S \circ \pi$. Furthermore $S$ is a bounded linear bijection and $\|S \|= \|T \|$.
		\end{enumerate}
	\end{ex}
	
	\begin{proof}\
		\begin{enumerate}
			\item Since $T$ is continuous and $\ker T = T^{-1}(\{0\})$, we have that $\ker T$ is closed.
			\item Suppose that there exists $S_1,S_2 \in L(X/ \ker T, T(X)) $ such that $T = S_1 \circ \pi$ and  $T = S_2 \circ \pi $. Let $x \in X$. Then $$S_1(x + \ker T) = S_1(\pi(x)) = T(x) = S_2(\pi(x)) = S_2(x + \ker T)$$ So $S_1 = S_2$. Therefore such a map is unique.\\
			Define $S: X / \ker T \rightarrow T(X)$ by $S(x+ \ker T) = T(x)$. Then $S$ is clearly a linear bijection that satisfies $T = S \circ \pi$. Let $x \in X$ and $z \in \ker T$. Then 
			\begin{align*}
				\|S(x+ \ker T) \|
				& = \|T(x) \|\\
				& = \|T(x+z) \|\\
				& \leq \|T \|\|x+ z \|
			\end{align*} 
			Thus $$\|S(x+ \ker T) \|\leq \|T \|\inf_{z \in \ker T}  \|x + z \|= \|T \|\|x + \ker T \|$$
			So $S$ is bounded and $\|S \|\leq \|T \|$. This implies that $$\|T \|= \|S \circ \pi \|\leq \|S \|\|\pi \|= \|S \|$$
			Thus $\|S \|= \|T \|$.
		\end{enumerate}
	\end{proof}
	
	\begin{ex} \lex{}
		Let $X, Y$ be normed vector spaces. Define $\phi: L(X; Y) \times X \rightarrow Y$ by \\$\phi(T,x) = Tx$. Then $\phi$ is continuous.
	\end{ex}
	
	\begin{proof}
		Let $(T_1, x_1) \in L(X; Y) \times X$ and $\ep > 0$. Choose $\del = \min \{\frac{\ep}{2(\|x_1 \|+ \|T_1 \|+1)}, \frac{\sqrt{\ep}}{\sqrt{2}} \}$. Let $(t_2, x_2) \in L(X; Y) \times X$. Suppose that $$\|(T_1, x_1) - (T_2, x_2) \|= \max \{\|T_1 - T_2\|, \|x_1 -x_2 \|\} < \del$$ Then 
		\begin{align*}
			\|\phi(T_1, x_1) - \phi(T_2-x_2) \|
			&= \|T_1 x_ - T_2 x_2 \|\\
			&= \|T_1 x_1 - T_2 x_1 + T_2 x_1 - T_2 x_2 \|\\
			& \leq \|(T_1 - T_2) x_1 \|+ \|T_2(x_1 -x_2) \|\\
			& \leq \|T_1 -T_2 \|\|x_1 \|+ \|T_2 \|\|x_1 -x_2 \|\\
			& \leq \|T_1 -T_2 \|\|x_1 \|+ \big(\|T_1 - T_2 \|+ \|T_1 \|\big)\|x_1 -x_2 \|\\
			& < \del \|x_1 \|+ (\del + \|T_1 \|) \del \\
			&= \del (\|T_1 \|+ \|x_1 \|) + \del^2\\
			& < \frac{\ep}{2} + \frac{\ep}{2}\\
			&= \ep
		\end{align*}
		So $\phi$ is continuous.
	\end{proof}
	
	\begin{ex} \lex{}
		Let $X$ be a normed vector space and $M \subset X$ a subspace. Then $\ol{M}$ is a subspace.
	\end{ex}
	
	\begin{proof}
		Let $x,y \in \ol{M}$ and $\al \in \C$. Then there exist sequences $(x_n)_{n \in \N} \subset M$ and $(y_n)_{n \in \N} \subset M$ such that $x_n \conv{} x$ and $y_n \conv{} y$. Since $M$ is a subspace, $(x_n +y_n)_{n \in \N} \subset M$ and $(\al x_n)_{n \in \N} \subset M$. Since addition and scalar multiplication are continuous, we have that $x_n + y_n \conv{} x+y$ and $\al x_n \conv{} \al x$. Thus $x+y \in \ol{M}$ and $\al x \in \ol{M}$ and hence $\ol{M}$ is a subspace.
	\end{proof}
	
	\newpage
	
	
	
	
	
	
	
	



	
	
	
	
	
	
	
	
	
	
	
	
	

	

	\newpage
	\section{Applications of the Hahn-Banach Theorem}
	
	
	
	
	
	
	\begin{defn} \ld{55002}\
	Let $X$ be a normed vector space over $\C$, and $T :X \rightarrow \C$. Then $T$ is said to be a \tbf{bounded linear functional on} $X$ if $T \in L(X, \C)$. We define the \tbf{dual space of} $X$, denoted $X^*$, by $X^* = L(X, \C)$.
	\end{defn}
	
	\begin{note}
	We define $X^*$ similarly when $X$ is a normed vector space over $\R$.
	\end{note}

	\begin{defn} \ld{55009}
		Let $X$ be a normed vector space and $p:X \rightarrow \R$ a sublinear functional. Then $p$ is said to be \tbf{bounded} if there exists $M >0$ such that for each $x \in X$, $p(x) \leq M\|x\|$. 
	\end{defn}
	
	\begin{ex} \lex{55010}
		Let $X$ be a normed vector space and $p:X \rightarrow \R$ a sublinear functional. Then $p$ is bounded iff $p$ is Lipschitz. 
	\end{ex}
	
	\begin{proof}
		Suppose that $p$ is bounded. Then there exists $M >0$ such that for each $x \in X$, $p(x) \leq M\|x\|$. Let $x, y \in X$. Then the previous exercise implies that 
		\begin{align*}
			-M\|x-y\| 
			&= -M\|y-x\| \\
			& \leq -p(y-x) \\
			& \leq p(x)-p(y) \\
			& \leq p(x-y) \\
			& \leq M \| x-y\| 
		\end{align*}
		So that $$|p(x) - p(y)| \leq  M\|x-y\|$$
		and $p$ is Lipschitz.
		Conversely, suppose that $p$ is Lipschitz. Then there exists $M >0 $ such that for each $x ,y \in X$, $|p(x) - p(y)| \leq  M\|x-y\|$. Let $x \in X$. Then 
		\begin{align*}
			p(x) 
			& \leq |p(x)| \\
			& = |p(x) - p(0)| \\
			& \leq M\|x - 0\| \\
			& \leq M\|x\| 
		\end{align*}
		So $p$ is bounded.
	\end{proof}
	
	\begin{ex} \lex{55013}
	Let $X$ be a normed vector space, $p:X \rightarrow \R$ a bounded sublinear functional and $\phi:X \rightarrow \R$ a linear functional. If $\phi \leq p$, then $\phi \in X^*$. 
	\end{ex}
	
	\begin{proof}
	Since $p$ is Lipschitz, there exists $M >0$ such that for each $x \in X$, 
	\begin{align*}
	p(x) 
	&\leq |p(x)| \\
	&\leq M \|x\|
	\end{align*}
	Let $x \in X$. Then 
	\begin{align*}
	\phi(x) 
	&\leq p(x) \\
	&\leq |p(x)| \\
	&\leq M \|x\| 
	\end{align*}
	and therefore  
	\begin{align*}
	- M \|x\| 
	&= -M \|-x\| \\
	& \leq -p(-x) \\
	& \leq - \phi(-x) \\
	&= \phi(x) 
	\end{align*}
	So that $|\phi(x)| \leq  M\|x\|$ and $\phi \in X^*$.
	\end{proof}
	
	\begin{ex} \lex{55014}
	Let $X$ be a normed vector space and $p:X \rightarrow \R$ a bounded sublinear functional. Then there exists $\phi \in X^*$ such that for each $x \in X$, $\phi(x) \leq p(x)$.
	\end{ex}
	
	\begin{proof}
	\tcr{A previous exercise in the section on sublinear functionals in the topologoical vector space chapter} implies there exists $\phi: X \rightarrow \R$ such that $\phi$ is linear and $\phi \leq p$. \tcr{The previous exercise} implies that $\phi \in X^*$.
	\end{proof}
	
	\begin{ex} \lex{55015} \tbf{Equivalency of linearity (Bounded Case)}\\
	Let $X$ be a normed vector space and $p:X \rightarrow \R$ a bounded sublinear functional. Then the following are equivalent:
	\begin{enumerate}
	\item there exists a unique $\phi \in X^*$ such that $\phi \leq p$
	\item for each $x \in X$, $-p(-x) = p(x)$
	\item $p$ is linear
\end{enumerate}	
	\end{ex}
	
	\begin{proof}
	Basically the same as last time.
	\end{proof}
	
	\begin{ex} \lex{55016}
		Let $X$ be a normed vector space, $M \subset X$ a subspace and $f \in M^*$. Then there exists $F \in X^*$ such that $\|F \|= \|f \|$ and $F|_M = f$.  
	\end{ex}
	
	\begin{proof}
		If $f =0$, Choose $F=0$. Suppose $f \neq 0$. Then $\|f \|\neq 0$ and there exists $x_0 \in M$ such that $x_0  \neq 0$. Thus $\|f \| \neq 0$. Define $p:X \rightarrow \Rg$ by $ p(x) = \|f \|\|x \|$. Then $p$ is a sublinear functional on $X$ and for each $x \in M$, $\vert f(x) \vert \leq p(x)$. So there exists a linear functional $F:X \rightarrow \C$ such that for each $x \in X$, $\vert F(x) \vert \leq p(x) = \|f \|\|x \|$ and $F|_M = f$. Thus $F \in X^*$ with $\|F \|\leq \|f \|$. Also $$\|F \|= \sup_{\substack{ x \in X \\ \|x \|= 1}} \vert F(x) \vert \geq  \sup_{\substack{ x \in M \\ \|x \|= 1}} \vert F(x) \vert = \sup_{\substack{ x \in M \\ \|x \|= 1}} \vert f(x) \vert = \|f \|$$
		
		So $\|F \|= \|f \|$.
	\end{proof}
	
	\begin{ex} \lex{55017}
		Let $X$ be a normed vector space, $M \subsetneq X$ a proper closed subspace and $x \in X \setminus M$. Then there exists $F \in X^*$ such that $F|_M = 0$, $\|F \|=1$ and $ F(x) = \|x+M \|\neq 0$. \\
		\tbf{Hint:} Consider $f:M+\C x \rightarrow \C$ defined by $f(m+\lam x) = \lam \|x +M \|$.
	\end{ex}
	
	\begin{proof}
		Define $f:M+\C x \rightarrow \C$ as above. Clearly $f$ is linear and $f|_M = 0$. Let $m \in M$ and $\lam \in \C$. If $\lam = 0$, then $\vert f(m +\lam x) \vert = 0 \leq \|m+ \lam x \|$. Suppose that $\lam \neq 0$. Then 
		\begin{align*}
			\vert f(m+\lam x) \vert 
			& = \vert \lam \vert \|x+M \|\\
			& =  \|\lam x+M \|\\
			& = \inf_{z \in M} \|z+ \lam x \|\\
			& \leq  \|m+ \lam x  \|\\
		\end{align*} 
		So $f \in (M+\C x )^*$ and $\|f \|\leq 1$. Let $\ep >0$. A previous exercise tells us that there exist $m \in M, \lam \in \C$ such that $\|m+ \lam x \|= 1$ and $\|m+ \lam x +M \|> 1- \ep$. Then 
		\begin{align*}
			\vert f(m + \lam x) \vert
			&= \vert \lam \vert \|x+M\|\\
			&=\|\lam x +M \|\\
			&= \|m + \lam x +M \|\\
			&> 1-\ep
		\end{align*}
		
		So $$ \|f \|= \sup_{\substack{z \in M + \C x \\ \|z \|=1}} \vert f(z) \vert \geq 1$$ Hence $\|f \|=1$. 
		The same exercise also tells us that $f(x) = \|x+M\|\neq 0$. Using the previous exercise, there exists $F \in X^*$ such that $\|F \|= \|f \|= 1$ and $F|_{M+\C x} = f$.
	\end{proof}
	
	\begin{ex} \lex{55018}
		Let $X$ be a normed vector space and $x \in X$. If $x \neq 0$, then there exists $F \in X^*$ such that $\|F \|= 1$ and $F(x) = \|x \|$.
	\end{ex}
	
	\begin{proof}
		Define $f:\C x \rightarrow \C$ by $f(\lam x) = \lam \|x \|$. Then $f$ is linear and $f(x) = \|x \|$. Clearly $$\sup_{\substack{z \in \C x \\ \|z \|=1}}\vert f(z) \vert = 1$$ 
		So $f \in (\C x)^*$ and $\|f \|= 1$. By a previous exercise, there exists $F \in X^*$ such that $\|F \|= \|f \|=1$ and $F|_{\C x} = f$. 
	\end{proof}
	
	\begin{ex} \lex{55019}
	Let $X$ be a normed vector space and $x \in X$. Then $x = 0$ iff for each $\phi \in X^*$, $\phi(x) = 0$.
	\end{ex}
	
	\begin{proof}
	\tcr{Clear by previous exercise.}
\end{proof}		
		
	\begin{ex} \lex{55020}
		Let $X$ be a normed vector space. Then $X^*$ separates the points of $X$. 
	\end{ex}
	
	\begin{proof}
		Let $x, y \in X$. Suppose that $x \neq y$. Then $x-y \neq 0$. The previous exercies implies that there exists $F \in X^*$ such that $\|F \|= 1$ and $$F(x) - F(y) = F(x-y) = \|x-y \|\neq 0$$ Thus $F(x) \neq F(y)$ and $X^*$ separates the points of $X$.
	\end{proof}
	
	
	\begin{ex} \lex{55021}
		Let $X$ be a normed vector space and $f:X \rightarrow \C$ a linear functional on $X$. Then $f$ is bounded iff $\ker f$ is closed. 
	\end{ex}
	
	\begin{proof}
		Suppose that $f$ is continuous. Since $\{0\}$ is closed, we have that $\ker f = f^{-1}(\{0\})$ is closed. Conversely, suppose that $\ker f$ is closed. If $\ker f = X$, then $f =0$ and $f$ is continuous. Suppose that $\ker f \neq X$. Then $\ker f$ is a proper, closed subspace of $X$. A previous exercise tells us that there exists $x \in X$ such that $\|x \|= 1$ and $\|x + \ker f \|> \frac{1}{2}$. Let $y \in X$. Suppose that $\|y \|< \frac{1}{2}$. Then for each $z \in \ker f$, 
		\begin{align*}
			\|z -  (x+y)\|
			& = \|(z-x) -y \|\\
			& \geq \|z-x \|- \|y \|\\
			& > \frac{1}{2} - \frac{1}{2} \\
			&=0
		\end{align*}
		
		So $x+y \not \in \ker f$. Therefore $f(B(x,\frac{1}{2})) \cap \{0\} = \varnothing$. If $f(B(x,\frac{1}{2})) $ is unbounded, then $f(B(x,\frac{1}{2})) = \C$ by linearity. This is a contradiction since $0 \not \in f(B(x,\frac{1}{2}))$. So There exists $s > 0$ such that $f(B(x,\frac{1}{2})) \subset B(0,s)$ and thus $f$ is bounded. 
	\end{proof}
	
	\begin{ex} \lex{55022}
		Let $X$ be a normed vector space. 
		\begin{enumerate}
			\item Let $M \subset X$ be a proper closed subspace of $X$ and $x \in X \setminus M$. Then $M + \C x$ is closed.
			\item Let $M \subset X$ be a finite dimensional subspace of $X$. Then $M$ is closed.
		\end{enumerate}
	\end{ex}
	
	\begin{proof}
		\begin{enumerate}
			\item Let $y \in X$ and $(y_n)_{n \in \N} \subset M+ \C x$. Suppose that $y_n \conv{} y$. If $y \in M$, then $y \in M+ \C x$. Suppose that $y \not \in M$. For each $n \in \N$, there exists $m_n \in M$ and $\lam_n \in \C$ such that $y_n = m_n + \lam_nx$. A previous exercise tells us that there exists $F \in X^*$ such that $\|F \|= 1$, $F|_M = 0$ and $F(x) = \|x+M \|\neq 0$. Since $F$ is continuous, $F(y_n) \conv{} F(y)$. Since for each $n \in \N$, $$F(y_n) = F(m_n + \lam_n x) = F(m_n)+ \lam_n (F_x) = \lam_n F(x)$$ we have that $\lam_n F(x) \conv{} F(y)$. Since $F(x) \neq 0$, this implies that $\lam_n \conv{} F(x)^{-1} F(y)$. It follows that $\lam_n x \conv{}F(x)^{-1}F(y)x$. Since  for each $n \in \N$, $m_n = y_n - \lam_nx$, we know that $m_n \conv{} y-F(x)^{-1}F(y)x$. Since $(m_n)_{n \in \N} \subset M$ and $M$ is closed, we have that $y-F(x)^{-1}F(y)x \in M$ and therefore $y \in M+\C x$. Hence $M+\C x$ is closed. \vspace{.5cm}\\
			\item If $M = X$, then $M$ is closed. Suppose that $M \neq X$. Let $(x_i)_{i=1}^n$ be a basis for $M$. Define $N_0 = \{0\}$ and for each $i =1,2, \cdots, n$, define $N_i = N_{i-1}+\C x_i$. Since $N_0$ is a proper closed subpace of $X$ and $x_1 \in X \setminus N_0$, (1) implies that $N_1$ is closed. Proceed inductively to obtain that $M = N_n$ is closed.
		\end{enumerate}
	\end{proof}
	
	\begin{ex} \lex{55023}
		Let $X$ be an infinite-dimensional normed vector space. 
		\begin{enumerate}
			\item There exists a sequence $(x_n)_{n\in \N} \subset X$ such that for each $m, n \in \N$, $\|x_n \|= 1$ and if $m \neq n$, then $\|x_m - x_n \|> \frac{1}{2}$.
			\item $X$ is not locally compact. 
		\end{enumerate}
	\end{ex}
	
	\begin{proof}\
		\begin{enumerate}
			\item Define $N_0 = \{0\}$. Then $N_0$ is a closed proper subspace of $X$. Choose $x_1 \in X$ such that $\|x_1 \|= 1$. Using the results of previous exercises, we proceed inductively. For each $n \geq 2$ we define $N_{n-1} = \text{span}(x_1, x_2, \cdots, x_{n-1})$. Then $N_{n-1}$ is a closed proper subspace of $X$. Thus we may choose $x_n \in X$ such that $\|x_n \|= 1$ and $\|x_n + N_{n-1} \|>  \frac{1}{2}$. Let $m,n \in \N$. Suppose that $m<n$. Then $x_m \in N_{n-1}$. Thus $\|x_n - x_m \|\geq \|x_n + N_{n-1} \|> \frac{1}{2}$\vspace{.5cm}\\
			\item Suppose that $X$ is locally compact. Then $\bar{B}(0,1)$ is compact and therefore sequentially compact. Using $(x_n)_{n \in \N} \subset \bar{B}(0,1)$ defined in (1), we see that there exists a subsequence $(x_{n_k})_{k \in \N}$, $x \in \bar{B}(0,1)$ such that $x_{n_k} \conv{} x$. Then $(x_{n_k})_{k \in \N}$ is Cauchy. So there exists $N \in N$ such that for each $j, k \in \N$, if $j, k \geq N$, then $\|x_{n_j} - x_{n_k} \|< \frac{1}{2}$. Then $\|x_{n_N} - x_{n_{N+1}} \| < \frac{1}{2}$. This is a contradiction since by construction, $\|x_{n_N} - x_{n_{N+1}} \| > \frac{1}{2}$. Thus $X$ is not locally compact.
		\end{enumerate}
	\end{proof}
	
	
	
	
	
	
	
	
	
	
	
	
	
	
	
	
	
	

	
	
	
	
	
	
	
	
	
	
	
	
	
	
	
	
	
	
	
	
	
	
	
	\newpage
	\section{Applications of the Baire Category Theorem} 
	
	
	\begin{thm} \lex{ex:banach:baire_cat:00001} \tbf{Open Mapping Theorem:} \\
		Let $X, Y$ be Banach spaces and $T\in L(X; Y)$. If $T$ is surjective, then $T$ is open.
	\end{thm}
	
	\begin{cor} \lex{ex:banach:baire_cat:00002}
		Let $X, Y$ be Banach spaces and $T \in L(X; Y)$. If $T$ is a bijection, then $T^{-1} \in L(X; Y)$.
	\end{cor}
	
	\begin{defn} \ld{def:banach:baire_cat:00003}
		Let $X,Y$ be sets and $f:X \rightarrow Y$. We define the \tbf{graph of f}, $\Gam(f)$, by $\Gam(f) = \{(x,y) \in X \times Y: f(x) = y\}$.
	\end{defn}
	
	\begin{thm} \lex{ex:banach:baire_cat:00004} \tbf{Closed Graph Theorem:} \\		
		Let $X, Y$ be Banach spaces and $T:X \rightarrow Y$ a linear map. If $\Gam(T)$ is closed, then $T \in L(X; Y)$.  
	\end{thm}
	
	\begin{note}
		We recall that $\Gam(T)$ is closed iff for each $(x_n)_{n \in \N} \subset X$, $x \in X$ and $y \in Y$, $x_n \conv{} x$ and $T(x_n) \conv{} y$ implies that $T(x) = y$. 
	\end{note}
	
	\begin{ex} \lex{ex:banach:baire_cat:00005} \tbf{Uniform Boundedness Principle:} \\		
		Let $X, Y$ be Banach spaces and $S \subset L(X; Y)$. If for each $x \in X$, $$\sup_{T \in S} \|Tx \|< \infty$$ then $$\sup_{T \in S} \|T \|< \infty$$
	\end{ex}

	\begin{proof}
		\tcb{Finish!!! }
	\end{proof}
	
	\begin{ex} \lex{ex:banach:baire_cat:00006}
		Let $\mu$ be counting measure on $(N, \MP(\N))$. Define $h: \N \rightarrow \N$ and $ \nu$ on $(N, \MP(\N))$ by $h(n) = n$ and $d \nu = h d \mu$. Define $X=L^1(\nu)$ and $Y = L^1(\mu)$. Equip both $X$ and $Y$ with the $L^1$ norm with respect to $\mu$. 
		\begin{enumerate}
			\item We have that $X$ is a proper subspace of $Y$ and therefore $X$ is not complete.
			\item Define $T: X \rightarrow Y$ by $Tf(n) = nf(n)$. Then $T$ is linear, $\Gam(T)$ is closed, and $T$ is unbounded.
			\item Define $S:Y \rightarrow X$ by $Sg(n) = \frac{1}{n}g(n)$. Then $S \in L(Y,X)$, $S$ is surjective and $S$ is not open. 
		\end{enumerate}
	\end{ex}
	
	\begin{proof}\
		\begin{enumerate}
			\item Note that for each $f: \N \rightarrow \C$, 
			\begin{align*}
				{\|f \|}_{\mu, 1}
				&= \sum_{n=1}^{\infty} \vert f(n) \vert  \\
				& \leq \sum_{n=1}^{\infty} n \vert f(n) \vert  \\
				& = \|f \|_{\nu,1} 
			\end{align*} 
			Hence $X$ is a subspace of $Y$. Define $f : \N \rightarrow \C$ by $f(n) = \frac{1}{n^2}$. Then $$\|f \|_{\mu, 1} = \sum_{n=1}^{\infty} \frac{1}{n^2} < \infty$$ So  $f \in Y$. 
			However 
			$$\|f \|_{\nu, 1} = \sum_{n=1}^\infty \frac{1}{n} = \infty$$ 
			So $f \not \in X$. Thus $X$ is a proper subspace of $Y$. Let $g \in Y$ and $\ep >0$. Since the simple functions are dense in $L^1(\mu)$, there exists $\phi \in L^1(\mu)$ such that $\phi$ is simple and $\|g - \phi \|_{\mu ,1} < \ep$. Then there exist $(c_i)_{i=1}^k \subset \C$ and $ (E_i)_{i=1}^k \subset \MP(\N)$ such that for each $i,j \in  \{1,2,\cdots, k\}$, $E_i$ is finite, $i \neq j$ implies that $E_i \cap E_j = \varnothing$ and  
			$$\phi = \sum_{i=1}^kc_i \chi_{E_i}$$ 
			Define $c = \max\{\vert c_i \vert: i=1,2,\cdots k\}$ and $m = \max \bigg[ \bigcup\limits_{i=1}^k E_i \bigg]$. Then 
			\begin{align*}
				\|\phi \|_{\nu,1} 
				&=  \sum_{n=1}^m n \vert \phi(n) \vert \\
				& \leq \sum_{n=1}^m  mc \\
				& = c m^2 \\
				& < \infty
			\end{align*}
			Hence $\phi \in X$ and $X$ is dense in $Y$. Since $X$ is a dense, proper subspace, it is not closed. Since $Y$ is complete and $X \subset Y$ is not closed, we have that $X$ is not complete.
			\item Clearly $T$ is linear. Let $(f_j)_{j \in \N} \subset X$, $f \in X$ and $g \in Y$. Suppose that $f_j \conv{L^1(\mu)} f$ and $Tf_j \conv{L^1(\mu)} g$. 
			
			Note that for each $j \in \N$ and $n \in \N$, $$\vert f_j(n) - f(n) \vert \leq \sum_{n =1}^{\infty}\vert f_j(n) - f(n) \vert = \|f_j-f \|_{\mu, 1}$$ and $$\vert nf_j(n) - g(n) \vert \leq \sum_{n =1}^{\infty}\vert nf_j(n) - g(n) \vert = \|Tf_j - g\|_{\mu, 1}$$  
			Thus for each $n \in \N$, $f_j(n) \conv{j} f(n)$ and $nf_j(n) \conv{j} g(n)$. This implies that for each $n \in \N$, $nf(n) = g(n)$. Thus $Tf = g$ which implies that $\Gam(T)$ is closed. Suppose, for the sake of contradiction, that $T$ is bounded. Then there exists $C \geq 0$ such that for each $f \in X$, $\|Tf \|_{\mu,1} \leq C \|f \|_{\mu, 1}$. Choose $n \in \N$ such that $n > C$. Define $f: \N \rightarrow \C$ by $f = \chi_{\{n\}}$. As established above, $S^+ \subset L^1(\mu)$. Then $\|f \|_{\mu,1} = 1$ and
			\begin{align*}
				\|Tf \|_{\mu,1}
				& = n \\
				&> C\\
				& = C \|f \|_{\mu,1}
			\end{align*}
			which is a contradiction. So $T$ is unbounded.
			\item Clearly $S$ is linear. Let $g \in Y$. Then \begin{align*}
				\|Sg \|_{\mu,1} 
				&= \sum_{n =1}^{\infty} \frac{1}{n} \vert g(n) \vert \\
				& \leq  \sum_{n =1}^{\infty} \vert g(n) \vert \\
				& = \|g \|_{\mu,1}
			\end{align*}
			So $S$ is bounded and $\|S \|\leq 1$. Thus $S \in L(Y,X)$. Let $f \in X$. Define $g: \N \rightarrow \C$ by $g(n) = nf(n)$. By defnition, $g \in Y$ and we have that
			\begin{align*}
				Sg(n) 
				&= \frac{1}{n}g(n) \\
				& = f(n)
			\end{align*}
			Hence $Sg =f$ and thus $S$ is surjective. Let $g \in Y$. Suppsose that $Sg = 0$. Then $$\sum_{n=1}^{\infty} \frac{1}{n}\vert g(n)\vert =\|Sg \| = 0$$ Thus for each $n \in \N$, $g(n) = 0$. Hence $\ker S = \{0\}$ and $S$ is injective. Note that for each $A \subset Y$, $S(A)= T^{-1}(A)$. If $S$ is open, then $T$ is continuous which as shown above is a contradiction. So $g$ is not open. 
		\end{enumerate}
	\end{proof}
	
	\begin{ex} \lex{ex:banach:baire_cat:00007}
		Let $X = C^1([0,1])$ and $Y=C([0,1])$. Equip both $X$ and $Y$ with the uniform norm. 
		\begin{enumerate}
			\item Then $X$ is not complete
			\item Define $T: X \rightarrow Y$ by $Tf = f'$. Then $\Gam(T)$ is closed and $T$ is not bounded. 
		\end{enumerate}
	\end{ex}
	
	\begin{proof}
		\begin{enumerate}
			\item Recall that for each $a,b \geq 0$ and $p \in \N$, $$(a^{\frac{1}{p}}+b^{\frac{1}{p}})^p = \sum_{n=0}^p  {p \choose n} a^{\frac{n}{p}}b^{\frac{p-n}{p}} \geq a + b$$ Thus $(a+b)^{\frac{1}{p}} \leq a^{\frac{1}{p}}+b^{\frac{1}{p}}$.\\
			For each $n \in \N$, define $f_n: [0,1] \rightarrow \C$ by $f_n(x) = \sqrt{(x-\frac{1}{2})^2+ \frac{1}{n^2}}$. Then $(f_n)_{n \in \N} \subset X$. Define $f:[0,1] \rightarrow \C$ by $f(x) = \vert x-\frac{1}{2}\vert$. Then $f \in Y \cap X^c$. Note that for each $n \in \N$, $f \leq f_n$. Our observation above implies that for each $x \in X$,
			\begin{align*}
				f_n(x) 
				&= \bigg[ (x-\frac{1}{2})^2 + \frac{1}{n^2} \bigg]^{\frac{1}{2}}\\
				& \leq \vert x-\frac{1}{2} \vert + \frac{1}{n}
			\end{align*}
			Thus $0 \leq f_n - f \leq \frac{1}{n} $. This implies that $f_n \convt{u} f$. Since $f \not \in X$, $X$ is not complete. \vspace{.5cm}\\
			\item Let $(f_n)_{n \in \N} \subset X$, $f \in X$ and $g \in Y$. Suppose that $f_n \convt{u} f$ and $Tf_n \convt{u} g$. Let $x \in [0,1]$. Then $f_n(x) \conv{} f(x)$ and $f_n(0) \conv{} f(0)$ and $f_n' \convt{u} g$. Applying the DCT to this sequence of integrable functions that converges uniformly to an integrable function on a finite measure space (a previous exercise) we have that
			\begin{align*}
				f_n(x) - f_n(0) 
				&= \int_{[0,x]} f_n' dm \\
				& \rightarrow \int_{[0,x]} g dm \\ 
			\end{align*} 
			Since $f_n(x) - f_n(0) \conv{} f(x) - f(0)$, we know that $$f(x) - f(0) = \int_{[0,x]} g dm$$ Thus $Tf = g$ and $\Gam(T)$ is closed. \\
			By \rex{42002}, $T$ is not bounded.
		\end{enumerate}
	\end{proof}
	
	\begin{ex} \lex{ex:banach:baire_cat:00008}
		Let $X, Y$ be Banach spaces and $T \in L(X; Y)$. Then $X/\ker T \cong T(X)$ iff $T(X)$ is closed.
	\end{ex}
	
	\begin{proof}
		Since $X$ is a banach space and $T$ is continuous, we have that $\ker T$ is closed and $X/ \ker T$ is a Banach space. Suppose that $X/ \ker T \cong T(X)$. Then $T(X)$ is complete. Since $Y$ is complete, this implies that $T(X)$ is closed. \\
		Conversely Suppose that $T(X)$ is closed. Then $T(X)$ is complete. Define $S: X/ \ker T \rightarrow T(X)$ by $S(x + \ker T) = T(x)$. A previous exercise tells us that the map $S: X/ \ker T \rightarrow T(X)$ defined by $S(x + \ker T) = T(x)$ is a bounded linear bijection. Since $T(X)$ is complete and $S$ is surjective, $S^{-1}$ is bounded and thus $S$ is an isomorphism.   
	\end{proof}
	
	\begin{ex} \lex{ex:banach:baire_cat:00009}
		Let $X$ be a separable Banach space. Define $B_X = \{x \in X: \|x \|< 1\}$. Let $(x_n)_{n \in \N} \subset B_X $ a dense subset of the unit ball and $\mu$ the counting measure on $(\N, \MP(\N))$. Define $T: L^1(\mu) \rightarrow X$ by $$Tf = \sum_{n=1}^{\infty}f(n)x_n$$ Then 
		\begin{enumerate}
			\item $T$ is well defined and $T \in L(L^1(\mu), X)$
			\item $T$ is surjective
			\item There exists a closed subspace $K \subset L^1(\mu)$ such that $L^1(\mu)/K \cong X$ 
		\end{enumerate} 
	\end{ex}
	
	\begin{proof}
		\begin{enumerate}
			\item Let $f \in L^1(\mu)$. Since $X$ is complete and 
			\begin{align*}
				\sum_{n=1}^{\infty}\|f(n)x_n \|
				& = \sum_{n=1}^{\infty} \vert f(n) \vert \|x_n \|\\
				& \leq \sum_{n=1}^{\infty} \vert f(n) \vert \\
				&< \infty 
			\end{align*}
			we have that $\sum_{n=1}^{\infty} f(n)x_n $ converges and thus $Tf \in X$. Hence $T$ is well defined. \vspace{.5cm}\\
			Clearly $T$ is linear. Let $f \in L^1(\mu)$. Then
			\begin{align*}
				\|Tf \|
				&= \| \sum_{n=1}^{\infty} f(n)x_n \|\\
				& \leq \sum_{n=1}^{\infty} \|f(n)x_n \|\\
				& \leq \sum_{n=1}^{\infty} \vert f(n) \vert \\
				&= \|f \|_1
			\end{align*}
			So $T$ is bounded with $\|T \|\leq 1$.\vspace{.5cm}\\
			\item Let $x \in X$. Suppose that $\|x \|< 1$. Then $x \in B_X$. So there exists $n_1 \in \N$ such that $\|x - x_{n_1} \|< \frac{1}{2}$. Then $2(x-x_{n_1}) \in B_X$. Since for each $j \in \N$, $B_X\setminus (x_n)_{n=1}^j$ is dense in $B_X$, there exists $n_2 \in \N$ such that $x_{n_2} \not \in (x_n)_{n=1}^{n_1}$ and $\|2(x- x_{n_1}) - x_{n_2} \|< \frac{1}{2}$ which implies that $\|x- (x_{n_1} - \frac{1}{2}x_{n_2}) \|< \frac{1}{4}$. \vspace{.5cm}\\ 
			Proceed inductively to obtain a subsequence $(x_{n_k})_{k \in \N}$ such that for each $k \geq 2$, $x_{n_k} \not \in (x_n)_{n=1}^{n_{k-1}}$ and $\|x - \sum_{j=1}^k 2^{1-j}x_{n_j} \|< \frac{1}{2^k}$. Then $x = \sum_{k=1}^{\infty}2^{1-k}x_{n_k}$. \vspace{.5cm} \\ 
			Define $f:\N \rightarrow \C$ by $f = \sum_{k=1}^{\infty}2^{1-k}\chi_{\{n_k\}}$. Then $\|f \|_1 = \sum_{k=1}^{\infty}2^{1-k}< \infty$, so $f \in L^1(\mu)$ and $Tf = \sum_{k=1}^{\infty}2^{1-k}x_{n_k} = x$. Now, suppose that $\|x \|\geq 1$, then $\frac{1}{2\|x \|}x \in B_X$. The above argument shows that there exists $f \in L^1(\mu)$ such that $Tf = \frac{1}{2\|x \|}x$. Then $2 \|x \|f \in L^1(\mu)$ and $T(2 \|x \|f) = 2 \|x \|Tf =x$. \\
			So for each $x \in X$, there exists $f \in L^1(\mu)$ such that $Tf = x$ and thus $T$ is surjective. 
			\item Since $X$ is a Banach space and $T$ is surjective, the previous exercise implies that $L^1(\mu)/\ker T \cong X$. 
		\end{enumerate}
	\end{proof}
	
	












































	\newpage
	\section{Duality}

	\begin{note} 
		Let $X$ be a normed vector space. Then $X^*$ is a normed vector space. In general the weak-$*$ topology on $X^*$ is not necessarily the same as the norm topology on $X^*$. In the context of normed vector spaces, we will write $X^{**}$ to denote $(X^*)^*$ when $X^*$ is equipped with the norm topology and $\hat{X}$ to denote $(X^*)^*$ when $X^*$ is equipped with the weak-$*$ topology. 
	\end{note}
	
	\begin{ex} \lex{ex:banach:duality:0001}
		Let $X$ be a normed vector space and $x \in X$. Define $\hat{x}:X^* \rightarrow \C$ by $\hat{x}(f) = f(x)$. Then $\hat{x} \in X^{**}$ and $\|\hat{x} \|= \|x \|$. \\
		\tbf{Hint:} Hahn-Banach theorem
	\end{ex}
	
	\begin{proof}
		Let $f,g \in X^*$ and $\lam \in \C$. Then $$\hat{x}(f+\lam g) = (f+ \lam g)(x) = f(x) + \lam g(x) = \hat{x}(f) + \lam \hat{x}(g)$$
		So $\hat{x}$ is linear. For each $f \in X^*$, $$\vert \hat{x}(f) \vert = \vert f(x) \vert \leq \|x \|\|f \|$$ Hence $\hat{x} \in X^{**}$ with $\|\hat{x} \|\leq \|x \|$. If $x=0$, then $\hat{x} = 0$ and $\|\hat{x} \|= \|x \|$. Suppose that $x \neq 0$. Then a previous exercise implies that there exists $F \in X^*$ such that $\|F \|=1$ and $F(x) = \|x \|$. Then we have that $$\sup_{\substack{f \in X^* \\ \|f \|= 1 } } \vert \hat{x}(f) \vert  = \sup_{\substack{f \in X^* \\ \|f \|= 1 }}  \vert f(x) \vert \geq \vert F(x) \vert = \|x \|$$
		Hence $\|\hat{x} \|= \|x \|$.
	\end{proof}

	\begin{ex} \lex{ex:banach:duality:0002}
		Let $X$ be a topological vector space. Then 
		\begin{enumerate}
			\item $\MT_{w^*} \subset \MT_{X^*}$
			\item For each $E \subset X^*$, if $E$ is weak-* closed, then $E$ is norm closed 
		\end{enumerate}
	\end{ex} 
	
	\begin{proof}\
		\begin{enumerate}
			\item Since $\hat{X} \subset X^{**}$, we have that 
			\begin{align*}
				\MT_{w^*}
				& = \tau_{X^*}(\hat{X}) \\
				& \subset \tau_{X^*}(X^{**}) \\
				& = \MT_{X^*}
			\end{align*}
			\item Let $E \subset X^*$. Suppose that $E$ is weak-* closed. Then 
			\begin{align*}
				E^c 
				& \in \MT_{w^*} \\
				& \subset \MT_{X^*} \\
			\end{align*}
			So $E$ is norm closed. 
		\end{enumerate}
	\end{proof}

	\begin{ex} \lex{ex:banach:duality:0003}
		Let $X$ be a normed vector space. If $X$ is separable, then there exist $(\phi_n)_{n \in \N} \subset X^*$ such that for each $n \in \N$, $\|\phi_n\| = 1$ and for each $x \in X$, 
		$$\|x\| = \sup_{n \in \N} |\phi_n(x)|$$  
		\tbf{Hint:} Let $(x_n)_{n \in \N} \subset X$ be a dense subset. A previous exercise on the Hahn-Banach theorem implies that for each $n$, there exists $\phi_n \in X^*$ such that $\|\phi_n\| = 1$ and $\phi_n(x_n) = \|x_n\|$. Then for each $x \in X$, $$\|x\| = \sup_{n \in \N} |\phi_n(x)|$$. 
	\end{ex}
	
	\begin{proof}
		Suppose that $X$ is separable. Then there exists $(x_n)_{n \in \N} \subset X$ such that $(x_n)_{n \in \N}$ is dense in $X$. A previous exercise on the Hahn-Banach theorem implies that for each $n$, there exists $\phi_n \in X^*$ such that $\|\phi_n\| = 1$ and $\phi_n(x_n) = \|x_n\|$. Let $x \in X$. Then 
		\begin{align*}
			\|x\| 
			&= \|\hat{x}\| \\
			&= \sup_{\substack{\phi \in X^* \\ \|\phi\| = 1}} \|\hat{x}(\phi)\| \\
			&= \sup_{\substack{\phi \in X^* \\ \|\phi\| = 1}} \|\phi(x)\| \\
			& \geq \sup_{n \in \N} \|\phi_n(x)\| \\
		\end{align*}
		Let $\ep > 0$. Choose $N \in \N$ such that $\|x - x_N\| < \ep/2$. Then
		\begin{align*}
			\|x\| 
			& \leq \|x - x_N\| + \|x_N\| \\
			& = \|x - x_N\| + |\phi_N(x_N)| \\
			& \leq \|x - x_N\| + |\phi_N(x_N - x)| + |\phi_N(x)| \\
			& \leq \|x - x_N\| + \|\phi_N\|\|x_N - x\| + |\phi_N(x)| \\
			& \leq 2\|x - x_N\| + |\phi_N(x)| \\
			& < 2 \frac{\ep}{2} +  |\phi_N(x)| \\
			& \leq \ep + \sup_{n \in \N} |\phi_n(x)|
		\end{align*}
		Since $\ep >0$ is arbitrary, $\|x\| \leq \sup\limits_{n \in \N} |\phi_n(x)|$. So $\|x\| = \sup\limits_{n \in \N} |\phi_n(x)|$.
	\end{proof}
	
	
	\begin{ex} \lex{ex:banach:duality:0004}
		Let $X$ be a normed vector space. Define $\phi : X \rightarrow X^{**}$ by $\phi(x) = \hat{x}$. Then $\phi$ is a linear isometry. 
	\end{ex}
	
	\begin{proof}
		Let $x,y \in X$ and $\lam \in \C$. Then for each $f \in X^*$, we have that 
		\begin{align*}
			\phi(x+ \lam y)(f) 
			&= \widehat{x+ \lam y}(f) \\
			&= f(x+\lam y) \\
			&= f(x) + \lam f(y) \\
			&= \hat{x}(f) + \lam \hat{y}(f)\\
			&= \phi(x)(f) + \lam \phi(y)(f)
		\end{align*} 
		So $\phi(x+ \lam y) = \phi(x) + \lam \phi(y)$ and $\phi$ is linear. The previous exercise tells us that 
		\begin{align*}
			\|\phi(x) - \phi(y) \|
			&= \|\phi(x-y)\|\\
			&= \|\widehat{x-y} \|= \|x-y \|
		\end{align*}
		So $\phi$ is an isometry.
	\end{proof}
	
	\begin{defn} \ld{def:banach:duality:0005}
		Let $X$ be a normed vector space and define $\phi:X \rightarrow X^{**}$ as above. We define $\widehat{X} = \phi(X) \subset X^{**}$. Since $\widehat{X}$ and $X$ are isomorphic, we may identify $X$ as a subset of $X^{**}$. 
	\end{defn}
	
	\begin{defn} \ld{def:banach:duality:0006}
		Let $X$ be a normed vector space and define $\phi:X \rightarrow X^{**}$ as above. Then $X$ is said to be \tbf{reflexive} if $\phi$ is surjective. In this case $\phi$ is then an isomorphism
	\end{defn}

	\begin{defn} \ld{def:banach:duality:0007}
	Let $X,Y$ be normed vector spaces and $T \in L(X; Y)$. Define the \tbf{adjoint of $T$}, denoted  $T^*:Y^* \rightarrow X^*$, by $T^*(f) = f \circ T$. 
	\end{defn}
	
	\begin{ex} \lex{ex:banach:duality:0008}
		Let $X,Y$ be normed vector spaces and $T \in L(X; Y)$. 
		\begin{enumerate}
			\item Then $T^* \in L(Y^*, X^*)$.
			\item Applying the result from (1) twice, we have that $T^{**} \in L(X^{**},Y^{**})$. We have that for each $x \in X$, $T^{**}(\hat{x}) = \widehat{T(x)}$.
			\item $T^*$ is injective iff $T(X)$ is dense in $Y$.
			\item If $T^*(Y^*)$ is dense in $X^*$, then $T$ is injective. The converse is true if $X$ is reflexive.
		\end{enumerate}
	\end{ex}
	
	\begin{proof}\
		\begin{enumerate}
			\item Let $f \in Y^*$. Then $\|T^* (f) \|= \|f \circ T \|\leq  \|T \| \|f \|$. So $T^* \in L(Y^*, X^*)$ with $\|T^* \|\leq \|T \|$.\vspace{.5cm}\\
			\item Let $x \in X$. Let $f \in Y^*$. Then 
			\begin{align*}
				T^{**}(\hat{x})(f) 
				&= \hat{x} \circ T^{*}(f) \\
				&= \hat{x}(T^* (f)) \\
				&= \hat{x}(f \circ T) \\
				&= f \circ T (x) \\
				&= f(T(x)) \\
				&= \widehat{T(x)}(f)
			\end{align*} 
			Hence $T^{**}(\hat{x}) = \widehat{T(x)}$.\vspace{.5cm}\\
			\item Suppose that $T(X)$ is not dense in $Y$. Then $\cl T(X) \neq Y$. So $T(X)$ is a proper closed subspace of $Y$ and there exists $y \in Y$ such that $y \not \in \cl T(X)$. By a previous exercise, there exists $f \in Y^*$ such that $f(y) = \|y+\cl T(X) \|\neq 0$, $\|f \|=1$ and $f|_{\cl T(X)} = 0$. Let $x \in X$. Then $T^*(f)(x) = f \circ T(x) = 0$. Hence $T^*(f) = 0 = T^*(0)$. Since $f \neq 0$, $T^*$ is not injective.\\ Now suppose that $T(X)$ is dense in $Y$. Let $f,g \in Y^*$. Define $h \in Y^*$ by $h = f-g$ Suppose that $T*(f) = T^*(g)$ Then $T^*(h) = 0$. So for each $x \in X$, $h(T(x)) = 0$. Let $y \in Y$ and $\ep >0$. By continuity, there exists $\del > 0 $ such that for each $y' \in Y$, if $\|y - y' \|< \del$, then $\|h(y) - h(y') \|< \ep$. Since $T(X)$ is dense in $Y$, there exists $x \in X$ such that $\|y - T(x) \|< \del$. Thus 
			\begin{align*}
				\|h (y) \|
				&\leq \|h(y) - h(T(x)) \|+ \|h(T(x)) \|\\
				& = \|h(y) - h(T(x)) \| \\
				& < \ep
			\end{align*} 
			Since $\ep > 0$ is arbitrary, $\|h(y) \|= 0$. This implies that $h(y) = 0$ and therefore $f(y) = g(y) $. Since $y \in Y$ is arbitrary, $f=g$ and $T^*$ is injective. \vspace{.5cm}\\
			\item For the sake of contradiction, suppose that $T^*(Y^*)$ is dense in $X^*$ and $T$ is not injective. Then there exist $x_1, x_2 \in X$ such that $x_1 \neq x_2$ and $T(x_1) = T(x_2)$. Define $x = x_1-x_2$. Then $x \neq 0$ and $T(x) = 0$. A previous exercise implies that there exists $F \in X^*$ such that $F(x) = \|x\|\neq 0$ and $\|F \|= 1$. Let $\ep >0$. Choose $g \in Y^*$ such that $\|F - T^*(g) \|< \ep$. Then 
			\begin{align*}
				\|x \|
				&= \vert F(x) \vert \\
				&\leq \vert F(x) - T^*(g)(x) \vert + \vert T^*(g)(x) \vert \\
				& < \ep \|x \|+ \vert g(T(x)) \vert\\
				&= \ep \|x \|
			\end{align*}
			
			Since $\ep > 0$ is arbitrary, we have that $\|x \|=0$ which is a contradiction. Hence if $T^*(Y^*) $ is dense in $X^*$, then $T$ is injective. \vspace{.5cm}\\ 
			Now, suppose that $X$ is reflexive and $T$ is injective. Let $\phi_1, \phi_2 \in X^{**}$. Suppose that $T^{**}(\phi_1) = T^{**}(\phi_2)$. Then $T^{**}(\phi_1 - \phi_2) = 0$. Since $X$ is reflexive, there exist $x_1, x_2 \in X$ such that $\phi_1 = \hat{x_1}$ and $\phi_2 = \hat{x_2}$. Define $x = x_1 - x_2$. Then $T^{**}(\hat{x}) = 0$. So for each $f \in Y^*$, 
			\begin{align*}
				T^{**}(\hat{x})(f) 
				&= \hat{x} \circ T^*(f)\\
				&= \hat{x}( T^*(f))\\
				&= \hat{x} (f \circ T)\\
				&= f \circ T(x)\\
				&= f(T(x))\\
				&= 0 
			\end{align*}
			Suppose that $T(x) \neq 0$. Then a previous exercise implies that there exists $g \in Y^*$ such that $g(T(x)) = \|T(x) \|\neq 0$ and $\|g \| = 1$. This is a contradiction since $g(T(x)) = 0$. So $T(x) = 0$. Since $T$ is injective, this implies that $x = 0$. Hence $\hat{x}=0$ and thus $\phi_1 = \phi_2$. Thus $T^{**}$ is injective. By (3), we have that $T^*(Y^*)$ is dense in $X^*$.
		\end{enumerate}
	\end{proof}
	
	\begin{ex} \lex{ex:banach:duality:0009}
		Let $X$ be a normed vector space. Then $X$ is reflexive iff $X^*$ is reflexive. 
	\end{ex}
	
	\begin{proof}
		Suppose that $X$ is reflexive. Let $\al \in X^{***}$. Define $f :X \rightarrow \C$ by $f(x) = \al(\hat{x})$. Clearly $f$ is linear and a previous exercise tells us that for each $x \in X$, 
		\begin{align*}
			\vert f(x) \vert 
			& \leq \|\al \|\|\hat{x} \|\\
			&= \|\al \|\|x \|
		\end{align*}
		So $f \in X^*$.
		Let $\phi \in X^{**}$. Since $X$ is reflexive, there exists $x \in X$ such that $\phi = \hat{x}$. Then 
		\begin{align*}
			\al(\phi)
			&= \al(\hat{x})\\
			&= f(x)\\
			&= \hat{x}(f)\\
			&= \hat{f}(\hat{x})\\
			&= \hat{f}(\phi)
		\end{align*}
		Hence $\al = \hat{f}$. Thus the map $X^* \rightarrow X^{***}$ given by $f \mapsto \hat{f} $ is surjective and so $X^{*}$ is reflexive.\vspace{.5cm}\\
		Conversely, suppose that $X^*$ is reflexive. Since $\phi:X \rightarrow X^{**}$ given by $\phi(x) = \hat{x}$ is an isometry, $\widehat{X} \subset X^{**}$ is closed. For the sake of contradiction, suppose that $\widehat{X} \neq X^{**}$. Then there exists $\al \in X^{**}$ such that $\al \not \in \widehat{X}$. Thus there exists $F \in X^{***}$ such that $\|F \|= 1$, $F(\al) = \|\al + \widehat{X} \|\neq 0$ and $F|_{\widehat{X}}=0$. Since $X^*$ is reflexive, there exists $f \in X^*$ such that $F = \hat{f}$. A previous exercise tells us that $\|f \|= \|\hat{f} \|= \|F \|= 1$. Since for each $x \in X$, $f(x) = \hat{x}(f) = \hat{f}(\hat{x}) = F(\hat{x}) = 0$, we have that $f = 0$. Thus $\|f \|= 0$, a contradiction. So $\widehat{X} = X^{**}$ and $X$ is reflexive.
	\end{proof}
	
	
	\begin{defn} \ld{def:banach:duality:00010}
	Let $X$ be a normed vector space, $M \subset X$ and $N \subset X^*$. We define the \tbf{annihilator} of $M$ and the annihilator of $N$, denoted by $M^{\perp} \subset X^*$ and $^{\perp}N \subset X$ respectively, by 
	\begin{align*}
	 M^{\perp} &= \{\phi \in X^*: \text{for each $x \in M$, $\phi(x) = 0$}\} \\
	 ^{\perp}N &= \{x \in X: \text{for each $\phi \in N$, $\phi(x) = 0$}\}
	\end{align*}
	\end{defn}	
	
	\begin{ex} \lex{ex:banach:duality:0011}
	Let $X$ be a normed vector space, $M \subset X$ and $N \subset X^*$. Then 
	\begin{enumerate}
	\item $$M^{\perp} = \bigcap_{x \in M} \ker \hat{x}$$
	\item $$^{\perp}N = \bigcap_{\phi \in N} \ker \phi $$
	\end{enumerate}
	\end{ex}
	
	\begin{proof}\
	\begin{enumerate}
	\item 
	\begin{align*}
	M^{\perp} 
	&= \{\phi \in X^*: \text{for each $x \in M$, $\phi(x) = 0$}\} \\
	&= \bigcap_{x \in M} \{\phi \in X^*: \phi(x) = 0\} \\
	&= \bigcap_{x \in M} \{\phi \in X^*: \hat{x}(\phi) = 0\} \\
	&= \bigcap_{x \in M} \ker \hat{x}
	\end{align*}
	\item 
	\begin{align*}
	^{\perp}N 
	&= \{x \in X: \text{for each $\phi \in N$, $\phi(x) = 0$}\} \\
	&= \bigcap_{\phi \in N} \{x \in X: \phi(x) = 0\} \\
	&= \bigcap_{\phi \in N} \ker \phi
	\end{align*}
	\end{enumerate}
	\end{proof}
	
	\begin{ex} \lex{ex:banach:duality:0012}
	Let $X$ be a normed vector space, $M \subset X$ and $N \subset X^*$. Then 
	\begin{enumerate}
	\item $M^{\perp}$ is weak-* closed
	\item $^{\perp} N$ is closed
	\end{enumerate}
	\end{ex}
	
	\begin{proof}\
	\begin{enumerate}
	\item Let $(\phi_n)_{n \in \N} \subset M^{\perp}$ and $\phi \in X^*$. Suppose that $\phi_n \conv{w^*} \phi$. Then for each $x \in X$, $\phi_n(x) \rightarrow \phi(x)$. Let $x \in M$. By definition, for each $n \in \N$, $\phi_n(x) = 0$. Thus $\phi_n(x) \rightarrow 0$ which implies that $\phi(x) = 0$ and $\phi \in \ker \hat{x}$. Since $x \in M$ is arbitrary, 
	\begin{align*}
	\phi 
	&\in \bigcap_{x \in M} \ker \hat{x} \\
	&= M^{\perp}
	\end{align*}
	\item Let $(x_n)_{n \in \N} \subset {^{\perp} N}$ and $x \in X$. Suppose that $x_n \rightarrow x$. Let $\phi \in N$. Continuity implies that $\phi(x_n) \rightarrow \phi(x)$. By definition, for each $n \in \N$, $\phi(x_n) = 0$. Thus $\phi(x_n) \rightarrow 0$ which implies that $\phi(x) = 0$. So $x \in \ker \phi$. Since $\phi \in N$ is arbitrary, 
	\begin{align*}
	x 
	&\in \bigcap_{\phi \in N} \ker \phi \\
	&= {^{\perp}N}
	\end{align*}
	\end{enumerate}
	\end{proof}
	
	\begin{ex} \lex{ex:banach:duality:0013}
	Let $X$ be a normed vector space, $M \subset X$ and $N \subset X^*$. Then 
	\begin{enumerate}
	\item $^{\perp}(M^{\perp}) = \cl M$, i.e. the norm closure of $M$
	\item $({^{\perp}N})^{\perp} = \text{cl}_{w^*}(N)$, i.e. the weak-* closure of $N$.
	\end{enumerate}
	\end{ex}
	
	\begin{proof}\
	\begin{enumerate}
	\item Let $x \in M$, then by definition, for each $\phi \in M^{\perp}$, $\phi(x) = 0$. Again by definition, $x \in {^{\perp}(M^{\perp})}$. So $M \subset {^{\perp}(M^{\perp})}$. Since ${^{\perp}(M^{\perp})}$ is closed, $\cl M \subset {^{\perp}(M^{\perp})}$. For the sake of contradiction, suppose that ${^{\perp}(M^{\perp})} \not \subset \cl M$. Then there exists $x \in {^{\perp}(M^{\perp})}$ such that $x \not \in \cl M$. \rex{55017} implies that there exists $\phi \in X^*$ such that $\phi|_{\cl M} = 0$, $\|\phi\| = 1$ and $\phi(x) = \|x + \cl M\| > 0$. By definition, $\phi \in M^{\perp}$. Since $\phi(x) \neq 0$, we have that $x \not \in {^{\perp}(M^{\perp})}$. This is a contradiction and so ${^{\perp}(M^{\perp})} \subset \cl M$.
	\item 
	\end{enumerate}
	\end{proof}
	
	
	
	\begin{ex} \lex{ex:banach:duality:0014} \tbf{Banach-Alaoglu Theorem:} \\
		Let $X$ be a normed vector space. Then $\bar{B}(0,1)$ is $w^*$-compact. 
	\end{ex}

	\begin{proof}
		For $x \in X$, define $D_x \subset \C$ by $D_x \defeq \bar{B}_{\C}(0, \|x\|)$. Then for each $x \in X$, $D_x$ is compact. Define $D \subset \C^X$ by $D \defeq \prod\limits_{x \in X} D_x$. Tychonoff's theorem \rex{ex:topology:compactness_prod_spaces:00018} implies that $D$ is compact. Let $\phi \in \bar{B}_{X^*}(0,1)$. Then for each $x \in X$, $|\phi(x)| \leq \|x\|$. Hence $\phi \in D$. Since $\phi \in \bar{B}_{X^*}(0,1)$ is arbitrary, we have that $\bar{B}_{X^*}(0,1) \subset D$. Let $(\phi_{\al})_{\al \in A} \subset \bar{B}_{X^*}(0,1)$ and $\phi \in D$. Suppose that $\phi_{\al} \convt{p.w.} \phi$. Then for each $x,y \in X$ and $\lam \in \C$,  
		\begin{align*}
			\phi(x + \lam y)
			& = \lim_{\al \in A} \phi_{\al}(x+ \lam y) \\
			& = \lim_{\al \in A} [\phi_{\al}(x) + \lam \phi_{\al} (y)] \\
			& = \lim_{\al \in A} \phi_{\al}(x) + \lam \lim_{\al \in A} \phi_{\al}(y) \\
			& = \phi(x) + \lam \phi(y).
		\end{align*}
		So $\phi$ is linear. Let $x \in X$. Suppose that $\|x\|=1$. Since $|\cdot|: \C \rightarrow \C$ is continuous, we have that $|\phi_{\al}(x)| \rightarrow |\phi(x)|$. Since $\|x\| = 1$, we have that 
		\begin{align*}
			|\phi_{\al}(x)| 
			& \leq \|\phi_{\al}\|\|x\| \\
			& \leq \|x\| \\
			& = 1.
		\end{align*}
		Since $x \in X$ with $\|x\|=1$ is arbitrary, we have that 
		\begin{align*}
			\|\phi\|
			& = \sup\limits_{\|x\|=1} |\phi(x)| \\
			& \leq 1.
		\end{align*}
		Thus $\phi \in \bar{B}_{X^*}(0,1)$. Since $(\phi_{\al})_{\al \in A} \subset \bar{B}_{X^*}(0,1)$ and $\phi \in D$ with $\phi_{\al} \convt{p.w.} \phi$ are arbitrary, we have that for each $(\phi_{\al})_{\al \in A} \subset \bar{B}_{X^*}(0,1)$ and $\phi \in D$, $\phi_{\al \in A} \convt{p.w.} \phi$ implies that $\phi \in \bar{B}_{X^*}(0,1)$. Hence $\bar{B}_{X^*}(0,1)$ is closed in $D$. Since $D$ is compact, $\bar{B}_{X^*}(0,1)$ is compact. Since for each $(\phi_{\al})_{\al \in A} \subset \bar{B}_{X^*}(0,1)$ and $\phi \in \bar{B}_{X^*}(0,1)$, $\phi_{\al} \conv{w^*} \phi$ iff $\phi_{\al} \convt{p.w.} \phi$, we have that $\bar{B}_{X^*}(0,1)$ in weak-$*$ is compact. 
 	\end{proof}







	
	
	
	






\newpage
\section{Compact Operators}


\begin{defn}

\end{defn}























	\newpage
	\section{Multilinear Maps}	
	
	\begin{defn} \ld{69001}
		Let $X_1, \cdots, X_n, Y$ be normed vector spaces and $T : \prod\limits_{j=1}^n X_j \rightarrow Y$. Suppose that $T$ is multilinear. Then $T$ is said to be \tbf{bounded} if there exists $C \geq 0$ such that for each $x_1, \cdots, x_n \in X$, $$\|T(x_1, \cdots, x_n)\| \leq C \|x_1\| \cdots \|x_n\|$$
		We define $$L^n (X_1, \dots, X_n; Y) = \bigg\{T : \prod\limits_{j=1}^n X_j \rightarrow Y: T \text{ is multilinear and bounded}\bigg \}$$ 
		If $X_1 = \cdots = X_n = X$, we write $L^n(X;Y)$ in place of $L^n (X, \dots, X; Y) $. 
	\end{defn}
	
	\begin{note}
	For the remainder of this section we will primarily consider $L^2(X_1, X_2; Y)$ to avoid notational clutter, but all results immediately generalize to $L^n(X_1, \ldots, X_n;Y)$
	\end{note}
	
	\begin{ex} \lex{69002}
	Let $X_1, X_2$ and $Y$ be normed vector spaces and $T: X_1 \times X_2 \rightarrow Y$ bilinear. Then the following are equivalent:
	\begin{enumerate}
			\item $T$ is continuous
			\item $T$ is continuous at $(0,0)$
			\item $T$ is bounded
		\end{enumerate}
	\end{ex}
	
	\begin{proof}\
		\begin{itemize}
		\item $(1) \implies (2)$:\\
		Trivial
		\item  $(2) \implies (3)$:\\ 
		Suppose that $T$ is continuous at $(0, 0)$. For the sake of contradiction, suppose that $T$ is not bounded. Then for each $C \geq 0$, there exist $(x_1, x_2) \in X_1 \times X_2$ such that $\|T(x_1, x_2)\| > C\|x_1\| \|x_2\|$. Hence there exist $(a_n)_{n \in \N} \subset X_1$ and $(b_n)_{n \in \N} \subset X_2$ such that for each $n \in \N$, $ \| T(a_n, b_n) \| > n^2 \|a_n\| \|b_n\|$. Hence for each $n \in \N$, $\|a_n\|$, $\|b_n\| > 0$. Define $$(a'_n)_{n \in \N} \subset X_1$$ and $(b'_n)_{n \in \N} \subset X_2$ by $a_n' = \frac{a_n}{n\|a_n\|}$ and $b_n' = \frac{b_n}{n\|b_n\|}$. Then $(a_n', b_n') \rightarrow (0,0)$. Continuiuty implies that $T(a_n',b_n') \rightarrow 0$. By construction, for each $n \in \N$,
		\begin{align*}
		\|T(a_n',b_n')\| 
		&= \frac{1}{n^2 \|a_n\| \|b_n\|} T(a_n, b_n) \\
		& > \frac{n^2 \|a_n\| \|b_n\|}{n^2 \|a_n\| \|b_n\|} \\
		&= 1
		\end{align*}
		which is a contradiction. So $T$ is bounded.
		\item  $(3) \implies (1)$:\\ 
		Suppose that $T$ is bounded. Then there exists $C > 0$ such that for each $(x_1, x_2) \in X_1 \times X_2$, $\| T(x_1, x_2) \| \leq C\|x_1\| \|x_2\|$. Let $(a, b) \in X_1 \times X_2$ and $(a_n, b_n)_{n \in \N} \subset X_1 \times X_2$. Suppose that $(a_n, b_n) \rightarrow (a,b)$. Then $a_n \rightarrow a$, $b_n \rightarrow b$ and $(a_n)_{n \in \N}$, $(b_n)_{n \in \N}$ are bounded. So there exists $B \geq 0$ such that for each $n \in \N$ $\|b_n\| \leq B$. Hence 
		\begin{align*}
		\|T(a_n,b_n) - T(a,b) \|
		&= \|T(a_n,b_n) - T(a, b_n) + T(a, b_n) - T(a,b) \| \\
		& \leq \|T(a_n,b_n) - T(a, b_n) \| + \|T(a, b_n) - T(a,b) \| \\
		&= \|T(a_n - a,b_n) \| + \|T(a, b_n - b)\| \\
		& \leq C(\|a_n - a\| \|b_n\| + \|a\|\|b_n - b\|) \\
		& \leq C(\|a_n - a\| B + \|a\|\|b_n - b\|) \\
		& \rightarrow 0
		\end{align*}
		Thus $T$ is continuous. 
		\end{itemize}
	\end{proof}
	
	\begin{defn} \ld{69005}
	Let $X_1, X_2$ and $Y$ be normed vector spaces and $T \in L^2(X_1, X_2;Y)$. We define the \tbf{operator norm} on $L^2(X_1, X_2;Y)$, denoted $\|\cdot\|: L^2(X_1, X_2; Y) \rightarrow [0, \infty)$, by  $$\|T\| =  \inf \{C \geq 0: \text{for each } (x_1, x_2) \in X_1 \times X_2\text{, } \|T(x_1, x_2) \|\leq C\|x_1\|\|x_2\|\}$$
	\end{defn}
	
	\begin{ex} \lex{42006}
		Let $X_1, X_2$ and $Y$ be normed vector spaces. If $X_1 \neq \{0\}$ and $ X_2 \neq \{0\}$, then the operater norm on $L^2(X_1, X_2; Y)$ is given by: 
		\begin{enumerate}
			\item $\|T\| = \sup\limits_{\|x_1\|=1 ,\|x_2\| = 1 }\|T(x_1, x_2)\|$
			\item $\|T\| = \sup\limits_{x_1 \neq 0, x_2 \neq 0 } \|x_1\|^{-1} \|x_2\|^{-1}\|T(x_1, x_2)\|$
			\item $\|T\| = \inf \{C \geq 0: \text{for each } (x_1, x_2) \in X_1 \times X_2\text{, } \|T(x_1, x_2) \|\leq C\|x_1\|\|x_2\|\}$
		\end{enumerate}
	\end{ex}
	
	\begin{proof} Since $X_1 \neq \{0\}$ and $ X_2 \neq \{0\}$, the supremums in (1) and (2) are well defined. Let $T \in L^2(X_1, X_2; Y)$. Bilinearity of $T$ implies that the sets over which the supremums are taken in (1) and (2) are the same. So (1) and (2) are equal.\\
		Now, set 
		$$M = \sup\limits_{\|x_1\|=1 ,\|x_2\| = 1 }\|T(x_1, x_2)\|$$ 
		and 
		$$m = \inf \{C \geq 0: \text{for each } (x_1, x_2) \in X_1 \times X_2\text{, } \|T(x_1, x_2) \|\leq C\|x_1\|\|x_2\|\}$$ 
		Let $(x_1,x_2) \in X_1 \times X_2$. If $\|x_1 \|=0$ or $\|x_2 \|=0$, then $T(x_1, x_2) = 0$ and $\|T(x_1, x_2) \|\leq M \|x_1 \| \|x_2\|$. Suppose that $\|x_1 \| \neq 0$ and $ \|x_2\|\neq 0$. Then 
		\begin{align*}
			\|T(x_1, x_2)\|
			&= \bigg(\big\|T(\|x_1\|^{-1} x_1, \|x_2\|^{-1} x_2 )\big\|\bigg)\|x_1 \| \|x_2\|\\
			& \leq M \|x_1\| \|x_2\|
		\end{align*}
		Hence $M \in \{C \geq 0: \text{ for each }x \in X\text{, } \|Tx \|\leq C \|x \|\}$ and $m \leq M$.
		Let $C \in \{C \geq 0: \text{for each } (x_1, x_2) \in X_1 \times X_2\text{, } \|T(x_1, x_2) \|\leq C\|x_1\|\|x_2\|\}$. Suppose that $\|x_1 \|=1$ and $\|x_2\| = 1$. Then $\| T (x_1, x_2) \| \leq C \|x_1 \| \|x_2 \|= C$. So $M \leq C$. Therefore $M \leq m$. So $M=m$ and the supremum in (1) is the same as the infimum in (3). 
	\end{proof}

	\begin{ex}
		Let $X_1, X_2$ and $Y$ be normed vector spaces and $T \in L(X_1, X_2;Y)$. Then for each $(x_1, x_2) \in X_1 \times X_2$, $\|T(x_1, x_2)\| \leq \|T\|\|x_1\|\|x_2\|$. 
	\end{ex}

	\begin{proof}
		Let $(x_1, x_2) \in X_1 \times X_2$. If $x_1 = 0$ or $x_2 = 0$, then 
		\begin{align*}
			\|T(x_1, x_2)\|
			& = \|0\| \\
			& = 0 \\
			& = \|T\|\|x_1\| \|x_2\| \\
		\end{align*}
		Suppose that $x_1 \neq 0$ and $x_2 \neq 0$. The previous exercise implies that 
		\begin{align*}
			\|T(x_1, x_2)\| 
			& = \|T(\|x_1\|^{-1}\|x_1\|, \|x_2\|^{-1}\|x_2\|)\| \|x_1\| \|x_2\| \\
			& \leq \bigg( \sup\limits_{\|x_1\|=1 ,\|x_2\| = 1 }\|T(x_1, x_2)\| \bigg)  \|x_1\| \|x_2\| \\
			& = \|T\| \|x_1\| \|x_2\|
		\end{align*}
	\end{proof}
	
	\begin{ex}
	Let $X_1, X_2$ and $Y$ be normed vector spaces. Then $L(X_1, X_2;Y)$ is a vector space and $\|\cdot\|: L^2(X_1, X_2; Y) \rightarrow [0, \infty)$ is a norm. 
	\end{ex}
	
	\begin{proof}
		Let $S,T \in L(X_1,X_2;Y)$ and $\lam \in \C$.
		\begin{itemize}
			\item It is clear that $S+T : X_1 \times X_2 \rightarrow Y$ is multilinear. For each $(x_1, x_2) \in X_1 \times X_2$,
			\begin{align*}
				\|(S+T)(x_1, x_2)\|
				& = \|S(x_1, x_2) + T(x_1, x_2)\| \\
				& \leq  \|S(x_1, x_2) \| + \|T(x_1, x_2)\| \\
				& \leq \|S\|\|x_1\| \|x_2\| + \|T\|\|x_1\| \|x_2\| \\
				& = (\|S\| + \|T\|)\|x_1\| \|x_2\|
			\end{align*}
			So $S+T \in L(X_1, X_2;Y)$ and $\|S+T\|\leq \|S \|+ \|T \|$.
			\item It is clear that $\lam T : X_1 \times X_2 \rightarrow Y$ is multilinear. For each $(x_1, x_2) \in X_1 \times X_2$,
			\begin{align*}
				\|(\lam T)(x_1, x_2)\| 
				& = \|\lam T(x_1, x_2)\| \\
				& = |\lam| \|T(x_1, x_2)\| \\
				& \leq |\lam| \|T\| \|x_1\| \|x_2\|
			\end{align*}
			So $\lam T \in L(X_1, X_2;Y)$ and $\|\lam T\| \leq |\lam| \|T\|$.
			\item Suppose that $\|T\| = 0$. Let $(x_1, x_2) \in X_1 \times X_2$. If $x_1 = 0$ or $x_2 = 0$, then $T(x_1,x_2) = 0$. Suppose that $x_1 \neq 0$ and $x_2 \neq 0$. Then 
			\begin{align*}
				\|T(x_1, x_2)\|
				& \leq  \|T\|\|x_1\|\|x_2\| \\
				& = 0
			\end{align*}
			So $T(x_1, x_2) = 0$. Since $(x_1, x_2) \in X_1 \times X_2$ is arbitrary, $T = 0$.
		\end{itemize}
		Therefore $L(X_1, X_2; Y)$ is a vector space and $\|\cdot\|: L(X_1, X_2; Y) \rightarrow \Rg$ is a norm.
	\end{proof}

	\begin{note}
	Let $X, Y, Z$ be sets. We recall the definition of $\cur: Y^{X \times Y} \rightarrow (Z^Y)^X$ from \rd{def:set_theory:products:0009}.
	\end{note}
	
	\begin{ex} \lex{69004}
	Let $X_1, X_2, Y$ be normed vector spaces. Then 
	\begin{enumerate}
		\item $\cur: (X_1 \times X_2)^Y \rightarrow (Y^{X_2})^{X_1}$ is linear
		\item $\cur|_{L(X_1, X_2;Y)}: L(X_1, X_2;Y) \rightarrow L(X_1; L(X_2;Y))$
		\item $\cur|_{L(X_1, X_2;Y)}$ is an isometry
		\item $\cur|_{L(X_1, X_2;Y)}$ is surjective
		\item $\cur|_{L(X_1, X_2;Y)}: L(X_1, X_2;Y) \rightarrow L(X_1; L(X_2;Y))$ is an isometric isomorphism.
	\end{enumerate}
	\end{ex}
	
	\begin{proof}\
	\begin{enumerate}
		\item Let $S, T \in (X_1 \times X_2)^Y$, $\lam \in \C$ and $(x_1, x_2) \in X_1 \times X_2$. Then 
		\begin{align*}
			\cur(S + \lam T)(x_1)(x_2)
			& = (S  + \lam T)(x_1, x_2) \\
			& = S(x_1, x_2) + \lam T (x_1, x_2) \\
			& = \cur(S)(x_1)(x_2) + \lam \cur(T)(x_1)(x_2) \\
			& \cur(S)(x_1)(x_2) + \lam \cur(T)(x_1)(x_2) \\
			& = [\cur(S) + \lam \cur(T)](x_1)(x_2)
		\end{align*}
		Since $(x_1, x_2) \in X_1 \times X_2$ is arbitrary, $\cur(S + \lam T) = \cur(S) + \lam \cur(T)$. Since  $S, T \in (X_1 \times X_2)^Y$ and $\lam \in \C$ are arbitrary, $\cur: (X_1 \times X_2)^Y \rightarrow  (Y^{X_2})^{X_1}$ is linear.
		\item Let $T \in L(X_1, X_2; Y)$ and $x_1 \in X_1$. Since $T$ is bilinear, for each $u,v \in X_2$ and $\lam \in \C$,  
		\begin{align*}
			\cur(T)(x_1)(u + \lam v)
			& = T(x_1, u + \lam b) \\
			& = T(x_1, u) + \lam T(x_1, v) \\
			& = \cur(T)(x_1)(u) +  \lam \cur(T)(x_1)(v) \\
		\end{align*}	
		So for each $x_1 \in X_1$, $\cur(T)(x_1)$ is linear. Let $a, b \in X_1$, $\al \in \C$ and $x_2 \in X_2$. Then
		\begin{align*}
			\cur(T)(a + \al b)(x_2) 
			& = T(a + \al b, x_2) \\
			& = T(a, x_2) + \al T(b, x_2) \\
			& = \cur(T)(a)(x_2) +  \al  \cur(T)(b)(x_2) \\
			& = [\cur(T)(a) + \al \cur(T)(b)](x_2) 
		\end{align*}
		Since $x_2 \in X_2$ is arbitrary, $\cur(T)(a + \al b) = \cur(T)(a) + \al \cur(T)(b)$. Since $a,b \in X_1$ and $\al \in \C$ are arbitrary, $\cur(T)$ is linear. 
		Let $(x_1, x_2) \in X_1 \times X_2$. Then
		\begin{align*}
			\| \cur(T)(x_1)(x_2) \|
			&= \|T(x_1, x_2) \| \\
			&\leq (\|T\| \|x_1\|) \|x_2\| 
		\end{align*}
		So $\cur(T)(x_1) \in L(X_2,Y)$ and $\|\cur(T)(x_1)\| \leq \|T\|\|x_1\|$. Since $x_1 \in X_1$ is arbitrary, $\cur(T) \in  L(X_1; L(X_2;Y))$ and $\|\cur T\| \leq \|T\|$. Since $T \in L(X_1, X_2;Y)$ is arbitrary, $\cur(L(X_1, X_2; Y)) \subset L(X_1; L(X_2; Y))$. Therefore $\cur|_{L(X_1, X_2;Y)}: L(X_1, X_2;Y) \rightarrow L(X_1; L(X_2;Y))$.
		\item Let $T \in L(X_1, X_2; Y)$. A previous exercise and an exercise in the section on real numbers imply that 
		\begin{align*}
			\|\cur(T)\|
			& = \sup_{\|x_1\| = 1} \|T(x_1)\| \\
			& = \sup_{\|x_1\| = 1} \bigg[ \sup_{\|x_2\| = 1}\|T(x_1)(x_2)\| \bigg] \\
			& = \sup_{\|x_1\| = 1, \|x_2\| = 2} \|T(x_1)(x_2)\| \\
			& = \|T\|
		\end{align*}
		So $\cur|_{L(X_1, X_2;Y)}: L(X_1, X_2;Y) \rightarrow L(X_1; L(X_2;Y))$ is an isometry. 
		\item Let $T \in L(X_1; L(X_2; Y))$. Define $S: X_1 \times X_2 \rightarrow Y$ by $S(x_1, x_2) = T(x_1)(x_2)$. It is straightforward to show that $S$ is bilinear and for each $(x_1, x_2) \in X_1 \times X_2$, 
		\begin{align*}
			\|S(x_1, x_2)\|
			& = \|T(x_1)(x_2)\| \\
			& \leq \|T(x_1)\|\|x_2\| \\
			& \leq \|T\| \|x_1\| \|x_2\| 
		\end{align*} 
		So $S \in L(X_1, X_2; Y)$ and $\|S\| \leq \|T\|$. By construction $\cur(S) = T$. Since $T \in L(X_1; L(X_2; Y))$ is arbitrary, we have that for each $T \in L(X_1; L(X_2; Y))$, there exists $S \in L(X_1, X_2; Y)$ such that $T = \cur(S)$. Hence $\cur|_{L(X_1, X_2;Y)}$ is surjective.
		\item  Since $\cur|_{L(X_1, X_2;Y)}$ is an isometry, $\cur|_{L(X_1, X_2;Y)}$ is injective. From the previous part, we know that $\cur|_{L(X_1, X_2;Y)}$ is surjective. Hence $\cur|_{L(X_1, X_2;Y)}$ is a bijection. The first part implies that $\cur|_{L(X_1, X_2;Y)}$ is linear. Hence $\cur|_{L(X_1, X_2;Y)}: L(X_1, X_2;Y) \rightarrow L(X_1; L(X_2;Y))$ is an isometric isomorphism.
	\end{enumerate}
	\end{proof}	

	\begin{ex}
		Let $X_1, X_2, Y$ be normed vector spaces. If $Y$ is complete, then $L(X_1, X_2; Y)$ is complete.
	\end{ex}
	
	\begin{proof}
		Suppose that $Y$ is complete. Then $L(X_1; L(X_2; Y))$ is complete. Since $L(X_1; L(X_2; Y))$ is isometrically isomorphic to $L(X_1, X_2; Y)$, we have that $L(X_1, X_2; Y)$ is complete. 
	\end{proof}
	
	\begin{defn}
	Let $X_1, X_2$ be normed vector spaces, $\phi_1 \in X_1^*$ and $\phi_2 \in X_2^*$. Define $\phi_1 \otimes \phi_2: X_1 \times X_2$ by $\phi_1 \otimes \phi_2(x_1, x_2) = \phi_1(x_1)\phi_2(x_2)$. 
	\end{defn}
	
	\begin{ex}
	Let $X_1, X_2$ be normed vector spaces, $\phi_1 \in X_1^*$ and $\phi_2 \in X_2^*$. Then $\phi_1 \otimes \phi_2 \in L^2(X_1, X_2; \C)$.
	\end{ex}
	
	\begin{proof}
	Clear.
	\end{proof}
	
	\begin{ex}
	Let $X_1, X_2$ be normed vector spaces and $(x_1, x_2) \in X_1 \times X_2$. If for each $\phi_1 \in X_1^*$ and $\phi_2 \in X_2^*$, $\phi_1 \otimes \phi_2 (x_1, x_2) = 0$, then $x_1 = 0$ or $x_2 = 0$. 
	\end{ex}
	
	\begin{proof}
	Suppose that $x_1 \neq 0$ and $x_2 \neq 0$. The previous section implies that there exist $\phi_1 \in X_1^*$ and $\phi_2 \in X_2^*$ such that $\phi_1(x_1) = \|x_1\| \neq 0$ and $\phi_2(x_2) = \|x_2\| \neq 0$. Then 
	\begin{align*}
	\phi_1 \otimes \phi_2 (x_1, x_2) 
	& = \phi_1(x_1) \phi_2(x_2) \\
	& \neq 0
	\end{align*}
	\end{proof}









































\newpage
\section{Tensor Products of Banach Spaces}

\tcr{(cite "intro to tensor products of banach spaces by Ryan")}


\section{Injective Tensor Product}

\begin{defn}
	Let $X, Y$ be vector spaces, $x \in X$ and $y \in Y$. We define $x \otimes_{\ep} y: X^* \times Y^* \rightarrow  \K$ by $x \otimes_{\ep} y(\phi, \psi) \defeq \phi(x) \psi(y)$.  
\end{defn}


\begin{ex}
	Let $X, Y$ be vector spaces, $x \in X$ and $y \in Y$. Then $x \otimes_{\ep} y \in L^2(X^*, Y^*; \K)$. 
\end{ex}

\begin{proof}
	Let $\phi_1, \phi_2 \in X^*$, $\psi \in Y^*$ and $\lam \in \K$. Then 
	\begin{align*}
		x \otimes_{\ep} y(\phi_1 + \lam \phi_2, \psi) 
		& = [\phi_1 + \lam \phi_2 (x)] \psi(y) \\
		& = \phi_1(x)\psi(y) + \lam \phi_2(x)\psi(y) \\
		& = x \otimes_{\ep} y(\phi_1, \psi) + \lam x \otimes_{\ep} y(\phi_2, \psi)
	\end{align*}
	Since $\phi_1, \phi_2 \in X^*$, $\psi \in Y^*$ and $\lam \in \K$ are arbitrary, we have that for each $\psi \in Y^*$, $x \otimes_{\ep} y(\cdot, \psi)$ is linear. Similarly for each $\phi \in X^*$, $x \otimes_{\ep} y(\phi, \cdot)$ is linear. Hence $x \otimes_{\ep} y$ is bilinear and $x \otimes_{\ep} y \in L^2(X^*, Y^*; \K)$. 
\end{proof}

\begin{defn}
	Let $X, Y$ be vector spaces. We define  
	\begin{itemize}
		\item the \tbf{injective tensor product of $X$ and $Y$}, denoted $X \otimes Y \subset L^2(X^*, Y^*; \K)$, by 
		$$X \otimes_{\ep} Y \defeq \spn(\text{$x \otimes_{\ep} y: x \in X$ and $y \in Y$}),$$
		\item the \tbf{injective tensor map}, denoted $\otimes_{\ep}: X \times Y \rightarrow X \otimes Y$, by $\otimes_{\ep}(x, y) \defeq x \otimes_{\ep} y$.
	\end{itemize}
\end{defn}

\begin{ex}
	Let $X,Y$ be vector spaces, $(x_j)_{j=1}^n \subset X$ and $(y_j)_{j=1}^n \subset Y$. The following are equivalent:
	\begin{enumerate}
		\item $\sum\limits_{j=1}^n  x_j \otimes y_j = 0$
		\item for each $\phi \in X^*$ and $\psi \in Y^*$, $\sum\limits_{j=1}^n  \phi(x_j) \psi(y_j) = 0$
		\item for each $\phi \in X^*$, $\sum\limits_{j=1}^n  \phi(x_j)  y_j = 0$
		\item for each $\psi \in Y^*$, $\sum\limits_{j=1}^n  \psi(y_j) x_j = 0$
	\end{enumerate}
\end{ex}

\begin{proof}\
	\begin{enumerate}
		\item $(1) \implies (2):$ \\
		Suppose that $\sum\limits_{j=1}^n x_j \otimes y_j = 0$. Let $\phi \in X^*$ and $\psi \in Y^*$. Then 
		\begin{align*}
			\sum\limits_{j=1}^n  \phi(x_j) \psi(y_j)
			& = \phi \bigg( \sum\limits_{j=1}^n \psi(y_j) x_j \bigg) \\
			& = 
		\end{align*}
		\item 
		\item 
	\end{enumerate}
\end{proof}






















































\newpage
\section{Projective Tensor Product}

\begin{defn}
	Let $X, Y$ be vector spaces, $x \in X$ and $y \in Y$. We define $x \otimes_{\pi} y: L^2(X, Y; \K) \rightarrow  \K$ by $x \otimes_{\pi} y(\phi, \psi) \defeq \phi(x) \psi(y)$.  
\end{defn}

\begin{ex}
	Let $X, Y$ be vector spaces, $x \in X$ and $y \in Y$. Then $x \otimes_{\pi} y \in L^2(X, Y; \K)^*$. 
\end{ex}

\begin{proof}
	Let $A, B \in L^2(X, Y; \K)$ and $\lam \in \K$. Then 
	\begin{align*}
		x \otimes_{\pi} y(A + \lam B) 
		& = (A + \lam B) (x,y) \\
		& = A(x,y) + \lam B(x,y) \\
		& = x \otimes_{\pi} y(A) + \lam x \otimes_{\pi} y(B)
	\end{align*}
	and 
	\begin{align*}
		|x \otimes_{\pi} y (A)|
		& = |A(x,y)| \\
		& \leq \|x\| \|y\| \|A\| 
	\end{align*}
	Since $A, B \in L^2(X, Y; \K)$ and $\lam \in \K$ are arbitrary, we have that $x \otimes_{\pi} y$ is linear and $\|x \otimes_{\pi} y\| \leq \|x\| \|y\|$. Hence $x \otimes_{\pi} y \in L^2(X, Y; \K)^*$ 
\end{proof}

\begin{defn}
	Let $X, Y$ be vector spaces. We define  
	\begin{itemize}
		\item the \tbf{projective tensor product of $X$ and $Y$}, denoted $X \otimes Y \subset L^2(X^*, Y^*; \K)$, by 
		$$X \otimes_{\pi} Y \defeq \spn(\text{$x \otimes_{\pi} y: x \in X$ and $y \in Y$}),$$
		\item the \tbf{projective tensor map}, denoted $\otimes_{\pi}: X \times Y \rightarrow X \otimes Y$, by $\otimes(x, y) \defeq x \otimes_{\pi} y$.
	\end{itemize}
\end{defn}

\begin{ex}
	Let $X,Y$ be vector spaces, $(x_j)_{j=1}^n \subset X$ and $(y_j)_{j=1}^n \subset Y$. The following are equivalent:
	\begin{enumerate}
		\item $\sum\limits_{j=1}^n  x_j \otimes y_j = 0$
		\item for each $\phi \in X^*$ and $\psi \in Y^*$, $\sum\limits_{j=1}^n  \phi(x_j) \psi(y_j) = 0$
		\item for each $\phi \in X^*$, $\sum\limits_{j=1}^n  \phi(x_j)  y_j = 0$
		\item for each $\psi \in Y^*$, $\sum\limits_{j=1}^n  \psi(y_j) x_j = 0$
	\end{enumerate}
	\tbf{Hint:} For $(4) \implies (1)$, set $E \defeq \spn (x_j)_{j=1}^n$, $F \defeq \spn(y_j)_{j=1}^n$ and define $B \in L^2(E,F;\K)^*$ by $B \defeq A|_{E \times F}$. Use the fact that $L^2(E, F; \K) = E^* \otimes_{\ep} F^*$ \tcr{make this an exercise in the section on multilinear maps}.
\end{ex}

\begin{proof}\
	\begin{enumerate}
		\item $(1) \implies (2):$ \\
		Suppose that $\sum\limits_{j=1}^n x_j \otimes_{\pi} y_j = 0$. Let $\phi \in X^*$ and $\psi \in Y^*$. Define $B_{\phi, \psi} \in L^2(X,Y;\K)$ by $B_{\phi, \psi}(x,y) \defeq \phi(x)\psi(y)$. Then 
		\begin{align*}
			\sum\limits_{j=1}^n  \phi(x_j) \psi(y_j)
			& = \sum\limits_{j=1}^n  B_{\phi, \psi}(x_j,y_j) \\
			& = \sum\limits_{j=1}^n  x_j \otimes_{\pi} y_j(B_{\phi, \psi}) \\
			& = \bigg[ \sum\limits_{j=1}^n   x_j \otimes_{\pi} y_j \bigg] (B_{\phi, \psi}) \\
			& = u(B_{\phi, \psi}) \\
			& = 0
		\end{align*}
		Since $\phi \in X^*$ and $\psi \in Y^*$ are arbitrary, for each $\phi \in X^*$ and $\psi \in Y^*$, $\sum\limits_{j=1}^n  \phi(x_j) \psi(y_j) = 0$. 
		\item $(2) \implies (3):$ \\
		Suppose that for each $\phi \in X^*$ and $\psi \in Y^*$, $\sum\limits_{j=1}^n  \phi(x_j) \psi(y_j) = 0$. Let $\phi \in X^*$ and $\psi \in Y^*$. Then  
		\begin{align*}
			\psi \bigg( \sum\limits_{j=1}^n  \phi(x_j)  y_j \bigg)
			& = \sum\limits_{j=1}^n  \phi(x_j) \psi(y_j) \\
			& = 0.
		\end{align*}
		Since $\psi \in Y^*$ is arbitrary, \rex{55019} implies that $\sum\limits_{j=1}^n  \phi(x_j)  y_j = 0$. Since $\phi \in X^*$ is arbitrary, we have that for each $\phi \in X^*$, $\sum\limits_{j=1}^n  \phi(x_j)  y_j = 0$.
		\item $(3) \implies (4)$: \\
		Suppose that for each $\phi \in X^*$, $\sum\limits_{j=1}^n  \phi(x_j)  y_j = 0$. Let $\psi \in Y^*$. Then 
		\begin{align*}
			\phi \bigg( \sum\limits_{j=1}^n  \psi(y_j)  x_j \bigg)
			& = \sum\limits_{j=1}^n  \phi(x_j) \psi(y_j) \\
			& = \psi \bigg( \sum\limits_{j=1}^n  \phi(x_j)  y_j \bigg) \\
			& = \psi(0) \\
			& = 0.
		\end{align*}
		Since $\phi \in X^*$ is arbitrary, we have that $\sum\limits_{j=1}^n  \psi(y_j)  x_j = 0$. Since $\psi \in Y^*$ is arbitrary, we have that for each $\psi \in Y^*$, $\sum\limits_{j=1}^n  \psi(y_j) x_j = 0$.
		\item $(4) \implies (1)$: \\
		Suppose that for each $\psi \in Y^*$, $\sum\limits_{j=1}^n  \psi(y_j) x_j = 0$. Set $E \defeq \spn(x_j:j \in [n])$, $F \defeq \spn(y_j:j \in [n])$ and define $B \in L^2(E , F; \K)$ by $B \defeq A|_{E \times F}$. \tcr{make exercise about $\phi_{0,k} \otimes_{\ep} \psi_{0,l}: k,l$ is a basis when $(\phi_{0,k})_{k=1}^n$ and $(\psi_{0,l})_{l=1}^n$ are bases for $E^*$, which can be obtained from a basis for $E$ and $F$}, there exist $(\phi_{0,j})_{j=1}^n \subset E^*$ and $(\psi_{0,j})_{j=1}^n \subset F^*$ such that $B = \sum_{k=1}^n \phi_{0,k} \otimes_{\ep} \psi_{0,k}$. Let $j \in [n]$. \rex{55016} implies that there exist $(\phi_j)_{j=1}^n \subset X^*$ and $(\psi_j)_{j=1}^n \subset Y^*$ such that for each $j \in [n]$, $\|\phi_j\| = \|\phi_{0,j}\|$, $\|\psi_j\| = \|\psi_{0,j}\|$, $\phi_j|_E = \phi_{0, j}$ and $\psi_j|_F = \psi_{0, j}$. Then
		\begin{align*}
			u(A)
			& = \sum_{j=1}^n x_j \otimes y_j(A) \\
			& = \sum_{j=1}^n A(x_j, y_j) \\
			& = \sum_{j=1}^n B(x_j, y_j) \\
			& = \sum_{j = 1}^n \bigg[ \sum_{k=1}^n \phi_{0,k} \otimes_{\ep} \psi_{0,k}(x_j, y_j) \bigg] \\
			& = \sum_{j = 1}^n \bigg[ \sum_{k=1}^n \phi_{0,k}(x_j)\psi_{0,k}(y_j) \bigg] \\
			& = \sum_{j = 1}^n \bigg[ \sum_{k=1}^n \phi_k(x_j)\psi_k(y_j) \bigg] \\
			& = \sum_{k = 1}^n \bigg[ \sum_{j=1}^n \phi_k(x_j)\psi_k(y_j) \bigg] \\
			& = \sum_{k=1}^n \bigg[ \phi_k \bigg( \sum_{j = 1}^n \psi_k(y_j) x_j \bigg) \bigg] \\ 
			& = \sum_{k=1}^n \phi_k(0) \\
			& = 0
		\end{align*}
		Since $A \in L^2(X,Y;\K)$ is arbitrary, we have that for each $A \in L^2(X,Y;\K)$, $u(A) = 0$. Hence $u = 0$.
	\end{enumerate}
\end{proof}

\begin{defn} \ld{def:banach_spaces:projective_norm:0001}
	Let $(X, \| \cdot \|_X), (Y, \| \cdot \|_Y)$ be Banach spaces. We define the \tbf{projective norm on $X \otimes Y$}, denoted $\|\cdot\|_{\pi}: X \otimes Y \rightarrow \Rg$, by 
	$$\|u \|_{\pi} \defeq \inf \bigg\{ \text{$\sum_{j=1}^n \|x_j\|_X \|y_j\|_Y : (x_j)_{j=1}^n \subset X, (y_j)_{j=1}^n \subset Y $ and $u = \sum_{j=1}^n x_j \otimes y_j$ }\bigg\}$$ 
\end{defn}

\begin{ex}
	Let $X, Y$ be Banach spaces. Then 
	\begin{enumerate}
		\item $\| \cdot \|_{\pi}$ is a norm on $X \otimes Y$
		\item for each $x \in X$ and $y \in Y$, $\|x \otimes_{\pi} y \|_{\pi} = \|x\|\|y\|$.
	\end{enumerate}
\end{ex}

\begin{proof}\
	\begin{enumerate}
		\item 
		\begin{enumerate}
			\item Let $u \in X \otimes Y$. Suppose that $\|u\|_{\pi} = 0$. Set 
			$$V_u \defeq \bigg\{ \text{$\sum_{j=1}^n \|x_j\| \|y_j\| : (x_j)_{j=1}^n \subset X, (y_j)_{j=1}^n \subset Y $ and $u = \sum_{j=1}^n x_j \otimes y_j$ }\bigg\}$$ 
			Let $\phi \in X^*$, $\psi \in Y^*$ and $\ep > 0$. Then there exist $(x_j)_{j=1}^n \subset X$ and $(y_j)_{j=1}^n \subset Y$ such that $u = \sum\limits_{j=1}^n x_j \otimes y_j$ and  
			\begin{align*}
				\sum\limits_{j=1}^n \|x_j\| \|y_j\| 
				& < \inf V_u + \ep/(\|\phi\|\|\psi\| + 1) \\
				& = \|u\|_{\pi} + \ep/(\|\phi\|\|\psi\| + 1) \\
				& = \ep/(\|\phi\|\|\psi\| + 1).
			\end{align*}
			
			\begin{align*}
				|u(\phi, \psi)|
				& = \sum\limits_{j=1}^n |\phi(x_j) \psi(y_j)| \\
				& \leq \sum\limits_{j=1}^n \|\phi\| \|\psi\| \|x_j\| \|y_j\| \\
				& = \|\phi\| \|\psi\| \sum\limits_{j=1}^n  \|x_j\| \|y_j\| \\
				& < \|\phi\| \|\psi\|  \frac{\ep}{\|\phi\|\|\psi\| + 1} \\
				& \leq   \ep 
			\end{align*}
			Since $\ep > 0$ is arbitrary, we have that $u(\phi, \psi) = 0$. Since $\phi \in X^*$ and $\psi \in Y^*$ are arbitrary, we have that for each $\phi \in X^*$ and $\psi \in Y^*$, $u(\phi, \psi) = 0$. Therefore $u = 0$. 
			\item Let $u \in X \otimes Y $ and $\lam \in \K$. Set 
			$$V_u \defeq \bigg\{ \text{$\sum_{j=1}^n \|x_j\| \|y_j\| : (x_j)_{j=1}^n \subset X, (y_j)_{j=1}^n \subset Y $ and $u = \sum_{j=1}^n x_j \otimes y_j$ }\bigg\}$$ 
			and 
			$$V_{\lam u} \defeq \bigg\{ \text{$\sum_{j=1}^n \|x_j\| \|y_j\| : (x_j)_{j=1}^n \subset X, (y_j)_{j=1}^n \subset Y $ and $\lam u = \sum_{j=1}^n x_j \otimes y_j$ }\bigg\}.$$ 
			Let $\ep > 0$. Since $\|u\|_{\pi} = \inf V_u$, there exists $a \in V_u$ such that $a < \|u\|_{\pi} + \ep/(|\lam| +1)$. Therefore, there exists $(x_j)_{j=1}^n \subset X$ and $(y_j)_{j=1}^n \subset Y$ such that $u = \sum\limits_{j=1}^n x_j \otimes y_j$ and $a = \sum_{j=1}^n \|x_j\| \|y_j\| $. Then $\lam u = \sum\limits_{j=1}^n (\lam x_j) \otimes y_j $ and 
			\begin{align*}
				\|\lam u\|_{\pi}
				& = \inf V_{\lam u} \\
				& \leq \sum_{j=1}^n \|\lam x_j\| \|y_j\| \\
				& = |\lam| \sum_{j=1}^n \| x_j\| \|y_j\| \\
				& = |\lam| a \\
				& < |\lam| \bigg( \|u\|_{\pi} + \frac{\ep}{|\lam| +1} \bigg) \\
				& = |\lam| \|u\|_{\pi} + \ep \frac{|\lam|}{|\lam| + 1} \\
				& < |\lam| \|u\|_{\pi} + \ep
			\end{align*}
			Since $\ep > 0$ is arbitrary, we have that $\|\lam u\|_{\pi} \leq |\lam| \|u\|_{\pi}$.
			\begin{itemize}
				\item Suppose that $\lam = 0$. Then $\lam u = 0 = 0 \otimes 0$. Hence
				\begin{align*}
					0 \leq 
					& \|\lam u\|_{\pi} \\
					& = \inf V_{\lam u} \\
					& \leq \|0\| \|0\| \\
					& = 0
				\end{align*} 
				Thus 
				\begin{align*}
					\|\lam u\|_{\pi}
					& = 0 \\
					& = |\lam| \|u\|_{\pi}
				\end{align*}
			\item Suppose that $\lam \neq 0$. Then
			\begin{align*}
				\|u\|_{\pi}
				& = \|\lam^{-1} (\lam u)\|_{\pi} \\
				& \leq |\lam^{-1}|\|\lam u\|_{\pi} \\
				& = |\lam|^{-1} \|\lam u\|_{\pi}
			\end{align*}
			Hence $|\lam|\|u\|_{\pi} \leq \|\lam u\|_{\pi} $. Thus $ \|\lam u\|_{\pi} = |\lam|\|u\|_{\pi}$.
			\end{itemize}
			\item 
		\end{enumerate}
		\item 
	\end{enumerate}
\end{proof}














































	
	
	
	











	
	
	
	
	
	
	
	
	
	
	
	
	
	
	
	
	\newpage
	\chapter{Hilbert Spaces}
	
	\section{TODO}
	\begin{itemize}
		\item Express $V^* \cong \ol{V}$ where $\ol{V}$ is just $V$, but with $\lam * v  = \lam^* v$. so Rieze rep theorem reads $V \cong \ol{V^*}$ or $V \cong \ol{V}^*$
		\item discuss projection maps
		\item show internal direct sum isomorphic to external
		\item discuss quotient hilbert space?
		\item discus subspaces
	\end{itemize}
	
	
	
	
	
	
	
	
	
	
	
	
	
	
	
	\section{Introduction}
	
	\begin{defn} \ld{def:hilbert:intro:00001}
		Let $H$ be a vector space and $\l \cdot, \cdot \r: H \times H \rightarrow \C$. Then 
		\begin{itemize}
			\item $\l \cdot, \cdot \r$ is said to be an \tbf{inner product} on $H$ if for each $x,y,z \in H$ and $\lam \in \C$
			\begin{enumerate}
				\item $\l x , y + \lam z\r = \l x , y \r + \lam \l x , z\r $
				\item $\l x , y \r = \l y , x\r^*$
				\item $\l x , x \r \geq 0$
				\item if $\l x ,x \r = 0$, then $x = 0$.  
			\end{enumerate}
			\item $(H, \l \cdot, \cdot \r)$ is said to be a \tbf{inner product space} if $\l \cdot, \cdot \r$ is an inner product on $H$.
		\end{itemize}
	\end{defn}

	\begin{note}
		When the context is clear, we supress the inner product $\l \cdot, \cdot \r: H \times H \rightarrow \C$.
	\end{note}
	
	\begin{note}
	In mathematics, inner products are conventionally linear in the first argument. The convention in physics is linearity in the second argument. The physics convention notationally generalizes the dot product as matrix multiplication when identifying $\C^n$ with $\C^{n \times 1}$ as is done in an introductory linear algebra class. For example, for $x,y \in \C^n$ $\l x, y \r = \bar{x}^{\top} y$.
	\end{note}
	 
\begin{defn}
	Let $(H, \l \cdot, \cdot \r)$ be an inner product space. We define $\l \cdot, \cdot \r_{\bar{H}}: \bar{H} \times \bar{H} \rightarrow \C$ by $\l x, y \r_{\bar{H}} \defeq \l x, y \r^*$. 
\end{defn}

\begin{ex}
	Let $(H, \l \cdot, \cdot \r)$ be an inner product space. Then $(\bar{H}, \l \cdot, \cdot \r_{\bar{H}})$ is an inner product space.
\end{ex}

\begin{proof}
	Let $x, y, z \in \bar{H}$ and $\lam \in \C$. Then
	\begin{enumerate}
		\item 
		\begin{align*}
			\l x, y + \lam \bar{\cdot} z \r_{\bar{H}} 
			& = \l x, y + \lam \bar{\cdot} z\r^*  \\
			& = \l x, y \r^* + \l x, \lam \bar{\cdot} z\r^*  \\
			& = \l x, y \r^* + \l x, \lam^* z\r^* \\
			& = \l x, y \r^* + \l \lam^* z, x\r \\
			& = \l x, y \r^* + \lam \l z, x\r \\
			& = \l x, y \r^* + \lam \l x, z\r^* \\
			& = \l x, y \r_{\bar{H}} + \lam \l x, z\r_{\bar{H}} \\
		\end{align*}
		\item 
		\begin{align*}
			\l x, y \r_{\bar{H}}
			& = \l x, y \r^* \\
			& = (\l y, x \r^*)^* \\
			& =  \l y, x\r_{\bar{H}}^* 
		\end{align*}
		\item 
		\begin{align*}
			\l x, x \r_{\bar{H}} 
			& = \l x, x\r^* \\
			& = \l x, x \r \\
			& \geq 0
		\end{align*}
		\item Suppose that $\l x, x \r_{\bar{H}} = 0$. Then 
		\begin{align*}
			\l x, x \r
			& = \l x, x \r^* \\
			& = \l x, x \r_{\bar{H}} \\
			& = 0.
		\end{align*}
		Therefore $x = 0$. 
	\end{enumerate}
	Thus $\l \cdot, \cdot \r_{\bar{H}}$ is an inner product on $\bar{H}$ and $(\bar{H}, \l \cdot, \cdot \r_{\bar{H}})$ is an inner product space.  
\end{proof}
	
	 
\begin{ex} \lex{ex:hilbert:intro:00002}
	Let $H$ be an inner product space, $(x_j)_{j =1}^n$, $(y_j)_{j =1}^n \subset H$ and $(\al_j)_{j=1}^n$, $(\be_j)_{j=1}^n \subset \C$. Then $$\bigg \l \sum_{i=1}^n \al_i x_i , \sum_{j=1}^n \be_j y_j \bigg \r = \sum_{i=1}^n \sum_{j=1}^n \al_i^*\be_j \l x_i , y_j \r $$
\end{ex}

\begin{proof}
Clear.
\end{proof}

\begin{defn} \ld{def:hilbert:intro:00003}
Let $(H, \l \cdot, \cdot \r)$ be an inner product space. Define the \tbf{induced norm}, denoted $\|\cdot \|: \rightarrow \C$, by $$\|x\| = \l x, x\r^{1/2}$$
\end{defn}

\begin{note}
	Unless otherwise specified, we only consider the induced norm on any given Hilbert space.
\end{note}

\begin{ex} \lex{ex:hilbert:intro:00004} \tbf{Cauchy-Schwarz Inequality}\\
Let $H$ be an inner product space. Then for each $x,y \in H$, $| \l x, y\r | \leq \|x\| \| y\|$ and $| \l x, y\r | = \|x\| \| y\|$ iff $x \in \spn(y)$. \\
\tbf{Hint:} For $x, y \in H$, put $z = \sgn\l x, y \r^*y$ and Consider $f: \R \rightarrow \Rg$ defined by $f(t) = \|x - tz\|^2$
\end{ex}

\begin{proof}
Let $x,y \in H$. If $y = 0$, then the claim holds trivially. Suppose that $y \neq 0$. Put $z = \sgn\l x, y \r^*y$. So $\l x, z\r = |\l x,y \r |$ and $\|z\| = \|y\|$. Define $f: \R \rightarrow \Rg$ by 
$$f(t) = \|x - tz\|^2$$ 
Then for each $t \in \R$, 
\begin{align*}
0 
& \leq f(t) \\
&=  \|x - tz\|^2 \\
&= \|x\|^2 + |t|^2\|z\|^2 - 2 \Re(t \l x,z \r) \\
&= \|x\|^2 + t^2\|y\|^2 - 2 t |\l x,y \r| \\
\end{align*} 
Thus $f$ is a quadratic with a minimum at $t_0 = \frac{|\l x, y \r|}{\|y\|^2}$. Hence 
\begin{align*}
0 
&\leq f(t_0) \\
&= \|x\|^2 +  \frac{|\l x, y \r|}{\|y\|^2} - 2\frac{|\l x, y \r|}{\|y\|^2} \\
& = \|x\|^2 -  \frac{|\l x, y \r|}{\|y\|^2}
\end{align*}
Which implies that $$| \l x, y\r |^2 \leq \|x\|^2 \| y\|^2$$ and hence the claim holds. Clearly if $x \in \spn(y)$, then equality holds. Conversely, if equality holds, then $x-z = 0$ which implies that $x \in \spn(y)$.
\end{proof}

\begin{ex} \lex{ex:hilbert:intro:00005}
Let $H$ be an inner product space. Then the induced norm, $\| \cdot\|: H \rightarrow \C$, is a norm. 
\end{ex}

\begin{proof}Let $x,y \in H$ and $\lam \in \C$. Then
\begin{enumerate}
\item By definition, if $\|x\| = 0$, then $\l x, x \r =0$, which implies that $x =0$.
\item Note that 
\begin{align*}
\| \lam x \|^2 
&= \l \lam x, \lam x \r \\
&= \lam *\lam  \l x, x\r \\
&= |\lam |^2\| x \|^2
\end{align*}
So $\| \lam x \| = |\lam |\|x\|$
\item The Cauchy-Schwarz inequality implies that
\begin{align*}
\|x + y\|^2 
&= \|x\|^2 + \|y\|^2 + 2 \Re(\l x, y\r) \\
& \leq \|x\|^2 + \|y\|^2 + 2 |\l x, y\r | \\
& \leq \|x\|^2 + \|y\|^2 + 2 \|x\| \|y\| \\
&= (\|x\| + \|y\|)^2
\end{align*}
Hence $\|x + y\| \leq \|x\| + \|y\|$.
\end{enumerate}
\end{proof}

\begin{ex} \lex{ex:hilbert:intro:00006}
	Let $H$ be an inner-product space and $y \in H$. Then $y = 0$ iff for each $x \in H$, $\l y, x\r = 0$. 
\end{ex}

\begin{proof}\
	\begin{itemize}
		\item $(\implies)$: \\
		Suppose that $y = 0$. Let $x \in H$. Then  
		\begin{align*}
			\l y, x \r 
			& = \l 0, x \r \\
			& = \l 0 + 0, x \r \\
			& = \l0, x \r + \l 0, x \r  \\
			& = \l y, x \r  + \l y, x \r 
		\end{align*}
		Hence $\l y, x \r  = 0$. Since $x \in H$ is arbitrary, we have that for each $x \in H$, $\l y, x \r =0$.
		\item $(\impliedby)$: \\
		Suppose that $y \neq 0$. Then
		\begin{align*}
			0
			& \neq \|y\| \\
			& = \l y, y \r
		\end{align*}
		Define $x \in H$ by $x \defeq y$. Then $\l y, x \r \neq 0$. Hence $y \neq 0$ implies that there exists $x \in H$ such that $\l y, x \r \neq 0$. By contrapositive, if for each $x \in H$, $\l y, x \r = 0$, then $y = 0$. 
	\end{itemize}
\end{proof}

\begin{ex} \lex{ex:hilbert:intro:00006.0.1}
	Let $H_1, H_2 \in \Obj(\Hilb)$ and $A \in L(H_1, H_2)$. Then $A = 0$ iff for each $x \in H_1$ and $y \in H_2$, $\l y, Ax \r = 0$.
\end{ex}

\begin{proof}\
	\begin{itemize}
		\item $(\implies)$: \\
		Suppose that $A = 0$. Then for each for each $x \in H_1$ and $y \in H_2$, 
		\begin{align*}
			\l y, Ax \r
			& = \l y, 0 \r \\
			& = 0.
		\end{align*}
		\item $(\impliedby)$: \\
		Suppose that for each $x \in H_1$ and $y \in H_2$, $\l y, Ax \r = 0$. Let $x \in H_1$. Then for each $y \in H_2$,  
		\begin{align*}
			\l Ax, y \r
			& = \l y, Ax \r^* \\
			& = 0.
		\end{align*}
		\rex{ex:hilbert:intro:00006} implies that $Ax = 0$. Since $x \in H_1$ is arbitrary, we have that for each $x \in H_1$, $Ax = 0$. Hence $A = 0$.
	\end{itemize}
\end{proof}

\begin{defn} \ld{def:hilbert:intro:00006.1}
	Let $X$ be a normed vector space. Then $X$ is said to satisfy the \tbf{parallelogram law} if for each $x,y \in X$, 
	$$ \|x+y\|^2 + \|x-y\|^2 = 2(\|x\|^2 + \|y\|^2).$$ 
\end{defn}

\begin{ex} \lex{ex:hilbert:intro:00006.2} \tbf{Polarization Identity:} \\
	Let $H$ be an inner product space. For each $x, y \in H$, 
	$$\l x, y \r = \frac{1}{4} \bigg[  \|x+y\|^2 - \|x-y\|^2 - i\|x+iy\|^2 + \|x-iy\|^2 \bigg].$$
\end{ex}

\begin{proof}
	Let $x,y \in H$. Then 
	\begin{align*}
		\|x+y\|^2 - \|x-y\|^2 - i\|x+iy\|^2 +i \|x-iy\|^2
		& = 4 \Re \l x, y \r - 4i \Re \l x, iy \r \\
		& = 4 \Re \l x, y \r - 4i \Re i \l x, y \r \\
		& = 4 \Re \l x, y \r + 4i \Im \l x, y \r \\
		& = 4 \l x, y \r. 
	\end{align*}
	Hence 
	$$\l x, y \r = \frac{1}{4} \bigg[  \|x+y\|^2 - \|x-y\|^2 - i\|x+iy\|^2 + i\|x-iy\|^2 \bigg].$$
	Since $x,y \in H$ are arbitrary, we have that for each $x, y \in H$, 
	$$\l x, y \r = \frac{1}{4} \bigg[  \|x+y\|^2 - \|x-y\|^2 - i\|x+iy\|^2 + i\|x-iy\|^2 \bigg].$$
\end{proof}

\begin{ex} \lex{ex:hilbert:intro:00007} \tbf{Paralellogram Law Equivalence:} \\
	Let $X$ be a normed vector space. There exists $\l \cdot, \cdot \r : X \times X \rightarrow \C$ such that $(X, \l \cdot, \cdot \r)$ is an inner product space and for each $x \in X$, $\|x\|^2 = \l x, x \r$ iff $X$ satisfies the parallelogram law.
\end{ex}

\begin{proof}\
	\begin{itemize}
		\item $(\implies)$: \\
		Suppose that there exists $\l \cdot, \cdot \r : X \times X \rightarrow \C$ such that $(X, \l \cdot, \cdot \r)$ is an inner product space and for each $x \in X$, $\|x\|^2 = \l x, x \r$. Let $x,y \in X$. Then 
		\begin{align*}
			\|x+y\|^2 + \|x-y\|^2
			& = (\|x\|^2 + \|y^2\| + 2 \Rl(\l x,y\r)) + (\|x\|^2 + \|y^2\| - 2 \Rl(\l x,y\r)) \\
			& = 2(\|x\|^2 + \|y^2\|)
		\end{align*}
		Since $x,y \in H$ are arbitrary, we have that for each $x,y \in X$, $ \|x+y\|^2 + \|x-y\|^2 = 2(\|x\|^2 + \|y\|^2)$. Hence $X$ satisfies the parallelogram law.  
		\item $(\impliedby)$: \\
		Suppose that $X$ satisfies the parallelogram law. Define $\l \cdot, \cdot \r : X \times X \rightarrow \C$ by 
		$$\l x, y \r \defeq \frac{1}{4} \bigg[  \|x+y\|^2 - \|x-y\|^2 - i\|x+iy\|^2 + i\|x-iy\|^2 \bigg].$$
	\end{itemize}
	Let $x,y,z \in X$ and $\lam \in \C$.
	\begin{enumerate}
		\item We note that 
		\begin{align*}
			\l y, x \r
			& = \frac{1}{4} \bigg[  \|y+x\|^2 - \|y-x\|^2 - i\|y+ix\|^2 + i\|y-ix\|^2 \bigg] \\
			& = \frac{1}{4} \bigg[  \|x+y\|^2 - \|x-y\|^2 - i\|iy - x\|^2 + i\|iy+x\|^2 \bigg] \\
			& = \frac{1}{4} \bigg[  \|x+y\|^2 - \|x-y\|^2 - i\|x-iy\|^2 + i\|x+iy\|^2 \bigg] \\
			& = \frac{1}{4} \bigg[  \|x+y\|^2 - \|x-y\|^2 - i\|x+iy\|^2 + i\|x-iy\|^2 \bigg]^* \\
			& = \l x, y \r^*.
		\end{align*}
		\item 
		\begin{itemize}
			\item By repeated application of the parallelogram law, we have that
			\begin{align*}
				4 \l x, y +  z \r
				& = \|x+(y +  z)\|^2 - \|x-(y +  z)\|^2 - i \|x + i(y+z)\|^2 + i \|x - i(y + z) \|^2 \\
				& = \|(x+y) + z\|^2 - \|(x-y) - z\|^2 - i\|(x + iy) + iz\|^2 + i \| (x-iy) - iz\|^2  \\
				& = \bigg[ 2(\|x+y\|^2 + \|z\|^2) - \|(x+y) -z\|^2 \bigg] \\
				& \quad \quad - \bigg[ 2(\|x-y\|^2 + \|z\|^2) - \|(x-y) + z\|^2 \bigg] \\
				& \quad \quad -i \bigg[ 2(\|x+iy\|^2 + \|z\|^2) - \|(x+iy) - iz\|^2 \bigg]  \\
				& \quad \quad +i \bigg[ 2(\|x-iy\|^2 + \|z\|^2) - \|(x-iy) + iz\|^2 \bigg] \\
				& = 2(\|x+y\|^2 - \|x-y\|^2 -i \|x+iy\|^2  + i \|x-iy\|^2 ) \\
				& \quad \quad + (\|(x - y) + z\|^2 - \|(x + y) - z\|^2 -i\|(x - iy) + iz\|^2 +i \|(x + iy) -iz\|^2) \\
				& = 8 \l x, y \r + \|(x - y) + z\|^2 - \|(x + y) - z\|^2 -i\|(x - iy) + iz\|^2 +i \|(x + iy) -iz\|^2.
			\end{align*}
			Applying the parallelogram, we obtain
			\begin{align*}
				\|(x - y) + z\|^2 - \|(x + y) - z\|^2
				& = \|(x+ z) -y\|^2 - \|(x - z) +y\|^2 \\
				& = \bigg[ 2(\|x+z\|^2 + \|y\|^2) - \|(x+z) +y\|^2 \bigg] \\
				& \quad \quad - \bigg[ 2(\|x-z\|^2 + \|y\|^2) - \|(x-z) -y\|^2 \bigg] \\
				& = 2(\|x+z\|^2 - \|x-z\|^2) + \|(x-z) -y\|^2 - \|(x+z) +y\|^2 \\
			\end{align*}
			and 
			\begin{align*}
				\|(x + iy) -iz\|^2 - \|(x - iy) + iz\|^2 
				& =  \|(x - iz) +iy \|^2 - \|(x + iz) - iy\|^2 \\
				& = \bigg[ 2(\|x - iz\|^2 + \|y\|^2) - \|(x - iz) - iy \|^2 \bigg] \\
				& \quad \quad - \bigg[ 2(\|x + iz\|^2 + \|y\|^2) - \|(x + iz) + iy\|^2 \bigg] \\
				& = 2(\|x - iz\|^2 - \|x + iz\|^2) + \|(x + iz) + iy\|^2 - \|(x - iz) - iy \|^2.
			\end{align*}
			Therefore
			\begin{align*}
				\|(x - y) + z\|^2 - \|(x + y) - z\|^2 \\
				-i \|(x - iy) + iz\|^2 + i\|(x + iy) -iz\|^2  
				& = 2(\|x+z\|^2 - \|x-z\|^2) +  \|(x-z) -y\|^2 - \|(x+z) +y\|^2  \\
				& \quad \quad + 2(i\|x - iz\|^2 - i\|x + iz\|^2) + i\|(x + iz) + iy\|^2 - i\|(x - iz) - iy \|^2 \\
				& = 2(\|x+z\|^2 - \|x-z\|^2 - i\|x + iz\|^2 +  i\|x - iz\|^2) \\
				& \quad \quad + \bigg[ \|(x-z) -y\|^2 - \|(x+z) +y\|^2 \\
				& \quad \quad + i\|(x + iz) + iy\|^2 - i\|(x - iz) - iy \|^2 \bigg] \\
				& = 8 \l x, z \r + \bigg[ \|x-(y+z)\|^2 - \|x+ (y +z)\|^2 \\
				& \quad \quad + i\|x+ i(y + z)\|^2 - i\|x - i(y+z) \|^2 \bigg] \\
				& = 8 \l x, z \r - \bigg[ \|x+ (y +z)\|^2 - \|x-(y+z)\|^2 \\
				& \quad \quad - i\|x+ i(y + z)\|^2 + i\|x - i(y+z) \|^2 \bigg] \\
				& =  8 \l x, z \r - 4 \l x, y+z \r.
			\end{align*}
			Combining the previous equations, we have that
			\begin{align*}
				4 \l x, y +  z \r
				& = 8 \l x, y \r + 8 \l x, z \r - 4 \l x, y+z \r \\
				& = 8 (\l x, y \r +  \l x, z \r) - 4 \l x, y+z \r.
			\end{align*}
			Thus $8 \l x, y +  z \r = 8 (\l x, y \r +  \l x, z \r)$ and finally $\l x, y +  z \r = \l x, y \r +  \l x, z \r$. 
			\item By induction and additivity, for each $m \in \Z$, $\l x, m z\r = m \l x, z\r$. Let $q \in \Q$. Then there exist $m \in \Z$ and $n \in \N$ such that $q = m/n$. Then 
			\begin{align*}
				\l x, q z\r
				& = m \l x, n^{-1}z \r \\
				& = m \frac{1}{4} \bigg[  \|x+ n^{-1}z\|^2 - \|x- n^{-1}z\|^2 - i\|x+in^{-1}z\|^2 + i\|x-in^{-1}z\|^2 \bigg] \\
				& = m \frac{1}{4} \bigg[  n^2\|nx+z\|^2 - n^2\|x- n^{-1}z\|^2 - n^2i\|x+in^{-1}z\|^2 + n^2i\|x-in^{-1}z\|^2 \bigg] \\
				& = m \frac{1}{4} \bigg[  n^{-2}\|nx+z\|^2 - n^{-2}\|nx- z\|^2 - n^{-2}i\|nx+iz\|^2 + n^{-2}i\|nx-iz\|^2 \bigg] \\
				& = mn^{-2} \l nx, z \r \\
				& = mn^{-2} \l z, nx \r^* \\
				& = mn^{-2} [n\l z, x \r]^* \\ 
				& = mn^{-2} n\l z, x \r^* \\
				& = q \l x, z \r.
			\end{align*}
			Since $q \in \Q$ is arbitrary, we have that for each $q \in \Q$, $\l x, q z\r = q \l x, z \r$. We note that
			\begin{align*}
				\l x, i z\r
				& = \frac{1}{4} \bigg[  \|x+ (iz)\|^2 - \|x- (iz)\|^2 - i\|x+ i(iz)\|^2 + i\|x-i(iz)\|^2 \bigg] \\
				& = \frac{1}{4} \bigg[  \|x+ iz\|^2 - \|x- iz\|^2 - i\|x-z\|^2 + i\|x+z\|^2 \bigg] \\
				& = \frac{1}{4} \bigg[ i\|x+z\|^2  - i\|x-z\|^2 + \|x+ iz\|^2 - \|x- iz\|^2 \bigg] \\
				& = i \frac{1}{4} \bigg[ \|x+z\|^2  - \|x-z\|^2 - i\|x+ iz\|^2 + i\|x- iz\|^2 \bigg] \\
				& = i \l x, z\r.
			\end{align*}
			Let $\lam_0 \in \Q[i]$. Then there exist $a,b \in \Q$ such that $\lam_0 = a + ib$. Then from before, we have that
			\begin{align*}
				\l x, \lam_0 z\r
				& = \l x, (a + ib) z\r \\
				& = \l x, az + ib z \r \\
				& = \l x, az + ib z \r \\
				& = \l x, az \r + \l x, ib z \r \\
				& = a \l x, z \r + i\l x, b z \r \\
				& = a \l x, z \r + ib \l x,  z \r \\
				& = (a + ib) \l x, z \r \\
				& = \lam_0 \l x, z \r.
			\end{align*} 
			Since $\lam_0 \in \Q[i]$ is arbitrary, we have that for each $\lam_0 \in \Q[i]$, $\l x, \lam_0 z\r = \lam_0 \l x, z \r $. Since $\Q[i]$ is dense in $\C$, there exists $(\lam_n)_{n \in \N} \subset \Q[i]$ such that $\lam_n \rightarrow \lam$. Since $\l \cdot, \cdot\r$ is continuous, we have that
			\begin{align*}
				\l x, \lam z\r
				& = \limn \l x, \lam_n z\r \\
				& = \limn [\lam_n \limn \l x, z\r] \\
				& = \lam \l x, z\r.
			\end{align*}
		\end{itemize}
		Hence $\l x, y + \lam z\r = \l x, y\r + \lam \l x, z\r$.
		\item The parallelogram law implies that 
		\begin{align*}
			\l x , x \r
			& = \frac{1}{4} \bigg[  \|x+x\|^2 - \|x-x\|^2 - i\|x+ix\|^2 + i\|x-ix\|^2 \bigg] \\
			& = \frac{1}{4} \bigg[  4\|x\|^2 - i\|(1+i)x\|^2 + i\|(1-i)x\|^2 \bigg] \\
			& = \frac{1}{4} \bigg[  4\|x\|^2 - i|1+i|\|x\|^2 + i|1-i|\|x\|^2 \bigg] \\
			& = \frac{1}{4} \bigg[  4\|x\|^2 + i(|1-i| - |1+i| )\|x\|^2  \bigg] \\
			& = \frac{1}{4} \bigg[  4\|x\|^2 + i(|1-i| - |1+i| )\|x\|^2  \bigg] \\
			& = \frac{1}{4} \bigg[  4\|x\|^2 + i(\sqrt{2} - \sqrt{2} )\|x\|^2  \bigg] \\
			& = \|x\|^2 \\
			& \geq 0.
		\end{align*}
		\item if $\l x ,x \r = 0$, then from before, we have that 
		\begin{align*}
			\|x\|^2
			& = \l x, x\r \\
			& = 0.
		\end{align*}
		Since $\|\cdot\|$ is a norm on $X$, $x = 0$. 
	\end{enumerate}
	Thus $\l \cdot, \cdot \r$ is an inner product on $X$ and $\l x, x \r = \|x\|^2$.
\end{proof}

\begin{defn} \ld{def:hilbert:intro:00008} 
	Let $H$ be an inner product space, $x, y \in H$ and $S \subset H$. Then
	\begin{enumerate}
	\item $x$ and $y$ are said to be \tbf{orthogonal}, denoted $x \perp y$, if $\l x,y\r = 0$. 
	\item $S$ is said to be \tbf{orthogonal} if for each $x,y \in S$, $x \perp y$. 
	\end{enumerate}
\end{defn}

\begin{defn} \ld{def:hilbert:intro:00009} 
	Let $H$ be an inner product space and $E \subset H$ a closed subspace. We define the \tbf{orthogonal complement of $E$}, denoted $E^{\perp}$, by 
	$$E^{\perp} = \{x \in H: \text{ for each $y \in E$, $x \perp y$}\}$$
\end{defn}

\begin{ex} \lex{ex:hilbert:intro:00010}
	Let $H$ be an inner product space and $E \subset H$. Then $E^{\perp}$ is a closed subspace of $H$.
\end{ex}

\begin{proof}\
	\begin{itemize}
		\item Let $x,y \in E^{\perp}$ and $\lam \in \C$. Then for each $z \in E$, 
		\begin{align*}
			\l x+ \lam y,  z\r
			& = \l x, z \r + \lam \l y, z \r \\
			& = 0
		\end{align*}
		Hence $x + \lam y \in E^{\perp}$. Thus $E^{\perp}$ is a subspace of $H$. 
		\item Let $(x_n)_{n \in \N} \subset E^{\perp}$ and $x \in H$. Suppose that $x_n \rightarrow x$. Then for each $z \in E$, continuity implies that
		\begin{align*}
			\l x, z \r
			& = \limn \l x_n, z \r \\
			& = \limn 0 \\
			& = 0
		\end{align*}
		Hence $x \in E^{\perp}$. Since $(x_n)_{n \in \N} \subset E^{\perp}$ and $x \in H$ with $x_n \rightarrow x$ are arbitrary, we have that $E^{\perp}$ is closed.
	\end{itemize}
\end{proof}

\begin{ex} \lex{ex:hilbert:intro:00011}
	Let $H$ be an inner product space. Then 
	\begin{enumerate}
		\item $H^{\perp} = \{0\}$.
		\item $\{0\}^{\perp} = H$
	\end{enumerate}
\end{ex}


\begin{proof}\
	\begin{enumerate}
		\item 
		Let $z \in H^{\perp}$. By definition, for each $x \in H$, $\l z, x \r = 0$. \tcb{A previous exercise} implies that $z = 0$. Since $z \in H^{\perp}$ is arbitrary, $H^{\perp} = \{0\}$. 
		\item Let $z \in H$. Trivially, for each $x \in \{0\}$, $\l z, x \r = 0$. Thus $z \in \{0\}^{\perp}$. Since $z \in H$ is arbitrary, $H \subset \{0\}^{\perp}$. Since trivially, $\{0\}^{\perp} \subset H$, we have that $\{0\}^{\perp} = H$.
	\end{enumerate}
\end{proof}

\begin{ex} \lex{ex:hilbert:intro:00012}
	Let $H$ be an inner product space and $E,F \subset H$. If $E \subset F$, then $F^{\perp} \subset E^{\perp}$.
\end{ex}

\begin{proof}
	Suppose that $E \subset F$. Let $x \in F^{\perp}$ and $z \in E$. By definition, for each $y \in F$, $\l x, y\r = 0$. Since $E \subset F$, $z \in F$. Hence $\l x, z \r = 0$. Since $x \in E$ is arbitrary, we have that for each $z \in E$, $\l x, z \r = 0$. Hence $x \in E^{\perp}$. Since $x \in F^{\perp}$ is arbitrary, we have that $F^{\perp} \subset E^{\perp}$. 
\end{proof}

\begin{ex} \lex{ex:hilbert:intro:00013} \tbf{Pythagorean theorem:}\\
	Let $H$ be an inner product space and $(x_j)_{j =1}^n \subset H$ an orthogonal set. Then $$\bigg \|\sum\limits_{j = 1}^n x_j  \bigg \|^2 = \sum\limits_{j =1}^n \|x_j \|^2$$
\end{ex}

\begin{proof}
	We have that
	\begin{align*}
		\bigg \| \sum\limits_{j = 1}^n  x_j\bigg \|^2
		&= \bigg \l \sum\limits_{i =1}^n x_i , \sum\limits_{j =1}^n x_j \bigg \r \\
		&= \sum\limits_{i =1}^n \sum\limits_{j =1}^n \l x_j , x_j \r \\
		&= \sum\limits_{j =1}^n \l x_j , x_j \r \\
		&= \sum\limits_{j =1}^n \| x_j \|^2
	\end{align*}
\end{proof}

\begin{ex} \lex{ex:hilbert:intro:00014}
	Let $H$ be an inner product space and $S \subset H$. Suppose that $0 \not \in S$. If $S$ is orthogonal, then $S$ is linearly independent.
\end{ex}

\begin{proof}
	Let $x_1, \cdots, x_n \in S$ and $c_1, \cdots, c_n \in \C$. Suppose that $\sum\limits_{j =1}^n c_j x_j = 0 $. Since $(c_j x_j)_{j=1}^n$ is orthogonal, the Pythagorean theorem implies that 
	\begin{align*}
		0
		&= \bigg \| \sum_{i=1}^n c_i x_i \bigg \| \\
		&= \sum_{j=1}^n  |c_j|^2 \| x_j\| 
	\end{align*}
	So for each $j \in \{ 1 , \cdots, n\}$, $c_j = 0$ and $S$ is linearly independent.
\end{proof}

\begin{defn} \ld{def:hilbert:intro:00015}
	Let $H$ be an inner product space and $S \subset H$. Then $S$ is said to be \tbf{orthonormal} if $S$ is orthogonal and for each $x \in S$, $\|x \| = 1$.
\end{defn}

\begin{ex} \lex{ex:hilbert:intro:00016} \tbf{Bessel's Inequality:}\\
Let $H$ be an inner product space and $S \subset H$. If $S$ is orthonormal, then for each $x \in H$, 
\begin{enumerate}
	\item $\sum\limits_{u \in S} | \l u, x \r |^2  \leq \|x\|$
	\item  $\{u \in S: \l u, x\r \neq 0\}$ is countable.
\end{enumerate}
\end{ex}

\begin{proof}\
\begin{enumerate}
	\item Suppose that $S$ is orthonormal. Let $x \in H$. We consider the measure space $(S, \MP(S), \#)$. Define $f_x: S \rightarrow \Rg$ by $f_x(u) = \l u, x\r $. Basic results about counting measure imply that
	\begin{align*}
		\sum_{u \in S} |\l u, x\r |^2
		& = \int |f_x|^2 \, d\# \\
		& = \sup \bigg \{\sum_{u \in F} |f_x(u)|^2: F \subset S \text{ and } \#(F) < \infty \bigg\}
	\end{align*}
	Let $F \subset S$ finite. Then the Pythagorean theorem implies that  
	\begin{align*}
		0 
		& \leq \bigg \|x - \sum_{u \in F} \l u, x \r u \bigg \|^2 \\
		&= \|x\|^2 + \bigg \| \sum_{u \in F} \l u, x \r u \bigg \|^2 - 2 \Rl \bigg \l x, \sum_{u \in F} \l u, x \r u \bigg \r  \\
		&= \|x\|^2 +  \sum_{u \in F} |\l u, x \r|^2 \|u\|^2 - 2 \sum_{u \in F} | \l u, x \r|^2  \\
		&= \|x\|^2 -  \sum_{u \in F} |\l u, x \r|^2 
	\end{align*}
	Thus 
	$$\sum_{u \in F} | \l u, x \r |^2  \leq \|x\|$$
	Since $F \subset X$ such that $\#(F) < \infty$ was arbitrary, we have that  
	$$\sum_{u \in S} | \l u, x \r |^2  \leq \|x\|$$
	\item Since 
	$$\int |f_x|^2 \dsh < \infty$$
	basic results about counting measure imply that $\{u \in S: \l u, x\r \neq 0\}$ is countable.
\end{enumerate}
\end{proof}

\begin{defn} \ld{def:hilbert:intro:00017}
	Let $(H, \l \cdot, \cdot \r)$ be an inner product space. Then $(H, \l \cdot, \cdot \r)$ is said to be a \tbf{Hilbert space} if $(H, \| \cdot \|)$ is a Banach space.
\end{defn}

\begin{ex} \lex{ex:hilbert:intro:00018}
	Let $H$ be a Hilbert space, $E \subset H$ a closed subspace of $H$, $x \in H$, $y \in E$ and $z \in E^{\perp}$. If $x = y + z$, then $\|x + E\|_{H/E} = \|z\|$.
\end{ex}

\begin{proof}
	Suppose that $x = y+z$. Let $y' \in E$. The Pythagorean theorem implies that
	\begin{align*}
		\|x - y'\|^2
		& = \|y + z - y'\|^2 \\
		& = \|y - y'\|^2 + \|z\|^2 \\
		& \geq \|z\|^2 \\
	\end{align*}
	Thus 
	\begin{align*}
		\|x +E\|
		& = \inf_{y' \in E} \|x - y'\| \\
		& \geq \|z\| \\
	\end{align*}
	Also,
	\begin{align*}
		\|x + E\| 
		& \leq \|x - y\| \\ 
		& = \|z\| 
	\end{align*}
	Hence $\|x + E\| = \|z\|$.
\end{proof}

\begin{ex} \lex{ex:hilbert:intro:00019}
	Let $H$ be a Hilbert space, $E \subset H$ a closed subspace of $H$ and $x \in H$. Then 
	\begin{enumerate}
		\item there exists a unique $y_0 \in E$ such that $\|x - y_0\| = \|x+E\|$ \\
		\tbf{Hint:} Suppose $(y_n)_{n \in \N} \subset E$ satisfies $\|x-y_n\| \rightarrow \|x + E\|$. Show that $(y_n)_{n \in \N}$ is Cauchy using the parallelogram law.
		\item there exist unique $y_0\in E$ and $z_0 \in E^{\perp}$ such that $x = y_0 + z_0$ and $\|z_0\| = \|x+E\|$ \\
		\tbf{Hint:} Set $z_0 \defeq x-y_0 $ and for $u \in E$, choose $\lam \in \C$ such that $\l x, \lam u \r \in \R$. Consider $f(t) = \|z - t\lam u\|^2$.
	\end{enumerate}
\end{ex}

\begin{proof} Set $s \defeq \|x+E\|$.
	\begin{enumerate}
		\item \begin{itemize}
			\item \tbf{(Existence):} \\
			Choose $(y_n)_{n \in \N} \subset E$ such that for each $n \in \N$, $\|x - y_n\| < s + 1/n$. Define $(a_n)_{n \in \N} \subset H$ by $a_n = x - y_n$. Let $m,n \in \N$. The parallelogram law implies that
			\begin{align*}
				2(\|x - y_n\|^2 + \|x-y_m\|^2)
				& = 2 (\|a_n\|^2 + \|a_m\|^2) \\
				& = \|a_n + a_m\|^2 + \|a_n - a_m\|^2 \\
				& = \|2x - (y_n + y_m)\|^2 +  \|y_m - y_n\|^2  \\
				& = 4 \|x - 2^{-1}(y_n + y_m)\|^2 +  \|y_m - y_n\|^2
			\end{align*}
			Since $y_n,y_m \in E$, $2^{-1}(y_n + y_m) \in E$ and therefore $\|x - 2^{-1}(y_n + y_m)\| \geq s$. Sine $m,n \in \N$ are arbitrary, we have that for each $m,n \in \N$, 
			\begin{align*}
				\|y_m - y_n\|^2  
				& = 2(\|x - y_n\|^2 + \|x-y_m\|^2) - 4 \|x - 2^{-1}(y_n + y_m)\|^2 \\
				& \leq 2(\|x - y_n\|^2 + \|x-y_m\|^2) - 4s^2 
			\end{align*}
			Let $\ep > 0$. Choose $N \in \N$ such that $1/N < (s^2 + \ep^2/4)^{1/2} - s$. Let $m,n \in \N$. Suppose that $m,n \geq N$. Then 
			\begin{align*}
				\|y_m - y_n\|^2
				& = 2\|x - y_n\|^2 + 2\|x-y_m\|^2 - 4s^2 \\
				& < 2(s + 1/n)^2 + 2(s + 1/m)^2 -4s^2 \\
				& < 2(s + 1/N)^2 + 2(s + 1/N)^2 -4s^2 \\
				& = 4(s + 1/N)^2 -4s^2 \\
				& < 4(s^2 + \ep^2/4) - 4s^2 \\
				& = \ep^2 
			\end{align*}
			Thus $\|y_m - y_n\| < \ep$. Since $\ep > 0$ is arbitrary, we have that for each $\ep > 0$, there exists $N \in \N$ such that for each $m,n \in \N$, $m,n \geq N$ implies that $\|y_m - y_n\| < \ep$. So $(y_n)_{n \in \N}$ is Cauchy and since $H$ is complete, there exists $y_0 \in H$ such that $y_n \rightarrow y$. Coninutiy of the inner product, addition and scalar multiplication implies that
			\begin{align*}
				\|x-y_0\|
				& = \lim_{n \rightarrow \infty} \|x - y_n\| \\
				& = s \\
				& = \|x + E\|
			\end{align*}
			\item \tbf{(Uniqueness):} \\
			Let $y_1 \in E$. Suppose that $\|x-y_1\| = \|x + E\|$. Similarly to part $(1)$, the parallelogram law implies that 
			\begin{align*}
				\|y_1 - y_0\|^2 
				& \leq 2(\|x - y_1\|^2 + \|x-y_0\|^2) - 4s^2 \\
				& = 2(s^2 + s^2) - 4s^2 \\
				& = 0
			\end{align*}
			Hence $\|y_1 - y_0\| = 0$ and $y_1 = y_0$. 
		\end{itemize}
		\item 
		\begin{itemize}
			\item \tbf{(Existence):} \\
			Set $z_0 \defeq x - y_0$. Let $u \in E$. Set 
			\[
			\lam \defeq 
			\begin{cases}
				(\sgn\l z_0, u \r)^{-1} & \l z_0, u \r \neq 0 \\
				1 & \l z_0, u \r = 0
			\end{cases}
			\] and $v \defeq \lam u$. So $\l z_0, v \r \in \R$. Define $f: \R \rightarrow \R$ by $f(t) = \|z_0 - tv\|^2$.
			By construction, since $y_0,v \in E$, we have that for each $t \in \R$, $y_0 - tv \in E$ and therefore
			\begin{align*}
				f(t)
				& = \|z_0 - tv\|^2 \\
				& = \|x - (y_0 - tv)\|^2 \\
				& \geq \inf_{y \in E} \|x - y\|^2 \\
				& = \|x - y_0\|^2 \\
				& = f(0)
			\end{align*}
			Since $f$ is smooth and has a local minimum at $t=0$, $f'(0) = 0$. Furthermore, for each $t \in \R$, 
			\begin{align*}
				f(t) 
				& = \|z_0\|^2 - 2 t \Rl(\l z_0, v \r) + t^2\|v\|^2 \\
				& = \|z_0\|^2 - 2 t \l z_0, v \r + t^2\|v\|^2 
			\end{align*}
			so that $f'(0) =  2 \l z_0, v \r$. Thus 
			\begin{align*}
				\l z_0, u \r 
				& = \lam^{-1} \l z_0, \lam u \r \\
				& = \lam^{-1} \l z_0, v \r \\
				& = 0
			\end{align*} 
			Since $u \in E$ is arbitrary, we have that $z_0 \in E^{\perp}$. 
			\tbf{(Uniqueness):} \\
			Suppose that there exist $y_1 \in E$ and $z_1 \in E^{\perp}$ such that $x = y_1 + z_1$ and $\|z_1\| = \|x+E\|$. Since $z_1 = x - y_1$, by assmuption, 
			\begin{align*}
				\|x - y_1\| 
				& = \|z_1\| \\
				& = \|x+E\|
			\end{align*}
			Since $y_1 \in E$, uniqueness in part $(1)$ implies that $y_1 = y_0$. Hence 
			\begin{align*}
				z_1
				& = x - y_1 \\
				& = x - y_0 \\
				& = z_0
			\end{align*}
		\end{itemize}
	\end{enumerate}
\end{proof}

\begin{ex} \lex{ex:hilbert:intro:00020}
	Let $H$ be a Hilbert space and $E \subset H$ a closed subspace of $H$. Then $(E^{\perp})^{\perp} = E$. 
\end{ex}

\begin{proof}\
	\begin{itemize}
		\item Let $x \in (E^{\perp})^{\perp}$ and $x_0 \in H$. \tcb{The previous exercise} implies that there exist unique $y,y_0 \in E$ and $z,z_0 \in E^{\perp}$ such that $x = y + z$, $x_0 = y_0 + z_0$, $\|x\| = \|y+E\|$ and $\|z_0\| = \|x + E\|$. Since $x \in (E^{\perp})^{\perp}$, $z_0 \in E^{\perp}$ and $y_0 \in E$, we have that
		\begin{align*}
			\l x, x_0\r
			& = \l x, y_0 + z_0 \r \\
			& = \l x, y_0 \r + \l x, z_0 \r \\
			& = \l x, y_0 \r \\
			& = \l y + z , y_0 \\
			& = \l y, y_0 \r + \l z, y_0 \r  \\
			& = \l y, y_0 \r 
		\end{align*}
		Similarly, since $y \in E$ and $z_0 \in E^{\perp}$, we have that
		\begin{align*}
			\l y, x_0 \r
			& = \l y, y_0 + z_0\r \\
			& = \l y, y_0 \r + \l y, z_0 \r \\
			& = \l y, y_0 \r    
		\end{align*}
		Therefore 
		\begin{align*}
			\l z, x_0 \r
			& =  \l x - y, x_0\r \\
			& = \l x, x_0\r - \l y, x_0\r \\
			& = \l y, y_0 \r - \l y, y_0 \r \\
			& = 0
		\end{align*}
		Since $x_0 \in H$ is arbitrary, \tcb{a previous exercise} implies that $z = 0$. Hence 
		\begin{align*}
			x
			& = y + z \\
			& = y \\
			& \in E
		\end{align*}
		Since $x \in (E^{\perp})^{\perp}$ is arbitrary, we have that $(E^{\perp})^{\perp} \subset E$. 
		\item Let $y \in E$ and $z \in E^{\perp}$. By definition of $E^{\perp}$, $\l y, z \r = 0$. Since $z \in E^{\perp}$ is arbitrary, we have that for each $z \in E^{\perp}$, $\l y, z \r = 0$. Hence $y \in (E^{\perp})^{\perp}$. Since $y \in E$ is arbitrary, we have that $E \subset (E^{\perp})^{\perp}$. 
	\end{itemize}
	Since $(E^{\perp})^{\perp} \subset E$ and $E \subset (E^{\perp})^{\perp}$, we have that $(E^{\perp})^{\perp} = E$.
\end{proof}

\begin{ex} \lex{ex:hilbert:intro:00021}
	Let $H$ be a Hilbert space and $E,F \subset H$ closed subspaces of $H$. Then $E = F$ iff $E^{\perp} = F^{\perp}$.
\end{ex}

\begin{proof}
	If $E = F$, then clearly $E^{\perp} = F^{\perp}$. \\
	Conversely, suppose that $E^{\perp} = F^{\perp}$. Then \tcb{the previous exercise} and part $(1)$ imply that
	\begin{align*}
		E
		& = (E^{\perp})^{\perp} \\
		& = (F^{\perp})^{\perp} \\
		& = F
	\end{align*}
\end{proof}

\begin{ex} \lex{ex:hilbert:intro:00022} \tbf{Riesz Representation Theorem:} \\
	Let $H$ be a Hilbert space. For each $\phi \in H^*$, there exists a unique $y \in H$ such that for each $x \in H$, $\phi(x) = \l y, x \r$. \\
	\tbf{Hint:} If $x \not \in \ker \phi$, then there exists $z \in (\ker \phi)^{\perp}$ such that $\|z\| = 1$. Consider $u \defeq \phi(x)z - \phi(z)x$. Then $u \in \ker \phi$ and consider $\l z, u \r$. 
\end{ex}

\begin{proof} Let $\phi \in H^*$.
		\begin{itemize}
		\item \tbf{(Existence):} \\
		\begin{itemize}
			\item Suppose that $\phi = 0$. Set $y \defeq 0$. \tcb{A previous exercise} implies that for each $x \in H$, $\phi(x) = \l y, x\r$. 
			\item Suppose that $\phi \neq 0$. Set $E \defeq \ker \phi$. Since $\phi$ is continuous, $E$ is a closed subspace of $H$. Since $\phi \neq 0$, $E \neq H$. \tcb{The previous exercise} then implies that $E^{\perp} \neq \{0\}$. Thus there exists $z \in E^{\perp}$ such that $\|z\| = 1$. Define $y \in H$ by $y \defeq \phi(z)^*z$. Let $x \in H$. Define $u \in H$ by $u \defeq \phi(x)z - \phi(z)x$. Then
			\begin{align*}
				\phi(u)
				& = \phi(x)\phi(z) - \phi(z)\phi(x) \\
				& = 0
			\end{align*}
			Therefore $u \in E$. Since $z \in E^{\perp}$, we have that 
			\begin{align*}
				0
				& = \l z, u \r \\
				& = \l z, \phi(x)z - \phi(z)x \r \\
					& =  \l z, \phi(x)z \r -  \l z, \phi(z)x \r \\
				& = \phi(x) \l z, z \r - \phi(z) \l z, x \r \\
				& = \phi(x) \|z\|^2 - \phi(z) \l z, x \r \\ 
				& = \phi(x) - \l \phi(z)^*z, x \r \\
				& = \phi(x) - \l y, x \r 
			\end{align*}
			Since $x \in H$ is arbitrary, we have that for each $x \in H$, $\phi(x) = \l y, x \r$.
		\end{itemize}
		\item \tbf{(Uniqueness):} \\
		Let $y' \in H$. Suppose that for each $x \in H$, $\phi(x) = \l y', x \r$. Then for each $x \in H$, $\l y - y', x \r = 0$. \tcb{A previous exercise} implies that $y - y' = 0$. Thus $y' = y$. 
	\end{itemize}
\end{proof}

\begin{defn} \lex{ex:hilbert:intro:00022.1}
	Let $H$ be a Hilbert space and $y \in H$. We define the \tbf{adjoint of $y$}, denoted $y^* \in H^*$, by $y^*(x) \defeq \l y, x \r$. 
\end{defn}

\begin{ex} \lex{ex:hilbert:intro:00022.2}
	Let $H$ be a Hilbert Space. We define $\flat: \bar{H} \rightarrow H^*$ by $\flat(y) \defeq y^*$. Then $\flat \in L(\bar{H}, H^*)$ and $\flat$ is an isometric isomorphism.
\end{ex}

\begin{proof}\
	\begin{itemize}
		\item Let $y, z \in \bar{H}$ and $\lam \in \C$. Then for each $x \in H$,
		\begin{align*}
			\flat(y + \lam \bar{\cdot} z)(x)
			& = (y + \lam \bar{\cdot} z)^*(x) \\
			& = \l y + \lam \bar{\cdot} z, x \r \\
			& = \l y + \lam^* z, x \r \\
			& = \l y , x \r +  \l \lam^* z, x \r \\
			& = \l y , x \r +  \lam \l z, x \r \\
			& = y*(x) + \lam z^*(x) \\
			& = (y^* + \lam z^*)(x) \\
			& = (\flat(y) + \lam \flat(z))(x). 
		\end{align*}
		Thus $\flat(y + \lam \bar{\cdot} z) = \flat(y) + \lam \flat(z)$. Since $y,z \in \bar{H}$ and $\lam \in \C$ are arbitrary, we have that $\flat$ is linear. 
		\item Let $y \in \bar{H}$. The Cauchy-Schwarz inequality implies that for each $x \in H$, 
		\begin{align*}
			|\flat(y)(x)|
			& |y^*(x)| \\
			& = |\l y, x\r| \\
			& \leq \|y\| \|x\|.
		\end{align*}
		Thus $\|\flat(y)\| \leq \|y\|$. Since $y \in \bar{H}$ is arbitrary, we have that for each $y \in \bar{H}$, $\|\flat(y)\| \leq \|y\|$. Hence $\flat \in L(\bar{H}, H^*)$ and $\|\flat\| \leq 1$. We note that
		\begin{align*}
			\|\flat\|
			& = \sup_{\|y\| = 1} \|\flat(y)\| \\
			& = \sup_{\|y\| = 1} \|y^*\| \\
			& = \sup_{\|y\|=1} \bigg[\sup_{\|x\|=1} |y^*(x)| \bigg] \\
			& = \sup_{\|y\|=1} \bigg[\sup_{\|x\|=1} |\l y, x \r| \bigg] \\
			& \geq \sup_{\|y\|=1} |\l y, y \r|  \\
			& = \sup_{\|y\|=1} \| y\|^2  \\
			& = 1,
		\end{align*}
		Therefore $\|\flat\| = 1$. \rex{ex:banach:bounded_ops:0003} implies that $\flat$ is an isometry.
		\item Since $\flat$ is an isometry, $\flat$ is injective. \rex{ex:hilbert:intro:00022} implies that $\flat$ is surjective. Hence $\flat$ is a bijection. Since $\flat$ is an isometry, $\| \flat^{-1} \| = 1$. Hence $\flat$ is an isomorphism. \tcr{reword this an make exercise in previous chapter about surjective isometries being isomorphisms}
	\end{itemize}
\end{proof}

\begin{defn} \lex{ex:hilbert:intro:00023}
	Let $H$ be a Hilbert Space. We define $\sharp: H^* \rightarrow \bar{H}$ by $\sharp \defeq \flat^{-1}$.
\end{defn}

\begin{ex} \lex{ex:hilbert:intro:00024}
Let $H$ be a Hilbert space and $S \subset H$. Suppose that  $S$ is orthonormal. Then the following are equivalent: 
\begin{enumerate}
\item For each $x \in H$, if for each $u \in S$, $\l u, x \r = 0$, then $x =0$.
\item For each $x \in H$, there exist $(u_j)_{j\in \N} \subset S$ such that $x = \sum\limits_{j \in \N} \l u_j, x\r u_j$ and for each $u \not \in (u_j)_{j\in \N}$, $\l u, x\r =0$.
\item For each $x \in H$, $\|x\|^2 = \sum\limits_{u \in S} | \l u, x \r |^2$.
\end{enumerate}
\end{ex}

\begin{proof}\
\begin{itemize}
		\item $(1) \implies (2)$:\\
		Suppose that for each $x \in H$, if for each $u \in S$, $\l u, x \r = 0$, then $x =0$. Let $x \in H$. Set $S_{*} \defeq \{u \in S: \l u, x \r \neq 0 \}$. The previous exercise implies that $S_{*}$ is countable. Write $S_* = (u_j)_{j \in \N}$. The previous exercise tells us that $\sum\limits_{j \in \N} |\l u_j, x \r|^2 \leq \|x\|^2$ and hence converges. Thus for $\ep >0$, there exist $N \in \N$ such that for each  $m,n \in \N$, $m, n \geq N$ implies that if $m < n$, then $$\sum_{m+1}^{n} |\l u_j, x \r |^2 < \ep$$
		Define $(y_n)_{n \in \N} \subset H$ by $$y_n = \sum_{j=1}^{n} \l u_j, x \r u_j$$ 
		Then for each $m,n \in \N$, $m, n \geq N$ implies that if $m < n$, then 
		\begin{align*}
		\|y_n - y_m\|^2 
		& = \bigg \|\sum_{1}^{n} \l u_j, x \r u_j  - \sum_{1}^{m} \l u_j, x \r u_j  \bigg \|^2 \\
		&= \bigg \|\sum_{m+1}^{n} \l u_j, x \r u_j \bigg \|^2 \\
		&= \sum_{m+1}^{n} |\l u_j, x \r |^2 \\
		& < \ep
		\end{align*}
		So $(y_n)_{n \in \N}$ is Cauchy. Since $H$ is complete, there exists $y \in H$ such that $y_n \rightarrow y$. By definition, $$y = \sum\limits_{j \in \N}\l u_j, x \r u_j $$
		Continuity of $\l \cdot, \cdot\r: H \times H \rightarrow \C$ implies that 
		\begin{enumerate}
		\item for each $u \in S \setminus  S_*$, 
		\begin{align*}
		\l u, x - y\r 
		&= \l u, x \r  - \l u, y \r \\
		&=  \l u, x \r - \limn \l u, y_n\r \\
		&=  \l u, x \r - \limn \sum_{j=1}^n \l u_j, x\r \l u, u_j \r \\
		&= 0 - 0 \\
		&=0
		\end{align*}
		\item for each $k \in \N$, 
		\begin{align*}
		\l u_k, x - y\r 
		&= \l u_k, x \r  - \l u_k, y \r \\
		&= \l u_k, x \r - \limn \l u_k, y_n\r \\
		&= \l u_k, x \r - \limn \sum_{j=1}^n \l u_j, x\r \l u_k, u_j \r \\
		&= \l u_k, x \r  - \l u_k, x\r \\
		&= 0
		\end{align*}
		\end{enumerate}
		So for each $u \in S$, $\l u, x-y\r =0$. By assumption, $x-y = 0$ and hence 
		$$x = \sum\limits_{j \in \N}\l u_j, x \r u_j$$
		
		\item $(2) \implies (3)$:\\
		Suppose that for each $x \in H$, there exist $(u_j)_{j\in \N} \subset S$ such that $x = \sum\limits_{j \in \N} \l u_j, x\r u_j$ and for each $u \not \in (u_j)_{j\in \N}$, $\l u, x\r =0$. Then continuity of $\|\cdot\|:H \rightarrow \Rg$ implies that
		\begin{align*}
		\|x\|^2 
		&= \bigg \|  \limn \sum_{j=1}^n \l u_j, x\r u_j \bigg \|^2 \\
		&= \limn \bigg \|\sum_{j=1}^n \l u_j, x\r u_j \bigg \|^2 \\
		&= \limn \sum_{j=1}^n |\l u_j, x\r |^2 \\
		&= \sum_{j \in \N}|\l u_j, x\r |^2  \\
		&= \sum_{u \in S}|\l u, x\r |^2 
		\end{align*}
		\item $(3) \implies (1)$:\\
		Suppose that for each $x \in H$, $\|x\|^2 = \sum\limits_{u \in S} | \l u, x \r |^2$. Let $x \in H$. Suppose that for each $u \in S$, $\l u,x \r = 0$. Then 
		\begin{align*}
		\| x\|^2 
		&= \sum_{u \in S}  | \l u, x \r |^2 \\
		&= 0
		\end{align*}
		So $x =0$
\end{itemize}
\end{proof}

\begin{defn} \ld{def:hilbert:intro:00025}
	Let $H$ be a Hilbert space and $S \subset H$. Then $S$ is said to be an \tbf{orthonormal basis for $H$} if 
	\begin{enumerate}
	\item $S$ is orthonormal
	\item for each $x \in H$, if for each $u \in S$, $\l u, x \r = 0$, then $x =0$
	\end{enumerate}
\end{defn}






































\newpage

\section{Operators on Hilbert Spaces}

\subsection{Adjoint Operators}

\begin{note}
	We recall the definition of the adjoint of a bounded linear map from \rex{def:banach:duality:0007}.
\end{note}

\begin{ex} \lex{ex:hilbert:operators:0001}
	Let $H_1,H_2$ be Hilbert spaces and $A \in L(H_1, H_2)$. Then there exists a unique $B \in L(H_2, H_1)$ such that for each $x_1 \in H_1$ and $x_2 \in H_2$, 
	$$\l x_1 , Bx_2 \r = \l A x_1 , x_2 \r$$  
	\tbf{Hint:} $B$ is essentially $A^*$ where we identify $H^*$ with $\bar{H}$. 
\end{ex}

\begin{proof}
	 Define $B \defeq \sharp \circ A^* \circ \flat$.  
\end{proof}

\begin{defn} \ld{def:hilbert:operators:0002} \tbf{Adjoint of an Operator:} \\
	Let $H_1,H_2$ be a Hilbert space and $A \in L(H_1, H_2)$. We define the \tbf{adjoint of $A$}, denoted $A^*$, to be the unique $B \in L(H_2, H_1)$ such that for each $x_1 \in H_1$ and $x_2 \in H_2$, 
	$$\l x_1 , Bx_2 \r = \l A x_1 , x_2 \r$$  
\end{defn}

\begin{ex} \lex{ex:hilbert:operators:0003}
	Let $H_1, H_2, H_3 \in \Obj(\Hilb)$.  
	\begin{enumerate}
		\item For each $A, B \in L(H_1, H_2)$ and $\lam \in \C$,
		\begin{enumerate}
			\item $(A^*)^* = A$
			\item $(A + B)^* = A^* + B^*$
			\item $(\lam A)^* = \lam^*A^*$
		\end{enumerate}
		\item For each $A \in L(H_1, H_2)$ and $B \in L(H_2, H_3)$,
		\begin{enumerate}
			\item $(BA)^* = A^*B^*$
			\item $H_1 = H_3$ implies that $A$ and $B$ commute iff $A^*$ and $B^*$ commute.
		\end{enumerate}
	\end{enumerate}
\end{ex}

\begin{proof} \
	\begin{enumerate}
		\item Let $A, B \in L(H_1, H_2)$, $\lam \in \C$.  
		\begin{enumerate}
			\item Let $x_1 \in H_1$ and $x_2 \in H_2$. Then 
			\begin{align*}
				\l x_2 , A x_1 \r^*	
				& = \l A x_1 , x_2 \r \\
				& = \l  x_1, A^*x_2  \r^* \hspace{.5cm} \text{(by definition)}\\
				& = \l  A^*x_2, x_1 \r^*.
			\end{align*}
			Thus $\l x_2 , A x_1 \r = \l  A^*x_2, x_1 \r$. Since $x_1 \in H_1$ and $x_2 \in H_2$ are arbitrary, we have that for each $x_1 \in H_1$ and $x_2 \in H_2$, $\l x_2 , A x_1 \r = \l  A^*x_2, x_1 \r$. Uniqueness of $(A^*)^*$ implies that $(A^*)^* = A$.
			\item For each $x_1 \in H_1$, $x_2 \in H_2$,
			\begin{align*}
				\l (A+B) x_1 , x_2 \r 
				& = \l A x_1 + B x_1 , x_2 \r \\ 
				& = \l Ax_1 ,  x_2 \r + \l B x_1 , x_2 \r \\
				& = \l x_1 ,A^* x_2 \r + \l x_1 , B^*x_2 \r \\
				& = \l x_1 ,A^* x_2  +  B^*x_2 \r \\
				& = \l x_1 ,(A^* + B^*) x_2 \r.
			\end{align*}
			Uniqueness of $(A+B)^*$ implies that $(A+B)^* = A^* + B^*$.
			\item For each $x_1 \in H_1$, $x_2 \in H_2$, 
			\begin{align*}
				\l (\lam A) x_1 ,  x_2 \r 
				& = \l \lam (A x_1) ,  x_2 \r \\
				& = \lam^* \l Ax_1 ,  x_2 \r \\
				& = \lam^* \l x_1 , A^{*}x_2 \r \\
				& = \l x_1 , \lam^* (A^{*}x_2) \r \\
				& = \l x_1 , (\lam^* A^{*})x_2 \r .
			\end{align*}
			Uniqueness of $(\lam A)^*$ implies that $(\lam A)^* = \lam^*A^*$.
		\end{enumerate}
		\item Let $A \in L(H_1, H_2)$ and $B \in L(H_2, H_3)$.
		\begin{enumerate}
			\item For each $x_1 \in H_1$, $x_2 \in H_2$,
			\begin{align*}
				\l BAx_1 , x_2 \r  
				&= \l Ax_1 , B^* x_2 \r \\
				&= \l x_1 , A^*B^* x_2 \r 
			\end{align*}
			Uniqueness of $(BA)^*$ implies that $(BA)^* = A^*B^*$.
			\item Suppose that $H_1 = H_3$. If $A$ and $B$ commute, then
			\begin{align*}
				A^*B^*
				&= (BA)^* \\
				&= (AB)^* \\
				&= B^*A^*.
			\end{align*}
			Conversely, if $A^*$ and $B^*$ commute then 
			\begin{align*}
				AB
				&= (B^*A^*)^* \\
				&= (A^*B^*)^* \\
				&= BA.
			\end{align*}
		\end{enumerate}
	\end{enumerate}
\end{proof}

\begin{defn} \ld{def:hilbert:operators:0004} \tbf{Unitary Maps:}
	Let $H_1, H_2 \in \Obj(\Hilb)$ and $U \in L(H_1, H_2)$. Then $U$ is said to be \tbf{unitary} if $U$ is invertible and for each $x,y \in H_1$, $\l Ux, Uy \r = \l x, y \r $.  
\end{defn}

\begin{ex} \lex{ex:hilbert:operators:0005}
	Let $H_1, H_2 \in \Obj(\Hilb)$ and $U \in L(H_1, H_2)$. Then the following are equivalent:
	\begin{enumerate}
		\item $U$ is unitary
		\item $U$ is invertible and $U^{-1} = U^*$
		\item $U$ is surjective and $U$ is an isometry \\
	\end{enumerate}
\end{ex}

\begin{proof}\
	\begin{itemize}
		\item $(1) \implies (2)$: \\ 
		Suppose that $U$ is unitary. Then $U$ is invertible and for each $x_1, x_2 \in H_1$, $\l Ux_1, U x_2 \r = \l x_1, x_2 \r$. Let $x,y \in H$. Then
		\begin{align*}
			\l y, x\r
			& = \l Uy, Ux \r \\
			& = \l U^*Uy, x\r.
		\end{align*}
		Therefore
		\begin{align*}
			0
			& = \l U^*U y, x\r - \l y, x \r \\
			& = \l U^*U y - y, x \r \\
			& = \l (U^*U -I)y, x \r.
		\end{align*}
		Since $x \in H_1$ and $y \in H_2$ are arbitrary, for each $x \in H_1$ and $y \in H_2$, $\l (U^*U - I)y, x \r = 0$. \rex{ex:hilbert:intro:00006.0.1} implies that $(U^*U - I) = 0$. Hence $U^*U = I$. Therefore
		\begin{align*}
			U
			& = UI \\
			& = U(U^*U) \\
			& = (UU^*)U.
		\end{align*}
		Since $U$ is invertible, we have that
		\begin{align*}
			I
			& = UU^{-1} \\
			& = [(UU^*)U]U^{-1} \\
			& = (UU^*)(UU^{-1}) \\
			& = (UU^*)I \\
			& = UU^*.
		\end{align*}
		Thus $U^*U = I$ and $UU^* = I$. Hence $U^{-1} = U^*$.
		\item $(2) \implies (3)$: \\ 
		Suppose that $U$ is invertible and $U^{-1} = U^*$. Since $U$ is invertible, $U$ is surjective. Let $x \in H_1$. Then 
		\begin{align*}
			\|Ux\|^2 
			& = \l Ux, Ux \r \\
			& = \l x, U^*Ux \r \\
			& = \l x, x \r \\
			& = \|x\|^2.
		\end{align*}
		Hence $\|Ux\| = \|x\|$. Since $x \in X$ is arbitrary, we have that for each $x \in X$, $\|Ux\| = \|x\|$. \rex{ex:banach:bounded_ops:0003} implies that $U$ is an isometry.
		\item $(3) \implies (1)$: \\ 
		Suppose that $U$ is surjective and $U$ is an isometry. Since $U$ is an isometry, $U$ is injective. Hence $U$ is a bijection. Since $U$ is an isometry, $U^{-1}$ is an isometry. Thus $U^{-1} \in L(H_2, H_1)$ and $U$ is invertible. Let $x,y \in H_1$. The polarization identity \rex{ex:hilbert:intro:00006.2} and \rex{ex:banach:bounded_ops:0003} imply that  
		\begin{align*}
			\l Ux, Uy \r
			& = \frac{1}{4} \bigg[  \|Ux+Uy\|^2 - \|Ux-Uy\|^2 - i\|Ux+iUy\|^2 + i\|Ux-iUy\|^2 \bigg] \\
			& = \frac{1}{4} \bigg[  \|U(x+y)\|^2 - \|U(x-y)\|^2 - i\|U(x+iy)\|^2 + i\|U(x-iy)\|^2 \bigg] \\
			& = \frac{1}{4} \bigg[  \|x+y\|^2 - \|x-y\|^2 - i\|x+iy\|^2 + i\|x-iy\|^2 \bigg] \\
			& = \l x, y \r.
		\end{align*}
		Since $x, y \in H_1$ are arbitrary, we have that for each $x, y \in H_1$, $\l Ux, Uy \r = \l x, y \r$. Hence $U$ is unitary. 
	\end{itemize}
\end{proof}

\begin{defn} \ld{def:hilbert:operators:0006}
	Let $H_1, H_2 \in \Obj(\Hilb)$. We define $U(H_1, H_2) \defeq \{U \in L(H_1, H_2): \text{$U$ is unitary}\}$. When $H_1 = H_2 = H$, we write $U(H)$ in place of $U(H, H)$. 
\end{defn}

\begin{ex} \lex{ex:hilbert:operators:0007}
	Let $H \in \Obj(\Hilb)$. Then $U(H)$ is a group.
\end{ex}

\begin{proof}
		\tcr{FINISH!!!}
\end{proof}

\begin{ex} \lex{ex:hilbert:operators:0007.1}
	Let $H \in \Obj(\Hilb)$. Then $\flat \in U(\bar{H}, H^*)$. 
\end{ex}

\begin{proof}
	\tcr{FINISH!!!}
\end{proof}

\begin{defn} \ld{def:hilbert:operators:0008}
	Let $H$ be a Hilbert space and $Q \in L(H)$. Then $Q$ is said to be \tbf{self-adjoint} if $$Q = Q^{*}$$
\end{defn}

\begin{ex} \lex{ex:hilbert:operators:0009}
	Let $H \in \Obj(\Hilb)$. Then $I$ is unitary and self-adjoint.
\end{ex}

\begin{proof}
	For each $x, y \in H$, $\l I x, I y \r = \l x, y\r$. So $I$ is unitary. \rex{ex:hilbert:operators:0005} implies that 
	\begin{align*}
		I^*
		& = I^{-1} \\
		& = I.
	\end{align*}
	Thus $I$ is self-adjoint.
\end{proof}






















\vspace{10cm}
\tcr{\subsection{Spectral Theory (move to section on compact operators on hilbert space or banach algebra)} }


\begin{ex} \lex{}
	Let $H$ be a Hilbert space and $Q \in L(H)$. If $Q$ is a self-adjoint then 
	\begin{enumerate}
		\item the eigenvalues of $Q$ are real. \tcr{reword}
		\item the eigenvectors of $Q$ corresponding to distinct eigenvalues are orthogonal.
	\end{enumerate}
\end{ex}

\begin{proof}
	Suppose that $Q$ is self-adjoint.
	\begin{enumerate}
		\item Let $\lam$ be an eigenvalue of $Q$ with corresponding eigenvector $x$. Then 
		\begin{align*}
			\lam \l x , x\r
			&= \l x , Q x\r \\
			&= \l Q x , x\r \\
			&= \lam^* \l x , x\r
		\end{align*}
		Thus $\lam = \lam^*$ and is real
		
		\item Let $\lam_1$ and $\lam_2$ be eigenvalues of $Q$ with corresponding eigenvectors $x_1$ and $x_2$. Suppose that $\lam_1 \neq \lam_2$. Then 
		\begin{align*}
			\lam_2 \l x_1 ,  x_2\r
			&= \l x_1 , Q x_2\r\\
			&= \l Q x_1 ,  x_2\r\\
			&= \lam_1 \l x_1 ,  x_2\r
		\end{align*}
		So $(\lam_2 - \lam_1)\l x_1 ,  x_2\r = 0$. Which implies that $\l x_1 ,  x_2\r=0$
	\end{enumerate}
\end{proof}

\begin{ex} \lex{}
	Let $H$ be a Hilbert space, $A, B \in L(H)$ and $ \lam \in \R$. Suppose that $A, B$ are self-adjoint. If $A$ and $B$ commute and then $\lam AB$ is self-adjoint.
\end{ex}

\begin{proof}
	\begin{align*}
		(\lam AB)^{*}
		&= \lam^* (AB)^{*} \\
		&= \lam B^{*} A^{*} \\
		&= \lam B A \\
		&= \lam AB
	\end{align*}
\end{proof}

\begin{defn} \ld{}\tbf{Adjoint of a Vector:} \\
	Let $H$ be a Hilbert space and $x \in H$. We define the \tbf{adjoint} of $x$, denoted $x^* \in H^*$, by $x^* y = \l x, y \r$. 
\end{defn}

\begin{note}
	In mathematics, where linearity of the inner product is in the first argument, $x^{*}$ is typically referred to by $u_{x} \in H^{*} $ where $u_{x}(y) = \l y, x\r$. In physics, where the inner product with linearity in the second argument, $x^{*} \phi$ is usually written in the so-called ``bra-ket" notation as $\l x | \phi \r$ which works smoothly since it aligns with the linearity of $u_{x}(\phi_1 + \lam \phi_2)$ and the conjugate-linearity of $u_{x_1 + \lam x_2}(\phi)$. In this way, it generalizes the notation for $\l x, y\r = x^T y$ for $\R^n$ to $\l x, y\r = x^*y$ for $\C^n$. 
\end{note}

\begin{ex} \lex{}
	Let $H$ be a Hilbert space, $x, y \in H$ and $\lam \in \C$. Then 
	\begin{enumerate}
		\item $(x + y)^* =  x^* + y^*$
		\item $(\lam x)^* = \lam^* x^*$
	\end{enumerate}
\end{ex}

\begin{proof}
	Clear.
\end{proof}

\begin{defn} \ld{}
	Let $H$ be a Hilbert space, $x, y \in H$ and $A \in L(H)$. We define 
	\begin{enumerate}
		\item $x^* A \in H^*$ by $(x^*A) y = x^*(A y)$
		\item $x y^* \in L(H)$ by $(x y^*) z = (y^*z) x$
	\end{enumerate}
\end{defn}

\begin{ex} \lex{}
	Let $H$ be a Hilbert space, $A \in L(H)$ and $x \in H$. Then $$(A x)^*= x^*A^*$$
\end{ex}

\begin{proof}
	Let $y \in H$. Then 
	\begin{align*}
		(Ax)^*	y 
		&= \l Ax, y \r \\
		&= \l x, A^* y \r \\
		&= x^*A^* y
	\end{align*}
\end{proof}

\begin{defn} \ld{}\tbf{Commutator:} \\
	Let $H$ be a Hilbert space and $A, B \in L(H)$. The \tbf{commutator} of $A$ and $B$, denoted $[A,B]$, is defined by $$[A,B] = AB - BA$$
\end{defn}

\begin{ex} \lex{}
	Let $H$ be a Hilbert space and $A,B, C \in L(H)$. Then 
	\begin{enumerate}
		\item $[AB,C] = A[B,C] + [A,C]B$
		\item $[A, BC] = B[A, C] + [A,B]C$
	\end{enumerate}
\end{ex}

\begin{proof} \
	\begin{enumerate}
		\item 
		\begin{align*}
			[AB,C]
			&= ABC - CAB\\
			&= ABC - ACB + ACB -CAB\\
			&= A(BC - CB) + (AC-CA)B\\
			&= A[B,C] + [A,C]B
		\end{align*}
		\item Similar to (1).
	\end{enumerate}
\end{proof}























\begin{ex}
	Let $H_1,H_2, H_3, H_4 \in \Obj(\Hilb)$ and $A \in L(H_3, H_4)$, $B \in L(H_2,H_3)$ and $C \in L(H_1, H_2)$. Suppose that $H,K, L$ are separable. Let $(e^2_j)_{j \in \N}$ and $(e^3_j)_{j \in \N}$ be orthonormal bases for $H_2, H_3$ and $H_4$ respectively. Then 
	$$ABC = \sum_{j \in \N} \sum_{k \in \N} \l e^3_j, Be^2_k \r (Ae^3_j)^*(e^2_k C)$$ 
	\tcr{FINISH!!!}
\end{ex}

\begin{proof}
	content...
\end{proof}
































\newpage
\section{Subspaces of Hilbert Spaces}


\subsection{Introduction}

\begin{ex}
	Let $(H, \l \cdot, \cdot \r) \in \Obj(\Hilb)$ and $E \subset H$ a closed subspace of $H$. Then $(E, \l \cdot, \cdot \r|_{E \times E})$ is a Hilbert space.  
\end{ex}

\begin{proof}
	Clearly $(E, \l \cdot, \cdot \r|_{E \times E})$ is an inner product space. Since $E$ is closed and $H$ is complete, $E$ is complete. Hence $(E, \l \cdot, \cdot \r|_{E \times E})$ is a Hilbert space.
\end{proof}












































\subsection{Orthogonal Projections}

\begin{defn}
	Let $H \in \Obj(\Hilb)$ and $P \in L(H)$. Then $P$ is said to be \tbf{idemptoent} if $P^2 = P$.
\end{defn}

\begin{ex}
	Let $H \in \Obj(\Hilb)$ and $P \in L(H)$. Then $P$ is idempotent iff $I - P$ is idempotent.
\end{ex}

\begin{proof}\
	\begin{itemize}
		\item $(\implies)$: \\
		Suppose that $P$ is idempotent. Then 
		\begin{align*}
			(I - P)^2
			& = (I - P)(I - P) \\
			& = I^2 - IP - PI + P^2 \\
			& = I -2P + P \\
			& = I - P
		\end{align*}
		So $I -P$ is idempotent.
		\item $(\impliedby):$ \\
		Suppose that $I - P$ is idempotent. Part $(1)$ implies that $I - (I - P) = P$ is idempotent.
	\end{itemize}
\end{proof}

\begin{ex}
	Let $H \in \Obj(\Hilb)$, $E \subset H$ a closed subspace of $H$ and $P \in L(H)$. If $P|_E = I_E$ and $\Img P \subset E$, then $P$ is idempotent
\end{ex}

\begin{proof}
	Suppose that $P|_E = I_E$ and $\Img P \subset E$. Let $x \in H$ and define $y \in E$ by $y \defeq P(x)$. Then 
	\begin{align*}
		P^2(x)
		& = P(y) \\
		& = P|_E(y) \\
		& = I_E(y) \\
		& = y \\
		& = P(x) 
	\end{align*} 
	Since $x \in H$ is arbitrary, we have that $P^2 = P$ and $P$ is idempotent.
\end{proof}

\begin{defn}
	Let $H \in \Obj(\Hilb)$ and $E \subset H$ a closed subspace of $H$. We define the \tbf{orthogonal projection onto $E$}, denoted $P_E: H \rightarrow E$ by 
	$$P_E(x) = \argmin\limits_{y \in E^{\perp}} \| x - y\|$$
\end{defn}

\begin{note}
	\rex{ex:hilbert:intro:00019} implies that $P_E$ is well-defined. 
\end{note}

\begin{ex}
	Let $H \in \Obj(\Hilb)$ and $E \subset H$ a closed subspace of $H$. Then 
	\begin{enumerate}
		\item $P_E|_E = I_E$
		\item $\Img P_E = E$
		\item $P_E$ is linear
		\item $P_E \in L(H, E)$ and $\|P_E\| =  
		\begin{cases}
			0, & E = \{0\} \\
			1 & E \neq \{0\}
		\end{cases}$
		\item $P_E$ is self-adjoint
		\item $P_E$ is idempotent
		\item $\ker P_E = E^{\perp}$
	\end{enumerate}
\end{ex}

\begin{proof}\
	\begin{enumerate}
		\item Let $x \in E$. Then 
		\begin{align*}
			P_E(x)
			& = \argmin_{y \in E^{\perp}} \|x - y\| \\
			& = x
		\end{align*}
		Since $x \in E$ is arbitrary, $P_E|_E = I_E$.
		\item By definition, $\Img P_E \subset E$. The previous part implies that $E \subset \Img P_E$. Thus $\Img P_E = E$. 
		\item  Let $x_1,x_2 \in H$ and $\lam \in \C$. \tcb{A previous exercise in the introduction section} implies that there exist unique $y_1, y_2 \in E$ and $z_1, z_2 \in E^{\perp}$ such that $x_1 = y_1 + z_1$, $x_2 = y_2 + y_2$, $\|x_1 + E\| = \|z_1\|$ and $\|x_2 + E\| = \|z_2\|$. Then $y_1 + \lam y_2 \in E$, $z_1 + \lam z_2 \in E^{\perp}$ and $x_1 + \lam x_2 = (y_1 + \lam y_2) + (z_1 + \lam z_2)$. \tcb{An exercise in the introduction section} implies that $\|x_1 + \lam x_2 + E\| = \|z_1 + \lam z_2\|$. Uniqueness implies that 
		\begin{align*}
			P_E(x_1 + \lam x_2) 
			& = y_1 + \lam y_2 \\
			& = P_E(x_1) + \lam P_E(x_2) 
		\end{align*} 
		Since $x_1,x_2 \in H$ and $\lam \in \C$ are arbitrary, $P_E$ is linear. 
		\item Let $x \in H$. Then there exist unique $y \in E$ and $z \in E^{\perp}$ such that $x = y+z$ and $\|x + E\| = \|z\|$. The Pythagorean theorem implies that 
		\begin{align*}
			\|P_E(x)\|^2
			& = \|y\|^2 \\
			& \leq \|y\|^2 + \|z\|^2 \\
			& = \|y + z\|^2 \\
			& = \|x\|^2
		\end{align*}
		So $\|P_E(x)\| \leq \|x\|$. Since $x \in H$ is arbitrary, $P_E \in L(H, E)$ and $\|P_E\| \leq 1$. If $E = \{0\}$, then $P_E = 0$ and therefore $\|P_E\| = 0$. Suppose that $E \neq \{0\}$. Then there exists $y \in E$ such that $\|y\| = 1$. Hence 
		\begin{align*}
			\|P_E\|
			& = \sup_{y' \neq 0} \|y'\|^{-1}\|P_E(y')\| \\
			& \geq \| P_E(y) \| \\
			& = \| y \| \\
			& = 1  
		\end{align*}
		So $\|P_E\| = 1$.
		\item \tcb{FINISH!!!}
		\item Since $P \in L(H)$, $P|_E = I_E$ and $\Img P \subset E$, \tcb{a previous exercise} implies that $P_E$ is idempotent.
		\item 
		\begin{itemize}
			\item Let $x \in \ker P_E$. \rex{ex:hilbert:intro:00019} implies that there exists unique $z_0 \in E^{\perp}$ such that $x = P_E(x) + z_0$ and $\|z_0\| = \|x + E\|$. Since $x \in \ker P_E$, we have that 
			\begin{align*}
				x =  \\
				& = P_E(x) + z_0 \\
				& = 0 + z_0 \\
				& = z_0.
			\end{align*}
			Hence $x \in E^{\perp}$. Since $x \in \ker P$ is arbitrary, we have that $\ker P \subset E^{\perp}$. 
			\item Let $x \in E^{\perp}$. Then \rex{ex:hilbert:intro:00019} implies that $P_E(x) = 0$. Thus $x \in \ker P_E$. Since $x \in E^{\perp}$ is arbitrary, we have that $E^{\perp} \subset \ker P_E$.
		\end{itemize}
		Since $E^{\perp} \subset \ker P_E$ and $P_E \subset \ker E^{\perp}$, we have that $\ker P_E = E^{\perp}$.
	\end{enumerate}
\end{proof}

\begin{ex}
	\tcr{FINISH!!!}
\end{ex}

\begin{proof}
	\begin{align*}
		\|A\|^2
		& = (\sup_{\|x\| = 1} \|Ax\|)^2 \\
		& = \sup_{\|x\| = 1} \|Ax\|^2 \\
		& = \sup_{\|x\| = 1} \l Ax, Ax \r \\
		& = \sup_{\|x\| = 1} \l x, A^*A x \r \\
		& = 
	\end{align*}
	\tcr{FINISH!!!}
\end{proof}

\begin{ex}
	Let $H \in \Obj(\Hilb)$, $E \subset H$ a closed subspace of $H$. Then there exists a unique $P \in L(H)$ such that 
	\begin{enumerate}
		\item $P$ is self-adjoint
		\item $P$ is idempotent
		\item $\Img P = E$ 
	\end{enumerate}
	\tbf{Hint:} for uniqueness, if $P$ and $Q$ satsify $(1)-(3)$, consider $(P - Q)^*(P - Q)$
\end{ex}

\begin{proof}\
	\begin{itemize}
		\item \tbf{(Existence):} \\
		$P_E$ satisfies $(1) - (3)$.
		\item \tbf{(Uniqueness):} \\
		Let $Q \in L(H, E)$. Suppose that $Q$ is self-adjoint, $Q$ is idempotent and $\Img Q = E$. The previous exercise implies that $P|_E, Q|_E = I_E$. Then 
		\begin{align*}
			(P - Q)^*(P - Q)
			& = P^*P - P^*Q - Q^*P + Q^*Q \\
			& = P^2 - PQ - QP + Q^2 \\
			& = P^2 - P|_E Q - Q|_E P + Q^2 \\
			& = P - Q - P + Q \\
			& = 0
		\end{align*}
		\tcr{make exercise about $\|T\| = \|T^*\|$ and $\|T\| = \max \l x , T y\r: \|x\|, \|y\| = 1$} 
		\begin{align*}
		\|(P-Q\|^2 
		& = \|(P-Q)^*(P-Q)\| \\
		& = \|0\| \\
		& = 0.
		\end{align*}
		Thus $\|P-Q\| = 0$ and therefore $P - Q = 0$. Hence $P=Q$
		Thus $\|(P-Q)^*(P-Q)\| $
		\tcr{FINISH!!!}
	\end{itemize}
\end{proof}











































\newpage
\section{Direct Sums of Hilbert spaces}

\tcr{need to find universal property for direct sum}

\begin{defn}
	Let $(H_\al, \l\cdot, \cdot \r_{\al})_{\al \in A} \subset \Obj(\Hilb)$. We define $s_A:\prod\limits_{\al \in A} H_{\al} \rightarrow \Rg^A$ by $s_A(x) = (\|x_{\al}\|_{\al})_{\al \in A}$.
\end{defn}

\begin{ex}	
	Let $(H_\al, \l\cdot, \cdot \r_{\al})_{\al \in A} \subset \Obj(\Hilb)$. Then  
	\begin{enumerate}
		\item for each $x,y \in s_A^{-1}(l^2(A))$, $(\l x_{\al}, y_{\al} \r_{\al})_{\al \in A} \in l^1(A)$
		\item for each $x,y \in \prod\limits_{\al \in A} H_{\al}$, $\|s_A(x) - s_A(y)\|_2 \leq \|s_A(x - y)\|_2$
	\end{enumerate}
		\item 
\end{ex}

\begin{proof}\
	\begin{enumerate}
		\item Let $x,y \in s_A^{-1}(l^2(A))$. Then $s_A(x), s_A(y) \in l^2(A)$. Therefore $s_A(x)s_A(y) \in l^1(A)$ and
		\begin{align*}
			\|\l x_{\al}, y_{\al} \r_{\al}\|_1
			& = \sum_{\al \in A} |\l x_{\al}, y_{\al} \r_{\al}| \\
			& \leq \sum_{\al \in A} \| x_{\al} \|_{\al} \|y_{\al}\|_{\al} \\
			& = \|s_A(x)s_A(y)\|_1 \\
			& < \infty 
		\end{align*}
		\item Let $x,y \in \prod\limits_{\al \in A} H_{\al}$. The reverse triangle inequality implies that 
		\begin{align*}
			\|s_A(x) - s_A(y)\|_2^2
			& = \|(\|x_{\al}\|_{\al})_{\al \in A} - (\|y_{\al}\|_{\al})_{\al \in A} \|_2^2 \\
			& = \|(\|x_{\al}\|_{\al} - \|y_{\al}\|_{\al})_{\al \in A} \|_2^2 \\
			& = \sum_{\al \in A} |\|x_{\al}\|_{\al} - \|y_{\al}\|_{\al}|^2 \\
			& \leq \sum_{\al \in A} \|x_{\al} - y_{\al} \|_{\al}^2 \\
			& = \|(\| (x- y)_{\al} \|_{\al})_{\al \in A}\|_2^2 \\
			& = \|s_A(x - y)\|_2^2
		\end{align*}
		Thus $\|s_A(x) - s_A(y)\|_2 \leq \|s_A(x - y)\|_2$.
	\end{enumerate}
\end{proof}

\begin{defn}
	Let $(H_\al, \l\cdot, \cdot \r_{\al})_{\al \in A} \subset \Obj(\Hilb)$. We define $\bigoplus\limits_{\al \in A} H_{\al} \subset \prod\limits_{\al \in A} H_{\al}$ and $\l\cdot, \cdot \r: \bigg[ \bigoplus\limits_{\al \in A} H_{\al} \bigg]^2 \rightarrow \C$ by  
	$$\bigoplus\limits_{\al \in A} H_{\al} = s_A^{-1}(l^2(A))$$
	and 
	$$\l x, y \r = \sum\limits_{\al \in A} \l x_{\al}, y_{\al} \r_{\al}$$
	We define the \tbf{direct sum of $(H_\al, \l\cdot, \cdot \r_{\al})_{\al \in A}$} to be $(\bigoplus\limits_{\al \in A} H_{\al}, \l \cdot, \cdot\r)$.
\end{defn}
	
\begin{ex}
	Let $(H_\al, \l\cdot, \cdot \r_{\al})_{\al \in A} \subset \Obj(\Hilb)$. Then 
	\begin{enumerate}
		\item for each $x \in \prod\limits_{\al \in A} H_{\al}$, $x \in \bigoplus\limits_{\al \in A} H_{\al}$ iff $s_A(x) \in l^2(A)$
		\item $s_A|_{\bigoplus\limits_{\al \in A} H_{\al}}: \bigoplus\limits_{\al \in A} H_{\al} \rightarrow l^2(A)$
	\end{enumerate}
\end{ex}

\begin{proof}\
	Immediate by definition
\end{proof}

\begin{ex}
	Let $(H_\al, \l \cdot, \cdot \r_{\al})_{\al \in A} \subset \Obj(\Hilb)$. Set $H \defeq \bigoplus\limits_{\al \in A} H_{\al}$. Then
	\begin{enumerate}
		\item $H$ is a vector space
		\item $\l \cdot, \cdot\r: H \times H \rightarrow \C$ is an inner product
		\item $(H, \l \cdot, \cdot \r)$ is a Hilbert space
	\end{enumerate}
\end{ex}

\begin{proof}\
	\begin{enumerate}
		\item Clearly $\prod\limits_{\al \in A} H_{\al}$ is a vector space. Let $x,y \in H$ and $\lam \in \C$. \tcb{The previous exercise} implies that $s_A(x), s_A(y) \in l^2(A)$. Therefore $s_A(x) s_A(y) \in L^1(A)$ and
		\begin{align*}
			\sum\limits_{\al \in A}\|(x + \lam y)_{\al}\|_{\al}^2
			& = \sum\limits_{\al \in A}\|x_{\al} + \lam y_{\al}\|_{\al}^2 \\
			& = \sum\limits_{\al \in A} \bigg[ \|x_{\al}\|_{\al}^2 + |\lam|^2 \|y_{\al}\|_{\al}^2 + 2 \Rl(\l x_{\al}, y_{\al} \r_{\al}) \bigg] \\
			& \leq \sum\limits_{\al \in A} \|x_{\al}\|_{\al}^2 + |\lam|^2 \sum\limits_{\al \in A} \|y_{\al}\|_{\al}^2 + 2\sum\limits_{\al \in A} \| x_{\al}\|_{\al} \|y \|_{\al}  \\
			& = \|s_A(x)\|_2^2 + |\lam|^2 \|s_A(y)\|_2^2 + 2 \|s_A(x) s_A(y)\|_1 \\
			& < \infty 
		\end{align*}
		Thus $H$ is a vector space.
		\item Let $x,y,z \in H$ and $\lam \in \C$. Then 
		\begin{enumerate}
			\item 
			\begin{align*}
				\l x, y + \lam z \r
				& = \sum\limits_{\al \in A} \l x_{\al}, (y + \lam z)_{\al} \r_{\al} \\
				& = \sum\limits_{\al \in A} \l x_{\al}, y_{\al} + \lam z_{\al} \r_{\al} \\
				& = \sum\limits_{\al \in A} \bigg[ \l x_{\al}, y_{\al} \r_{\al} + \lam \l x_{\al}, z_{\al} \r_{\al} \bigg] \\
				& =  \sum\limits_{\al \in A} \l x_{\al}, y_{\al} \r_{\al} + \lam \sum\limits_{\al \in A} \l x_{\al}, z_{\al} \r_{\al}  \\
				& = \l x, y \r + \lam \l x, z \r 
			\end{align*}
			\item 
			\begin{align*}
				\l x,y \r
				& = \sum\limits_{\al \in A} \l x_{\al}, y_{\al} \r_{\al} \\
				& = \sum\limits_{\al \in A} \l y_{\al}, x_{\al} \r_{\al}^* \\
				& = \bigg( \sum\limits_{\al \in A} \l y_{\al}, x_{\al} \r_{\al} \bigg)^* \\
				& = \l y, x \r 
			\end{align*}
			\item 
			\begin{align*}
				\l x, x \r 
				& = \sum\limits_{\al \in A} \l x_{\al}, x_{\al} \r_{\al} \\
				& \geq 0
			\end{align*}
			\item Suppose that $\l x, x \r = 0$. Then 
			\begin{align*}
				0
				& = \l x, x \r \\
				& = \sum\limits_{\al \in A} \l x_{\al}, x_{\al} \r_{\al} \\
				& = \sum\limits_{\al \in A} \|x_{\al}\|_{\al}^2  
			\end{align*}
			Thus for each $\al \in A$, $\|x_{\al} \|_{\al} = 0$. Therefore for each $\al \in A$, $x_{\al} = 0$. Hence $x = 0$.
		\end{enumerate}
		So $\l \cdot, \cdot \r: H \times H \rightarrow \C$ is an inner product on $H$.
		\item Let $(x_j)_{j \in \N} \subset H$. Suppose that $(x_j)_{j \in \N}$ Cauchy. Let $\al \in A$ and $\ep > 0$. Since $(x_j)_{j \in \N}$ Cauchy, there exists $N \in \N$ such that for each $m,n \in \N$, $m,n \geq N$ implies that $\|x_m - x_n\| < \ep$. Let $m,n \in \N$. Suppose that $m,n \geq N$. Then 
		\begin{align*}
			\|x_{m,\al} - x_{n, \al}\|_{\al}^2 
			& \leq \sum_{\be \in A} \|x_{m,\be} - x_{n,\be}\|_{\be}^2 \\
			& = \|x_{m} - x_{n}\|^2 \\
			& < \ep^2
		\end{align*}
		Thus $\|x_{m,\al} - x_{n, \al}\|_{\al} \leq \ep$. Since $\ep > 0$ is arbitrary, we have that for each $\ep >0$, there exists $N \in \N$ such that for each $m,n \in \N$, $m,n \geq N$ implies that $\|x_{m,\al} - x_{n, \al}\|_{\al}$. Hence $(x_{j,\al})_{j \in \N}$ is Cauchy. Since $H_{\al}$ is complete, there exists $x_{\al} \in H_{\al}$ such that $x_{j,\al} \rightarrow x_{\al}$. Since $\al \in A$ is arbitrary, we have that for each $\al \in A$, there exists $x_{\al} \in H_{\al}$ such that $x_{j, \al} \rightarrow x_{\al}$. Define $x \in \prod\limits_{\al \in A} H_{\al}$ by $x = (x_{\al})_{\al \in A}$. 

		Let $\ep > 0$. Since $(x_n)_{n \in \N}$ is Cauchy, there exists $N \in \N$ such that for each $m,n \in \N$, $m,n \geq N$ implies that $\|x_m - x_n\| \leq \ep$. Let $n \in \N$. Suppose that $n \geq N$. Fatou's lemma imply that
		\begin{align*}
			\|s_A(x - x_n)\|_2^2 
			& = \sum_{\al \in A} \|x_{\al} - x_{n,\al}\|_{\al}^2 \\
			& = \sum_{\al \in A} \lim\limits_{m \rightarrow \infty} \|x_{m, \al} - x_{n,\al}\|_{\al}^2 \\
			& \leq \liminf\limits_{m \rightarrow \infty} \sum_{\al \in A} \|x_{m, \al} - x_{n,\al}\|_{\al}^2 \\
			& = \liminf\limits_{m \rightarrow \infty} \|x_m - x_n\|^2 \\
			& \leq \ep^2
		\end{align*}
		Thus $\|s_A(x - x_n)\|_2 \leq \ep$. Since $\ep > 0$ is arbitrary, we have that for each $\ep > 0$, there exists $N_{\ep} \in \N$ such that for each $n \in \N$, $n \geq N_{\ep}$ implies that $\|s_A(x - x_n)\|_2 \leq \ep$.
		
		In particular, setting $\ep = 1$ and $n = N_1$, \tcb{A previous exercise} implies that
		\begin{align*}
			\|s_A(x)\|_2
			& \leq \|s_A(x) - s_A(x_{N_1})\|_2 + \|s_A(x_{N_1})\|_2 \\
			& = \|s_A(x) - s_A(x_{N_1})\|_2 + \|x_{N_1}\| \\
			& \leq \|s_A(x - x_{_{\ep}})\|_2 + \|x_{N_1}\| \\
			& \leq 1 + \|x_{N_1}\| \\
			& < \infty 
		\end{align*}
		so that $s_A(x) \in l^2(A)$ and therefore $x \in \bigoplus\limits_{\al \in A} H_{\al}$. 
		
		Since $x \in H$, we have that for each $n \in \N$, $\|x - x_n\| = \|s_A(x - x_n)\|_2$. Thus from before, for each $\ep > 0$, there exists $N_{\ep} \in \N$ such that for each $n \in \N$, $n \geq N_{\ep}$ implies that $\|x - x_n\| \leq \ep$. Hence $x_n \rightarrow x$. Since $(x_n)_{n \in \N} \subset H$ with $(x_n)_{n \in \N}$ Cauchy is arbitrary, we have that for each $(x_n)_{n \in \N} \subset H$, $(x_n)_{n \in \N}$ is Cauchy implies that there exists $x \in H$ such that $x_n \rightarrow x$. Hence $H$ is complete. Thus $(H, \l \cdot, \cdot \r)$ is a Hilbert space.
	\end{enumerate}
\end{proof}

\begin{note}
	This construction might work for Banach spaces with norms satisfying $\|x_{\al}+y_{\al}\|_{\al}^2 \leq \|x_{\al}\|_{\al}^2 + \|y_{\al}\|_{\al}^2 + 2\|x_{\al}\|_{\al}\|y_{\al}\|_{\al}$.
\end{note}

\begin{ex}
	Let $H \in \Obj(\Hilb)$, $A$ an index set and for each $\al \in A$, $E_{\al}$ a closed subspace of $H$. Then $H \cong \bigoplus_{\al \in A} E_{\al}$ iff 
	\begin{enumerate}
		\item for each $\al, \be \in A$, $\al \neq \be$ implies that $E_{\al} \cap E_{\be} = \{0\}$
		\item for each $x \in H$, there exist $(x_{\al})_{\al \in A} \in \prod\limits_{\al \in A} E_{\al}$ such that $x = \sum\limits_{\al \in A}$
	\end{enumerate}
\end{ex}

\begin{proof}\
	\begin{itemize}
		\item ($\implies$): \\
		Suppose that $H \cong \bigoplus_{\al \in A} E_{\al}$. Let $x \in H$. 
		\item ($\impliedby$): \\
	\end{itemize}
\end{proof}

\begin{ex}
	Let $H \in \Obj(\Hilb)$ and $E \subset H$ a closed subspace of $H$. Then $E^{\perp} \oplus E$ something about $E^{\perp} \cong H/E$ and adding up to $H$. \tcr{FINISH!!!}
\end{ex}

\begin{proof}
	
\end{proof}


















































%\begin{defn} \ld{}\tbf{(Tensor Product):} \\ 
%	Let $H_1, H_2$ be Hilbert spaces. Define $$\otimes: H_1 \times H_2 \rightarrow \C^{H_1 \times H_2} ,\hspace{.5cm} (x, \phi) \mapsto x \otimes \phi$$ by $$x \otimes \phi(x,y) = \l x, x  \r \l y, \phi\r$$ 
%\end{defn}
%
%\begin{note}
%	For the remainder of this section, we assume that $H_1$ and $H_2$ are Hilbert spaces.
%\end{note}
%
%\begin{ex} \lex{}
%	We have that $\otimes: H_1 \times H_2 \rightarrow \C^{H_1 \times H_2}$ is bilinear.
%\end{ex}
%
%\begin{proof}
%	Clear.
%\end{proof}
%
%\begin{defn} \ld{}
%	Define $T(H_1, H_2) = \spn \{x \otimes \phi: x \in H_1 \text{ and } \phi \in H_2\}$ and define $\l \cdot, \cdot \r : T(H_1, H_2) \rightarrow \C $ by $$\l x_1 \otimes \phi_1 , x_2 \otimes \phi_2 \r = \l x_1, x_2 \r \l \phi_1 ,  \phi_1\r$$ and extending sesquilinearly so that $$ \l \sum_{i=1}^m \al_i x_i \otimes \phi_i , \sum_{j=1}^n \beta_j \Xi_j \otimes \Gam_j\r = \sum_{i=1}^m \sum_{j=1}^n \al_i^* \beta_j \l x_i \otimes \phi_i,  \Xi_j \otimes \Gam_j \r$$
%\end{defn}
%
%\begin{ex} \lex{}
%	We have that  $\l \cdot, \cdot \r : T(H_1, H_2) \rightarrow \C $ is an inner product on $T(H_1, H_2)$.
%\end{ex}
%
%\begin{proof} 
%	Clear.
%\end{proof}
%
%\begin{defn} \ld{}
%	Define $H_1 \otimes H_2$ to be the completion of $T(H_1, H_2)$.
%\end{defn}
%	
%
%	\begin{ex} \lex{}
%		Let $H_1, H_2$ be Hilbert spaces. If $(x_j)_{j\in \N}$ is an orthonormal basis for $H_1$ and $(\phi_j)_{j \in \N}$ is an orthonormal basis for $H_2$, then $(x_i \otimes \phi_j)_{i,j \in \N}$ is an orthonormal basis for $H_1 \otimes H_2$. 
%	\end{ex}
%
%	\begin{proof}
%		Since 
%		\begin{align*}
%			\l x_{i_1} \otimes \phi_{j_1} , x_{i_2} \otimes \phi_{j_2} \r 
%			&= \l x_{i_1} , x_{i_2} \r \l \phi_{j_1} , \phi_{j_2} \r \\
%			&= \del_{i_1, i_2} \del_{j_1, j_2}
%		\end{align*}
%		we have that $(x_i \otimes \phi_j)_{i,j \in \N}$ is orthonormal. Let $x = \sum\limits_{i \in \N} a_i x_i \in H_1$ and $\phi = \sum\limits_{j \in \N} b_j \phi_j \in H_2$. Then $\phi \otimes x = \sum\limits_{i \in \N} \sum\limits_{j \in \N} a_ib_j x_i \otimes \phi_j$, so $(x_i \otimes \phi_j)_{i,j \in \N}$ is a dense subset of $T(H_1, H_2)$, which is dense in $H_1 \otimes H_2$. Hence $(x_i \otimes \phi_j)_{i,j \in \N}$ is dense in $H_1 \otimes H_2$ and is a basis.
%	\end{proof}
%
%	\begin{note}
%		If $H_1$ and $H_2$ are function spaces over sets $S_1$ and $S_2$ respectively, then $H_1 \otimes H_2$ can be identified with the function space over $S_1 \times S_2$ given by  $f_1 \otimes f_2(s_1, s_2) = f_1(s_1)f_2(s_2)$.
%	\end{note}
%
%	\begin{defn} \ld{}
%		Let $A$ and $B$ be operators on $H_1$ and $H_2$ respectively. We define the operator $A \otimes B$ on $H_1 \otimes H_2$ by setting $$A \otimes B (x \otimes \phi) = A x \otimes B \phi$$ and extending linearly to $T(H_1, H_2)$ and then extending continuously to $H_1 \otimes H_2$
%	\end{defn}



















\newpage
	\section{Tensor Products}

	\begin{ex}
		Let $H_1, \ldots, H_n \in \Obj(\Hilb)$, $\phi \in L^n(H_1, \ldots, H_n; \C)$. Suppose that for each $j \in [n]$, $H_j$ is separable. For each $j \in [n]$, let $B_j \subset H_j$ be an orthonormal basis for $H_j$. Then
		$$\sum_{e_1 \in B_1} \cdots \sum_{e_n \in B_n} |\phi(e_1, \ldots, e_n)|^2 = \sum_{e_1' \in B_1'} \cdots \sum_{e_n' \in B_n'} |\phi(e_1', \ldots, e_n')|^2.$$
	\end{ex}

	\begin{proof}
		\tcr{Induction on $n$}
	\end{proof}

	\begin{defn}
		Let $H_1, \ldots, H_n \in \Obj(\Hilb)$, $\phi \in L^n(H_1, \ldots, H_n; \C)$. Suppose that for each $j \in [n]$, $H_j$ is separable. For each $j \in [n]$, let $B_j \subset H_j$ be an orthonormal basis for $H_j$. We define $\|\cdot\|_{HS}: L^n(H_1, \ldots, H_n; \C) \rightarrow \RG$ by 
		$$\| \phi \|_{HS} \defeq \bigg[ \sum_{e_1 \in B_1} \sum_{e_1 \in B_1} \cdots \sum_{e_n \in B_n} |\phi(e_1, \ldots, e_n)|^2 \bigg]^{1/2}.$$ 
		We define the \tbf{Hilbert-Schmidt functionals on $H_1, \ldots, H_n$}, denoted $HS^n(H_1, \ldots, H_n; \C)$, by 
		$$HS^n(H_1, \ldots, H_n; \C) \defeq \{\phi \in L^n(H_1, \ldots, H_n; \C): \|\phi\|_{HS} < \infty\}.$$
		We define $\l \cdot, \cdot \r_{HS} : HS^n(H_1, \ldots, H_n; \C) \rightarrow \C$ by 
		$$\l \phi, \psi \r_{HS} \defeq \sum_{e_1 \in B_1} \cdots \sum_{e_n \in B_n} \phi(e_1, \ldots, e_n)^* \psi(e_1, \ldots, e_n).$$
		When $H_1 = \cdots = H_n = H$, we write $HS^n(H; \C)$ in place of $HS^n(H_1, \ldots, H_n; \C)$. 
	\end{defn}

	\begin{ex}
		Let $H_1, \ldots, H_n \in \Obj(\Hilb)$, $\phi \in L^n(H_1, \ldots, H_n; \C)$. Suppose that for each $j \in [n]$, $H_j$ is separable. For each $j \in [n]$, let $B_j \subset H_j$ be an orthonormal basis for $H_j$. 
		\begin{enumerate}
			\item $\l \cdot, \cdot \r_{HS}$ is an inner product on $HS^n(H_1, \ldots, H_n; \C)$ 
			\item for each $\phi \in HS^n(H_1, \ldots, H_n; \C)$, $\l \phi, \phi \r_{HS} = \|\phi\|_{HS}$.
		\end{enumerate}
	\end{ex}

	\begin{proof}\
		\begin{enumerate}
			\item Let $\phi, \psi, \chi \in HS^n(H_1, \ldots, H_n; \C)$ and $\lam \in \C$. Then 
			\begin{enumerate}
				\item 
				\begin{align*}
					\l \phi , \psi + \lam \chi \r_{HS}
					& = \sum_{e_1 \in B_1} \cdots \sum_{e_n \in B_n} \phi(e_1, \ldots, e_n)^* \bigg[ \psi(e_1, \ldots, e_n) + \lam \chi(e_1, \ldots, e_n) \bigg] \\
					& = \sum_{e_1 \in B_1} \cdots \sum_{e_n \in B_n} \phi(e_1, \ldots, e_n)^*\psi(e_1, \ldots, e_n) + \lam \sum_{e_1 \in B_1} \cdots \sum_{e_n \in B_n} \phi(e_1, \ldots, e_n)^* \chi(e_1, \ldots, e_n) \bigg] \\
					& = \l \phi , \psi  \r_{HS} + \lam \l \phi , \chi \r_{HS}.
				\end{align*}
				\item 
				\begin{align*}
					\l \phi , \psi \r _{HS} 
					& = \sum_{e_1 \in B_1} \cdots \sum_{e_n \in B_n} \phi(e_1, \ldots, e_n)^* \psi(e_1, \ldots, e_n) \\
					& = \bigg[ \sum_{e_1 \in B_1} \cdots \sum_{e_n \in B_n} \phi(e_1, \ldots, e_n) \psi(e_1, \ldots, e_n)^* \bigg]^* \\
					& = \bigg[ \sum_{e_1 \in B_1} \cdots \sum_{e_n \in B_n} \psi(e_1, \ldots, e_n)^* \phi(e_1, \ldots, e_n) \bigg]^* \\
					& = \l \psi , \phi \r _{HS}^*. 
				\end{align*}
				\item 
				\begin{align*}
					\l \phi , \psi \r _{HS} 
					& = \sum_{e_1 \in B_1} \cdots \sum_{e_n \in B_n} \phi(e_1, \ldots, e_n)^* \phi(e_1, \ldots, e_n) \\
					& = \sum_{e_1 \in B_1} \cdots \sum_{e_n \in B_n} |\phi(e_1, \ldots, e_n)|^2 \\
					& = \|\phi\|_{HS}^2 \\
					& \geq 0
				\end{align*}
				\item Suppose that $\l \phi, \phi \r_{HS} = 0$. Then 
				\begin{align*}
					\sum_{e_1 \in B_1} \cdots \sum_{e_n \in B_n} |\phi(e_1, \ldots, e_n)|^2 
					& = \l \phi, \phi \r_{HS} \\
					& = 0.
				\end{align*}
				Thus for each $e_{1, j_1} \in B_1, \ldots, e_{n, j_n} \in B_n$, $\phi(e_1, \ldots, e_n) = 0$. \tcr{make exercise about how multilinear maps are determined by value on tuples of orthonormal basis elements} \rex{} implies that $\phi = 0$.
			\end{enumerate}
			\item Clear.
		\end{enumerate}
		\tcr{FINISH!!!}
	\end{proof}

	\begin{ex}
		Let $H \in \Obj(\Hilb)$. Suppose that $H$ is separable. Then 
		\begin{enumerate}
			\item for each $\phi \in H^*$, $\|\phi\|_{HS} = \|\phi\|$ 
			\item $HS(H;\C) = \H^*$. 
		\end{enumerate}
	\end{ex}

	\begin{proof}
		Since $H$ is separable, there exists $(e_j)_{j \in \N} \subset H$ such that $(e_j)_{j \in \N}$ is an orthonormal basis for $H$. 
		\begin{enumerate}
			\item Let $\phi \in H^*$. Then 
			\begin{align*}
				\|\phi\|
				& = \sup_{\|x\|=1} |\phi(x)| \\
				& \leq 
			\end{align*}
			\item 
		\end{enumerate}
	\end{proof}

	\begin{ex}
		content...
	\end{ex}
	
	\begin{ex}
		\tcr{associativity of tensor product}
	\end{ex}
	
	\begin{defn}
		Let $H_1, H_2, H \in \Obj(\Hilb)$, $\al \in L^2(H_1, H_2; H)$. Suppose that for each $j \in [n]$, $H_j$ is separable. For $j \in [n]$, let $B_j \subset H_j$ be an orthonormal basis for $H_j$. Then $\al$ is said to be \tbf{Hilbert-Schmidt} if 
		$$\sum_{e^1 \in B_1} \cdots \sum_{e^n \in B_n} |\phi(e^1, \cdots, e^n)|^2 < \infty $$
	\end{defn}
	
	\begin{defn}
		Let $H_1, H_2, H \in \Obj(\Hilb)$, $\al \in L^2(H_1, H_2; H)$, $(e_j)_{j \in \N}$ and $(f_k)_{k \in \N}$ orthonormal bases for $H_1$ and $H_2$ respectively. Then $\al$ is said to be \tbf{weakly Hilbert-Schmidt} if there exists $M \geq 0$ such that for each $x \in H$,   
	\end{defn}
	
	\begin{defn}
		Let $H_1, H_2, K \in \Obj(\Hilb)$ and $\al \in L^2(H_1, H_2; K)$. Then $(K, \al)$ is said to be a \tbf{tensor product of $H_1$ and $H_2$} if for each vector space $Z$ and $\be \in L^2(X, Y; Z)$, there exists a unique $\phi \in L^1(T;Z)$ such that $\phi \circ \al = \be$, i.e. the following diagram commutes:
		\[ 
		\begin{tikzcd}
			X \times Y \arrow[r, "\al"] \arrow[dr, "\be"'] 	
			& T  \arrow[d, dashed, "\phi"] \\
			& Z 
		\end{tikzcd}
		\] 
	\end{defn}
	
	
	
	
	
	
	
	
	
	
	
	
	
	
	
	
	
	
	
	
	
	
	
	
	
	
	\newpage
	\tcr{OLD STUFF}
	\begin{note}
	This section assumes familiarity with the algebraic tensor product of two vector spaces. See section ??? of \cite{algebra} for details. 
	\end{note}	
	
	\begin{defn}
	Let $X, Y$ and $Z$ be Banach spaces and $\phi \in L^2(X,Y ; Z) $. Then $(Z, \phi)$ is said to be a \tbf{tensor product} of $X$ with $Y$ if 
	\begin{enumerate}
	\item $\spn \phi(X \times Y)$ is dense in $Z$
	\item for each Banach space $W$ and $\psi \in L^2(X,Y;W)$, there exists a unique $\psi' \in L(Z, W)$ such that $\psi' \circ \phi = \psi$, i.e. such that the following diagram commutes: 
	\[ \begin{tikzcd}
	X \times Y \arrow[r, "\phi"] \arrow[dr, "\psi"'] 	
	& Z  \arrow[d, "\psi'"] \\
	& W 
\end{tikzcd}
	\]
	\end{enumerate}
	If $(Z, \phi)$ is a tensor product of $X$ with $Y$. We often write $Z = X \otimes Y$ and for each $x\in X$, $y \in Y$, we often write $\phi(x,y) = x \otimes y$.
	\end{defn}	
	
	\begin{ex}
	Let $X$ and $Y$ be Banach spaces, $U \subset X$ and $V \subset Y$. Set $W = \{u \otimes v: u \in U \text{ and } v \in V\} \subset X \otimes Y$. If $U$ and $V$ are linearly independent, then $W$ is linearly independent.\\
	\tbf{Hint:} For $\phi \in X^*$, $\psi \in Y^*$, define $T \in L^2(X, Y; \C)$ by $T(x,y) = \phi(x)\psi(y)$.
	\end{ex}	
	
	\begin{proof}
	Let $w = \sum\limits_{u \in U} \sum\limits_{v \in V}\lam_{u,v} u \otimes v$. Suppose that $w = 0$. Let $\phi \in X^*$ and $\psi \in Y^*$. Define $T \in L^2(X, Y; \C)$ by $T(x,y) = \phi(x)\psi(y)$. By definition of the tensor product, there exists a unique $T' \in L(X \otimes Y, \C)$ such that for each $x \in X$ and $y \in Y$, $T'(x \otimes y) = T(x,y)$. Then 
	\begin{align*}
	0 
	&= T'(w) \\
	&= T'(\sum\limits_{u \in U} \sum\limits_{v \in V}\lam_{u,v} u \otimes v) \\
	&= \sum\limits_{u \in U} \sum\limits_{v \in V}\lam_{u,v} T'(u \otimes v) \\
	&= \sum\limits_{u \in U} \sum\limits_{v \in V}\lam_{u,v} T(u, v) \\
	&= \sum\limits_{u \in U} \sum\limits_{v \in V}\lam_{u,v} \phi(u)\psi(v) \\
	&= \phi\bigg( \sum\limits_{u \in U} \sum\limits_{v \in V}\lam_{u,v}  \psi(v) u \bigg)
	\end{align*}
	Since $\phi \in X^*$ is arbitary, a previous exercise in the section on linear functionals implies that 
	\begin{align*}
	0 
	&= \sum\limits_{u \in U} \sum\limits_{v \in V}\lam_{u,v}  \psi(v) u \\
	&= \sum\limits_{u \in U} \bigg (\sum\limits_{v \in V}\lam_{u,v}  \psi(v)  \bigg) u
	\end{align*}
	Linear independence of $U$ implies that for each $u \in U$, 
	\begin{align*}
	0 
	&= \sum\limits_{v \in V}\lam_{u,v}  \psi(v) \\
	&= \psi \bigg( \sum\limits_{v \in V}\lam_{u,v}  v \bigg )
	\end{align*}
	Since $\psi \in Y^*$ is arbitary, for each $u \in U$, $$\sum\limits_{v \in V}\lam_{u,v}  v = 0$$
	Linear independence of $V$ implies that for each $u \in U, v \in V$, $\lam_{u,v} = 0$. Hence $W$ is linearly independent. 
	\end{proof}
	
	\begin{ex} \tbf{Uniqueness:}\\
	Let $X, Y$ and $Z$ be Banach spaces and $\phi \in L^2(X,Y ; Z)$. Suppose that $(Z, \phi)$ is a tensor product of $X$ with $Y$. Then $(Z, \phi)$ is unique up to isomorphism. 
	\end{ex}
	
	\begin{proof}
	Let $W$ be a Banach space and $\psi \in  L^2(X,Y ; W)$. Suppose that $(W, \psi)$ is a tensor product of $X$ with $Y$. Since $(Z, \phi)$ is a tensor product of $X$ with $Y$, there exists a unique $\psi' \in L(Z, W)$ such that $\psi' \circ \phi = \psi$. Since $(W, \psi)$ is a tensor product of $X$ with $Y$, there exists a unique $\phi' \in L(W,Z)$ such that $\phi' \circ \psi = \phi$. Thus the following diagram commutes: 
	\[ \begin{tikzcd}
	& W  \arrow[d, "\phi'"] \\
	X \times Y \arrow[r, "\phi"] \arrow[dr, "\psi"'] \arrow[ur, "\psi"]	
	& Z  \arrow[d, "\psi'"] \\
	& W 
\end{tikzcd}
	\]
	On the other hand, since $(W, \psi)$ is a tensor product of $X$ with $Y$, there exists a unique $\Psi \in L(W)$ such that $\Psi \circ \psi = \psi$. Thus the following diagram commutes:  
	\[ \begin{tikzcd}
	X \times Y \arrow[r, "\psi"] \arrow[dr, "\psi"'] 	
	& W  \arrow[d, "\Psi"] \\
	& W 
\end{tikzcd}
	\]
	Since $I_W \in L(W)$ and $I_W \circ \psi = \psi$, uniqueness of $\Psi$ implies that $\Psi = I_W$. 
	From the first diagram, we see that $\psi' \circ \phi'$ satisfies $(\psi' \circ \phi') \circ \psi  = \psi$. Since $\psi' \circ \phi' \in L(W)$, uniqueness of $\Psi$ implies that $\Psi = \psi' \circ \phi'$. Thus $\psi' \circ \phi'  = I_W$. \\
	Similarly, we could have initially considered the following diagram: 
	\[ \begin{tikzcd}
	& Z  \arrow[d, "\psi'"] \\
	X \times Y \arrow[r, "\psi"] \arrow[dr, "\phi"'] \arrow[ur, "\phi"]	
	& W  \arrow[d, "\phi'"] \\
	& Z
\end{tikzcd}
	\]
	Playing a similar game, we could use the fact that there exists a unique $\Phi \in L(Z)$ such that $\Phi \circ \phi = \phi$ to obtain the following diagram:
	\[ \begin{tikzcd}
	X \times Y \arrow[r, "\phi"] \arrow[dr, "\phi"'] 	
	& Z  \arrow[d, "\Phi"] \\
	& Z 
\end{tikzcd}
	\]
	As before, uniqueness enables us to conclude that $\phi' \circ \psi' = I_Z$. Thus $\psi'$ and $\phi'$ are isomorphisms and $Z \cong W$.
	\end{proof}
	
	
	\begin{note}
	The following definitions and exercises will cover the explicit construction of a tensor product of Banach spaces.
	\end{note}	
	
	\begin{defn}
	Let $X$ and $Y$ be Banach spaces. Define $X \otimes^{\text{alg}} Y = \spn \{ x \otimes y: x \in X \text{ and } y \in Y \}$ to be the algebraic tensor product of $X$ with $Y$ (see section ??? of \cite{algebra} for details). 
	\end{defn}
	
	\begin{ex}
	Let $X$ and $Y$ be Banach spaces and $x \otimes y \in X \otimes^{\text{alg}} Y$. If for each $\phi \in X^*$ and $\psi \in Y^*$, $\phi \otimes \psi(x,y) = 0$, then $x \otimes y = 0$.
	\end{ex}
	
	\begin{proof}
	The previous section tells us that for each $\phi \in X^*$ and $\psi \in Y^*$, $\phi \otimes psi(x,y) = 0$, then $x = 0$ or $y = 0$. This implies that $x \otimes y = 0$.
	\end{proof}
	
	\begin{defn}\tbf{The Projective Norm:} \\
	Define $\|\cdot \|_{\pi}:X \otimes^{\text{alg}} Y \rightarrow \Rg$ by $$\|u\|_{\pi} = \inf \bigg \{ \sum_{j=1}^n \|x_j\| \|y_j\|: (x_j)_{j=1}^n \subset X \text{, }  (y_j)_{j=1}^n \subset Y \text{ and } u = \sum_{j=1}^n x_j \otimes y_j  \bigg \}$$
	\end{defn}
	
	\begin{ex}
	Let $X$ and $Y$ be Banach spaces. Then $\| \cdot \|_{\pi}: X \otimes^{\text{alg}} Y \rightarrow \Rg$ is a norm on $X \otimes^{\text{alg}} Y$.
	\end{ex}
	
	\begin{proof}\
	\begin{itemize}
	\item Let $\lam \in \C$, $u \in X \otimes^{\text{alg}} Y$. If $\lam = 0$, then $\lam u = 0  u = 0 \otimes 0$ and clearly $\|\lam u\|_{\pi} = 0 = |\lam|\|u\|_{\pi}$. Suppose that $\lam \neq 0$. Let $\ep >0$. Then there exist $(x_j)_{j=1}^n \subset X$ and $(y_j)_{j=1}^n \subset Y$ such that $u = \sum\limits_{j=1}^n x_j \otimes y_j $ and $\sum\limits_{j=1}^n \|x_j\| \|y_j\| < \|u\|_{\pi} + \ep/|\lam| $. Then $\lam u = \sum\limits_{j=1}^n (\lam x_j) \otimes y_j $.
	Therefore
	\begin{align*}
	\|\lam u\|_{\pi} 
	& \leq \sum\limits_{j=1}^n \|\lam x_j\| \|y_j\| \\
	& \leq |\lam| \sum\limits_{j=1}^n \| x_j\| \|y_j\| \\
	& < |\lam| \bigg( \|u\|_{\pi} + \frac{\ep}{|\lam|} \bigg) \\
	& = |\lam| \|u\|_{\pi} + \ep
	\end{align*}
	Since $\ep >0$ is arbitrary, $\|\lam u\|_{\pi} \leq |\lam| \|u\|_{\pi} $. For the sake of contradiction, suppose that $\|\lam u\|_{\pi} < |\lam| \|u\|_{\pi} $. Then there exists $(x_j)_{j=1}^n \subset X$ and $(y_j)_{j=1}^n \subset Y$ such that $\lam u = \sum\limits_{j=1}^n x_j \otimes y_j $ and $\sum\limits_{j=1}^n \|x_j\| \|y_j\| < |\lam| \|u\|_{\pi}$. Hence $u = \sum\limits_{j=1}^n (\lam^{-1}x_j) \otimes y_j$. This implies that 
	\begin{align*}
	\|u \|_{\pi} 
	& \leq \sum\limits_{j=1}^n \|\lam^{-1}x_j\| \|y_j\| \\
	&= |\lam |^{-1}\sum\limits_{j=1}^n \|x_j\| \|y_j\| \\
	&< |\lam |^{-1} |\lam| \|u\|_{\pi} \\
	&= \|u\|_{\pi}
	\end{align*}
	which is a contradiction. Therefore $\|\lam u\|_{\pi} \geq |\lam| \|u\|_{\pi}$ which implies that $\|\lam u\|_{\pi} = |\lam| \|u\|_{\pi}$
	\item Let $u, v \in X \otimes^{\text{alg}} Y$ and $\ep >0$. Then there exist $(x_j)_{j=1}^n$, $(a_k)_{k=1}^m \subset X$ and $(y_j)_{j=1}^n$, $(b_k)_{k=1}^m \subset Y$ such that $u = \sum\limits_{j=1}^n x_j \otimes y_j $, $v = \sum\limits_{k=1}^m a_k \otimes b_k $, $\sum\limits_{j=1}^n \|x_j\| \|y_j\| < \|u\|_{\pi} + \ep/2$ and $\sum\limits_{k=1}^m \|a_k\| \|b_k\| < \|u\|_{\pi} + \ep/2$. Then $u+v = \sum\limits_{j=1}^n x_j \otimes y_j + \sum\limits_{k=1}^m a_k \otimes b_k $ which implies that 
	\begin{align*}
	\|u + v\|_{\pi} 
	& \leq \sum\limits_{j=1}^n \|x_j\| \|y_j\| + \sum\limits_{k=1}^m \|a_k\| \|b_k\| \\
	&< \|u\|_{\pi} + \ep/2 + \|v\|_{\pi} + \ep/2 \\
	&= \|u\|_{\pi} + \|v\|_{\pi} + \ep
	\end{align*}
	Since $\ep >0$ is arbitrary, $\|u + v\|_{\pi} \leq \|u\|_{\pi} + \|v\|_{\pi}$.
	\item Let $u \in X \otimes^{\text{alg}} Y$. Suppose that $\|u\| = 0$. Let $\phi \in X^*$ and $\psi \in Y^*$ and $\ep >0$. Then there exist $(x_j)_{j=1}^n \subset X$ and $(y_j)_{j=1}^n \subset Y$ such that $u = \sum\limits_{j=1}^n x_j \otimes y_j $ and $$\sum\limits_{j=1}^n \|x_j\| \|y_j\| < \frac{\ep}{\|\phi\| \|\psi\| + 1}$$
	Then 
	\begin{align*}
	\sum_{j=1}^n |\phi \otimes \psi(x_j,y_j)|
	&=  \sum_{j=1}^n |\phi(x_j)\psi(y_j)| \\
	& \leq  \sum_{j=1}^n \|\phi\|\|x_j\| \|\psi\|\|y_j\| \\
	&= \|\phi\|\|\psi\|\sum_{j=1}^n \|x_j\| \|y_j\| \\
	& < \|\phi\|\|\psi\|\frac{\ep}{\|\phi\| \|\psi\| + 1} \\
	& < \ep
	\end{align*}
	Then for each $j \in \{1, \ldots, n\}$, $|\phi \otimes \psi(x_j,y_j)| < \ep$.
	\tbf{FINISH!!!} Try using sequences and continuity and a common refinement of representation and averaging
	\end{itemize}
	\end{proof}
	
	\begin{ex} \tbf{Existence:} \\
	
	\end{ex}
	
	\begin{proof}
	
	\end{proof}	
	
	
	
	
	
	
	\newpage
	\section{Rigged Hilbert Spaces}
	
	\tcr{Introduce rigged hilbert spaces, talk about how schwarz space is a rigged hilbert space, then introduce gateaux derivative $\frac{d L}{\p \phi(x)} \defeq dL(\phi)(\del(x))$ of $L$ at $\phi$ in the direction of $\del(x)$ (the dirac delta function), for functionals $L$ on schwarz space, then talk about equations of motion and qft, see "quantum theory for gifted amateur" book section on functionals for details, try to formalize.}
	
	
	
	\newpage
	
	
	\section{MISC, unitary transformations}
	
	\begin{defn}
		Let $H$ be a Hilbert space, $T \in GL(H)$ and $E \subset H$ a closed subspace. Then $E$ is said to be \tbf{invariant under $T$} if $T(E) = E$.  
	\end{defn}
	
	\begin{ex}
		Let $H$ be a Hilbert space, $U \in U(H)$ and $E \subset H$ a closed subspace. If $E$ is invariant under $U$, then $U(E^{\perp}) = U(E)^{\perp}$.
	\end{ex}

	\begin{proof}
		Suppose that $E$ is invariant under $U$. Let $y \in E$ and $x_0 \in E^{\perp}$. Since $E$ is invariant under $U$, $U(E) = E$. Hence there exists $x \in E$ such that $Ux = y$. Since $x_0 \in E^{\perp}$ and $x \in E$, $\l x_0, x \r = 0$. Since $U \in U(H)$, 
		\begin{align*}
			\l U(x_0), y\r
			& = \l U(x_0), y \r \\
			& = \l U(x_0), U(x) \r \\
			& =  \l x_0, x \r \\
			& = 0
		\end{align*}  
		Since $y \in E$ is arbitrary, we have that for each $y \in E$, $\l U(x_0), y\r = 0$. Therefore $U x_0 \in E^{\perp}$. Since $x_0 \in E^{\perp}$ is arbitrary, we have that for each $x_0 \in E^{\perp}$, $U x_0 \in E^{\perp}$. Thus $U(E^{\perp}) \subset E^{\perp}$. For the sake of contradiction, suppose that $E^{\perp} \not \subset U(E^{\perp})$. Then there exists $y \in E^{\perp}$ such that $y \not \in U(E^{\perp})$. Since Since $H = E \oplus E^{\perp}$ and \tcb{FINISH!!!} show $U(E \oplus E^{\perp}) = U(E) \oplus U(E^{\perp})$.
	\end{proof}
	
	
	
	
	
	
	
	
	
	
	
	
	
	
	
	
	
	
	
	\newpage
	\chapter{Differentiation}
	
	\newpage
	
	\section{TODO}
	\begin{itemize}
		\item Finish implicit and inverse function theorems
	\end{itemize}
	
	\newpage
	\begin{note}
		Much of the material in this chapter discusses maps $f: X \rightarrow Y$ where $X$ and $Y$ are Banach spaces. It is often the case that a discussion requires the base fields of $X$ and $Y$ to agree or to both be real. We note that in these cases, every complex vector space is also a real vector space. In particular, if $X$ is a finite dimensional complex vector space with dimension $n$, then $X$ is a finite dimensional real vector space of dimension $2n$.
	\end{note}
	
	\section{The Gateaux Derivative}
	
	\begin{defn} \ld{61001}
	Let $X,Y$ be a Banach spaces, $A \subset X$ open, $f:A \rightarrow Y$, $x_0 \in A$ and $x \in X$. Then $f$ is said to be 
	\begin{enumerate}
	\item \tbf{right-hand-differentiable at $x_0$ in the direction $x$} if the limit
	$$  \lim_{t \rightarrow 0^+} \frac{f(x_0 +tx) - f(x_0)}{t}$$
	exists. If $f$ is right-hand-differentiable at $x_0$ in the direction $x$, we define the \tbf{right-hand derivative} of $f$ at $x_0$ in the direction $x$, denoted by  $d^+ f(x_0; x)$, to be the above limit. 
	
	\item \tbf{left-hand-differentiable at $x_0$ in the direction $x$} if the limit
	$$  \lim_{t \rightarrow 0^-} \frac{f(x_0 +tx) - f(x_0)}{t}$$
	exists. If $f$ is right-hand-differentiable at $x_0$ in the direction $x$, we define the \tbf{left-hand derivative} of $f$ at $x_0$ in the direction $x$, denoted by  $d^- f(x_0; x)$, to be the above limit. 
	
	\item \tbf{differentiable at $x_0$ in the direction $x$} if the limit
	$$  \lim_{t \rightarrow 0} \frac{f(x_0 +tx) - f(x_0)}{t}$$
	exists. If $f$ is differentiable at $x_0$ in the direction $x$, we define the \tbf{derivative} of $f$ at $x_0$ in the direction $x$, denoted by $d f(x_0; x)$, to be the above limit. 
	\end{enumerate}
	\end{defn}	
	
	\begin{ex} \lex{61002}
	Let $X, Y$ be Banach spaces, $A \subset X$ open, $f:A \rightarrow \R$ and $x_0 \in A$. Then $df(x_0; 0) = 0$.
	\end{ex}
	
	\begin{proof}
	Clear.
	\end{proof}
	
	\begin{defn} \ld{}\tbf{The Gateaux Derivative:}\\
	Let $X,Y$ be Banach spaces, $A \subset X$ open, $f:A \rightarrow Y$ and $x_0 \in A$. Then $f$ is said to be 
	\begin{enumerate}
	\item \tbf{right-hand Gateaux differentiable} at $x_0$ if for each $x \in X$, $d^+ f(x_0; x)$ exits. We define the \tbf{right-hand Gateaux derivative} of $f$ at $x_0$, denoted $d^+ f(x_0) : X \rightarrow \R$, to be $$d^+ f(x_0)(x) = d^+ f(x_0;x)$$ 
	
	\item \tbf{left-hand Gateaux differentiable} at $x_0$ if for each $x \in X$, $d^- f(x_0; x)$ exits. We define the \tbf{left-hand Gateaux derivative} of $f$ at $x_0$, denoted $d^- f(x_0) : X \rightarrow \R$, to be $$d^- f(x_0)(x) = d^- f(x_0;x)$$
	
	\item \tbf{Gateaux differentiable} at $x_0$ if for each $x \in X$, $d f(x_0; x)$ exits. We define the \tbf{Gateaux derivative} of $f$ at $x_0$, denoted $d f(x_0) : X \rightarrow \R$, to be $$d f(x_0)(x) = d f(x_0;x)$$
	\end{enumerate}
	\end{defn}
	
	\begin{defn} \ld{61003}
Let $Y$ be a Banach space, $A \subset \R$ open and $f:A \rightarrow Y$. Then $f$ is said to be \tbf{Gateaux differentiable} if for each $x \in A$, $f$ is Gateaux differentiable at $x$. If $f$ is Gateaux differentiable, we define $df:A \rightarrow Y^X$ by $x_0 \mapsto df(x_0)$.
\end{defn}	
	
	\begin{ex} \lex{61004}
	Let $X, Y$ be Banach spaces, $A \subset X$ open, $f,g :A \rightarrow Y$, $\lam \in \R$ and $x_0 \in A$. If $f, g$ are Gateaux differentiable at $x_0$, then $f + \lam g$ is Gateaux differentiable at $x_0$ and $d[f+\lam g](x_0) = df(x_0) + \lam dg(x_0)$.
	\end{ex}
	
	\begin{proof}
	Similar to the case of the derivative from Calc I. 
	\end{proof}		
	
	\begin{ex} \lex{61005}
	Let $X, Y$ be Banach spaces, $A \subset X$ open, $f:A \rightarrow Y$ and $x_0 \in A$. Suppose that $f$ is Gateaux differentiable at $x_0$. Then for each $\lam \in \R$ and $x \in X$, $$df(x_0)(\lam x) = \lam df(x_0)(x)$$
	\end{ex}
	
	\begin{proof}
	Let $\lam \in \R$ and $x \in X$. Then 
	\begin{align*}
	df(x_0)(\lam x) 
	&= \lim_{t \rightarrow 0} \frac{f(x_0 + t \lam x) - f(x_0)}{t} \\
	&= \lim_{t \rightarrow 0} \lam \frac{f(x_0 + t \lam x) - f(x_0)}{\lam t} \\
	&= \lam \lim_{t \rightarrow 0}  \frac{f(x_0 + t \lam x) - f(x_0)}{\lam t} \\
	&= \lam \lim_{t \rightarrow 0}  \frac{f(x_0 + t x) - f(x_0)}{t} \\
	&= \lam df(x_0)(x) 
	\end{align*}
	\end{proof}
	
	\begin{ex} \lex{61006}
	Let $X, Y$ be Banach spaces, $A \subset X$ open, $f:A \rightarrow Y$. If $f$ is constant, then $f$ is Gateaux differentiable and for each $x_0\in A, x \in X$, $$df(x_0)(x) = 0$$
	\end{ex}
	
	\begin{proof}
	Suppose that $f$ is constant. Then there exists $c \in Y$ such that for each $x \in A$, $f(x) = c$. Let $x_0 \in A, x \in X$. Then 
	\begin{align*}
	df(x_0)(x) 
	&= \lim_{t \rightarrow 0} \frac{f(x_0 + t x) - f(x_0)}{t} \\
	&= \lim_{t \rightarrow 0} \frac{c - c}{t} \\
	&= 0
	\end{align*}
	\end{proof}
	
	\begin{ex} \lex{61007}
	Let $X, Y$ be Banach spaces, $A \subset X$ open, $f:A \rightarrow Y$. If $f$ is linear, then $f$ is Gateaux differentiable and for each $x_0\in A, x \in X$, $$df(x_0)(x) = f(x)$$
	\end{ex}
	
	\begin{proof}
	Suppose that $f$ is linear. Let $x_0\in A, x \in X$. Then 
	\begin{align*}
	df(x_0)(x) 
	&= \lim_{t \rightarrow 0} \frac{f(x_0 + t x) - f(x_0)}{t} \\
	&= \lim_{t \rightarrow 0} \frac{f(x_0) + t f(x) - f(x_0)}{t} \\
	&= f(x)
	\end{align*}
	\end{proof}		
	
	\begin{ex} \lex{61008}
	There exist Banach spaces $X,Y$, and $f:X \rightarrow Y$ such that $f$ is Gateaux differentiable and $f$ is nowhere continuous. \\
	\tbf{Hint:} use \rex{61007}
	\end{ex}
	
	\begin{proof}
	Set $X = C^1([0, 1])$ and $Y = C([0,1])$. Equip both $X$ and $Y$ with the sup norm. Define $T: X \rightarrow Y$ by $Tf = f'$. Then \rex{42002} implies that $T$ is not bounded. Since $T$ is linear, \rex{61007} implies that $T$ is Gateaux differentiable. Since $T$ is not bounded, \rex{42004} implies that $T$ is not continuous at $0$. Then \rex{42003.1} tells us that $T$ is nowhere continuous. 
	\end{proof}
	
	\begin{ex} \lex{61008.5}
	Set $A = \{(x,y) \in \R^2: y = -x^2 \text{ and } x \neq 0\}$. Define $f: \R^2 \setminus A \rightarrow \R$ by 
	\[f(x,y) = 
	\begin{cases}
	0 & (x,y) = (0,0) \\
	\frac{x^4y}{x^6 + y^3} & \text{otherwise}
	\end{cases}	
	\]
	Then $f$ is Gateaux differentiable at $(0,0)$ and $f$ is not continuous at $(0,0)$. \\
	\tbf{Hint:} Consider the set $B = \{(x, x^2:x \in \R)\} \subset \R^2 \setminus A$. 
	\end{ex}
	
	\begin{proof}
	
	\end{proof}
	
	
	
	\begin{ex} \lex{61009}
	Let $Y$ be a Banach space, $A \subset \R$ open, $f:A \rightarrow Y$ and $x_0 \in A$. Suppose that $f$ is Gateaux differentiable at $x_0$. Then $df(x_0) \in L(\R,Y)$.
	\end{ex}
	
	\begin{proof}\
	\begin{enumerate}
	\item \rex{61005} implies that for each $x,y ,\lam \in \R$,
	\begin{align*}
	df(x_0)(x + \lam y) 
	&= df(x_0)((x+\lam y)1)  \\
	&= (x+\lam y)df(x_0)(1) \\
	&= xdf(x_0)(1) + \lam y df(x_0)(1) \\
	&= df(x_0)(x) + \lam df(x_0)(y)
	\end{align*}
	So $df(x_0) \in \ML(\R, Y)$.
	\item Since for each $x \in \R$,
	\begin{align*}
	\|df(x_0)(x)\| 
	&= \|xdf(x_0)(1)\| \\
	&= |x| \|df(x_0)(1)\| \\
	\end{align*}	
	We have that $df(x_0):\R \rightarrow Y$ is bounded with $\|df(x_0)\| \leq \|df(x_0)(1)\|$. So $df(x_0)$ is continuous.
	\end{enumerate}
	Thus $df(x_0) \in L(\R, Y)$.
	\end{proof}
	
	\begin{ex} \lex{61010}
	Let $X$ be a Banach space, $A \subset X$ open, $f:A \rightarrow \R$ and $x_0 \in A$. If $f$ is Gateaux differentiable at $x_0$ and $f$ has a local extremum at $x_0$, then $df(x_0) = 0$.
	\end{ex}	
	
	\begin{proof}
	Suppose that $f$ is Gateaux differentiable at $x_0$ and $f$ has a local minimum point at $x_0$. Then there exists $\del >0 $ such that $B(x_0, \del) \subset A$ and for each $y \in B(x_0, \del)$, $f(x_0) \leq f(y)$. \\
	For the sake of contradiction, suppose that $df(x_0) \neq 0$. Then there exists $x \in X$ such that $x \neq 0$ and $df(x_0)(x) \neq 0$. \\
	First, suppose that $df(x_0)(x) < 0$. Choose $\ep = -df(x_0)(x) >0$. Then there exists $t_0 >0$ such that for each $t \in B^*(0, t_0)$, $x_0 + tx \in B(x_0, \del)$ and $$\bigg | \frac{f(x_0 + tx) - f(x_0)}{t} - df(x_0)(x) \bigg | < \ep$$ 
	This implies that for each $t \in B^*(0, t_0)$,
	\begin{align*}
	\frac{f(x_0 + tx) - f(x_0)}{t}  
	&< \ep + df(x_0)(x) \\
	&= 0
	\end{align*} 
	and hence $f(x_0 + tx) < f(x_0)$, which is a contradiction. \\
	Now, suppose that $df(x_0)(x) > 0$. Then 
	\begin{align*}
	df(x_0)(-x) 
	&= -df(x_0)(x) \\
	& < 0
	\end{align*}
	Similarly to above, this implies that there exists $t_0 >0$ such that for each $t \in B^*(0, t_0)$, $x_0 - tx \in B(x_0, \del)$ and $f(x_0 - tx) < f(x_0)$ which is a contradiction. So $df(x_0)(x) = 0$ and $df(x_0) = 0$. \\
	If $f$ has a local maximum at $x_0$, then $-f$ has a local minimum point at $x_0$. Then 
	\begin{align*}
	df(x_0)
	&= -d[-f](x_0) \\
	&= -0 \\
	&= 0
\end{align*}	 
	\end{proof}
	
	\begin{ex} \lex{61011}
	Let $X, Y, Z$ be a Banach spaces, $A \subset X$ open, $B \subset Y$ open, $f:A \rightarrow Y$, $g:B \rightarrow Z$ and $x_0 \in A$. Suppose that $f$ is affine. If $g$ is Gateaux differentiable at $f(x_0)$, then $g \circ f$ is Gateaux differentiable at $x_0$ and $$d(g \circ f)(x_0)(x) = dg(f(x_0))(df(x_0)(x))$$
	\end{ex}
	
	\begin{proof}
	Suppose that $g$ is Gateaux differentiable at $f(x_0)$. Since $f$ is affine, there exists $h:A \rightarrow Y$ and $c \in Y$ such that $h$ is linear and $f = h+c$. Then 
	\begin{align*}
	df(x_0) 
	&= dh(x_0) \\
	&= h
	\end{align*}
	Let $x \in X$. Choose $\del >0$ such that for each $t \in B(0, \del) \subset \R$, $f(x_0) + th(x) \in B$. Then for each $t \in B^*(0, \del)$,
	\begin{align*}
	g \circ f(x_0 + tx)
	&= g \bigg ( f(x_0) + t \frac{f(x_0 + tx) - f(x_0)}{t} \bigg) \\
	&=  g ( f(x_0) + th(x)) 
	\end{align*}
	This implies that
	\begin{align*}
	d(g \circ f)(x_0)
	&= \lim_{t \rightarrow 0 }\frac{g \circ f (x_0 + tx) - g(f(x_0))}{t} \\  
	&= \lim_{t \rightarrow 0} \frac{g ( f(x_0) + th(x)) - g(x_0)}{t} \\
	&= dg(f(x_0))(h(x)) \\
	&= dg(f(x_0))(df(x_0)(x)) \\
	\end{align*}
	\end{proof}
	
	
	
	
	
	
	
	
	
	
	
	
	
	
	
	\newpage
	\section{The Frechet Derivative}
	
	\begin{ex} \lex{62001}
	Let $X,Y$ be a normed vector spaces and $\phi: X \rightarrow Y$ linear. If $\phi(h) = o(\|h\|)$ as $h \rightarrow 0$, then $\phi = 0$. 
	\end{ex}
	
	\begin{proof}
	Let $h_0 \in X$. If $h_0 = 0$, then $\phi(h_0) = 0$. Suppose that $h_0 \neq 0$. Define $(h_n)_{n \in \N} \subset X$ by $$h_n = \frac{h_0}{n}$$ Then $h_n \rightarrow 0$. By continuity of $\phi$ and our initial assumption we have that 
	\begin{align*}
	\| h_0 \|^{-1} \phi ( h_0 ) 
	&= \phi \bigg( \frac{h_0}{\| h_0 \|} \bigg) \\
	&= \phi \bigg( \frac{h_n}{\| h_n \|} \bigg) \\
	&= \frac{\phi(h_n)}{\| h_n \|} \\
	& \rightarrow 0
	\end{align*}
	which implies that $\| h_0 \|^{-1}\phi ( h_0 ) = 0$. So $\phi(h_0) = 0$ and hence $\phi = 0$.
	\end{proof}
	
	\begin{ex} \lex{62002}
	Let $X, Y$ be a normed vector spaces, $A \subset X$ open, $f:A \rightarrow Y$ and $x_0 \in A$. Suppose that there exists $\phi: X \rightarrow Y$ such that $\phi$ is linear and $$f(x_0 + h) = f(x_0) + \phi(h) + o(\|h\| ) \hspace{.5cm} \text{ as } h \rightarrow 0$$ then $\phi$ is unique. 
	\end{ex}
	
	\begin{proof}
	Suppose that there exists $\psi : X \rightarrow Y$ such that $\psi$ is linear and such that
	$$f(x_0 + h) = f(x_0) + \psi(h) + o(\|h\| ) \hspace{.5cm} \text{ as } h \rightarrow 0$$ 
	Then $\phi(h) - \psi(h) = o(h)$. Since $\phi - \psi$ is linear, the previous exercise implies that $\phi = \psi$.
	\end{proof}
	
	\begin{note}
	Recall that for Banach spaces $X$ and $Y$, $$\cur : L^n(X; Y) \rightarrow L(X; L(X; \cdots; L(X; Y)) \cdots)$$ is an  isometric isomorphism and we may identify $L(X; L(X; \cdots; L(X; Y)) \cdots)$ as $L^n(X; Y)$. 
	\end{note}	
	
	\begin{defn} \ld{62003}\tbf{Frechet Derivative:} \\
	Let $X, Y$ be a banach spaces, $A \subset X$ open, $f:A \rightarrow Y$ and $x_0 \in A$. 
	\begin{enumerate}
	\item 
	\begin{itemize}
	\item Then $f$ is said to be \tbf{ Frechet differentiable at $x_0$} if there exists $Df(x_0) \in L(X; Y)$ such that, $$f(x_0 + h) = f(x_0) + Df(x_0)(h) + o(\|h\| ) \hspace{.5cm} \text{ as } h \rightarrow 0$$  
	\item If $f$ is Frechet differentiable at $x_0$, we define the \tbf{ Frechet derivative of $f$ at $x_0$} to be $Df(x_0)$. 
	\item We say that $f$ is \tbf{ Frechet differentiable} if for each $x \in A$, $f$ is Frechet differentiable at $x$. 
	\item If $f$ is Frechet differentiable, we define the \tbf{ Frechet derivative of $f$}, denoted $Df:A \rightarrow L(X; Y)$, by $x \mapsto D^{(1)}f(x)$. 
	\end{itemize}
	\item Continuing inductively, we set $D^0f = f$ and for $n \geq 2$, 
	\begin{itemize}
	\item $f$ is said to be \tbf{$n$-th order Frechet differentiable at $x_0$} if $f$ is $(n-1)$-th order Frechet differentiable and $D^{n-1}f$ is Frechet differentiable at $x_0$. 
	\item If $f$ is $n$-th order Frechet differentiable at $x_0$, we define $D^nf(x_0) \in L^n(X; Y)$ by 
	$$D^nf(x_0) = D[D^{n-1}f](x_0)$$ 
	\item We say that $f$ is \tbf{$n$-th order Frechet differentiable} if $f$ is $(n-1)$-th order Frechet differentiable and for each $x \in A$, $D^{n-1}f$ is Frechet differentiable at $x$. 
	\item If $f$ is $n$-th order Frechet differentiable, we define the \tbf{$n$-th order Frechet derivative of $f$}, denoted $D^nf:A \rightarrow  L^n(X; Y)$, by $x \mapsto D^{n}f(x)$ \\
	\end{itemize}		
	\item If $f$ is $n$-th order differentiable, then $f$ is said to be \tbf{continuously $n$-th order differentiable} if $D^nf$ is continuous. We define $$C^n(A,Y) = \{f:A \rightarrow Y: f \text{ is continuously $n$-th order differentiable}\}$$
	
	\end{enumerate}
	\end{defn}
	
	\begin{ex} \lex{62004}
	Let $X, Y$ be a banach spaces, $A \subset X$ open, $f,g:A \rightarrow Y$, $\lam \in \R$ and $x_0 \in A$. If $f$ and $g$ are Frechet differentiable at $x_0$, then $f+ \lam g$ is Frechet differentiable at $x_0$ and $D(f+\lam g)(x_0) = Df(x_0) + \lam Dg(x_0)$.
	\end{ex}
	
	\begin{proof}
	Suppose that $f$ and $g$ are Frechet differentiable at $x_0$. Then $$f(x_0 + h) = f(x_0) + Df(x_0)(h) + o(\|h\| ) \hspace{.5cm} \text{ as } h \rightarrow 0$$  and $$g(x_0 + h) = g(x_0) + Dg(x_0)(h) + o(\|h\| ) \hspace{.5cm} \text{ as } h \rightarrow 0$$  
	This implies that 
	\begin{align*}
	(f+\lam g)(x_0 + h) 
	&= f(x_0 + h) +\lam g(x_0 + h) \\
	&= f(x_0) + Df(x_0)(h) + o(\|h\| ) + \lam g(x_0) + \lam Dg(x_0)(h) + o(\|h\| ) \\
	&= (f+\lam g)(x_0) + [Df(x_0)+ \lam Dg(x_0)](h) + o(\|h\|) \hspace{.5cm} \text{ as } h \rightarrow 0
	\end{align*}
	Since $Df(x_0)+\lam Dg(x_0) \in L(X; Y)$, $f+\lam g$ is Frechet differentiable at $x_0$ and $D(f+\lam g)(x_0) = Df(x_0) + \lam Dg(x_0)$. 
	\end{proof}
	
	\begin{ex}\lex{62004.5}
	Let $X, Y$ be a banach spaces, $A \subset X$ open, $f:A \rightarrow Y$ and $x_0 \in A$. If $f$ is Frechet differentiable at $x_0$, then $f$ is continuous at $x_0$. 
	\end{ex}
	
	\begin{proof}
	Suppose that $f$ is Frechet differentiable at $x_0$. Then $f(x) - f(x_0) = Df(x_0)(x - x_0) + o(\|x - x_0\|)$ as $x \rightarrow x_0$. Hence $\|f(x) - f(x_0)\| \leq \| Df(x_0)\| \|x - x_0 \| + o(\|x - x_0\|)$ as $x \rightarrow x_0$. This implies that $f(x) \rightarrow f(x_0)$ as $x \rightarrow x_0$ and therefore $f$ is continuous at $x_0$.
	\end{proof}
	
	\begin{ex} \lex{62005}
	Let $X, Y$ be a banach spaces, $A \subset X$ open, $f:A \rightarrow Y$ and $x_0 \in A$. If $f$ is Frechet differentiable at $x_0$, then $f$ is Gateaux differentiable at $x_0$ and $df(x_0) = Df(x_0)$.
	\end{ex}
	
	\begin{proof}
	Suppose that $f$ is Frechet differentiable at $x_0$. Then $f(x_0 + h) = f(x_0) + Df(x_0)(h) + o(\|h\| )$ as $h \rightarrow 0$. Let $x \in X$. Then $f(x_0 + tx) - f(x_0) = tDf(x_0)(x) + o(t)$ as $t \rightarrow 0$. This implies that $f$ is differentiable at $x_0$ in the direction $x$ and 
	\begin{align*}
	df(x_0)(x) 
	&= \lim_{t \rightarrow 0} \frac{f(x_0 + tx) - f(x_0)}{t} \\
	&= Df(x_0)(x)
	\end{align*}
	Since $x \in X$ is arbitrary, $f$ is Gateaux differentiable at $x_0$ and $df(x_0) = Df(x_0)$.
	\end{proof}
	
	\begin{ex} \lex{62006}
	Let $X$ be a Banach space, $A \subset X$ open, $f:A \rightarrow \R$ and $x_0 \in A$. If $f$ is Frechet differentiable at $x_0$ and $f$ has a local extremum at $x_0$, then $Df(x_0) = 0$.
	\end{ex}	
	
	\begin{proof}
	Suppose that $f$ is Frechet differentiable at $x_0$ and $f$ has a local extremum at $x_0$. Two previous exercises imply that $f$ is Gateaux differentiable at $x_0$ and 
	\begin{align*}
	Df(x_0) 
	&= df(x_0) \\
	&= 0
	\end{align*}	
	\end{proof}
	
	\begin{defn} \ld{62007}
	Let $X, Y$ be a banach spaces, $A \subset X$ open, $f:A \rightarrow Y$ and $x_0 \in A$. Suppose that $f$ is Frechet differentiable at $x_0$. Define $R_f(x_0): A - x_0 \rightarrow Y$ by $$R_f(x_0)(h) = f(x_0 + h) - f(x_0) - Df(x_0)(h)$$
	\end{defn}
	
	\begin{ex} \lex{62008}
	Let $X, Y$ be a banach spaces, $A \subset X$ open, $f:A \rightarrow Y$ and $x_0 \in A$. If $f$ is Frechet differentiable at $x_0$, then $$f(x_0+h) - f(x_0) = O(\|h\|) \hspace{.1cm} \text{ as } h \rightarrow 0$$ 
	\end{ex}
	
	\begin{proof}
	Suppose that $f$ is Frechet differentiable at $x_0$. Then $R_f(h) = o(\|h\|)$ as $h \rightarrow 0$. Hence there exists $\del >0$ such that $B(0, \del) \subset A - x_0$ and for each $h \in B(0, \del)$, $\|R_f(h)\| \leq \|h\|$. Hence for each $h \in B(0, \del)$
	\begin{align*}
	\|f(x_0+h) - f(x_0) \| 
	&= \|Df(x_0)(h) + R_f(x_0)(h)\| \\
	& \leq \|Df(x_0)(h)\| + \|R_f(x_0)(h)\|  \\
	& \leq \|Df(x_0)\| \|(h)\| + \|h\| \\
	& = (\|Df(x_0)\| + 1) \|h\|
	\end{align*}
	\end{proof}
	
	\begin{ex} \lex{62009}\tbf{Chain Rule:}\\
	Let $X, Y, Z$ be a Banach spaces, $A \subset X$ open, $B \subset Y$ open, $f:A \rightarrow Y$, $g:B \rightarrow Z$ and $x_0 \in A$. Suppose that $f(x_0) \in B$. If $f$ is Frechet differentiable at $x_0$ and $g$ is Frechet differentiable at $f(x_0)$, then $g \circ f$ is Frechet differentiable at $x_0$ and $$D(g \circ f)(x_0) = Dg(f(x_0)) \circ Df(x_0)$$
	\end{ex}
	
	\begin{proof}
	Suppose that $f$ is Frechet differentiable at $x_0$ and $g$ is Frechet differentiable at $f(x_0)$. 
	
	\begin{itemize}
	\item The previous exercise implies that there exists $\del^* >0$ and $K > 0$ such that for each $h \in B(0, \del^*)$, $\| f(x_0 + h) - f(x_0) \| \leq K \|h\|$. Let $\ep >0$. Since $R_g(f(x_0))(h') = o(\|h'\|)$ as $h' \rightarrow 0$, there exists $\del' >0$ such that for each $h' \in B(0, \del')$, $\|R_g(f(x_0))(h')\| \leq \frac{\ep}{K} \|h'\|$. \\ Choose $\del = \min(\del' / K, \del^*)$. Let $h \in B(0, \del)$. Then 
	\begin{align*}
	\| f(x_0 + h) - f(x_0) \| 
	& \leq K \|h\| \\
	&< \del' 
	\end{align*}
	This implies that 
	\begin{align*}
	\|R_g(f(x_0))(f(x_0 + h) - f(x_0))\| 
	& \leq \frac{\ep}{K} \|f(x_0 + h) - f(x_0)\| \\
	& \leq \frac{\ep}{K} K\|h\| \\
	& \leq \ep \|h\|
	\end{align*}
	So $R_g(f(x_0))(f(x_0 + h) - f(x_0)) = o(\|h\|)$ as $h \rightarrow 0$. \\
	\item Since $\|Dg(f(x_0))(R_f(x_0)(h))\| \leq \|Dg(f(x_0)) \| \|R_f(x_0)(h)\|$ and $R_f(x_0)(h) = o(h)$ as $h \rightarrow 0$, we have that $Dg(f(x_0))(R_f(x_0)(h)) = o(h)$ as $h \rightarrow 0$. \\
	\item Combining the previous two observations, we have that $Dg(f(x_0))(R_f(x_0)(h)) + R_g(f(x_0))(f(x_0 + h) - f(x_0)) = o(\|h\|)$ as $h \rightarrow 0$. \\
	\item All together, we obtain 
	\begin{align*}
	g \circ f(x_0 + h) 
	&=  g (f(x_0)) + f(x_0 + h) - f(x_0)) \\
	&= g(f(x_0)) + Dg(f(x_0))(f(x_0 + h) - f(x_0)) + R_g(f(x_0))(f(x_0 + h) - f(x_0)) \\
	&= g(f(x_0)) + Dg(f(x_0))(Df(x_0)(h) + R_f(x_0)(h)) \\
	& \hspace{2.05cm} +  R_g(f(x_0))(f(x_0 + h) - f(x_0)) \\
	&= g(f(x_0)) + Dg(f(x_0))(Df(x_0)(h)) + Dg(f(x_0))(R_f(x_0)(h)) \\
	& \hspace{2.05cm} +  R_g(f(x_0))(f(x_0 + h) - f(x_0)) \\
	&= g \circ f(x_0) + Dg(f(x_0)) \circ Df(x_0)(h) + o(\|h\|) \text{ as } h \rightarrow 0
	\end{align*}
	So $g \circ f$ is Frechet differentiable at $x_0$ and $D(g \circ f)(x_0) = Dg(f(x_0)) \circ Df(x_0)$.
	\end{itemize}
	\end{proof}
	
	\begin{ex} \lex{62010}
	Let $Y$ be a Banach space, $A \subset \R$ open and $f:A \rightarrow Y$. Then $f$ is Gateaux differentiable iff $f$ is Frechet differentiable.
	\end{ex}
	
	\begin{proof}
	Suppose that $f$ is Gateaux differentiable. Let $x_0 \in A$. \rex{61009} implies that $df(x_0) \in L(\R, Y)$. By defintion, $$  \lim_{h \rightarrow 0} \bigg \| \frac{f(x_0 + h) - f(x_0)}{h} - df(x_0)(1) \bigg \| = 0$$ 
	This is equivalent to saying that $$f(x_0 + h) = f(x_0) + df(x_0)(h) + o(|h|) \hspace{.5cm} \text{ as } h \rightarrow 0$$
	So $f$ is Frechet differentiable at $x_0$ and $Df(x_0) = df(x_0)$.
	\end{proof}
	
	
	
	
	
	
	
	
	\newpage
	\section{The Calc I Derivative}
	\begin{defn} \ld{}\tbf{Calc I Derivative:}\\
	Let $Y$ be a Banach space, $A \subset \R$ or $\C$ open, $f:A \rightarrow Y$ and $x_0 \in A$. 
	\begin{enumerate}
	\item 
	\begin{itemize}
	\item If $f$ is Frechet differentiable at $x_0$, we define the \tbf{calc I derivative of $f$ at $x_0$}, denoted $$f'(x_0) \text{ or } \frac{df}{dt}(x_0)$$ by
	\begin{align*}
	f'(x_0) 
	& \defeq \lim_{t \rightarrow 0} \frac{f(x_0 + t) - f(x_0)}{t} \\
	&= df(x_0)(1) \\
	&= Df(x_0)(1)
	\end{align*}
	\item If $f$ is Frechet differentiable, we define $f':A \rightarrow Y$ by $x \mapsto f'(x)$. 
	\end{itemize}
	\item Continuing inductively, we set $f^{(0)} = f$ and for $n \geq 1$,
	\begin{itemize}
	\item  if $f^{(n-1)}$ is Frechet differentiable at $x_0$, we define the \tbf{$(n)$-th order calc I derivative of $f$ at $x_0$}, denoted $f^{(n)}(x_0)$, by $$f^{(n)}(x_0) = [f^{(n-1)}]'(x_0)$$ 
	\item if $f^{(n-1)}$ is Frechet differentiable, we define $f^{(n)}:A \rightarrow Y$ by $$f^{(n)} = [f^{(n - 1)}]'$$ 
	\end{itemize}
	\end{enumerate}
	\end{defn}	
	
	\begin{ex} \lex{}
	Let $Y$ be a Banach space, $A \subset \R$ open and $f:A \rightarrow Y$. If $f$ is $n$-th order Frechet differentiable, then for each $x_0 \in A$ and $k \in \{1, \cdots, n\}$, $$f^{(k)}(x_0) = D^kf(x_0)(1^{\oplus k})$$
	\end{ex}
	
	\begin{proof}
	Let $x_0 \in A$. We proceed by induction. The base case is true by definition. Let $k \in \{1, \cdots, n\}$. Suppose the claim is true for $k-1$. Then $$f^{(k-1)}(x_0) = D^{k-1}f(x_0)(1^{\oplus (k-1)})$$
	Since $f$ is $n$-th order Frechet differentiable, $$D^{k-1}f(x_0+h) = D^{k-1}f(x_0) + D^kf(x_0)(h) + o(\|h\|) \hspace{.5cm} \text{ as } h \rightarrow 0$$ 
	This implies that 
	\begin{align*}
	f^{(k-1)}(x_0+h) 
	&=  D^{k-1}f(x_0+h)(1^{\oplus (k-1)}) \\
	&= D^{k-1}f(x_0)(1^{\oplus (k-1)}) + D^kf(x_0)(h)(1^{\oplus (k-1)}) + o(\|h\|) \hspace{.5cm} \text{ as } h \rightarrow 0
	\end{align*}
	Therefore for each $h \in \R$, $$Df^{(k-1)}(x_0)(h) = D^kf(x_0)(h)(1^{\oplus (k-1)})$$
	and by definition,
	\begin{align*}
	f^{(k)}(x_0) 
	&= [f^{(k-1)}]'(x_0) \\
	&= Df^{(k-1)}(x_0)(1) \\
	&=  D^kf(x_0)(1^{\oplus k})
	\end{align*}
	\end{proof}
	
	
	
	
	\begin{ex} \lex{}
	Let $X,Y$ be Banach spaces, $A \subset X$ open, $f \in C^n(A, Y), x_0 \in A$, and $h \in X$. Suppose that $\{x_0 +th: t \in [0,1]\} \subset A$. Define and $g:(0,1) \rightarrow Y$ by $$g(t) = f(x_0 + th)$$
	Then for each $k \in \{1 \dots, n\}$ and $t \in (0,1)$, $$g^{(k)}(t) = D^kf(x_0 + th)(h^{\oplus k})$$
	\end{ex}
	
	\begin{proof}
	We proceed by induction. It is straightforward to show that the claim is true for $k=1$.\\
	Let $k \in \{1 \dots, n\}$. Suppose that $g^{(k-1)}(t) = D^{k-1}f(x_0 + th)(h^{\oplus (k-1)})$. Since $f \in C^k(A, Y)$, $$D^{k-1}f(x_0 + s_0h + th)= D^{k-1}f(x_0 + s_0h) + D^k f(x_0 + s_0h)(th) + o(\|t\|) \hspace{.2cm} \text{ as }t \rightarrow 0 $$ 
	The previous exercise implies that 
	\begin{align*}
	g^{(k-1)}(s_0 + t)
	&= D^{k-1}g(s_0+t)(1^{\oplus (k-1)}) \\
	&= D^{k-1}f(x_0 + s_0h + th)(h^{\oplus (k-1)}) \\
	&= D^{k-1}f(x_0 + s_0h)(h^{\oplus (k-1)}) + D^kf(x_0 + s_0h)(th)(h^{\oplus (k-1)}) + o(\|t\|) \hspace{.2cm} \text{ as }t \rightarrow 0
	\end{align*}
	Hence $$Dg^{(k-1)}(s_0)(t) = D^kf(x_0 + s_0h)(th)(h^{\oplus (k-1)})$$ 
	and 
	\begin{align*}
	g^{(k)}(t)
	&= Dg^{(k-1)}(t)(1) \\
	&= D^kf(x_0 + th)(h^{\oplus k})
	\end{align*}
	\end{proof}
	
	
	
	
	
	
	
	\newpage
	\section{Mean Value Theorem}	
	
	\begin{ex} \lex{64001}
	Let $X$ be a Banach space, $A \subset X$ open and convex, and $f:A \rightarrow \R$. If $f$ is continuous and Gateaux differentiable, then for each $x,y \in A$, there exists $t^* \in (0,1)$ such that $f(x) - f(y) = df(t^*x + (1-t^*)y)(x - y)$. 
	\end{ex}
	
	\begin{proof}
	Suppose that $f$ is continuous and Gateaux differentiable. Let $x,y \in A$. Define $h: [0,1] \rightarrow X$ by $h(t) = tx +(1-t)y$. Set $g = f \circ h: [0,1] \rightarrow \R$. Then $g$ is continuous on $[0,1]$ and \rex{61011} implies that $g$ is Gateaux differentiable on $(0,1)$. Then \rex{62010} \rex{61011} and the mean value theorem implies that there exists $t^* \in (0,1)$ such that
	\begin{align*}
	f(x) - f(y)
	&= g(1) - g(0) \\
	&=g'(t^*) \\ 
	&= dg(t^*)(1) \\
	&= df(h(t^*))(dh(t^*)(1)) \\
	&= df(h(t^*))(h'(t^*)) \\
	&= df(t^*x + (1-t^*)y)(x -y)
	\end{align*}
	\end{proof}
	
	\begin{ex} \lex{64002}
	Let $X$ be a Banach space, $A \subset X$ open and convex, and $f:A \rightarrow \R$. If $f$ is Frechet differentiable, then for each $x,y \in A$, there exists $t^* \in (0,1)$ such that $f(x) - f(y) = Df(t^*x + (1-t^*)y)(x - y)$. 
	\end{ex}
	
	\begin{proof}
	Suppose that $f$ is Frechet differentiable. Then $f$ is continuous and Gateaux differentiable. Now apply the previous exercise.	
	\end{proof}
	
	\begin{ex} \lex{64003}\tbf{Mean Value Theorem:}\\
	Let $X, Y$ be a Banach spaces, $A \subset X$ open and convex and $f:A \rightarrow Y$. If $f$ is Frechet differentiable, then for each $x,y \in A$, there exists $t^* \in (0,1)$ such that $$\|f(x) - f(y)\| \leq \|Df(t^*x + (1-t^*)y)\|\|x-y\|$$
	\tbf{Hint:} For $x,y \in A$ with $f(x) \neq f(y)$, using a Hahn-Banach argument, find $\lam \in Y^*$ such that $\|\lam\| = 1 $ and $\lam (f(x) - f(y)) = \|f(x) - f(y)\|$.
	\end{ex}
	
	\begin{proof}
	Suppose that $f$ is Frechet differentiable. Let $x,y \in A$. The claim is clearly true when $f(x) = f(y)$. Suppose that $f(x) \neq f(y)$. An exercise in the section on linear functionals implies that there exists $\lam \in Y^*$ such that $\lam(f(x)-f(y)) = \| f(x) - f(y)\|$ and $\|\lam \| = 1$
	Define $g:[0,1] \rightarrow \R$ by $$g(t) = \lam(f(tx +(1-t)y))$$ Then $g$ is continuous and (Frechet) differentiable on $(0,1)$ with $$Dg(t)(h) = \lam \circ Df(tx+(1-t)y)((x-y)h)$$ which implies that
	\begin{align*}
	g'(t) 
	&= Dg(t)(1)\\
	&= \lam \circ Df(tx+(1-t)y)((x-y))
	\end{align*}
	The mean value theorem implies that there exists $t^* \in (0,1)$ such that 
	\begin{align*}
	\|f(x) - f(y)\|
	&= \lam(f(x) - f(y)) \\
	&= g(1) - g(0) \\
	&= g'(t^*)\\
	&= \lam \circ Df(t^*x+(1-t^*)y)((x-y))
	\end{align*}
	Taking absolute values, we see that 
	\begin{align*}
	\|f(x) - f(y)\|
	&= |\lam \circ Df(t^*x+(1-t^*)y)((x-y))| \\
	& \leq \|\lam \| \|Df(t^*x+(1-t^*)y)\|\|x-y\| \\
	& \leq \|Df(t^*x+(1-t^*)y)\|\|x-y\|
	\end{align*}
	\end{proof}
	
	\begin{ex} \lex{64004}
	Let $X, Y$ be Banach spaces, $A \subset X$ open and convex and $f:A \rightarrow Y$. Suppose that $f$ is Frechet differentiable. If for each $x \in A$, $Df(x) = 0$, then $f$ is constant.
	\end{ex}
	
	\begin{proof}
	Suppose that for each $x \in A$, $Df(x) = 0$. Let $x,y \in A$. Then the mean value theorem implies that there exists $t \in (0, 1)$ such that 
	\begin{align*}
	\|f(x) - f(y)\| 
	&\leq \|Df(tx + (1-t)y)\| \|x-y\| \\
	&= 0
	\end{align*}
	So $f(x) = f(y)$. 
	\end{proof}
	
	\begin{ex} \lex{64005}
	Let $X, Y$ be Banach spaces, $A \subset X$ open and convex and $f,g:A \rightarrow Y$. Suppose that $f$ and $g$ are Frechet differentiable. If $Df = Dg$, then there exists $c \in Y$ such that $f = g+c$.
	\end{ex}
	
	\begin{proof}
		Suppose that $Df = Dg$. Then $D(f-g) = 0$ and the previous exercise implies that $f -g$ is constant.
	\end{proof}		
	
	\begin{ex} \lex{64006}
	Let $X, Y$ be a Banach spaces, $A \subset \R$ open and $f:A \rightarrow Y$. Suppose that $f$ is Frechet differentiable. Then $f' \in C(A,Y)$ iff $f \in C^1(A,Y)$.
	\end{ex}
	
	\begin{proof}
	Suppose that $f' \in C(A, Y)$. Let $x,y \in A$ and $h \in \R$. Then 
	\begin{align*}
	\|(Df(x)- Df(y))(h)\| 
	&= \|Df(x)(h) - Df(y)(h)\| \\
	&=  \|hf'(x) - hf'(y)\| \\
	&= \|h(f'(x) - f'(y))\| \\
	&= \|f'(x) - f'(y)\||h|
	\end{align*}
	So $\|Df(x) - Df(y)\| \leq \|f'(x) - f'(y)\|$. Hence continuity of $f'$ implies continuity of $Df$ and $f \in C^1(A, Y)$.
	Conversely, suppose that $f \in C^1(A, Y)$. Let $x,y \in A$. Then 
	\begin{align*}
	\|f'(x) - f'(y)\| 
	&= \|Df(x)(1) - Df(y)(1)\| \\
	&= \|(Df(x) - Df(y))(1)\| \\
	& \leq \| Df(x) - Df(y)\|
	\end{align*}
	Hence continuity of $Df$ implies continuity of $f'$ and $f' \in C(A, Y)$.
	\end{proof}

	\begin{ex}
		Let $X,Y$ be Banach spaces, $A \subset X$ open and convex and $f:A \rightarrow Y$. Suppose that $f$ is Frechet differentiable. Then $f$ is Lipschitz iff $Df$ is bounded.
	\end{ex}

	\begin{proof}
		Suppose that $f$ is Lipschitz. Then there exists $M > 0$ such that for each $x, y \in A$, $\|f(y) - f(x)\| \leq M \|y - x\|$. Let $x \in A$ and $h \in X$. Suppose that $\|h\| = 1$. Since $f(x+th) = f(x) + Df(x)(th) + o(|t|)$ as $t \rightarrow 0$, we have that 
		\begin{align*}
			|t|\|Df(x)(h)\| 
			& = \|Df(x)(th)\| \\
			& \leq \|f(x +th) - f(x)\| + o(|t|) \text{ as $t \rightarrow 0$} \\
			& \leq M \|th\|  + o(|t|)  \text{ as $t \rightarrow 0$}\\
			& = M |t|  + o(|t|)  \text{ as $t \rightarrow 0$}\\
		\end{align*}
		Hence $\|Df(x)(h)\| \leq M + o(1) $ as $t \rightarrow 0$ which implies that $\|Df(x)(h)\| \leq M $. Thus 
		\begin{align*}
			\|Df(x)\| 
			& = \sup \{\|Df(x)(h)\|: h \in X \text{ and $\|h\| =1 $}\} \\
			& \leq M
		\end{align*}
		Since $x \in A$ is arbitrary, $Df$ is bounded.\\
		Conversely, suppose that $Df$ is bounded. Then there exists $M > 0$ such that for each $x \in A$, $\|Df(x)\| \leq M$. Let $x,y \in A$. The mean value theorem implies that there exists $t^* \in (0,1)$ such that 
		\begin{align*}
			\|f(x) - f(y)\| 
			& \leq \|Df(t^*x + (1-t^*)y)\|\|x-y\| \\
			& \leq M \|x-y\|
		\end{align*}
		Therefore $f$ is Lipschitz.
	\end{proof}
	
	
	
	
	
	
	
	
	
	
	\newpage
	\section{Taylor's Theorem}
	
	\begin{ex} \lex{}
	Let $Y$ be a separable Banach space, $f:[a,b] \rightarrow Y$ continuous so that $f$ is Bochner-integrable. Define $F:(a,b) \rightarrow Y$ by  $$F(x) = \int_{(a, x]}f dm$$ Then $F \in C^1((a,b), Y)$ and for each $x_0 \in (a,b)$ and $F'(x_0) = f(x_0)$.
	\end{ex}
	
	\begin{proof}
	Let $x_0 \in (a,b)$ and $h \in (0, b-x_0)$. Then continuity implies that
	\begin{align*}
	\frac{1}{\|h\|} \bigg | \int_{(x_0, x_0 + h]}f - f(x_0) dm \bigg |
	& \leq  \frac{1}{\|h\|} \max_{x \in (x_0, x_0+h]} |f(x) - f(x_0)| \|h\| \\
	&= \max_{x \in [x_0, x_0+h]} |f(x) - f(x_0)| \\
	& \rightarrow 0  \text{ as } h \rightarrow 0
\end{align*}	  
So $$\int_{(x_0, x_0 + h]}f - f(x_0) dm = o(\|h\|) \hspace{1cm}\text{ as }h \rightarrow 0$$ 
	Therefore 
	\begin{align*}
	F(x_0 + h)
	&= \int_{(a, x_0 + h]} f dm  \\
	&= \int_{(a, x_0]} f dm + \int_{(x_0, x_0 + h]} fdm \\
	&= \int_{(a, x_0]} f dm + hf(x_0) + \int_{(x_0, x_0 + h]} f - f(x_0) dm \\ 
	&= F(x_0 ) + hf(x_0) + o(\|h\|) \hspace{1cm }\text{ as } h \rightarrow 0\\
	\end{align*}
	The case is similar for $h \in (x_0 - b, 0)$. Since the map $h \mapsto f(x_0)h$ is bounded, $F$ is Frechet differentiable at $x_0$ and $DF(x_0)(h) = f(x_0)h$. This implies that $F'(x_0) = f(x_0)$ and a previous exercise implies tells us that continuity of $f$ implies continuity of $DF$. So $F \in C^1(A, Y)$.
	\end{proof}
	
	\begin{ex} \lex{}\tbf{Fundamental Theorem of Calculus:}
	Let $Y$ be a separable Banach space and $f \in C^1((a,b), Y)$. Then for each $x, x_0 \in (a,b)$, $x_0 < x$ implies that 
	\begin{enumerate}
	\item $f'$ is Bochner integrable on $(x_0, x]$ 
	\item  $$f(x) - f(x_0) = \int_{(x_0, x]}f'dm$$ 
	\end{enumerate}
	\end{ex}

	\begin{proof}\
	\begin{enumerate}
	\item Since $f \in C^1((a,b), Y)$, a previous exercise tells us that $f' \in C_Y(a,b)$. Let $x, x_0 \in (a,b)$. Suppose that $x_0 < x$. Choose $c,d \in (a,b)$ such that $a < c < x_0< x< d < b$. Then $f'$ is continuous on $[c,d]$ and hence Bochner-integrable on $(c,d]$ and $(x_0,x]$. 
	\item Define $g: (c,d) \rightarrow Y$ by $$g(\xi) = \int_{(c, \xi]}f'dm$$
	Then the previous exercise implies that $g \in C^1_Y(c,d)$ and for each $t \in (c, d)$, $g'(t) = f'(t)$. Let $t \in (c,d)$ and $h \in \R$. Then
	\begin{align*}
	Dg(t)(h) 
	&= hg'(t) \\
	&= hf'(t) \\
	&= Df(t)(h)
	\end{align*}
	So $Dg = Df$ on $(c,d)$. A previous exercise implies that there exists $c \in Y$ such that $f = g + c$ on $(c, d)$. Then 
	\begin{align*}
	f(x) - f(x_0)
	&= g(x)+c - (g(x_0)+c) \\
	&= g(x) - g(x_0) \\
	&= \int_{(c, x]}f'dm - \int_{(c, x_0]}f'dm\\
	&= \int_{(x_0, x]}f'dm
	\end{align*}
	\end{enumerate}
	\end{proof}
	
	
	\begin{ex} \lex{}
	Let $Y$ be a Banach space, $A \subset \R$ open and $g:A \rightarrow Y$. If $g$ is $n$-th order Frechet differentiable, then 
	$$\frac{d}{dt} \sum_{k=0}^{n-1} \frac{(1-t)^k}{k!}g^{(k)}(t) = \frac{(1-t)^{n-1}}{(n-1)!}g^{(n)}(t)$$
	\end{ex}
	
	\begin{proof}
	Taking the derivative yields a telescoping series.
	\end{proof}
	
	
	
	\begin{ex} \lex{} \tbf{Taylor's Theorem I:}\\
	Let $X$ be a Banach space, $Y$ a separable Banach space, $A \subset X$ open and convex, $f\in C^{n+1}(A, Y)$, $x_0 \in A$, and $h \in A - x_0$. Then $$f(x_0 + h) = \sum_{k=0}^{n} \frac{1}{k!} D^k f(x_0)(h^{\oplus k}) + R(x_0)(h)$$ 
	where $R(x_0): A - x_0 \rightarrow Y$ is defined by $$R(x_0)(h) = \frac{1}{n!}\int_{(0,1)} (1-t)^{n}D^{n+1}f(x_0 + th)(h^{\oplus (n+1)})d m(t)$$
	and $R(x_0)(h) = o(\|h\|^{n})$ as $h \rightarrow 0$.\\
	\tbf{Hint:} Define $g: (0,1) \rightarrow Y$ by $$g(t) = f(x_0 +t h)$$ Then use the previous exercise and the fundamental theorem of calculus.
	\end{ex}
	
	\begin{proof}
	Let $h \in X$. Suppose that $x_0 + h \in A$. Define $g: (0,1) \rightarrow Y$ by 
	$$g(t) = f(x_0 +t h)$$ 
	For each $k \in \{1, \dots, n+1\}$, a previous exercise implies that $g^{(k)}(t) = D^kf(x_0 + th)(h^{\oplus k})$, so $g^{(k)}(0) = D^kf(x_0)(h^{\oplus k})$. The previous exercise and the fundamental theorem of calculus tell us that 
	\begin{align*}
	f(x_0 +h) - \sum_{k=0}^{n} \frac{1}{k!}D^kf(x_0)(h^{\oplus k})
	&= g(1) - \sum_{k=0}^{n} \frac{1}{k!}g^{(k)}(0)\\
	&= \int_{(0,1)} \bigg [\frac{d}{dt} \sum_{k=0}^{n} \frac{(1-t)^k}{k!}g^{(k)}(t)\bigg ] dm(t) \\
	&= \int_{(0,1)} \frac{(1-t)^{n}}{n!}g^{(n+1)}(t) dm(t) \\
	&= \frac{1}{n!}\int_{(0,1)} (1-t)^{n}D^{n+1}f(x_0 + th)(h^{\oplus (n+1)})d m(t) \\
	&= R(x_0)(h)
	\end{align*}	
	Note that $$\frac{1}{n+1} = \frac{1}{n!}\int_{(0,1)} (1-t)^{n} dm(t)$$ 
	Since $D^{n+1}f$ is continuous at $x_0$, there exists $\del_1 >0$ such that for each $h \in B(0, \del_1)$, $x_0 + h \in A$ and 
	$$\|D^{n+1} f(x_0+h) - D^{n+1}f(x_0)\| < 1 $$  
	Let $\ep >0$. Choose $\del_2 >0$ such that $$\frac{1}{n+1} \bigg( \|D^{n+1}f(x_0 )\|  +  1 \bigg) \del_2 < \ep$$ Set $\del = \min(\del_1, \del_2)$. Let $h \in B(0, \del)$. Then
	\begin{align*}
	\|R(x_0)(h)\| 
	&= \bigg \| \int_{(0,1)} \frac{1}{n!}\int_{(0,1)} (1-t)^{n}D^{n+1}f(x_0 + th)(h^{\oplus (n+1)})d m(t) \bigg\| \\
	&\leq \frac{1}{n!}\int_{(0,1)} \|(1-t)^{n}D^{n+1}f(x_0 + th)(h^{\oplus (n+1)}) \|dm(t)\\
	&\leq \max_{t \in [0,1]}\|D^{n+1}f(x_0 + th)\| \|h\|^{n+1} \frac{1}{n!}\int_{(0,1)} (1-t)^{n} dm(t)  \\
	&\leq \frac{1}{n+1}  \bigg(\|D^{n+1}f(x_0 )\| +  \max_{t \in [0,1]} \|D^{n+1} f(x_0+th) - D^{n+1}f(x_0)\| \bigg)\|h\|^{n+1}  \\
	&< \frac{1}{n+1}\bigg(\|D^{n+1}f(x_0 )\|  +  1 \bigg)\|h\|^{n+1}  \\
	&<\ep \|h\|^n
	\end{align*}
	So $R(x_0)(h) = o(\|h\|^{n})$ as $h \rightarrow 0$.
	\end{proof}
	
	
	\begin{ex} \lex{} \tbf{Taylor's Theorem II:}\\
	Let $X$ be a Banach space, $Y$ a separable Banach space, $A \subset X$ open and convex, $f\in C^{n}(A, Y)$, $x_0 \in A$, and $h \in A - x_0$. Then there exists $R(x_0): A - x_0 \rightarrow Y$ such that $$f(x_0 + h) = \sum_{k=0}^{n} \frac{1}{k!} D^k f(x_0)(h^{\oplus k}) + R(x_0)(h)$$ and $R(x_0)(h) = o(\|h\|^n)$ as $ h \rightarrow 0$. \\
	\tbf{Hint:} use Taylor's theorem and expand the derivative inside the integral.
	\end{ex}
	
	\begin{proof}
	This is clear by definition for $n = 1$. Suppose that $n \geq 2$. Taylor's theorem implies that $$f(x_0 + h) = \sum_{k=0}^{n-2} \frac{1}{k!} D^k f(x_0)(h^{\oplus k}) + S(x_0)(h)$$ 
	where $S(x_0): A - x_0 \rightarrow Y$ is defined by 
	$$S(x_0)(h) = \frac{1}{(n-2)!}\int_{(0,1)} (1-t)^{n-2}D^{n-1}f(x_0 + th)(h^{\oplus (n-1)})d m(t)$$
	
	and $S(x_0; h) = o(\|h\|^{n-2})$ as $h \rightarrow 0$.
	Define $T^{n-1}(x_0):A-x_0 \rightarrow L^{n-1}(X;Y)$ by 
	$$T^{n-1}(x_0)(h) = D^{n-1}f(x_0 + h) - D^{n-1}f(x_0) - D^nf(x_0)(h)$$ 
	so that 
	$$D^{n-1}f(x_0 + h) = D^{n-1}f(x_0) + D^nf(x_0)(h) + T^{n-1}(x_0)(h)$$ 
	and $T^{n-1}(x_0)(h) = o(\|h\|)$ as $h \rightarrow 0$. \\
	Define $R(x_0): A - x_0 \rightarrow Y$ by 
	$$R(x_0)(h) = \frac{1}{(n-2)!} \int_{(0,1)} (1-t)^{n-2}T^{n-1}(x_0)(th)(h^{\oplus (n-1)}) dm(t) $$
	Note that 
	\begin{itemize}
	\item $$\int_0^1 (1-t)^{n-2} dt = \frac{1}{n-1}$$
	\item $$\int_0^1 (1-t)^{n-2}t dt = \frac{1}{n(n-1)}$$
	\end{itemize}
	Let $\ep >0$. Choose $\del >0$ such that for each $h \in B(0, \del)$, $h \in A - x_0$ and $$\|T^{n-1}(x_0)(h)\| \leq \ep n! \|h\|$$ Let $h \in B(0, \del)$. Then 
	\begin{align*}
	\|R(x_0)(h)\|
	&= \bigg \|  \frac{1}{(n-2)!} \int_{(0,1)} (1-t)^{n-2}T^{n-1}(x_0)(th)(h^{\oplus (n-1)}) dm(t) \bigg \| \\
	& \leq  \frac{1}{(n-2)!} \int_{(0,1)} (1-t)^{n-2}\|T^{n-1}(x_0)(th)(h^{\oplus (n-1)})\| dm(t) \\
	& \leq  \frac{1}{(n-2)!} \int_{(0,1)} (1-t)^{n-2} \|T^{n-1}(x_0)(th) \| \|h\|^{n-1} dm(t) \\
	& \leq  \frac{\ep}{(n-2)!}  n!  \|h\|^n \int_{(0,1)} (1-t)^{n-2} t  dm(t) \\		
	&= \ep \|h\|^n  
	\end{align*}
	So that $R(x_0)(h) = o(\|h\|^n)$ as $h \rightarrow 0$. 
	
	Then 
	\begin{align*}
	S(x_0)(h) 
	&= \frac{1}{(n-2)!}\int_{(0,1)} (1-t)^{n-2}D^{n-1}f(x_0 + th)(h^{\oplus (n-1)})d m(t) \\
	&= \frac{1}{(n-2)!}\int_{(0,1)} (1-t)^{n-2} D^{n-1}f(x_0)(h^{\oplus (n-1)}) dm(t) \\ 
	& \hspace{1.5cm} + \frac{1}{(n-2)!} \int_{(0,1)} (1-t)^{n-2}t D^{n}f(x_0)(h)(h^{\oplus (n-1)}) dm(t) \\
	& \hspace{1.5cm} + \frac{1}{(n-2)!} \int_{(0,1)} (1-t)^{n-2}T^{n-1}(x_0)(th)(h^{\oplus (n-1)}) dm(t) \\
	&= \frac{1}{(n-1)!}D^{n-1}f(x_0)(h^{\oplus (n-1)}) + \frac{1}{n!}D^{n}f(x_0)(h^{\oplus n}) + R_f(x_0)(h)
	\end{align*}
	Hence 
	\begin{align*}
	f(x_0 + h) 
	&= \sum_{k=0}^{n-2} \frac{1}{k!} D^k f(x_0)(h^{\oplus k}) + S(x_0)(h) \\
	&= \sum_{k=0}^{n} \frac{1}{k!} D^k f(x_0)(h^{\oplus k}) + R(x_0)(h) 
	\end{align*}
	
	\end{proof}
	
	
	\begin{ex} \lex{} \tbf{Taylor's Theorem III:}\\
	Let $X$ be a Banach space, $A \subset X$ open and convex, $f \in C^{n}(A)$, $x_0 \in A$, and $h \in A - x_0$. Then there exists $t^* \in (0,1)$ such that $$f(x_0 + h) = \sum_{k=0}^{n-1} \frac{1}{k!} D^k f(x_0)(h^{\oplus k}) + \frac{1}{(n-1)!} (1-t^*)^{n-1}D^{n}f(x_0 + t^*h)(h^{\oplus n})$$ \\
	\tbf{Hint:} use Taylor's theorem and the mean value theorem.
	\end{ex}	
	
	\begin{proof}
	Taylors Theorem implies that 
	$$f(x_0 + h) = \sum_{k=0}^{n-1} \frac{1}{k!} D^k f(x_0)(h^{\oplus k}) + R(x_0)(h)$$ 
	where 
	$$R(x_0)(h) =  \frac{1}{(n-1)!}\int_{(0,1)} (1-t)^{n-1}D^{n}f(x_0 + th)(h^{\oplus n})d m(t)$$ 
	Define $F \in C^1([0,1])$ by $$F(t) = \int_{(0,t]}\frac{1}{(n-1)!}(1-s)^{n-1}D^{n}f(x_0 + sh)(h^{\oplus n})d m(s)$$ Then the fundamental theorem of calculus implies that 
	$$F'(t) = \frac{1}{(n-1)!}(1-t)^{n-1}D^{n}f(x_0 + th)(h^{\oplus n})$$ 
	The mean value theorem implies that there exists $t^* \in (0,1)$ such that 
	\begin{align*}
	R(x_0)(h)
	&= F(1) - F(0) \\
	&= F'(t^*) \\
	&= \frac{1}{(n-1)!}(1-t^*)^{n-1}D^{n}f(x_0 + t^*h)(h^{\oplus n})
	\end{align*}
	\end{proof}	

	
	
	
	\begin{ex} \lex{}
	Let $X$ be a Banach space, $A \subset X$ open and convex and $f\in C^{2}(A)$, $x_0 \in A$. If $f$ has a local minimum at $x_0$, then $D^2f(x_0)$ is positive semidefinite.   
	\end{ex}
	
	\begin{proof}
	Suppose that $f$ has a local minimum at $x_0$, then $Df(x_0) = 0$. Let $x \in X$. Then 
	\begin{align*}
	0 
	& \leq f(x+h) - f(x_0) \\
	&= \frac{1}{2} D^2f(x_0)(h,h) +o(\|h\|^2) \hspace{.2cm} \text{ as } h \rightarrow 0
	\end{align*}
	Let $h \in X$. Then $$0 \leq \frac{1}{2} t^2 D^2f(x_0)(h,h) +o(t^2) \hspace{.2cm} \text{ as } t \rightarrow 0$$
	This implies that $D^2f(x_0)(h,h) \geq 0$. So $D^2f(x_0)$ is positive semidefinite.
	\end{proof}





























	\newpage
	\section{Implicit and Inverse Function Theorems}
	\begin{defn}
		Let $(x_0,y_0) \in U$. Then $f$ is said to be \tbf{partial Frechet differentiable with respect to $X$ at $(x_0, y_0)$} if $f^{y_0}$ is Frechet differentiable at $x_0$. \\
		Suppose that $f$ is partial Frechet differentiable with respect to $X$ at $(x_0, y_0)$. We define the \tbf{partial Frechet derivative of $f$ with respect to $X$ at $(x_0, y_0)$}, denoted $D_X f(x_0, y_0) \in L(X, Z)$, by 
		$$D_Xf(x_0, y_0) = Df^{y_0}(x_0)$$
		Suppose that for each $y \in Y$, $f^{y}$ is Frechet differentiable. We define the \tbf{partial Frechet derivative of $f$ with respect to $X$}, denoted $D_X f: X \times Y \rightarrow L(X, Z)$, by 
		$$D_Xf(x, y) = Df^{y}(x)$$
		We define partial Frechet differentiability with respect to $Y$ similarly.
	\end{defn}

	\begin{ex}
		Let $X, Y$ and $Z$ be Banach spaces, $f: X \times Y \rightarrow Z$ and $(x_0, y_0) \in X \times Y$. If $f$ is Frechet differentiable at $(x_0, y_0)$, then $f$ is partial Frechet differentiable at $(x_0, y_0)$ with respect to $X$ and $Y$ and for each $h_X \in X$, $h_Y \in Y$,  
		$$Df(x_0, y_0)(h_X, h_Y) = D_Xf(x_0, y_0) (h_X) + D_Yf(x_0, y_0) (h_Y)$$
	\end{ex}

	\begin{proof}
		Suppose that $f$ is Frechet differentiable at $(x_0, y_0)$. Then 
		$$f[(x_0,y_0) + (h_X, h_Y)] = f(x_0, y_0) + Df(x_0, y_0)(h_X, h_Y) + o(\|(h_X, h_Y)\|_{X \oplus Y}) \text{ as } (h_X, h_Y) \rightarrow (0,0)$$ Since there exist $C_1, C_2 > 0$ such that for each $h_X \in X$ and $h_Y \in Y$, 
		$$C_1(\|x\| + \|y\|) \leq \|(h_x, h_y)\|_{X \oplus Y} \leq C_2(\|x\| + \|y\|)$$
		we have that 
		$$f^{y_0}(x_0 + h_X) = f^{y_0}(x_0) + Df(x_0, y_0)(h_X, 0) + o(\|h_X\|) \text{ as } h_X \rightarrow 0$$
		Therefore $f^{y_0}:X \rightarrow Z$ is Frechet differentiable at $x_0$ and $Df^{y_0}(x_0) = Df(x_0, y_0)(h_X, 0)$. Hence $f$ is partial Frechet differentiable at $(x_0, y_0)$ with respect to $X$ and for each $h_X \in X$, $D_Xf(x_0, y_0)(h_x) = Df(x_0, y_0)(h_X, 0)$. Similarly, $f$ is partial Frechet differentiable at $(x_0, y_0)$ with respect to $Y$ and for each $h_Y \in Y$, $D_Yf(x_0, y_0)(h_Y) = Df(x_0, y_0)(0, h_Y)$. Let $h_X \in X$ and $h_Y \in Y$. Then
		\begin{align*}
			Df(x_0, y_0)(h_X, h_Y) 
			& = Df(x_0, y_0)[(h_X, 0) + (0, h_Y)] \\
			& = Df(x_0, y_0)(h_X, 0) + Df(x_0, y_0)(0, h_Y) \\
			& = D_Xf(x_0, y_0)(h_x) +  D_Yf(x_0, y_0)(h_Y)
		\end{align*}
	\end{proof}

	\begin{ex}
		Let $X, Y$ and $Z$ be Banach spaces, $U \subset X \times Y$ open, $f: U \rightarrow Z$ and $n \in \N$. If $f$ is $C^1(U, Z)$, then $D_Xf$,$D_Yf \in C(U, Z)$.
	\end{ex}

	\begin{proof}
		Suppose that $f$ is $C^1(U, Z)$. Then $Df \in C(U, Z)$. Define $\phi_X:X \rightarrow X \times Y$ and $\phi_Y:Y \rightarrow X \times Y$ by $\phi_X(x) = (x, 0)$ and $\phi_Y(y) = (0, y)$. Then $\phi_X \in L(X, X \times Y)$ and $\phi_Y \in L(Y, X \times Y)$. The previous exercise implies that for each $(x,y) \in U$, $D_Xf(x,y) = Df(x, y) \circ \phi_X $. Let $(x,y)$, $(x_0, y_0) \in U$. Then 
		\begin{align*}
			\|D_Xf(x,y) - D_Xf(x_0, y_0)\|
			& = \|Df(x, y) \circ \phi_X - Df(x_0, y_0) \circ \phi_X\| \\
			& = \|(Df(x, y) - Df(x_0, y_0)) \circ \phi_X\| \\
			& \leq \|Df(x, y) - Df(x_0, y_0) \| \|\phi_X\| \\
		\end{align*}
	\end{proof}
	

	\begin{ex}
		Let $X, Y$ and $Z$ be Banach spaces, $U \subset X \times Y$ open, $F: U \rightarrow Z$, $(x_0, y_0) \in U$. Suppose that $F$ is partial Frechet differentiable with respect to $Y$ on $U$ and $F$ and $D_YF$ continuous at $(x_0, y_0)$. Then there  
	\end{ex}

	\begin{proof}
		Set $L = D_YF(x_0, y_0)$. Define $G: U \rightarrow Z$ by $G(x,y) = y - L^{-1}F(x,y)$. Then $G(x_0, y_0) = y_0$ and since $F \in C^1(U, Z)$, $G \in C^1(U, Z)$. The previous exercise implies that $D_YG \in C(U, Z)$. Note that for each $(x,y) \in U$,
		\begin{align*}
			D_YG(x, y) 
			& = \id_Y - L^{-1} D_YF(x,y) \\
			& = L^{-1}(L - D_YF(x,y))
		\end{align*} 
		which implies that $D_YG(x_0, y_0) = 0$. Set $\ep = 1/2$. Since $U$ is open and $D_YG$ is continuous at $(x_0, y_0)$ there exist $\del_X$, $\del_Y > 0$ such that for each $x \in B(x_0, \del_X)$ and $y \in B(y_0, \del_Y)$, $(x, y) \in U$ and  
		\begin{align*}
			\|D_YG(x, y)\| 
			& =  \|D_YG(x, y) - D_YG(x_0, y_0)\| \\
			& < \ep
		\end{align*}
		Set $A = B(x_0, \del_X)$ and $B = B(y_0, \del_Y)$. Let $x \in A$ and $y_1, y_2 \in B$. Define $l: [0,1] \rightarrow B$ by $l(t) = ty_1 + (1-t)y_2$. The mean value theorem implies that 
		\begin{align*}
			\|G(x, y_1) - G(x, y_2)\| 
			& \leq \sup_{t \in [0,1]} \|D_YG(x, l(t))\|\|y_1 - y_2\| \\
			& \leq \ep \|y_1 - y_2\| \\
			& = \frac{1}{2}\|y_1 - y_2\| 
		\end{align*} 
		Hence, for each $x \in X$ and $y \in Y$, 
		$\|G(x, y)\| \leq \frac{1}{2}\|y_1 - y_2\|$  
		For $x \in A$, define $T_x: B \rightarrow B$ by $T_x(y) = G(x,y)$. 
	\end{proof}


	
	
	
	
	
	
	
	
	
	
	
	
	
	
	
	
	
	
	
	
	
	
	
	
	\newpage
	\section{The Gradient}
	
	\begin{defn} \ld{}
	Let $H$ be a Hilbert space, $f: H \rightarrow \C$ and $x_0 \in H$. Suppose that $f$ is Frechet differentiable at $x_0$. Then $Df(x_0) \in H^*$. We define the \tbf{gradient of $f$ at $x_0$}, denoted $\nabla f(x_0) \in H$, via the Riesz representation theorem to be the unique element of $H$ satisfying $$\l \nabla f(x_0), y \r = Df(x_0)(y) \hspace{.3cm} \text{ for each } y \in H$$
	\end{defn}


	


	\begin{defn}
		Let $H \subset Y^X$ be a Hilbert space and $F: H \rightarrow \R$. We define the \tbf{functional derivative $F$} with respect to $\phi(x)$ of $f:X \rightarrow Y$, denoted $\frac{\del F}{\del \phi}: X \rightarrow \R$, by .  
		\href{https://en.wikipedia.org/wiki/Functional_derivative#Functional_derivative}{functional derivative} \tcr{relate this to rigged hilbert spaces and $\frac{\del F}{\del \phi(x_0)} = DF(\phi)(\del_{x_0})$}
	\end{defn}



















































\newpage
\chapter{Banach Algebras}

\section{Introduction}

\begin{defn} \ld{def:banach_alg:intro:0001}
	Let $X$ be a Banach space and $\mu: X \times X \rightarrow X$. Then $(X, \mu)$ is said to be a \tbf{Banach algebra} if
	\begin{enumerate}
		\item $(X, \mu)$ is an associative algebra
		\item $\mu \in L^2(X)$ and $\|\mu\| \leq 1$
	\end{enumerate}
\end{defn}

\begin{note}
	By definition in the section on multilinear maps, condition $(2)$ is equivalent to the assumption that for each $x,y \in X$, $\|xy\| \leq \|x\|\|y\|$.
\end{note}

\begin{defn} \ld{def:banach_alg:intro:0002}
	Let $X$ be a Banach algebra and $e \in X$. Then $e$ is said to be an \tbf{identity} if for each $x \in X$, $ex = xe = x$. 
\end{defn}


\begin{defn} \ld{def:banach_alg:intro:0003}
	Let $X$ be a Banach algebra. Then $X$ is said to be \tbf{unital} if there exists $e \in X$ such that $e$ is an identity.
\end{defn}	

\begin{ex} \lex{ex:banach_alg:intro:0004}
	Let $X$ be a unital Banach algebra. Then there exists a unique $e \in X$ such that $e$ is an identity.
\end{ex} 

\begin{proof}\
	\begin{itemize}
		\item \tbf{Existence: } \\
		By definition, there exists $e \in X$ such that $e$ is an identity.
		\item \tbf{Uniqueness: }\\
		Let $e' \in X$. Suppose that $e'$ is an identity. Then 
		\begin{align*}
			e' 
			& = e' e \\
			& = e'
		\end{align*}	
	\end{itemize}
\end{proof}

\begin{ex} \lex{ex:banach_alg:intro:0005}
	Let $X$ be a unital Banach algebra. If $X \neq \{0\}$, then $1 \leq \|e \|$. 
\end{ex}

\begin{proof}
	Suppose that $X \neq \{0\}$. Then $e \neq 0$ which implies that $\|e \| > 0$. Since 
	\begin{align*}
		\|e \|
		= \|e e \|
		& \leq \|e \|\|e \|
	\end{align*}
	we have that $1 \leq \|e \|$.
\end{proof}

\begin{ex} \lex{ex:banach_alg:intro:0006} \lex{}\tbf{Fundamental Example:} \\
	Let $X$ be a Banach space. Then $GL(X)$ is a unital Banach algebra.
\end{ex}

\begin{proof}
	Clear.
\end{proof}

\begin{defn} \ld{def:banach_alg:intro:0007}
	Let $X$ be a unital Banach algebra and $x,y \in X$. Then $y$ is said to be an 
	\tbf{inverse} of $x$ if $xy = yx = e$.
\end{defn}

\begin{defn} \ld{def:banach_alg:intro:0008}
	Let $X$ be a unital Banach algebra and $x \in X$. Then $y$ is said to be 
	\tbf{invertible} if there exists $y \in X$ such that $y$ is an inverse of $x$.
\end{defn}

\begin{ex} \lex{ex:banach_alg:intro:0009}
	Let $X$ be a unital Banach algebra and $x \in X$. If $x$ is invertible, then there exists a unique $y \in X$ such that $y$ is an inverse of $x$.
\end{ex}

\begin{proof}
	Suppose that $x$ is invertible.
	\begin{itemize}
		\item \tbf{Existence: } \\
		By definition, there exists $y \in X$ such that $y$ is an inverse of $x$.
		\item \tbf{Uniqueness: }\\
		Let $y' \in X$. Suppose that $y'$ is an inverse of $x$. Then 
		\begin{align*}
			y'
			& = y'e \\
			& = y'(xy) \\
			& = (y'x)y \\
			& = ey \\
			& = y
		\end{align*}
	\end{itemize}
\end{proof}

\begin{defn} \ld{def:banach_alg:intro:0010}
	Let $X$ be a unital Banach algebra. We define $G(X) = \{x \in X: x \text{ is invertible}\}$.
\end{defn}

\begin{ex} \lex{ex:banach_alg:intro:0011}
	Let $X$ be a unital Banach algebra. Then $G(X)$ is a group.  
\end{ex}

\begin{proof}
	Clear.
\end{proof}

\begin{defn} \ld{def:banach_alg:intro:0012}
	Let $X$ be a unital Banach algebra and $x \in G(X)$. We define the \textbf{inverse of $x$}, denoted $x^{-1}$, to be the unique $y \in X$ such that $yx = xy = e$.
\end{defn}

\begin{ex} \lex{ex:banach_alg:intro:0013}
	Let $X$ be a unital Banach algebra, $x \in G(X)$ and $\lam \in \C^{\times}$. Then $\lam x \in G(X)$ is and $(\lam x)^{-1} = \lam^{-1}x^{-1}$.
\end{ex}

\begin{proof}
	We have that 
	\begin{align*}
		(\lam^{-1}x^{-1})( \lam x)
		& = ((\lam^{-1} \lam )x^{-1}) x \\
		& = (1x^{-1}) x \\
		& = x^{-1}x \\
		& = e \\
	\end{align*}
	Similarly, $( \lam x)(\lam^{-1}x^{-1}) = e$. Hence $\lam x \in G(X)$ and $(\lam x)^{-1} = \lam^{-1}x^{-1}$.
\end{proof}

\begin{ex} \lex{ex:banach_alg:intro:0014}
	Let $X$ be a unital Banach algebra and $x,y \in G(X)$. Then $xy \in G(X)$ is and $(xy)^{-1} = y^{-1}x^{-1}$.
\end{ex}

\begin{proof}
	We have that 
	\begin{align*}
		(y^{-1}x^{-1})(xy)
		& = y^{-1}(x^{-1} (x y)) \\
		& = y^{-1}((x^{-1} x) y) \\
		& = y^{-1} (e y) \\
		& = y^{-1} y \\
		& = e
	\end{align*}
	Similarly, $(xy)(y^{-1}x^{-1}) = e$. Hence $xy \in G(X)$ and $(xy)^{-1} = y^{-1}x^{-1}$.
\end{proof}

\begin{ex} \lex{ex:banach_alg:intro:0015}
	Let $X$ be a unital Banach algebra and $x,y \in X$. 
	\begin{enumerate}
		\item If $xy \in G(X)$ and $y \in G(X)$, then $x \in G(X)$.
		\item If $xy\in G(X)$ and $yx \in G(X)$, then $x \in G(X)$ and $y \in G(X)$.
		\item If $xy = yx$ and $x \not \in G(X)$, then $xy \not \in G(X)$.
	\end{enumerate}
\end{ex}

\begin{proof}\
	\begin{enumerate}
	\item Suppose that $xy \in G(X)$ and $y \in G(X)$. Since $xy \in G(X)$, there exists $z \in G(X)$ such that $z(xy) = (xy)z = e$. Since $z, y \in G(X)$, we have that $yz \in G(X)$ and and $(yz)^{-1} = z^{-1}y^{-1}$. Therefore
	\begin{align*}
		z(xy) = e
		& \implies xy = z^{-1} \\
		& \implies x = z^{-1} y^{-1} \\
		& \implies x = (yz)^{-1}
	\end{align*} 
	Hence $x \in G(X)$.
	\item Suppose that $xy, yx \in G(X)$. Then there exists $z \in G(X)$ such that $z(xy) = (xy)z = e$. Then $x(yz) = e$ and since $yx \in G(X)$, we have that
	\begin{align*}
		z(xy) = e
		& \implies (zx)y = e \\
		& \implies (zx)yx = x \\
		& \implies zx = x(yx)^{-1} \\
		& \implies y(zx) = y(x(yx)^{-1}) \\
		& \implies (yz)x = (yx)(yx)^{-1} \\
		& \implies (yz)x = e \\
	\end{align*}
	Since $(yz)x = x(yz) = e$, we have that $x \in G(X)$. Similarly, $y \in G(X)$.
	\item Suppose that $xy = yx$ and $x \not \in G(X)$. Part $(2)$ implies that $xy \not \in G(X)$ or $yx \not \in G(X)$. Since $xy = yx$, we have that $xy \not \in G(X)$.
	\end{enumerate}
\end{proof}

\begin{ex} \lex{ex:banach_alg:intro:0016}
	Let $X$ be a unital Banach algebra. 
	\begin{enumerate}
		\item For each $x \in X$, if $\|x \|< 1$, then $e - x \in G(X)$ and $$(e-x)^{-1} = \sum_{n=0}^{\infty}x^n$$
		\item For each $x \in X$ and $\lam \in \C$, if $\|x\| < |\lam |$, then $\lam e - x \in G(X)$ and 
		$$(\lam e - x)^{-1} = \sum_{n=0}^{\infty} \lam^{-(n+1)} x^n $$
		\item For each $x,y \in X$, if $x \in G(X)$  and $\|y \|< \|x^{-1} \|^{-1}$, then $x - y \in G(X)$ and $$(x-y)^{-1} = x^{-1}\sum_{n=0}^{\infty} (yx^{-1})^n$$
		\item For each $x,y \in X$, if $x \in G(X)$  and $\| x - y \|< \|x^{-1} \|^{-1}$, then $y \in G(X)$ and $$y^{-1} = x^{-1}\sum_{n=0}^{\infty} (e - yx^{-1})^n$$
		\item $G(X)$ is open
	\end{enumerate}
\end{ex}

\begin{proof}\
	\begin{enumerate}
		\item Let $x \in X$. Suppose that $\|x \|< 1$. Then $$\sum_{n=0}^{\infty} \|x^n \| \leq \sum_{n=0}^{\infty} \|x \|^{n} < \infty$$ Since $X$ is a complete, $\sum\limits_{n=0}^{\infty}x^n$ converges in $X$.\\
		Define $(s_k)_{k=0}^{\infty} \subset X$ and $s \in X$ by $s_k = \sum\limits_{n=0}^{k} x^n$ and  $s = \sum\limits_{n=0}^{\infty}x^n$. Then for each $k \in \N$,
		\begin{align*}
			(e - x) s_k
			& = s_k - x s_k \\
			&= e - x^{k+1} 
		\end{align*}
		Since $x^k \rightarrow 0$ as $k \rightarrow \infty$,  we have that $(e- x)s_k \rightarrow e$ as $k \rightarrow \infty$. Since multiplication on Banach algebras is continuous, we have that $(e - x)s_k \rightarrow (e-x)s$ as $k \rightarrow \infty$. Uniqueness of limits implies that $(e-x)s = e$. A similar argument implies that $s(e-x) = e$. Thus $e - x \in G(X)$ and $(e-x)^{-1} = \sum\limits_{n=0}^{\infty}x^n$. \vspace{.5cm}\\
		\item Let $x \in X$ and $\lam \in \C$. Suppose that $\|x\| < |\lam |$. Then $\lam \neq 0$ and
		\begin{align*}
			\|\lam^{-1} x\| \\
			& = |\lam^{-1}| \|x \| \\
			& = |\lam |^{-1} \|x\| \\
			& < |\lam |^{-1} |\lam | \\
			& = 1
		\end{align*}
		By $(1)$, we have that $e - \lam^{-1}x \in G(X)$ and 
		\begin{align*}
			(e - \lam^{-1}x)^{-1} 
			& = \sum_{n=0}^{\infty} (\lam^{-1}x)^n \\
			& = \sum_{n=0}^{\infty} \lam^{-n} x^n 
		\end{align*}
		Therefore, 
		\begin{align*}
			\lam e - x
			& = \lam (e - \lam^{-1}x) \\
			& \in G(X)
		\end{align*}
		and 
		\begin{align*}
			(\lam e - x)^{-1}
			& = (\lam (e - \lam^{-1}x))^{-1} \\
			& = \lam^{-1}(e - \lam^{-1}x)^{-1} \\
			& = \lam^{-1} \sum_{n=0}^{\infty} \lam^{-n} x^n  \\
			& = \sum_{n=0}^{\infty} \lam^{-(n+1)} x^n 
		\end{align*}
		\item  Let $x, y \in X$. Suppose that $x \in G(X)$ and $\|y \|< \|x^{-1} \|^{-1}$. Then 
		\begin{align*}
			\|yx^{-1} \|
			& \leq \|y\| \|x^{-1} \| \\
			& <  \|x^{-1} \|^{-1} \|x^{-1} \| \\
			& = 1
		\end{align*}
		Hence $e - yx^{-1} \in G(X)$ and
		$$(e - yx^{-1}) = \sum_{n=0}^{\infty} (y x^{-1})^n$$
		This implies that 
		\begin{align*}
			x - y
			& = (e - yx^{-1}) x \\
			& \in G(X)
		\end{align*}
		and 
		\begin{align*}
			(x - y)^{-1}
			& = ((e - yx^{-1}) x)^{-1} \\
			& = x^{-1} (e - yx^{-1})^{-1} \\
			& = x^{-1} \sum_{n=0}^{\infty} (y x^{-1})^n
		\end{align*}
		\vspace{.5cm}\\
		\item Let $x, y \in X$. Suppose that $x \in G(X)$ and $\|x-y \|< \|x^{-1} \|^{-1}$. Then $(2)$ implies that 
		\begin{align*}
			y 
			& = x - (x -y) \\
			& \in G(X)
		\end{align*}
		and 
		\begin{align*}
			y^{-1} 
			& = (x - (x -y))^{-1} \\
			& = x^{-1} \sum_{n=0}^{\infty} ((x - y) x^{-1})^n \\
			& = x^{-1} \sum_{n=0}^{\infty} (e - y x^{-1})^n
		\end{align*}
		\item Let $x \in G(X)$. Choose $\del = \|x^{-1}\|^{-1}$. By $(3)$, $B(x, \del) \subset G(X)$. Since $x \in G(X)$ is arbitrary, $G(X)$ is open. 
	\end{enumerate}
\end{proof}	

\begin{defn} \ld{def:banach_alg:intro:0017}
	Let $X$ be a unital Banach algebra. We define $\inv: G(X) \rightarrow G(X)$ by $\inv(x) = x^{-1}$. 
\end{defn}

\begin{ex} \lex{ex:banach_alg:intro:0018}
	Let $X$ be a unital Banach algebra. Then
	\begin{enumerate}
		\item for each $x,y \in X$, if $x \in G(X)$ and $\|y\| \leq \frac{1}{2} \|x^{-1}\|^{-1}$ so that $x - y \in G(X)$, then  
		$$\|(x-y)^{-1} - x^{-1}\| \leq 2 \|x^{-1}\|^2 \|y\|$$
		\item $\inv: G(X) \rightarrow G(X)$ is continuous
		\item $G(X)$ is a topological group
	\end{enumerate}
\end{ex}

\begin{proof}\
	\begin{enumerate}
		\item Let $x,y \in X$. Suppose that $x \in G(X)$ and $\|y\| \leq 2^{-1} \|x^{-1}\|^{-1}$. The previous exercise implies that 
		\begin{align*}
			\|(x - y)^{-1} - x^{-1}\|
			& = \bigg \| x^{-1}\sum_{n=0}^{\infty}(yx^{-1})^n  - x^{-1}\bigg\| \\
			& = \bigg \| x^{-1}\sum_{n=1}^{\infty}(yx^{-1})^n  \bigg\| \\
			& \leq \|x^{-1}\| \sum_{n=1}^{\infty} (\|y\|\|x^{-1}\|)^n \\
			& = \|x^{-1}\|^{2} \|y\| \sum_{n=0}^{\infty} (\|y\|\|x^{-1}\|)^n \\ 
			& = \|x^{-1}\|^{2} \|y\| \sum_{n=0}^{\infty} \frac{1}{2^n} \\
			& = 2 \|x^{-1}\|^{2} \|y\| 
		\end{align*}
		\item Let $(x_n)_{n \in \N} \subset G(X)$ and $x \in G(X)$. Suppose that $x_n \rightarrow x$ in $G(X)$. Then $x_n \rightarrow x$ in $X$. Define $(y_n)_{n \in \N} \subset X$ by $y_n = x - x_n$. Then $y_n \rightarrow 0$ in $X$. Let $\ep > 0$. Then there exists $N \in \N$ such that for each $n \in \N$, $n \geq N$ implies that 
		$$\|y_n \| < \max \bigg(\frac{\ep}{2\|x^{-1}\|^2}, \frac{1}{2} \|x^{-1}\|^{-1} \bigg)$$
		Let $n \in \N$. Suppose that $n \geq N$. By $(1)$, we have that 
		\begin{align*}
			\|\inv(x_n) - \inv(x)\|
			& = \|x_n^{-1} - x^{-1}\| \\
			& = \|(x - y_n)^{-1} - x^{-1}\| \\
			& \leq 2\|x^{-1}\|^2\|y_n\| \\
			& < \ep 
		\end{align*} 
		Hence $\inv(x_n) \rightarrow \inv(x)$ in $X$. Thus $\inv(x_n) \rightarrow \inv(x)$ in $G(X)$. So $\inv: G(X) \rightarrow G(X)$ is continuous. 
		\item Since multiplication $G(X) \times G(X) \rightarrow G(X)$ and multiplicative inversion $\inv: G(X) \rightarrow G(X)$ are continuous, $G(X)$ is a topological group. 
	\end{enumerate}
\end{proof}

\begin{ex} \lex{ex:banach_alg:intro:0019}
	Let $X$ be a unital Banach algebra. Then $\inv: G(X) \rightarrow G(X)$ is Frechet differentiable and for each $x \in G(X)$, $h \in X$,  $D \inv(x)(h) = x^{-1}hx^{-1}$.
\end{ex}

\begin{proof}
	Let $x \in G(X)$ and $h \in B(x, \|x^{-1}\|^{-1})$. A previous exercise implies that $x + h \in G(X)$ and 
	\begin{align*}
		\inv(x+h) 
		& = (x+h)^{-1} \\
		& = x^{1-} \sum_{n=0}^{\infty} [(-h) x^{-1}]^n \\
		& = x^{-1} - x^{-1}hx^{-1} + x^{-1}\sum_{n=2}^{\infty} [(-h) x^{-1}]^n \\
		& = x^{-1} - x^{-1}hx^{-1} + o(\|h\|) \text{ as $h \rightarrow 0$}
	\end{align*}
	Since the map $X \rightarrow X$ given by $h \mapsto - x^{-1}hx^{-1}$ is a bounded linear operator and  $x^{-1}\sum\limits_{n=2}^{\infty} [(-h) x^{-1}]^n = o(\|h\|) \text{ as $h \rightarrow 0$}$, we have that $\inv$ is differentiable at $x$ and for each $h \in X$, $D\inv(x)(h) = - x^{-1}hx^{-1}$. Since $x \in G(X)$ is arbitrary, we have that $\inv(x)$ is differentiable. 
\end{proof}


\tcb{do all the other derivatives like power rule, product rule, etc}
\begin{ex}
	
\end{ex}

\begin{proof}
	content...
\end{proof}

\begin{ex}
	
\end{ex}

\begin{proof}
	content...
\end{proof}


\begin{ex}
	
\end{ex}

\begin{proof}
	content...
\end{proof}










































\newpage
\section{Spectral Theory}

\begin{defn} \ld{def:banach_alg:spectral:0001}
	Let $X$ be a unital Banach algebra and $x \in X$. We define the
	\begin{itemize}
		\item \textbf{resolvent} of $x$, denoted $\rho(x)$, by 
		$$\rho(x) = \{\lam \in \C: \lam e - x \in G(X)\}$$
		\item \textbf{spectrum} of $x$, denoted $\sig(x)$, by 
		$$\sig(x) = \rho(x)^c$$ 
	\end{itemize}
\end{defn}

\begin{ex} \lex{ex:banach_alg:spectral:0002}
	Let $X$ be a unital Banach algebra and $x \in X$. Then 
	\begin{enumerate}
		\item $\rho(x)$ is open
		\item $\sig(x)$ is closed
		\item $\sig(x) \subset \bar{B}(0, \|x\|)$
		\item  $\sig(x)$ is compact
	\end{enumerate} 
\end{ex}

\begin{proof}\
	\begin{enumerate}
		\item Let $\lam \in \rho(x)$. Set $\del = (\|(\lam e - x)^{-1}\| \|e\| )^{-1} > 0$. Let $\lam' \in B(\lam, \del)$. Then 
		\begin{align*}
			\|(\lam e - x) - (\lam' e - x)\| 
			& = |\lam - \lam '|\|e\| \\
			& <  \del \|e\| \\
			& = \|(\lam e - x)^{-1}\|^{-1} \\
		\end{align*}
		A previous exercise implies that $\lam' e - x \in G(X)$. Hence $\lam' \in \rho(x)$. Since $\lam' \in B(\lam, \del)$ is arbitrary, we have that $B(\lam, \del) \subset \rho(x)$. So for each $\lam \in \rho(x)$, there exists $\del >0$ such that $B(\lam ,\del) \subset \rho(x)$. Hence $\rho(x)$ is open.
		\item Since $\sig(x) = \rho(x)^c$ and $\rho(x)$ is open, we have that $\sig(x)$ is closed.
		\item Let $\lam \in \sig(x)$. For the sake of contradiction, suppose that $\|x\| < |\lam|$. A previous exercise implies that $\lam e - x \in G(X)$. Hence 
		\begin{align*}
			\lam 
			& \in \rho(x) \\
			& = \sig(x)^c
		\end{align*}
		which is a contradiction. So $|\lam| \leq \|x\|$ and thus $\lam \in \bar{B}(0, \|x\|)$. Since $\lam \in \sig(x)$ is arbitrary, $\sig(x) \subset \bar{B}(0, \|x\|)$.
		\item Since $\sig(x) \subset \C$ is closed and bounded, $\sig(x)$ is compact.
	\end{enumerate} 
\end{proof}

\begin{ex} \lex{ex:banach_alg:spectral:0003}
	Let $X$ be a unital Banach algebra and $x \in X$ and $p \in \C[t]$. Suppose that $\deg p \geq 1$. Then $\sig(p(x)) = \{p(\lam): \lam \in \sig(x)\}$. \\
	\tbf{Hint:} Consider the roots of $p(x) - p(\lam)e$. 
\end{ex}

\begin{proof}
	Let $\lam \in \sig(x)$. Then $\lam e$ is a root of $p(t) - p(\lam) e$. Therefore, there exists $q \in \C[t]$ such that $\deg q = \deg p -1$ and $p(x) - p(\lam) e = (x - \lam e)q(x)$. Since $q(x) (x - \lam e) = (x - \lam e) q(x)$ and $(x - \lam e) \not \in G(X)$, a previous exercise implies that $p(x) - p(\lam) e \not \in G(X)$. Thus $p(\lam) \in \sig(p(x))$. Since $\lam \in \sig(x)$ is arbitrary, we have that $\{p(\lam): \lam \in \sig(x)\} \subset \sig(p(x))$. \\
	Conversely, let $\mu \in \sig(p(x))$. Set $n = \deg p$. Then there exist $(a_j)_{j=1}^n \subset \C$ and $a \in \C$ such that 
	$$p(x) - \mu e = a \prod_{j=1}^{n} (x - a_j e)$$ 
	Since $p(x) - \mu e \not \in G(X)$, there exists $j \in \{1, \ldots, n\}$ such that $(x - a_je) \not \in G(X)$. Thus $a_j \in \sig(x)$. By construction 
	\begin{align*}
		(p(a_j) - \mu )e 
		& = p(a_j)e - \mu e \\
		& = p(a_j e) - \mu e \\
		& = 0
	\end{align*}
	Thus 
	\begin{align*}
		\mu 
		& = p(a_j) \\
		& \in \{p(\lam): \lam \in \sig(x)\}
	\end{align*}
	Since $\mu \in \sig(p(x))$ is arbitrary, we have that $\sig(p(x)) \subset \{p(\lam): \lam \in \sig(x)\}$. Hence $\sig(p(x)) = \{p(\lam): \lam \in \sig(x)\}$.
\end{proof}

\begin{defn} \ld{def:banach_alg:spectral:0004}
	Let $X$ be a unital Banach algebra and $x \in X$. We define the \textbf{resolvent function} of $x$, denoted $R_x: \rho(x) \rightarrow G(X)$, by  
	$$R_x(\lam) = (\lam e - x)^{-1}$$
\end{defn}

\begin{ex} \lex{ex:banach_alg:spectral:0005}
	Let $X$ be a unital Banach algebra and $x \in X$. Then 
	\begin{enumerate}
		\item $R_x: \rho(x) \rightarrow G(X)$ is Frechet differentiable and for each $\lam \in \rho(x)$, 
		$$R_x' = -R_x^2$$
		\item $R_x \in C^{\infty}(\rho(x))$ and for each $n \in \N$, $R_x^{(n)} = (-1)^n n! R_x^{n+1}$
	\end{enumerate}
\end{ex}

\begin{proof}\
	\begin{enumerate}
		\item Define $S_x: \rho(x) \rightarrow G(X)$ by $S_x(\lam) = \lam e - x$. Then $R_x = \inv \circ S_x$. Since $S_x$ and $\inv$ are differentiable, $R_x = \inv \circ S_x$ is differentiable. Previous exercises imply that for each $\lam \in \rho(x)$, we have that
		\begin{align*}
			R_x'(\lam)
			& = DR_x(\lam)(1) \\
			& = [D \inv(S_x(\lam)) \circ DS_x(\lam)](1) \\
			& = D \inv(S_x(\lam)) (DS_x(\lam)(1)) \\
			& = D \inv(S_x(\lam)) (e) \\
			& = -S_x(\lam)^{-1} e S_x(\lam)^{-1} \\
			& = -S_x(\lam)^{-2} \\
			& = -R_x(\lam)^2
		\end{align*}
		\item Let $n \in \N$. Suppose that $R_x \in C^{n-1}(\rho(x))$ and $R_x^{(n-1)} =  (-1)^{n-1} (n-1)! R_x^{n}$. Then 
		\begin{align*}
			R_x^{(n)}
			& = (R_x^{(n-1)})' \\
			& = [(-1)^{n-1}(n-1)! R_x^{n}]' \\
			& = (-1)^{n-1}(n-1)! (n R_x^{n-1}) (-R_x^2) \\
			& = (-1)^n n! R_x^{n+1}  
		\end{align*}
		By induction, for each $n \in \N$, $R_x \in C^{n}(\rho(x))$ and $R_x^{(n)} = (-1)^n n! R_x^{n+1}$. A previous exercise in the section of differentiability implies that for each $n \in \N$, $R_x \in C^n(\rho(x))$ iff $R_x^{(n)} \in C^0(\rho(x))$. Hence for each $n \in \N$, $R_x \in C^n(\rho(x))$ and therefore $R_x \in C^{\infty}(\rho(x))$.
	\end{enumerate}
\end{proof}



\begin{ex} \lex{ex:banach_alg:spectral:0006}
	Let $X$ be a unital Banach algebra and $x \in X$. Then $\sig(x) \neq \varnothing$. \\
	\tbf{Hint:} $R_x$ is bounded and apply Louiville's theorem
\end{ex}

\begin{proof}
	Suppose that $\sig(x) = \varnothing$. Then $\rho(x) = \C$ and the previous exercise implies that $R_x: \C \rightarrow G(X)$ is differentiable. We observe that for each $\lam \in \C^{\times}$, 
	\begin{align*}
		R_x(\lam)
		& = (\lam e - x)^{-1} \\
		& = \lam^{-1} (e - \lam^{-1}x)^{-1} \\
	\end{align*} 
	Since $\lam^{-1} \rightarrow 0$ as $\lam \rightarrow \infty$ and $\inv: G(X) \rightarrow G(X)$ is continuous, we have that
	\begin{align*}
		(e - \lam^{-1}x)^{-1}
		& \rightarrow e^{-1} \\
		&= e
	\end{align*}
	as $\lam \rightarrow \infty$. Hence $R_x(\lam) \rightarrow 0$ as $\lam \rightarrow \infty$. Thus $R_x: \C \rightarrow G(X)$ is bounded. Louiville's theorem implies that $R_x = 0$. This is a contradiction since $0 \not \in G(X)$.   
\end{proof}

\begin{ex} \lex{ex:banach_alg:spectral:0007} \tbf{Gelfand-Mazur Theorem:} \\
	Let $X$ be a unital Banach algebra. Suppose that $G(X) = X \setminus \{0\}$. Then $X = \C e$. 
\end{ex}

\begin{proof}
	For the sake of contradiction, suppose that $X \neq \C e$. Then there exists $x \in X$ such that for each $\lam \in \C$, $\lam e - x \neq 0$. Let $\lam \in \C$. Since $G(X) = X \setminus \{0\}$, $\lam e - x \in G(X)$ and $\lam \in \rho(x)$. Since $\lam \in \C$ is arbitrary, we have that $\rho(x) = \C$. Therefore $\sig(x) = \varnothing$. This is a contradiction since \rex{} \tcr{the previous exercise} implies that $\sig(x) \neq \varnothing$. Thus $X = \C e$. 
\end{proof}

\begin{note}
	In particular, if $X \setminus \{0\} = G(X)$, then $X \cong \C$.
\end{note}

	\begin{defn} \ld{def:banach_alg:spectral:0008}
		Let $X$ be a unital Banach algebra and $x \in X$. We define the \tbf{spectral radius of $x$}, denoted by $r(x)$, by $$r(x) = \sup \{|\lam|: \lam \in \sig(x)\}$$
	\end{defn}

	\begin{ex} \lex{ex:banach_alg:spectral:0009}
		Let $X$ be a unital Banach algebra and $x \in X$. Then
		\begin{enumerate}
			\item $r(x) \leq \liminf \|x^n\|^{1/n}$
			\item for each $n \in \N$, $\lam \in \bar{B}(0, r(x))^c$ and $\phi \in X^*$, $|\phi((\lam x)^n)| < \infty$ and $\phi \in X^*$, $\phi \circ R_x(\lam) = \sum_{n=0}^{\infty} \lam^{-(n+1)} \phi(x)^n$ is
			\tbf{Hint:} uniform boundedness principle  
		\end{enumerate}
	\end{ex}

	\begin{proof}\
		\begin{enumerate}
			\item Let $\lam \in \sig(x)$ and $n \in \N$. The previous exercise implies that $\lam^n \in \sig(x^n)$. Since $\lam^ne - x^n \not \in G(X)$, we have that
			\begin{align*}
				|\lam|^n
				& = |\lam^n| \\
				& \leq \|x^n\|
			\end{align*}
			Therefore $|\lam| \leq \|x^n\|^{1/n}$. Since $n \in \N$ is arbitrary, $|\lam| \leq \liminf \|x^n\|^{1/n}$. Since $\lam \in \sig(x)$ is arbitrary, we have that 
			\begin{align*}
				r(x)
				& = \sup_{\lam \in \sig(x)} |\lam| \\
				& \leq \liminf \|x^n\|^{1/n}
			\end{align*} 
			\item Let $\lam \in \C$. Suppose that $|\lam| > r(x)$. Then $\lam \in \rho(x)$. \tcr{A previous exercise} implies that $|\lam| > \|x\|$  Let $\phi \in X^*$. Then
			\begin{align*}
				\phi \circ R_x(\lam) 
				& = \phi \bigg( \sum\limits_{n=0}^{\infty} \lam^{-(n+1)} x^n \bigg) \\
				& = \lam^{-1} \sum\limits_{n=0}^{\infty} \phi((\lam^{-1} x)^n).
			\end{align*}
			Hence 
			\begin{align*}
				\|\phi \circ R_x(\lam)\|
				& = \sum\limits_{n=0}^{\infty} \|\phi\| |\lam|^{-(n+1)}  \|x\|^n \\
				& < \infty 
			\end{align*}
			and therefore $|\phi \circ R_x(\lam)| \leq \sum\limits_{n=0}^{\infty} r(x)^{-(n+1)}\|\phi\|^n\|x\|^n$. Thus for each $n \in \N$ and $\lam \in \bar{B}(0, r(x))^c$, \tcr{FINISH!!!, need to use holomorphicity, make section on complex analysis and also for Banach algebra valued functions, show that for $D \subset \C$ open, $f:D \rightarrow X $ is holomorphic iff for each $\phi \in X^*$, $\phi \circ f$ is holomorphic.}
			
		\end{enumerate}
	\end{proof}
	
	\begin{ex} \lex{ex:banach_alg:spectral:0010}
		Let $X$ be a unital Banach algebra and $x \in X$. Then 
		$$r(x) = \lim_{n \rightarrow \infty} \|x^n\|^{1/n}$$
	\end{ex}

	\begin{defn} \ld{def:banach_alg:spectral:0011}
		Let $X, Y$ be Banach algebras and $\phi: X \rightarrow Y$. Then $\phi$ is said to be a \tbf{Banach algebra homomorphism} if 
		\begin{enumerate}
			\item $\phi \in L(X, Y)$,
			\item for each $a, b \in X$, $\phi(ab) = \phi(a) \phi(b)$.
		\end{enumerate}
	\end{defn}

	\begin{ex} \lex{ex:banach_alg:spectral:0012}
		Let $X, Y$ be Banach algebras and $\phi: X \rightarrow Y$. Suppose that $X$ is unital and $\phi$ is a Banach algebra homomorphism. Then 
		\begin{enumerate}
			\item $\phi(e) = 0$ iff $\phi = 0$,
			\item $Y = \C$ implies that $\phi(e) \neq 0$ iff $\phi(e) = 1$.
		\end{enumerate}
	\end{ex}

	\begin{proof}\
		\begin{enumerate}
			\item 
			\begin{itemize}
				\item $(\implies):$ \\
				Suppose that $\phi(e) = 0$. Then for each $x \in X$, 
				\begin{align*}
					\phi(x)
					& = \phi(x e) \\
					& = \phi(x) \phi(e) \\
					& = \phi(x) 0 \\
					& = 0.
				\end{align*}
				Thus $\phi = 0$
				\item $(\impliedby):$ \\ 
				Suppose that $\phi = 0$. Then $\phi(e) = 0$.
			\end{itemize}
			\item Suppose that $Y = \C$.
			\begin{itemize}
				\item $(\implies):$ \\
				Suppose that $\phi(e) \neq 0$. Then 
				\begin{align*}
					\phi(e)
					& = \phi(e e) \\
					& = \phi(e) \phi(e).
				\end{align*}
				Thus $\phi(e)(\phi(e) - 1) = 0$. Since $\phi(e) \neq 0$, $\phi(e) = 1$.
				\item $(\impliedby):$ \\ 
				Suppose that $\phi(e) = 1$. Then clearly $\phi(e) \neq 0$.
			\end{itemize}
		\end{enumerate}
	\end{proof}

	\begin{defn} \ld{def:banach_alg:spectral:0013}
		\tcr{Category of Banach algebras}
		We define the \tbf{category of Banach algebras}, denoted $\BanAlg$
	\end{defn}

	\begin{defn} \ld{def:banach_alg:spectral:0014}
		Let $X$ be a unital Banach algebra. We define the \tbf{spectrum of $X$}, denoted $\sig(X)$, by $\sig(X) \defeq \Hom_{\BanAlg}(X, \C) \setminus \{0\}$.
	\end{defn}

	\begin{ex} \lex{ex:banach_alg:spectral:0015}
		Let $X$ be a unital Banach algebra and $\phi \in \sig(X)$. Then 
		\begin{enumerate}
			\item $\phi(e) \neq 0$,
			\item $\phi(e) = 1$,
			\item for each $x \in G(X)$, $\phi(x) \neq 0$,
			\item $\|\phi\| \leq 1$.
		\end{enumerate}
	\end{ex}

	\begin{proof}\
		\begin{enumerate}
			\item Since $\phi \in \sig(X)$, $\phi \neq 0$. \tcr{The previous exercise} implies $\phi(e) \neq 0$.
			\item \tcr{the previous exercise} implies that $\phi(e) = 1$ iff $\phi(e) \neq 0$. Since the previous part implies that $\phi(e) \neq 0$, we have that $\phi(e) = 1$. 
			\item Let $x \in G(X)$. Then 
			\begin{align*}
				\phi(x) \phi(x^{-1}) 
				& = \phi(x x^{-1}) \\
				& = \phi(e) \\
				& = 1 \\
			\end{align*}
			and similarly, $\phi(x^{-1}) \phi(x) = 1$. Hence $\phi(x) \neq 0$ and $\phi(x)^{-1} = \phi(x^{-1})$.
			$\phi(x) \neq 0$.
			\item Let $x \in X$. If $\phi(x) = 0$, then 
			\begin{align*}
				|\phi(x)| 
				& = |0| \\
				& = 0 \\
 				& \leq \|x\|.
			\end{align*}
			Suppose that $\phi(x) \neq 0$. For the sake of contradiction, suppose that $|\phi(x)| > \|x\|$. Define $\lam \in \C$ by $\lam \defeq \phi(x)$. Since $|\lam| > \|x\|$, \tcr{A previous exercise} then implies that $\lam e - x \in G(X)$. \tcr{The previous part} implies that 
			\begin{align*}
				\phi(x) - \phi(x) 
				& = \lam - \phi(x) \\
				& = \lam \phi(e) - \phi(x) \\
				& = \phi(\lam e - x) \\
				& \neq 0. 
			\end{align*}
			This is a contradiction. Hence $|\phi(x)| \leq \|x\|$. Since $x \in X$ is arbitrary, we have that for each $x \in X$, $|\phi(x)| \leq \|x\|$. Thus $\|\phi\| \leq 1$.
		\end{enumerate}
	\end{proof}

	\begin{ex} \lex{ex:banach_alg:spectral:0016}
		Let $X$ be a unital Banach algebra. Then $\sig(X)$ is weak-* compact.
	\end{ex}

	\begin{proof}
		The Banach-Alaoglu theorem \rex{arg1} \tcr{(reference here)} implies that $\bar{B}_{X^*}(0,1)$ is weak-$*$ compact. \rex{} \tcr{the previous exercise} implies that $\sig(X) \subset \bar{B}_{X^*}(0,1)$. Let $(\phi_{\al})_{\al \in A} \subset \sig(X)$ and $\phi \in \bar{B}_{X^*}(0,1)$. Suppose that $\phi_{\al} \conv{w^*} \phi$. Then 
		\begin{itemize}
			\item Let $x, y \in X$. Since multiplication $\mu:X \times X \rightarrow X$ is norm continuous, $\mu$ is weak-$*$ continuous. Thus
			\begin{align*}
				\phi(xy)
				& = \lim_{\al \in A} \phi_{\al}(xy) \\
				& = \lim_{\al \in A} [\phi_{\al}(x) \phi_{\al}(y)] \\
				& = [\lim_{\al \in A} \phi_{\al}(x)] [\lim_{\al \in A} \phi_{\al}(y)] \\
				& = \phi(x) \phi(y).
			\end{align*} 
			So $\phi \in \Hom_{\BanAlg}(X, \C)$.
			\item Since for each $\al \in A$, $\phi_{\al} \in \sig(X)$, \tcr{a previous exercise} \rex{arg1} implies that for each $\al \in A$, $\phi_{\al}(e) = 1$. Thus 
			\begin{align*}
				\phi(e) 
				& = \lim_{\al \in A} \phi_{\al}(e) \\
				& = 1. 
			\end{align*}
			Hence $\phi \neq 0$. 
		\end{itemize}
		Thus $\phi \in \sig(X)$. Since $(\phi_{\al})_{\al \in A} \subset \sig(X)$ and $\phi \in \bar{B}_{X^*}(0,1)$ with $\phi_{\al} \conv{w^*} \phi$  are arbitrary, we have that for each $(\phi_{\al})_{\al \in A} \subset \sig(X)$ and $\phi \in \bar{B}_{X^*}(0,1)$, $\phi_{\al} \conv{w^*} \phi$ implies that $\phi \in \sig(X)$. Hence $\sig(X)$ is weak-* closed in $\bar{B}_{X^*}(0,1)$. Thus $\sig(X)$ is weak-* compact.
	\end{proof}

	\begin{defn} \lex{ex:banach_alg:spectral:0017}
		Let $X$ be a Banach algebra and $J \subset X$ a subalgebra. Then $J$ is said to be a 
		\begin{itemize}
			\item \tbf{left ideal of $X$} if for each $x \in X$, $y \in J$, $xy \in J$,
			\item \tbf{right ideal of $X$} if for each $x \in J$, $y \in X$, $xy \in J$,
			\item \tbf{ideal of $X$} if $J$ is a left ideal of $X$ and $J$ is a right ideal of $X$.
		\end{itemize}
	\end{defn}

	\begin{defn}
		Let $X$ be a Banach algebra. We define $\MI(X) \defeq \{\text{$J \subset X: J$ is an ideal of $X$} \}$
	\end{defn}

	\begin{ex} \lex{ex:banach_alg:spectral:0018}
		Let $X$ be a Banach algebra and $J \subset X$ an ideal of $X$. Then $\cl J$ is a ideal of $X$.
	\end{ex}

	\begin{proof}
		Let $x \in X$ and $y \in \cl J$. Then there exists $(y_n)_{n \in \N} \subset J$ such that $y_n \rightarrow y$. Since multiplication $\mu: X \times X \rightarrow X$ is continuous, we have that $x y_n \rightarrow xy$.  Since $J$ is a left ideal of $X$, we have that $(xy_n)_{n \in \N} \subset J$. Since there exists $(z_n)_{n \in \N} \subset J$ such that $z_n \rightarrow xy$, we have that $xy \in \cl J$. Since $x \in X$ and $y \in \cl J$ are arbitrary, we have that for each $x \in X$ and $y \in \cl J$, $xy \in \cl J$. So $\cl J$ is a left ideal of $X$. Similarly, $\cl J$ is a right ideal of $X$. Hence $\cl J$ is an ideal of $X$.
	\end{proof}

	\begin{defn} \ld{def:banach_alg:spectral:0019}
		Let $X$ be a Banach algebra and $J \in \MI(X)$. Then $J$ is said to be
		\begin{itemize}
			\item \tbf{proper} if $J \neq X$,
			\item \tbf{maximal} if $J$ is proper and for each $J' \in \MI(X)$, $J \subset J'$ implies that $J = J'$ or $J' = X$.
		\end{itemize}
	\end{defn}

	\begin{ex} \lex{ex:banach_alg:spectral:0020}
		Let $X$ be a unital Banach algebra and $J \in \MI(X)$. Then the following are equivalent:
		\begin{enumerate}
			\item $J$ is proper
			\item $e \not \in J$
			\item $J \cap G(X) = \varnothing$
		\end{enumerate}
	\end{ex}

	\begin{proof}\
		\begin{enumerate}
			\item $(1) \implies (2):$ \\
			Suppose that $e \in J$. Then for each $x \in X$, 
			\begin{align*}
				x
				& = xe \\
				& \in J.
			\end{align*}
			So $X \subset J$ and $J = X$. Hence $J$ is not proper. 
			\item $(2) \implies (3):$ \\
			Suppose that $J \cap G(X) \neq \varnothing$. Then there exists $x \in G(X)$ such that $x \in J$. Then 
			\begin{align*}
				e
				& = x^{-1} x \\
				& \in J. 
			\end{align*}
			\item $(3) \implies (1):$ \\
			Suppose that $J \cap G(X) = \varnothing$. Since $e \in G(X)$, we have that $e \not \in J$. Hence $J \neq X$ and $J$ is proper.
		\end{enumerate}
	\end{proof}

	\begin{ex} \lex{ex:banach_alg:spectral:0021}
		Let $X$ be a unital Banach algebra and $J,K \in \MT(X)$. Then 
		\begin{enumerate}
			\item $J+K \in \MI(X)$
			\item $JK \in \MI(X)$
		\end{enumerate}
		\tcr{FINISH!!!}
	\end{ex}

	\begin{proof}
		content...
	\end{proof}

	\begin{ex} \lex{ex:banach_alg:spectral:0022}
		Let $X$ be a unital Banach algebra and $J \in \MI(X)$. Suppose that $J$ is a proper. Then
		\begin{enumerate}
			\item $\cl J$ is proper,
			\item there exists $J' \in \MI(X)$ such that $J'$ is a maximal and $J \subset J'$,
			\item if $J$ is maximal, then $J$ is closed.
		\end{enumerate}
	\end{ex}

	\begin{proof}\
		\begin{enumerate}
			\item \rex{ex:banach_alg:spectral:0018} implies that $\cl J$ is an ideal of $X$. Since $J$ is proper, \rex{ex:banach_alg:spectral:0020} implies that $J \subset G(X)^c$. \rex{ex:banach_alg:intro:0016} implies that $G(X)^c$ is closed. Hence $\cl J \subset G(X)^c$. Another application of \rex{ex:banach_alg:spectral:0020} then implies that $\cl J$ is proper.
			\item \tcr{zorn's lemma FINISH!!!}
			\item Suppose that $J$ is maximal. Since $J \subset \cl J$ and $J$ is maximal, we have that $J = \cl J$ or $\cl J = X$. Since $J$ is maximal, $J$ is proper. The previous part implies that $\cl J$ is proper. Thus $\cl J \neq X$. Hence $J = \cl J$.
		\end{enumerate}
	\end{proof}

	\begin{ex} \lex{ex:banach_alg:spectral:0023}
		Let $X$ be a unital Banach algebra and $J \in \MI(X)$. If $J$ is maximal, then $X/J$ is a Banach algebra.
	\end{ex}

	\begin{proof}
		Suppose that $J$ is maximal. \tcr{A previous exercise in section on Banach spaces} implies that $X/J$ is a Banach space. Let $a,b \in X/J$ and $\ep > 0$. Then there exist $x,y \in X$ such that $a = x+J$ and $b = y+J$. Set 
		$$\ep_0 \defeq \min \bigg( \frac{\ep}{3 (\|x + J\| + 1)}, \frac{\ep}{3 (\|y + J\| + 1)}, \sqrt{\frac{\ep}{3}} \bigg)$$ 
		Then $\ep_0 > 0$. Thus there exist $z,w \in J$ such that $\|x + z\| < \|x + J\| + \ep_0$ and $\|y + w\| < \|y + J\| + \ep_0$. Since $J$ is an ideal of $X$, we have that $xw, zy, zw \in J$. Therefore $xw + zy + zw \in J$ and 
		\begin{align*}
			\|ab\|
			& = \|xy + J\| \\
			& = \inf_{q \in J} \|xy + q \| \\
			& = \leq \|xy + xw + zy + zw \| \\
			& = \|(x + z)(y + w) \| \\
			& \leq \|x+z\| \|y+w\| \\
			& < (\|x + J\| + \ep_0) (\|y + J\| + \ep_0) \\
			& = \|x + J\| \|y + J\| + \ep_0\|x + J\| + \ep_0 \|y + J\| + \ep_0^2 \\
			& < \|x + J\| \|y + J\| + \frac{\ep}{3} + \frac{\ep}{3} + \frac{\ep}{3} \\
			& = \|x + J\| \|y + J\| + \ep \\
			& = \|a\| \|b\| + \ep. 
		\end{align*}
		Since $\ep > 0$ is arbitrary, we have that for each $\ep >0$, $\|ab\| < \|a\| \|b\| + \ep$. Therefore $\|ab\| \leq \|a\| \|b\|$. Since $a,b \in X/J$ are arbitrary, we have that for each $a,b \in X/J$, $\|ab\| \leq \|a\| \|b\|$. Hence $X/J$ is a Banach algebra.
	\end{proof}

	\begin{note}
		We denote the canonical projection of $X$ onto $X/J$ by $\pi:X \rightarrow X/J$.
	\end{note}

	\begin{ex} \lex{ex:banach_alg:spectral:0024}
		Let $X$ be a unital Banach algebra, $J \in \MI(X)$ and $K \in \MI(X/J)$. Then 
		\begin{enumerate}
			\item $\pi^{-1}(K)$ is an ideal of $X$ and $J \subset \pi^{-1}(K)$,
			\item $X$ is commutative implies that $J$ is maximal iff that $X/J = \C(e + J)$.
		\end{enumerate}
	\end{ex}

	\begin{proof}\
		\begin{enumerate}
			\item \tcr{The preimage of an ideal is an ideal.}
			Let $x \in X$ and $y \in \pi^{-1}(K)$. Then $\pi(y) \in K$. Since $K \in \MI(X/J)$, we have that $\pi(x) \pi(y) \in K$. Therefore
			\begin{align*}
				\pi(xy)
				& = \pi(x) \pi(y) \\
				& \in K.
			\end{align*}
			Hence $xy \in \pi^{-1}(K)$. Since $x \in X$ and $y \in \pi^{-1}(K)$ is arbitrary, we have that for each $x \in X$ and $y \in \pi^{-1}(K)$, $xy \in \pi^{-1}(K)$. 
			\item Suppose that $X$ is commutative. \tcr{A previous exercise} implies that $X/J$ is a commutative unital Banach algebra. Since $J$ is maximal, \tcr{a previous exercise, or exercise in appendices} implies that $X/J$ is a field. The Gelfand-Mazur theorem \rex{} \tcr{(reference)} implies that $X/J = \C(e + J)$. 
		\end{enumerate}
	\end{proof}

	\begin{ex}\
		\begin{enumerate}
			\item $\MI(M_2(\C)) = \{(0)\}$. 
			\item $M_2(\C) / (0)$ is not a division ring.
		\end{enumerate} 
	\end{ex}
	
	\begin{proof}\
		\begin{enumerate}
			\item 
			\item 
		\end{enumerate}
	\end{proof}

	\begin{defn}
		Let $X$ be a Banach algebra. We define $\Imax(X) \defeq \{J \in \MI(X): \text{$J$ is a maximal}\}$.
	\end{defn}

	\begin{ex} \lex{ex:banach_alg:spectral:0025}
		Let $X$ be a unital Banach algebra.
		\begin{enumerate}
			\item For each $\phi \in \sig(X)$, $\ker \phi \in \Imax(X)$. \\
			\tbf{Hint:} Let $J' \in \MI(X)$. Suppose that $\ker \phi \subset J'$ and $\ker \phi \neq J$. Write $e = a + b$ with $a \in \ker \phi$ and $b \in J' \setminus \ker \phi$.
			\item Define $\al: \sig(X) \rightarrow \Imax(X)$ by $\al(\phi) = \ker \phi$. Then $\al$ is a bijection.
		\end{enumerate}
	\end{ex}

	\begin{proof}\
		\begin{enumerate}
			\item Let $\phi \in \sig(X)$. Let $x \in X$ and $y \in \ker \phi$. Then  
			\begin{align*}
				\phi(xy)
				& = \phi(x) \phi(y) \\
				& = \phi(x) 0 \\
				& = 0.
			\end{align*}
			Thus $xy \in \ker \phi$. Since $x \in X$ and $y \in \ker \phi$ are arbitrary, we have that for each $x \in X$ and $y \in \ker \phi$, $xy \in \ker \phi$. Hence $\ker \phi$ is a left ideal of $X$. Similarly,  $\ker \phi$ is a right ideal of $X$. Hence $\ker \phi \in \MI(X)$. Since $\phi \in \sig(X)$, $\phi \neq 0$ and therefore $\ker \phi \neq X$. So $\ker \phi$ is proper. Let $J' \in \MI(X)$. Suppose $\ker \phi \subset J'$ and $\ker \phi \neq J'$. Then there exists $x \in J'$ such that $x \not \in \ker \phi$. Hence $\phi(x) \neq 0$. Set $x' \defeq \phi(x)^{-1} x$. Then 
			\begin{align*}
				\phi(e - x')
				& = \phi(e) - \phi(x') \\
				& = 1 - \phi(x)^{-1}\phi(x) \\
				& = 1 - 1 \\
				& = 0.
			\end{align*}
			Thus  
			\begin{align*}
				e - x' 
				& \in \ker \phi \\
				& \subset J'.
			\end{align*}
			Since $J'$ is an ideal of $X$ and $x \in J'$, we have that $[\phi(x)^{-1}e] x \in J'$. Therefore 
			\begin{align*}
				e
				& = e - x' + x' \\
				& = (e - x') + [\phi(x)^{-1}e] x \\
				& \in J'
			\end{align*}
			\tcr{A previous exercise} implies that $J' = X$. Since $J' \in \MI(X)$ with $\ker \phi \subset J'$ and $\ker \phi \neq J'$ is arbitrary, we have that for each $J' \in \MI(X)$, $\ker \phi \subset J'$ and $\ker \phi \neq J'$ implies that $J' = X$. Thus $\ker \phi$ is maximal. 
			\item Gelfand-Mazur Theorem
		\end{enumerate}
	\end{proof}

	\begin{ex}
		\tcr{If $\sig(X) \neq \varnothing$ and $\dim X \geq 2$ (i.e. $X \neq \C e$), then $\MI(X) \neq \{(0)\}$.}
		\tcr{If $X$ is commutative, then for each $x \in G(X)$, there exists $J \in \MI(X)$ such that $x \in J$} (so that if $x \neq 0$ and $x \in G(X)$, then $\MI(X) \neq \{(0)\}$. Thus there exists $\phi \in \sig(X)$ since $\sig(X)$ is bijective to $\MI(X)$), \tcr{so commutativity implies that}. 
	\end{ex}

	\tcr{note: For example $\MI(\C^{n \times n}) = \{(0)\}$. why isnt then $\C^{n \times n}/ (0)$ a division ring? so it must not be the case that division by a maximal ideal always results in a division ring \href{https://math.stackexchange.com/questions/1150727/m-maximal-in-a-ring-r-what-is-r-m?rq=1}{see first stackexchange answer}} 
	
	\begin{proof}
		Let $\phi \in \sig(X)$. Then  $\ker \phi$ is a maximal ideal of $X$. Since $\dim X \geq 2$, $\dim \ker \phi \geq 1$. Hence $\ker \phi \neq (0)$.
	\end{proof}
	

	
	
	
	
	
	
	
	
	\begin{itemize}
		\item $\implies:$ \\
		\item $\impliedby:$ \\
	\end{itemize}
	
	\begin{itemize}
		\item $\implies:$ \\
		
		\item $\impliedby:$ \\
		
	\end{itemize}
	














































	\newpage
	\subsection{$*$-Algebras}

	\begin{defn}
		Let $X$ be an algebra and $*: X \rightarrow X$. 
		\begin{itemize}
			\item Then $*$ is said to be an \tbf{involution on $X$} if 
			for each $x, y \in X$ and $\lam \in \C$, 
			\begin{enumerate}
				\item $(x + y)^* = x^* + y^*$, 
				\item $(\lam x)^* = \lam^* x^*$, 
				\item $(xy)^* = y^*x^*$, 
				\item $(x^*)^* = x$.
			\end{enumerate}
			\item If $X$ is a Banach algebra, then $(X, *)$ is said to be a \tbf{$C^*$-algebra} if for each $x \in X$, $\|x^*x\| = \|x\|^2$.
		\end{itemize}
	\end{defn}

	\begin{note}
		We note that $* \in \Hom_{\VectC}(X, \cnj(X))$ and therefore $0^* = 0$. \tcr{(need to define the conjugate of a vector space $\cnj(X)$)}
	\end{note}

	\begin{ex}
		Let $X$ be a $C^*$-algebra. Then  
		\begin{enumerate}
			\item for each $x \in X$, $\|x\| = \|x^*\|$
			\item for each $x,y \in X$, $\|x^*y\| = \|y^*x\|$
			\item $* \in L(X, \cnj(X))$.
		\end{enumerate}
	\end{ex}

	\begin{proof}
		\begin{enumerate}
			\item  Let $x \in X$. If $x = 0$, then $x^* = 0$ and therefore $\|x\| = \|x^*\|$. Suppose that $\|x\| \neq 0$. Then 
			\begin{align*}
				\|x\|^2
				& = \|x^* x\| \\
				& \leq \|x^*\| \|x\|. 
			\end{align*}
			Thus $\|x\| \leq \|x^*\|$. Similarly, we have that $\|x^*\| \leq \|(x^*)^*\|$. Therefore
			\begin{align*}
				\|x\| 
				& \leq \|x^*\| \\ 
				& \leq \|(x^*)^*\| \\
				& = \|x\|.
			\end{align*}
			Hence $\|x^*\| = \|x\|$.
			\item Let $x, y \in X$. We have that
			\begin{align*}
				\|x^* y\|
				& = \|(x^*y)^*\| \\
				& = \|y^*(x^*)^*\| \\
				& = \|y^* x\|.
			\end{align*}
			\item \tcr{As noted above}, $* \in \Hom_{\VectC}(X, \cnj(X))$. \tcr{The previous part} implies that for each $x \in X$, $\|x^*\| = \|x\|$. Hence $\|*\| \leq 1$. Similarly, 
			\begin{align*}
				\|*\| 
				& = \sup\limits_{\|x\| = 1} \|x^*\| \\ 
				& = 
			\end{align*}
		\end{enumerate}
	\end{proof}

	
	\begin{ex}
		Let $X$ be a $C^*$-algebra. Then $*:X \rightarrow X$ is continuous. 
	\end{ex}

	\begin{proof}
		
	\end{proof}

































\newpage
\chapter{Semigroup Theory}
	
	
	
	
	
	
	
	
	
	
	
	
	
	
	
	
	
	
	
	
	
	
	
	
	
	
	
	
	
	
	
	\newpage
	
	\chapter{Banach Modules}
	
	\section{Introduction}
	
	\begin{defn}
		Let $A$ be a Banach algebra and $X$ a Banach space and $\mu: A \times X \rightarrow X$. Then $(X, A, \mu)$ is said to be a \tbf{left Banach $A$-module} if 
		\begin{enumerate}
			\item $(X, A, \mu)$ is a left $A$-module
			\item $\mu \in L(A, X; X)$ and $\|\mu\| \leq 1$
		\end{enumerate}
	\end{defn}
	
	\begin{note}
		Condition $(2)$ is equivalent to the assumption that for each $a \in A$ and $x \in X$, $\|ax\|_X \leq \|a\|_A\|x\|_X$. 
	\end{note}
	
	
	
	
	
	
	
	
	
	
	
	
	
	
	
	
	
	

	
	
	
	
	
	
	\newpage
	\chapter{Convexity}
	
	\section{Introduction}

	\begin{note}
	In this section, we assume all vector spaces are real.
	\end{note}

	\begin{defn} \ld{91001}
	Let $X$ be a vector space and $A \subset X$. Then $A$ is said to be $\tbf{convex}$ if for each $x, y \in A$, and $t \in [0,1]$,  $tx + (1-t)y \in A$. 
	\end{defn}	
	
	\begin{defn} \ld{91002}
	Let $X$ be a vector space and $f:A \rightarrow R$. Then $f$ is said to be \tbf{convex} if for each $x,y \in A$, $t \in \ui$, $$f(tx + (1-t)y) \leq tf(x) + (1-t)f(y)$$
	\end{defn}
	
	\begin{defn} \ld{91003}
	Let $X$ be a vector space and $f:A \rightarrow R$. Then $f$ is said to be \tbf{strictly convex} if for each $x,y \in A$, $t \in (0,1)$, $x \neq y$ implies that $$f(tx + (1-t)y) < tf(x) + (1-t)f(y)$$
	\end{defn}
	
	\begin{ex} \lex{91004}
	Let $X$ be a vector space, $f \in X^*$ and $g: X \rightarrow \R$ constant. Then $f$ and $g$ are convex. 
	\end{ex}
	
	\begin{proof}
		Let $x, y \in X$ and $t \in \ui$. Put $c = g(0)$. Then $$f(tx + (1-t)y) = tf(x) + (1-t)f(y)$$ and 
		\begin{align*}
		g(tx + (1-t)y) 
		&= c\\ 
		&= tc + (1-t)c \\
		&= tg(x) + (1-t)g(y)
		\end{align*}
		So $f$ and $g$ are convex.
	\end{proof}		

	\begin{ex} \tbf{Star-shapedness:}
		Let $f:\Rg \rightarrow \R$ be convex. If $f(0) \leq 0$, then for each $x \in \Rg$, $t \in [0,1]$, $f(tx) \leq tf(x)$.
	\end{ex}

	\begin{proof}
		Suppose that $f(0) \leq 0$. Let $x \in \Rg$ and $t \in [0,1]$. Then 
		\begin{align*}
			f(tx)
			&= f(tx + (1-t)0) \\
			& \leq tf(x) + (1-t)f(0) \\
			& \leq tf(x)
		\end{align*}
	\end{proof}

	\begin{ex} \tbf{Superadditivity:} \lex{91004.2}\\
		Let $f:\Rg \rightarrow \R$ be convex. If $f(0) = 0$, then for each $x,y \in \Rg$, $$f(x) + f(y) \leq f(x+y)$$
		\tbf{Hint:} $f(x) = f \bigg( \frac{x}{x+y}(x+y)\bigg)$
	\end{ex}

	\begin{proof}
		Suppose that $f(0) = 0$. Let $x, y \in \Rg$. If $x+y = 0$, then $x=y=0$ and $f(x) + f(y) = 0 = f(x+y)$. Suppose that $x+y \neq 0$. Then the previous exercise implies that 
		\begin{align*}
			f(x) + f(y) 
			&= f \bigg( \frac{x}{x+y}(x+y)\bigg) + f \bigg( \frac{y}{x+y}(x+y)\bigg) \\
			& \leq \frac{x}{x+y}f(x+y) + \frac{y}{x+y}f(x+y) \\
			&= f(x+y)
		\end{align*}
	\end{proof}
	
	\begin{ex} \lex{91005}
	Let $X$ be a vector space, $A \subset X$ convex, $f,g:A \rightarrow \R$ and $\lam \geq 0$. If $f,g$ are convex, then 
	\begin{enumerate}
	\item $f + g$ is convex 
	\item $\lam f$ is convex
	\end{enumerate}
	\end{ex}
	
	\begin{proof}
	Suppose that $f$ and $g$ are convex. Let $x,y \in A$ and $t \in [0,1]$. Then 
	\begin{align*}
	(f + \lam g)(tx + (1-t)y) 
	&= f(tx + (1-t)y) + \lam g(tx + (1-t)y) \\
	& \leq tf(x) + (1-t)f(y) +  t \lam g(x) + (1-t)\lam g(y) \\
	&= t(f(x) + \lam g(x)) + (1-t)(f(y) + \lam g(y))\\
	& = t(f + \lam g)(x) + (1-t)(f + \lam g)(y)
\end{align*}		 
	\end{proof}
	
	
	\begin{defn} \ld{91006}
	Let $X$ be a vector space and $f: X \rightarrow \R$. Then $f$ is said to be \tbf{affine} if there exists $\phi \in X^*$, $a \in \R$ constant such that $f = \phi + a$.\\
	\end{defn}
	
	\begin{ex} \lex{91007}
	Let $X$ be a vector space and $f: X \rightarrow \R$. If $f$ is affine, then $f$ is convex.
	\end{ex}
	
	\begin{proof}
	Suppose that $f$ is affine. Then there exists $\phi \in X^*$, $a \in R$ constant such that $f = \phi + a$. Then $\phi$ is convex and $g: X \rightarrow \R$ defined by $g(x) = a$ is convex. So $f = \phi + g$ is convex.
	\end{proof}
	
	\begin{ex} \lex{91008}
	Let $X$ be a vector space, $A \subset X$ convex, $f:\R \rightarrow \R$ and $g: A \rightarrow \R$. If $f$ is convex and increasing and $g$ is convex, then $f \circ g$ is convex.
	\end{ex}	
	
	\begin{proof}
	Let $t \in [0,1]$ and $x, y \in A$. Then convexity of $g$ implies that $$g(tx +(1-t)y) \leq tg(x) + (1-t)g(y)$$ and we have
	\begin{align*}
	f\circ g(tx +(1-t)y) 
	&= f(g(tx +(1-t)y)) \\
	& \leq f(tg(x) + (1-t)g(y)) \hspace{2cm} (f \text{ increasing)}\\
	& \leq tf(g(x)) + (1-t)f(g(y)) \hspace{2cm}  (f \text{ convex)}\\	
	&= tf \circ g(x) + (1-t)f \circ g(y)
\end{align*}	 
So $f \circ g$ is convex.
	\end{proof}
	
	\begin{ex} \lex{91009}
	Let $X$ be a vector space, $A \subset X$ convex, $f:A \rightarrow \R$ convex and $x_0 \in A$. Then $f$ has a local minimum point at $x_0$ iff $f$ has a global minimum point at $x_0$.
	\end{ex}	
	
	\begin{proof}
	If $f$ has a global minimum point at $x_0$, then $f$ has a local minimum point at $x_0$. Conversely, suppose that $f$ has a local minimum point at $x_0$. Then there exists $\del >0$ such that for each $x \in B(x_0, \del) \cap A$, $f(x_0) \leq f(x)$. For the sake of contradiction, suppose that $f$ does not have a global minimum point at $x_0$. Then there exits $x' \in A$ such that $f(x') < f(x_0)$. Put $t_0 = \min(\frac{\del}{\|x' - x_0\| + 1}, 1) >0$. Let $t \in (0, t_0)$, then
	\begin{align*}
	\|(tx' + (1-t)x_0) - x_0\| 
	&= t\|x' -x_0 \| \\
	& <   \frac{\|x' -x_0 \|\del}{\|x' -x_0\| + 1} \\
	& < \del
	\end{align*} 
	so that $tx' + (1-t)x_0 \in B(x_0, \del) \cap A$ and hence $f(x_0) \leq f(tx' + (1-t)x_0)$.  Therefore  
	\begin{align*}
	f(x_0) 
	& \leq f(tx' + (1-t)x_0) \\
	& \leq tf(x') + (1-t)f(x_0)  \hspace{.5cm} (\text{convexity of }f)\\
	& < tf(x_0) + (1-t)f(x_0) \\
	&= f(x_0)
	\end{align*}
	which is a contradiction. Hence $f$ has a global minimum point at $x_0$.
	\end{proof}
	
	\begin{ex} \lex{91010}
	Let $X$ be a vector space, $A \subset X$ convex, $f:A \rightarrow \R$ strictly convex and $x_0 \in X$. If $f$ has a local minimum point at $x_0$, then $f$ has a unique global minimum point at $x_0$.  
	\end{ex}
	
	\begin{proof}
	Suppose that $f$ has a local minimum point at $x_0$. The previous exercise implies that $f$ has a global minimum point at $x_0$. For the sake of contradiction suppose that there exists $x_1 \in X$ such that $f$ has a global minimum point at $x_1$ and $x_0 \neq x_1$. This implies $f(x_0) = f(x_1)$. Set $t = 1/2$. Strict convexity implies that 
	\begin{align*}
	f(tx_0 + (1-t)x_1) 
	&< tf(x_0) + (1-t)f(x_1)  \\
	&= f(x_0) 
	\end{align*}
	which is a contradiction since $f$ has a global minimum point at $x_0$.
	\end{proof}

	
	\begin{defn} \ld{91011}
	Let $X, Y$ be vector spaces, $A \subset X \oplus Y$. For $y \in Y$, define $$A^y = \{x \in X: (x,y) \in A \}$$ and $f^y:A^y \rightarrow \R$ by $$f^y(x) = f(x,y)$$
	\end{defn}
	
	\begin{ex} \lex{91012}
	Let $X, Y$ be vector spaces, $A \subset X \oplus Y$ convex and $f:A \rightarrow \R$ convex. Then for each $y \in \pi_2(A)$,
	\begin{enumerate}
	\item $A^y$ is convex 
	\item $f^y$ is convex  
	\end{enumerate}	  
	where $\pi_2: X\times Y \rightarrow Y$, the canonical projection of $X \times Y$ onto $Y$ given by $\pi_2(x,y) = y$.
	\end{ex}
	
	\begin{proof}
	Let $y \in \pi_2(A)$, $x_1, x_2 \in A^y$ and $t \in [0,1]$. Then by definition, $(x_1, y)$, $(x_2, y) \in A$.
	\begin{enumerate}
	\item  Convexity of $A$ implies that $(tx_1 + (1-t)x_2, y) \in A$. Hence $tx_1 + (1-t)x_2 \in A^y$ and $A^y$ is convex. 
	\item Convexity of $f$ implies that 
	\begin{align*}
	f^y(tx_1 + (1-t)x_2)
	&= f(tx_1 + (1-t)x_2, y) \\
	&= f(t(x_1, y) + (1-t)(x_2,y)) \\
	& \leq tf(x_1, y) + (1-t) f(x_2,y) \\
	&= tf^y(x_1) + (t-t)f^y(x_2)
\end{align*}	  
	and so $f^y$ is convex.
	\end{enumerate}
	\end{proof}
	
	\begin{ex} \lex{91013}
	Let $X$, $Y$ be vector spaces and $A\subset X, B \subset Y$. If $A$ and $B$ are convex, then $A \times B \subset X \oplus Y$ is convex.
	\end{ex}	
	
	\begin{proof}
	Suppose that $A$ and $B$ are convex. Let $(x_1,y_1), (x_2,y_2) \in A \times B$ and $t \in [0,1]$. Convexity of $A$ and $B$ implies that $tx_1 + (1-t)x_2 \in A$ and $ty_1 + (1-t)y_2 \in B$. Therefore 
	\begin{align*}
	t(x_1,y_1) + (1-t)(x_2,y_2) 
	&= (tx_1 + (1-t)x_2, ty_1 + (1-t)y_2) \\
	& \in A \times B
\end{align*}	 
	\end{proof}
	
	\begin{ex} \lex{91014}
	Let $X, Y$ be vector spaces and $A \subset X$, $B \subset Y$ convex (implying that $A \times B$ is convex)  and $f:A \times B \rightarrow \R$ convex. Suppose that for each $y \in B$, $\{f(x, y): x \in A\}$ is bounded below. Then $\inf\limits_{y \in B}f^y$ is convex
	\end{ex}
	
	\begin{proof}
	Put $g = \inf\limits_{y \in B}f^y$. 
	Let $x_1, x_2 \in A$, $y_1, y_2 \in B$ and $t \in [0,1]$. Put $y'= ty_1 + (1-t)y_2$. Then convexity of $f$ implies that
	\begin{align*}
	g(tx_1 + (1-t)x_2) 
	& \leq f^{y'}(tx_1 + (1-t)x_2) \\
	&= f(tx_1 + (1-t)x_2, ty_1 + (1-t)y_2)\\
	&= f(t(x_1,y_1) + (1-t)(x_2, y_2)) \\
	& \leq tf(x_1, y_1) + (1-t)f(x_2, y_2) \\
	&= tf^{y_1}(x_1) + (1-t)f^{y_2}(x_2) \\
	\end{align*}
	Since $y_1 \in B$ is arbitrary, we have that $$g(tx_1 + (1-t)x_2) \leq tg(x_1) + (1-t)f^{y_2}(x_2)$$ Similarly, since $y_2 \in B$ is arbitrary, we have that $$g(tx_1 + (1-t)x_2) \leq tg(x_1) + (1-t)g(x_2)$$ and $f$ is convex.
	\end{proof}	

	\begin{ex} \lex{91015}
	Let $X$ be a vector space, $A \subset X$ convex and $ (f_{\lam})_{\lam \in \Lam} \subset \R^A$. Suppose that for each $\lam \in \Lam$, $f_\lam$ is convex. Define 
	\begin{enumerate}
		\item $A^* = \{x \in A: \sup\limits_{\lam \in \Lam} f_{\lam}(x) < \infty\}$
		\item $f^*:A^* \rightarrow \R$ by $f^*(x) =  \sup\limits_{\lam \in \Lam} f_{\lam}(x)$
	\end{enumerate}
Then
	\begin{enumerate}
		\item $A^*$ is convex 
		\item $f^*$ is convex
	\end{enumerate}
	\end{ex}
	
	\begin{proof}
		\begin{enumerate}
			\item Let $x, y \in A$ and $t \in [0,1]$. By definition, $ \sup\limits_{\lam \in \Lam} f_{\lam}(x)$,  $\sup\limits_{\lam \in \Lam} f_{\lam}(y) < \infty$. Therefore 
			\begin{align*}
				 \sup\limits_{\lam \in \Lam} f_{\lam}(tx + (1-t)y) 
				&\leq \sup\limits_{\lam \in \Lam} [tf_{\lam}(x) + (1-t)f_{\lam}(y) ]  \\
				& \leq t \sup\limits_{\lam \in \Lam} f_{\lam}(x) + (1-t) \sup\limits_{\lam \in \Lam} f_{\lam}(y) \\
				& < \infty
			\end{align*}
			So $tx + (1-t)y \in A$.
			\item  By definition, the previous part implies that for each $x,y \in A^*$, $f^*(tx + (1-t)y) \leq t f^*(x) + (1-t)f^*(y)$. So $f^*:A^* \rightarrow \R$ is convex.
		\end{enumerate}
	\end{proof}
	
	
	
	
	\begin{ex} \lex{91016}
	Let $X$ be a normed vector space, $A \subset X$ open and convex, $f:A \rightarrow \R$ convex and $x_0 \in A$. If $f$ is continuous at $x_0$, then $f$ is locally Lipschitz at $x_0$. \\
	\tbf{Hint:} Given $x_1, x_2$ near $x_0$ Choose a $z$ near $x_0$ s.t. $x_1$ is a convex combination of $x_2$ and $z$. Then repeat but with $x_2$ as a convex combination of $x_1$ and $z$
	\end{ex}
	
	\begin{proof}
	By continuity, $f$ is locally bounded at $x_0$. So there exist $M, \del >0$ such that $B(x_0, \del) \subset A$ and for each $x \in B(x_0, \del)$, $|f(x)| \leq M$. Put $\del' = \frac{\del}{2}$ and choose $U = B(x_0, \del')$. Then $U \subset A$ and $U \in \MN(x_0)$. \\
	Let $x_1, x_2 \in U$. Suppose that $x_1 \neq x_2$. Define $\al = \|x_1 - x_2\| >0$, $p = \frac{\al}{\al + \del'}$, $q = 1-p$ and $z = p^{-1}(x_1 - qx_2)$. Then $x_1 = pz + qx_2$ and 
	\begin{align*}
	\|z - x_1\| 
	&= \|(p^{-1} - 1)x_1 - p^{-1}qx_2\| \\
	&= \frac{1-p}{p} \al \\
	&= \frac{\del'}{\al} \al \\
	&= \del ' 
	\end{align*}
	Therefore 
	\begin{align*}
	\|z - x_0\| 
	& \leq \|z - x_1\| + \|x_1 - x_0\| \\
	& <  \del '  + \del '  \\
	&= \del
\end{align*}	  
	So $z \in B(x_0, \del)$, which implies that 
	\begin{align*}
	f(z) - f(x_2) 
	& \leq |f(z) - f(x_2)|\\ 
	&\leq |f(z)| + |f(x_2)| \\
	&\leq 2M
\end{align*}		
	Since $x_1 = pz + qx_2$, convexity of $f$ implies that $f(x_1) \leq pf(z) + qf(x_2)$. Hence 
	\begin{align*}
	f(x_1) - f(x_2) 
	& \leq pf(z) -pf(x_2) \\
	&= p(f(z) - f(x_2)) \\
	& \leq p 2M \\
	&= \frac{\al}{\al + \del'} 2M \\
	& \leq \al 2M \\
	&= 2M \|x_1 - x_2 \|
	\end{align*}
	Similarly, choosing $z = p^{-1}(x_2 - qx_1)$, yields $f(x_2) - f(x_1) \leq 2M \|x_1 - x_2 \|$ which implies that $$|f(x_1) - f(x_2)| \leq 2M \|x_1 - x_2 \|$$ and $f$ is Lipschitz on $U$. 
 	\end{proof}


	
	
	
	
	
	
	
	
	
	
	
	
	
	
	
	
	\newpage
	\section{The Subdifferential}
	
	\begin{ex} \lex{}
	Let $X$ be a Banach space, $A \subset X$ open and convex, $f:A \rightarrow \R$ convex, $x_0 \in A$ and $x \in X$. Define $T = \{ t \in \R: x_0+tx \in A\}$. Then there exist $a, b \in (0, \infty]$ such that $T = (-a, b)$.
	\end{ex}
	
	\begin{proof}
	Continuity of scalar multiplication and addition implies that $T$ is an open neighborhood of $0$. Let $t > 0$ and $s \in [0,t]$. Then $\frac{s}{t} \in [0, 1]$ and by convexity of $A$, $x_0 + tx \in A$ implies that
	\begin{align*}
	x_0 + sx 
	&= \frac{s}{t}(x_0 + tx) + \bigg(1-\frac{s}{t} \bigg )x_0\\
	& \in A
	\end{align*} 
	Thus $[0,t] \subset T$. Similarly, $x_0 - tx \in A$ implies that $[-t, 0] \subset T$. \\
	Define $a,b \in (0, \infty]$ by $a = \sup \{t > 0: x_0 -tx \in A\}$ and $b = \sup\{t > 0: x_0 +tx \in A\}$. Then $(-a, b) = T$.
	\end{proof}
	
	\begin{defn} \ld{}
	Let $X$ be a Banach space, $A \subset X$ open and convex, $f:A \rightarrow \R$ convex, $x_0 \in A$ and $x \in X$. Define $T$ as in the previous exercise and choose $t_0 >0$ such that $(-t_0, t_0) \subset T$. For $t \in (0,t_0)$, define the difference quotient $q: (-t_0, t_0) \setminus \{0\} \rightarrow \R$ by$$q(t) = \frac{f(x_0 + tx) - f(x_0)}{t}$$ 
	\end{defn}	
	
	\begin{ex} \lex{}
	Let $X$ be a Banach space, $A \subset X$ open and convex, $f:A \rightarrow \R$ convex, $x_0 \in A$ and $x \in X$. Define $t_0$ as above.
	Then
	\begin{enumerate}
	\item $q(t)$ is increasing on $(0, t_0)$
	\item $q(-t)$ decreasing on $(0, t_0)$
\end{enumerate}	 
	 \tbf{Hint:} As an example, look at the graph of $f(x) = x^2$. For the algebra, start at the desired end inequality and work backwards
	\end{ex}	
	
	\begin{proof}\
	\begin{enumerate}
	\item Let $s, t \in (0, t_0)$ and suppose that $s \leq t$. Then $x_0 +sx$, $x_0 + tx \in A$. Note that since $0 < s \leq t$, $\frac{s}{t} \in (0, 1]$ and $1- \frac{s}{t} = \frac{t-s}{t} \in (0, 1]$. Also, since $A$ is convex, we have that $$ \bigg( \frac{t-s}{t} \bigg) x_0 +  \bigg(\frac{s}{t} \bigg) (x_0 + tx)  \in A$$
	Convexity of $f$ implies that 
	\begin{align*}
	f(x_0 + sx)
	&= f\bigg ( \bigg( \frac{t-s}{t} \bigg) x_0 +  \bigg(\frac{s}{t} \bigg) (x_0 + tx) \bigg) \\
	& \leq \bigg( \frac{t-s}{t} \bigg) f(x_0) + \bigg(\frac{s}{t} \bigg) f(x_0 + tx)
	\end{align*}
	This implies that $$tf(x_0 + sx) \leq (t-s) f(x_0) + s f(x_0 + tx)$$
	and after rearranging, we get $$t f(x_0 + sx) - tf(x_0) \leq s f(x_0 + tx) - sf(x_0)$$
	and so finally, dividing both sides by $st$, we obtain 
	\begin{align*}
	q(s)
	&= \frac{f(x_0 + sx) - f(x_0)}{s} \\
	& \leq \frac{f(x_0 + tx) - f(x_0)}{t} \\
	&= q(t) 
	\end{align*}
	as desired.
	\item Similar to $(1)$.
	\end{enumerate}
	\end{proof}
	
	\begin{ex} \lex{}
	Let $X$ be a Banach space, $A \subset X$ open and convex, $f:A \rightarrow \R$ convex, $x_0 \in A$ and $x \in X$. Define $t_0$ as before. Then for each $t \in (0, t_0)$, $$q(-t) \leq q(t)$$ \\
	\tbf{Hint:} for sufficiently small $t$, convexity of $f$ implies that $f(x_0) \leq \frac{1}{2} f(x_0 - 2tx) + \frac{1}{2} f(x_0 + 2tx)$
	\end{ex}
	
	\begin{proof}
	Choose $t_0$ as in the previous exercise. Since convexity of $f$ implies that for each $t \in (0, t_0/2)$,
	$$f(x_0) \leq \frac{1}{2} f(x_0 - 2tx) + \frac{1}{2} f(x_0 + 2tx)$$
	we have that for each $t \in (0, t_0/2)$,
	\begin{align*}
	q(-2t) 
	&= \frac{f(x_0 - 2tx) - f(x_0)}{-2t} \\
	&\leq \frac{f(x_0 + 2tx) - f(x_0)}{2t} \\
	&= q(2t)
	\end{align*}	 
	So for each $t \in (0, t_0)$, $q(-t) \leq q(t)$.
	\end{proof}	
	
	\begin{ex} \lex{}
	Let $X$ be a Banach space, $A \subset X$ open and convex, $f:A \rightarrow \R$ convex and $x_0 \in A$. Then 
	\begin{enumerate}
	\item $f$ is left-hand and right-hand Gateaux differentiable at $x_0$ with $d^-f(x_0) \leq d^+f(x_0)$ 
	\item for each $x \in X$, $d^-f(x_0)(x) = - d^+f(x_0)(-x)$
	\end{enumerate}
	\end{ex}	
	
	\begin{proof}\
	\begin{enumerate}
	\item Let $x \in X$. Choose $t_0 >0$ as in the previous two exercises. Let $t, u \in (0,t_0)$. Choose $s \in (0, \min(u, t))$. The previous two exercises imply that 
	\begin{align*}
	q(-u) 
	& \leq q(-s) \\ 
	&\leq q(s) \\
	&\leq q(t)
	\end{align*} and therefore $q(t)$ is an upper bound for $\{q(-u): u \in (0,t_0)\}$ and $d^-f(x_0)(x) = \sup\limits_{u \in (0,t_0)}q(-u)$ exists with $d^-f(x_0)(x) \leq q(t)$.\\
	Since $t \in (0, t_0)$ is arbitrary, $d^-f(x_0)(x)$ is a lower bound for $\{q(t): t \in (0, t_0)\}$. Therefore $$d^+f(x_0)(x) = \inf_{t \in (0,t_0)}q(t)$$ exists with $d^+f(x_0)(x) \geq d^-f(x_0)(x)$. 
	\item By definition, we have 
	\begin{align*}
	d^-f(x_0)(x)
	&= \lim_{t \rightarrow 0^+} \frac{f(x_0 + -tx) - f(x_0)}{-t} \\
	&= - \lim_{t \rightarrow 0^+} \frac{f(x_0 + -tx) - f(x_0)}{t} \\
	&= - d^+f(x_0)(-x)
	\end{align*}
	\end{enumerate}
	\end{proof}
	
	\begin{ex} \lex{}
	Let $X$ be a Banach space, $A \subset X$ open and convex, $f:A \rightarrow \R$ convex and $x_0 \in A$. Then $d^+f(x_0):X \rightarrow \R$ is a sublinear functional.
	\end{ex}	
	
	\begin{proof}
	Let $x,y \in X$ and $k \geq 0$. If $k = 0$, then clearly
	\begin{align*}
	d^+f(x_0)(kx)
	&= k d^+(x_0)(x)
	\end{align*}
	If $k >0$. Then 
	\begin{align*}
	d^+f(x_0)(kx)
	&= \lim_{t \rightarrow 0^+} \frac{f(x_0 + tkx) - f(x_0)}{t} \\
	&= k\lim_{t \rightarrow 0^+} \frac{f(x_0 + tkx) - f(x_0)}{tk}\\
	&= kd^+f(x_0)(x)
	\end{align*}
	Define $t_0 >0$ as before and let $t \in (0, \frac{t_0}{2})$. Note that $$x_0 + tx + ty = \frac{1}{2}(x_0 + 2tx) + \frac{1}{2}(x_0 + 2ty)$$ 
	Convexity of $f$ implies that $$f(x_0 + tx + ty) \leq \frac{1}{2}f(x_0 + 2tx) + \frac{1}{2}f(x_0 + 2ty)$$
	which implies that $$\frac{f(x_0 + tx + ty) - f(x_0)}{t} \leq \frac{f(x_0 + 2tx) - f(x_0)}{2t} + \frac{f(x_0 + 2ty) - f(x_0)}{2t}$$
	Therefore 
	\begin{align*}
	d^+f(x_0)(x+y) 
	&= \lim_{t \rightarrow 0^+} \frac{f(x_0 + t(x + y)) - f(x_0)}{t} \\
	&= \lim_{t \rightarrow 0^+} \frac{f(x_0 + tx + ty) - f(x_0)}{t} \\
	& \leq \lim_{t \rightarrow 0^+} \bigg[ \frac{f(x_0 + 2tx) - f(x_0)}{2t} + \frac{f(x_0 + 2ty) - f(x_0)}{2t} \bigg ] \\
	&= \lim_{t \rightarrow 0^+}  \frac{f(x_0 + 2tx) - f(x_0)}{2t} + \lim_{t \rightarrow 0^+} \frac{f(x_0 + 2ty) - f(x_0)}{2t} \\
	&= d^+f(x_0)(x) + d^+f(x_0)(y) 
	\end{align*}
	\end{proof}
	
	\begin{ex} \lex{}
	Let $X$ be a Banach space, $A \subset X$ open and convex, $f:A \rightarrow \R$ convex and $x_0 \in A$. Then for each $x \in A$, $$d^+f(x_0)(x-x_0) \leq f(x) - f(x_0)$$
	\end{ex}	
	
	\begin{proof}
	Let $x \in A$. Define $T = \{t \in \R: x_0 + t(x-x_0) \in A\}$ similarly to earlier. Clearly $1 \in T$ and  
	\begin{align*}
	d^+f(x_0)(x - x_0) 
	&= \inf_{t \in (0,1]} \frac{f(x_0 + t(x-x_0)) - f(x_0)}{t} \\
	& \leq f(x) - f(x_0)
	\end{align*}
	\end{proof}
	
	\begin{ex} \lex{}
	Let $X$ be a Banach space, $A \subset X$ open and convex, $f:A \rightarrow \R$ convex and $x_0 \in A$. If $f$ is continuous at $x_0$, then $d^+f(x_0)$ is Lipschitz (equivalently bounded). 
	\end{ex}	
	
	\begin{proof}
	Suppose that $f$ is continuous at $x_0$. A previous exercise about convex functions tells us that $f$ is locally Lipschitz at $x_0$, so there exists $\del,M >0$ such that for each $x_1, x_2 \in B(x_0, \del)$, $|f(x_1) - f(x_2)| \leq M\|x_1 - x_2\|$. Let $x \in X$ and define $t_0 = \frac{\del}{\|x\| + 1}$ so that for each $t \in (0, t_0)$,
	\begin{align*}
	\|(x_0 +tx) - x_0\|
	& = t\|x\| \\
	& \leq t_0 \| x\| \\
	&= \frac{\del \| x\|}{\|x\| + 1} \\
	& < \del
	\end{align*}  
	and $x_0 +tx \in B(x_0, \del)$.
	Then for each $t \in (0, t_0)$, 
	\begin{align*}
	d^+f(x_0)(x) 
	& \leq \frac{f(x_0 + tx) - f(x_0)}{t} \\
	& \leq \frac{|f(x_0 + tx) - f(x_0)|}{t} \\
	& \leq t^{-1}M \| (x_0 + tx) - x_0\| \\
	&= M\|x\|
	\end{align*}
	Thus $d^+f(x_0)$ is a bounded sublinear functional and  a previous exercise in the section on sublinear functionals implies this is eqivalent to $d^+f(x_0)$ being Lipschitz.
	\end{proof}
	
	\begin{ex} \lex{}
	Let $X$ be a Banach space, $A \subset X$ open and convex, $f:A \rightarrow \R$ convex and $x_0 \in A$. If $f$ is continuous at $x_0$, then there exists $\phi \in X^*$ such that $\phi \leq d^+f(x_0)$.
	\end{ex}	
	
	\begin{proof}
	Suppose that $f$ is continuous at $x_0$. The previous exercise implies that $d^+f(x_0)$ is Lipschitz (equivalently bounded). A previous exercise in the section discussing sublinear functionals tells us that boundedness of $d^+f(x_0)$ implies that there exists $\phi \in X^*$ such that $\phi \leq d^+f(x_0)$.
	\end{proof}
	
	\begin{defn} \ld{}\tbf{Subdifferential:}\\
	Let $X$ be a Banach space, $A \subset X$ open and convex, $f:A \rightarrow \R$ convex and $x_0 \in A$. We define the \tbf{subdifferential of $f$ at $x_0$}, denoted $\p f(x_0)$, to be $$\p f(x_0) = \{ \phi \in X^*: \text{for each } x \in A, f(x_0) + \phi(x-x_0) \leq f(x)\}$$
	\end{defn}
	
	\begin{ex} \lex{}
	Let $X$ be a Banach space, $A \subset X$ open and convex, $f:A \rightarrow \R$ convex and $x_0 \in A$. If $f$ is continuous at $x_0$, then $\p f(x_0) \neq \varnothing$.
	\end{ex}
	
	\begin{proof}
	Suppose that $f$ is continuous at $x_0$. The previous exercise tells us that there exists $\phi \in X^*$ such that $\phi \leq d^+f(x_0)$. Let $x \in A$. A previous exercise implies that
	\begin{align*}
	\phi(x-x_0) 
	& \leq d^+f(x_0)(x - x_0) \\
	& \leq f(x) - f(x_0)
	\end{align*}
	Then $f(x_0) + \phi(x-x_0) \leq f(x)$.
	\end{proof}
	
	\begin{ex} \lex{}
	Let $X$ be a Banach space, $A \subset X$ open and convex, $f:A \rightarrow \R$ convex, $\phi \in X^*$ and $x_0 \in A$. Then 
	\begin{enumerate}
	\item for each $x \in A$, $$\phi(x-x_0) \leq f(x) - f(x_0)$$ iff  $$\phi \leq d^+f(x_0)$$ 
	\item  $\p f(x_0) = \{ \phi \in X^*: \phi \leq d^+ f(x_0)\}$
	\end{enumerate}
	\end{ex}	
	
	\begin{proof}\
	\begin{enumerate}
	\item Suppose that for each $x \in A$, $\phi(x-x_0) \leq f(x) - f(x_0)$. Let $x \in X$. Define $t_0$ as before. Then for each $t \in (0, t_0)$, 
	\begin{align*}
	t\phi(x)
	&= \phi((x_0 + tx) - x_0) \\
	& \leq f(x_0 + tx) - f(x_0)
	\end{align*}	 
	This implies that $\phi(x) \leq d^+f(x_0)(x)$.\\
	Conversely, suppose that $\phi \leq d^+f(x_0)$. Let $x \in A$. A previous exercise implies that, 
	\begin{align*}
	\phi(x-x_0) 
	& \leq d^+f(x_0)(x-x_0) \\
	&\leq f(x) - f(x_0)
	\end{align*}
	\item Clear.
	\end{enumerate}
	\end{proof}
	
	\begin{ex} \lex{}
	Let $X$ be a Banach space, $A \subset X$ open and convex, $f:A \rightarrow \R$ convex and $x_0 \in A$. If $f$ is continuous at $x_0$, then the following are equivalent:
	\begin{enumerate}
	\item $f$ is Gateaux differentiable at $x_0$    
	\item $d^+f(x_0)$ is linear 
	\item $\# \p f(x_0) = 1$
	\end{enumerate}
	\end{ex}	
	
	\begin{proof}
	Suppose that $f$ is continuous at $x_0$. Then $d^+f(x_0)$ is Lipschitz and bounded.
	\begin{itemize}
	\item $(1) \implies (2)$: \\ 
	Suppose that $f$ is Gateaux differentiable at $x_0$. Let $x \in X$. Then a previous exercise implies that 
	\begin{align*}
	-df^+(x_0)(-x) 
	&= df^-f(x_0)(x) \\
	&= df^+f(x_0)(x)
	\end{align*}
	An exercise in the section on sublinear functionals implies that $df^+f(x_0)$ is linear.
	\item $(2) \implies (3)$: \\  
	Suppose that $df^+f(x_0)$ is linear. Let $\phi \in \p f(x_0)$. The previous exercise implies that $\phi \leq df^+f(x_0)$. Equivalence of linearity in the section on sublinear functionals implies that $d^+f(x_0) = \phi$. 
	\item $(3) \implies (1)$: \\  
	Suppose that $\# \p f(x_0) = 1$. Since $\p f(x_0) = \{ \phi \in X^*: \phi \leq d^+ f(x_0) \}$, equivalence of linearity in the section on sublinear functionals implies that $d^+ f(x_0)$ is linear. This implies that $d^+ f(x_0) = d^- f(x_0)$ and which implies that $f$ is Gateaux differentiable at $x_0$.
	\end{itemize}
	\end{proof}
	
	\begin{ex}
	Let $X$ be a Banach space, $A \subset X$ open and convex, $f,g:A \rightarrow \R$ convex, $\lam \geq 0$ and $x_0 \in A$. Then $$\p f(x_0) + \lam \p g(x_0) \subset \p[f + \lam g](x_0)$$
	\end{ex}
	
	\begin{proof}
	Let $\zeta \in \p f(x_0) + \lam \p g(x_0)$. Then there exist $\phi \in \p f(x_0)$ and $\psi \in \p g(x_0)$ such that $\zeta = \phi + \lam \psi$. A previous exercise implies that $\phi \leq d^+f(x_0)$ and $\lam \psi \leq \lam d^+g(x_0) = d^+[\lam g](x_0)$. Hence 
	\begin{align*}
	\zeta
	&= \phi + \lam \psi \\
	&\leq d^+f(x_0) + d^+[\lam g](x_0) \\
	&= d^+[f + \lam g](x_0)
	\end{align*}
	So $\zeta \in \p [f+\lam g](x_0)$
	\end{proof}
	
	\begin{ex} \lex{}
	Let $X$ be a Banach space, $A \subset X$ open and convex, $f:A \rightarrow \R$ convex and $x_0 \in A$. If $f$ is continuous at $x_0$, then $f$ has a global minimum point at $x_0$ iff $0 \in \p f(x_0)$.
	\end{ex}
	
	\begin{proof}
	Suppose that $f$ has a global minimum point at $x_0$. Let $x \in X$. Then 
	\begin{align*}
	d^+f(x_0)(x) 
	&= \lim_{t \rightarrow 0^+} \frac{f(x_0 + tx) - f(x_0)}{t} \\
	& \geq 0
	\end{align*}
	So $0 \leq df^+(x_0)$ and $0 \in \p f(x_0)$.\\
	Conversely, suppose that $0 \in \p f(x_0)$. Let $x \in A$. Then 
	\begin{align*}
	0
	& = 0(x - x_0) \\
	& \leq f(x) - f(x_0)
	\end{align*}
	So that $f(x_0) \leq f(x)$ which implies that $f$ has a global minimum point at $x_0$.
	\end{proof}
	
	\begin{ex}
	et $X$ be a Banach space, $A \subset X$ open and convex, $f:A \rightarrow \R$ convex and $x_0 \in A$. If $f$ is Frechet differentiable at $x_0$, then $\p f(x_0) = \{Df(x_0)\}$. 
	\end{ex}	
	
	\begin{proof}
	Clear.
	\end{proof}
	
	\begin{ex}
	Let $X$ be a Banach space, $A \subset X$ open and convex, $f:A \rightarrow \R$ convex and $x_0 \in A$. Suppose that $f$ is Frechet differentiable at $x_0$. If $Df(x_0) = 0$, then $f$ has a global minimum point at $x_0$. 
	\end{ex}
	
	\begin{proof}
	Suppose that $Df(x_0) = 0$. Since $\p f(x_0) = \{Df(x_0)\}$, a previous exercise implies that $f$ has a global minimum point at $x_0$. 
	\end{proof}
	
	\begin{ex}
	Let $X$ be a Banach space, $A \subset X$ open and convex, $f:A \rightarrow \R$ convex and $x_0 \in A$. Suppose that $f$ is Frechet differentiable at $x_0$. Then for each $x \in A$, $f(x) \geq f(x_0) + Df(x_0)(x - x_0)$
	\end{ex}
	
	\begin{proof}
	Since $Df(x_0) \in \p f(x_0)$, for each $x \in A$, $Df(x_0)(x - x_0) \leq f(x) - f(x_0)$.
	\end{proof}
	
	\begin{ex}
	Let $X$ be a Banach space, $A \subset X$ open and convex, $f:A \rightarrow \R$. Suppose that $f$ is Frechet differentiable. Then $f$ is convex iff for each $x_0, x \in A$, $f(x) \geq f(x_0) + Df(x_0)(x - x_0)$.
	\end{ex}
	
	\begin{proof}
	Suppose that $f$ is convex. Then the previous exercise implies that for each $x_0,x \in A$, $f(x) \geq f(x_0) + Df(x_0)(x - x_0)$. Conversely, suppose that for each $x_0,x \in A$, $f(x) \geq f(x_0) + Df(x_0)(x - x_0)$. Let $x_0, x, y \in A$. Then $f(x) \geq f(x_0) + Df(x_0)(x - x_0)$ and $f(y) \geq f(x_0) + Df(x_0)(y - x_0)$. \\
	\tbf{FINISH!!!}
	\end{proof}
	
	\begin{ex} \lex{}
	Let $X$ be a Banach space, $A \subset X$ open and convex, and $f \in C^2(A)$. Then $f$ is convex iff for each $x_0 \in A$, $D^2f(x_0)$ is positive semidefinite.\\
	\tbf{Hint:} Define $g:A \rightarrow \R$ by $g(x) = f(x) - Df(x_0)(x - x_0)$ and show $g$ is convex and use Taylor's Theorem
	\end{ex}
	
	\begin{proof}
	Suppose that $f$ is convex. Let $x_0 \in X$. Define $g:A \rightarrow \R$ by $g(x) = f(x) - Df(x_0)(x - x_0)$. Since $g$ is the sum of a convex function and an affine function, $g$ is convex. Since $f \in C^2(A)$, we have that $g \in C^2(A)$ and it is straightforward to show that for each $x \in A$, $Dg(x) = Df(x) - Df(x_0)$ and $D^2g(x) = D^2f(x)$. In particular, $Dg(x_0) = 0$. Hence $g$ has a global minimum point at $x_0$. This implies that $D^2f(x_0)$ is positive semidefinite. 
	Conversely, suppose that for each $x_0 \in A$, $D^2f(x_0)$ is positive semidefinite. Let  \\
	\tbf{FINISH!!!}
	\end{proof}
	
	\newpage 
	\section{Conjugacy}
	
	\begin{defn} \ld{}
	Let $X$ be a Banach space, $A \subset X$ and $f:A \rightarrow \R$. Define 
	\begin{enumerate}
		\item $A^* \subset X^*$ and $f^*: A^* \rightarrow \R$ 
		\item $A^{**} \subset X$ and $f^{**}:A^{**} \rightarrow \R$
	\end{enumerate}
	by 
	\begin{enumerate}
		\item $$A^* = \bigg \{\phi \in X^*: \sup_{x \in A} \bigg[ \phi(x) - f(x) \bigg] < \infty \bigg  \}$$ and $$f^*(\phi) = \sup_{x \in A} \bigg[ \phi(x) - f(x) \bigg] $$  
		\item $$A^{**} = \bigg \{x \in X: \sup_{\phi \in A^*} \bigg[ \hat{x}(\phi) - f^*(\phi) \bigg] < \infty \bigg \}$$ and $$f^{**}(x) = \sup_{\phi \in A^*} \bigg[ \hat{x}(\phi) - f^*(\phi) \bigg]$$
	\end{enumerate}
	\end{defn} 

	\begin{note}
		If $X$ is a Hilbert space, we may define $A^* \subset X$ and $f^*: A^* \rightarrow \R$ via the Riesz representation theorem by $$A^* = \bigg \{y \in X: \sup_{x \in A} \bigg[ \l y, x \r - f(x) \bigg] < \infty \bigg  \}$$ and $f^*: A^* \rightarrow \R$ and $$ f^*(y) = \sup_{x \in A} \bigg[ \l y, x \r - f(x) \bigg] $$
	\end{note}
	
	\begin{ex} \lex{}
	Let $X$ be a Banach space, $A \subset X$ and $f:A \rightarrow \R$. Then 
	\begin{enumerate}
		\item $A^*$ is convex and $f^*:A^* \rightarrow \R$ is convex and weak* lower semicontinuous.
		\item $A^{**}$ is convex and $f^{**}:A^{**} \rightarrow \R$ is convex and weakly lower semicontinuous.
	\end{enumerate} 
	\end{ex}
	
	\begin{proof} \
		\begin{enumerate}
			\item For $x \in A$, define $g_x: X^* \rightarrow \R$ by $g_x(\phi) = \hat{x}(\phi) - f(x)$. Then for each $x \in A$, $g_x$ is convex and weak* \lsc \, since it is affine and weak* continuous. \rex{91015} implies that $A^* = \{\phi \in X^*: \sup\limits_{x \in A} g_x(\phi) < \infty \}$ is convex and  $f^* = \sup\limits_{x \in A} g_x$ is convex.
			\item For $\phi \in A^*$, define $h_{\phi}: X \rightarrow \R$ by $h_{\phi}(x) = \phi(x) - f^{*}(\phi)$. Then for each $\phi \in A^*$, $h_{\phi}$ is convex and weakly \lsc \, since it is affine and weakly continuous. \rex{91015} implies that $A^{**} = \{x \in X: \sup\limits_{\phi \in A^*} h_{\phi}(x) < \infty \}$ is convex and   $f^{**} = \sup\limits_{\phi \in A^*} h_{\phi}$ is convex. 
		\end{enumerate}		 
	\end{proof}
	
	\begin{ex} \lex{}
		Let $X$ be a Banach space, $A \subset X$ and $f:A \rightarrow \R$. Then for each $x \in A$ and $\phi \in A^*$, $f^*(\phi) \geq \phi(x) - f(x)$.	
	\end{ex}
	
	\begin{proof}
	Clear by definition.
	\end{proof}
	
	\begin{ex} \lex{}
	Let $X$ be a Banach space, $A \subset X$ and $f:A \rightarrow \R$. Then $A \subset A^{**}$.
	\end{ex}

	\begin{proof}
		Let $x \in A$. Then the previous exercise implies that
		\begin{align*}
			\sup_{\phi \in A^*} [\phi(x) - f^{*}(\phi)] 
			& \leq f(x) \\
			& < \infty  
		\end{align*}
		So $x \in A^{**}$.
	\end{proof}

	\begin{ex}
		Let $X$ be a Banach space, $A \subset X$ convex, $f:A \rightarrow \R$ convex and lower semicontinuous and $x_0 \in A$. 
		\begin{enumerate}
			\item if $x_0 \in A$, then for each $\ep >0$, there exists $\phi \in A^*$ such that for each $x \in A$, $f(x) > f(x_0) + \phi(x - x_0) - \ep$
			\item if $x_0 \not \in A$, then for each $M \in \R$, there exists $\phi \in A^*$ such that for each $x \in A$, $f(x) > M + \phi(x - x_0)$
		\end{enumerate}
	\tbf{Hint:} Apply second Hahn-Banach separation theorem to $\{(x_0, f(x_0) - \ep)\}$ and $\epi f$.
	\end{ex}

	\begin{proof}\
		\begin{enumerate}
			\item Suppose that $x_0 \in A$. Let $\ep >0$. Since $f$ is convex and lower semicontinuous, $\epi f \subset X \times \R$ is convex and closed, $\{(x_0, f(x_0) - \ep)\} \subset X \times \R$ is convex and compact and $\{(x_0, f(x_0) - \ep)\} \cap \epi f = \varnothing$. Thus, there exists $\lam \in \R$, $\psi \in X^*$ and $k \in \R$ such that for each $x \in A$ and $r \geq f(x)$, 
			$$\psi(x) + \lam r < k < \psi(x_0) + \lam (f(x_0) - \ep)$$
			Taking $(x, r) = (x_0, f(x_0))$ implies that $0 < -\lam \ep $ and hence that $\lam < 0$. Set $\phi = |\lam|^{-1}\psi$. For $x \in A$, set $r = f(x)$. Then 
			\begin{align*}
				& \hspace{1.2cm} \psi(x) -|\lam| f(x) < \psi(x_0) - |\lam| (f(x_0) - \ep) \\
				& \iff |\lam|^{-1} \psi(x) - f(x) < |\lam|^{-1}\psi(x_0) - (f(x_0) - \ep) \\
				& \iff \phi(x) - f(x) < \phi(x_0) - (f(x_0) - \ep) \\
				& \iff f(x) > f(x_0) + \phi(x-x_0) - \ep 
			\end{align*}
			Since for each $x \in A$, $\phi(x) - f(x) < \phi(x_0) - f(x_0) + \ep$, we have that 
			\begin{align*}
				\sup_{a \in A}[\phi(x) - f(x)] 
				& \leq \phi(x_0) - f(x_0) + \ep \\
				&< \infty
			\end{align*}
			So $\phi \in A^*$.
			\item Suppose that $x_0 \not \in A$. Let $M \in \R$. Repeat the previous argument for $(x_0, M)$ and $\epi f$.
		\end{enumerate}
	\end{proof}

	\begin{ex}
		Let $X$ be a Banach space, $A \subset X$ convex and $f:A \rightarrow \R$ convex and lower semicontinuous. Then 
		\begin{enumerate}
			\item $A = A^{**}$
			\item $f = f^{**}$
		\end{enumerate}
	\end{ex}

	\begin{proof}\
		\begin{enumerate}
			\item A previous exercise implies that $A \subset A^{**}$. Let $x_0 \in X$. Suppose that $x_0 \not \in A$. Let $M \in \R$. The previous exercise implies that there exists $\phi_0 \in A^*$ such that for each $x \in A$, $f(x) > M + \phi_0(x - x_0)$. Then 
			\begin{align*}
				\phi_0(x_0) - f^*(\phi_0) 
				&= \phi_0(x_0) - \sup_{x \in A}[\phi_0(x) - f(x)] \\
				&= \phi_0(x_0) + \inf_{x \in A}[f(x) - \phi_0(x)] \\
				& \geq \phi_0(x_0) + (M - \phi_0(x_0)) \\
				&= M
			\end{align*}
		Therefore 
		\begin{align*}
			\sup\limits_{\phi \in A^*}[\phi(x_0) - f^*(\phi)] 
			& \geq \phi_0(x_0) - f^*(\phi_0) \\
			& \geq M
		\end{align*}
		Since $M \in \R$ is arbitrary, $$\sup\limits_{\phi \in A^*}[\phi(x_0) - f^*(\phi)] = \infty $$ and $x_0 \not \in A^{**}$. So $A^c \subset (A^{**})^c$, which implies that $A^{**} \subset A$. Thus $A^{**} = A$.
		\item Part $(1)$ and a previous exercise imply that $f^{**} \leq f$. Suppose that $f \not \leq f^{**}$. Then there exists $x_0 \in A$ such that $f(x_0) > f^{**}(x_0)$. Choose $\ep > 0$ such that $f(x_0) > f^{**}(x_0) + 2 \ep$. A previous exercise implies that there exists $\phi \in A^*$ such that for each $x \in A$, $f(x) > f(x_0) + \phi(x - x_0) - \ep$. Choose $a \in A$ such that $f^*(\phi) - \ep < \phi(a) - f(a)$. Then 
		\begin{align*}
			f(x_0)
			& > f^{**}(x_0) + 2 \ep \\
			& \geq \phi(x_0) - f^*(\phi) + 2 \ep \\
			& > \phi(x_0 -a) + f(a) + \ep \\
			& > \phi(x_0 -a) + f(x_0) + \phi(a - x_0) - \ep + \ep \\
			&= f(x_0) 
		\end{align*}
		which is a contradiction. So $f \leq f^{**}$ and hence $f = f^{**}$. 
		\end{enumerate}
		
	\end{proof}
	

	
	
	
	
	
	
	
	
	
	\begin{defn} \ld{}
	Let 
	\end{defn}
	
	\begin{defn} \ld{}
	$\partial f$
	\end{defn}	
	
	\begin{ex} \lex{}
	
	\end{ex}
	
	
	
	
	
	
	
	
	
	
	
	
	
	
	
	
	
	
	
	
	
	\newpage
	\chapter{Topological Groups}
	
	
	\section{Introduction}

	\begin{defn}
		Let $G$ be a group, we define $\mult:G \times G \rightarrow G$ and $\inv: G \rightarrow G$ by $\mult(g, h) = gh$ and $\inv(g) = g^{-1}$ respectively.
	\end{defn}
	
	\begin{defn} \ld{00000} 
		Let $G$ be a group and $\MT$ a topology on $G$. Then $(G, \MT)$ is said to be a \tbf{topological group} if $\mult:G \times G \rightarrow G$ and $\inv: G \rightarrow G$ are continuous.
	\end{defn}

	\begin{note}
		For the remainder of this chapter, measurablility is in reference to $(G, \MB(\MT))$. That is, the measurable sets are the Borel sets.
	\end{note}
	
	\begin{defn} \ld{00000} 
		Let $G$ be a topological group. We define $$\Homeo(G) = \{\phi:G \rightarrow G: \phi \text{ is a homeomorphism}\}$$
	\end{defn}
	
	\begin{note}
	Let $G$ be a topological group. Then $\Homeo(G)$ is a group.
	\end{note}
	
	\begin{ex} \lex{00000} 
		Let $G$ be a topological group. Then $\inv \in \Homeo(G)$.
	\end{ex}

	\begin{proof}
		By assumption $\inv$ is continuous. We know from basic group theory that $\inv$ is a bijection with $\inv^{-1} = \inv$. 
	\end{proof}

	\begin{defn} \ld{00000} 
		Let $G$ be a group and $S \subset G$, then $S$ is said to be \tbf{symmetric} if $\inv(S) = S$, ( i.e. $S^{-1} = S$).
	\end{defn}
	
	\begin{defn} \ld{00000} 
		Let $G$ be a topological group and $\phi:G \rightarrow G$. Then $\phi$ is said to be an \tbf{automorphism} of $G$ if $\phi$ is a homomorphism and a homeomorphism. We define $$\Aut(G) = \{\phi:G \rightarrow G: \phi \text{ is an automorphism}\}$$
	\end{defn}
	
	\begin{ex} \lex{00000} 
	Let $G$ be a topological group. Then $\inv \in \Aut(G)$ iff $G$ is abelian. 
	\end{ex}
	
	\begin{proof}
	Basic group theory tells us that $\inv$ is a homomorphism iff $G$ is abelian.
	\end{proof}
	
	\begin{defn} \ld{00000} 
		Let $G$ be a group and $g \in G$. We define the \tbf{left and right translation maps}, denoted $l_g:G \rightarrow G$ and $r_g:G \rightarrow G$ respectively, by $l_g(x) = gx$ and $r_g(x) = xg^{-1}$. 
	\end{defn}
	
	\begin{ex} \lex{00000} 
		Let $G$ be a topological group and $g \in G$. Then $l_g, r_g \in \Homeo(G)$.
	\end{ex}
	
	\begin{proof}
		By assumption $l_g$ and $r_g$ are continuous. We know from basic group theory that $l_g$ and $r_g$ are bijections with $l_g^{-1} = l_{g^{-1}}$ and $r_g^{-1} = r_{g^{-1}}$ so $l_g$ and $r_g$. are homeomorphisms. 
	\end{proof}
	
	\begin{ex} \lex{00000} 
	Let $G$ be a toplogical group. Define $\phi, \psi:G \rightarrow \Homeo(G)$ by $\phi(g) = l_g$ and $\psi(g) = r_g$. Then $\phi, \psi$ are homomorphisms.
	\end{ex}
	
	\begin{proof}
	Let $g_1, g_2 \in G$. Then $$l_{g_1} \circ l_{g_2}(x) = l_{g_1}(g_2 x) = g_1 g_2 x= l_{g_1 g_2}(x)$$ and $$r_{g_1} \circ r_{g_2} (x) = r_{g_1}(x g_2^{-1})= xg_2^{-1}g_1^{-1} = x(g_1g_2)^{-1} = r_{g_1g_2}(x)$$ 
	\end{proof}
	
	\begin{ex} \lex{00000} 
		Let $G$ be a topological group. Then for each $U \subset G$ and $g \in G$, if $U$ is open, then $gU$, $Ug$ and $U^{-1}$ are open. 
	\end{ex}
	\begin{proof}
		Let $U \subset G$ and $g \in G$. Suppose that $U$ is open. Since $l_g, r_g$ and $\inv$ are homeomorphisms, $l_g(U) = gU$, $r_g(U) = Ug$ and $\inv(U) = U^{-1}$ are open. 
	\end{proof}
	
	\begin{defn} \ld{00000} 
		Let $G$ be a topological group, $y \in G$ and $f \in L^0$.  Define $L_y, R_y: L^0(G) \rightarrow L^0(G)$ by $L_y f = f \circ l_y^{-1}$ and $R_y f = f \circ r_y^{-1}$, that is, $L_yf(x) = f(y^{-1}x)$ and $R_yf(x) = f(xy)$.
	\end{defn}
	
	\begin{ex} \lex{00000} 
	Let $G$ be a topological group and $y \in G$. Then $L_y, R_y \in \Sym(L^0(G))$. 
	\end{ex}
	
	\begin{proof}
	It is straight forward to show that $L_y^{-1} = L_{y^{-1}}$ and $R_y^{-1} = R_{y^{-1}}$. 
	\end{proof}
	
	\begin{ex} \lex{00000} 
	Let $G$ be a topological group. Define $\phi, \psi: G \rightarrow \Sym(L^0(G))$ by $\phi(y) = L_y$ and $\psi(y) = R_y$. Then $\phi$ and $\psi$ are homomorphisms.
	\end{ex}

	\begin{proof}
		Let $y,z \in G$ and $f \in L^0(G)$. Then 
		\begin{align*}
			L_y \circ L_z(f)
			& = L_y (L_z (f))  \\
			& = L_y (f \circ l_z^{-1})  \\
			& = (f \circ l_z^{-1}) \circ l_y^{-1} \\
			&= f \circ (l_z^{-1} \circ l_y^{-1}) \\
			& = f \circ (l_y \circ l_z)^{-1}  \\
			& = f \circ l_{yz}^{-1} \\
			&= L_{yz} (f)
		\end{align*}
		
		The case is similar for $R_y$ and $ R_z$.
	\end{proof}
	
	\begin{ex} \lex{00000} 
		Let G be a topological group, $U \in \MB(G)$ and $y \in G$. Then $L_y\chi_U = \chi_{yU}$ and $R_y\chi_U = \chi_{Uy^{-1}}$. 
	\end{ex}
	
	\begin{proof}
		Let $x \in G$. Then 
		\begin{align*}
			L_y\chi_U(x) = 1
			& \iff y^{-1}x \in U\\
			& \iff x \in yU \\
			& \iff \chi_{yU}(x) = 1
		\end{align*}
		The case is similar for $R_y$
	\end{proof}
	
	\begin{ex} \lex{00000} 
		Let G be a topological group, $y \in G$ and $f \in L^0(G)$. Then $\supp(L_yf) = y\supp(f)$ and $\supp(R_yf) = \supp(f)y^{-1}$
	\end{ex}
	
	\begin{proof}
		Put $A = \{x \in G: L_yf(x) \neq 0 \}$ and $B = \{x \in G: f(x) \neq 0 \}$. Then 
		\begin{align*}
			x \in A
			& \iff L_yf(x) \neq 0 \\
			& \iff f(y^{-1}x) \neq 0 \\
			& \iff y^{-1}x \in B \\
			& \iff x \in yB
		\end{align*}
		Thus $A = yB$ which implies that $\cl A = y \cl B$. Therefore $\supp(L_yf) = y\supp(f)$.
	\end{proof}
	
	\begin{ex} \lex{00000} 
		Let $G$ be a topological group and $y \in G$. Then $L_y, R_y$ are linear and if we restrict to the bounded measurable functions, then  $L_y, R_y \in L(B(G))$ and $\|L_y\|, \|R_y\| = 1$. 
	\end{ex}
	
	\begin{proof}
		Let $f, g \in L^0(G)$ and $\lam \in \C$. Then 
		\begin{align*}
			L_y(\lam f+g)(x)
			& = (\lam f+g)(y^{-1}x) \\
			& = \lam f(y^{-1}x) + g(y^{-1}x) \\
			& = \lam L_yf(x) + L_yg(x)
		\end{align*}
		So $L_y$ is linear. Next, we restrict to $B(G) \cap L^0$. We note that $$\{|f(y^{-1}x)|: x \in y\supp(f)\} = \{|f(x)|: x \in \supp(f)\}$$ This implies that 
		\begin{align*}
			\|L_yf \|_u 
			& = \sup_{x \in \supp(L_yf)} |L_yf(x)| \\
			& = \sup_{x \in y\supp(f)} |f(y^{-1}x)| \\
			& = \sup_{x \in \supp(f)} |f(x)| \\ 
			& = \|f\|_u
		\end{align*} 
		So $L_y$ is bounded. Hence $L_y \in L(L^0)$. The case is similar for $R_y$.
	\end{proof}
	
	\begin{defn} \ld{00000} 
		Let $G$ be a topological group. We say that $G$ is a \tbf{locally compact group} if $G$ is locally compact and Hausdorff.
	\end{defn}






























	\newpage
	\section{Group Actions}
	
	\subsection{Introduction}
	
	\begin{note}
		Let $X,Y,X$ be sets. We recall that for $f:X \times Y \rightarrow Z$, $a \in X$ and $b \in Y$, the maps $f_a:Y \rightarrow Z$ and $f^b:X \rightarrow Z$ are defined by 
		$$f_a(y) = f(a,y) \quad f^b(x) = f(x, b)$$
	\end{note}
	
	\begin{defn}
		Let $\MC$ a concrete category with products, $G, X \in \Obj(C)$ and $\phi \in \Hom_{\MC}( G \times X,  X)$. Suppose that $G$ is a group. Then $\phi$ is said to be a \tbf{group action} of $G$ on $X$ if 
		\begin{enumerate}
			\item for each $x \in X$, $\phi_e = \id_X$ 
			\item for each $g, h \in G$, $\phi_{gh} = \phi_g \circ \phi_h$
		\end{enumerate}
	\end{defn}

	\begin{note}
		When the context is clear, we will write $g \cdot x$ in place of $\phi(g, x)$. 
	\end{note}

	\begin{ex}
		Let $\MC$ a category with products, $G, X \in \Obj(C)$ and $\phi \in \Hom_{\MC}( G \times X,  X)$. Suppose that $G$ is a group and $\phi$ group action. Then for each $g \in G$, $\phi_g \in \Aut(X)$.
	\end{ex}

	\begin{proof}
		Let $g \in G$. Then 
		\begin{align*}
			\phi_g \circ \phi_{g^{-1}}(x) 
			& = \phi_g (\phi_{g^{-1}}(x)) \\
			& = g \cdot (g^{-1} \cdot x) \\
			& = (g g^{-1}) \cdot x \\
			& = e \cdot x \\
			& = x 
		\end{align*} 
		Since $x \in X$ is arbitrary, $\phi_g \circ \phi_{g^{-1}} = \id_X$. Similarly, $\phi_{g^{-1}} \circ \phi_g = \id_X$. Hence $\phi_g \in \Aut(X)$.
	\end{proof}

	\begin{defn}
		Let $\MC$ a category with products, $G, X \in \Obj(C)$ and $\phi \in \Hom_{\MC}( G \times X,  X)$. Suppose that $G$ is a group and $\phi$ group action. We define $\hat{\phi}: G \rightarrow \Aut(X)$ by $\hat{\phi}(g) = \phi_g$.
	\end{defn}

	\begin{ex}
		Let $\MC$ a category with products, $G, X \in \Obj(C)$ and $\phi \in \Hom_{\MC}( G \times X,  X)$. Suppose that $G$ is a group and $\phi$ group action. Then $\hat{\phi}: G \rightarrow \Aut(X)$ is a group homomorphism.
	\end{ex}

	\begin{proof}
		Clear by definition.
	\end{proof}

























	\subsection{Homogeneous Spaces}
	
	\begin{defn}
		Let $G$ be a topological group, $X$ a topological space and $\phi: G \times X \rightarrow X$ a continous group action. Then $(X, \phi)$ is said to be a \tbf{homogeneous $G$-space} if
		\begin{itemize}
			\item $\phi$ is transitive
			\item for each $x \in X$, $\phi_x : G \rightarrow X$ is open
		\end{itemize} 
	\end{defn}

	\begin{defn}
		Let $G$ be a topological group, $H < G$. We define $\phi_H: H \times G \rightarrow G$ by $\phi(h, g) = g h^{-1}$.
	\end{defn}

	\begin{ex}
		
	\end{ex}

	\begin{ex}
		Let $G$ be a topological group, $H < G$ a closed subgroup of $G$. Then $(G/H, \phi_H)$
	\end{ex}
	
	\begin{ex}
		Let $G$ be a topological group, $H < G$ a closed subgroup of $G$ and $(X, \phi)$ a homogeneous $G$-space. 
	\end{ex}
	
	
	
	
	
	
	\subsection{Common Examples}
		\begin{ex}
		Let $H$ be a Hilbert space and $x, y \in H$. Then $\| x \| = \|y\|$ iff there exists $U \in U(H)$ such that $x = Uy$.   
	\end{ex}
	
	\begin{proof}\
		\begin{itemize}
			\item ($\implies$) :
			Suppose that $\| x \| = \|y\|$. An exercise 
			\item ($\impliedby$): 
		\end{itemize}
	\end{proof}
	
	\begin{ex}
		Let $H$ be a Hilbert space. Then 
		\begin{enumerate}
			\item $\|\cdot\|: H \rightarrow \Rg$ is a quotient map
			\item $H / U(H)$ is homeomorphic to $\Rg$ 
		\end{enumerate} 
	\end{ex}
	
	\begin{proof}
		content...
	\end{proof}
	
	\begin{defn}
		Let $n,k \in \N$. Suppose that $n \geq k$.  We define the \tbf{Stiefel manifold}, denoted $V_k(\R^n)$, by 
		$$V_k(\R^n) = \{A \in \R^{n \times k}: A^* A = I\}$$ 
		We define the \tbf{orthogonal matrices}, denoted by $O(n)$, by 
		$$O(n) = V_n(\R^n)$$ 
	\end{defn}
	
	\begin{note}
		We note that for each $X \in V_k(\R^n)$, $\rnk X = k$ and for each $U \in O(n)$, $UU^* = I$.
	\end{note}
	
	\begin{ex}
		Let $X, Y \in \R^{n \times k}$. Suppose that $\rnk X = k$ and $\rnk Y = k$. Then $XX^* = YY^*$ iff there exists $U \in O(k)$ such that $X = YU$. \\
		\tbf{Hint:} $\rnk X = \rnk X^*X$.
	\end{ex}
	
	\begin{proof}\
		\begin{itemize}
			\item $(\implies)$: \\
			Suppose that $XX^* = YY^*$. Since $\rnk X = k$, we have that  
			\begin{align*}
				\rnk XX^* 
				& = \rnk X \\
				& = k
			\end{align*}
			Since $X^*X \in \R^{k \times k}$, $X^*X$ is invertible. Hence 
			\begin{align*}
				X
				& = XI \\
				& = X(X^*X)(X^*X)^{-1} \\
				& = (XX^*) X (X^*X)^{-1} \\
				& = (YY^*) X (X^*X)^{-1} \\
				& = Y(Y^*X)(X^*X)^{-1}
			\end{align*}
			Set $U =(Y^*X)(X^*X)^{-1}$. Then $X = YU$ and  
			\begin{align*}
				U^*U
				& = \bigg( (Y^*X)(X^*X)^{-1} \bigg)^* (Y^*X)(X^*X)^{-1} \\
				& = (X^*X)^{-1} (X^* Y) (Y^*X) (X^*X)^{-1} \\
				& = (X^*X)^{-1} X^* (YY^*)X (X^*X)^{-1} \\
				& = (X^*X)^{-1} X^* (XX^*)X (X^*X)^{-1} \\
				& = (X^*X)^{-1} (X^* X) (X^*X) (X^*X)^{-1} \\
				& = I
			\end{align*}
			Thus $U \in O(k)$.
			\item $(\impliedby)$: \\
			Suppose that there exists $U \in O(k)$ such that $X = YU$. Then 
			\begin{align*}
				XX^*
				& = (YU) (YU)^* \\
				& = (YU) (U^* Y^*)  \\
				& =  Y (UU^*) Y^* \\
				& = Y I Y^* \\
				& = YY^*
			\end{align*}
		\end{itemize}
	\end{proof}
	
	\begin{ex}
		Define $f: V$
	\end{ex}
	
	
	
	
	
	
	
	
	
	
	
	
	
	
	
	
	
	
	
	
	
	
	
	
	
	
	
		\section{Quotient Groups}
	\begin{defn}
		Let 
	\end{defn}
	
	
	
	
	
	
	
	
	
	
	
	
	
	
	
	
	
	
	
	
	
	\newpage
	\section{Automorphism Groups of Metric Spaces}
	
	\begin{defn} \ld{}
	Let $(X, \tau)$ be a topological space. Define $$\Aut(X) = \{\sig:X\rightarrow X: \sig \text{ is a homeomorphism} \}$$ 
	\end{defn}	
	
	\begin{ex} \lex{}
	Let $(X, d)$ be a compact metric space. Then $(\Aut(X), d_{u} )$ is a topological group.
	\end{ex}
	
	\begin{proof}
	Let $(\sig_n)_{n \in \N}, (\tau_n)_{n \in \N} \subset \Aut(X)$ and $\sig,\tau \in \Aut(X)$. Suppose that $\sig_n \convt{u} \sig$ and $\tau_n \convt{u} \tau$.
	\begin{enumerate}
	\item Let $\ep >0$. Since $X$ is compact and $\sig$ is continuous, $\sig$ is uniformly continuous. Then there exists $\del >0$ such that for each $x, y \in X$, $d(x,y) < \del$ implies that $d(\sig(x), \sig(y)) \leq \ep/2$.  Choose $N_\sig \in \N$ such that for each $n \in \N$, $ n \geq \N$ implies that $d_u(\sig_n, \sig) < \ep/2$. Choose $N_\tau \in \N$ such that for each $n \in \N$, $ n \geq \N$ implies that $d_u(\tau_n, \tau) < \del$. Put $N = \max(N_\sig, N_\tau)$. Let $n \in \N$ and $x \in X$. Suppose that $n \geq N$. Then 
	\begin{align*}
		d(\sig_n \circ \tau_n (x) ,\sig \circ \tau (x) ) 
		&\leq  d(\sig_n(\tau_n(x)),  \sig(\tau_n(x))) + d( \sig(\tau_n (x)), \sig( \tau (x))) \\
		& < \ep / 2 +\ep / 2 \\
		&= \ep 
	\end{align*}
	So $d_u(\sig_n \circ \tau_n, \sig\circ \tau) \leq \ep$ and $\circ: \Aut(X)^2 \rightarrow \Aut(X)$ is continuous. 
	\item Suppose that $\sig = \id_X$. Let $\ep >0$. Then there exists $N \in \N$ such that for each $n \in \N$, $n \geq N$ implies that $d_u(\sig_n, \id_X) < \ep$. Let $n \in \N$. Suppose that $n \geq N$. Then 
	\begin{align*}
	\sup_{x \in X} d(\sig^{-1}_n(x), x) 
	&= \sup_{x \in \sig_n(X)}d(\sig^{-1}_n(x), x) \\
	&= \sup_{x \in X}d(\sig^{-1}_n(\sig_n(x)), \sig_n(x)) \\
	&= \sup_{x \in X}d(x, \sig_n(x)) \\
	&< \ep
	\end{align*}
	So $\sig^{-1}_n \convt{u} \id_X$. Now suppose that $\sig \neq \id_X$. Since $\sig_n \convt{u} \sig$, part $(1)$ implies that $\sig^{-1} \circ \sig_n \convt{u} \id_X$. Applying the result from above, we get that $\sig_n^{-1} \circ \sig \convt{u} \id_X$. Applying part $(1)$ again implies that $\sig_n^{-1}  \convt{u}  \sig^{-1}$. So the map $\sig \mapsto \sig^{-1}$ is continuous. 
	\end{enumerate}
	Hence $\Aut(X)$ is a topological group. 
	\end{proof}
	
	\begin{defn} \ld{}
	Let $(X, d)$ be a metric space. Define 
	$$\Aut(X, d) = \{\sig:X\rightarrow X: \sig \text{ is an isometric isomorphism} \}$$  
	\end{defn}
	
	\begin{ex} \lex{}
	Let $(X, d)$ be a compact metric space. Then $(\Aut(X, d), d_u)$ is a compact subgroup of $(\Aut(X), d_u)$.
	\end{ex}
	
	\begin{proof}
	Clearly, $(\Aut(X, d), d_u)$ is a topological subgroup. To show compactness, use the Arzela Ascoli theorem.
	\end{proof}
	
	\begin{defn} \ld{}
	Let $(X, \tau)$ be a topological space and $\mu: \MB(X) \rightarrow \R$ a Borel measure. Define $$\Aut(X, \mu) = \{\sig \in \Aut(X): \sig_* \mu = \mu\}$$ 
	\end{defn}	
	
	\begin{ex} \lex{}
	Let $(X,d)$ be a compact metric space and $\mu: \MB(X) \rightarrow \R$ an outer-regular Borel measure. Then $\Aut(X, \mu)$ is a closed subgroup of $\Aut(X)$.
	\end{ex}
	
	\begin{proof}
	It is clear that $\Aut(X, \mu)$ is a subgroup of $\Aut(X)$. Let $(\sig_n)_{n \in \N} \subset \Aut(X, \MB(X), \mu)$ and $\sig \in \Aut(X)$. Suppose that $\sig_n \convt{u} \sig$. Let $E \subset X$ be closed, $U \subset X$ open and suppose that $E \subset U$. An exercise in the section on metric spaces tells us that there exists $N \in \N$ such that for each $n \in \N$, $n \geq N$ implies that $\sig(E) \subset \sig_n(U)$. Then 
	\begin{align*}
	\mu(\sig(E)) 
	&\leq \mu(\sig_N(U)) \\
	&= \mu(U) 
	\end{align*}
	Therefore, since $\mu$ is outer regular, $\mu(\sig(E)) \leq \mu(E)$. Since $\sig_n^{-1} \convt{u} \sig^{-1}$, we may apply the above argument to obtain that 
	\begin{align*}
	\mu(E) 
	&= \mu(\sig^{-1}(\sig (E))) \\
	&\leq  \mu(\sig(E))
\end{align*}	 
Hence $\mu(E) = \mu(\sig(E))$. Applying the whole argument above thus far to $\sig^{-1}$, we see that $\mu(E) = \mu(\sig^{-1}(E))$. Since $E \subset X$ is an arbitrary closed set and $\MB(X) = \sig(E \subset X: E \text{ is closed})$, we have that $\mu = \sig_*\mu$. Thus $\sig \in \Aut(X, \mu)$ which implies that $\Aut(X, \mu)$ is closed. 
	\end{proof}
	
	\begin{defn} \ld{}
	Let $(X,d)$ be a compact metric space and $\mu: \MB(X) \rightarrow \R$ an outer-regular Borel measure. Define $\Aut(X, d, \mu) = \Aut(X, d) \cap \Aut(X, \mu)$.
	\end{defn}
	
	\begin{ex} \lex{}
	Let $(X,d)$ be a compact metric space and $\mu: \MB(X) \rightarrow \R$ an outer-regular Borel measure. Then $\Aut(X, d, \mu)$ is compact.
	\end{ex}
	
	\begin{proof}
	Since $\Aut(X, d)$ is compact and $\Aut(X, \mu)$ is closed, $\Aut(X, d, \mu)$ is compact.
	\end{proof}
	
	
	
	
	
	
	
	
	
	
	
	
	
	

	
	
	
	
	
	
	
	
	
	
	\newpage
	\chapter{Group Actions}
	
	\section{Introduction}
	\begin{note}
	For a set $X$, a group $G$ and a (left) group action $\phi: G \times X \rightarrow X$, we will write $\phi(g, x)$ as $g \cdot x$.
	\end{note}	
	
	\begin{defn} \ld{00000} 
		Let $X$ be a set, $G$ a group, $\phi: G \times X \rightarrow X$ a group action and $g \in G$. Define $l_g:X \rightarrow X$ by 
		\begin{equation*}
		l_g(x) = g \cdot x
		\end{equation*}
	\end{defn}
	
	\begin{defn}
	Let $X$ be a topological space, $G$ a group and $\phi: G \times X \rightarrow X$ a group action. Then $\phi$ is said to be $X$-continuous if for each $g \in G$, $\phi_g$ is continuous.
	\end{defn}
	
	\begin{ex}
	Let $X$ be a topological space, $G$ a group and $\phi: G \times X \rightarrow X$ an $X$-continuous group action. Then for each $g \in G$, $\phi_g \in \Homeo(X)$.
	\end{ex}
	
	\begin{proof}
	Let $g \in G$, then $\phi_g$ and $\phi_{g}^{-1} = \phi_{g^{-1}}$ are continuous, so $\phi_g \in \Homeo(G)$. 
	\end{proof}
	
	\begin{defn} \ld{}
	Let $(X, d)$ be a metric space, $G$ a group, and $\phi: G \times X \rightarrow X$ a group action. Then $\phi$ is said to be an \tbf{isometric group action} if for each $g \in G$, $\phi_g:X \rightarrow X$ is an isometry. 
	\end{defn}
	
	\begin{ex}
	Let $(X, d)$ be a metric space, $G$ a group, and $\phi: G \times X \rightarrow X$ an isometric group action. Then $\phi$ is $X$-continuous.
	\end{ex}
	
	\begin{proof}
	Clear since isometries are continuous.
	\end{proof}		
	
	\begin{defn}
		Let $X$ be a set, $G$ a group, $\phi: G \times X \rightarrow X$ a group action. We define the relation $\sim_{\phi} \subset X \times X$ by 
		$$\sim_{\phi} = \{(a, b) \in X \times X: \text{ there exists $g \in G: a = g \cdot b$} \}$$
	\end{defn}

	\begin{ex}
		Let $X$ be a set, $G$ a group, $\phi: G \times X \rightarrow X$ a group action. Then $\sim_{\phi}$ is an equivalence relation on $X$.
	\end{ex}

	\begin{proof} Let $a, b, c \in X$.
		\begin{itemize}
			\item \tbf{(reflexivity): } Then $a = e \cdot a$. Hence $a \sim_{\phi} a$.
			\item \tbf{(symmetry): }Suppose that $a \sim_{\phi} b$. Then there exists $g \in G$ such that $a = g \cdot b$. Hence $b = g^{-1} \cdot a$. Thus $b \sim_{\phi} a$.
			\item \tbf{(transitivity): }Suppose that $a \sim_{\phi} b$ and $b \sim_{\phi} c$. Then there exist $g, h \in G$ such that $a = g \cdot b$ and $b = h \cdot c$. Then 
			\begin{align*}
				a
				& = g \cdot b \\
				& = g \cdot (h \cdot c) \\
				& = (gh) \cdot c 
			\end{align*}
			Hence $a \sim_{\phi} c$.
		\end{itemize}
	\end{proof}

	\begin{defn}
		Let $X$ be a set, $G$ a group and $\phi: G \times X \rightarrow X$ a group action. We define the \tbf{quotient of $X$ by $G$}, denoted $X / G$, by 
		$$X/ G = X / \sim_{\phi}$$
		We denote the projection from $X$ onto $X/G$ by $\pi: X \rightarrow X/ G$.
	\end{defn}
	
	\begin{defn}
	Let $X$ be a set, $G$ a group, $\phi: G \times X \rightarrow X$ a group action and $f:X \rightarrow \C$. Then $f$ is said to be \tbf{$\phi$-invariant} if for each $g \in G$ and $x \in X$, $f(g \cdot x) = f(x)$.
	\end{defn}
	
	\begin{ex}
	Let $X$ be a set, $G$ a group, $\phi: G \times X \rightarrow X$ a group action and $f:X \rightarrow \C$. Then $f$ is $\phi$-invariant iff $f$ is $\sim_{\phi}$-invariant.  
	\end{ex}
	
	\begin{proof}\
		\begin{itemize}
		\item $(\implies):$ \\ 
		Suppose that $f$ is $\phi$-invariant. Let $a,b \in X$. Suppose that $a \sim_{\phi} b$. Then there exists $g \in G$ such that $a = g \cdot b$. Since $f$ is $\phi$-invariant, 
		\begin{align*}
			f(a)
			& = f(g \cdot b) \\
			& = f(b)		
		\end{align*}
		Since $a,b \in X$ such that $a \sim_{\phi} b$ are arbitrary, we have that $f$ is $\sim_{\phi}$-invariant.
 		\item $(\impliedby):$ \\
 		Suppose that $f$ is $\sim_{\phi}$-invariant. Let  $g \in G$ and $x \in X$. By definition, $x \sim_{\phi} g \cdot x$. Since $f$ is $\sim_{\phi}$-invariant, $f(g \cdot x) = f(x)$. Since $g \in G$ and $x \in X$ are arbitrary, $f$ is $\phi$-invariant.
		\end{itemize}
	\end{proof}
	
	\begin{ex}
		Let $X, Y$ be a topological spaces, $G$ a topological group, $\phi: G \times X \rightarrow X$ a continuous group action and $f:X \rightarrow Y$ a homeomorphism.
	\end{ex}

	
	
	
	\tcr{maybe introduce the unitary matrices, the stiefel manifold, the grassman manifold here, maybe introduce in geometry notes}
	
	
	
	
	
	
	
	
	
	
	
	
	
	
	
	
	
	
	
	
	
	
	
	
	
	
	
	
	
	
	
	
	
	
	
	
	
	
	
	
	\newpage
	\section{Group Actions on Metric Spaces}
	
	\begin{note}
	This section establishes the criteria for the existence of a metric on the orbit space of a metric space under a group action. 
	\end{note}
	
	\begin{defn} \ld{}
	Let $(X, d)$ be a metric space, $G$ a group, and $\phi: G \times X \rightarrow X$ a group action. We define 
	$\bar{d}: X/G \times X / G \rightarrow \Rg$ by 
	$$\bar{d}(\bar{x}, \bar{y}) = \inf_{\substack{a \in \bar{x} \\ b \in \bar{y}}} d(a,b) $$
	\end{defn}
	
	\begin{ex} \lex{}
	Let $(X, d)$ be a metric space, $G$ a group, and $\phi: G \times X \rightarrow X$ an isometric group action. Then for each $x, y \in X$, $$\bar{d}(\bar{x}, \bar{y}) = \inf_{g \in G} d(g \cdot x, y)$$
	\end{ex}
	
	\begin{proof}
	Let $x, y \in X$, $a \in \bar{x}$ and $b \in \bar{y}$. Then there exists there exists $g_a, g_b \in G$ such that $a = g_a \cdot x$ and $b = g_b \cdot y$. Set $g = g_b^{-1}g_a$. Since the map $z \mapsto g_b^{-1} \cdot z$ is an isometry, 
	\begin{align*}
	d(a,b) 
	&= d(g_a \cdot x, g_b \cdot y) \\
	&= d(g_b^{-1}g_a \cdot x, y)\\
	&= d(g\cdot x, y)
	\end{align*}
	Let $\ep >0$. Then there exist $a^* \in \bar{x}$ and $b^* \in \bar{y}$ such that $d(a^*,b^*) < \bar{d}(\bar{x},\bar{y}) + \ep$. The above argument implies that that there exists $g^* \in G$ such that 
	\begin{align*} 
	\inf_{g \in G} d(g \cdot x, y) 
	& \leq d(g^* \cdot x, y) \\
	&= d(a^*, b^*) \\
	& < \bar{d}(\bar{x}, \bar{y}) + \ep
\end{align*}	 
	Since $\ep >0$ is arbitrary, $$\inf_{g \in G} d(g \cdot x, y) \leq \bar{d}(\bar{x}, \bar{y})$$
	Conversely, since $\{(g \cdot x, y): g \in G\} \subset \{(a,b): a \in \bar{x}, b \in \bar{y}\}$, we have that 
	$$\inf_{g \in G} d(g \cdot x, y) \geq \bar{d}(\bar{x}, \bar{y})$$ 
	\end{proof}
	
	\begin{ex} \lex{}
	Let $(X, d)$ be a metric space, $G$ a group, and $\phi: G \times X \rightarrow X$ an isometric group action. Then for each $x, y, z \in X$, $$\bar{d}(\bar{x}, \bar{y}) \leq \bar{d}(\bar{x}, \bar{z}) + \bar{d}(\bar{z}, \bar{y})$$
	\end{ex}
	
	\begin{proof}
	Let $x, y, z \in X$. An exercise in section $(2.1)$ implies that $d(\bar{x}, \bar{y}) \leq d(\bar{x}, z) + d(z, \bar{y})$. The previous exercise implies that 
	\begin{align*}
	d(\bar{x}, z) 
	&= \inf_{a \in \bar{x}} d(a, z) \\
	&= \inf_{g \in G} d(g \cdot x, z) \\
	&= \bar{d}(\bar{x}, \bar{z})
	\end{align*}
	Similarly, $d(z, \bar{y}) = \bar{d}(\bar{z}, \bar{y})$. Then 
	\begin{align*}
	d(\bar{x}, \bar{y}) 
	&\leq d(\bar{x}, z) + d(z, \bar{y}) \\
	&= \bar{d}(\bar{x}, \bar{z}) + \bar{d}(\bar{z}, \bar{y})
	\end{align*}
	\end{proof}
	
	\begin{ex} \lex{}
	Let $(X, d)$ be a metric space, $G$ a group, and $\phi: G \times X \rightarrow X$ an isometric group action. If for each $x \in X$, $\bar{x}$ is closed, then for each $x, y \in X$, $\bar{d}(\bar{x}, \bar{y}) =0$ implies that $\bar{x} = \bar{y}$.
	\end{ex}
	
	\begin{proof}
	Suppose that for each $x \in X$, $\bar{x}$ is closed. Let $x,y \in X$. Suppose that $\bar{d}(\bar{x} , \bar{y}) = 0$. Then $\inf\limits_{ g \in G} d(g \cdot x, y) = 0$. Hence there exists $(g_n)_{n \in N} \subset G$ such that $g_n \cdot x \rightarrow y$. Since $(g_n \cdot x)_{n \in \N} \subset \bar{x}$ and $\bar{x}$ is closed, $y \in \bar{x}$. Thus $\bar{x} = \bar{y}$. 
	\end{proof}
	
	\begin{ex} \lex{}
	Let $(X, d)$ be a metric space, $G$ a group, and $\phi: G \times X \rightarrow X$ an isometric group action. If for each $x \in X$, $\bar{x}$ is closed, then $\bar{d}$ is a metric on $X/G$.
	\end{ex}
	
	\begin{proof}
	Clear by preceeding exercises.
	\end{proof}
	
	\begin{ex} \lex{}
	Let $(X, d)$ be a metric space, $(G, \MT_G)$ a topological group, and $\phi: G \times X \rightarrow X$ an isometric group action. Suppose that $G$ is compact and for each $x \in X$, the map $g \mapsto g \cdot x$ is $(\MT_G, \MT_d)$-continuous. Then $\bar{d}$ is a metric on $X/G$. 
	\end{ex}
	
	\begin{proof}
	Let $x \in X$. Since $G$ is compact and the map $g \mapsto g \cdot x$ is $(\MT_G, \MT_d)$-continuous, $\bar{x} = G \cdot x$ is compact and therefore closed. The previous exercise implies that $\bar{d}$ is a metric.
	\end{proof}
	
	\begin{ex} \lex{}
	Let $(X, d)$ be a metric space, $G$ a group, and $\phi: G \times X \rightarrow X$ an isometric group action. Suppose that $\bar{d}$ is a metric on $X/G$. Then the projection map $\pi: X \rightarrow X/G, $ is $(d, \bar{d})$-Lipschitz and therefore $(\MT_d, \MT_{\bar{d}})$-continuous.
	\end{ex}
	
	\begin{proof}
	Let $x,y \in X$. Then
	\begin{align*}
	\bar{d}(\pi(x), \pi(y)) 
	&= \bar{d}(\bar{x}, \bar{y}) \\
	&= \inf_{g \in G} d(g \cdot x, y)\\
	& \leq d(x,y)  \\
	\end{align*}
	\end{proof}
	
	\begin{ex} \lex{}
	Let $(X, d)$ be a metric space, $G$ a group, and $\phi: G \times X \rightarrow X$ an isometric group action. Suppose that $\bar{d}$ is a metric on $X/G$. Let $(x_n)_{n \in \N} \subset X$ and $x \in X$. Then $\bar{x}_n \conv{\bar{d}} \bar{x}$ iff there exists a sequence $(g_n)_{n \in \N}$ such that $g_n \cdot x_n \conv{d} x$.
	\end{ex}
	
	\begin{proof} 
	Suppose that $\bar{x}_n \conv{\bar{d}} \bar{x}$. For $n \in \N$, choose $g_n \in G$ such that $d(g_n \cdot x_n, x) < \bar{d}(\bar{x}_n, \bar{x}) + 2^{-n}$. Then $d(g_n \cdot x_n, x) \rightarrow 0$ and $g_n \cdot x_n \conv{d} x$.  \\
	Conversely, suppose that that there exists a sequence $(g_n)_{n \in \N}$ such that $g_n \cdot x_n \conv{d} x$. Since $\pi:  X \rightarrow X/G$ is $(\MT_{d}, \MT_{\bar{d}})$-continuous, we have that
	\begin{align*}
	g_n \cdot x_n \conv{d} x
	& \implies \pi(g_n \cdot x_n) \conv{\bar{d}} \pi(x)\\
	& \implies \bar{x}_n  \conv{\bar{d}} \bar{x}
	\end{align*}
	\end{proof}		
	
	\begin{ex} \lex{}
	Let $X$ be a set, $d_1, d_2: X^2 \rightarrow \Rg$ metrics, $G$ a group and $\phi: G \times X \rightarrow X$ an isometric group action. Suppose that $d_1$ and $d_2$ are $\Top$-equivalent. 
	\begin{enumerate}
	\item Then $\bar{d}_1$ is a metric on $X/G$ iff $\bar{d}_2$ is a metric on $X/G$
	\item If $\bar{d}_1$ and $\bar{d}_2$ are metrics, then $\bar{d}_1$ and $\bar{d}_2$ are $\Top$-equivalent. 
	\end{enumerate}
	\end{ex}
	
	\begin{proof}\
	\begin{enumerate}
	\item 
	\begin{itemize}
	\item $\implies$ Suppose that $\bar{d}_1$ is a metric. Let $x,y \in X$. Suppose that $\bar{d}_2(\bar{x}, \bar{y}) = 0$. Then there exist $(g_n)_{n \in \N} \subset G$ such that $d_2(g_n \cdot x, y) \rightarrow 0$. Since $d_1$ and $d_2$ are $\Top$-equivalent, $d_1(g_n \cdot x, y) \rightarrow 0$. Thus $\bar{d}_1(\bar{x}, \bar{y}) = 0$. Since $\bar{d}_1$ is a metric, $\bar{x} = \bar{y}$. Hence $\bar{d}_2$ is a metric. 
	\item $\impliedby$ Similar.
	\end{itemize}
	\item Suppose that $\bar{d}_1$ and $\bar{d}_2$ are metrics. Let $(\bar{x}_n)_{n \in \N} \subset X/G$ and $\bar{x} \in X/G$. 
	\begin{itemize}
	\item Suppose that $\bar{x}_n \conv{\bar{d}_1} \bar{x}$. Then there exists a sequence $(g_n)_{n \in \N}$ such that $g_n \cdot x_n \conv{d_1} x$. Since $d_1$ and $d_2$ are $\Top$-equivalent, $g_n \cdot x_n \conv{d_2} x$. This implies that $\bar{x}_n \conv{\bar{d}_2} \bar{x}$. 
	\item Suppose that $\bar{x}_n \conv{\bar{d}_2} \bar{x}$. Then similarly to above, $\bar{x}_n \conv{\bar{d}_1} \bar{x}$.
	\end{itemize}
	\end{enumerate}
	\end{proof}	
	
	\begin{ex} \lex{}
	Let $X$ be a set, $d_1, d_2: X^2 \rightarrow \Rg$ metrics on $X$, $G$ a group and $\phi: G \times X \rightarrow X$ an isometric group action. Suppose that $d_1$ and $d_2$ are equivalent. If $\bar{d}_1$ and $\bar{d}_2$ are metrics, then $\bar{d}_1$ and $\bar{d}_2$ are equivalent.
	\end{ex}
	
	\begin{proof} Suppose that $\bar{d}_1$ and $\bar{d}_2$  are metrics. Since $d_1$ $d_2$ are equivalent, there exist $C_1, C_2 >0$ such that for each $x,y \in X$, $C_1d_1(x,y) \leq d_2(x,y) \leq C_2d_1(x,y)$. Let $x,y \in X$. Then
	\begin{align*}
	C_1\bar{d}_1(\bar{x}, \bar{y}) 
	&= C_1 \inf_{g \in G} d_1(g \cdot x, y) \\
	&=  \inf_{g \in G} C_1 d_1(g \cdot x, y) \\
	&\leq \inf_{g \in G} d_2(g \cdot x, y) \\
	&= \bar{d}_2(\bar{x}, \bar{y}) \\
	\end{align*}	 
	and 
	\begin{align*}
	\bar{d}_2(\bar{x}, \bar{y}) 
	&= \inf_{g \in G} d_2(g \cdot x, y) \\	
	& \leq \inf_{g \in G} C_2 d_1(g \cdot x, y) \\
	&= C_2 \inf_{g \in G}  d_1(g \cdot x, y) \\
	&= C_2 \bar{d}_1(\bar{x}, \bar{y})
	\end{align*}
	So that $C_1 \bar{d}_1 \leq \bar{d}_2 \leq C_2 \bar{d}_1$
	\end{proof}
	
	\begin{ex}
	Let $(X,d)$ be a metric space, $G$ a group and $\phi: G \times X \rightarrow X$ an isometric group action. Suppose that $\bar{d}$ is a metric. Then $\pi: X \rightarrow X/G$ is a $(\MT_{d}, \MT_{\bar{d}})$-quotient map.
	\end{ex}
	
	\begin{proof}\
	\begin{itemize}
	\item Clearly $\pi$ is surjective. 
	\item Let $C \subset X/G$. Suppose that $C$ is closed. Since $\pi$ is $(\MT_{d}, \MT_{\bar{d}})$-continuous, if $\pi^{-1}(C)$ is closed. \\
	Conversely, suppose that $\pi^{-1}(C)$ is closed. Let $(\bar{x}_n)_{n \in \N} \subset C$ and $\bar{x} \in X/G$. Suppose that $\bar{x}_n \conv{\bar{d}} \bar{x}$. Then there exists $(g_n)_{\al \in A} \subset G$ such that $g_n \cdot x_n \conv{d} x$. Since $(g_n \cdot x_n)_{n \in \N} \subset \pi^{-1}(C)$, $x \in \pi^{-1}(C)$. Hence $\bar{x} \in C$ and $C$ is closed. Then \rex{ex:quotient_topology:0002} implies that $\pi$ is a $(\MT_{d}, \MT_{\bar{d}})$-quotient map.
	\end{itemize}
	\end{proof}
	
	\begin{ex}
	Let $(X,d)$ be a metric space, $G$ a group and $\phi: G \times X \rightarrow X$ an isometric group action. Suppose that $\bar{d}$ is a metric. Then $\pi: X \rightarrow X/G$ is $(\MT_{d}, \MT_{\bar{d}})$-open.
	\end{ex}
	
	\begin{proof}
	Let $U \subset X$. Suppose that $U$ is open. Then 
	\begin{equation*}
	\pi^{-1}(\pi(U)) = \bigcup_{g \in G} g \cdot U
	\end{equation*}		
	Since for each $g \in G$, $\phi_g$ is an isometry and thus a homeomorphism, we have that for each $g \in G$, $g \cdot U$ is open. Therefore 
	$$\pi^{-1}(\pi(U)) = \bigcup\limits_{g \in G} g \cdot U$$
	is open. \rex{ex:quotient_topology:0008} implies that $\pi$ is open.
	\end{proof}

	\begin{ex}
		Let $(X,d)$ be a metric space, $G$ a group and $\phi: G \times X \rightarrow X$ an isometric group action. Suppose that $\bar{d}$ is a metric. Then $\bar{\pi}:{X/G} \rightarrow {X/G}$ is a $(\MT_{X/G}, \MT_{\bar{d}})$-homeomorphism.
	\end{ex}

	\begin{proof}
		The previous exercises imply that $\pi: X \rightarrow X/G$ is a $(\MT_d, \MT_{\bar{d}})$-quotient map and $(\MT_d, \MT_{\bar{d}})$-open. Since for each $a,b \in X$, $a \sim b$ iff $\pi(a) = \pi(b)$, \rex{36012} implies that $\bar{\pi}:{X/G} \rightarrow {X/G}$ is a $(\MT_{X/G}, \MT_{\bar{d}})$-homeomorphism.
	\end{proof}
	
	\begin{ex}
	Let $(X,d)$ be a metric space, $G$ a group and $\phi: G \times X \rightarrow X$ an isometric group action. Suppose that $\bar{d}$ is a metric. Then $\bar{d}$ metrizes the quotient topology $\pi_*\MT_d$ on $X/G$.
	\end{ex}
	
	\begin{proof}
	Immediate by the previous exercise.
	\end{proof}
	
	
	
	
	
	
	
	
	
	
	
	
	
	
	
	
	
	
	
	
	
	
	

	
	
	
	
	
	
	
	
	
	
	\newpage
	\section{Fundamental Examples}
	\begin{note}
	This section uses results from the previous two sections to establish metrics on some fundamental orbit spaces of metric spaces under a group action. 
	\end{note} 
		
	
	\begin{ex} \lex{} \tbf{Procrustes Distance:} \\
	Consider the metric space $(\C^{n \times d}, \|\cdot\|_F)$, topological group $(U_d, \|\cdot\|_F)$ and  the (right) action $\phi: X \times U_d \rightarrow X$ by $X \cdot U = XU$. Then 
	\begin{enumerate}
	\item $\phi$ is a continuous isometric group action 
	\item $U_d$ is compact 
	\item $\bar{d}$ is a metric on $\C^{n \times d}/ U_d$
	\end{enumerate}
	\end{ex}
	
	\begin{proof}
	Clear.
	\end{proof}		
	
	\begin{ex} \lex{}
	Let $X$ be a compact metric space and $\mu:\MB(X) \rightarrow \RG$ a Borel measure. Define the (right) group action $\phi: L^1(\mu) \times \Aut(X, \mu) \rightarrow L^1(\mu) $ by $$f \cdot \sig = f \circ \sig$$ Then $\phi$ is an isometric group action. 
	\end{ex}
	
	\begin{proof}
	Let $\sig \in \Aut(X, \mu)$ and $f \in L^1(\mu)$. 
Then 
	\begin{align*}
	\|f \cdot \sig\|_1
	&=  \int_X |f \circ \sig| d\mu \\
	&=  \int_X |f| \circ \sig d\mu \\
	&=  \int_{\sig(X)} |f| d \sig_* \mu  \\
	&=  \int_{\sig(X)} |f| d \mu \\
	&=  \int_{X} |f| d \mu \\
	&= \|f\|_1 
	\end{align*}	 
	\end{proof}
	
	\begin{ex} \lex{}
	Let $X$ be a compact metric space and $\mu:\MB(X) \rightarrow \RG$ a Radon measure. Define the (right) group action $\phi: L^1(\mu) \times \Aut(X, \mu) \rightarrow L^1(\mu) $ by 
	$$f \cdot \sig = f \circ \sig$$	
	Then for each $f \in L^1(\mu)$, the map $\sig \mapsto  f \cdot \sig$ is continuous.  
	\end{ex}
	
	\begin{proof}
	Let $f \in L^1(\mu)$, $(\sig_n)_{n \in \N} \subset \Aut(X, \mu)$ and $\sig \in \Aut(X, \mu)$. Suppose that $\sig_n \convt{u} \sig$. Since $\mu$ is Radon, $C_c(X)$ is dense in $L^1(\mu)$ and therefore, there exists $\phi \in C_c(X)$ such that $\|\phi - f\| < \ep/3$. Since $X$ is compact and $\mu$ is Radon, $\mu(X) < \infty$. Since $\phi$ is uniformly continuous, \rex{211111111} implies that $\phi \circ \sig_n \convt{u} \phi \circ \sig$.  So there exists $N \in \N$ such that for each $n \in \N$, $n \geq N$ implies that $\|\phi \circ \sig_n - \phi \circ \sig\|_u < \frac{\ep}{3 (\mu(X)+1)}$. Let $n \in \N$. Suppose that $n \geq \N$. Then 
	\begin{align*}
	\|f \circ \sig_n - f \circ \sig\|_1 
	&\leq \|f \circ \sig_n - \phi \circ \sig_n \|_1 + \|\phi \circ \sig_n - \phi \circ \sig\|_1 + \|\phi \circ \sig - f \circ \sig\|_1 \\
	& = \|(f - \phi) \circ \sig_n \|_1 + \|\phi \circ \sig_n - \phi \circ \sig\|_1 + \|(\phi - f) \circ \sig\|_1 \\
	&= \|f - \phi  \|_1 + \|\phi \circ \sig_n - \phi \circ \sig\|_1 + \|\phi - f \|_1 \\
	&= \|f - \phi  \|_1 + \|\phi \circ \sig_n - \phi \circ \sig\|_u \mu(X) + \|\phi - f \|_1 \\
	&< \frac{\ep}{3} + \frac{\ep}{3} + \frac{\ep}{3} \\
	&= \ep
	\end{align*}
	So that $f \circ \sig_n \convt{u} f \circ \sig$ which implies that the map $\sig \mapsto  f \cdot \sig$ is continuous. 
	\end{proof}
	
	\begin{ex} \lex{} \tbf{Cut Distance:} \\
	Let $X$ be a compact metric space and $\mu:\MB(X) \rightarrow \RG$ a Radon measure. Define the (right) group action $\phi: L^1(\mu) \times \Aut(X, \mu) \rightarrow L^1(\mu) $ by 
	$$f \cdot \sig = f \circ \sig$$	
	Then 
	\begin{enumerate}
	\item $\phi$ is an isometric group action 
	\item $\Aut(X, d, \mu)$ is compact 
	\item for each $f \in L^1(\mu)$, the map $\sig \mapsto  f \cdot \sig$ is continuous.  
	\item $\bar{d}$ is a metric on $L^1(\mu) / \Aut(X, d, \mu)$
	\end{enumerate}
	\end{ex}
	
	\begin{proof}
	Clear by the preceeding exercises.
	\end{proof}
	
	\begin{note}
	The preceeding distance is not quite the Cut distance, as the Cut norm only considers a subset of measurable sets for a function of two variables, but with some work, maybe I can show it is a distance.
	\end{note}














































	\newpage
	\section{Optimization over Compact Groups}
	
	\begin{defn}
		We define $W \cdot U \defeq W U$ and $W \trl \Sig \defeq W^* \Sig W$. Define $S(n) \defeq \{\Sig \in \C^{n \times n}: \Sig^* = \Sig\}$. Define $U(n,k) \defeq \{U \in \C^{n \times k}: U^*U = I\}$. 
	\end{defn}
	
	\begin{ex}
		Let $n, k \in \N$. Suppose that $k \leq n$. Define $f: U(n, k) \times S(n) \rightarrow \R$ and $F: S(H) \rightarrow \R$ by $f(U, \Sig) \defeq \tr (UU^* \Sig)$ and $F(\Sig) \defeq \sup\limits_{U \in U(n, k)} f(U, \Sig)$. Let $\Sig \in S(n)$. Let $\Sig \in S(n)$. Suppose that there exists $V \in U(n)$ and $\Lam \in S(n)$ such that $\Lam$ is diagonal and $\Sig = V \Lam V^*$. Then
		\begin{enumerate}
			\item for each $W \in U(n)$ and $U \in U(n, k)$, $f(W \cdot U, \Sig) = f(U, W \trl \Sig)$,
			\item $F(\Sig) = F(\Lam)$, 
			\item Suppose that $\Lam = \diag(\lam_1, \cdots, \lam_n)$ and $\lam_ \geq \cdots \geq \lam_n \geq 0$. Then $\argmin\limits_{U \in U(n,k)} f(U, \Sig) = VI_{n, k}$. 
		\end{enumerate}
	\end{ex}

	\begin{proof}\
		\begin{enumerate}
			\item Let $W \in U(n)$ and $U \in U(n, k)$. Then 
			\begin{align*}
				f(W \cdot U, \Sig)
				& = \tr (WUU^* W^* \Sig) \\
				& = \tr (U^* W^* \Sig W U ) \\
				& = \tr (U^* (W \trl \Sig) U) \\
				& = \tr (U^* U(W \trl \Sig)) \\
				& = f(U, W \trl \Sig).
			\end{align*}
			\item The previous part implies that for each $W \in U(n)$ and $U \in U(n,k)$, 
			\begin{align*}
				f(U, \Sig)
				& = f(U, V^* \trl \Lam) \\
				& = f(V^* \cdot U, \Lam).
			\end{align*}
			Therefore
			\begin{align*}
				F(\Sig) 
				& = \sup\limits_{U \in U(n, k)} f(U, \Sig) \\
				& = \sup\limits_{U \in U(n, k)} f(V^* \cdot U, \Lam) \\
				& = \sup\limits_{U \in U(n, k)} f(U, \Lam) \\
				& = F(\Lam).
			\end{align*}
			\item 
			Let $U \in U(n,k)$. For each $j \in [k]$, we define $u_j \defeq Ue_j$. We know that
			\begin{align*}
				\tr(UU^* \Lam)
				& = \tr(U^* \Lam U) \\
				& = \sum_{j=1}^k \l u_j , \Lam u_j \r \\
				& = \sum_{j=1}^k \lam_j \l u_j , u_j \r \\
				& = \sum_{j=1}^k \lam_j \| u_j \|^2 \\
				& = \sum_{j=1}^k \lam_j.
			\end{align*}
			Therefore
			\begin{align*}
				F(\Sig)
				& = F(\Lam) \\
				& = \sup\limits_{U \in U(n,k)} \tr(U^*U \Lam) \\
				& = \sup\limits_{U \in U(n,k)} \sum_{j=1}^k \lam_j \\
				& = \sum_{j=1}^k \lam_j.
			\end{align*}
			Since 
			\begin{align*}
				f(VI_{n,k}, \Sig)
				& = \tr(VI_{n,k}(VI_{n,k})^* \Sig) \\
				& = \tr((VI_{n,k})^* \Sig VI_{n,k}) \\
				& = \tr(I_{k,n} V^* \Sig V I_{n,k}) \\
				& = \tr(I_{k,n} V^* V  \Lam V^* V I_{n,k}) \\
				& = \tr(I_{k,n} \Lam I_{n,k}) \\
				& = \sum_{j=1}^k \lam_j,
			\end{align*}
			we have that $\argmin\limits_{U \in U(n,k)} f(U, \Sig) = VI_{n,k}$. 
		\end{enumerate}
	\end{proof}























	
	
	
	
	
	
	
	
	
	
	
	
	\appendix
	
	\chapter{Summation}
	
	\begin{defn} \ld{}
		Let $f:X \rightarrow \Rg$, Then we define $$\sum_{x \in X} f(x) := \sup_{\substack{F \subset X \\ F \text{ finite}}} \sum_{x \in F} f(x)$$ This definition coincides with the usual notion of summation when $X$ is countable. For $f:X \rightarrow \C$, we can write $f = g +ih$ where $g,h:X \rightarrow \R$. If $$\sum_{x \in X}|f(x)| < \infty,$$ then the same is true for $g^+,g^-,h^+,h^-$. In this case, we may define $$\sum_{x \in X} f(x)$$ in the obvious way.
	\end{defn} 
	
	The following note justifies the notation $\sum_{x \in X}f(x)$ where $f:X \rightarrow \C$.
	
	\begin{note}
		Let $f:X \rightarrow \C$ and $\al:X \rightarrow X$ a bijection. If $\sum_{x \in X}|f(x)|< \infty$, then $\sum_{x \in X}f( \al (x)) = \sum_{x \in X}f(x) $.
	\end{note}
	
	\newpage	
	
	\chapter{Asymptotic Notation}
	
	\begin{defn} \ld{}
	Let $X$ be a topological space, $Y, Z$ be normed vector spaces, $f:X \rightarrow Y$, $g: X \rightarrow Z$ and $x_0 \in X \cup \{\infty\}$. Then we write $$f = o(g) \hspace{.5cm} \text{ as } x \rightarrow x_0$$ if for each $\ep >0$, there exists $U \in \MN(x_0)$ such that for each $x \in U$, $$\|f(x)\| \leq \ep\|g(x)\|$$
	\end{defn}
	
	\begin{ex} \lex{}
	Let $X$ be a topological space, $Y, Z$ be normed vector spaces, $f:X \rightarrow Y$, $g: X \rightarrow Z$ and $x_0 \in X \cup \{\infty\}$. If there exists $U \in \MN(x_0)$ such that for each $x \in U \setminus \{x_0\}$, $g(x) > 0$, then $$f = o(g) \text{ as } x \rightarrow x_0 \hspace{.25cm} \text{ iff } \hspace{.25cm}  \lim_{x \rightarrow x_0} \frac{\| f(x) \|}{\| g(x) \|} = 0$$
	\end{ex}	
	
	\begin{ex} \lex{}
	Let $X$ and $Y$ a be normed vector spaces, $A \subset X$ open and $f:A \rightarrow Y$. Suppose that $0 \in A$. If $f(h) = o(\|h\|)$ as $h \rightarrow 0$, then for each $h \in X$,  $f(th) = o(|t|)$ as $t \rightarrow 0$.
	\end{ex}	
	
	\begin{proof}
	Suppose that $f(h) = o(\|h\|)$ as $h \rightarrow 0$.  Let $h \in X$ and $\ep >0$. Choose $\del' >0 $ such that for each $h' \in B(0, \del')$, $h' \in A$ and 
	$$\|f(h')\| \leq \frac{\ep}{\|h\|+1} \|h'\|$$ 
	Choose $\del >0$ such that for each $t \in B(0,\del)$, $th \in B(0,\del')$. Let $t \in B(0,\del)$. Then 
	\begin{align*}
	\|f(th)\| 
	&\leq \frac{\ep}{\|h\|+1} |t|\|h\| \\
	&< \ep |t|
	\end{align*}
	So $f(th) = o(|t|)$ as $t \rightarrow 0$.
	\end{proof}		
	
	
	
	
	\begin{defn} \ld{}
	Let $X$ be a topological space, $Y, Z$ be normed vector spaces, $f:X \rightarrow Y$, $g: X \rightarrow Z$ and $x_0 \in X \cup \{\infty\}$. Then we write $$f = O(g) \hspace{.5cm} \text{ as } x \rightarrow x_0$$ if there exists $U \in \MN(x_0)$ and $M \geq 0$ such that for each $x \in U$, $$\|f(x)\| \leq M\|g(x)\|$$
	\end{defn}
	
	
	
	
	
	
	
	
	
	
	
	
	
	
	
	
	
	
	
	
	
	
	
	
	
	
	
	\newpage
	\chapter{Categories}
	
	\tcr{move to notation?}
	
	\begin{defn}
		We define the category of topological measure spaces, denoted $\TopMsrpos$, by 
		\begin{itemize}
			\item $\Obj(\TopMsrpos) \defeq \{(X, \mu): X \in \Obj(\Top) \text{ and } \mu \in M(X)\}$			
			\item $\Hom_{\TopMsrpos}((X, \mu), (Y, \nu)) \defeq \Hom_{\Top}(X, Y) \cap \Hom_{\Msrpos}((X, \MB(X), \mu), (Y, \MB(Y), \nu))$
		\end{itemize}
	\end{defn}
	
































	






	\newpage
	\chapter{Rings}
	
	\section{Introduction}
	
	\begin{defn}
		Let $R$ be a ring and $I, J \subset R$ ideals. We define 
		\begin{itemize}
			\item $I+J \subset R$ by $I + J \defeq \{i + j: i \in I \text{ and } j \in J\}$,
			\item $IJ \subset R$ by $IJ \defeq \{i_1j_1 + i_2j_2: i_1,i_2 \in I \text{ and } j_1, j_2 \in J\}$.
		\end{itemize}
	\end{defn}

	\begin{ex}
		Let $R$ be a ring and $I, J \subset R$ ideals. Then $I + J$ and $IJ$ are ideals.
	\end{ex}

	\begin{proof}
		\tcr{FINISH!!!}
	\end{proof}
	
	\section{Quotient Rings}
	
	\begin{ex}
		Let $R$ be a commutative unital ring and $I \subset R$ and ideal. Then $I$ is maximal iff $R/I$ is a field. 
	\end{ex}

	\begin{proof}
		\tcr{A previous exercise} implies that $R/I$ is a commutative unital ring. 
		\begin{itemize}
			\item $(\implies):$ \\
			Suppose that $I$ is maximal. Let $k+I \in R/I$. Suppose that $k+I \neq 0+I$. Then $k \not \in I$. Define $J \subset R$ by $J \defeq (k) + I$. \tcr{A previous exercise} implies that $J$ is an ideal. By construction $I \subset J$ and $k \in J$. Hence $I \neq J$. Since $I$ is maximal, $J = R$. Since $1 \in J$, there exists $r \in R$ and $i \in I$ such that $1 = rk + i$. Hence 
			\begin{align*}
				1 +I 
				& = rk + i +I \\
				& = rk + I \\
				& = (r + I)(k+I).
			\end{align*} 
			Thus $(k+I)$ is invertible. Since $k +I \in R/I$ such that $k+I \neq 0+I$ is arbitrary, we have that for each $k+I \in R/I$, $k+I \neq 0 + I$ implies that $k+I$ is invertible. Thus $R/I$ is a field.
			\item $(\impliedby):$ \\
			Suppose that $R/I$ is a field. Let $J \subset R$. Suppose that $J$ is an ideal, $I \subset J$ and $I \neq J$. Then there exists $j \in J$ such that $j \not \in I$. Thus $j+I \neq 0+I$. Since $R/I$ is a field, $j+I$ is invertible. Hence there exists $k \in R$ such that $jk + I = 1+I$. Hence
			\begin{align*}
				1 - jk 
				& \in I \\
				& \subset J.
			\end{align*}
			Since $J$ is an ideal, $jk \in J$ and therefore
			\begin{align*}
				1
				& = (1 - jk) + jk \\
				& \in J.
			\end{align*}
			Since $1 \in J$, $J = R$. Since $J \subset R$ with $J$ an ideal and $I \subset J$ is arbitrary, we have that for each $J \subset R$, $J$ an ideal and $I \subset J$ implies that $J = I$ or $J = X$. Hence $I$ is maximal. 
		\end{itemize}
	\end{proof}
	
	
	
	
	
	
	
	
	
	
	
	
	
	
	
	
	
	
	
	
	
	
	
	
	
	
	
	
	
	
	\newpage
	\chapter{Vector Spaces}
	\tcr{it might be better to cover some category theory and write everything in terms of $\Hom_{\VectK}$ and $\Obj(\VectK)$}
	
	\section{Introduction}
	
	\begin{defn}
		Let $X$ be a set, $\K$ a field, $+:X \times X \rightarrow X$ and $\cdot:\K \times X \rightarrow X$. Then $(X, +, \cdot)$ is said to be a \tbf{$\K$-vector space} if 
		\begin{enumerate}
			\item $(X, +)$ is an abelian group
			\item 
		\end{enumerate} 
	\end{defn}

	\begin{defn}
		Let $(X, +, \cdot) \in \Obj(\VectC)$. We define the \tbf{conjugate of $(X, +, \cdot)$}, denoted $(\bar{H}, \bar{+}, \bar{\cdot})$, by 
		\begin{itemize}
			\item $\bar{X} \defeq X$,
			\item $x \bar{+} y \defeq x + y$,
			\item $\lam \bar{\cdot} x \defeq \lam^* \cdot x$.
		\end{itemize}
	\end{defn}

	\begin{ex}
		Let $(X, +, \cdot) \in \Obj(\VectC)$. Then $(\bar{H}, \bar{+}, \bar{\cdot}) \in \Obj(\Vect_{\C})$.
	\end{ex}

	\begin{proof}
		\tcr{FINISH!!!}
	\end{proof}


	\begin{defn}
		Let $(X, +_X, \cdot_X)$ and $(E, +_E, \cdot_E)$ be vector spaces. Suppose that $E \subset X$. Then $(E, +_E, \cdot_E)$ is said to be a subspace of $X$ if 
		\begin{enumerate}
			\item $+_E = +_X|_{E \times E}$
			\item $\cdot_E = \cdot_X|_{\K \times E}$
		\end{enumerate}
	\end{defn}

	\begin{ex}
		Let $(X, +_X, \cdot_X)$ and $(E, +_E, \cdot_E)$ be vector spaces. Suppose that $E \subset X$. 
	\end{ex}
	
	\begin{ex}
		Let $(X, +, \cdot)$ be a vector space and $E \subset X$. Then $E$ is a subspace of $X$
	\end{ex}
	
	
	\begin{defn}
		Let $X$ be a vector space and $(E_j)_{j \in J}$ a collection of subspaces of $X$. Then $\bigcap\limits_{j \in J}E_j$ is a subspace of $X$. 
	\end{defn}

	\begin{proof}
		Set $E \defeq \bigcap\limits_{j \in J}E_j$. Let $x,y \in E$ and $\lam \in \K$. Then for each $j \in J$, $x,y \in E_j$. Since for each $j \in J$, $E_j$ is a subspace of $X$, we have that for each $j \in J$, $x+ \lam y \in E_j$. Thus $x+\lam y \in E$. Since $x,y \in E$ and $\lam \in \K$ are arbitrary, \tcr{(cite exercise here)} we have that $E$ is a subspace of $X$. 
	\end{proof}
	
	
	
	
	
	
	
	
	
	
	
	
	
	
	
	
	
	
	
	
	
	
	
	
	
	
	\begin{defn}
		Let $X, Y$ be vector spaces and $T:X \rightarrow Y$. Then $T$ is said to be \tbf{linear} if for each $x_1, x_2 \in X$ and $\lam \in \Lam$, 
		\begin{enumerate}
			\item $T(x_1 + x_2) = T(x_1) + T(x_2)$,
			\item $T(\lam x_1) = \lam T(x_1)$.
		\end{enumerate}
		We define $\ML(X;Y) \defeq \{T:X \rightarrow Y: \text{ $T$ is linear}\}$. 
	\end{defn}
	
	\begin{ex}
		Let $X,Y$ be vector spaces and $T : X \rightarrow Y$. Then $T$ is linear iff for each $x_1, x_2 \in X$ and $\lam \in \Lam$, 
		$$T(x_1 + \lam x_2) = T(x_1) + \lam T(x_2)$$
	\end{ex}
	
	\begin{proof}
		Clear. \tcr{(add details)}
	\end{proof}

	\begin{ex}
		$\ML(X;Y)$ is a vector space \tcr{FINISH!!!}
	\end{ex}

	\begin{proof}
		content...
	\end{proof}
	
	\begin{defn}
		\tcr{define addition/scalar multiplication of linear maps}
	\end{defn}

	\begin{ex}
		Let $X,Y$ be vector spaces over $\K$. Then $\ML(X;Y)$ is a $\K$-vector space. 
	\end{ex}

	\begin{proof}
		Clear
	\end{proof}

	
	\begin{ex}
	Let $X,Y, Z$ be vector spaces over $\K$, $T \in \ML(X.Y)$ and $S \in \ML(Y, Z)$. Then $S \circ T \in \ML(X, Z)$. 
	\end{ex}

	\begin{proof}
		\tcr{FINISH!!!}
	\end{proof}

	
	
	
	
	
	
	
	
	
	
	
	
	
	
	
	
	
	
	
	
	
	
	
	
	
	
	
	
	
	
	\section{Bases}

	\begin{defn}
		Let $X$ be a vector space and $(e_{\al})_{\al \in A} \subset X$. Then $(e_{\al})_{\al \in A}$ is said to be
		\begin{itemize}
			\item \tbf{linearly independent} if for each $(\al_j)_{j=1}^n \subset A$, $(\lam_j)_{j=1}^n \subset \K$, $\sum\limits_{j=1}^n \lam_j e_{\al_j} = 0$ implies that for each $j \in [n]$, $\lam_j = 0$.  
			\item a \tbf{Hamel basis for $X$} if $(e_{\al})_{\al \in A}$ is linearly independent and $\spn (e_{\al})_{\al \in A} = X$. 
		\end{itemize}
	\end{defn}
	
	\begin{ex}
		\tcr{every vector space has a Hamel basis}
	\end{ex}

	\begin{proof}
		
	\end{proof}

	\begin{ex}
		
	\end{ex}


	\begin{ex}
		Let $X$ be a $\K$-vector space and $x \in X$. Then $x = 0$ iff for each $\phi \in X^*$, $\phi(x) = 0$. 
	\end{ex}

	\begin{proof}\
		\begin{itemize}
			\item $(\implies):$ \\
			Suppose that $x = 0$. Linearity implies that for each $\phi \in X^*$ $\phi(x) = 0$. 
			\item $(\impliedby):$ \\
			Conversely, suppose that $x \neq 0$. Define $\ep_x: \spn(x) \rightarrow \K$ by $\ep_x(\lam x) \defeq \lam$. Let $u,v \in \spn(x)$. Then there exists $\lam_u, \lam_v \in \K$ such that $u = \lam_u x$ and $v = \lam_v x$. Suppose that $u = v$. Then 
			\begin{align*}
				(\lam_u - \lam_v)x
				& = \lam_u x - \lam_v x \\
				& = u - v \\
				& = 0
			\end{align*}
			Since $x \neq 0$, we have that $\lam_u - \lam_v = 0$ and therefore $\lam_u = \lam_v$. Hence  
			\begin{align*}
				\lam_u 
				& = \ep_x(u) \\
				& = \ep_x(v) \\
				& = \lam_v.
			\end{align*}
			Thus $\ep_x$ is well defined. 
		\end{itemize}
	\end{proof}

























































\newpage 
\section{Duality}

\begin{defn} \ld{55001}\
	Let $X$ be a vector space over $\K$. We define the \tbf{algebraic dual space of $X$}, denoted $X'$, by $X' \defeq \ML(X;\K)$. 
\end{defn}


\begin{ex}
	Let $X$ be a vector space. Then $X'$ is a vector space. 
\end{ex}

\begin{proof}
	Clear.
\end{proof}


\begin{defn}
	Let $X, Y$ be vector spaces over $\K$. We define the algebraic adjoint of $T \in \ML(X;Y)$, denoted $T' \in \ML(Y', X')$, by $T'(\phi) \defeq \phi \circ T$. 
\end{defn}















































	\newpage
	\section{Multilinear Maps}

	\begin{defn}
		Let $X_1, \cdots, X_n, Y$ be vector spaces and $T: \prod\limits_{j=1}^n X_j \rightarrow \K$. Then $T$ is said to be \tbf{multilinear} if for each $j_0 \in [n]$ and $(x_j)_{j=1}^n \in \prod\limits_{j=1}^n X_j$, $T(x_1, \ldots, x_{j_0 - 1}, \cdot, x_{j_0 + 1})$ is linear. $$L^n(X_1, \dots, X_n; Y) = \bigg\{T : \prod\limits_{j=1}^n X_j \rightarrow Y: T \text{ is multilinear}\bigg \}$$ 
		If $X_1 = \cdots = X_n = X$, we write $L^n(X;Y)$ in place of $L^n (X, \dots, X; Y) $. 
	\end{defn}

	\begin{defn}
		\tcr{define addition and scalar mult of multilinear maps}
	\end{defn}

	\begin{ex}
		Let $X_1, \cdots, X_n, Y$ be vector spaces. Then $L^n(X_1, \ldots, X_n;Y)$ is a $\K$-vector space.
	\end{ex}

	\begin{proof}
		content...
	\end{proof}

	\begin{ex}
		Let $X_1, \cdots, X_n, Y, Z$ be $\K$-vector spaces, $\al \in L^n(X_1, \ldots, X_n;Y)$ and $\phi \in L^1(Y;Z)$. Then $\phi \circ \al \in L^n(X_1, \ldots, X_n; Z)$. 
	\end{ex}

	\begin{proof}
		Let $(x_j)_{j=1}^n \in \prod\limits_{j=1}^n X_j$ and $j_0 \in [n]$. Define $f:X_{j_0} \rightarrow Y$ by 
		$$f(a) \defeq \al(x_1, \ldots, x_{j_0-1}, a , x_{j_0+1}, \ldots, x_n) $$
		Since $\al \in L^n(X_1, \ldots, X_n;Y)$, $f$ is linear. Since $\phi$ is linear, and $\phi \circ f$ is linear. Since $(x_j)_{j=1}^n \in \prod\limits_{j=1}^n X_j$ and $j_0 \in [n]$ are arbitrary, we have that $\phi \circ \al \in L^n(X_1, \ldots, X_n;Y)$. 
	\end{proof}

	








































	\newpage
	\section{Tensor Products}
	
	\begin{defn}
		Let $X, Y$ and $T$ be vector spaces over $\K$ and $\al \in L^2(X, Y; T)$. Then $(T, \al)$ is said to be a \tbf{tensor product of $X$ and $Y$} if for each vector space $Z$ and $\be \in L^2(X, Y; Z)$, there exists a unique $\phi \in L^1(T;Z)$ such that $\phi \circ \al = \be$, i.e. the following diagram commutes:
		\[ 
		\begin{tikzcd}
			X \times Y \arrow[r, "\al"] \arrow[dr, "\be"'] 	
			& T  \arrow[d, dashed, "\phi"] \\
			& Z 
		\end{tikzcd}
		\] 
	\end{defn}
	
	\begin{ex}
		Let $X, Y, S, T$ be vector spaces, $\al \in L^2(X, Y; S)$ and $\be \in L^2(X, Y; T)$. Suppose that $(S, \al)$ and $(T, \be)$ are tensor products of $X$ and $Y$. Then $S$ and $T$ are isomorphic. 
	\end{ex}
	
	\begin{proof}
		Since $(T, \be)$ is a tensor product of $X$ and $Y$, $\be \in L^2(X,Y; T)$ there exists a unique $f \in L^1(T;T)$ such that $f \circ \be = \be$, i.e. the following diagram commutes: 
		\[ 
		\begin{tikzcd}
			& T \arrow[dd, dashed, "f"] \\
			X \times Y \arrow[ur, "\be"] \arrow[dr, "\be"'] 
			&   \\
			& T 
		\end{tikzcd}
		\] 
		Since $\id_T \in L^1(T;T)$ and $\id_T \circ \be = \be$, we have that $f = \id_T$. Since $(S, \al)$ is a tensor product of $X$ and $Y$, there exists a unique $\phi: S \rightarrow T$ such that $\phi \circ \al = \be$, i.e. the following diagram commutes: 
		\[ 
		\begin{tikzcd}
			X \times Y \arrow[r, "\al"] \arrow[dr, "\be"'] 	
			& S  \arrow[d, dashed, "\phi"] \\
			& T 
		\end{tikzcd}
		\] 
		Similarly, since $(T, \be)$ is a tensor product of $X$ and $Y$, there exists a unique $\psi: T \rightarrow S$ such that $\psi \circ \be = \al$, i.e. the following diagram commutes: 
		\[ 
		\begin{tikzcd}
			X \times Y \arrow[r, "\be"] \arrow[dr, "\al"'] 	
			& T  \arrow[d, dashed, "\psi"] \\
			& S 
		\end{tikzcd}
		\] 
		Therefore 
		\begin{align*}
			(\phi \circ \psi) \circ \be 
			& = \phi \circ (\psi \circ \be) \\
			& = \phi \circ \al \\
			& = \be, 
		\end{align*}
		i.e. the following diagram commutes:
		\[ 
		\begin{tikzcd}
			& T \arrow[d, dashed, "\psi"] \\
			X \times Y \arrow[ur, "\be"] \arrow[dr, "\be"'] \arrow[r, "\al"]	
			& S  \arrow[d, dashed, "\phi"] \\
			& T 
		\end{tikzcd}
		\implies
		\begin{tikzcd}
			& T \arrow[dd, dashed, "\phi \circ \psi"] \\
			X \times Y \arrow[ur, "\be"] \arrow[dr, "\be"'] 
			&  \\
			& T 
		\end{tikzcd}
		\] 
		By uniqueness of $f \in L^1(T;T)$, we have that 
		\begin{align*}
			\id_T
			& = f \\
			& = \phi \circ \psi 
		\end{align*}
		A similar argument implies that $\psi \circ \phi = \id_S$. Hence $\phi$ and $\psi$ are isomorphisms with $\phi^{-1} = \psi$. Hence $S$ and $T$ are isomorphic.
	\end{proof}
	
	\begin{defn}
		Let $X, Y$ be vector spaces, $x \in X$ and $y \in Y$. We define $x \otimes y: X^* \times Y^* \rightarrow  \K$ by $x \otimes y(\phi, \psi) \defeq \phi(x) \psi(y)$.  
	\end{defn}
	
	\begin{ex}
		Let $X, Y$ be vector spaces, $x \in X$ and $y \in Y$. Then $x \otimes y \in L^2(X^*, Y^*; \K)$. 
	\end{ex}

	\begin{proof}
		Let $\phi_1, \phi_2 \in X^*$, $\psi \in Y^*$ and $\lam \in \K$. Then 
		\begin{align*}
			x \otimes y(\phi_1 + \lam \phi_2, \psi) 
			& = [\phi_1 + \lam \phi_2 (x)] \psi(y) \\
			& = \phi_1(x)\psi(y) + \lam \phi_2(x)\psi(y) \\
			& = x \otimes y(\phi_1, \psi) + \lam x \otimes y(\phi_2, \psi)
		\end{align*}
		Since $\phi_1, \phi_2 \in X^*$, $\psi \in Y^*$ and $\lam \in \K$ are arbitrary, we have that for each $\psi \in Y^*$, $x \otimes y(\cdot, \psi)$ is linear. Similarly for each $\phi \in X^*$, $x \otimes y(\phi, \cdot)$ is linear. Hence $x \otimes y$ is bilinear and $x \otimes y \in L^2(X^*, Y^*; \K)$. 
	\end{proof}
	
	\begin{defn}
		Let $X, Y$ be vector spaces. We define  
		\begin{itemize}
			\item the \tbf{tensor product of $X$ and $Y$}, denoted $X \otimes Y \subset L^2(X^*, Y^*; \K)$, by 
			$$X \otimes Y \defeq \spn(\text{$x \otimes y: x \in X$ and $y \in Y$}),$$
			\item the \tbf{tensor map}, denoted $\otimes: X \times Y \rightarrow X \otimes Y$, by $\otimes(x, y) \defeq x \otimes y$.
		\end{itemize}
	\end{defn}

	\begin{ex}
		Let $X,Y$ be vector spaces, $(x_j)_{j=1}^n \subset X$ and $(y_j)_{j=1}^n \subset Y$. The following are equivalent:
		\begin{enumerate}
			\item $\sum\limits_{j=1}^n  x_j \otimes y_j = 0$
			\item for each $\phi \in X^*$ and $\psi \in Y^*$, $\sum\limits_{j=1}^n  \phi(x_j) \psi(y_j) = 0$
			\item for each $\phi \in X^*$, $\sum\limits_{j=1}^n  \phi(x_j)  y_j = 0$
			\item for each $\psi \in Y^*$, $\sum\limits_{j=1}^n  \psi(y_j) x_j = 0$
		\end{enumerate}
	\end{ex}

	\begin{proof}\
		\begin{enumerate}
			\item $(1) \implies (2):$ \\
			Suppose that $\sum\limits_{j=1}^n x_j \otimes y_j = 0$. Let $\phi \in X^*$ and $\psi \in Y^*$. Then 
			\begin{align*}
				\sum\limits_{j=1}^n  \phi(x_j) \psi(y_j)
				& = \phi \bigg( \sum\limits_{j=1}^n \psi(y_j) x_j \bigg) \\
				& = 
			\end{align*}
			\item 
			\item 
		\end{enumerate}
	\end{proof}

	\begin{ex}
		Let $X, Y$ be vector spaces. Then $(X \otimes Y, \otimes)$ is a tensor product of $X$ and $Y$. 
	\end{ex}

	\begin{proof}
		Let $Z$ be a vector space and $\al \in L^2(X, Y; Z)$. Define $\phi: X \otimes Y \rightarrow Z$ by $\phi \bigg( \sum\limits_{j=1}^n \lam_j x_j \otimes y_j) \defeq \sum\limits_{j =1}^n \lam_j \al(x_j, y_j)$.  
		\begin{itemize}
			\item \tbf{(well defined):} \\
			Let $u \in X \otimes Y$. Then there exist $(\lam_j)_{j=1}^n \subset \K$, $(x_j)_{j=1}^n \subset X$, $(y_j)_{j=1}^n \subset Y$ such that $u = \sum\limits_{j=1}^n \lam_j x_j \otimes y_j$. Suppose that $u = 0$. Let $\phi \in Z^*$. Then $\phi \circ \al \in L^2(X,Y; Z)$.  
		\end{itemize}
	\end{proof}

	
	
	
	
	
	
	
	
	
	
	
	
	
	
	
	
	
	
	
	
	
	
	
	
	
	
	
	
	
	
	
	
	
	
	
	
	
	
	
	
	
	
	
	
	
	
	
	
	
	
	
	
	
	
	
	
	
	
	
	
	
	
	
	
	
	
	
	
	
	
	
	
	
	
	
	
	
	\backmatter
	\begin{thebibliography}{4}
\bibitem{algebra} \href{https://github.com/carsonaj/Mathematics/blob/master/Introduction\%20to\%20Algebra/Introduction\%20to\%20Algebra.pdf}{Introduction to Algebra}

\bibitem{analysis}  \href{https://github.com/carsonaj/Mathematics/blob/master/Introduction\%20to\%20Analysis/Introduction\%20to\%20Analysis.pdf}{Introduction to Analysis}	

\bibitem{foranal}  \href{https://github.com/carsonaj/Mathematics/blob/master/Introduction\%20to\%20Fourier\%20Analysis/Introduction\%20to\%20Fourier\%20Analysis.pdf}{Introduction to Fourier Analysis}

\bibitem{measure}  \href{https://github.com/carsonaj/Mathematics/blob/master/Introduction\%20to\%20Measure\%20and\%20Integration/Introduction\%20to\%20Measure\%20and\%20Integration.pdf}{Introduction to Measure and Integration}



\end{thebibliography}


	
	
	
	
	
	
	
	
	
	
	
	
	
	
	
	
	
	
	
	
	
\end{document}
	
	