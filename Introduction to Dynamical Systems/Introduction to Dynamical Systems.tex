%% filename: amsbook-template.tex
%% version: 1.1
%% date: 2014/07/24
%%
%% American Mathematical Society
%% Technical Support
%% Publications Technical Group
%% 201 Charles Street
%% Providence, RI 02904
%% USA
%% tel: (401) 455-4080
%%      (800) 321-4267 (USA and Canada only)
%% fax: (401) 331-3842
%% email: tech-support@ams.org
%% 
%% Copyright 2006, 2008-2010, 2014 American Mathematical Society.
%% 
%% This work may be distributed and/or modified under the
%% conditions of the LaTeX Project Public License, either version 1.3c
%% of this license or (at your option) any later version.
%% The latest version of this license is in
%%   http://www.latex-project.org/lppl.txt
%% and version 1.3c or later is part of all distributions of LaTeX
%% version 2005/12/01 or later.
%% 
%% This work has the LPPL maintenance status `maintained'.
%% 
%% The Current Maintainer of this work is the American Mathematical
%% Society.
%%
%% ====================================================================

%    AMS-LaTeX v.2 driver file template for use with amsbook
%
%    Remove any commented or uncommented macros you do not use.

\documentclass{book}

%    For use when working on individual chapters
%\includeonly{}

%    For use when working on individual chapters
%\includeonly{}

%    Include referenced packages here.
\usepackage[left =.5in, right = .5in, top = 1in, bottom = 1in]{geometry} 
\usepackage{amsmath}
\usepackage{amsthm}
\usepackage{amssymb}
\usepackage{setspace}
\usepackage{mathtools}
\usepackage{tikz}  
\usepackage{tikz-cd}
\usepackage{tkz-fct}
\usepackage{pgfplots}
\usepackage{environ}
\usepackage{tikz-cd} 
\usepackage{enumitem}
\usepackage{color}   %May be necessary if you want to color links
%\usepackage{xr}

\usepackage{hyperref}
\hypersetup{
	colorlinks=true, %set true if you want colored links
	linktoc=all,     %set to all if you want both sections and subsections linked
	linkcolor=black,  %choose some color if you want links to stand out
	urlcolor=cyan
}
\usepackage[symbols,nogroupskip,sort=none]{glossaries-extra}

\pgfplotsset{every axis/.append style={
		axis x line=middle,    % put the x axis in the middle
		axis y line=middle,    % put the y axis in the middle
		axis line style={<->,color=black}, % arrows on the axis
		xlabel={$x$},          % default put x on x-axis
		ylabel={$y$},          % default put y on y-axis
}}


\theoremstyle{definition}
\newtheorem{definition}{Definition}[subsection]
\newtheorem{defn}[definition]{Definition}
\newtheorem{note}[definition]{Note}
\newtheorem{ax}[definition]{Axiom}
\newtheorem{thm}[definition]{Theorem}
\newtheorem{lem}[definition]{Lemma}
\newtheorem{prop}[definition]{Proposition}
\newtheorem{cor}[definition]{Corollary}
\newtheorem{conj}[definition]{Conjecture}
\newtheorem{ex}[definition]{Exercise}
\newtheorem{exmp}[definition]{Example}
\newtheorem{soln}[definition]{Solution}

\setcounter{tocdepth}{3}

% hide proofs
\newif\ifhideproofs
%\hideproofstrue %uncomment to hide proofs
\ifhideproofs
\NewEnviron{hide}{}
\let\proof\hide
\let\endproof\endhide
\fi

% lower-case greek
\newcommand{\al}{\alpha}
\newcommand{\be}{\beta}
\newcommand{\gam}{\gamma}
\newcommand{\del}{\delta}
\newcommand{\ep}{\epsilon}
\newcommand{\ze}{\zeta} 
\newcommand{\kap}{\kappa} 
\newcommand{\lam}{\lambda}  
\newcommand{\sig}{\sigma} 
\newcommand{\omi}{\omicron}
\newcommand{\up}{\upsilon}
\newcommand{\om}{\omega}

% upper-case greek
\newcommand{\Gam}{\Gamma}
\newcommand{\Del}{\Delta}
\newcommand{\Lam}{\Lambda} 
\newcommand{\Sig}{\Sigma} 
\newcommand{\Om}{\Omega}

% blackboard bold
\newcommand{\C}{\mathbb{C}}
\newcommand{\E}{\mathbb{E}}
\newcommand{\F}{\mathbb{F}}
\renewcommand{\H}{\mathbb{H}}
\newcommand{\K}{\mathbb{K}}
\newcommand{\N}{\mathbb{N}}
\renewcommand{\O}{\mathbb{O}}
\newcommand{\Q}{\mathbb{Q}}
\newcommand{\R}{\mathbb{R}}
\renewcommand{\S}{\mathbb{S}}
\newcommand{\T}{\mathbb{T}}
\newcommand{\V}{\mathbb{V}}
\newcommand{\Z}{\mathbb{Z}}

% math caligraphic
\newcommand{\MA}{\mathcal{A}}
\newcommand{\MB}{\mathcal{B}}
\newcommand{\MC}{\mathcal{C}}
\newcommand{\MD}{\mathcal{D}}
\newcommand{\ME}{\mathcal{E}}
\newcommand{\MF}{\mathcal{F}}
\newcommand{\MG}{\mathcal{G}}
\newcommand{\MH}{\mathcal{H}}
\newcommand{\MI}{\mathcal{I}}
\newcommand{\MJ}{\mathcal{J}}
\newcommand{\MK}{\mathcal{K}}
\newcommand{\ML}{\mathcal{L}}
\newcommand{\MM}{\mathcal{M}}
\newcommand{\MN}{\mathcal{N}}
\newcommand{\MO}{\mathcal{O}}
\newcommand{\MP}{\mathcal{P}}
\newcommand{\MQ}{\mathcal{Q}}
\newcommand{\MR}{\mathcal{R}}
\newcommand{\MS}{\mathcal{S}}
\newcommand{\MT}{\mathcal{T}}
\newcommand{\MU}{\mathcal{U}}
\newcommand{\MV}{\mathcal{V}}
\newcommand{\MW}{\mathcal{W}}
\newcommand{\MX}{\mathcal{X}}
\newcommand{\MY}{\mathcal{Y}}
\newcommand{\MZ}{\mathcal{Z}}

% mathfrak
\newcommand{\MFX}{\mathfrak{X}}
\newcommand{\MFg}{\mathfrak{g}}
\newcommand{\MFh}{\mathfrak{h}}

% tilde 
\newcommand{\tMA}{\tilde{\MA}}
\newcommand{\tMB}{\tilde{\MB}}
\newcommand{\tU}{\tilde{U}}
\newcommand{\tV}{\tilde{V}}
\newcommand{\tphi}{\tilde{\phi}}
\newcommand{\tpsi}{\tilde{\psi}}
\newcommand{\tF}{\tilde{F}}

\newcommand{\iid}{\stackrel{iid}{\sim}}





% label/reference
% internal label/reference
\newcommand{\lex}[1]{\label{ex:#1}}
\newcommand{\rex}[1]{Exercise \ref{ex:#1}}

\newcommand{\ld}[1]{\label{defn:#1}}
\newcommand{\rd}[1]{Definition \ref{defn:#1}}

\newcommand{\lax}[1]{\label{ax:#1}}
\newcommand{\rax}[1]{Axiom \ref{ax:#1}}

\newcommand{\lfig}[1]{\label{fig:#1}}
\newcommand{\rfig}[1]{Figure \ref{fig:#1}}

% external reference
\newcommand{\extrex}[2]{Exercise \ref{#1-ex:#2}}

\newcommand{\extrd}[2]{Definition \ref{#1-defn:#2}}

\newcommand{\extrax}[2]{Axiom \ref{#1-ax:#2}}

\newcommand{\extrfig}[2]{Figure \ref{#1-fig:#2}}

% external documents (EDIT HERE)
%\externaldocument[analysis-]{"/home/carson/Desktop/Github/Mathematics/Introduction to Analysis/Introduction to Analysis.tex"}




% math operators
\DeclareMathOperator{\supp}{supp}
\DeclareMathOperator{\sgn}{sgn}
\DeclareMathOperator{\spn}{span}
\DeclareMathOperator{\Iso}{Iso}
\DeclareMathOperator{\Eq}{Eq}
\DeclareMathOperator{\id}{id}
\DeclareMathOperator{\Aut}{Aut}
\DeclareMathOperator{\Endo}{End}
\DeclareMathOperator{\Homeo}{Homeo}
\DeclareMathOperator{\Sym}{Sym}
\DeclareMathOperator{\Alt}{Alt}
\DeclareMathOperator{\cl}{cl}
\DeclareMathOperator{\Int}{Int}
\DeclareMathOperator{\bal}{bal}
\DeclareMathOperator{\cyc}{cyc}
\DeclareMathOperator{\cnv}{conv}
\DeclareMathOperator{\epi}{epi}
\DeclareMathOperator{\dom}{dom}
\DeclareMathOperator{\cod}{cod}
\DeclareMathOperator{\codim}{codim}
\DeclareMathOperator{\Obj}{Obj}
\DeclareMathOperator{\Derivinf}{Deriv^{\infty}}
\DeclareMathOperator{\Hom}{Hom}
\DeclareMathOperator*{\argmax}{arg\,max}
\DeclareMathOperator*{\argmin}{arg\,min}
\DeclareMathOperator{\diam}{\text{diam}}
\DeclareMathOperator{\rnk}{\text{rank}}
\DeclareMathOperator{\tr}{\text{tr}}
\DeclareMathOperator{\prj}{\text{proj}}
\DeclareMathOperator{\nab}{\nabla}
\DeclareMathOperator{\diag}{\text{diag}}
\DeclareMathOperator*{\ind}{\text{ind}}
\DeclareMathOperator*{\ar}{\text{arity}}
\DeclareMathOperator*{\cur}{\text{cur}}
\DeclareMathOperator*{\Part}{\text{Part}}
\DeclareMathOperator{\Var}{\text{Var}}
\DeclareMathOperator*{\FIP}{\text{FIP}} 
\DeclareMathOperator*{\Fun}{\text{Fun}} 
\DeclareMathOperator*{\Rel}{\text{Rel}} 
\DeclareMathOperator*{\Cons}{\text{Cons}} 
\DeclareMathOperator*{\Sg}{\text{Sg}} 
\DeclareMathOperator*{\ot}{\otimes}
\DeclareMathOperator{\uni}{Uni}

% Algebra
\DeclareMathOperator{\inv}{\text{inv}}
\DeclareMathOperator{\mult}{\text{mult}}
\DeclareMathOperator{\smult}{\text{smult}}

% category theory
\DeclareMathOperator*{\Set}{\text{\tbf{Set}}}
\DeclareMathOperator*{\BanAlg}{\text{\tbf{BanAlg}}}
\DeclareMathOperator*{\Meas}{\text{\tbf{Meas}}}
\DeclareMathOperator*{\TopMeas}{\text{\tbf{TopMeas}}}
\DeclareMathOperator*{\Msrpos}{\text{\tbf{Msr}}_{+}}
\DeclareMathOperator*{\TopMsrpos}{\text{\tbf{TopMsr}}_{+}}
\DeclareMathOperator*{\TopRadMsrpos}{\text{\tbf{TopRadMsr}}_{+}}
\DeclareMathOperator*{\TopRadMsrone}{\text{\tbf{TopRadMsr}}_{1}}
\DeclareMathOperator*{\MsrC}{\text{\tbf{Msr}}_{\C}} 
\DeclareMathOperator*{\TopMsrC}{\text{\tbf{TopMsr}}_{\C}} 
\DeclareMathOperator*{\TopRadMsrC}{\text{\tbf{TopRadMsr}}_{\C}} 
\DeclareMathOperator*{\Maninf}{\text{\tbf{Man}}^{\infty}} 
\DeclareMathOperator*{\ManBndinf}{\text{\tbf{ManBnd}}^{\infty}} 
\DeclareMathOperator*{\Man0}{\text{\tbf{Man}}^{0}}
\DeclareMathOperator*{\Buninf}{\text{\tbf{Bun}}^{\infty}} 
\DeclareMathOperator*{\VecBuninf}{\text{\tbf{VecBun}}^{\infty}} 
\DeclareMathOperator*{\Field}{\text{\tbf{Field}}} 
\DeclareMathOperator*{\Mon}{\text{\tbf{Mon}}} 
\DeclareMathOperator*{\Grp}{\text{\tbf{Grp}}}
\DeclareMathOperator*{\Semgrp}{\text{\tbf{Semgrp}}}
\DeclareMathOperator*{\LieGrp}{\text{\tbf{LieGrp}}} 
\DeclareMathOperator*{\Alg}{\text{\tbf{Alg}}} 
\DeclareMathOperator*{\Vect}{\text{\tbf{Vect}}} 
\DeclareMathOperator*{\Mod}{\text{\tbf{Mod}}}
\DeclareMathOperator*{\Rep}{\text{\tbf{Rep}}} 
\DeclareMathOperator*{\URep}{\text{\tbf{URep}}}
\DeclareMathOperator*{\Ban}{\text{\tbf{Ban}}} 
\DeclareMathOperator*{\Hilb}{\text{\tbf{Hilb}}} 
\DeclareMathOperator*{\Prob}{\text{\tbf{Prob}}} 
\DeclareMathOperator*{\PrinBuninf}{\text{\tbf{PrinBun}}^{\infty}}

\DeclareMathOperator*{\Top}{\text{\tbf{Top}}}
\DeclareMathOperator*{\TopField}{\text{\tbf{TopField}}} 
\DeclareMathOperator*{\TopMon}{\text{\tbf{TopMon}}} 
\DeclareMathOperator*{\TopGrp}{\text{\tbf{TopGrp}}}
\DeclareMathOperator*{\TopVect}{\text{\tbf{TopVect}}} 
\DeclareMathOperator*{\TopEq}{\text{\tbf{TopEq}}}

\DeclareMathOperator*{\VectR}{\text{\tbf{Vect}}_{\R}}
\DeclareMathOperator*{\VectC}{\text{\tbf{Vect}}_{\C}} 
\DeclareMathOperator*{\VectK}{\text{\tbf{Vect}}_{\K}}
\DeclareMathOperator*{\Cat}{\text{\tbf{Cat}}}
\DeclareMathOperator*{\0}{\mbf{0}}
\DeclareMathOperator*{\1}{\mbf{1}}


\DeclareMathOperator*{\Cone}{\text{\tbf{Cone}}}

\DeclareMathOperator*{\Cocone}{\text{\tbf{Cocone}}}


% Algebra
\DeclareMathOperator{\End}{\text{End}} 
\DeclareMathOperator{\rep}{\text{Rep}} 




% notation
\renewcommand{\r}{\rangle}
\renewcommand{\l}{\langle}
\renewcommand{\div}{\text{div}}
\renewcommand{\Re}{\text{Re} \,}
\renewcommand{\Im}{\text{Im} \,}
\newcommand{\Img}{\text{Img} \,}
\newcommand{\grad}{\text{grad}}
\newcommand{\tbf}[1]{\textbf{#1}}
\newcommand{\tcb}[1]{\textcolor{blue}{#1}}
\newcommand{\tcr}[1]{\textcolor{red}{#1}}
\newcommand{\mbf}[1]{\mathbf{#1}}
\newcommand{\ol}[1]{\overline{#1}}
\newcommand{\ub}[1]{\underbar{#1}}
\newcommand{\tl}[1]{\tilde{#1}}
\newcommand{\p}{\partial}
\newcommand{\Tn}[1]{T^{r_{#1}}_{s_{#1}}(V)}
\newcommand{\Tnp}{T^{r_1 + r_2}_{s_1 + s_2}(V)}
\newcommand{\Perm}{\text{Perm}}
\newcommand{\wh}[1]{\widehat{#1}}
\newcommand{\wt}[1]{\widetilde{#1}}
\newcommand{\defeq}{\vcentcolon=}
\newcommand{\Con}{\text{Con}}
\newcommand{\ConKos}{\text{Con}_{\text{Kos}}}
\newcommand{\trl}{\triangleleft}
\newcommand{\trr}{\triangleright}
\newcommand{\alg}{\text{alg}}
\newcommand{\Triv}{\text{Triv}}
\newcommand{\Der}{\text{Der}}
\newcommand{\cnj}{\text{conj}}

\newcommand{\lcm}{\text{lcm}}
\newcommand{\Imax}{\MI_{\text{max}}}


\DeclareMathOperator*{\Rl}{\text{Re}}
\DeclareMathOperator*{\Imn}{\text{Imn}}



% limits
\newcommand{\limfn}{\liminf \limits_{n \rightarrow \infty}}
\newcommand{\limpn}{\limsup \limits_{n \rightarrow \infty}}
\newcommand{\limn}{\lim \limits_{n \rightarrow \infty}}
\newcommand{\convt}[1]{\xrightarrow{\text{#1}}}
\newcommand{\conv}[1]{\xrightarrow{#1}} 
\newcommand{\seq}[2]{(#1_{#2})_{#2 \in \N}}

% intervals
\newcommand{\RG}{[0,\infty]}
\newcommand{\Rg}{[0,\infty)}
\newcommand{\Rgp}{(0,\infty)}
\newcommand{\Ru}{(\infty, \infty]}
\newcommand{\Rd}{[\infty, \infty)}
\newcommand{\ui}{[0,1]}

% integration \newcommand{\dm}{\, d m}
\newcommand{\dmu}{\, d \mu}
\newcommand{\dnu}{\, d \nu}
\newcommand{\dlam}{\, d \lambda}
\newcommand{\dP}{\, d P}
\newcommand{\dQ}{\, d Q}
\newcommand{\dm}{\, d m}
\newcommand{\dsh}{\, d \#}

% abreviations 
\newcommand{\lsc}{lower semicontinuous}

% misc
\newcommand{\as}[1]{\overset{#1}{\sim}}
\newcommand{\astx}[1]{\overset{\text{#1}}{\sim}}
\newcommand{\io}{\text{ i.o.}}
%\newcommand{\ev}{\text{ ev.}}
\newcommand{\Ll}{L^1_{\text{loc}}(\R^n)}

\newcommand{\loc}{\text{loc}}
\newcommand{\BV}{\text{BV}}
\newcommand{\NBV}{\text{NBV}}
\newcommand{\TV}{\text{TV}}

\newcommand{\op}[1]{\mathcal{#1}^{\text{op}}}


% Glossary - Notation
\glsxtrnewsymbol[description={finite measures on $(X, \MA)$}]{n000001}{$\MM_+(X, \MA)$}
\glsxtrnewsymbol[description={velocity}]{v}{\ensuremath{v}}


\makeindex

\begin{document}
	
	\frontmatter
	
	\title{Introduction to Dynamical Systems}
	
	%    Remove any unused author tags.
	
	%    author one information
	\author{Carson James}
	\thanks{}
	
	\date{}
	
	\maketitle
	
	%    Dedication.  If the dedication is longer than a line or two,
	%    remove the centering instructions and the line break.
	%\cleardoublepage
	%\thispagestyle{empty}
	%\vspace*{13.5pc}
	%\begin{center}
	%  Dedication text (use \\[2pt] for line break if necessary)
	%\end{center}
	%\cleardoublepage
	
	%    Change page number to 6 if a dedication is present.
	\setcounter{page}{4}
	
	\tableofcontents
	\printunsrtglossary[type=symbols,style=long,title={Notation}]
	
	%    Include unnumbered chapters (preface, acknowledgments, etc.) here.
	%\include{}
	
	\mainmatter
	%    Include main chapters here.
	%\include{}
	
	\chapter*{Preface}
	\addcontentsline{toc}{chapter}{Preface}
	
	\begin{flushleft}
		\href{https://creativecommons.org/licenses/by-nc-sa/4.0/legalcode.txt}{cc-by-nc-sa}
	\end{flushleft}
	
	\newpage
	
	
	
	
	
	
	
	
	
	
	
	
	
	
	
	
	\chapter{Basic Concepts}
	
	\section{Measure Preserving Transformations}
	
	\begin{defn}
		We define $\Meas$ by 
		\begin{itemize}
			\item $\Obj(\Meas) \defeq \{(X, \MA): (X, \MA) \text{ is a measurable space}\}$.
			\item for $(X, \MA), (Y, \MB) \in \Obj(\Meas)$, 
			$$\Hom_{\Meas}((X, \MA), (Y, \MB)) \defeq \{f:X \rightarrow Y: f \text{ is $(\MA, \MB)$-measurable} \}$$
			\item for $(X, \MA), (Y, \MB), (Z, \MC) \in \Obj(\Meas)$, $f \in \Hom_{\Meas}((X, \MA), (Y, \MB))$ and \\
			$g \in \Hom_{\Meas}((Y, \MB), (Z, \MC))$, 
			$$g \circ_{\Meas} f \defeq g \circ_{\Set} f $$
		\end{itemize}
	\end{defn}

	\begin{ex}
		We have that $\Meas$ is a category.
	\end{ex}

	\begin{proof}
		
	\end{proof}

	\begin{ex}
		We have that $\Meas$ is a Cartesian monoidal category. 
	\end{ex}
	
	\begin{defn}
		Let $(X, \MA, \mu), (Y, \MB, \nu)$ be probability spaces and $f \in \Hom_{\Meas}((X, \MA), (Y, \MB))$. Then $T$ is said to be \tbf{measure preserving} if $f_* \mu = \nu$. 
	\end{defn}

	\begin{ex}
		Let $(X, \MA, \mu), (Y, \MB, \nu)$ be probability spaces and $f \in \Hom_{\Meas}((X, \MA), (Y, \MB))$. Then $f$ is measure preserving iff for each $\phi \in L^1(Y, \MB, \nu)$, $\phi \circ f \in L^1(X, \MA, \mu)$ and 
		$$\int_Y \phi \dnu = \int_X \phi \circ f \dmu $$
	\end{ex}

	\begin{proof}\
		\begin{itemize}
			\item $(\implies)$: \\
			Suppose that $f$ is measure preserving. $\phi \in L^1(Y, \MB, \nu)$. Then the \tcb{a basic result on the change of variables} implies that $\phi \circ f \in L^1(X, \MA, \mu)$ and 
			\begin{align*}
				\int_Y \phi \dnu 
				& = \int_Y \phi d \, f_* mu \\
				& = \int_X \phi \dmu \\
			\end{align*} 
			\item $(\impliedby)$: \\
			Suppose that for each $\phi \in L^1(Y, \MB, \nu)$, $\phi \circ f \in L^1(X, \MA, \mu)$ and 
			$$\int_Y \phi \dnu = \int_X \phi \circ f \dmu $$
			Let $B \in \MB$. Since $\nu$ is a probability measure, $\chi_B \in L^1(Y, \MB, \nu)$. Thus 
			\begin{align*}
				\nu(B)
				& = \int_Y \chi_B \dnu \\
				& = \int_X \chi_B \circ f \dmu \\
				& = \int_X \chi_{f^{-1}(B)} \dmu \\
				& = \mu(f^{-1}(B)) \\
				& = f_*\mu(B)
			\end{align*}
			Since $B \in \MB$ is arbitrary, $f_* \mu = \nu$. 
		\end{itemize}
	\end{proof}

	\begin{defn}
		We define $\Prob$ by 
		\begin{itemize}
			\item $\Obj(\Prob) = \{(X, \MA, \mu): (X, \MA, \mu) \text{ is a probability space}\}$.
			\item for $(X, \MA, \mu), (Y, \MB, \nu) \in \Obj(\Prob)$, 
			$$\Hom_{\Prob}((X, \MA, \mu), (Y, \MB, \nu)) = \{f \in  \Hom_{\Meas}((X, \MA), (Y, \MB)): f \text{ is measure preserving} \}$$
			\item for $(X, \MA, \mu), (Y, \MB, \nu), (Z, \MC, \lam) \in \Obj(\Prob)$, $f \in \Hom_{\Prob}((X, \MA, \mu), (Y, \MB, \nu))$ and \\
			$g \in  \Hom_{\Prob}((Y, \MB, \nu), (Z, \MC, \lam))$, 
			$$g \circ_{\Prob} f \defeq g \circ_{\Set} f$$
		\end{itemize}
	\end{defn}
	
	\begin{ex}
		We have that $\Prob$ is a category.
	\end{ex}
	
	\begin{proof}
		
	\end{proof}
	
	\begin{ex}
		We have that $\Prob$ is not a Cartesian monoidal category. 
	\end{ex}

	\begin{proof}
		content...
	\end{proof}
	
	Even though $\Prob$ does not have products, when applying the forgetful functor $U: \Prob \rightarrow \Meas$, we get a category with products $\Meas$, so in some sense, an object in $\Meas$ is an equivalence class of objects in $\Prob$ where we ignore our notions of size/interaction of sub-objects. After applying the $U$ to a potential product $(Z, \MC, \lam) \in \Obj(\Prob)$ (i.e. there are associated measure preserving maps $f_X:Z \rightarrow X$ and $f_Y:Z \rightarrow Y$) to get $(Z, \MC) \in \Obj(\Meas)$, then $(Z, \MC) \in \Obj(\Meas)$ is a potential product with the same associated maps and we get the unique map $h: Z \rightarrow X \times Y$ in $\Meas$ yielding the typical commutative diagram for products in $\Meas$ (i.e. $h = f_X, f_Y$). In general $h$ does not preserve measure unless $\lam$ can be written as a tensor product. We can quantify how far off a potential product $(Z, \MC, \lam) \in \Obj(\Prob)$ (i.e. an element of the equivalence class) is from being a product by looking at the information loss (relative entropy) across $h$  
	
	
	
	
	
	
	
	
	
	
	
	
	
	
	
	
	
	
	
	
	
	
	
	
	
	
	
	
	
	
	\section{Measure Preserving Systems}
	
	\begin{defn}
		Let $(X, \MA) \in \Obj(\Meas)$, $f \in \End_{\Meas}(X, \MA)$ and $\mu \in \MM(X, \MA)$. Then $\mu$ is said to be \tbf{$f$-invariant} if $f_* \mu = \mu$. 
	\end{defn}
	
	\begin{ex}
		Let $X$ be a compact metric space and $f \in \End_{\Top}(X)$. Then there exists $\mu \in \MP(X, \MA)$ such that $\mu$ is $f$-invariant. \\
		\tbf{Hint:} 
	\end{ex}

	\begin{proof}
		
	\end{proof}
	
	\begin{defn}
		Let $(X, \MA, \mu) \in \Prob$ and $f \in \End_{\Prob}(X, \MA, \mu)$. Then $(X, \MA, \mu, f)$ is said to be a \tbf{measure-preserving dynamical system}. 
	\end{defn}
	
	\begin{ex}
		
	\end{ex}
	
	
	
	
	
	
	
	
	
	
	
	
	
	
	
	
	
	
	
	
	
	
	
	
	
	
	
	
	
	
	\chapter{Thoughts}
	
	\begin{ex}
		Try showing classical and quantum versions of Khintchine's recurrence theorem for observables in the Heisenberg picture. \\
		
		First, classically on phase space $X = \R^{2n}$ with Hamiltonian flow $(\phi_t)_{t \in \R} \subset Iso(X)$, the schrodinger picture is where the state evolves $(p(t),q(t)) = \phi_t(p(0), q(0))$ and satisfies hamilton's equations and the Heisenberg picture is where the observables evolve $f_t = f_0 \circ \phi_t$ and satisfies Louiville's theorem.
		
		In the quantum case on phase space $H = L^2(\R^n)$ with unitary flow $(U_t)_{t \in \R} \subset \Iso(H)$, the schodinger picture is where the state evolves $\psi(t) = U_t \psi(0)$ and satisfies the schrodinger equation and the heisenberg picture is where the observable evolves $A(t) = U_t^*A(0)U_t$ and satisfies von Neumann's equation. 
		
		Classically in ergodic theory, we care about Khintchine's recurrence theorem: Given measure space $(X, \MA, \mu)$ and $T$ which is $\mu$-invariant, then for each $A \in \MA$, $\ep > 0$, there is a relatively dense $\MN \subset \N$ such that for each $n \in \MN$, $\mu(T^{-n}(A) \cap A) > \mu(A)^2 - \ep$. Here relatively dense means that for some $L > 0$, every interval in $\N$ of length $\geq L$, contains an element of $\MN$. \\
		
		This should generalize to the Heisenberg picture $\int_X f \circ \phi_t f \dmu > \bigg( \int_X f \dmu \bigg)^2 - \ep$, i.e. $\int_X f_t f \dmu > \bigg( \int_X f_0 \dmu \bigg)^2 - \ep$. This is strong if we recall cauchy schwarz for $L^2$ and the fact that $\mu$ is $\phi_t$-invariant, so that if $f \geq 0$, then 
		\begin{align*}
			\bigg( \int_X f_0 \dmu \bigg)^2
			& = \int_X f_t \dmu \int_X f_0 \dmu \\
			& \geq \int_X f_t f_0 \dmu \\
			& > \bigg( \int_X f_0 \dmu \bigg)^2 - \ep
		\end{align*}
		. If $\mu$ is a probability measure we interpret this to mean that for an observable $f$, recurrently after enough time, we expect to measure the same value (in some weak sense) as in the outset, or at least the observable at the outset and at time $t$ overlap and are indistinguishable. Then follwoing the idea in nestruevs smooth manifold book about how knowing the state space is equivalent to knowing the observables, as we can determine one from the other, we would like to know if for sufficiently many observables to determine the state well enough $(f_t^{(j)})_{j \in [J]} \in C^{\infty}(X; \Rg^J)$, whether or not we get enough overlap between $(f_t^{(j)})_{j \in [J]}$ and $(f_0^{(j)})_{j \in [J]}$ recurrently in the $l_2$-sense. \href{https://joelmoreira.wordpress.com/2012/04/15/recurrence-theorems/}{I cant believe its not random}\\
		
		This should generalize to the quantum case as in 
		
		\href{http://math.ucv.ro/~niculescu/articles/before2000/Hilbert%20Revue%201999.pdf}{quantum ergodic theorems}\\
		\href{https://arxiv.org/pdf/quant-ph/0312051}{quantum ergodic phd thesis}
	\end{ex}
	
	
	
	
	
	
	
	
	
	
	
	
	
	
	
	
	
	
	
	
	
	
	
	
	\appendix
	\chapter{App}
	
	\section{Reading Diagrams and associated digraphs of diagrams}
	
	\begin{defn}
		Let 
		\[ 
		\begin{tikzcd}[baseline= 7]
			C \arrow[r, "g"] \arrow[d, "h"'] & A \arrow[d, "f"] \\
			A \arrow[r, "f"'] & B\\
		\end{tikzcd}
		\implies
		\begin{tikzcd}[baseline= 7]
			C \arrow[bend right=60, swap]{r}{h} \arrow[bend right=-60]{r}{g} & A  \\
		\end{tikzcd}
		\]
		
		see an intro to the language of category theory by roman for description
	\end{defn}
	
	
	\begin{defn}
		A diagram is said to be \tbf{commutative} if for each path of length $\geq 2$, in the associated digraph gives the same morphism.  
	\end{defn}
	
	
	
	\backmatter
	
	
	
	
	
	
	
	
	
	
	
	
	
	
\end{document}