\documentclass[12pt]{amsart}
\usepackage[margin=1in]{geometry} 
\usepackage{amsmath,amsthm,amssymb,amsfonts,setspace}
\usepackage[shortlabels]{enumitem}
\usepackage{exercise, chngcntr}
\usepackage{cite}
\usepackage{physics}



%
%
%
\newif\ifhideproofs
\hideproofstrue %uncomment to hide proofs
%
%
%
%


\newtheorem{thm}{Theorem}[section]
\newtheorem{lem}[thm]{Lemma}
\newtheorem{prop}[thm]{Proposition}
\newtheorem{cor}[thm]{Corollary}
\newtheorem{conj}{Conjecture}
\newtheorem{defn}[thm]{Definition}
\newtheorem{intp}[thm]{Interpretation}
\newtheorem{note}[thm]{Note}
\newtheorem{ex}[thm]{Exercise}
\newtheorem{exm}[thm]{Example}


\newenvironment{sol}
  {\renewcommand\qedsymbol{$\blacksquare$}\begin{proof}[Solution]}
  {\end{proof}}

\renewcommand{\r}{\rangle}
\renewcommand{\l}{\langle}

\newcommand{\sch}{Schr\"{o}dinger }

\newcommand\Item[1][]{%
  \ifx\relax#1\relax  \item \else \item[#1] \fi
  \abovedisplayskip=0pt\abovedisplayshortskip=0pt~\vspace*{-\baselineskip}}

\newcommand{\al}{\alpha}
\newcommand{\Gam}{\Gamma}
\newcommand{\gam}{\gamma}
\newcommand{\be}{\beta} 
\newcommand{\del}{\delta} 
\newcommand{\Del}{\Delta}
\newcommand{\lam}{\lambda}  
\newcommand{\Lam}{\Lambda} 
\newcommand{\ep}{\epsilon}
\newcommand{\sig}{\sigma} 
\newcommand{\om}{\omega}
\newcommand{\Om}{\Omega}
\newcommand{\C}{\mathbb{C}}
\newcommand{\N}{\mathbb{N}}
\newcommand{\kap}{\kappa}
\renewcommand{\H}{\mathbb{H}}
\newcommand{\Z}{\mathbb{Z}}
\newcommand{\R}{\mathbb{R}}
\newcommand{\Q}{\mathbb{Q}}
\renewcommand{\P}{\mathbb{P}}
\newcommand{\MA}{\mathcal{A}}
\newcommand{\MB}{\mathcal{B}}
\newcommand{\MF}{\mathcal{F}}
\newcommand{\MG}{\mathcal{G}}
\newcommand{\ML}{\mathcal{L}}
\newcommand{\MN}{\mathcal{N}}
\newcommand{\MS}{\mathcal{S}}
\newcommand{\MP}{\mathcal{P}}
\newcommand{\ME}{\mathcal{E}}
\newcommand{\MT}{\mathcal{T}}
\newcommand{\MM}{\mathcal{M}}
\newcommand{\MW}{\mathcal{W}}

\renewcommand{\MR}{\mathcal{R}}


\newcommand{\RG}{[0,\infty]}
\newcommand{\Rg}{[0,\infty)}
\newcommand{\limfn}{\liminf \limits_{n \rightarrow \infty}}
\newcommand{\limpn}{\limsup \limits_{n \rightarrow \infty}}
\newcommand{\limn}{\lim \limits_{n \rightarrow \infty}}
\newcommand{\convt}[1]{\xrightarrow{\text{#1}}}
\newcommand{\conv}[1]{\xrightarrow{#1}} 
\newcommand{\p}[1]{\frac{\partial}{\partial{#1}}}
\newcommand{\Ll}{L^1_{\text{loc}}(\R^n)}
\newcommand{\seq}[1]{(x_{#1})_{#1 \in \N}}
\newcommand{\n}{\Vert}

\DeclareMathOperator*{\argmax}{argmax}
\DeclareMathOperator*{\argmin}{argmin}
\DeclareMathOperator*{\E}{\mathbb{E}}
 
\begin{document}

\title{Quantum Mechanics Notes}
\author[James]{Carson James}
\maketitle


\tableofcontents

\section{Introduction}
\subsection{Schr\"{o}dinger Equation}

\begin{defn}
A particle with potential energy $V(x,t)$ is completely decribed by its \textbf{position wavefunction} $\Psi(x,t)$, which satisfies the \textbf{Schr\"{o}dinger equation}: 
\begin{align*}
i\hbar \p{t} \Psi 
&= -\frac{\hbar^2}{2m} \Del \Psi + V \Psi\\
\end{align*}
\end{defn}

\begin{intp}
We interpret $\vert\Psi(x,t)\vert^2$ to be the \textbf{probability density} for the position, $x$, of the particle at time $t$. Therefore, we require that for each $t \in \R$, $$\int_{\R^n}\Psi(x,t)^* \Psi(x,t) dx = 1$$
\end{intp}

\subsection{Operators}

\begin{defn}

We define the $j^{\text{th}}$ \textbf{position} and \textbf{momentum coordinate operators} $X_j,P_j$, (in position space) by $$X_j \Psi(x,t) = x_j \Psi(x,t)$$ and $$P_j \Psi(x,t) = -i \hbar \p{x_j} \Psi(x,t)$$ 
We define the \textbf{position} and \textbf{momentum} operators, $X$ and $P$, by $$X = (X_1, X_2, \cdots, X_n)$$ and $$P = (P_1, P_2, \cdots, P_n)$$
We denote $P \cdot P$ by $P^2$. Note that $$P^2 = -\hbar^2 \Del$$
If the partical has potential energy $V(x,t)$, we define the \textbf{Hamiltonian} operator, $H$, by $$H = \frac{P^2}{2m} + V$$ Thus the Schr\"{o}dinger equation reads $$i\hbar \p{t}\Psi = H \Psi$$ 
\end{defn}

\begin{note}
If the potential energy doesn't depend on time, we may write $$H = \frac{P^2}{2m} + V(X)$$ meaning Hamiltonian only depends on the position and momentum operators, $X$ and $P$. For the rest of these notes, we assume that the potential energy $V$ does not depend on time.
\end{note}

\begin{defn}
Let $A$ and $B$ be operators. Then $B$ is said to be the \textbf{adjoint} of $A$ if for each $\Psi_1$, $\Psi_2$, $$\l \Psi_1 \vert A\Psi_2 \r = \l B \Psi_1 \vert \Psi_2 \r$$ i.e. $$\int_{\R^n}\Psi_1^* (A\Psi_2) dx = \int_{\R^n}(B\Psi_1)^* \Psi_2 dx$$ If B is the adjoint of $A$, we write $$B = A^{\dagger}$$
\end{defn}

\begin{ex}
Let $A$ be an operator, then \begin{enumerate}
\item for each $\Psi_1, \Psi_2$, $\l A\Psi_1 \vert \Psi_2 \r = \l   \Psi_1 \vert A^{\dagger} \Psi_2 \r$ 
\item $(A^{\dagger})^{\dagger} = A$
\end{enumerate}
\end{ex}

\begin{proof}
\begin{enumerate}
\item For wavefunctions $\Psi_1$, $\Psi_2$, we have
\begin{align*}
\l A \Psi_1 \vert \Psi_2 \r
&= \l \Psi_2 \vert A \Psi_1 \r^*\\
&= \l A^{\dagger}\Psi_2 \vert  \Psi_1 \r^* \hspace{.5cm} \text{(by definition)}\\
&= \l  \Psi_1 \vert A^{\dagger}\Psi_2 \r
\end{align*}
\item For each $\Psi_1, \Psi_2$, we have that
\begin{align*}
\l  A\Psi_1 \vert \Psi_2 \r
&= \l   \Psi_1 \vert A^{\dagger}\Psi_2 \r \\
&= \l  (A^{\dagger})^{\dagger} \Psi_1 \vert \Psi_2 \r
\end{align*}

This implies that for each $\Psi_1, \Psi_2$, $$\big \l \big[A-(A^{\dagger})^{\dagger} \big] \Psi_1, \Psi_2 \big \r = 0$$ Therefore for each $\Psi_1$, $$ \big[A-(A^{\dagger})^{\dagger} \big] \Psi_1 = 0$$ Hence $ \l A-(A^{\dagger})^{\dagger}  = 0$ and $A = (A^{\dagger})^{\dagger}$.
\end{enumerate}
\end{proof}

\begin{defn}
An linear operator $Q$ is \textbf{self-adjoint} if $$Q = Q^{\dagger}$$
\end{defn}

\begin{intp}
For each measurable, observable quantity $\hat{Q}$, there is a self-adjoint operator $Q$ whose eigenvalues are the possible measurment values and whose eigenfunctions are the possible states of the system at measurment.
\end{intp}

\begin{ex}
The operators $X_j, P_j$ and $H$ are self adjoint. \\Hint: for $H$, use Green's second identity.
\end{ex}

\begin{proof}
Since $x_j$ is real, clearly $$\l \Psi_1 \vert X_j \Psi_2 \r = \l X_j\Psi_1 \vert \Psi_2 \r $$ Similarly, we have that 
\begin{align*}
\l \Psi_1 \vert P_j \Psi_2 \r
&=  \int_{\R^n} \Psi_1^* \bigg(-i\hbar\p{x_j} \Psi_2\bigg)dx \\
&= -i \hbar\int_{\R^n} \Psi_1^*  \bigg(\p{x_j} \Psi_2\bigg)dx\\
&= i\hbar \int_{\R_n} \bigg( \p{x_j} \Psi_1^* \bigg) \Psi_2 dx \hspace{1cm } \text{(integration by parts)}\\
&= \int_{\R^n} \bigg( -i \hbar \p{x_j} \Psi_1 \bigg)^* \Psi_2 dx\\
&= \l P \Psi_1 \vert \Psi_2 \r
\end{align*}

Finally 
\begin{align*}
\l \Psi_1 \vert H \Psi_2 \r - \l H \Psi_1 \vert  \Psi_2 \r
&= \int_{\R^n} \Psi_1^* \bigg(-\frac{\hbar^2}{2m}\Del \Psi_2 + V \Psi_2\bigg)dx - \int_{\R^n} \bigg(-\frac{\hbar^2}{2m}\Del \Psi_1 + V\Psi_1\bigg)^*  \Psi_2 dx \\
&= \frac{\hbar^2}{2m}\int_{\R^n} (\Del \Psi_1^* )\Psi_2 - \Psi_1^*(\Del \Psi_2)dx\\
&= 0 \hspace{1cm} \text{(Green's second identity)}
\end{align*}
\end{proof}

\begin{ex}
Let $Q$ be a self-adjoint operator. Then 
\begin{enumerate}
\item the eigenvalues of $Q$ are real.
\item the eigenfunctions of $Q$ corresponding to distinct eigenvalues are orthogonal.
\end{enumerate}
\end{ex}

\begin{proof}
\ \begin{enumerate}
\item Let $\lam$ be an eigenvalue of $Q$ with corresponding eigenfunction $\Psi$. Then 
\begin{align*}
 \lam \l \Psi \vert \Psi\r
&= \l \Psi \vert Q \Psi\r \\
&= \l Q \Psi \vert \Psi\r \\
&= \lam^* \l \Psi \vert \Psi\r
\end{align*}
Thus $\lam = \lam^*$ and is real

\item Let $\lam_1$ and $\lam_2$ be eigenvalues of $Q$ with corresponding eigenfunctions $\Psi_1$ and $\Psi_2$. Suppose that $\lam_1 \neq \lam_2$. Then 
\begin{align*}
\lam_2 \l \Psi_1 \vert  \Psi_2\r
&= \l \Psi_1 \vert Q \Psi_2\r\\
&= \l Q \Psi_1 \vert  \Psi_2\r\\
&= \lam_1 \l \Psi_1 \vert  \Psi_2\r
\end{align*}
So $(\lam_2 - \lam_1)\l \Psi_1 \vert  \Psi_2\r = 0$. Which implies that $\l \Psi_1 \vert  \Psi_2\r=0$
\end{enumerate}
\end{proof}

\begin{defn}
Let $A$ and $B$ be operators. The \textbf{commutator} of $A$ and $B$, $[A,B]$, is defined by $$[A,B] = AB - BA$$
\end{defn}

\begin{ex}
We have $[X_j, P_j] = i\hbar$.
\end{ex}

\begin{proof}
For a position wave function $\Psi$, 
\begin{align*}
[X_j, P_j]\Psi(x,t)
&= [x_j, -i\hbar \p{x_j}]\Psi(x,t)\\
&= (-i\hbar) \bigg[x_j \p{x_j}\Psi(x,t)- \p{x_j}x_j\Psi(x,t)\bigg]\\
&= (-i\hbar)\bigg[ x_j \p{x_j}\Psi(x,t)- \Psi(x,t) - x_j \p{x_j}\Psi(x,t)\bigg]\\
&=i\hbar \Psi(x,t)
\end{align*}

Hence $[X_j, P_j] = i\hbar$
\end{proof}

\subsection{Continuity Equation}

\begin{ex}
If $V$ is real and $\Psi$ satisfies the Schr\"{o}dinger equation, then $$i\hbar \p{t} \Psi^* = -H\Psi^* $$
\end{ex}

\begin{proof}
We have that 
\begin{align*}
i \hbar \p{t} \Psi^{*} 
&= \bigg(-i \hbar \p{t} \Psi\bigg)^*\\
&=\bigg( - \bigg[-\frac{\hbar^2}{2m}\Del \Psi + V \Psi \bigg] \bigg)^*\\
&= - \bigg[ -\frac{\hbar^2}{2m}\Del \Psi^* + V \Psi^*\bigg]\\
&= -H \Psi^*
\end{align*}
\end{proof}

\begin{ex}
We have that $$\p{t} (\Psi^* \Psi) + \frac{\hbar}{2mi} \nabla \cdot \bigg[ \Psi^* (\nabla \Psi) - (\nabla \Psi^*) \Psi\bigg] = 0 $$
\end{ex}

\begin{proof}
\begin{align*}
\p{t}(\Psi^* \Psi) 
&= \bigg(\p{t} \Psi^* \bigg) \Psi + \Psi^* \bigg(\p{t} \Psi \bigg)\\
&= \bigg( \frac{\hbar}{2mi} (\Del \Psi^*) \Psi - \frac{1}{i \hbar }V \Psi^* \Psi\bigg) + \bigg( -\frac{\hbar}{2mi}  \Psi^* (\Del \Psi) + \frac{1}{i \hbar }V \Psi^* \Psi \bigg)\\
&= \frac{\hbar}{2mi} \bigg[ (\Del \Psi^*) \Psi - \Psi^* (\Del \Psi) \bigg]\\
&= -\frac{\hbar}{2mi} \bigg[ \Psi^* (\Del \Psi) - (\Del \Psi^*) \Psi\bigg]\\
&= - \frac{\hbar}{2mi} \nabla \cdot \bigg[\Psi^* (\nabla \Psi) - (\nabla \Psi^*) \Psi \bigg]
\end{align*}

Therefore  $$\p{t} (\Psi^* \Psi) + \frac{\hbar}{2mi} \nabla \cdot \bigg[ \Psi^* (\nabla \Psi) - (\nabla \Psi^*) \Psi\bigg] = 0 $$
\end{proof}

\begin{defn}
We define the \textbf{probability current density}, $j$, of the particle to be $$j = \frac{\hbar}{2mi} \bigg[ \Psi^* (\nabla \Psi) - (\nabla \Psi^*) \Psi\bigg]$$ 
\end{defn}
\subsection{Position and Momentum Space}
\begin{defn}
We define the \textbf{momemtum wavefunction}, $\Phi$, of the particle to be the Fourier transform of the position wavefunction: 
\begin{align*}
\Phi(p,t) 
&= F[\Psi](p,t)\\
&= \frac{1}{(2 \pi \hbar)^{n/2}} \int_{\R ^n}\Psi(x,t)e^{-i \frac{p \cdot x}{\hbar} }dx
\end{align*}
\end{defn}

\begin{note}
We recall the following facts about Fourier transforms:
\begin{enumerate}
\item $$\Phi(p,t) = \frac{1}{(2 \pi \hbar)^{n/2}} \int_{\R ^n}\Psi(x,t)e^{-i \frac{p \cdot x}{\hbar} }dx $$ and $$\Psi(x,t) = \frac{1}{(2 \pi \hbar)^{n/2}} \int_{\R ^n}\Phi(p,t)e^{i \frac{p \cdot x}{\hbar} }dp $$

\item $$F\bigg[\p{x_j} \Psi \bigg] = \frac{i p_j}{\hbar}F[\Psi]$$
and $$F^{-1}\bigg[\p{p_j} \Phi \bigg] = -\frac{i x_j}{\hbar}F[\Psi]$$

\item $$\int_{\R^n} \Psi_1^* \Psi_2 dx = \int_{\R^n} F[\Psi_1]^* F[\Psi_2]dx$$
\end{enumerate}
\end{note}

\begin{note}
Let $Q(X,P)$ be a self-adjoint operator. Then the properties of the Fourier transform inmply that:
\[
Q(X,P)=
\begin{cases}
Q(x, -i\hbar \nabla) & \text{(position space)}\\
Q(i\hbar \nabla, p) & \text{(momentum space)}
\end{cases}
\]
\end{note}

\begin{ex}
If $\Psi$ satisfies the Schr\"{o}dinger equation, then $\Phi$ satisfies $$i\hbar \p{t}\Phi = \frac{p^2}{2m}\Phi + V(i \hbar \nabla)\Phi$$
\end{ex}

\begin{proof}
Starting with the Schr\"{o}dinger equation, we have 
\begin{align*}
i\hbar \p{t} \Psi 
&= \bigg[\frac{P^2}{2m} + V(X)\bigg] \Psi\\
&= \bigg[\frac{-\hbar^2}{2m}\Del + V(x)\bigg] \Psi \hspace{1cm} \text{(position space)}
\end{align*} 
Taking Fourier transforms of both sides, we see that 
\begin{align*}
i\hbar \p{t} \Phi 
&= \bigg[\frac{P^2}{2m} + V(X)\bigg] \Phi\\
&= \bigg[\frac{p^2}{2m} + V(i \hbar \nabla)\bigg] \Phi \hspace{1cm} \text{(position space)}
\end{align*} 
\end{proof}

\begin{intp}
We interpret $\vert \Phi (p,t) \vert^2$ to be the probability density for the momentum, $p$, of the particle at time $t$.  
\end{intp}

\begin{note}
For a self-adjoint operator $Q(X,P)$, the expected value of $Q$,  is given by 

\[ 
\l Q \r = 
\begin{cases}
\l \Psi(x,t) \vert Q(x, -i\hbar \nabla) \Psi(x,t)\r & \hspace{1cm} \text{(position space)}\\
\l \Phi(p,t) \vert Q(i\hbar \nabla, p) \Phi(p,t) \r & \hspace{1cm} \text{(momentum space)}\\
\end{cases}
\]
\end{note}

\subsection{Stationary States}
\begin{defn}
When the potential energy $V$ doesn't depend on time, we look for solutions to the \sch equation of the form $$\Psi(x,t) = \psi(x) \varphi(t)$$ With a closer look, we find that

\begin{enumerate}
\item $H\psi = E \psi$
\item $\varphi(t) = e^{-i\frac{E}{\hbar}t}$
\end{enumerate}
Statement $(1)$ is referred to as the \textbf{time-independent \sch equation}. Eigenfuntions of the Hamiltonian operator are called \textbf{stationary states}. If the possible eigenvalues for the Hamiltonian operator are discreet $(E_n)_{n\in \N}$ with stationary states $(\psi_n)_{n \in \N}$, then the general solution to the \sch equation is $$\Psi(x,t) = \sum_{n \in \N} c_n \psi_n(x) e^{-i\frac{E_n}{\hbar}t}$$ where $$c_n = \int_{\R^n}\psi_n^*(x) \Psi(x,0)dx$$
\end{defn}

\begin{defn}
An energy eiganvalue $E_n$ of $H$ is said to have a \textbf{degeneracy of degree} $k$ if it corresponds to $k$ orthonomal stationary states. 
\end{defn}

\begin{note}
If the energy eigenvalues $(E_n)_{n \in \N}$ have degeneracies of degrees $(k_n)_{n \in \N}$ with corresponding orthonormal stationary states $(\psi_{n,j})_{j=1}^{k_n}$ and $$\Psi(x,t) = \sum_{n \in \N} \sum_{j =1}^{k_n}c_{n,j} \psi_{n,j}(x)e^{-i \frac{E_n}{\hbar}t}$$ Then the probability of measuring the energy $E_n$ is $$\P(E_n) = \sum_{j=1}^{k_n} \vert c_{n,j}\vert^2$$
\end{note}

\begin{defn}
If the spectrum of the Hamiltonian is discreet, the stationary state with the least energy is called the \textbf{ground state}. The stationary states that are not the ground state are called \textbf{excited states}.
\end{defn}

\section{Fundamental Examples in One Dimension}

\subsection{The Infinite Square Well}

\begin{defn}
The infinite square well is defined by the potential 
\[
V(x) = 
\begin{cases}
\infty & x \in I_1 = (-\infty, a]\\
0 & x \in I_2 = (0,a)\\
\infty &x \in I_3 = [a,\infty)
\end{cases}
\]
\end{defn}

\begin{ex}
By starting with a finite potental well and letting the height of the well go to infinity, show that the stationary states and their  energies are given by $$\psi_n(x)= 
\begin{cases}
\sqrt{\frac{2}{a}}\sin(\frac{n \pi}{a}x)  & x \in (0,a) \\
0 & x \not \in (0,a)
\end{cases} $$ 
and 
$$E_n = \frac{n^2 \pi^2 \hbar^2}{2ma^2}$$
\end{ex}

\begin{proof}
Define 
\[
V_\al(x) = 
\begin{cases}
\al & x \in I_1 \\
0 & x \in I_2\\
\al & x \in I_3
\end{cases}
\]

For the potential energy $V_\al$, in sections $I_1, I_3$ the \sch equation may be written as $$\dv[2]{\psi}{x} = \frac{2m}{\hbar^2}(\al-E)\psi$$ Assuming $\al > E$, we may write $l = \frac{\sqrt{2m(\al-E)}}{\hbar}$ and substitute to get $$\dv[2]{\psi}{x} = l^2\psi$$ \\Thus in region $I_1$, $\psi_1(x) = Ae^{lx} + Be^{-lx}$ and in region $I_3$, $\psi_3(x) = Fe^{lx} + Ge^{-lx}$. Since $e^{-lx}$ blows up as $x \rightarrow - \infty$, $B=0$. Since $e^{lx}$ blows up as $x \rightarrow  \infty$, $F=0$. \vspace{4mm}\\
In section $I_2$, the \sch equation may be written as $$\dv[2]{\psi}{x} = -\frac{2mE}{\hbar^2}\psi$$ We write $k = \frac{\sqrt{2mE}}{\hbar}$ and substitute to get $$\dv[2]{\psi}{x} = -k^2\psi$$  
Hence in region $I_2$, $\psi_2(x) = C\sin(kx) + D \cos(kx)$. \vspace{.4cm}\\
So far we have 
\[
\psi_\al(x)= 
\begin{cases}
Ae^{lx} & x \in I_1 \\
C\sin(kx) + D \cos(kx)  & x \in I_2 \\
Ge^{-lx} & x \in I_3
\end{cases}
\]

To find possible wavefunctions $\psi$ for the infinite potential, we let $\al \rightarrow \infty$. As $\al \rightarrow \infty$, we have that $l \rightarrow \infty$. Hence $\psi_1 \rightarrow 0$ and $\psi_3 \rightarrow 0$. So for the infinite potential, 
\[
\psi(x)= 
\begin{cases}
C\sin(kx) + D \cos(kx)  & x \in (0,a) \\
0 & x \not \in (0,a)
\end{cases}
\] 

By continuity at the points $x=0$ and $x=a$, we see that $0 = C\sin(0) + D \cos(0)$ which imples that $D= 0$ and $0 = C\sin(ka)$ which yields various solutions $$k_n = \frac{n \pi}{a} \hspace{.5cm} n \in \Z$$
To avoid non-normalizable solutions or linearly dependent solutions, we restrict $n \in \N$. Our energies are then $$E_n = \frac{\hbar^2k_n^2}{2m} = \frac{\hbar^2n^2 \pi^2}{2ma^2} \hspace{.4cm} n \in \N$$
and (after normalizing) our stationary states are 
\[
\psi_n(x)= 
\begin{cases}
\sqrt{\frac{2}{a}}\sin(\frac{n \pi}{a}x)  & x \in (0,a) \\
0 & x \not \in (0,a)
\end{cases}
\]
\end{proof}

\subsection{The Harmonic Oscillator}

\begin{defn}
The \textbf{harmonic oscillator} in one dimension is defined by the potential energy: $$V(x) = \frac{1}{2}m \om^2 x^2$$ We define the \textbf{lowering operator}, $a$, by $$a = \frac{1}{\sqrt{2 \hbar m \om }}\bigg(m\om X +iP\bigg)$$ 
\end{defn}

\begin{ex}
The adjoint of the lowering operator is $$a^{\dagger} = \frac{1}{\sqrt{2 \hbar m \om }}\bigg(m\om X -iP\bigg)$$
\end{ex}

\begin{proof}
For a wave functions $\Psi_1$, $\Psi_2$,
\begin{align*}
\int_{\R}\bigg[ \frac{1}{\sqrt{2 \hbar m \om }}\bigg(m\om X -iP\bigg)\Psi_1 \bigg]^* \Psi_2 dx
&= \frac{1}{\sqrt{2 \hbar m \om }}\int_{\R}(m\om x \Psi_1(x,t)^* \Psi_2(x,t) -\hbar \bigg(\p{x}\Psi_1(x,t)^*\bigg)\Psi_2(x,t) dx\\
&=\frac{1}{\sqrt{2 \hbar m \om }}\int_{\R}(m\om x \Psi_1(x,t)^* \Psi_2(x,t) +\hbar \Psi_1(x,t)^*\bigg(\p{x}\Psi_2(x,t)\bigg) dx \hspace{.5cm}\\
&=\int_{\R}\Psi_1^*\bigg[ \frac{1}{\sqrt{2 \hbar m \om }}\bigg(m\om X +iP\bigg)\Psi_2 \bigg]  dx
\end{align*}
\end{proof}

\begin{defn}
We call $a^{\dagger}$ the \textbf{raising operator} and together, $a$ and $a^{\dagger}$ are called the ladder operators.
\end{defn}

\begin{ex}
We have that 
\begin{enumerate}
\item $aa^{\dagger} = \frac{1}{\hbar \om}H + \frac{1}{2}$
\item $a^{\dagger}a = \frac{1}{\hbar \om}H - \frac{1}{2}$
\item $[a,a^{\dagger}] = 1$
\end{enumerate}
\end{ex}

\begin{proof}
\begin{enumerate}
\item \
\begin{align*}
a a^{\dagger}
&= \frac{1}{2\hbar m \om}\big(m \om X + iP \big) \big( m\om X - iP )\\
&= \frac{1}{2 \hbar m \om} \bigg[ \big(m^2 \om^2 X^2 + P^2 \big) - m\om i\big(XP - PX \big) \bigg]\\
&= \frac{1}{\hbar \om}\big(\frac{1}{2m}P^2 + \frac{1}{2}m \om^2 X^2 \big) - \frac{i}{2 \hbar}\big[X,P \big]\\
&= \frac{1}{\hbar \om}H + \frac{1}{2}
\end{align*}
\item Similar
\item Trivial
\end{enumerate}
\end{proof}

\begin{ex}
If $H\psi = E\psi$, then 
\begin{enumerate}
\item $Ha\psi = (E-\hbar \om) a \psi$
\item $Ha^{\dagger}\psi = (E+\hbar \om) a^{\dagger} \psi$
\end{enumerate}
\end{ex}

\begin{proof}\
\begin{enumerate}
\item \
\begin{align*}
Ha\psi 
&= \hbar \om \bigg(aa^{\dagger}-\frac{1}{2}\bigg)a \psi\\
&= \hbar \om \bigg(aa^{\dagger}a-\frac{1}{2}a\bigg) \psi\\
&= \hbar \om a\bigg(a^{\dagger}a-\frac{1}{2}\bigg) \psi\\
&= \hbar \om a\bigg(a^{\dagger}a+\frac{1}{2} -1\bigg) \psi\\
&= \hbar \om a\bigg(\frac{1}{\hbar \om}H -1\bigg) \psi\\
&= a H\psi -\hbar \om a \psi \\
&= (E - \hbar \om)a\psi 
\end{align*}
\item Similar
\end{enumerate}
\end{proof}

\begin{intp}
The lowering operator ``lowers"  a stationary state $\psi$ with energy $E$ to a stationary state $a\psi$ with energy $E-\hbar \om$ and the raising operator ``raises"  a stationary state $\psi$ with energy $E$ to a stationary state $a^{\dagger}\psi$ with energy $E+\hbar \om$.
\end{intp}

\begin{defn}
Since the zero function is a solution to the time-independent \sch equation, we define the ground state, $\psi_0$ of the harmonic oscillator to be the stationary state that satisfies $a\psi_0 = 0$. The excited states $\psi_n$, for $n \geq 1$, are obtained by applying the rasing operator $n$ times and then normalizing.
\end{defn}

\begin{ex}
We have that
\begin{enumerate}
\Item $$\psi_0= \bigg(\frac{m \om}{ \pi \hbar}\bigg)^{\frac{1}{4}}e^{-\frac{m \om}{2 \hbar}x^2}$$\vspace{2mm}
\Item $$E_0 = \frac{1}{2}\hbar \om$$ \vspace{2mm}
\Item $$\psi_n = c_n(a^{\dagger})^n\psi_0 \hspace{.4cm} (\text{for some constant } c_n)$$ \vspace{2mm}
\Item $$E_n = \hbar \om (n + \frac{1}{2})$$
\end{enumerate}

\end{ex}

\begin{proof}\
\begin{enumerate}
\item The simple differential equation $a\psi_0 = 0$ has the solution $$\psi_0 = Ae^{-\frac{m \om}{2 \hbar}x^2}$$ Thus $$\vert \psi_0\vert^2 = \vert A \vert^2 e^{-\frac{m \om}{ \hbar}x^2}$$ If we normalize this function, we obtain $$\psi_0= \bigg( \frac{m \om}{ \pi \hbar} \bigg)^{\frac{1}{4}} e^{-\frac{m \om}{2 \hbar}x^2}$$
\item It is tedious but straightforward to show that $$H\psi_0 = \frac{1}{2}\hbar \om\psi_0$$
\item Clear by definition.
\item Clear by previous exercise.
\end{enumerate} 
\end{proof}

\begin{ex}
\begin{enumerate}\
\item $\psi_{n+1} = \frac{1}{\sqrt{n+1}} a^{\dagger} \psi_n $
\item $\psi_{n-1} = \frac{1}{\sqrt{n}} a \psi_n $

\end{enumerate}
Hint: use the adjoint-ness of $a$ and $a^{\dagger}$
\end{ex}

\begin{proof}\
\begin{enumerate}
\item 
\begin{align*}
aa^{\dagger}\psi_n
&= \bigg(\frac{1}{\hbar \om}H +\frac{1}{2}\bigg)\psi_n\\
&= \frac{1}{\hbar \om}E_n \psi_n + \frac{1}{2} \psi_n\\
&= (n+1) \psi_n
\end{align*}
Since $\psi_{n+1} = ca^{\dagger}\psi_{n}$, we have that
\begin{align*}
1
&=\l \psi_{n+1}\vert \psi_{n+1}\r\\
&=\l c a^{\dagger}\psi_n \vert ca^{\dagger}\psi_n\r\\
&= \vert c\vert^2 \l a^{\dagger} \psi_n\vert a^{\dagger} \psi_n \r\\
&= \vert c\vert^2 \l a a^{\dagger}\psi_n \vert \psi_n \r\\
&= \vert c\vert^2 \l (n+1)\psi_n \vert \psi_n \r\\
&= \vert c\vert^2  (n+1) \l \psi_n \vert \psi_n \r\\
&= \vert c\vert^2  (n+1)\\
\end{align*}
So $c = \frac{1}{\sqrt{n+1}}$\vspace{2mm}

\item Similar to $(1)$.
\end{enumerate}
\end{proof}

\begin{ex}
The $n^{\text{th}}$ stationary state is given by $\psi_n = \frac{1}{\sqrt{n!}}(a^{\dagger})^n\psi_0$
\end{ex}

\begin{proof}
Clear by induction.
\end{proof}

\begin{ex}
Show that
\begin{enumerate}
\item $\psi_1(x) = \big(\frac{4m^3 \om^3}{\hbar^3 \pi} \big)xe^{-\frac{m \om}{2 \hbar}x^2}$
\item $E_1 = \frac{3}{2} \hbar \om$
\end{enumerate}
\end{ex}

\begin{proof}
Straightforward.
\end{proof}



\begin{ex}
If particle one is in state $\psi_0$ at time $t=0$, then the momentum wave function is $$\Phi(p,t) = \bigg(\frac{1}{m \om \pi \hbar}\bigg)^{\frac{1}{4}}e^{-\frac{1}{2m \om \hbar}p^2}e^{-i \frac{\om}{2}t}$$
\end{ex}

\begin{proof}
By assumption $$\Psi(x,t) = \psi_0(x)e^{-i \frac{\om}{2}t}$$ Thus $$\Phi(p,t) = \frac{1}{\sqrt{2 \pi \hbar}}\int_{\R}\Psi(x,t)e^{-i\frac{px}{h}}dx$$
The rest is straightforward.
\end{proof}

\section{Fundamental Examples in Three Dimensions}

\subsection{Spherical Coordinates}
\begin{defn}
We now set $n=3$, and work with spherical coordinates $(r, \theta, \phi)$ where $r$ is the distance in from the origin, $0 \leq \theta \leq \pi$ is the angle with initial side on the positive $z$-axis, and $0 \leq \phi < 2\pi$ is the angle in the $x$-$y$ plane with initial side on the positive $x$-axis going towards the positive $y$-axis.
\end{defn}

\begin{prop}
In spherical coordinates, the time independent \sch equation becomes $$-\frac{\hbar^2}{2m}\bigg[ \frac{1}{r^2} \p{r}\bigg( r^2 \p{r}\bigg) + \frac{1}{r^2 \sin \theta} \p{\theta} \bigg( \sin \theta \p{\theta}\bigg) + \frac{1}{r^2 \sin^2 \theta} \frac{\partial^2}{\partial \phi^2}  \bigg] \psi + V \psi = E \psi$$
\end{prop}

\begin{defn}
If the potential energy $V$ only depends on $r$, then we can solve for stationary solutions of the form $\psi(r, \theta, \phi) = R(r), Y(\theta, \phi)$. It results that there is some constant $l$ such that 

\begin{enumerate}
\Item 
$$\frac{1}{R} \frac{d}{dr} r^2 \frac{dR}{dr} - \frac{2m}{\hbar^2}r^2(V -E) = l(l+1)$$ \vspace{3mm}
\Item 
$$\frac{1}{Y}\bigg[\frac{1}{\sin \theta}\p{\theta} \bigg(\sin \theta \frac{\partial Y}{\partial \theta} \bigg) + \frac{1}{\sin^2 \theta}\frac{\partial^2Y}{\partial \phi^2} \bigg] = -l(l+1)$$

\end{enumerate} \vspace{3mm}
The number $l$ is called the \textbf{azimuthal quantum number}, equation $(1)$ is called the \textbf{radial equation} and equation $(2)$ is called the \textbf{angular equation}. 
\end{defn}

\begin{defn}
We can look for solutions to the angular equation of the form \\$Y(\theta, \phi) = \Theta(\theta)\Phi(\phi)$. It results that there is some constant $m$ such that 
\begin{enumerate}
\Item 
$$\frac{1}{\Theta}  \sin \theta \frac{d}{d\theta}\bigg( \sin \theta \frac{d\Theta}{d\theta} \bigg) + l (l+1) \sin^2 \theta = m^2$$
\vspace{3mm}
\Item $$\frac{1}{\Phi} \frac{d^2 \Phi}{d\phi^2} = -m^2$$
\end{enumerate} \vspace{3mm}
Equation $(2)$ has the solution $$\Phi(\phi) = e^{im\phi}$$
Since $(r,\theta, \phi)$ is the same point in space as $(r, \theta, \phi+2 \pi)$, we require that $\Phi(\phi) = \Phi(\phi+2\pi)$. This implies that $m \in \Z$. The integer $m$ is called the \textbf{magnetic quantum number}. \vspace{3mm}\\ 
If $l \in \N_0$ and $m \leq l$, then equation $(1)$ has the solution $$\Theta(\theta) = AP_l^m(\cos \theta)$$ where $P_l^m$ is the \textbf{associated Legendre} function given by $$P_l^m(x) = (1-x^2)^{\frac{\vert m \vert}{2}}\bigg(\frac{d}{dx} \bigg)^{\vert m \vert} P_l(x)$$ and $P_l(x)$ is the $l^{\text{th}}$ \textbf{Legendre polynomial} defined by $$P_l(x) = \frac{1}{2^l l!} \bigg(\frac{d}{dx} \bigg)^{l}(x^2 -1)^l$$ \vspace{3mm} The angular function $Y^m_l(\theta, \phi) = A_l^m P_l^m(\cos \theta)e^{im\phi}$ may be normalized by setting $$A_l^m = \ep \sqrt{\frac{(2l+1)}{4\pi} \frac{(l-\vert m \vert)!}{(l+ \vert m \vert)!}}$$ where $$\ep = 
\begin{cases}
(-1)^m & m \geq 0\\
1 & m < 0
\end{cases}$$ The normalized angular functions are called \textbf{spherical harmonics}.
\end{defn}

\begin{ex}
Compute some spherical harmonics.
\end{ex}

\begin{defn}
If we make the substitution $u(r) = rR(r)$, we may rewrite the radial equation as $$-\frac{\hbar^2}{2m}\frac{d^2u}{dr^2} + \bigg[V+ \frac{\hbar^2}{2m}\frac{l(l+1)}{r^2} \bigg]u = Eu$$ which looks like the one dimensional \sch equation. The function $$V+ \frac{\hbar^2}{2m}\frac{l(l+1)}{r^2}$$ is called the \textbf{effective potential}.
\end{defn}

\subsection{Spherical Harmonic Oscillator (Cartesian Coordinates)}

\begin{defn}
The spherical harmonic oscillator (in cartesian coordinates) is defined by the potential energy
$$V(x,y,z) = x^2 + y^2 + z^2$$
\end{defn}

\begin{ex}
In cartesian coordinates, the the stationary states of the harmonic oscillator are given by $$\psi_{n_x, n_y, n_z}(x,y,z) = \psi_{n_x}(x)\psi_{n_y}(y)\psi_{n_z}(z)$$ with energies $$E_{n_x,n_y, n_z} = \hbar \om \bigg (n_x + n_y + n_z + \frac{3}{2} \bigg)$$ where $\psi_{n_x}, \psi_{n_y}, \psi_{n_z}$ are stationary states for the one dimensional harmonic oscillator.
\end{ex}

\begin{proof}
We look for solutions of the form $\psi(x,y,z) = \psi_x(x) \psi_y(y) \psi_z(z)$. Plugging this into the time-independent \sch equation, we get $$-\frac{\hbar^2}{2m}\bigg[ \pdv[2]{\psi_x}{x}\psi_y \psi_z + \psi_x \pdv[2]{\psi_y}{y} \psi_z  + \psi_x \psi_y \pdv[2]{\psi_z}{z} \bigg] + \frac{1}{2}m \omega^2(x^2 + y^2 + x^2) \psi = E\psi$$ Dividing both sides by $\psi$ and rearranging, we obtain $$\bigg(-\frac{\hbar^2}{2m}\pdv[2]{\psi_x}{x}\frac{1}{\psi_x} + \frac{1}{2}m \om^2x^2\bigg) + \bigg(-\frac{\hbar^2}{2m}\pdv[2]{\psi_y}{y}\frac{1}{\psi_y} + \frac{1}{2}m \om^2y^2\bigg)+ \bigg(-\frac{\hbar^2}{2m}\pdv[2]{\psi_z}{z}\frac{1}{\psi_z} + \frac{1}{2}m \om^2z^2\bigg) = E $$ 
Thus each part is constant and we may write 
\begin{align*}
-\frac{\hbar^2}{2m}\pdv[2]{\psi_x}{x} + \frac{1}{2}m \om^2x^2\psi_x = E_x\psi_x \\ 
-\frac{\hbar^2}{2m}\pdv[2]{\psi_y}{y} + \frac{1}{2}m \om^2y^2\psi_y = E_y\psi_y \\
-\frac{\hbar^2}{2m}\pdv[2]{\psi_z}{z} + \frac{1}{2}m \om^2z^2\psi_z = E_z\psi_z
\end{align*}

So we have three one-dimensional harmonic oscillators and we have 
\begin{align*}
\psi_x = \psi_{n_x} = \frac{1}{\sqrt{{n_x}!}}(a^{\dagger})^{n_x} \psi_0 \text{ and } E_x = E_{n_x} = \hbar \om \bigg( n_x +\frac{1}{2}\bigg)\\
\psi_y = \psi_{n_y} = \frac{1}{\sqrt{{n_y}!}}(a^{\dagger})^{n_y} \psi_0 \text{ and } E_y = E_{n_y} = \hbar \om \bigg( n_y +\frac{1}{2}\bigg)\\
\psi_z = \psi_{n_z} = \frac{1}{\sqrt{{n_z}!}}(a^{\dagger})^{n_z} \psi_0 \text{ and } E_z = E_{n_z} = \hbar \om \bigg( n_z +\frac{1}{2}\bigg)
\end{align*}
Thus $$\psi = \psi_{n_x, n_y, n_z}(x,y,z) = \psi_{n_x}(x)\psi_{n_y}(y)\psi_{n_z}(z)$$ with energy $$E = E_{n_x,n_y, n_z} = \hbar \om \bigg (n_x + n_y + n_z + \frac{3}{2} \bigg)$$
\end{proof}

\begin{ex}
Show that the degeneracy of $E_n$ is $$deg(E_n) = {n+2 \choose 2}$$
\end{ex}

\begin{proof}
Stars and bars
\end{proof}

\begin{intp}
The energies of the three-dimensional harmonic oscillator are given by $E_n = \hbar \om \bigg( n + \frac{3}{2}\bigg)$ which correspond to $ n+2 \choose 2$ stationary states.
\end{intp}

\subsection{Spherical Harmonic Oscillator (Spherical Coordinates)}
\begin{defn}
The spherical harmonic oscillator (in spherical coordinates) is defined by the potential energy
$$V(r) = r^2$$
\end{defn}

\begin{ex}
Making the substitution $\kappa = \frac{\sqrt{2mE}}{\hbar}$, we can rewrite the radial equation for the harmonic oscillator as $$\frac{1}{\kap^2}\dv[2]{u}{r} = \bigg[ \frac{\hbar^2 \om^2 (\kap r)^2}{2^2E^2} + \frac{l(l+1)}{(\kap r)^2} - 1\bigg]u$$
\end{ex}

\begin{proof}
Straightforward
\end{proof}

\begin{ex}
Making the substitution $\rho = \kap r$ and $\rho_0 = \frac{\hbar \om}{2 E}$, we can rewrite the radial equation as $$\frac{1}{\kap^2}\dv[2]{u}{r} = \bigg[ \rho_0^2 \rho^2 + \frac{l(l+1)}{\rho^2} - 1\bigg]u$$
\end{ex}

\begin{proof}
Straightforward.
\end{proof}

\begin{ex}
We have $$\dv[2]{u}{\rho} = \frac{1}{\kap^2}\dv[2]{u}{r}$$ and thus we may rewrite the radial equation as $$\dv[2]{u}{\rho} = \bigg[ \rho_0^2 \rho^2 + \frac{l(l+1)}{\rho^2} - 1\bigg]u$$
\end{ex}

\begin{proof}
Straightforward by chain-rule.
\end{proof}

\begin{ex}
As $ \rho \rightarrow \infty$, $u \approx e^{-\frac{\rho_0}{2}\rho^2}$
\end{ex}

\begin{proof}
As $\rho \rightarrow \infty$, $$\dv[2]{u}{\rho} \approx \rho_0^2 \rho^2u$$ Trying the function $u(\rho) = e^{-\frac{\rho_0}{2}\rho^2}$, we see that
\begin{align*}
\dv[2]{u}{\rho} 
&= (\rho_0^2 \rho^2 - \rho_0)e^{-\frac{\rho_0}{2}\rho^2}\\
& \approx \rho_0^2 \rho^2 e^{-\frac{\rho_0}{2}\rho^2} \hspace{.5cm} \text{(as }  \rho  \rightarrow \infty \text{)}\\
&=  \rho_0^2 \rho^2 u
\end{align*}
\end{proof}

\begin{ex}
As $\rho \rightarrow 0$, $u \approx \rho^{l+1}$
\end{ex}

\begin{proof}
As $\rho \rightarrow 0$, $$\dv[2]{u}{\rho} \approx \frac{l(l+1)}{\rho^2}u$$
Trying the function $u(\rho) = \rho^{l+1}$, we see that 
\begin{align*}
\dv[2]{u}{\rho} 
&= l(l+1)\rho^{l-1}\\
&= \frac{l(l+1)}{\rho^2}u
\end{align*}
\end{proof}

\begin{note}
We can now, ``glue" these functions together with a third unknown function $v(\rho)$ to obtain the prototype solution $$u(\rho) = \rho^{l+1}e^{-\frac{\rho_0}{2}\rho^2}v(\rho)$$
\end{note}

\begin{ex}
Suppose that for some nice function $v(\rho)$, $$u(\rho) = \rho^{l+1}e^{-\frac{\rho_0}{2}\rho^2}v(\rho)$$ Then computing $\dv[2]{u}{\rho}$ and plugging into the radial equation and simplifying, we obtain the relation $$\rho \dv[2]{v}{\rho} +2(l+1 - \rho_0\rho^2)\dv{v}{\rho} + \rho(1-\rho_0(2l+3))v = 0$$
\end{ex}

\begin{proof}
Very tedious but straightforward.
\end{proof}

\begin{ex}
If $v(\rho)$ can be represented by a power series $$v(\rho) = \sum_{j=0}^{\infty}c_j\rho^j$$ then plugging in $v(\rho)$ into the previous relation combining like terms and solving for the coefficients yields the relations $$c_1 = 0$$ and $$c_{j+2} = \bigg[ \frac{\rho_0(2j+2l+3)-1}{(j+2)(j+2l+3)}\bigg]c_j \hspace{5mm} j \geq 0$$ \vspace{3mm} \\This implies that for each odd $j$, $c_j = 0$. \vspace{3mm}
\end{ex}

\begin{proof}
Tedious but straightforward.
\end{proof}

\begin{ex}
If for each $j \geq 0$, $c_{2j} \neq 0$, then $v$ behaves asymptotically like $e^{\rho_0\rho^2}$. Thus $u(\rho)$ behaves asymptotically like $\rho^{l+1}e^{\frac{\rho_0}{2}\rho^2}$. This implies that $R(r)$ is not normalizable. Therefore there exists $j_{max} \geq 0$ such that $c_{2j+2} = 0$ and $v(\rho)$ is a polynomial of degree $2j_{max}$ and consists of only even powers of $\rho$. 
\end{ex}

\begin{proof}
As $j \rightarrow \infty$, $c_{j+2} \approx \frac{2 \rho_0}{j}c_j$. Hence $v(\rho)$ behaves asymptotically like 
\begin{align*}
\sum_{j=0}^{\infty}\frac{2^j\rho_0^j}{\prod_{k=1}^j2k}\rho^{2j}
&= \sum_{j=0}^{\infty}\frac{(\sqrt{\rho_0} \rho)^{2j}}{j!}\\
&= e^{(\sqrt{\rho_0}\rho)^2}\\
&= e^{\rho_0 \rho^2}
\end{align*}
\end{proof}





\subsection{The Infinite Spherical Box}

\subsection{The Hydrogen Atom}








\subsection{Orbital Angular Momentum}



\end{document}

