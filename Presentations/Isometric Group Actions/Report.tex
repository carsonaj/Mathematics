\documentclass[12pt]{amsart}
\usepackage[margin=1in]{geometry} 
\usepackage{amsmath,amsthm,amssymb,setspace, mathtools, enumitem}
\usepackage{physics}
\usepackage{tikz-cd} 

\usepackage{color}   %May be necessary if you want to color links
\usepackage{hyperref}
\hypersetup{
	colorlinks=true, %set true if you want colored links
	linktoc=all,     %set to all if you want both sections and subsections linked
	linkcolor=black,  %choose some color if you want links to stand out
	urlcolor=cyan
}


%
%
%
\newif\ifhideproofs
%\hideproofstrue %uncomment to hide proofs
%
%
%
%
\ifhideproofs
\usepackage{environ}
\NewEnviron{hide}{}
\let\proof\hide
\let\endproof\endhide
\fi

\theoremstyle{definition}
\newtheorem{definition}{Definition}[subsection]
\newtheorem{defn}[definition]{Definition}
\newtheorem{note}[definition]{Note}
\newtheorem{thm}[definition]{Theorem}
\newtheorem{lem}[definition]{Lemma}
\newtheorem{prop}[definition]{Proposition}
\newtheorem{cor}[definition]{Corollary}
\newtheorem{conj}[definition]{Conjecture}
\newtheorem{ex}[definition]{Exercise}


\newcommand{\al}{\alpha}
\newcommand{\gam}{\gamma}
\newcommand{\Gam}{\Gamma}
\newcommand{\be}{\beta} 
\newcommand{\del}{\delta} 
\newcommand{\Del}{\Delta}
\newcommand{\lam}{\lambda}  
\newcommand{\Lam}{\Lambda} 
\newcommand{\ep}{\epsilon}
\newcommand{\sig}{\sigma} 
\newcommand{\om}{\omega}
\newcommand{\Om}{\Omega}
\newcommand{\C}{\mathbb{C}}
\newcommand{\N}{\mathbb{N}}
\newcommand{\E}{\mathbb{E}}
\newcommand{\Z}{\mathbb{Z}}
\newcommand{\R}{\mathbb{R}}
\newcommand{\T}{\mathbb{T}}
\newcommand{\Q}{\mathbb{Q}}
\renewcommand{\P}{\mathbb{P}}
\newcommand{\MA}{\mathcal{A}}
\newcommand{\MC}{\mathcal{C}}
\newcommand{\MB}{\mathcal{B}}
\newcommand{\MF}{\mathcal{F}}
\newcommand{\MG}{\mathcal{G}}
\newcommand{\ML}{\mathcal{L}}
\newcommand{\MN}{\mathcal{N}}
\newcommand{\MS}{\mathcal{S}}
\newcommand{\MP}{\mathcal{P}}
\newcommand{\ME}{\mathcal{E}}
\newcommand{\MT}{\mathcal{T}}
\newcommand{\MM}{\mathcal{M}}
\newcommand{\MI}{\mathcal{I}}
\newcommand{\MU}{\mathcal{U}}
\newcommand{\MO}{\mathcal{O}}
\newcommand{\MX}{\mathcal{X}}

\newcommand{\ui}{[0,1]}
\newcommand{\p}{\partial}

\newcommand{\io}{\text{ i.o.}}
%\newcommand{\ev}{\text{ ev.}}
\renewcommand{\r}{\rangle}
\renewcommand{\l}{\langle}

\newcommand{\RG}{[0,\infty]}
\newcommand{\Rg}{[0,\infty)}
\newcommand{\Ru}{(\infty, \infty]}
\newcommand{\Rd}{[\infty, \infty)}
\newcommand{\Ll}{L^1_{\text{loc}}(\R^n)}

\newcommand{\limfn}{\liminf \limits_{n \rightarrow \infty}}
\newcommand{\limpn}{\limsup \limits_{n \rightarrow \infty}}
\newcommand{\limn}{\lim \limits_{n \rightarrow \infty}}
\newcommand{\convt}[1]{\xrightarrow{\text{#1}}}
\newcommand{\conv}[1]{\xrightarrow{#1}} 
\newcommand{\seq}[2]{(#1_{#2})_{#2 \in \N}}

\newcommand{\lsc}{l.s.c. }

\newcommand{\as}[1]{\overset{#1}{\sim}}
\newcommand{\astx}[1]{\overset{\text{#1}}{\sim}}

\DeclareMathOperator{\supp}{supp}
\DeclareMathOperator{\sgn}{sgn}
\DeclareMathOperator{\spn}{span}
\DeclareMathOperator{\iso}{Iso}
\DeclareMathOperator{\id}{id}
\DeclareMathOperator{\argmax}{arg\,max}
\DeclareMathOperator{\argmin}{arg\,min}
\DeclareMathOperator{\Aut}{Aut}
\DeclareMathOperator{\Homeo}{Homeo}
\DeclareMathOperator{\Sym}{Sym}
\DeclareMathOperator{\cl}{cl}
\DeclareMathOperator{\intt}{int}
\DeclareMathOperator{\diam}{diam}



\newcommand{\lex}[1]{\label{ex:#1}}
\newcommand{\ld}[1]{\label{defn:#1}}
\newcommand{\rex}[1]{Exercise \ref{ex:#1}}
\newcommand{\rd}[1]{Definition \ref{defn:#1}}


\begin{document}
	
	\title{Orbit Space Metrics and Measures Induced by Isometric Group Actions}
	\author{Carson James}
	\maketitle
	
	\tableofcontents
	
	
	\section{Introduction}
	
	\subsection{Main Idea}
	In these notes we do the following: 
	\begin{itemize}
	\item for an isometric group action on metric spaces, we define an induced metric on the orbit space which metrizes the quotient topology
	\item for nice measures on metric spaces in the above case, we define nice induced measure on the orbit space
	\item give an application to Bayesian statistics
	\end{itemize}
	
	
	
	
	
	
	
	
	
	
	
	
	
	
	\newpage
	\section{Group Actions on Metric Spaces}
	
	\subsection{Introduction}
	\begin{note}
	For a set $X$, a group $G$ and a (left) group action $\phi: G \times X \rightarrow X$, we will write $\phi(g, x)$ as $g \cdot x$. We denote the projection map by $\pi: X \rightarrow X/G$.
	\end{note}	
	
	\begin{defn} \ld{00000} 
		Let $X$ be a set, $G$ a group, $\phi: G \times X \rightarrow X$ a group action and $g \in G$. Define $l_g:X \rightarrow X$ by 
		\begin{equation*}
		l_g(x) = g \cdot x
		\end{equation*}
	\end{defn}
	
	\begin{defn}
	Let $X$ be a topological space, $G$ a group and $\phi: G \times X \rightarrow X$ a group action. Then $\phi$ is said to be $X$-continuous if for each $g \in G$, $l_g$ is continuous.
	\end{defn}
	
	\begin{ex}
	Let $X$ be a topological space, $G$ a group and $\phi: G \times X \rightarrow X$ an $X$-continuous group action. Then for each $g \in G$, $l_g \in \Homeo(X)$.
	\end{ex}
	
	\begin{proof}
	Let $g \in G$, then $l_g$ and $l_{g}^{-1} = l_{g^{-1}}$ are continuous, so $l_g \in \Homeo(G)$. 
	\end{proof}
	
	\begin{defn} \ld{}
	Let $(X, d)$ be a metric space, $G$ a group, and $\phi: G \times X \rightarrow X$ a group action. Then $\phi$ is said to be an \textbf{isometric group action} if for each $g \in G$, $l_g:X \rightarrow X$ is an isometry. 
	\end{defn}
	
	\begin{ex}
	Let $(X, d)$ be a metric space, $G$ a group, and $\phi: G \times X \rightarrow X$ an isometric group action. Then $\phi$ is $X$-continuous.
	\end{ex}
	
	\begin{proof}
	Clear since isometries are continuous.
\end{proof}		
	
	\begin{defn}
	Let $X$ be a set, $G$ a group and $\phi: G \times X \rightarrow X$ an $X$-continuous group action. Let $g \in G$. Define $L_g:\C^X \rightarrow \C^X$ by 
	\begin{align*}
	L_g(f)(x) 
	&= f \circ l_g^{-1} \\
	&= f \circ l_{g^{-1}}
	\end{align*}
	\end{defn}
	
	
	\begin{defn}
	Let X be a set, $G$ a group, $\phi: G \times X \rightarrow X$ a group action and $f:X \rightarrow \C$. Then $f$ is said to be \textbf{$G$-invariant} if for each $g \in G$, $L_g f = f$.
	\end{defn}
	
	\begin{ex}
	Let X be a set, $G$ a group, $\phi: G \times X \rightarrow X$ a group action and $f:X \rightarrow \C$. Then $f$ is $G$-invariant iff for each $g \in G$ $x \in X$, $f(g \cdot x) = f(x)$.  
	\end{ex}
	
	\begin{proof}
	Clear.
	\end{proof}
	
	\begin{defn}
	Let X be a set, $G$ a group, $\phi: G \times X \rightarrow X$ a group action and $f:X \rightarrow \C$. Suppose that $f$ is $G$-invariant. Define $\bar{f}:X/ G \rightarrow \C$ by $\bar{f}(\bar{x}) = f(x)$. 
	\end{defn}
	
	\begin{ex}
	Let X be a set, $G$ a group, $\phi: G \times X \rightarrow X$ a group action and $f:X \rightarrow \C$. Suppose that $f$ is $G$-invariant. Then $f = \bar{f} \circ \pi$. 
	\end{ex}
	
	\begin{proof}
	Clear.
	\end{proof}
	
	
	
	
	
	
	
	
	
	\newpage
	\subsection{Induced Metrics on Orbit Spaces}
	
	\begin{note}
	This section establishes the criteria for the existence of a metric on the orbit space of a metric space under a group action. 
	\end{note}
	
	\begin{defn} \ld{}
	Let $(X, d)$ be a metric space, $G$ a group, and $\phi: G \times X \rightarrow X$ a group action. We define 
	$\bar{d}: X/G \times X / G \rightarrow \Rg$ by 
	$$\bar{d}(\bar{x}, \bar{y}) = \inf_{\substack{a \in \bar{x} \\ b \in \bar{y}}} d(a,b) $$
	\end{defn}
	
	\begin{ex} \lex{}
	Let $(X, d)$ be a metric space, $G$ a group, and $\phi: G \times X \rightarrow X$ an isometric group action. Then for each $x, y \in X$, $$\bar{d}(\bar{x}, \bar{y}) = \inf_{g \in G} d(g \cdot x, y)$$
	\end{ex}
	
	\begin{proof}
	Let $x, y \in X$, $a \in \bar{x}$ and $b \in \bar{y}$. Then there exists there exists $g_a, g_b \in G$ such that $a = g_a \cdot x$ and $b = g_b \cdot y$. Set $g = g_b^{-1}g_a$. Since the map $z \mapsto g_b^{-1} \cdot z$ is an isometry, 
	\begin{align*}
	d(a,b) 
	&= d(g_a \cdot x, g_b \cdot y) \\
	&= d(g_b^{-1}g_a \cdot x, y)\\
	&= d(g\cdot x, y)
	\end{align*}
	Let $\ep >0$. Then there exist $a^* \in \bar{x}$ and $b^* \in \bar{y}$ such that $d(a^*,b^*) < \bar{d}(\bar{x},\bar{y}) + \ep$. The above argument implies that that there exists $g^* \in G$ such that 
	\begin{align*} 
	\inf_{g \in G} d(g \cdot x, y) 
	& \leq d(g^* \cdot x, y) \\
	&= d(a^*, b^*) \\
	& < \bar{d}(\bar{x}, \bar{y}) + \ep
\end{align*}	 
	Since $\ep >0$ is arbitrary, $$\inf_{g \in G} d(g \cdot x, y) \leq \bar{d}(\bar{x}, \bar{y})$$
	Conversely, since $\{(g \cdot x, y): g \in G\} \subset \{(a,b): a \in \bar{x}, b \in \bar{y}\}$, we have that 
	$$\inf_{g \in G} d(g \cdot x, y) \geq \bar{d}(\bar{x}, \bar{y})$$ 
	\end{proof}
	
	\begin{ex} \lex{}
	Let $(X, d)$ be a metric space, $G$ a group, and $\phi: G \times X \rightarrow X$ an isometric group action. Then for each $x, y, z \in X$, $$\bar{d}(\bar{x}, \bar{y}) \leq \bar{d}(\bar{x}, \bar{z}) + \bar{d}(\bar{z}, \bar{y})$$
	\end{ex}
	
	\begin{proof}
	Let $x, y, z \in X$. An exercise in section $(2.1)$ implies that $d(\bar{x}, \bar{y}) \leq d(\bar{x}, z) + d(z, \bar{y})$. The previous exercise implies that 
	\begin{align*}
	d(\bar{x}, z) 
	&= \inf_{a \in \bar{x}} d(a, z) \\
	&= \inf_{g \in G} d(g \cdot x, z) \\
	&= \bar{d}(\bar{x}, \bar{z})
	\end{align*}
	Similarly, $d(z, \bar{y}) = \bar{d}(\bar{z}, \bar{y})$. Then 
	\begin{align*}
	d(\bar{x}, \bar{y}) 
	&\leq d(\bar{x}, z) + d(z, \bar{y}) \\
	&= \bar{d}(\bar{x}, \bar{z}) + \bar{d}(\bar{z}, \bar{y})
	\end{align*}
	\end{proof}
	
	\begin{ex} \lex{}
	Let $(X, d)$ be a metric space, $G$ a group, and $\phi: G \times X \rightarrow X$ an isometric group action. If for each $x \in X$, $\bar{x}$ is closed, then for each $x, y \in X$, $\bar{d}(\bar{x}, \bar{y}) =0$ implies that $\bar{x} = \bar{y}$.
	\end{ex}
	
	\begin{proof}
	Suppose that for each $x \in X$, $\bar{x}$ is closed. Let $x,y \in X$. Suppose that $\bar{d}(\bar{x} , \bar{y}) = 0$. Then $\inf\limits_{ g \in G} d(g \cdot x, y) = 0$. Hence there exists $(g_n)_{n \in N} \subset G$ such that $g_n \cdot x \rightarrow y$. Since $(g_n \cdot x)_{n \in \N} \subset \bar{x}$ and $\bar{x}$ is closed, $y \in \bar{x}$. Thus $\bar{x} = \bar{y}$. 
	\end{proof}
	
	\begin{ex} \lex{}
	Let $(X, d)$ be a metric space, $G$ a group, and $\phi: G \times X \rightarrow X$ an isometric group action. If for each $x \in X$, $\bar{x}$ is closed, then $\bar{d}$ is a metric on $X/G$.
	\end{ex}
	
	\begin{proof}
	Clear by preceeding exercises.
	\end{proof}
	
	\begin{ex} \lex{}
	Let $(X, d)$ be a metric space, $(G, \tau)$ a topological group, and $\phi: G \times X \rightarrow X$ an isometric group action. Suppose that $G$ is compact and for each $x \in X$, the map $g \mapsto g \cdot x$ is continuous. Then $\bar{d}$ is a metric on $X/G$. 
	\end{ex}
	
	\begin{proof}
	Let $x \in X$. Since $G$ is compact and the map $g \mapsto g \cdot x$ is continuous, $\bar{x} = G \cdot x$ is compact and therefore closed. The previous exercise implies that $\bar{d}$ is a metric.
	\end{proof}
	
	\begin{ex} \lex{}
	Let $(X, d)$ be a metric space, $G$ a group, and $\phi: G \times X \rightarrow X$ an isometric group action. Suppose that $\bar{d}$ is a metric on $X/G$. Then the projection map $\pi: X \rightarrow X/G$ is Lipschitz and therefore continuous.
	\end{ex}
	
	\begin{proof}
	Let $x,y \in X$. Then
	\begin{align*}
	\bar{d}(\pi(x), \pi(y)) 
	&= \bar{d}(\bar{x}, \bar{y}) \\
	&= \inf_{g \in G} d(g \cdot x, y)\\
	& \leq d(x,y)  \\
	\end{align*}
	\end{proof}
	
	\begin{ex} \lex{}
	Let $(X, d)$ be a metric space, $G$ a group, and $\phi: G \times X \rightarrow X$ an isometric group action. Suppose that $\bar{d}$ is a metric on $X/G$. Let $(x_n)_{n \in \N} \subset X$ and $x \in X$. Then $\bar{x}_n \conv{\bar{d}} \bar{x}$ iff there exists a sequence $(g_n)_{n \in \N}$ such that $g_n \cdot x_n \conv{d} x$.
	\end{ex}
	
	\begin{proof} 
	Suppose that $\bar{x}_n \conv{\bar{d}} \bar{x}$. For $n \in \N$, choose $g_n \in G$ such that $d(g_n \cdot x_n, x) < \bar{d}(\bar{x}_n, \bar{x}) + 2^{-n}$. Then $d(g_n \cdot x_n, x) \rightarrow 0$ and $g_n \cdot x_n \conv{d} x$.  \\
	Conversely, suppose that that there exists a sequence $(g_n)_{n \in \N}$ such that $g_n \cdot x_n \conv{d} x$. Since $\pi:X \rightarrow X/G$ is continuous, we have that
	\begin{align*}
	g_n \cdot x_n \conv{d} x
	& \implies \pi(g_n \cdot x_n) \conv{\bar{d}} \pi(x)\\
	& \implies \bar{x}_n  \conv{\bar{d}} \bar{x}
	\end{align*}
	\end{proof}		
	
	\begin{ex} \lex{}
	Let $X$ be a set, $d_1, d_2: X^2 \rightarrow \Rg$ metrics, $G$ a group and $\phi: G \times X \rightarrow X$ an isometric group action. Suppose that $d_1$ and $d_2$ are topologically equivalent. 
	\begin{enumerate}
	\item Then $\bar{d}_1$ is a metric on $X/G$ iff $\bar{d}_2$ is a metric on $X/G$
	\item If $\bar{d}_1$ and $\bar{d}_2$ are metrics, then $\bar{d}_1$ and $\bar{d}_2$ are topologically equivalent. 
	\end{enumerate}
	\end{ex}
	
	\begin{proof}\
	\begin{enumerate}
	\item 
	\begin{itemize}
	\item $\implies$ Suppose that $\bar{d}_1$ is a metric. Let $x,y \in X$. Suppose that $\bar{d}_2(\bar{x}, \bar{y}) = 0$. Then there exist $(g_n)_{n \in \N} \subset G$ such that $d_2(g_n \cdot x, y) \rightarrow 0$. Since $d_1$ and $d_2$ are topologically equivalent, $d_1(g_n \cdot x, y) \rightarrow 0$. Thus $\bar{d}_1(\bar{x}, \bar{y}) = 0$. Since $\bar{d}_1$ is a metric, $\bar{x} = \bar{y}$. Hence $\bar{d}_2$ is a metric. 
	\item $\impliedby$ Similar.
	\end{itemize}
	\item Suppose that $\bar{d}_1$ and $\bar{d}_2$ are metrics. Let $(\bar{x}_n)_{n \in \N} \subset X/G$ and $\bar{x} \in X/G$. 
	\begin{itemize}
	\item Suppose that $\bar{x}_n \conv{\bar{d}_1} \bar{x}$. Then there exists a sequence $(g_n)_{n \in \N}$ such that $g_n \cdot x_n \conv{d_1} x$. Since $d_1$ and $d_2$ are topologically equivalent, $g_n \cdot x_n \conv{d_2} x$. This implies that $\bar{x}_n \conv{\bar{d}_2} \bar{x}$. 
	\item Suppose that $\bar{x}_n \conv{\bar{d}_2} \bar{x}$. Then similarly to above, $\bar{x}_n \conv{\bar{d}_1} \bar{x}$.
	\end{itemize}
	\end{enumerate}
	\end{proof}	
	
	\begin{ex} \lex{}
	Let $X$ be a set, $d_1, d_2: X^2 \rightarrow \Rg$ metrics on $X$, $G$ a group and $\phi: G \times X \rightarrow X$ an isometric group action. Suppose that $d_1$ and $d_2$ are equivalent. If $\bar{d}_1$ and $\bar{d}_2$ are metrics, then $\bar{d}_1$ and $\bar{d}_2$ are equivalent.
	\end{ex}
	
	\begin{proof} Suppose that $\bar{d}_1$ and $\bar{d}_2$  are metrics. Since $d_1$ $d_2$ are equivalent, there exist $C_1, C_2 >0$ such that for each $x,y \in X$, $C_1d_1(x,y) \leq d_2(x,y) \leq C_2d_1(x,y)$. Let $x,y \in X$. Then
	\begin{align*}
	C_1\bar{d}_1(\bar{x}, \bar{y}) 
	&= C_1 \inf_{g \in G} d_1(g \cdot x, y) \\
	&=  \inf_{g \in G} C_1 d_1(g \cdot x, y) \\
	&\leq \inf_{g \in G} d_2(g \cdot x, y) \\
	&= \bar{d}_2(\bar{x}, \bar{y}) \\
	\end{align*}	 
	and 
	\begin{align*}
	\bar{d}_2(\bar{x}, \bar{y}) 
	&= \inf_{g \in G} d_2(g \cdot x, y) \\	
	& \leq \inf_{g \in G} C_2 d_1(g \cdot x, y) \\
	&= C_2 \inf_{g \in G}  d_1(g \cdot x, y) \\
	&= C_2 \bar{d}_1(\bar{x}, \bar{y})
	\end{align*}
	So that $C_1 \bar{d}_1 \leq \bar{d}_2 \leq C_2 \bar{d}_1$
	\end{proof}
	
	\begin{ex}
	Let $(X,d)$ be a metric space, $G$ a group and $\phi: G \times X \rightarrow X$ an isometric group action. Suppose that $\bar{d}$ is a metric. Then $\pi:X \rightarrow X/G$ is a quotient map.
	\end{ex}
	
	\begin{proof}\
	\begin{itemize}
	\item Clearly $\pi$ is surjective. 
	\item Let $C \subset X/G$. Suppose that $C$ is closed. Since $\pi$ is continuous, if $\pi^{-1}(C)$ is closed. \\
	Conversely, suppose that $\pi^{-1}(C)$ is closed. Let $(\bar{x}_{\al})_{\al} \subset C$ be a net and $\bar{x} \in X/G$. Suppose that $\bar{x}_{\al} \rightarrow \bar{x}$. Then there exists $(g_{\al})_{\al \in A} \subset G$ such that $g_{\al} \cdot x_{\al} \rightarrow x$. Since $(g_{\al} \cdot x_{\al})_{\al \in A} \subset \pi^{-1}(C)$, $x \in \pi^{-1}(C)$. Hence $\bar{x} \in C$ and $C$ is closed. Then \rex{34003} implies that $\pi$ is a quotient map.
	\end{itemize}
	\end{proof}
	
	\begin{ex}
	Let $(X,d)$ be a metric space, $G$ a group and $\phi: G \times X \rightarrow X$ an isometric group action. Suppose that $\bar{d}$ is a metric. Then $\pi:X \rightarrow X/G$ is open.
	\end{ex}
	
	\begin{proof}
	Let $U \subset X$. Suppose that $U$ is open. Then 
	\begin{equation*}
	\pi^{-1}(\pi(U)) = \bigcup_{g \in G} g \cdot U
	\end{equation*}		
	Since for each $g \in G$, $l_g \in \Homeo(X)$, we have that for each $g \in G$, $g \cdot U$ is open. Therefore $\bigcup\limits_{g \in G} g \cdot U$ is open. Hence $\pi^{-1}(\pi(U))$ is open. Then \rex{34005} implies that $\pi$ is open.
	\end{proof}
	
	\begin{ex}
	Let $(X,d)$ be a metric space, $G$ a group and $\phi: G \times X \rightarrow X$ an isometric group action. Suppose that $\bar{d}$ is a metric. Then $\bar{d}$ metrizes the quotient topology $\pi_*\tau(d)$ on $X/G$.
	\end{ex}
	
	\begin{proof}
	Immediate by the previous exercise and \rex{34008}.
	\end{proof}
	
	\begin{ex}
	Let $(X, d)$ be a metric space, $G$ a group, and $\phi: G \times X \rightarrow X$ an isometric group action. Let $f: X \rightarrow \C$. Suppose that $f$ is $G$-invariant. Suppose that $\bar{d}$ is a metric. If $f \in C(X)$, then $\bar{f} \in C(X/G)$.  \\
	\textbf{Hint:} Doob-Dynkin Lemma
	\end{ex}
	
	\begin{proof}
	Suppose that $f \in C(X)$. Let $(x_{\al})_{\al \in A}$ be a net in $X$ and $x \in X$. Suppose that $x_{\al} \rightarrow x$ in the $\tau(\pi)$ topology. Then $\bar{x}_{\al} \rightarrow \bar{x}$. This implies that there exists $(g_{\al})_{\al \in A} \subset G$ such that $g_{\al} \cdot x_{\al} \conv{d} x$. Since $f$ is $G$-invariant and continuous, we have that 
	\begin{align*}
	f(x_{\al})
	&= f(g_{\al} \cdot x_{\al}) \\
	& \rightarrow f(x)
	\end{align*}
	So $f$ is $\tau(\pi)$-$\tau(\C)$ continuous. The Doob-Dynkin lemma for continuous functions implies that there exists a continuous unique $g:X/G \rightarrow \C$ such that $f = g \circ \pi$. Since $f = \bar{f} \circ \pi$, we have that $\bar{f} = g$ and $\bar{f}$ is continuous.
	\end{proof}
	
	\begin{note}
	I would have liked to show that $f$ is $\sig(\pi)$-$\MB(\C)$ measurable and used the Doob-Dynkin lemma for measurable functions to show that $\bar{f}$ is measurable, but was unable to do this.
	\end{note}
	
	
	
	
	
	
	
	
	
	
	
	
	
	
	
	
	
	
	
	\newpage
	\subsection{Induced Measures on Isometric Orbit Spaces}
	
	\begin{note}
	This section assumes familiarity with induced metrics on orbit spaces of metric spaces under isometric group actions. See section $9.1$ of \cite{analysis} for details. 
	\end{note}
	
	\begin{note}
	
	\end{note}
	
	\begin{defn}
	Let $(X, d)$ be a metric space, $G$ a group, and $\phi: G \times X \rightarrow X$ an isometric group action. Suppose that $(X/G, \bar{d})$ is a metric space. Let $\mu: \MB(X) \rightarrow [0, \infty]$ be a measure on $X$. We define $\bar{\mu}: \MB(X/G) \rightarrow [0, \infty]$ by $\bar{\mu} = \pi_* \mu$. 
	\end{defn}
	
	\begin{note}
	If $\mu \ll H_p^X$, where $X$ has Hausdorff dimension $p$, I want to be able to define $\bar{\mu}$ in terms of $H_q^{X/G}$ where $X/G$ has Hausdorff dimension $q$. I was unable to do this. It might be possible with some manifold theory, for instance $O(2)$ acting on $\R^2$.
	\end{note}
	
	\begin{defn}
	Let $(X, d)$ be a metric space, $G$ a group, and $\phi: G \times X \rightarrow X$ an isometric group action. Suppose that $(X/G, \bar{d})$ is a metric space. Let $\mu: \MB(X) \rightarrow [0, \infty]$ be a measure on $X$. Then $\mu$ is said to be $G$-invariant if for each $g \in G$, $U \in \MB(X)$, 
	\begin{equation*}
	\mu(g \cdot U) = \mu(U)
	\end{equation*}
	\end{defn}
	
	\begin{ex}
	Let $X$ be a metric space, $G$ a group, and $\phi: G \times X \rightarrow X$ an isometric group action. Then for each $p \geq 0$, $H_p$ is $G$-invariant. 
	\end{ex}	
	
	\begin{proof}
	Clear.
	\end{proof}
	
	\begin{ex}
	Let $X$ be a metric space, $G$ a group, and $\phi: G \times X \rightarrow X$ an isometric group action. Let $\mu: \MB(X) \rightarrow [0, \infty]$ be a measure on $X$. Suppose that $\mu \ll H_p$. Then $\mu$ is $G$-invariant iff $d\mu /d H_p$ is $G$-invariant.
	\end{ex}	
	
	\begin{proof}
	Suppose that $\mu$ is $G$-invariant. Let $g \in G$ and $U \in \MB(X)$. Then 
	\begin{align*}
	\int_U L_g \frac{d\mu}{d H_p}(x) \, d  H_p (x)
	&= \int_U \frac{d\mu}{d H_p} \circ l_{g}^{-1}(x) \, d  H_p(x) \\
	&= \int_{l_{g}^{-1}( U) } \frac{d\mu}{d H_p}(x) \, d (l_{g}^{-1})_*H_p(x) \\
	&= \int_{g^{-1} \cdot U } \frac{d\mu}{d H_p}(x) \, d H_p(x) \\
	&= \mu(g^{-1} \cdot U) \\
	&= \mu (U)
	\end{align*}
	So that \begin{equation*}
	L_g \frac{d\mu}{d H_p} = \frac{d\mu}{d H_p}
	\end{equation*}
	The Converse is similar.
	\end{proof}
	
	\begin{ex}
	Let $(X, d)$ be a metric space, $G$ a group, and $\phi: G \times X \rightarrow X$ an isometric group action. Suppose that $\bar{d}$ is a metric. Let $\mu: \MB(X) \rightarrow [0, \infty]$ be a measure on $X$. Suppose that $\mu$ is $G$-invariant, $\mu \ll H_p^X$ and $d\mu / dH_p^X$ is continuous. Then $\bar{\mu} \ll \bar{H}_p^X$, $d\bar{\mu}/d \bar{H}_p^X$ is $G$-invariant, $d\bar{\mu}/d \bar{H}_p^X$ is continuous and 
	\begin{equation*}
	\frac{d \bar{\mu}}{d \bar{H}_p^X} = \overline{\frac{d \mu}{d H_p^X}}
	\end{equation*}
	\end{ex}
	
	\begin{proof}
	A previous exercise implies that $\bar{\mu} \ll \bar{H}_p^X$. Set $f = d \mu /d H_p^X$. Since $\mu$ is $G$-invariant, $f$ is $G$-invariant. Since $f$ is continuous, an exercise in section $9.2$ of \cite{analysis} implies that $\bar{f}$ is continuous and $f = \bar{f} \circ \pi$. Let $E \in \MB(X/G)$. Then 
	\begin{align*}
	\int_E \bar{f} d \bar{H}_p^X 
	&= \int_{\pi^{-1}(E)} \bar{f} \circ \pi dH_p^X \\
	&= \int_{\pi^{-1}(E)} f dH_p^X \\
	&= \mu(\pi^{-1}(E)) \\
	&= \bar{\mu}(E) \\
\end{align*}	 
	Therefore, by definition, we have that
	\begin{equation*}
	\frac{d \bar{\mu}}{d \bar{H}_p^X} = \bar{f} = \overline{\frac{d \mu}{d H_p^X}}
	\end{equation*}
	\end{proof}
	
	
	
	
	
	
	
	
	
	
	
	
	
	
	
	
	
	
	
	
	
	
	
	
	
	
	
	
	
	
	
	
	
	\newpage
	\section{Applications}
	
	\subsection{Applications to Bayesian Statistics}
	
	\begin{ex}
	Let $(\MX, \MA)$ be a measurable space $(\Theta, d)$ a metric space, $G$ a group, $\phi: G \times \Theta \rightarrow \Theta$ an isometric group action. Suppose that $\bar{d}$ is a metric on $\Theta / G$. Let 
	\begin{itemize}
	\item $H_p^{\Theta}$ be the Hausdorff measure on $\Theta$, $\mu_{\MX}$ a measure on $\MX$, 
	\item $p$ a denisty on $\Theta$ and for each $\theta \in \Theta$, $p(\cdot|\theta)$ a density on $\MX$. 
	\item $\theta_0 \in \Theta$ and for $j \in \N$, $X_j \sim p(x|\theta_0)$
	\end{itemize}
	Suppose that $p$ is $G$-invariant and continuous on $\Theta$ and for each $x \in \MX$, $p(x| \cdot)$ is $G$-invariant and continuous on $\Theta$. For $n \in \N$, set $p(\cdot|X^{(n)}) \propto f(X_1, \ldots, X_n| \cdot) p(\cdot)$. Define the posterior measure $P_{\Theta|X^{(n)}}: \MB(\Theta) \rightarrow [0, 1]$ by 
	\begin{equation*}
	d P_{\Theta|X^{(n)}} (\theta) = p(\theta |X^{(n)}) \, dH_p^{\Theta} (\theta)
	\end{equation*}
	Then there exists a continuous density $\bar{p}(\cdot|X^{(n)})$ on $\Theta / G$ such that 
	\begin{equation*}
	d \bar{P}_{\Theta|X^{(n)}}(\theta) = \bar{p}(\theta |X^{(n)}) \, d\bar{H}_p^{\Theta} (\theta)
	\end{equation*}
	\end{ex}
	
	\begin{proof}
	Clear from previous work.
	\end{proof}
	
	\begin{ex}
	Let $(\MX, \MA)$ be a measurable space $(\Theta, d)$ a metric space, $G$ a group, $\phi: G \times \Theta \rightarrow \Theta$ an isometric group action. Suppose that $\bar{d}$ is a metric on $\Theta / G$. Let 
	\begin{itemize}
	\item $H_p^{\Theta}$ be the Hausdorff measure on $\Theta$, $\mu_{\MX}$ a measure on $\MX$, 
	\item $p$ a denisty on $\Theta$ and for each $\theta \in \Theta$, $p(\cdot|\theta)$ a density on $\MX$. 
	\item $\theta_0 \in \Theta$ and for $j \in \N$, $X_j \sim p(x|\theta_0)$
	\end{itemize}
	Suppose that $p$ is $G$-invariant and continuous on $\Theta$ and for each $x \in \MX$, $p(x| \cdot)$ is $G$-invariant and continuous on $\Theta$. For $n \in \N$, set $p(\cdot|X^{(n)}) \propto f(X_1, \ldots, X_n| \cdot) p(\cdot)$. Define the posterior measure $P_{\Theta|X^{(n)}}: \MB(\Theta) \rightarrow [0, 1]$ by 
	\begin{equation*}
	d P_{\Theta|X^{(n)}} (\theta) = p(\theta |X^{(n)}) \, dH_p^{\Theta} (\theta)
	\end{equation*}
	Suppose that $(P_{\Theta|X^{(n)}})_{n \in \N}$ concentrates on $\bar{\theta}_0 \subset \Theta$ a.s. or in probability. Then $(\bar{P}_{\Theta|X^{(n)}})_{n \in \N}$ concentrates a.s. or in probability on $\{\bar{\theta_0}\} \subset \Theta / G$ (i.e. is consistent a.s. or in probability)
	\end{ex}
	
	\begin{proof}
	Let $V \in \MN_{\bar{\theta}_0}$. Then $\pi^{-1}(V) \in \MN_{\bar{\theta}_0}$. By definition, 
	\begin{align*}
	\bar{P}_{\Theta|X^{(n)}}(V^c)
	&= P_{\Theta|X^{(n)}}(\pi^{-1}(V^c)) \\
	&= P_{\Theta|X^{(n)}}(\pi^{-1}(V)^c) \\
	&\convt{a.s./$P$} 0 
	\end{align*}
	\end{proof}
	
	\begin{note}
	Some examples of $G$-invariant priors would be the uniform distribution, or $N_n(0, \sig^2I)$ on $\R^n$ when acted on by $O(n)$. An example of a $G$-invariant likelihood would be $f(A|Z) \sim \text{Ber}(ZZ^T)$ as in a latent position random graph model where $Z \in \R^{n \times d}$ is the parameter is invariant under right multiplication by $U \in O(d)$.
	\end{note}
	
	
	\begin{note}
	Next steps are to come up with a model that is computationally expensive, but on the oprbit space, computationally viable, get an estimate for the orbit of the parameter, map back.
	\end{note}
	
	
	
	
	
	
	
	
	
	
	
	
	
	
	
	\newpage
	\subsection{Applications to Network Statistics}
	
	\begin{note}
	We will define $L(d,n) = \{L \in \R^{d \times n}: L \text{ is lower-triangular}\}$.
	\end{note}
	
	\begin{note}
	Consider the isometric group action $O(d) \times \R^{d \times n} \rightarrow \R^{d \times n}$ given by $(Q, \theta) \mapsto Q \theta$. Using the QL decomposition, for each $\theta \in \R^{d \times n}$, there exists $Q \in O(d)$ and  $L \in L(d,n)$ such that $\theta = QL$. So that for each $\theta \in \R^{d \times n}$, there exists $L \in L(d,n)$ such that $\bar{\theta} = \bar{L}$ in $\R^{d \times n} / O(d)$. We may embed $\R^{{d+1 \choose 2}}$ into $L(d,n)$ via the map $\R^{{d+1 \choose 2}} \rightarrow L(d,n)$ given by $\eta \mapsto \theta_{\eta}$ where $\theta_{\eta}$ is lower triangular whose entries are the entries of $\eta$. 
	\end{note} 
	
	Define $\sig: \R \rightarrow (0,1)$ and $g:\R^{d \times n} \rightarrow \R^{n \times n}$ by $$\sig(t) = \frac{1}{1+ e^{-t}}$$ and $$g(\theta)_{i,j} = \sig((\theta^T \theta)_{i,j})$$.\\
	
	Consider the model 
	\begin{itemize}
	\item $A_{ij}^{(t)} \overset{iid}{\sim} \text{Bern}(g(\theta_0)_{i,j})$ for $i,j \in \{1, \ldots, n\}$ and $t \in \{1, \ldots, T\}$
	\item $\theta_0 \in \R^{d \times n}$
	\end{itemize}
	
	Write $\theta = QL$, let $\eta \in \R^{{d+1 \choose 2}}$ such that $\theta_{\eta} = L$ and write $\theta_{\eta} = (U_{\eta}, 0)$ where $U_{\eta} \in \R^{d \times d}$ is lower-triangular. Then $g(\theta) = g(\theta_{\eta})$ and 
\[ \theta_{\eta}^T\theta_{\eta} = 
\begin{pmatrix}
U_{\eta}^TU_{\eta} & 0 \\
0 & 0
\end{pmatrix}
\]

We can therefore consider the function $g': \R^{d \times d} \rightarrow \R^{d \times d}$ given by $$g'(U)_{i,j} = \sig((U^TU)_{i,j})$$

The likelihood $L(\theta)$ is given by 
\begin{align*}
L(\theta) 
&= L(\theta_{\eta}) \\
&= \prod_{t=1}^T \prod_{i < j} g(\theta_{\eta})_{i,j}^{A^{(t)}_{i,j}}(1 - g(\theta_{\eta})_{i,j})^{1 - A^{(t)}_{i,j}}\\
&= \prod_{t=1}^T \prod_{i < j \leq d} g(\theta_{\eta})_{i,j}^{A^{(t)}_{i,j}}(1 - g(\theta_{\eta})_{i,j})^{1 - A^{(t)}_{i,j}} (1/2)^{n^2 - d^2}\\
&= \prod_{t=1}^T \prod_{i < j \leq d} g'(U_{\eta})_{i,j}^{A^{(t)}_{i,j}}(1 - g'(U_{\eta})_{i,j})^{1 - A^{(t)}_{i,j}} (1/2)^{n^2 - d^2}\\
\end{align*}

Since $L(\theta) = L(\theta_{\eta})$, to optimize, we may drop the constant multiple (the only dependence on $n$) and maximize the log-likelihood $l'(\eta)$, which is given by 
\begin{align*}
l'(\eta) 
&= l(\theta_{\eta}) \\ 
&= \sum_{t=1}^T \sum_{i < j \leq d} A^{(t)}_{i,j} \log [g'(U_{\eta})_{i,j}] + (1 - A^{(t)}_{i,j}) \log[1 - g'(U_{\eta})_{i,j}] \\
\end{align*}
	
	
	
	
	
	
	
	
	
	
	
	
	
	
	
	
	
	
	\newpage
	\section{Appendix}
	
	\subsection{Quotient Topology}
	
	\begin{defn} \ld{34001}
	Let $(X, \MA)$, $(Y, \MB)$ be topological spaces and $f:X \rightarrow Y$. Suppose that $f$ is surjective. Then $f$ is said to be a \textbf{$\MA$-$\MB$ quotient map} if 
	\begin{enumerate}
	\item $f$ is surjective
	\item for each $V \subset Y$, $V \in \MB$ iff $f^{-1}(V) \in \MA$.
	\end{enumerate}
	\end{defn}
	
	\begin{note}
	We typically avoid specifying the topologies when they are clear from the context.
	\end{note}
	
	\begin{ex} \lex{34002}
	Let $(X, \MA)$, $(Y, \MB)$ be topological spaces and $f:X \rightarrow Y$. If $f$ is a quotient map, then $f$ is continuous.
	\end{ex}
	
	\begin{proof}
	Suppose that $f$ is a quotient map. Let $V \subset Y$. Suppose that $V$ is open. By definition, $f^{-1}(V)$ is open. Hence $f$ is continuous.  
	\end{proof}
	
	\begin{ex} \lex{34003}
	Let $(X, \MA)$, $(Y, \MB)$ be topological spaces and $f:X \rightarrow Y$. Suppose that $f$ is continuous and surjective. Then $f$ is a quotient map iff 
	\begin{equation*}
	\text{for each $C \subset Y$, $C$ is closed iff $f^{-1}(C)$ is closed} 
	\end{equation*}	
	\end{ex}
	
	\begin{proof}\
	\begin{itemize}
	\item ($\implies$) \\
	Suppose that $f$ is a quotient map.\\
	Let $C \subset Y$. If $C$ is closed, then continuity implies that $f^{-1}(C)$ is closed.\\ 
	Conversely, suppose that $f^{-1}(C)$ is closed. Then $f^{-1}(C^c) = (f^{-1}(C))^c$ is open. Since $f$ is a quotient map, $f(f^{-1}(C^c))$ is open. Surjectivity implies that $f(f^{-1}(C^c)) = C^c$. So $C$ is closed. 
	\item ($\impliedby$) \\
	Suppose that for each $C \subset Y$, $C$ is closed iff $f^{-1}(C)$ is closed. \\
	Let $V \subset Y$. If $V$ is open. Continuity implies that $f^{-1}(V)$ is open.\\ 
	Conversely, suppose that $f^{-1}(V)$ is open. Then $ f^{-1}(V^c) = (f^{-1}(V))^c$ is closed. Therefore, $f(f^{-1}(V^c))$ is closed. Surjectivity implies that $V^c = f(f^{-1}(V^c))$. So $U$ is open.
	\end{itemize}
	\end{proof}
	
	\begin{ex} \lex{34004}
	Let $(X, \MA)$, $(Y, \MB)$ be topological spaces and $f:X \rightarrow Y$. Suppose that $f$ is continuous and surjective. If $f$ is open or closed, then $f$ is a quotient map. 
	\end{ex}
	
	\begin{proof}\
	\begin{itemize}	
	\item Suppose that $f$ is open. Let $V \subset Y$. \\
	Suppose that $V$ is open. Then continuity implies that $f^{-1}(V)$ is open. Conversely, suppose that $f^{-1}(V)$ is open. Since $f$ is open $f(f^{-1}(V))$ is open. Surjectivity implies that $V = f(f^{-1}(V))$. So $V$ is open. By definition, $f$ is a quotient map.\\
	\item   
	Suppose that $f$ is open. Then similarly to above, $f$ is a quotient map.
	\end{itemize}
	\end{proof}
	
	\begin{ex} \lex{34005}
	Let $(X, \MA)$, $(Y, \MB)$ be topological spaces and $f:X \rightarrow Y$. Suppose that $f$ is a quotient map. Then $f$ is open iff 
	\begin{equation*}
	\text{for each $U \subset X$, $U$ is open implies that $f^{-1}(f(U))$ is open} 
	\end{equation*}
	\end{ex}
	
	\begin{proof}\
	\begin{itemize}	
	\item ($\implies$) \\
	Suppose that $f$ is open.\\
	Let $U \subset X$. Suppose that $U$ is open. Since $f$ is open, $f(U)$ is open. Continuity implies that $f^{-1}(f(U))$ is open.\\ 
	\item ($\impliedby$) \\
	Suppose that for each $U \subset X$, $U$ is open implies that $f^{-1}(f(U))$ is open. \\
	Since $f$ is a quotient map, $f(U)$ is open. So $f$ is open.
	
	\end{itemize}
	\end{proof}
	
	
	\begin{defn} \ld{34006}
	Let $(X, \MT)$ be a topological space, $Y$ a set and $f:X \rightarrow Y$. Suppose that $f$ is surjective.	
	 We call $f_* \MT$ the \textbf{quotient topology} on $Y$.  
	\end{defn}
	
	\begin{ex} \lex{34007}
	Let $(X, \MT)$ be a topological space, $Y$ a set and $f:X \rightarrow Y$. Suppose that $f$ is surjective. Then $f: X \rightarrow Y$ is a $\MT$-$f_*\MT$ quotient map. 
	\end{ex}
	
	\begin{proof}
	Clear.
	\end{proof}
	
	\begin{ex} \lex{34008}
	Let $(X, \MA)$, $(Y, \MB)$ be topological spaces, and $f:X \rightarrow Y$. Suppose that $f$ is surjective and continuous. If $f$ is open or closed, then $f_*\MA = \MB$.
	\end{ex}
	
	\begin{proof}
	Continuity, $\MB \subset f_*\MA$. 
	\begin{itemize}
	\item Suppose that $f$ is open. Let $V \in f_*\MA$. By definiiton, $f^{-1}(V) \in \MA$. Since $f$ is open, $f(f^{-1}(V)) \in \MB$. Surjectivity implies that $V = f(f^{-1}(V))$. 
	\item The case is similar if $f$ is closed.
	\end{itemize}
	\end{proof}
	
	\newpage
	\subsection{Hausdorff Measure}
	
	\begin{defn}
	Let $X$ be a metric space and $\mu^*: \MP(X) \rightarrow [0, \infty]$ an outer measure on $X$. Then $\mu^*$ is said to be a \textbf{metric outer measure on $X$} if for each $A, B \subset X$, $d(A,B) > 0$ implies that 
	\begin{equation*}
	\mu^*(A \cup B) = \mu^*(A) + \mu^*(B)
	\end{equation*}
	\end{defn}	
	
	\begin{ex}
	Let $X$ be a metric space and $\mu^*: \MP(X) \rightarrow [0, \infty]$ a metric outer measure on $X$.
	Then for each $A \in \MB(X)$, $A$ is $\mu^*$-outer measurable. 
	\end{ex}
	
	\begin{proof}
	
	\end{proof}
	
	
	\begin{defn}
	Let $X$ be a metric space, $E \subset X$ and $\del >0$. Define $\MA_{E, \del} \subset \MP(X)^{\N}$ by 
	\begin{equation*}
	 \MA_{E, \del} = \inf \bigg \{(A_j)_{j \in \N} \subset \MP(X): E \subset \bigcup\limits_{j \in \N}A_j \text{ and for each $j \in \N$, } \diam(A_j) < \del \bigg \}
	\end{equation*}
	\end{defn}
	
	\begin{ex}
	Let $X$ be a metric space, $E \subset X$ and $\del_1, \del_2 >0$. If $\del_1 \leq \del_2$, then $\MA_{E, \del_1} \subset \MA_{E, \del_2}$.
	\end{ex}
	
	\begin{proof}
	Clear.
	\end{proof}
	
	\begin{defn}
	Let $X$ be a metric space, $d \geq 0$ and $\del >0$. Define $H_{d, \del}: \MP(X) \rightarrow [0, \infty]$ by 
	\begin{equation*}
	H_{d, \del}(E) = \inf \bigg \{\sum_{j \in \N} \diam(A_j)^d: (A_j)_{j \in \N} \in \MA_{E, \del} \bigg \}
	\end{equation*}
	\end{defn}
	
	\begin{ex}
	Let $X$ be a metric space, $d \geq 0$ and $\del_1, \del_2 >0$. If $\del_1 \leq \del_2$, then $H_{d, \del_2} \leq H_{d, \del_1}$.
	\end{ex}
	
	\begin{proof}
	Clear.
	\end{proof}
	
	\begin{defn}
	Let $X$ be a metric space and $d \geq 0$. We define the \textbf{$d$-dimensional Hausdorff outer measure}, denoted $H_{d}: \MP(X) \rightarrow [0, \infty]$, by 
	\begin{align*}
	H_{d}(E) 
	&= \sup_{\del > 0} H_{d, \del}(E) \\
	&= \lim_{\del \rightarrow 0^+} H_{d, \del}(E)
	\end{align*}
	\end{defn}
	
	\begin{ex}
	Let $X$ be a metric space and $d \geq 0$. Then $H_d: \MP(X) \rightarrow [0, \infty]$ is an outer measure on $X$.
	\end{ex}
	
	\begin{proof}
	
	\end{proof}
	
	\begin{ex}
	Let $X$ be a metric space and $d \geq 0$. Then $H_d: \MP(X) \rightarrow [0, \infty]$ is a metric outer measure on $X$.
	\end{ex}
	
	\begin{proof}
	
	\end{proof}
	
	
	
	
	
	
	
	
	
	
	
	
	
	
	
	
	
	
	
	
	
	
	
	
	
	
	
	
	
	
	
	
	
	
	
	
	
	
	
	
	
	
	
	
	
	
	
	
	
	
	
	
	
	
	
	
	
	
	
	
	
	
	
	
	
	
	
	
	
	
	
	
	
	\newpage
	\begin{thebibliography}{4}
\bibitem{algebra} \href{https://github.com/carsonaj/Mathematics/blob/master/Introduction\%20to\%20Algebra/Introduction\%20to\%20Algebra.pdf}{Introduction to Algebra}

\bibitem{analysis}  \href{https://github.com/carsonaj/Mathematics/blob/master/Introduction\%20to\%20Analysis/Introduction\%20to\%20Analysis.pdf}{Introduction to Analysis}	

\bibitem{foranal}  \href{https://github.com/carsonaj/Mathematics/blob/master/Introduction\%20to\%20Fourier\%20Analysis/Introduction\%20to\%20Fourier\%20Analysis.pdf}{Introduction to Fourier Analysis}

\bibitem{measure}  \href{https://github.com/carsonaj/Mathematics/blob/master/Introduction\%20to\%20Measure\%20and\%20Integration/Introduction\%20to\%20Measure\%20and\%20Integration.pdf}{Introduction to Measure and Integration}



\end{thebibliography}

	
	
	
	
	
	
	
	
\end{document}