\documentclass[notheorems]{beamer}

\usepackage{amsmath,amsthm,amssymb,setspace, mathtools}
\usepackage{physics}

\usepackage{color}   %May be necessary if you want to color links
\usepackage{hyperref}
\hypersetup{
	colorlinks=true, %set true if you want colored links
	linktoc=all,     %set to all if you want both sections and subsections linked
	linkcolor=black,  %choose some color if you want links to stand out
	urlcolor=cyan
}


%
%
%
\newif\ifhideproofs
%\hideproofstrue %uncomment to hide proofs
%
%
%
%
\ifhideproofs
\usepackage{environ}
\NewEnviron{hide}{}
\let\proof\hide
\let\endproof\endhide
\fi

\newtheorem{thm}{Theorem}[subsection]
\newtheorem{lem}[thm]{Lemma}
\newtheorem{prop}[thm]{Proposition}
\newtheorem{cor}[thm]{Corollary}
\newtheorem{conj}{Conjecture}
\newtheorem{res}[thm]{Result}

\theoremstyle{definition}
\newtheorem{definition}{Definition}[subsection]
\newtheorem{defn}[definition]{Definition}


\theoremstyle{definition}
\newtheorem{ex}[definition]{Exercise}
\newtheorem{rem}[definition]{Remark}

\newcommand{\al}{\alpha}
\newcommand{\Gam}{\Gamma}
\newcommand{\be}{\beta} 
\newcommand{\del}{\delta} 
\newcommand{\Del}{\Delta}
\newcommand{\lam}{\lambda}  
\newcommand{\Lam}{\Lambda} 
\newcommand{\ep}{\epsilon}
\newcommand{\sig}{\sigma} 
\newcommand{\om}{\omega}
\newcommand{\Om}{\Omega}
\newcommand{\C}{\mathbb{C}}
\newcommand{\N}{\mathbb{N}}
\newcommand{\E}{\mathbb{E}}
\newcommand{\Z}{\mathbb{Z}}
\newcommand{\R}{\mathbb{R}}
\newcommand{\T}{\mathbb{T}}
\newcommand{\Q}{\mathbb{Q}}
\renewcommand{\P}{\mathbb{P}}
\newcommand{\MA}{\mathcal{A}}
\newcommand{\MC}{\mathcal{C}}
\newcommand{\MB}{\mathcal{B}}
\newcommand{\MF}{\mathcal{F}}
\newcommand{\MG}{\mathcal{G}}
\newcommand{\ML}{\mathcal{L}}
\newcommand{\MN}{\mathcal{N}}
\newcommand{\MS}{\mathcal{S}}
\newcommand{\MP}{\mathcal{P}}
\newcommand{\ME}{\mathcal{E}}
\newcommand{\MT}{\mathcal{T}}
\newcommand{\MM}{\mathcal{M}}
\newcommand{\MI}{\mathcal{I}}

\newcommand{\ui}{[0,1]}
\newcommand{\p}{\partial}

\newcommand{\io}{\text{ i.o.}}
%\newcommand{\ev}{\text{ ev.}}
\renewcommand{\r}{\rangle}
\renewcommand{\l}{\langle}

\newcommand{\RG}{[0,\infty]}
\newcommand{\Rg}{[0,\infty)}
\newcommand{\Ru}{(\infty, \infty]}
\newcommand{\Rd}{[\infty, \infty)}
\newcommand{\Ll}{L^1_{\text{loc}}(\R^n)}

\newcommand{\limfn}{\liminf \limits_{n \rightarrow \infty}}
\newcommand{\limpn}{\limsup \limits_{n \rightarrow \infty}}
\newcommand{\limn}{\lim \limits_{n \rightarrow \infty}}
\newcommand{\convt}[1]{\xrightarrow{\text{#1}}}
\newcommand{\conv}[1]{\xrightarrow{#1}} 
\newcommand{\seq}[2]{(#1_{#2})_{#2 \in \N}}

\newcommand{\lsc}{l.s.c. }

\DeclareMathOperator{\supp}{supp}
\DeclareMathOperator{\sgn}{sgn}
\DeclareMathOperator{\spn}{span}
\DeclareMathOperator{\iso}{Iso}
\DeclareMathOperator{\id}{id}


%Information to be included in the title page:
\title{Gradient Descent in Hilbert Space}
\author{Carson James}

\begin{document}

\frame{\titlepage}

\begin{frame}
\frametitle{Outline}
	\begin{enumerate}
	\item Banach Spaces
		\begin{itemize}
		\item bounded linear maps
		\item Frechet differentiation
		\end{itemize}
	\item Calculus 
		\begin{itemize}
		\item tools
		\item results
		\end{itemize}
	\item Hilbert Spaces
		\begin{itemize}
		\item Riesz representation theorem
		\item gradients
		\end{itemize}
	\item Convex Analysis 
		\begin{itemize}
		\item results for Frechet differentiable functions
		\end{itemize}
	\item Reproducing Kernel Hilbert Spaces
		\begin{itemize}
		\item representer theorem
		\end{itemize}
	\item Applications to Gaussian Processes
	\begin{itemize}
		\item predictive posterior 
		\end{itemize}
	
	\end{enumerate}
\end{frame}

\begin{frame}
\frametitle{Banach Spaces}
	
	\begin{defn}
		Let $X,Y$ be a normed vector spaces and $T:X \rightarrow Y$ a linear map. Then $T$ is said to be \textbf{bounded} if there exists $C \geq 0$ such that for each $x \in X$, $$\|Tx \|\leq C \|x \|$$ We define $$L(X,Y) = \{T:X \rightarrow Y: T \text{ is linear and bounded}\}$$
	\end{defn}
	\end{frame}
	
	
	
	
	
	
	
	
	
	
	
	
	
	
	
	
	
	\begin{frame}
	
	\begin{defn}
		Let $X_1, \cdots, X_n$ and $Y$ be a normed vector spaces and $T:\prod\limits_{j=1}^n X_j \rightarrow Y$ a multilinear linear map. Then $T$ is said to be \textbf{bounded} if there exists $C \geq 0$ such that for each $(x_j)_{j=1}^n \in \prod\limits_{j=1}^n X_j$, $$\|T(x_1, \cdots, x_n) \|\leq C \|x_1 \| \cdots \|x_n\|$$ 
		We define $$L^n\bigg(\prod_{j=1}^n X_j,Y\bigg) = \{T:X \rightarrow Y: T \text{ is multilinear and bounded}\}$$ 
		If $X_1, \cdots, X_n = X$, we write $L^n(X,Y)$ in place of  $L^n(\prod_{j=1}^n X_j,Y)$.
	\end{defn}
	
	\end{frame}
	
	
	
	
	
	

	
	
	
	
	
	
	
	
	
	
	
	
	
	\begin{frame}
	
	\begin{rem}
	
	Let $X$ and $Y$ be normed vector spaces. We may identify $L(X, L(X, \cdots, L(X, Y)) \cdots)$ and $L^n(X, Y)$ via the isometric isomorphism given by $\phi \mapsto \psi_{\phi}$ where $$\psi_{\phi}(x_1, x_2, \cdots, x_n) = \phi(x_1)(x_2),\cdots,(x_n)$$ 
	\end{rem}

	\begin{defn}
	Let $X$ be a normed vector space over $\R$. We define the \textbf{dual space of} $X$, denoted $X^*$, by $X^* = L(X, \R)$. Let $T: X \rightarrow \R$. Then $T$ is said to be a \textbf{bounded linear functional on} $X$ if $T \in X^*$.
	\end{defn}
	
\end{frame}

























\begin{frame}

	\begin{defn}
		Let $X$ be a normed vector space. Then $X$ is said to be a \textbf{Banach space} if $X$ is complete.  
	\end{defn}

	\begin{defn}
	Let $X, Y$ be a banach spaces, $A \subset X$ open, $f:A \rightarrow Y$ and $x_0 \in A$. Then $f$ is said to be \textbf{Frechet differentiable at $x_0$} if there exists $Df(x_0) \in L(X,Y)$ such that, $$f(x_0 + h) = f(x_0) + Df(x_0)(h) + o(\|h\| ) \hspace{.5cm} \text{ as } h \rightarrow 0$$  
	If $f$ is Frechet differentiable at $x_0$, we define the \textbf{Frechet derivative of $f$ at $x_0$} to be $Df(x_0)$.
	We say that $f$ is \textbf{Frechet differentiable} (or \textbf{$1$-st order Frechet differentiable}) if for each $x_0 \in A$, $f$ is Frechet differentiable at $x_0$. \\
	If $f$ is Frechet differentiable, we define the \textbf{Frechet derivative} of $f$, denoted $Df: A \rightarrow L(X,Y)$, by $$x \mapsto Df(x)$$
	\end{defn}

\end{frame}











\begin{frame}
\begin{defn}
Let $X, Y$ be a banach spaces, $A \subset X$ open, $f:A \rightarrow Y$. We define \textbf{$n$-th} order Frechet differentiablility inductively. \\
Since $f$  
\end{defn}
\end{frame}













\begin{frame}
\frametitle{Calculus}

\begin{rem}
The various tools used to obtain the main calculus results are the following:
	\begin{itemize}
	\item Frechet derivative 
	\item Hahn-Banach theorem (not introduced)
	\item Bochner Integral (not introduced)
	\end{itemize}
\end{rem}
\end{frame}


















\begin{frame}
\frametitle{Convex Analysis}

\begin{res}
The various tools used to obtain the main convex analysis results are the following:
	\begin{itemize}
	\item 
	\item Frechet derivative 
	\item Hahn-Banach theorem (not introduced)
	\item Bochner Integral (not introduced)
	\end{itemize}
\end{res}
\end{frame}



\end{document}