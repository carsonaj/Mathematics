%% filename: amsbook-template.tex
%% version: 1.1
%% date: 2014/07/24
%%
%% American Mathematical Society
%% Technical Support
%% Publications Technical Group
%% 201 Charles Street
%% Providence, RI 02904
%% USA
%% tel: (401) 455-4080
%%      (800) 321-4267 (USA and Canada only)
%% fax: (401) 331-3842
%% email: tech-support@ams.org
%% 
%% Copyright 2006, 2008-2010, 2014 American Mathematical Society.
%% 
%% This work may be distributed and/or modified under the
%% conditions of the LaTeX Project Public License, either version 1.3c
%% of this license or (at your option) any later version.
%% The latest version of this license is in
%%   http://www.latex-project.org/lppl.txt
%% and version 1.3c or later is part of all distributions of LaTeX
%% version 2005/12/01 or later.
%% 
%% This work has the LPPL maintenance status `maintained'.
%% 
%% The Current Maintainer of this work is the American Mathematical
%% Society.
%%
%% ====================================================================

%    AMS-LaTeX v.2 driver file template for use with amsbook
%
%    Remove any commented or uncommented macros you do not use.

\documentclass{book}

%    For use when working on individual chapters
%\includeonly{}

%    For use when working on individual chapters
%\includeonly{}

%    Include referenced packages here.
\usepackage[margin=1in]{geometry} 
\usepackage{amsmath}
\usepackage{amsthm}
\usepackage{amssymb}
\usepackage{setspace}
\usepackage{mathtools}
\usepackage{tikz}  
\usepackage{tikz-cd}
\usepackage{tkz-fct}
\usepackage{pgfplots}
\usepackage{environ}
\usepackage{tikz-cd} 
\usepackage{enumitem}
\usepackage{color}   %May be necessary if you want to color links
\usepackage{hyperref}
\hypersetup{
	colorlinks=true, %set true if you want colored links
	linktoc=all,     %set to all if you want both sections and subsections linked
	linkcolor=black,  %choose some color if you want links to stand out
	urlcolor=cyan
}
\usepackage[symbols,nogroupskip,sort=none]{glossaries-extra}

\pgfplotsset{every axis/.append style={
		axis x line=middle,    % put the x axis in the middle
		axis y line=middle,    % put the y axis in the middle
		axis line style={<->,color=black}, % arrows on the axis
		xlabel={$x$},          % default put x on x-axis
		ylabel={$y$},          % default put y on y-axis
}}


\theoremstyle{definition}
\newtheorem{definition}{Definition}[subsection]
\newtheorem{defn}[definition]{Definition}
\newtheorem{note}[definition]{Note}
\newtheorem{ax}[definition]{Axiom}
\newtheorem{thm}[definition]{Theorem}
\newtheorem{lem}[definition]{Lemma}
\newtheorem{prop}[definition]{Proposition}
\newtheorem{cor}[definition]{Corollary}
\newtheorem{conj}[definition]{Conjecture}
\newtheorem{ex}[definition]{Exercise}
\newtheorem{exmp}[definition]{Example}

\setcounter{tocdepth}{3}

% hide proofs
\newif\ifhideproofs
%\hideproofstrue %uncomment to hide proofs
\ifhideproofs
\NewEnviron{hide}{}
\let\proof\hide
\let\endproof\endhide
\fi

% lower-case greek
\newcommand{\al}{\alpha}
\newcommand{\be}{\beta}
\newcommand{\gam}{\gamma}
\newcommand{\del}{\delta}
\newcommand{\ep}{\epsilon}
\newcommand{\ze}{\zeta} 
\newcommand{\kap}{\kappa} 
\newcommand{\lam}{\lambda}  
\newcommand{\sig}{\sigma} 
\newcommand{\omi}{\omicron}
\newcommand{\up}{\upsilon}
\newcommand{\om}{\omega}

% upper-case greek
\newcommand{\Gam}{\Gamma}
\newcommand{\Del}{\Delta}
\newcommand{\Lam}{\Lambda} 
\newcommand{\Sig}{\Sigma} 
\newcommand{\Om}{\Omega}

% blackboard bold
\newcommand{\C}{\mathbb{C}}
\newcommand{\E}{\mathbb{E}}
\newcommand{\F}{\mathbb{F}}
\renewcommand{\H}{\mathbb{H}}
\newcommand{\K}{\mathbb{K}}
\newcommand{\N}{\mathbb{N}}
\renewcommand{\O}{\mathbb{O}}
\newcommand{\Q}{\mathbb{Q}}
\newcommand{\R}{\mathbb{R}}
\newcommand{\T}{\mathbb{T}}
\newcommand{\V}{\mathbb{V}}
\newcommand{\Z}{\mathbb{Z}}

% math caligraphic
\newcommand{\MA}{\mathcal{A}}
\newcommand{\MB}{\mathcal{B}}
\newcommand{\MC}{\mathcal{C}}
\newcommand{\MD}{\mathcal{D}}
\newcommand{\ME}{\mathcal{E}}
\newcommand{\MF}{\mathcal{F}}
\newcommand{\MG}{\mathcal{G}}
\newcommand{\MH}{\mathcal{H}}
\newcommand{\MI}{\mathcal{I}}
\newcommand{\MJ}{\mathcal{J}}
\newcommand{\MK}{\mathcal{K}}
\newcommand{\ML}{\mathcal{L}}
\newcommand{\MM}{\mathcal{M}}
\newcommand{\MN}{\mathcal{N}}
\newcommand{\MO}{\mathcal{O}}
\newcommand{\MP}{\mathcal{P}}
\newcommand{\MQ}{\mathcal{Q}}
\newcommand{\MR}{\mathcal{R}}
\newcommand{\MS}{\mathcal{S}}
\newcommand{\MT}{\mathcal{T}}
\newcommand{\MU}{\mathcal{U}}
\newcommand{\MV}{\mathcal{V}}
\newcommand{\MW}{\mathcal{W}}
\newcommand{\MX}{\mathcal{X}}
\newcommand{\MY}{\mathcal{Y}}
\newcommand{\MZ}{\mathcal{Z}}

% mathfrak
\newcommand{\MFX}{\mathfrak{X}}
\newcommand{\MFg}{\mathfrak{g}}
\newcommand{\MFh}{\mathfrak{h}}

% tilde 
\newcommand{\tMA}{\tilde{\MA}}
\newcommand{\tMB}{\tilde{\MB}}
\newcommand{\tU}{\tilde{U}}
\newcommand{\tV}{\tilde{V}}
\newcommand{\tphi}{\tilde{\phi}}
\newcommand{\tpsi}{\tilde{\psi}}
\newcommand{\tF}{\tilde{F}}

% label/reference
\newcommand{\lex}[1]{\label{ex:#1}}
\newcommand{\rex}[1]{Exercise \ref{ex:#1}}

\newcommand{\ld}[1]{\label{defn:#1}}
\newcommand{\rd}[1]{Definition \ref{defn:#1}}

\newcommand{\lax}[1]{\label{ax:#1}}
\newcommand{\rax}[1]{Axiom \ref{ax:#1}}

\newcommand{\lfig}[1]{\label{fig:#1}}
\newcommand{\rfig}[1]{Figure \ref{fig:#1}}

% math operators
\DeclareMathOperator{\supp}{supp}
\DeclareMathOperator{\sgn}{sgn}
\DeclareMathOperator{\spn}{span}
\DeclareMathOperator{\iso}{Iso}
\DeclareMathOperator{\id}{id}
\DeclareMathOperator{\Aut}{Aut}
\DeclareMathOperator{\Homeo}{Homeo}
\DeclareMathOperator{\Sym}{Sym}
\DeclareMathOperator{\Alt}{Alt}
\DeclareMathOperator{\cl}{cl}
\DeclareMathOperator{\Int}{Int}
\DeclareMathOperator{\bal}{bal}
\DeclareMathOperator{\cnv}{conv}
\DeclareMathOperator{\epi}{epi}
\DeclareMathOperator{\dom}{dom}
\DeclareMathOperator{\cod}{cod}
\DeclareMathOperator{\Obj}{Obj}
\DeclareMathOperator{\Hom}{Hom}
\DeclareMathOperator*{\argmax}{arg\,max}
\DeclareMathOperator*{\argmin}{arg\,min}
\DeclareMathOperator{\diam}{diam}
\DeclareMathOperator{\rnk}{rank}
\DeclareMathOperator{\prj}{\text{proj}}
\DeclareMathOperator{\nab}{\nabla}
\DeclareMathOperator{\diag}{\text{diag}}
\DeclareMathOperator*{\ind}{\text{\text{ind}}}
\DeclareMathOperator*{\ar}{\text{\text{arity}}}
\DeclareMathOperator*{\cur}{\text{\text{cur}}}


% category theory
\DeclareMathOperator*{\Set}{\text{\tbf{Set}}}
\DeclareMathOperator*{\Meas}{\text{\tbf{Meas}}}
\DeclareMathOperator*{\Maninf}{\text{\tbf{Man}}^{\infty}} 
\DeclareMathOperator*{\Man0}{\text{\tbf{Man}}^{0}}
\DeclareMathOperator*{\Buninf}{\text{\tbf{Bun}}^{\infty}} 
\DeclareMathOperator*{\VecBuninf}{\text{\tbf{VecBun}}^{\infty}} 
\DeclareMathOperator*{\VectR}{\text{\tbf{Vect}}_{\R}} 
\DeclareMathOperator*{\Cat}{\text{\tbf{Cat}}}
\DeclareMathOperator*{\0}{\mbf{0}}
\DeclareMathOperator*{\1}{\mbf{1}}
\DeclareMathOperator*{\TopEq}{\text{\tbf{TopEq}}}
\DeclareMathOperator*{\Top}{\text{\tbf{Top}}}
\DeclareMathOperator*{\Cone}{\text{\tbf{Cone}}}
\DeclareMathOperator*{\Cocone}{\text{\tbf{Cocone}}}

% notation
\renewcommand{\r}{\rangle}
\renewcommand{\l}{\langle}
\renewcommand{\div}{\text{div}}
\renewcommand{\Im}{\text{Im} \,}
\newcommand{\grad}{\text{grad}}
\newcommand{\tbf}[1]{\textbf{#1}}
\newcommand{\tcb}[1]{\textcolor{blue}{#1}}
\newcommand{\mbf}[1]{\mathbf{#1}}
\newcommand{\ol}[1]{\overline{#1}}
\newcommand{\p}{\partial}
\newcommand{\Tn}[1]{T^{r_{#1}}_{s_{#1}}(V)}
\newcommand{\Tnp}{T^{r_1 + r_2}_{s_1 + s_2}(V)}
\newcommand{\Perm}{\text{Perm}}



% limits
\newcommand{\limfn}{\liminf \limits_{n \rightarrow \infty}}
\newcommand{\limpn}{\limsup \limits_{n \rightarrow \infty}}
\newcommand{\limn}{\lim \limits_{n \rightarrow \infty}}
\newcommand{\convt}[1]{\xrightarrow{\text{#1}}}
\newcommand{\conv}[1]{\xrightarrow{#1}} 
\newcommand{\seq}[2]{(#1_{#2})_{#2 \in \N}}

% intervals
\newcommand{\RG}{[0,\infty]}
\newcommand{\Rg}{[0,\infty)}
\newcommand{\Ru}{(\infty, \infty]}
\newcommand{\Rd}{[\infty, \infty)}
\newcommand{\ui}{[0,1]}

% integration \newcommand{\dm}{\, d m}
\newcommand{\dmu}{\, d \mu}
\newcommand{\dnu}{\, d \nu}
\newcommand{\dlam}{\, d \lambda}
\newcommand{\dP}{\, d P}
\newcommand{\dQ}{\, d Q}
\newcommand{\dm}{\, d m}
\newcommand{\dsh}{\, d \#}

% abreviations 
\newcommand{\lsc}{lower semicontinuous}

% misc
\newcommand{\as}[1]{\overset{#1}{\sim}}
\newcommand{\astx}[1]{\overset{\text{#1}}{\sim}}
\newcommand{\io}{\text{ i.o.}}
%\newcommand{\ev}{\text{ ev.}}
\newcommand{\Ll}{L^1_{\text{loc}}(\R^n)}

\newcommand{\loc}{\text{loc}}
\newcommand{\BV}{\text{BV}}
\newcommand{\NBV}{\text{NBV}}
\newcommand{\TV}{\text{TV}}

\newcommand{\op}[1]{\mathcal{#1}^{\text{op}}}

% Glossary - Notation
\glsxtrnewsymbol[description={finite measures on $(X, \MA)$}]{n000001}{$\MM_+(X, \MA)$}
\glsxtrnewsymbol[description={velocity}]{v}{\ensuremath{v}}


\makeindex

\begin{document}
	
	\frontmatter
	
	\title{Introduction to Group Theory}
	
	%    Remove any unused author tags.
	
	%    author one information
	\author{Carson James}
	\thanks{}
	
	\date{}
	
	\maketitle
	
	%    Dedication.  If the dedication is longer than a line or two,
	%    remove the centering instructions and the line break.
	%\cleardoublepage
	%\thispagestyle{empty}
	%\vspace*{13.5pc}
	%\begin{center}
	%  Dedication text (use \\[2pt] for line break if necessary)
	%\end{center}
	%\cleardoublepage
	
	%    Change page number to 6 if a dedication is present.
	\setcounter{page}{4}
	
	\tableofcontents
	\printunsrtglossary[type=symbols,style=long,title={Notation}]
	
	%    Include unnumbered chapters (preface, acknowledgments, etc.) here.
	%\include{}
	
	\mainmatter
	%    Include main chapters here.
	%\include{}
	
	\chapter*{Preface}
	\addcontentsline{toc}{chapter}{Preface}
	
	\begin{flushleft}
		\href{https://creativecommons.org/licenses/by-nc-sa/4.0/legalcode.txt}{cc-by-nc-sa}
	\end{flushleft}
	
	\newpage
	
	
	\chapter{Representation Theory}
	
	\section{Tannaka-Krein Duality}
	
	\begin{defn}
		Let $G \in \Obj(\TopMon)$, $V \in \Obj(\Top\Vect_{\C})$ and $\pi \in \Hom_{\TopMon}(G, \End_{\TopVect_{\C}}(V))$. Then $(V, \pi)$ is said to be a \tbf{$\C$-representation of $G$}.  We denote the set of $\C$-representations of $G$ by $\MR(G, \C)$.
	\end{defn}

	\begin{defn}
		 Let $(V, \pi) \in \MR(G< \C)$. We define the \tbf{dimension of $(V, \pi)$}, denoted $\dim (V, \pi)$, by $\dim (V, \pi) = \dim V$. Then $(V, \pi)$ is said to be \tbf{finite dimensional} if $\dim (V, \pi) < \infty$. 
	\end{defn}

	\begin{defn}
		Let $G \in \Obj(\TopMon)$, $(V, \pi), (W, \rho) \in \MR(G, \C)$ and $T\in \Hom_{\TopVect_{\C}}(V, W)$ Then $T$ is said to be \tbf{$(\pi, \rho)$-equivariant} if for each $g \in G$, $T \circ \pi(g) = \rho(g) \circ T$, i.e. the following diagram commutes:
		\[
		\begin{tikzcd}
			V \arrow[r, "T"] \arrow[d, "\pi(g)"']  & W  \arrow[d, "\rho(g)"] \\
			V \arrow[r, "T"']                   & W
		\end{tikzcd}
		\]
	\end{defn}

	\begin{defn} \ld{13009}
		Let $G \in \Obj(\TopMon)$. We define $\Rep(G, \C)$ by 
		\begin{itemize}
			\item $\Obj(\Rep(G, \C)) = \MR(G, \C)$.
			\item for $(V, \pi),(W, \rho) \in \Obj(\Rep(G, \C))$, 
			$$\Hom_{\Rep(G, \C)}((V, \pi), (W, \rho)) = \{T \in \Hom_{\TopVect_{\C}}(V, W): T \text{ is $(\pi, \rho)$-equivariant} \}$$
			\item for $(V, \pi), (W, \rho), (Z, \mu) \in \Obj(\Rep(G, \C))$, $T \in \Hom_{\Rep(G, \C)}((V, \pi), (W, \rho))$ and $S \in  \Hom_{\Rep(G, \C)}((W, \rho), (Z, \mu))$, $$S \circ_{\Rep(G, \C)} T = S \circ T$$
		\end{itemize}
	\end{defn}
	
	\begin{ex}
		Let $G \in \Obj(\TopMon)$. Then $\Rep(G, \C)$ is a category.
	\end{ex}

	\begin{proof}
		
	\end{proof}

	\begin{defn}
		Let $G \in \Obj(\TopMon)$. We define the \tbf{forgetful functor from $\Rep(G, \C)$ to $\TopVect_{\C}$}, denoted $U: \Rep(G, \C) \rightarrow \TopVect_{\C}$, by 
		\begin{itemize}
			\item $U (V, \pi) = V$, \quad $(V, \pi) \in \Obj(\Rep(G, \C))$
			\item $U (T) = T$, \quad $T \in \Hom_{\Rep(G, \C)}((V, \pi), (W, \rho))$.
		\end{itemize}
	\end{defn}


	\begin{defn}
		Let $G \in \Obj(\TopMon)$ and $g \in G$. We define $\hat{g}: U \Rightarrow U$ by $\hat{g}_{(V, \pi)} = \pi(g)$. 
	\end{defn}

	\begin{ex}
		Let $G \in \Obj(\TopMon)$ and $g \in G$. Then 
		\begin{enumerate}
			\item $\hat{g}: U \Rightarrow U$ is a natural transformation.
			\item $\hat{g} \in \End_{\TopVect_{\C}^{\Rep(G, \C)}}(U)$
		\end{enumerate}
	\end{ex}

	\begin{proof}\
		\begin{enumerate}
			\item \begin{enumerate}
				\item Let $(V, \pi) \in \Obj(\Rep(G, \C))$. By definition,  
				\begin{align*}
					\hat{g}_{(V, \pi)}
					& = \pi(g) \\
					& \in \End_{\TopVect_{\C}}(V) \\
					& = \Hom_{\TopVect_{\C}}(U(V, \pi), U(V, \pi))
				\end{align*}
				\item Let $(V, \pi), (W, \rho) \in \Obj(\Rep(G, \C))$ and $T \in \Hom_{\Rep(G, \C)}((V, \pi), (W, \rho))$. By definition, $T \in \Hom_{\TopVect_{\C}}(V, W)$ and $T$ is $(\pi, \rho)$-equivariant. Therefore
				\begin{align*}
					U(T) \circ \hat{g}_{(V, \pi)}
					& = T \circ \pi(g) \\
					& = \rho(g) \circ T \\
					& = \hat{g}_{(W, \rho)} \circ U(T) 
				\end{align*}
				i.e. the following diagram commutes: 
				\[ 
				\begin{tikzcd}
					U(V, \pi)  \arrow[r, "\hat{g}_{(V,\pi)}"]  \arrow[d, "U(T)"']  & U(V, \pi)   \arrow[d, "U(T)"]\\
					U(W, \rho) \arrow[r, "\hat{g}_{(W, \rho)}"] &  U(W, \rho) \\
				\end{tikzcd}
				\]
			\end{enumerate}
			Thus $\hat{g}: U \Rightarrow U$ is a natural transformation. 
			\item The previous part implies that  
			\begin{align*}
				\hat{g} 
				& \in \Hom_{\TopVect_{\C}^{\Rep(G, \C)}}(U, U) \\
				& = \End_{\TopVect_{\C}^{\Rep(G, \C)}}(U)
			\end{align*} 
		\end{enumerate}
	\end{proof}
	
	\begin{defn}
		Let $G \in \Obj(\TopMon)$ and $(V, \pi) \in \Obj(\Rep(G, \C))$. We define the \tbf{$(V, \pi)$-projection}, denoted $\pi_{(V, \pi)}: \End_{\TopVect_{\C}^{\Rep(G, \C)}}(U) \rightarrow \End_{\TopVect_{\C}}(V)$, by $\pi_{(V, \pi)}(\al) = \al_{(V, \pi)}$. We define the \tbf{topology of endomorphisms of $U$}, denoted $\MT_{\ME(U)}$, by 
		$$\MT_{\ME(U)} = \tau(\pi_{(V, \pi)}: (V, \pi) \in \Rep(G, \C))$$
	\end{defn}
	
	\begin{defn}
		\tcb{define addition of endomorphisms of $U$ pointwise}
	\end{defn}
	
	\begin{ex}
		Let $G \in \Obj(\TopMon)$. Then $(\End_{\TopVect_{\C}^{\Rep(G, \C)}}(U), \MT_{\ME(U)})$ is a topological unital algebra.
	\end{ex}

	\begin{proof}
		
	\end{proof}
	
	
	
	
	
	
	
	
	
	
	
	
	
	
	
	
	
	
	
	
	
	
	
	
	
	
	
	
	
	
	
	
	
	
	
	
	
	
	\newpage
	\chapter{Groupoids}
	
	\begin{defn}
		
	\end{defn}
	
	
	
	
	
	
	
	
	
	
	
	
	
	\backmatter
	\begin{thebibliography}{4}
		\bibitem{algebra} \href{https://github.com/carsonaj/Mathematics/blob/master/Introduction\%20to\%20Algebra/Introduction\%20to\%20Algebra.pdf}{Introduction to Algebra}
		
		\bibitem{analysis}  \href{https://github.com/carsonaj/Mathematics/blob/master/Introduction\%20to\%20Analysis/Introduction\%20to\%20Analysis.pdf}{Introduction to Analysis}	
		
		\bibitem{foranal}  \href{https://github.com/carsonaj/Mathematics/blob/master/Introduction\%20to\%20Fourier\%20Analysis/Introduction\%20to\%20Fourier\%20Analysis.pdf}{Introduction to Fourier Analysis}
		
		\bibitem{measure}  \href{https://github.com/carsonaj/Mathematics/blob/master/Introduction\%20to\%20Measure\%20and\%20Integration/Introduction\%20to\%20Measure\%20and\%20Integration.pdf}{Introduction to Measure and Integration}
		
		
		
	\end{thebibliography}
	
	
	
	
	
	
	
	
	
	
	
	
	
	
	
	
	
	
	
	
	
	
	
\end{document}

