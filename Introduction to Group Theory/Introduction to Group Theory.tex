\documentclass[12pt]{amsart}
\usepackage[margin=1in]{geometry} 
\usepackage{amsmath,amsthm,amssymb,setspace, mathtools}

\usepackage{color}   %May be necessary if you want to color links
\usepackage{hyperref}
\hypersetup{
	colorlinks=true, %set true if you want colored links
	linktoc=all,     %set to all if you want both sections and subsections linked
	linkcolor=black,  %choose some color if you want links to stand out
	urlcolor=cyan
}


%
%
%
\newif\ifhideproofs
%\hideproofstrue %uncomment to hide proofs
%
%
%
%
\ifhideproofs
\usepackage{environ}
\NewEnviron{hide}{}
\let\proof\hide
\let\endproof\endhide
\fi

\theoremstyle{definition}
\newtheorem{definition}{Definition}[subsection]
\newtheorem{defn}[definition]{Definition}
\newtheorem{note}[definition]{Note}
\newtheorem{thm}[definition]{Theorem}
\newtheorem{lem}[definition]{Lemma}
\newtheorem{prop}[definition]{Proposition}
\newtheorem{cor}[definition]{Corollary}
\newtheorem{conj}[definition]{Conjecture}
\newtheorem{ex}[definition]{Exercise}



\DeclareMathOperator{\supp}{supp}

\newcommand{\p}{\partial}

\newcommand{\al}{\alpha}
\newcommand{\Gam}{\Gamma}
\newcommand{\bet}{\beta} 
\newcommand{\del}{\delta} 
\newcommand{\Del}{\Delta}
\newcommand{\lam}{\lambda}  
\newcommand{\Lam}{\Lambda} 
\newcommand{\ep}{\epsilon}
\newcommand{\sig}{\sigma} 
\newcommand{\om}{\omega}
\newcommand{\Om}{\Omega}
\newcommand{\C}{\mathbb{C}}
\newcommand{\N}{\mathbb{N}}
\newcommand{\E}{\mathbb{E}}
\newcommand{\Z}{\mathbb{Z}}
\newcommand{\R}{\mathbb{R}}
\newcommand{\T}{\mathbb{T}}
\newcommand{\Q}{\mathbb{Q}}
\renewcommand{\P}{\mathbb{P}}
\newcommand{\MA}{\mathcal{A}}
\newcommand{\MC}{\mathcal{C}}
\newcommand{\MB}{\mathcal{B}}
\newcommand{\MF}{\mathcal{F}}
\newcommand{\MG}{\mathcal{G}}
\newcommand{\ML}{\mathcal{L}}
\newcommand{\MN}{\mathcal{N}}
\newcommand{\MS}{\mathcal{S}}
\newcommand{\MP}{\mathcal{P}}
\newcommand{\ME}{\mathcal{E}}
\newcommand{\MT}{\mathcal{T}}
\newcommand{\MM}{\mathcal{M}}
\newcommand{\MI}{\mathcal{I}}

\newcommand{\io}{\text{ i.o.}}
\newcommand{\ev}{\text{ ev.}}
\renewcommand{\r}{\rangle}
\renewcommand{\l}{\langle}

\newcommand{\RG}{[0,\infty]}
\newcommand{\Rg}{[0,\infty)}
\newcommand{\Ll}{L^1_{\text{loc}}(\R^n)}

\newcommand{\limfn}{\liminf \limits_{n \rightarrow \infty}}
\newcommand{\limpn}{\limsup \limits_{n \rightarrow \infty}}
\newcommand{\limn}{\lim \limits_{n \rightarrow \infty}}
\newcommand{\convt}[1]{\xrightarrow{\text{#1}}}
\newcommand{\conv}[1]{\xrightarrow{#1}} 
\newcommand{\seq}[2]{(#1_{#2})_{#2 \in \N}}

\newcommand{\loc}{\text{loc}}

\DeclareMathOperator{\sgn}{sgn}
\DeclareMathOperator{\spn}{span}

\newcommand{\Hom}{\text{Hom}}



\newcommand{\lex}[1]{\label{ex:#1}}
\newcommand{\ld}[1]{\label{defn:#1}}
\newcommand{\rex}[1]{Exercise \ref{ex:#1}}
\newcommand{\rd}[1]{Definition \ref{defn:#1}}



\begin{document}
	
	\title{Introduction to Group Theory}
	\author{Carson James}
	\maketitle
	
	\tableofcontents
	
	\newpage
	
	
	\subsection{Direct Products}
	
	\begin{defn}
	Let $G,H$ be groups. Define a product $*:(G \times H) \times (G \times H) \rightarrow G \times H$ by 
	$$(x_1,y_1) * (x_2, y_2) = (x_1x_2, y_1y_2)$$
	Then $(G \times H, *)$ is called the \textbf{direct product of $G$ and $H$}.
	\end{defn}	
	
	\begin{ex}
	\lex{1} Let $G,H$ be groups. Then the direct product $G \times H$ is a group.
	\end{ex}
	\begin{proof}
	Clear.
	\end{proof}
	
	\begin{defn} 
	Let $G,H$ be groups. Define $\pi_G :G \times H \rightarrow G$ and $\pi_H :G \times H \rightarrow H$ by $\pi_G(x,y) = x$ and $\pi_H(x,y) = y$.  Then $\pi_G$ and $\pi_H$ are respectively called the \textbf{projection maps onto $G$ and $H$}.
	\end{defn}	
	
	\begin{ex}
	\lex{2} Let $G,H$ be groups. Then 
	\begin{enumerate}
	\item $\pi_G: G \times H \rightarrow G$ and $\pi_H : G \times H \rightarrow H$ are homomorphisms
	\item $\ker \pi_G \cong H$ and $\ker \pi_H \cong G$
\end{enumerate}	 
	\end{ex}
	
	\begin{proof}\
	\begin{enumerate}
	\item Clear
	\item Define $\iota_G:G \rightarrow \ker \pi_H$ by $$\iota_G(x) = (x, e_H)$$ Then $\iota_G$ is an isomorphism. Similarly, we can define $\iota_H:H \rightarrow \ker \pi_G$ and show that it is an isomorphism.
	\end{enumerate}
	\end{proof}
	
	\begin{defn}
	Let $G,H, K$ be groups, $\phi \in \Hom(G,K)$ and $\psi \in \Hom(H, K)$. We define $\phi \times \psi: G \times H \rightarrow K$ by $\phi \times \psi(x,y) = \phi(x) \psi(y)$ 
	\end{defn}	
	
	\begin{ex}
	\lex{3} Let $G,H, K$ be groups, $\phi \in \Hom(G,K)$ and $\psi \in \Hom(H, K)$. If $K$ is abelian, then $\phi \times \psi \in Hom(G \times H,K)$.
	\end{ex}
	
	\begin{proof}
	Let $x_1, x_2 \in G$ and $y_1, y_2 \in H$. Then 
	\begin{align*}
	\phi \times \psi[(x_1, y_1)(x_2, y_2)] 
	&= \phi \times \psi (x_1x_2, y_1y_2) \\
	&= \phi(x_1x_2) \psi(y_1y_2) \\
	&= \phi(x_1)\phi(x_2)\psi(y_1)\psi(y_2) \\
	&= \phi(x_1)\psi(y_1)\phi(x_2)\psi(y_2) \\
	&= [\phi \times \psi(x_1, y_1)] [\phi \times \psi(x_2, y_2) ]
	\end{align*}
	\end{proof}
	
	\begin{ex}
	\lex{4} Let $G,H, K$ be groups and $\phi \in \Hom(G \times H, K)$. Then there exist $\phi_G \in \Hom(G,K)$, $\phi_H \in \Hom(H, K)$ such that $\phi_G \times \phi_H = \phi$.
	\end{ex}
	
	\begin{proof}
	Suppose that $K$ is abelian. Define $\iota_G \in \Hom(G, \ker \pi_H)$ and $\iota_H \in \Hom(H, \ker \pi_G)$ as in part $(2)$ of \rex{2} Define $\phi_G \in \Hom(G, K)$ and $\phi_H \in \Hom(H,K)$ by  $\phi_G = \phi \circ \iota_G$ and $\phi_H = \phi \circ \iota_H $. Let $(x,y) \in G \times H$. Then 
	\begin{align*}
	\phi_G \times \phi_H(x,y) 
	&= \phi_G(x) \phi_H(y) \\
	&= \phi \circ \iota_G(x) \phi \circ \iota_H(y) \\
	&= \phi(x, e_H)\phi(e_G, y) \\
	&= \phi(x,y) \\
	\end{align*}
	So $\phi = \phi_G \times \phi_H$
	\end{proof}
\end{document}