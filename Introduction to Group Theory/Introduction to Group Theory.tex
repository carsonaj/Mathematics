%% filename: amsbook-template.tex
%% version: 1.1
%% date: 2014/07/24
%%
%% American Mathematical Society
%% Technical Support
%% Publications Technical Group
%% 201 Charles Street
%% Providence, RI 02904
%% USA
%% tel: (401) 455-4080
%%      (800) 321-4267 (USA and Canada only)
%% fax: (401) 331-3842
%% email: tech-support@ams.org
%% 
%% Copyright 2006, 2008-2010, 2014 American Mathematical Society.
%% 
%% This work may be distributed and/or modified under the
%% conditions of the LaTeX Project Public License, either version 1.3c
%% of this license or (at your option) any later version.
%% The latest version of this license is in
%%   http://www.latex-project.org/lppl.txt
%% and version 1.3c or later is part of all distributions of LaTeX
%% version 2005/12/01 or later.
%% 
%% This work has the LPPL maintenance status `maintained'.
%% 
%% The Current Maintainer of this work is the American Mathematical
%% Society.
%%
%% ====================================================================

%    AMS-LaTeX v.2 driver file template for use with amsbook
%
%    Remove any commented or uncommented macros you do not use.

\documentclass{book}

%    For use when working on individual chapters
%\includeonly{}

%    For use when working on individual chapters
%\includeonly{}

%    Include referenced packages here.
\usepackage[left =.5in, right = .5in, top = 1in, bottom = 1in]{geometry} 
\usepackage{amsmath}
\usepackage{amsthm}
\usepackage{amssymb}
\usepackage{setspace}
\usepackage{mathtools}
\usepackage{tikz}  
\usepackage{tikz-cd}
\usepackage{tkz-fct}
\usepackage{pgfplots}
\usepackage{environ}
\usepackage{tikz-cd} 
\usepackage{enumitem}
\usepackage{color}   %May be necessary if you want to color links
%\usepackage{xr}

\usepackage{hyperref}
\hypersetup{
	colorlinks=true, %set true if you want colored links
	linktoc=all,     %set to all if you want both sections and subsections linked
	linkcolor=black,  %choose some color if you want links to stand out
	urlcolor=cyan
}
\usepackage[symbols,nogroupskip,sort=none]{glossaries-extra}

\pgfplotsset{every axis/.append style={
		axis x line=middle,    % put the x axis in the middle
		axis y line=middle,    % put the y axis in the middle
		axis line style={<->,color=black}, % arrows on the axis
		xlabel={$x$},          % default put x on x-axis
		ylabel={$y$},          % default put y on y-axis
}}


\theoremstyle{definition}
\newtheorem{definition}{Definition}[subsection]
\newtheorem{defn}[definition]{Definition}
\newtheorem{note}[definition]{Note}
\newtheorem{ax}[definition]{Axiom}
\newtheorem{thm}[definition]{Theorem}
\newtheorem{lem}[definition]{Lemma}
\newtheorem{prop}[definition]{Proposition}
\newtheorem{cor}[definition]{Corollary}
\newtheorem{conj}[definition]{Conjecture}
\newtheorem{ex}[definition]{Exercise}
\newtheorem{exmp}[definition]{Example}
\newtheorem{soln}[definition]{Solution}

\setcounter{tocdepth}{3}

% hide proofs
\newif\ifhideproofs
%\hideproofstrue %uncomment to hide proofs
\ifhideproofs
\NewEnviron{hide}{}
\let\proof\hide
\let\endproof\endhide
\fi

% lower-case greek
\newcommand{\al}{\alpha}
\newcommand{\be}{\beta}
\newcommand{\gam}{\gamma}
\newcommand{\del}{\delta}
\newcommand{\ep}{\epsilon}
\newcommand{\ze}{\zeta} 
\newcommand{\kap}{\kappa} 
\newcommand{\lam}{\lambda}  
\newcommand{\sig}{\sigma} 
\newcommand{\omi}{\omicron}
\newcommand{\up}{\upsilon}
\newcommand{\om}{\omega}

% upper-case greek
\newcommand{\Gam}{\Gamma}
\newcommand{\Del}{\Delta}
\newcommand{\Lam}{\Lambda} 
\newcommand{\Sig}{\Sigma} 
\newcommand{\Om}{\Omega}

% blackboard bold
\newcommand{\C}{\mathbb{C}}
\newcommand{\E}{\mathbb{E}}
\newcommand{\F}{\mathbb{F}}
\renewcommand{\H}{\mathbb{H}}
\newcommand{\K}{\mathbb{K}}
\newcommand{\N}{\mathbb{N}}
\renewcommand{\O}{\mathbb{O}}
\newcommand{\Q}{\mathbb{Q}}
\newcommand{\R}{\mathbb{R}}
\renewcommand{\S}{\mathbb{S}}
\newcommand{\T}{\mathbb{T}}
\newcommand{\V}{\mathbb{V}}
\newcommand{\Z}{\mathbb{Z}}

% math caligraphic
\newcommand{\MA}{\mathcal{A}}
\newcommand{\MB}{\mathcal{B}}
\newcommand{\MC}{\mathcal{C}}
\newcommand{\MD}{\mathcal{D}}
\newcommand{\ME}{\mathcal{E}}
\newcommand{\MF}{\mathcal{F}}
\newcommand{\MG}{\mathcal{G}}
\newcommand{\MH}{\mathcal{H}}
\newcommand{\MI}{\mathcal{I}}
\newcommand{\MJ}{\mathcal{J}}
\newcommand{\MK}{\mathcal{K}}
\newcommand{\ML}{\mathcal{L}}
\newcommand{\MM}{\mathcal{M}}
\newcommand{\MN}{\mathcal{N}}
\newcommand{\MO}{\mathcal{O}}
\newcommand{\MP}{\mathcal{P}}
\newcommand{\MQ}{\mathcal{Q}}
\newcommand{\MR}{\mathcal{R}}
\newcommand{\MS}{\mathcal{S}}
\newcommand{\MT}{\mathcal{T}}
\newcommand{\MU}{\mathcal{U}}
\newcommand{\MV}{\mathcal{V}}
\newcommand{\MW}{\mathcal{W}}
\newcommand{\MX}{\mathcal{X}}
\newcommand{\MY}{\mathcal{Y}}
\newcommand{\MZ}{\mathcal{Z}}

% mathfrak
\newcommand{\MFX}{\mathfrak{X}}
\newcommand{\MFg}{\mathfrak{g}}
\newcommand{\MFh}{\mathfrak{h}}

% tilde 
\newcommand{\tMA}{\tilde{\MA}}
\newcommand{\tMB}{\tilde{\MB}}
\newcommand{\tU}{\tilde{U}}
\newcommand{\tV}{\tilde{V}}
\newcommand{\tphi}{\tilde{\phi}}
\newcommand{\tpsi}{\tilde{\psi}}
\newcommand{\tF}{\tilde{F}}

\newcommand{\iid}{\stackrel{iid}{\sim}}





% label/reference
% internal label/reference
\newcommand{\lex}[1]{\label{ex:#1}}
\newcommand{\rex}[1]{Exercise \ref{ex:#1}}

\newcommand{\ld}[1]{\label{defn:#1}}
\newcommand{\rd}[1]{Definition \ref{defn:#1}}

\newcommand{\lax}[1]{\label{ax:#1}}
\newcommand{\rax}[1]{Axiom \ref{ax:#1}}

\newcommand{\lfig}[1]{\label{fig:#1}}
\newcommand{\rfig}[1]{Figure \ref{fig:#1}}

% external reference
\newcommand{\extrex}[2]{Exercise \ref{#1-ex:#2}}

\newcommand{\extrd}[2]{Definition \ref{#1-defn:#2}}

\newcommand{\extrax}[2]{Axiom \ref{#1-ax:#2}}

\newcommand{\extrfig}[2]{Figure \ref{#1-fig:#2}}

% external documents (EDIT HERE)
%\externaldocument[analysis-]{"/home/carson/Desktop/Github/Mathematics/Introduction to Analysis/Introduction to Analysis.tex"}




% math operators
\DeclareMathOperator{\supp}{supp}
\DeclareMathOperator{\sgn}{sgn}
\DeclareMathOperator{\spn}{span}
\DeclareMathOperator{\Iso}{Iso}
\DeclareMathOperator{\Eq}{Eq}
\DeclareMathOperator{\id}{id}
\DeclareMathOperator{\Aut}{Aut}
\DeclareMathOperator{\Endo}{End}
\DeclareMathOperator{\Homeo}{Homeo}
\DeclareMathOperator{\Sym}{Sym}
\DeclareMathOperator{\Alt}{Alt}
\DeclareMathOperator{\cl}{cl}
\DeclareMathOperator{\Int}{Int}
\DeclareMathOperator{\bal}{bal}
\DeclareMathOperator{\cyc}{cyc}
\DeclareMathOperator{\cnv}{conv}
\DeclareMathOperator{\epi}{epi}
\DeclareMathOperator{\dom}{dom}
\DeclareMathOperator{\cod}{cod}
\DeclareMathOperator{\codim}{codim}
\DeclareMathOperator{\Obj}{Obj}
\DeclareMathOperator{\Derivinf}{Deriv^{\infty}}
\DeclareMathOperator{\Hom}{Hom}
\DeclareMathOperator*{\argmax}{arg\,max}
\DeclareMathOperator*{\argmin}{arg\,min}
\DeclareMathOperator{\diam}{\text{diam}}
\DeclareMathOperator{\rnk}{\text{rank}}
\DeclareMathOperator{\tr}{\text{tr}}
\DeclareMathOperator{\prj}{\text{proj}}
\DeclareMathOperator{\nab}{\nabla}
\DeclareMathOperator{\diag}{\text{diag}}
\DeclareMathOperator*{\ind}{\text{ind}}
\DeclareMathOperator*{\ar}{\text{arity}}
\DeclareMathOperator*{\cur}{\text{cur}}
\DeclareMathOperator*{\Part}{\text{Part}}
\DeclareMathOperator{\Var}{\text{Var}}
\DeclareMathOperator*{\FIP}{\text{FIP}} 
\DeclareMathOperator*{\Fun}{\text{Fun}} 
\DeclareMathOperator*{\Rel}{\text{Rel}} 
\DeclareMathOperator*{\Cons}{\text{Cons}} 
\DeclareMathOperator*{\Sg}{\text{Sg}} 
\DeclareMathOperator*{\ot}{\otimes}
\DeclareMathOperator{\uni}{Uni}

% Algebra
\DeclareMathOperator{\inv}{\text{inv}}
\DeclareMathOperator{\mult}{\text{mult}}
\DeclareMathOperator{\smult}{\text{smult}}

% category theory
\DeclareMathOperator*{\Set}{\text{\tbf{Set}}}
\DeclareMathOperator*{\BanAlg}{\text{\tbf{BanAlg}}}
\DeclareMathOperator*{\Meas}{\text{\tbf{Meas}}}
\DeclareMathOperator*{\TopMeas}{\text{\tbf{TopMeas}}}
\DeclareMathOperator*{\Msrpos}{\text{\tbf{Msr}}_{+}}
\DeclareMathOperator*{\TopMsrpos}{\text{\tbf{TopMsr}}_{+}}
\DeclareMathOperator*{\TopRadMsrpos}{\text{\tbf{TopRadMsr}}_{+}}
\DeclareMathOperator*{\TopRadMsrone}{\text{\tbf{TopRadMsr}}_{1}}
\DeclareMathOperator*{\MsrC}{\text{\tbf{Msr}}_{\C}} 
\DeclareMathOperator*{\TopMsrC}{\text{\tbf{TopMsr}}_{\C}} 
\DeclareMathOperator*{\TopRadMsrC}{\text{\tbf{TopRadMsr}}_{\C}} 
\DeclareMathOperator*{\Maninf}{\text{\tbf{Man}}^{\infty}} 
\DeclareMathOperator*{\ManBndinf}{\text{\tbf{ManBnd}}^{\infty}} 
\DeclareMathOperator*{\Man0}{\text{\tbf{Man}}^{0}}
\DeclareMathOperator*{\Buninf}{\text{\tbf{Bun}}^{\infty}} 
\DeclareMathOperator*{\VecBuninf}{\text{\tbf{VecBun}}^{\infty}} 
\DeclareMathOperator*{\Field}{\text{\tbf{Field}}} 
\DeclareMathOperator*{\Mon}{\text{\tbf{Mon}}} 
\DeclareMathOperator*{\Grp}{\text{\tbf{Grp}}}
\DeclareMathOperator*{\Semgrp}{\text{\tbf{Semgrp}}}
\DeclareMathOperator*{\LieGrp}{\text{\tbf{LieGrp}}} 
\DeclareMathOperator*{\Alg}{\text{\tbf{Alg}}} 
\DeclareMathOperator*{\Vect}{\text{\tbf{Vect}}} 
\DeclareMathOperator*{\Mod}{\text{\tbf{Mod}}}
\DeclareMathOperator*{\Rep}{\text{\tbf{Rep}}} 
\DeclareMathOperator*{\URep}{\text{\tbf{URep}}}
\DeclareMathOperator*{\Ban}{\text{\tbf{Ban}}} 
\DeclareMathOperator*{\Hilb}{\text{\tbf{Hilb}}} 
\DeclareMathOperator*{\Prob}{\text{\tbf{Prob}}} 
\DeclareMathOperator*{\PrinBuninf}{\text{\tbf{PrinBun}}^{\infty}}

\DeclareMathOperator*{\Top}{\text{\tbf{Top}}}
\DeclareMathOperator*{\TopField}{\text{\tbf{TopField}}} 
\DeclareMathOperator*{\TopMon}{\text{\tbf{TopMon}}} 
\DeclareMathOperator*{\TopGrp}{\text{\tbf{TopGrp}}}
\DeclareMathOperator*{\TopVect}{\text{\tbf{TopVect}}} 
\DeclareMathOperator*{\TopEq}{\text{\tbf{TopEq}}}

\DeclareMathOperator*{\VectR}{\text{\tbf{Vect}}_{\R}}
\DeclareMathOperator*{\VectC}{\text{\tbf{Vect}}_{\C}} 
\DeclareMathOperator*{\VectK}{\text{\tbf{Vect}}_{\K}}
\DeclareMathOperator*{\Cat}{\text{\tbf{Cat}}}
\DeclareMathOperator*{\0}{\mbf{0}}
\DeclareMathOperator*{\1}{\mbf{1}}


\DeclareMathOperator*{\Cone}{\text{\tbf{Cone}}}

\DeclareMathOperator*{\Cocone}{\text{\tbf{Cocone}}}


% Algebra
\DeclareMathOperator{\End}{\text{End}} 
\DeclareMathOperator{\rep}{\text{Rep}} 




% notation
\renewcommand{\r}{\rangle}
\renewcommand{\l}{\langle}
\renewcommand{\div}{\text{div}}
\renewcommand{\Re}{\text{Re} \,}
\renewcommand{\Im}{\text{Im} \,}
\newcommand{\Img}{\text{Img} \,}
\newcommand{\grad}{\text{grad}}
\newcommand{\tbf}[1]{\textbf{#1}}
\newcommand{\tcb}[1]{\textcolor{blue}{#1}}
\newcommand{\tcr}[1]{\textcolor{red}{#1}}
\newcommand{\mbf}[1]{\mathbf{#1}}
\newcommand{\ol}[1]{\overline{#1}}
\newcommand{\ub}[1]{\underbar{#1}}
\newcommand{\tl}[1]{\tilde{#1}}
\newcommand{\p}{\partial}
\newcommand{\Tn}[1]{T^{r_{#1}}_{s_{#1}}(V)}
\newcommand{\Tnp}{T^{r_1 + r_2}_{s_1 + s_2}(V)}
\newcommand{\Perm}{\text{Perm}}
\newcommand{\wh}[1]{\widehat{#1}}
\newcommand{\wt}[1]{\widetilde{#1}}
\newcommand{\defeq}{\vcentcolon=}
\newcommand{\Con}{\text{Con}}
\newcommand{\ConKos}{\text{Con}_{\text{Kos}}}
\newcommand{\trl}{\triangleleft}
\newcommand{\trr}{\triangleright}
\newcommand{\alg}{\text{alg}}
\newcommand{\Triv}{\text{Triv}}
\newcommand{\Der}{\text{Der}}
\newcommand{\cnj}{\text{conj}}

\newcommand{\lcm}{\text{lcm}}
\newcommand{\Imax}{\MI_{\text{max}}}


\DeclareMathOperator*{\Rl}{\text{Re}}
\DeclareMathOperator*{\Imn}{\text{Imn}}



% limits
\newcommand{\limfn}{\liminf \limits_{n \rightarrow \infty}}
\newcommand{\limpn}{\limsup \limits_{n \rightarrow \infty}}
\newcommand{\limn}{\lim \limits_{n \rightarrow \infty}}
\newcommand{\convt}[1]{\xrightarrow{\text{#1}}}
\newcommand{\conv}[1]{\xrightarrow{#1}} 
\newcommand{\seq}[2]{(#1_{#2})_{#2 \in \N}}

% intervals
\newcommand{\RG}{[0,\infty]}
\newcommand{\Rg}{[0,\infty)}
\newcommand{\Rgp}{(0,\infty)}
\newcommand{\Ru}{(\infty, \infty]}
\newcommand{\Rd}{[\infty, \infty)}
\newcommand{\ui}{[0,1]}

% integration \newcommand{\dm}{\, d m}
\newcommand{\dmu}{\, d \mu}
\newcommand{\dnu}{\, d \nu}
\newcommand{\dlam}{\, d \lambda}
\newcommand{\dP}{\, d P}
\newcommand{\dQ}{\, d Q}
\newcommand{\dm}{\, d m}
\newcommand{\dsh}{\, d \#}

% abreviations 
\newcommand{\lsc}{lower semicontinuous}

% misc
\newcommand{\as}[1]{\overset{#1}{\sim}}
\newcommand{\astx}[1]{\overset{\text{#1}}{\sim}}
\newcommand{\io}{\text{ i.o.}}
%\newcommand{\ev}{\text{ ev.}}
\newcommand{\Ll}{L^1_{\text{loc}}(\R^n)}

\newcommand{\loc}{\text{loc}}
\newcommand{\BV}{\text{BV}}
\newcommand{\NBV}{\text{NBV}}
\newcommand{\TV}{\text{TV}}

\newcommand{\op}[1]{\mathcal{#1}^{\text{op}}}

% Glossary - Notation
\glsxtrnewsymbol[description={finite measures on $(X, \MA)$}]{n000001}{$\MM_+(X, \MA)$}
\glsxtrnewsymbol[description={velocity}]{v}{\ensuremath{v}}


\makeindex

\begin{document}
	
	\frontmatter
	
	\title{Introduction to Group Theory}
	
	%    Remove any unused author tags.
	
	%    author one information
	\author{Carson James}
	\thanks{}
	
	\date{}
	
	\maketitle
	
	%    Dedication.  If the dedication is longer than a line or two,
	%    remove the centering instructions and the line break.
	%\cleardoublepage
	%\thispagestyle{empty}
	%\vspace*{13.5pc}
	%\begin{center}
	%  Dedication text (use \\[2pt] for line break if necessary)
	%\end{center}
	%\cleardoublepage
	
	%    Change page number to 6 if a dedication is present.
	\setcounter{page}{4}
	
	\tableofcontents
	\printunsrtglossary[type=symbols,style=long,title={Notation}]
	
	%    Include unnumbered chapters (preface, acknowledgments, etc.) here.
	%\include{}
	
	\mainmatter
	%    Include main chapters here.
	%\include{}
	
	\chapter*{Preface}
	\addcontentsline{toc}{chapter}{Preface}
	
	\begin{flushleft}
		\href{https://creativecommons.org/licenses/by-nc-sa/4.0/legalcode.txt}{cc-by-nc-sa}
	\end{flushleft}
	
	\newpage
	
	
	
	\chapter{Prelimiaries} 
	
	\section{Category Theory}
	
	\begin{itemize}
		\item $\Hilb$:  
		\begin{itemize}
			\item $\Obj(\Hilb) = \{H: H \text{ is a Hilbert space}\}$
			\item $\Hom_{\Hilb}(H_1, H_2) = \{T \in \Vect_{\C}(H_1, H_2): T \text{ is continuous}\}$
		\end{itemize}
		\item $\Mon$
	\end{itemize}

	
	
	
	
	
	
	
	
	
	
	
	
	
	
	
	
	
	
	
	
	
	
	
	
	
	
	
	
	
	


	\subsection{The Unitary Group}
	
	\begin{defn}
		Let $H_1, H_2 \in \Obj(\Hilb)$. We define the \tbf{unitary group from $H_1$ to $H_2$}, denoted $U(H_1, H_2)$, by 
		$$U(H_1, H_2) = \{T \in \Iso_{\Hilb}(H_1, H_2): T^* = T^{-1}\}$$ 
		We write $U(H)$ in place of $U(H,H)$. We equip $U(H_1, H_2)$ with the strong operator topology. 
	\end{defn}
	
	\begin{ex}
		Let $H \in \Obj(\Hilb)$. Then $\MT_{U(H)}^s = \MT_{U(H)}^w$. \tcb{strong weak operator topologies coincide}
	\end{ex}
	
	\begin{ex}
		Let $H \in \Obj(\Hilb)$. Then $U(H)$ is a topological group.
	\end{ex}
	
	\begin{proof}
		content...
	\end{proof}
	
	
	
	
	
	
	
	
	
	
	
	
	
	
	
	
	
	
	
	
	
	
	
	
	
	
	
	
	
	\newpage
	\chapter{Representation Theory}
	
	\section{Group Representations}

	
	
	
	
	
	
	
	
	
	
	
	
	
	
	
	
	
	
	
	
	
	
	
	
	
	
	\subsection{Unitary representations}
	
		\begin{defn}
		Let $G \in \Obj(\TopGrp)$, $H \in \Obj(\Hilb)$ and $\pi \in \Hom_{\TopGrp}(G, U(H))$. Then $(H, \pi)$ is said to be a \tbf{unitary representation of $G$}. We define the \tbf{dimension of $(H, \pi)$}, denoted $\dim (H, \pi)$, by $\dim (H, \pi) \defeq \dim V$.
	\end{defn}
	
	\begin{defn}
		Let $G \in \Obj(\TopGrp)$, $(H_{\pi}, \pi)$, $(H_{\rho}, \rho)$ unitary representations of $G$ and $T \in \Hom_{\Hilb}(H_{\pi}, H_{\rho})$. Then $T$ is said to be \tbf{$(\pi, \rho)$-equivariant} if for each $g \in G$, $T \circ \pi(g) = \rho(g) \circ T$, i.e. the following diagram commutes:
		\[
		\begin{tikzcd}
			H_{\pi} \arrow[r, "T"] \arrow[d, "\pi(g)"']  & H_{\rho}  \arrow[d, "\rho(g)"] \\
			H_{\pi} \arrow[r, "T"']                   & H_{\rho}
		\end{tikzcd}
		\]
	\end{defn}
	
	\begin{defn} \ld{13009}
		Let $G \in \Obj(\TopGrp)$. We define $\URep(G)$ by 
		\begin{itemize}
			\item $\Obj(\URep(G)) = \{(H, \pi): (H, \pi) \text{ is a unitary representation of $G$}\}$.
			\item for $(H_{\pi}, \pi),(H_{\rho}, \rho) \in \Obj(\URep(G))$, 
			$$\Hom_{\URep(G)}((H_{\pi}, \pi), (H_{\rho}, \rho)) = \{T \in \Hom_{\Hilb}(H_{\pi}, H_{\rho}): T \text{ is $(\pi, \rho)$-equivariant} \}$$
			\item for $(H_{\pi}, \pi), (H_{\rho}, \rho), (H_{\mu}, \mu) \in \Obj(\URep(G))$, $T \in \Hom_{\URep(G)}((H_{\pi}, \pi), (H_{\rho}, \rho))$ and \\
			$S \in  \Hom_{\URep(G)}((H_{\rho}, \rho), (H_{\mu}, \mu))$, 
			$$S \circ_{\URep(G)} T = S \circ T$$
		\end{itemize}
	\end{defn}
	
	\begin{ex}
		Let $G \in \Obj(\TopGrp)$. Then $\URep(G)$ is a category.
	\end{ex}
	
	\begin{proof}
		\tcb{FINISH!!!}
	\end{proof}
	
	\begin{defn}
		Let $G \in \Obj(\TopGrp)$ and $(H_{\pi}, \pi), (H_{\rho}, \rho) \in \URep(G)$. Then $(H_{\pi}, \pi)$ is said to be \tbf{unitarily equivalent} to $(H_{\rho}, \rho)$, denoted $(H_{\pi}, \pi) \equiv (H_{\rho}, \rho)$, if $\Hom_{\URep(G)}((H_{\pi}, \pi), (H_{\rho}, \rho)) \cap U(H_{\pi}, H_{\rho}) \neq \varnothing$.
	\end{defn}
	
	\begin{note}
		Let $\pi \in \Hom_{\TopGrp}(G, U(H))$. Since $U(H)$ is equipped with the strong operator topology, we have that for each $u \in H$, the map $g \mapsto \pi(g)u$ is continuous.  
	\end{note}
	
	\begin{defn}
		Let $G \in \Obj(\TopGrp)$ and $(H, \pi) \in \Obj(\URep(G))$. We define the \tbf{induced group action of $G$ on $H$}, denoted $\phi_{(H, \pi)}: G \times H \rightarrow H$, by 
		$$\phi_{(H, \pi)}(g, v) = \pi(g)v$$ 
	\end{defn}
	
	\begin{note}
		When the context is clear, we write $g \cdot v$ in place of $\phi_{(H, \pi)}(g, v)$. 
	\end{note}
	
	\begin{ex}
		Let $G \in \Obj(\TopGrp)$ and $(H, \pi) \in \Obj(\URep(G))$. Then 
		\begin{enumerate}
			\item $\phi_{(H, \pi)}$ is a linear group action. 
			\item $G$ is locally compact implies that $\phi_{(H, \pi)}$ is continuous
		\end{enumerate}
	\end{ex}
	
	\begin{proof}\
		\begin{enumerate}
			\item 
			\begin{itemize}
				\item Let $g,h \in G$ and $v \in H$. 
				\begin{enumerate}
					\item Since $\pi \in \Hom_{\TopGrp}(G, U(H))$, 
					\begin{align*}
						e \cdot v
						& = \pi(e)v \\
						& = \id_H v \\
						& = v
					\end{align*}
					\item Since $\pi \in \Hom_{\TopGrp}(G, U(H))$, 
					\begin{align*}
						g \cdot (h \cdot v) 
						& = \pi(g)[\pi(h) v] \\
						& = [\pi(g) \pi(h)] v \\
						& = \pi(gh) v \\
						& = (gh) \cdot v
					\end{align*}
				\end{enumerate}
				Since $g,h \in G$ and $v \in H$ are arbitrary, $\phi_{(H, \pi)}$ is a group action of $G$ on $H$.
				\item Let $g \in G$, $\lam \in \C$ and $v,w \in H$. Then 
				\begin{align*}
					g \cdot (\lam v + w) \\
					& = \pi(g)(\lam v + w) \\
					& = \lam \pi(g) v + \pi(g) w \\
					& = \lam g \cdot v + g \cdot w
				\end{align*} 
				Since $g \in G$, $\lam \in \C$ and $v,w \in H$ are arbitrary, $\phi_{(H, \pi)}$ is a linear action.
			\end{itemize}
			\item Suppose that $G$ is locally compact. Let $(g_0, v_0) \in G \times H$ and $\ep > 0$. Since $G$ is locally compact, there exists $K \subset G$ such that $g_0 \in \Int K$ and $K$ is compact. Let $v \in H$. Define $f_v:G \rightarrow H$ by $f_v(g) = g \cdot v$. Since $\pi: G \rightarrow U(H)$ is continuous, $f_v$ is continuous. Thus $\|f_v\|$ is continuous. Since $K$ is compact, $\|f_v\|(K)$ is compact. Thus 
			\begin{align*}
				\sup\limits_{g \in K} \|g \cdot v\| 
				& = \sup\limits_{g \in K} \|f_v(g)\| \\
				& < \infty
			\end{align*}  
			Since $v \in H$ is arbitrary, we have that for each $v \in H$, $\sup\limits_{g \in K} \|g \cdot v\| < \infty$. The uniform boundedness principle implies that  there exists $M > 0$ such that $\sup\limits_{g \in K} \| \pi(g) \| \leq M$. Since $f_{v_0}$ is continuous, there exists $U \subset K$ such that $U$ is open, $g_0 \in U$, and $f_{v_0}(U) \subset B(f_{v_0}(g_0), \ep/2)$. Let $(g_1, v_1) \in U \times B(v_0, (2M)^{-1} \ep)$. Then 
			\begin{align*}
				\|\phi_{(H, \pi)}(g_0, v_0) - \phi_{(H, \pi)}(g_1, v_1)\|
				& = \|g_0 \cdot v_0 - g_1 \cdot v_1\| \\
				& \leq 	\|g_0 \cdot v_0 - g_1 \cdot v_0\| + \|g_1 \cdot v_0 - g_1 \cdot v_1\| \\
				& = \|f_{v_0}(g_0) - f_{v_0}(g_1)\| + \|\pi(g_1)(v_0 - v_1)\| \\
				& \leq \|f_{v_0}(g_0) - f_{v_0}(g_1)\| + \|\pi(g_1)\| \|v_0 - v_1\| \\
				& \leq \|f_{v_0}(g_0) - f_{v_0}(g_1)\| + M \|v_0 - v_1\| \\
				& \leq \frac{\ep}{2} + M \frac{\ep}{2M} \\
				& = \ep
			\end{align*}
			Since $\ep > 0$ is arbitrary, we have that $\phi_{(H, \pi)}$ is continuous at $(g_0, v_0)$. Since $(g_0, v_0) \in G \times H$ is arbitrary, we have that $\phi_{(H, \pi)}: G \times H \rightarrow H$ is continuous.
		\end{enumerate}
	\end{proof}


	
	

































	
	
	
	
	
	
	\subsection{Subrepresentations}
	
	\begin{defn}
		Let $G \in \Obj(\TopGrp)$, $(H, \pi) \in \Obj(\URep(G))$ and $E \subset H$ a closed subspace. Then $E$ is said to be 
		\begin{itemize}
			\item \tbf{nontrivial} if $E \neq H, \varnothing$
			\item \tbf{$(H, \pi)$-invariant} if for each $g \in G$, $\pi(g)(E) = E$
		\end{itemize} 
	\end{defn}

	\begin{defn}
		Let $G \in \Obj(\TopGrp)$ and $\K \in \Obj(\Field)$ and $(H, \pi) \in \Obj(\URep(G))$. Then 
		\begin{itemize}
			\item $(H, \pi)$ is said to be \tbf{reducible} if there exists a closed subspace $E \subset H$ such that $E$ is not trivial and $E$ is $(H, \pi)$-invariant 
			\item $(H, \pi)$ is said to be \tbf{irreducible} if $(H, \pi)$ is not reducible.
		\end{itemize}
	\end{defn}

	\begin{ex}
		Let $G \in \Obj(\TopGrp)$ and $(H, \pi) \in \Obj(\URep(G))$ and $E \subset H$ a closed subspace. Suppose that $E$ is $(H, \pi)$-invariant. Then for each $g \in G$, $\pi(g)|_E \in U(E)$. 
	\end{ex}

	\begin{proof}
		Let $g \in G$. Since $E$ is $(H, \pi)$-invariant, for each $g \in G$, $\pi(g)(E) = E$. Since $\pi(g) \in U(H)$, $\pi(g)|_E \in U(E)$.
	\end{proof}

	\begin{defn}
		Let $G \in \Obj(\TopGrp)$ and $(H, \pi) \in \Obj(\URep(G))$ and $E \subset H$ a closed subspace. Suppose that $E$ is $(H, \pi)$-invariant. 
		\begin{itemize}
			\item We define $\pi^E \in \Hom_{\TopGrp}(G, U(E))$ by $\pi^E(g) \defeq \pi(g)|_E$
			\item We define the \tbf{restriction $(H, \pi)$ to $E$}, denoted $(H, \pi)|_E$, by $(H, \pi)|_E \defeq (E, \pi^E)$
		\end{itemize}
	\end{defn}

	\begin{ex}
		Let $G \in \Obj(\TopGrp)$ and $(H, \pi) \in \Obj(\URep(G))$ and $E \subset H$ a closed subspace. 
		\begin{enumerate}
			\item If $E$ is nontrivial, then $E^{\perp}$ is nontrivial.
			\item If $E$ is $(H, \pi)$-invariant, then $E^{\perp}$ is  $(H, \pi)$-invariant.  
		\end{enumerate}
	\end{ex}
	
	\begin{proof}\
		\begin{enumerate}
			\item Suppose that $E$ is nontrivial. Then $E \neq \{0\}, H$. Then $E^{\perp} \neq \{0\}, H$. Thus $E^{\perp}$ is nontrivial.
			\item Suppose that $E$ is $(H, \pi)$-invariant. Let $g \in G$. Since $\pi(g) \in U(H)$ and $\pi(g)(E) = E$, \tcb{An exercise in the analysis notes section on Hilbert spaces} implies that $\pi(g)(E^{\perp}) = E^{\perp}$. Since $g \in G$ is arbitrary, $E^{\perp}$ is $(H, \pi)$-invariant. 
		\end{enumerate}
	\end{proof}

	\begin{defn}
		Let $G \in \Obj(\TopGrp)$, $(H, \pi) \in \Obj(\URep(G))$ and $u \in H$. We define the \tbf{cyclic subspace of $H$ generated by $u$ under $(H, \pi)$}, denoted $\cyc_{(H, \pi)}(u)$, by 
		$$\cyc_{(H, \pi)}(u) \defeq \cl \spn (\phi_{(H, \pi)}(G, u))$$
	\end{defn}

	\begin{note}
		When the context is clear, we write $\cyc(u)$ in place of $\cyc_{(H, \pi)}(u)$.
	\end{note}
	
	\begin{ex}
		Let $G \in \Obj(\TopGrp)$, $(H, \pi) \in \Obj(\URep(G))$ and $u \in H$. Then $\cyc(u)$ is $(H, \pi)$-invariant. \tcb{this should largely be a result about linear group actions.}
	\end{ex}
	
	\begin{proof}
		Let $g \in G$. Since $G$ acts linearly and homeomorphically on $H$, 
		\begin{align*}
			g \cdot \cyc(u) 
			& = g \cdot \cl \spn (G \cdot u) \\
			& = \cl g \cdot \spn (G \cdot u) \\
			& = \cl \spn [g \cdot (G \cdot u)] \\
			& = \cl \spn (G \cdot u) \\
			& = \cyc(u)
		\end{align*}
		Since $g \in G$ is arbitrary, $\cyc(u)$ is $G$-invariant. 
	\end{proof}
	
	\begin{defn}
		Let $G \in \Obj(\TopGrp)$ and $(H, \pi) \in \Obj(\URep(G))$. 
		\begin{itemize}
			\item Let $u \in H$. Then $u$ is said to be \tbf{$(H, \pi)$-cyclic} if $\cyc(u) = H$. 
			\item Then $(H, \pi)$ is said to be \tbf{cyclic} if there exists $u \in H$ such that $u$ is $(H, \pi)$-cyclic.
		\end{itemize}
	\end{defn}

	











	







	\subsection{Direct Sum of Representations}
	
	\begin{defn}
		Let $G \in \Obj(\TopGrp)$ and $(H_{\al}, \pi_{\al})_{\al \in A} \subset \Obj(\URep(G))$. 
		\begin{itemize}
			\item We define $\bigoplus\limits_{\al \in A} \pi_{\al} \in \Hom_{\TopGrp}(G, U(\bigoplus\limits_{\al \in A} H_{\al}))$ by 
			$$\bigg[ \bigoplus\limits_{\al \in A} \pi_{\al} \bigg](g) = \bigoplus\limits_{\al \in A} \pi_{\al}(g)$$
			\item We define the \tbf{direct sum} of $(H_{\al}, \pi_{\al})_{\al \in A}$, denoted $\bigoplus\limits_{\al \in A} (H_{\al}, \pi_{\al})$, by 
			$$\bigoplus\limits_{\al \in A} (H_{\al}, \pi_{\al}) = \bigg(\bigoplus\limits_{\al \in A} H_{\al}, \bigoplus\limits_{\al \in A} \pi_{\al} \bigg)$$ 
		\end{itemize}
	\end{defn}

	\begin{note}
		\tcb{FINISH!!!} the last definition works for internal or external direct sum, just need to define inner or external sum of $H_{\al}$ and $\pi_{\al}$ in either case. 
	\end{note}

	\begin{ex}
		Let $G \in \Obj(\TopGrp)$, $(H, \pi) \in \Obj(\URep(G))$ and $E \subset H$ a closed subspace. If $E$ is nontrivial and $(H, \pi)$-invariant, then $(H, \pi) = (E \oplus E^{\perp}, \pi^E \oplus \pi^{E^{\perp}})$.
	\end{ex}
	
	\begin{proof}
		Suppose that  $E$ is nontrivial and $(H, \pi)$-invariant. \tcb{A previous exercise} implies that $E^{\perp}$ is nontrivial and $(H, \pi)$-invariant. \tcb{It is clear that} $H = E \oplus E^{\perp}$. Let $g \in G$ and $u \in H$. Since $H = E \oplus E^{\perp}$, there exists $v \in E$ and $w \in E^{\perp}$ such that $u = v + w$. Then  
		\begin{align*}
			\pi(g)(u)
			& = \pi(g)(v +w) \\
			& = \pi(g)(v) + \pi(g)(w) \\
			& = \pi(g)|_E(v) + \pi(g)|_{E^{\perp}}(w) \\
			& = \pi^E(g)(v) + \pi^{E^{\perp}}(g)(w) \\
			& = [\pi^E(g) \oplus \pi^E(g)](v+w) \\
			& = [\pi^E \oplus \pi^E](g)(v+w) \\
			& = [\pi^E \oplus \pi^E](g)(u)
		\end{align*}
		 Since $u \in H$ is arbitrary, $\pi(g) = [\pi^E \oplus \pi^E](g)$. Since $g \in G$ is arbitrary, $\pi = \pi^E \oplus \pi^E$.
	\end{proof}
	
	
	\begin{defn}
		Let $G \in \Obj(\TopGrp)$, $(H, \pi) \in \Obj(\URep(G))$ and $\ME \subset \MP(H)$. Then $\ME$ is said to be an \tbf{$(H, \pi)$-orthocyclic system} if for each $E, F \in \ME$,
		\begin{enumerate}
			\item $E$ is a closed subspace of $H$
			\item $(H, \pi)|_E$ is cyclic
			\item if $E \neq F$, then $E \perp F$
		\end{enumerate}
	\end{defn}

	\begin{ex}
		Let $G \in \Obj(\TopGrp)$ and $(H, \pi) \in \Obj(\URep(G))$. Then there exists $(H_{\al}, \pi_{\al})_{\al \in A} \subset \Obj(\URep(G))$ such that for each $\al \in A$, $(H_{\al}, \pi_{\al})$ is cyclic and $(H, \pi) =  \bigoplus\limits_{\al \in A} (H_{\al}, \pi_{\al}) $. \\
		\tbf{Hint:} Zorn's lemma
	\end{ex}



	\begin{proof}
		Define $\MP = \{\ME: \text{$\ME$ is an $(H, \pi)$-orthocyclic system}\}$. We partially order $\MP$ by inclusion. Let $\MC \subset \MP$ be a chain. Set $\ME_0 = \bigcup\limits_{\ME \in \MC} \ME$. Let $E_1, E_2 \in \ME_0$. Then there exist $\ME_1, \ME_2 \in \MC$ such that $E_1 \in \ME_1$ and $E_2 \in \ME_2$. Since $\MC$ is a chain, $\ME_1 \subset \ME_2$ or $\ME_2 \subset \ME_1$. 
		
		Suppose that $\ME_1 \subset \ME_2$. Then $E_1 \in \ME_2$. Since $\ME_2$ is an $(H, \pi)$-orthocyclic system, we have that $E_1$ is a closed subspaces of $H$, $(H, \pi)|_{E_1}$ is cyclic and if $E_1 \neq E_2$, then $E_1 \perp E_2$. Similarly, $\ME_2 \subset \ME_1$ implies the same conclusion. Since $E_1, E_2 \in \ME_0$ are arbitrary, we have that for each $E_1, E_2 \in \ME_0$ 
	 	\begin{enumerate}
	 		\item $E_1$ is a closed subspaces of $H$ and $E_1$ is $(H, \pi)$-invariant
	 		\item $(H, \pi)|_{E_1}$ is cyclic
	 		\item if $E_1 \neq E_2$, then $E_1 \perp E_2$
	 	\end{enumerate}
 	
 		Thus $\ME_0$ is an $(H, \pi)$-orthocyclic system. Hence $\ME_0 \in \MP$. By construction, for each $\ME \in \MC$, $\ME \subset \ME_0$. So $\ME_0$ is an upper bound of $\MC$. Since $\MC \subset \MP$ such that $\MC$ is a chain is arbitrary, we have that for each $\MC \subset \MP$, if $\MC$ is a chain, then there exists $\ME_0 \in \MP$ such that $\ME_0$ is an upper bound of $\MC$. Zorn's lemma implies that there exists $\ME \in \MP$ such that $\ME$ is maximal. Set $E =  \bigoplus\limits_{E_0 \in \ME} E_0$. For the sake of contradiction, suppose that $H \neq E$. Then $E^{\perp} \neq \{0\}$. Thus there exists $u \in E^{\perp}$ such that $u \neq 0$. Therefore $\cyc(u) \neq 0$ and $\cyc(u) \subset E^{\perp}$. Let $E_0 \in \ME$. By construction, $E_0 \subset E$. Thus
 		\begin{align*}
 			\cyc(u) 
 			& \subset E^{\perp} \\
 			& \subset E_0^{\perp}
 		\end{align*}
 		Since $E_0 \in \ME$ is arbitrary, we have that for each $E_0 \in \ME$, $\cyc(u) \subset E_0^{\perp}$. Set $\ME' = \ME \cup \{\cyc(u)\}$. Then for each $E,F \in \ME'$, 
 		\begin{enumerate}
 			\item $E$ is a closed subspaces of $H$ and $E$ is $(H, \pi)$-invariant
 			\item $(H, \pi)|_{E}$ is cyclic
 			\item if $E \neq F$, then $E \perp F$
 		\end{enumerate}
 		Hence $\ME' \in \MP$. Since $\ME \subset \ME'$ and $\ME$
	\end{proof}


	
	
	
	
	
	
	
	
	
	
	
	
	
	
	
	

	
	
	




















	\newpage 

	\section{Tannaka Duality}
	
	\begin{defn}
		Let $G \in \Obj(\TopGrp)$. We define the \tbf{forgetful functor from $\URep(G)$ to $\Hilb$}, denoted $U: \URep(G) \rightarrow \Hilb$, by 
		\begin{itemize}
			\item $U (H, \pi) = H$, \quad $(H, \pi) \in \Obj(\URep(G))$
			\item $U (T) = T$, \quad $T \in \Hom_{\URep(G)}((H_{\pi}, \pi), (H_{\rho}, \rho))$.
		\end{itemize}
	\end{defn}
	
	
	\tcb{Need to find out if quotienting by equivalence of isomorphism makes $\URep(G)$ a small category so that we can talk about the functor category $\Hilb^{\URep(G)}$ containing the forgetful functor as an object.}

	\begin{defn}
		Let $G \in \Obj(\TopGrp)$ and $g \in G$. We define $\hat{g}: U \Rightarrow U$ by 
		$$\hat{g}_{(H, \pi)} = \pi(g)$$
	\end{defn}

	\begin{ex}
		Let $G \in \Obj(\TopGrp)$ and $g \in G$. Then 
		\begin{enumerate}
			\item $\hat{g}: U \Rightarrow U$ is a natural transformation.
			\item $\hat{g} \in \Aut_{\Hilb^{\URep(G)}}(U)$
		\end{enumerate}
	\end{ex}

	\begin{proof}\
		\begin{enumerate}
			\item \begin{enumerate}
				\item Let $(H, \pi) \in \Obj(\URep(G))$. By definition,  
				\begin{align*}
					\hat{g}_{(H, \pi)}
					& = \pi(g) \\
					& \in U(H) \\
					& \subset \Aut_{\Hilb}(U(H, \pi))
				\end{align*}
				\item Let $(H_{\pi}, \pi), (H_{\rho}, \rho) \in \Obj(\URep(G))$ and $T \in \Hom_{\URep(G)}((H_{\pi}, \pi), (H_{\rho}, \rho))$. By definition, $T \in \Hom_{\Hilb}(H_{\pi}, H_{\rho})$ and $T$ is $(\pi, \rho)$-equivariant. Therefore
				\begin{align*}
					U(T) \circ \hat{g}_{(H_{\pi}, \pi)}
					& = T \circ \pi(g) \\
					& = \rho(g) \circ T \\
					& = \hat{g}_{(H_{\rho}, \rho)} \circ U(T) 
				\end{align*}
				i.e. the following diagram commutes: 
				\[ 
				\begin{tikzcd}
					U(H_{\pi}, \pi)  \arrow[r, "\hat{g}_{(H_{\pi} ,\pi)}"]  \arrow[d, "U(T)"']  & U(H_{\pi}, \pi)   \arrow[d, "U(T)"]\\
					U(H_{\rho}, \rho) \arrow[r, "\hat{g}_{(H_{\rho}, \rho)}"] &  U(H_{\rho}, \rho) \\
				\end{tikzcd}
				= 
				\begin{tikzcd}
					H_{\pi}  \arrow[r, "\pi(g)"]  \arrow[d, "T"']  & H_{\pi}   \arrow[d, "T"]\\
					H_{\rho} \arrow[r, "\rho(g)"] &  H_{\rho} \\
				\end{tikzcd}
				\]
			\end{enumerate}
			Thus $\hat{g}: U \Rightarrow U$ is a natural transformation. 
			\item Set $h = g^{-1}$. Part $(1)$ implies that $\hat{g}, \hat{h} \in \Endo_{\Hilb^{\URep(G)}}(U)$. Let $(H, \pi) \in \Obj(\URep(G))$. Then 
			\begin{align*}
				(\hat{g} \circ \hat{h})_{(H, \pi)}
				& = \hat{g}_{(H, \pi)}
			\end{align*}
			The previous part implies that  
			\begin{align*}
				\hat{g} 
				& \in \Hom_{\TopVect_{\C}^{\URep(G)}}(U, U) \\
				& = \End_{\TopVect_{\C}^{\URep(G)}}(U)
			\end{align*} 
		\end{enumerate}
	\end{proof}
	
	\begin{defn}
		Let $G \in \Obj(\TopGrp)$ and $(H, \pi) \in \Obj(\URep(G))$. We define the \tbf{$(H, \pi)$-projection}, denoted $\pi_{(H, \pi)}: \End_{\TopVect_{\C}^{\URep(G)}}(U) \rightarrow \End_{\TopVect_{\C}}(V)$, by $\pi_{(H, \pi)}(\al) = \al_{(H, \pi)}$. We define the \tbf{topology of endomorphisms of $U$}, denoted $\MT_{\ME(U)}$, by 
		$$\MT_{\ME(U)} = \tau(\pi_{(H, \pi)}: (H, \pi) \in \URep(G))$$
	\end{defn}
	
	\begin{defn}
		\tcb{define addition of endomorphisms of $U$ pointwise}
	\end{defn}
	
	\begin{ex}
		Let $G \in \Obj(\TopGrp)$. Then $(\Aut_{\TopVect_{\C}^{\URep(G)}}(U), \MT_{\ME(U)})$ is a topological unital algebra.
	\end{ex}

	\begin{proof}
		
	\end{proof}
	
	
	
	
	
	
	
	
	
	
	
	
	
	
	
	
	
	
	
	
	
	
	
	
	
	
	
	
	
	
	
	
	
	
	
	
	
	
	\newpage
	\chapter{Groupoids}
	
	\begin{defn}
		
	\end{defn}
	
	
	
	
	
	
	
	
	
	
	
	
	
	\backmatter
	\begin{thebibliography}{4}
		\bibitem{algebra} \href{https://github.com/carsonaj/Mathematics/blob/master/Introduction\%20to\%20Algebra/Introduction\%20to\%20Algebra.pdf}{Introduction to Algebra}
		
		\bibitem{analysis}  \href{https://github.com/carsonaj/Mathematics/blob/master/Introduction\%20to\%20Analysis/Introduction\%20to\%20Analysis.pdf}{Introduction to Analysis}	
		
		\bibitem{foranal}  \href{https://github.com/carsonaj/Mathematics/blob/master/Introduction\%20to\%20Fourier\%20Analysis/Introduction\%20to\%20Fourier\%20Analysis.pdf}{Introduction to Fourier Analysis}
		
		\bibitem{measure}  \href{https://github.com/carsonaj/Mathematics/blob/master/Introduction\%20to\%20Measure\%20and\%20Integration/Introduction\%20to\%20Measure\%20and\%20Integration.pdf}{Introduction to Measure and Integration}
		
		
		
	\end{thebibliography}
	
	
	
	
	
	
	
	
	
	
	
	
	
	
	
	
	
	
	
	
	
	
	
\end{document}

