\documentclass[12pt]{amsart}
\usepackage[margin=1in]{geometry} 
\usepackage{amsmath,amsthm,amssymb,setspace, mathtools}
\usepackage{physics}

\usepackage{color}   %May be necessary if you want to color links
\usepackage{hyperref}
\hypersetup{
	colorlinks=true, %set true if you want colored links
	linktoc=all,     %set to all if you want both sections and subsections linked
	linkcolor=black,  %choose some color if you want links to stand out
	urlcolor=cyan
}


%
%
%
\newif\ifhideproofs
%\hideproofstrue %uncomment to hide proofs
%
%
%
%
\ifhideproofs
\usepackage{environ}
\NewEnviron{hide}{}
\let\proof\hide
\let\endproof\endhide
\fi

\newtheorem{thm}{Theorem}[subsection]
\newtheorem{lem}[thm]{Lemma}
\newtheorem{prop}[thm]{Proposition}
\newtheorem{cor}[thm]{Corollary}
\newtheorem{conj}{Conjecture}

\theoremstyle{definition}
\newtheorem{definition}{Definition}[subsection]
\newtheorem{defn}[definition]{Definition}

\theoremstyle{remark}
\newtheorem{remark}{Note}[subsection]
\newtheorem{note}[remark]{Note}

\theoremstyle{definition}
\newtheorem{ex}[definition]{Exercise}

\newcommand{\al}{\alpha}
\newcommand{\gam}{\gamma}
\newcommand{\Gam}{\Gamma}
\newcommand{\bet}{\beta} 
\newcommand{\del}{\delta} 
\newcommand{\Del}{\Delta}
\newcommand{\lam}{\lambda}  
\newcommand{\Lam}{\Lambda} 
\newcommand{\ep}{\epsilon}
\newcommand{\sig}{\sigma} 
\newcommand{\om}{\omega}
\newcommand{\Om}{\Omega}
\newcommand{\C}{\mathbb{C}}
\newcommand{\N}{\mathbb{N}}
\newcommand{\E}{\mathbb{E}}
\renewcommand{\H}{\mathbb{H}}
\newcommand{\Z}{\mathbb{Z}}
\newcommand{\R}{\mathbb{R}}
\newcommand{\T}{\mathbb{T}}
\newcommand{\Q}{\mathbb{Q}}
\renewcommand{\P}{\mathbb{P}}
\newcommand{\MA}{\mathcal{A}}
\newcommand{\MC}{\mathcal{C}}
\newcommand{\MB}{\mathcal{B}}
\newcommand{\MF}{\mathcal{F}}
\newcommand{\MG}{\mathcal{G}}
\newcommand{\ML}{\mathcal{L}}
\newcommand{\MN}{\mathcal{N}}
\newcommand{\MS}{\mathcal{S}}
\newcommand{\MP}{\mathcal{P}}
\newcommand{\ME}{\mathcal{E}}
\newcommand{\MT}{\mathcal{T}}
\newcommand{\MI}{\mathcal{I}}
\newcommand{\MM}{\mathcal{M}}
\newcommand{\MX}{\mathcal{X}}


\newcommand{\p}{\partial}

\renewcommand{\r}{\rangle}
\renewcommand{\l}{\langle}

\newcommand{\RG}{[0,\infty]}
\newcommand{\Rg}{[0,\infty)}
\newcommand{\Ll}{L^1_{\text{loc}}(\R^n)}

\newcommand{\limfn}{\liminf \limits_{n \rightarrow \infty}}
\newcommand{\limpn}{\limsup \limits_{n \rightarrow \infty}}
\newcommand{\limn}{\lim \limits_{n \rightarrow \infty}}
\newcommand{\convt}[1]{\xrightarrow{\text{#1}}}
\newcommand{\conv}[1]{\xrightarrow{#1}} 
\newcommand{\seq}[2]{(#1_{#2})_{#2 \in \N}}


\DeclareMathOperator{\supp}{supp}
\DeclareMathOperator{\spn}{span}
\DeclareMathOperator{\sgn}{sgn}
\DeclareMathOperator{\iso}{Iso}
\DeclareMathOperator{\id}{id}

\begin{document}
	
	\title{Introduction to Differential Geometry}
	\author{Carson James}
	\maketitle
	
	\tableofcontents
	
	\section{Fundamental Definitions and Results}

	\subsection{Set Theory}
	
	\begin{defn}
		Let $\{A_i\}_{i \in I}$ be a collection of sets. The \textbf{disjoint union of} $\{A_i\}_{i \in I}$, denoted $\coprod\limits_{i \in I} A_i$, is defined by $$\coprod_{i \in I}A_i = \bigcup_{i\in I} \{i\} \times A_i$$ 
	\end{defn}

	\begin{note}
		In these notes, we will identify $\{i\} \times A_i$ and $A_i$.
	\end{note}

	\begin{defn}
		Let Let $\{A_i\}_{i \in I}$ be a collection of sets and $\sig: I \rightarrow \coprod\limits_{i \in I} A_i$. Then $\sig$ is said to be a \textbf{section of $\coprod\limits_{i \in I} A_i$} if for each $i \in I$, $\sig(i) \in A_i$.
	\end{defn}
	
	\subsection{Differentiation}
	
	\begin{defn}
		Let $n \geq 1$. For $i = 1, \cdots, n$, define $x_i: \R^n \rightarrow \R$ by $x_i(a_1, \cdots, a_n) = a_i$. The functions $(x_i)_{i=1}^n$ are called the \textbf{standard coordinate functions on $\R^n$}. 
	\end{defn}
	
	\begin{defn} 
		Let $U \subset \R^n$ be open, $f: U \rightarrow \R$ and $a \in U$. Then $f$ is said to be \textbf{differentiable with respect to $x_i$ at $a$} if $$\lim\limits_{h \rightarrow 0} \frac{f(a + he_i) - f(a)}{h}$$ exists. If $f$ is differentiable with respect to $x_i$ at $a$, we define the \textbf{partial derivative of $f$ with respect to $x_i$ at $a$}, denoted $$\frac{\partial f}{\partial x_i} (a) \text{ or } \eval{\pdv{x_i}}_{a} f $$ to be the limit above.
		
	\end{defn}
		
	\begin{defn}
		Let $U \subset \R^n$ be open and $f: U \rightarrow \R$. Then $f$ is said to be \textbf{differentiable with respect to $x_i$} if for each $a \in U$, $f$ is differentiable with respect to $x_i$ at $a$.
	\end{defn}

	\begin{ex}
		Let $U \subset \R^n$ be open, $f: U \rightarrow \R$ and $a \in U$. Suppose that $\frac{\partial ^2 f}{\partial x_i x_j}$ and $\frac{\partial ^2 f}{\partial x_j x_i}$ exist and are continuous at $a$. Then $$\frac{\partial ^2 f}{\partial x_i x_j} (a) = \frac{\partial ^2 f}{\partial x_j x_i} (a)$$
	\end{ex}

	\begin{proof}
		
	\end{proof}

	\begin{defn}
		Let $U \subset \R^n$ be open and $f: U \rightarrow \R$. Then $f$ is said to be \textbf{smooth} if for each $i_1, \cdots, i_k \in \{1, \cdots, n\}$, $\frac{\partial^k f}{\partial i_1 \cdots i_k}$ exists and is continuous on $U$.
	\end{defn}

	\begin{defn}
		Let $U \subset \R^n$, $f: U \rightarrow \R$. Then $f$ is said to be \textbf{smooth} if there exists $U' 
		\subset \R^n$ and $f':U' \rightarrow \R$ such that $U \subset U'$, $U'$ is open, $f'|_U = f$ and $f'$ is smooth. The set of smooth functions on $U$ is denoted $C^{\infty}(U)$.
	\end{defn}

	\begin{defn} 
		Let $U \subset \R^n$ and $p \in U$. Then $U$ is said to be \textbf{star-shaped} if for each $q \in U$, $\{p + t(q-p): 0 \leq t \leq 1\} \subset U$.
	\end{defn}

	\begin{thm} \textbf{(Taylor's Theorem)}
		Let $U \subset \R^n$ be open, $p \in U$ and $f \in C^{\infty}(U)$. Suppose that $U$ is star-shaped with respect to $p$. Then there exist $g_1, \cdots, g_n \in C^{\infty}(U)$ such that for each $x \in U$, $$f(x) = f(p) + \sum_{i=1}^n (x_i - p_i) g_i(x)$$ and for each $i \in \{1, \cdots, n\}$, $$g_i(p) = {\pdv{f}{x_i}}(p)$$
	\end{thm}

	\begin{proof}
		Let $x \in U$. Since $U$ is star-shaped with respect to $p$, $\{p + t(x-p): 0 \leq t \leq 1\} \subset U$. By the chain rule, 
		$${\dv{t}} \bigg[ f(p+t(x-p)) \bigg] = \sum_{i=1}^n {\pdv{f}{x_i}}(p+t(x-p)) (x_i - p_i)$$
		Integrating both sides with respect to $t$ from $0$ to $1$, we obtain
		$$f(x) - f(p) = \sum_{i=1}^n (x_i - p_i) \int_{0}^{1}  {\pdv{f}{x_i}}(p+t(x-p)) dt $$
		For $i \in \{1, \cdots, n\}$, define $g_i \in C^{\infty}(U)$ by $$g_i(x) = \int_{0}^{1}  {\pdv{f}{x_i}}(p+t(x-p)) dt$$
		Then for each $i \in \{1, \cdots, n\}$, $$g_i(p) = {\pdv{f}{x_i}}(p))$$
	\end{proof}

	
	\subsection{Smooth Maps}
	\begin{defn}
	Let $U \subset \R^n$ and $F: U \rightarrow \R^m$. Let $x_1, \cdots, x_n$ be the standard coordinate functions on $\R^n$ and $y_1, \cdots, y_m$ be the standard coordinate functions on $\R^m$. For $i \in \{1, \cdots, m\}$, we define the \textbf{$i$th component of $F$}, denoted $F_i: U \rightarrow \R$, by $$F_i = y_i \circ F$$ 
	Thus $F = (F_1, \cdots, F_m)$
	\end{defn}
	
	\begin{defn}
	Let $U \subset \R^n$ and $F: U \rightarrow \R^m$. Then $F$ is said to be \textbf{smooth} if for each $i \in \{1, \cdots, m\}$, the $i$th component of $F$, $F_i: U \rightarrow \R$, is smooth.
	\end{defn}

	\begin{defn}
		Let $U \subset \R^n$ and $V \subset \R^m$ and $F: U \rightarrow V$. Then $F$ is said to be a  \textbf{diffeomorphism} if $F$ is a homeomorphism and $F, F^{-1}$ are smooth. 
	\end{defn}
	
	\begin{defn}
	Let $U \subset \R^n$ be open, $p \in U$ and $F: U \rightarrow \R^m$. We define the \textbf{Jacobian of $F$ at $p$}, denoted $\frac{\p F}{\p x}(p) \in \R^{m \times n}$, by $$\bigg (\frac{\p F}{\p x}(p) \bigg )_{i,j} = \frac{\p F_i}{\p x_j}$$
	\end{defn}
	
	\begin{ex}\textbf{Inverse Function Theorem:}\\
	Let $U,V \subset \R^n$ be open and $F: U \rightarrow V$.
	\end{ex}
	
	\begin{ex}
		Let $U,V \subset \R^n$ and $F: U \rightarrow V$. Then $F$ is a diffeomorphism iff for each $p \in U$, there exists a relatively open neighborhood $N \subset U$ of $p$ such that $F|_N:N \rightarrow F(N)$ is a diffeomorphism
	\end{ex}
	
	\begin{proof}
		content...
	\end{proof}















\newpage
	\section{Multilinear Algebra}
	
	\begin{note}
		For the remainder of this section we let $V$ denote an $n$-dimensional vector space with basis $\{e_1, \cdots, e_n\}$ with dual space $V^*$ and dual basis $\{\ep_1, \cdots, \ep_n\}$ defined by $\ep_i(e_j) = \del_{i,j}$.
	\end{note}

	\subsection{$k$-Tensors}
	
	\begin{defn}
		Let $\al: V^k \rightarrow \R$. Then $\alpha$ is said to be \textbf{multilinear} or a \textbf{k-tensor on V} if for $i \in \{1, \cdots, k\}$, $w \in V$, $c \in \R$ and $v_1, \cdots, v_k \in V$, $$\al(v_1, \cdots, v_i + cw, \cdots, v_k) = \al(v_1, \cdots, v_i, \cdots, v_k) + c\al(v_1, \cdots, w, \cdots, v_k)$$
		The set of all $k$-tensors on $V$ is denoted by $T_k(V)$. Define $L_0(V) = \R$.
	\end{defn}

	\begin{ex}
		We have that $T_k(V)$ is a vector space. 
	\end{ex}

	\begin{proof}
		Clear.
	\end{proof}

	\begin{defn}
		For $\sig \in S_k$ and $\al \in T_k(V)$, define the $\sig \al : V^k \rightarrow \R$ by $$\sig \al(v_1, \cdots, v_k) = \al(v_{\sig(1)}, \cdots, v_{\sig(k)})$$  The map $\al \mapsto \sig \al$ is called the \textbf{permutation action} of $S_k$ on $T_k(V)$
	\end{defn}

	\begin{ex}
		The permutation action of $S_k$ on $T_k(V)$ is a group action.
	\end{ex}

	\begin{proof} \
		\begin{enumerate}
			\item Clearly for each $\sig \in S_k$ and $\al \in T_k(V), \sig \al \in T_k(V) $.
			\item Clearly for each $\al \in T_k(V)$, $e \al = \al$.
			\item Let $\tau, \sig \in S_k$ and $\al \in T_k(V)$. Then for each $v_1, \cdots, v_k \in V$, 
			\begin{align*}
				(\tau \sig) \al(v_1, \cdots, v_k) 
				&= \al(v_{\tau \sig (1)}, \cdots, v_{\tau \sig (k)}) \\
				&= \tau \al(v_{ \sig (1)}, \cdots, v_{ \sig (k)}) \\ 
				&= \tau (\sig \al) (v_1, \cdots, v_k) 
			\end{align*}
		\end{enumerate}
	\end{proof}

	\begin{ex}
		Let $\sig \in S_k$. Then $L_{\sig}: T_k(V) \rightarrow T_k(V)$ given by $ L_{\sig}(\al) = \sig \al$ is a linear transformation.
	\end{ex}

	\begin{proof}
		Let $\al, \beta \in T_k(V)$, $c \in \R$ and $v_1, \cdots, v_k \in V$. Then 
		\begin{align*}
			\sig(c\al + \beta)(v_1, \cdots, v_k) 
			&= (c\al + \beta)(v_{\sig(1)}, \cdots, v_{\sig(k)}) \\
			&= c \al(v_{\sig(1)}, \cdots, v_{\sig(k)}) + \beta(v_{\sig(1)}, \cdots, v_{\sig(k)}) \\
			&= c \sig \al(v_1, \cdots, v_k) + \sig \beta(v_1, \cdots, v_k)
		\end{align*}
		So $\sig(c \al + \beta) = c\sig \al + \sig \beta$.
	\end{proof}
	
	\begin{defn}
		Let $\al \in T_k(V)$. Then $\al$ is said to be \textbf{symmetric} if for each $\sig \in S_k$, $\sig \al = \al$. and $\al$ is said to be \textbf{alternating} if for each $\sig \in S_k$, $\sig \al = \sgn(\sig) \al$.  The set of symmetric $k$-tensors on $V$ is denoted $\Xi_k(V)$ and the set of alternating $k$-tensors on $V$ is denoted $\Lam_k(V)$.
	\end{defn}

	\begin{defn}
		Define the \textbf{symmetric operator} $S: T_k(V) \rightarrow \Xi_k(V)$ by $$S(\al) = \frac{1}{k!}\sum_{\sig \in S_k} \sig \al$$  Define the \textbf{alternating operator} $A: T_k(V) \rightarrow \Lam_k(V)$ by $$A(\al) = \frac{1}{k!}\sum_{\sig \in S_k} \sgn(\sig)\sig \al$$
	\end{defn}
	
	\begin{ex}\
		\begin{enumerate}
			\item For $\al \in T_k(V)$, $S(\al)$ is symmetric.
			\item For $\al \in T_k(V)$, $A(\al)$ is alternating.
		\end{enumerate}
	\end{ex}

	\begin{proof}\
		\begin{enumerate}
			\item Let $\al \in T_k(V)$ and $\sig \in S_k$. Then 
			\begin{align*}
				\sig S(\al) 
				&= \sig \bigg[ \frac{1}{k!}\sum_{\tau \in S_k} \tau \al \bigg]\\
				&= \frac{1}{k!} \sum_{\tau \in S_k} \sig \tau \al \\
				&= \frac{1}{k!} \sum_{\tau \in S_k} \tau \al \\
				&= S(\al)
			\end{align*} 
			\item Let $\al \in T_k(V)$ and $\sig \in S_k$. Then 
			\begin{align*}
				\sig A(\al) 
				&= \sig \bigg[ \frac{1}{k!}\sum_{\tau \in S_k} \sgn(\tau)\tau \al \bigg] \\
				&= \frac{1}{k!}\sum_{\tau \in S_k} \sgn(\tau) \sig \tau \al \\
				&= \frac{1}{k!} \sum_{\tau \in S_k} \sgn(\sig) \sgn(\sig \tau) \sig \tau \al \\
				&= \sgn(\sig) \frac{1}{k!} \sum_{\tau \in S_k} \sgn(\sig \tau) \sig \tau \al \\
				&= \sgn(\sig) \frac{1}{k!} \sum_{\tau \in S_k} \sgn(\tau)\tau \al \\
				&= \sgn(\sig)A(\al)
			\end{align*} 
		\end{enumerate}
	\end{proof}

	\begin{ex} \
		\begin{enumerate}
			\item For $\al \in \Xi_k(V)$, $S(\al) = \al$.
			\item For $\al \in \Lam_k(V)$, $A(\al) = \al$.
		\end{enumerate}
	\end{ex}

	\begin{proof}\
		\begin{enumerate}
			\item Let $\al \in \Xi_k(V)$. Then 
			\begin{align*}
				S(\al) 
				&= \frac{1}{k!} \sum_{\sig \in S_k} \sig \al \\
				&= \frac{1}{k!} \sum_{\sig \in S_k} \al \\
				&= \al
			\end{align*}
			\item Let $\al \in \Lam_k(V)$. Then 
			\begin{align*}
				A(\al) 
				&= \frac{1}{k!} \sum_{\sig \in S_k} \sgn(\sig)\sig \al \\
				&= \frac{1}{k!} \sum_{\sig \in S_k} \sgn(\sig)^2 \al \\
				&= \al
			\end{align*}
		\end{enumerate}
	\end{proof}

	\begin{ex}
		The symmetric operator $S: T_k(V) \rightarrow \Xi_k(V)$ and the alternating operator $A: T_k(V) \rightarrow \Lam_k(V)$ are linear.
	\end{ex}

	\begin{proof}
		Clear.
	\end{proof}
	
	\begin{defn}
		Let $\al \in T_k(V)$ and $\beta \in T_l(V)$. The \textbf{tensor product} of $\al$ and $\beta$ is defined to be the map $\al \otimes \beta \in T_{k+l}(V)$ given by $$\al \otimes \beta (v_1, \cdots, v_k, v_{k+1}, \cdots, v_{k+l}) = \al(v_1, \cdots, v_k) \beta(v_{k+1}, \cdots, v_{k+l})$$  
		Thus $\otimes: T_k(V) \times T_l(V) \rightarrow T_{k+l}(V)$.
	\end{defn}

	\begin{ex}
		The tensor product $\otimes: T_k(V) \times T_l(V) \rightarrow T_{k+l}(V)$ is associative.
	\end{ex}

	\begin{proof}
		Clear.
	\end{proof}
	
	\begin{ex}
		The tensor product $\otimes: T_k(V) \times T_l(V) \rightarrow T_{k+l}(V)$ is bilinear.
	\end{ex}

	\begin{proof}
		Clear.
	\end{proof}
	
	\begin{defn}
		Let $\al \in \Lam_k(V)$ and $\beta \in \Lam_l(V)$. The \textbf{exterior product} of $\al$ and $\beta$ is defined to be the map $\al \wedge \beta \in \Lam_{k+l}(V)$ given by $$\al \wedge \beta = \frac{(k+l)!}{k! l!} A(\al \otimes \beta)$$ 
		Thus $\wedge: \Lam_k(V) \times \Lam_l(V) \rightarrow \Lam_{k+l}(V)$.
	\end{defn}

	\begin{ex}
		The exterior product $\wedge: \Lam_k(V) \times \Lam_l(V) \rightarrow T_{k+l}(V)$ is bilinear.
	\end{ex}
	
	\begin{proof}
		Clear.
	\end{proof}

	\begin{ex}
		Let $\al \in T_k(V)$ and $\beta \in T_l(V)$. Then 
		\begin{enumerate}
			\item $A(A(\al) \otimes \beta) = A(\al \otimes \beta)$
			\item $A(\al \otimes A(\beta)) = A(\al \otimes \beta)$
		\end{enumerate}
	\end{ex}

	\begin{proof}
		First note that if we fix $\mu \in S_{k+1}$, then for each $\tau \in S_k$, choosing $\sig = \mu \tau^{-1}$ yields $\sig \tau = \mu$. For each $\mu \in S_{k+l}$, the map $\phi_{\mu}: S_{k} \rightarrow S_{k+l}$ given by $\phi_{\mu}(\tau) = \mu \tau^{-1}$ is injective. Thus for each $\mu \in S_{k+l}$, we have that $\# \{ (\sig, \tau) \in S_{k+l} \times S_{k}: \mu = \sig \tau \} = k!$ 
		\begin{enumerate}
			\item Then
			\begin{align*}
				A(A(\al) \otimes \beta)
				&= \frac{1}{(k+l)!} \sum_{\sig \in S_{k+l}} \sgn(\sig) \sig \bigg [A(\al) \otimes \beta \bigg] \\
				&= \frac{1}{(k+l)!} \sum_{\sig \in S_{k+l}} \sgn(\sig) \sig \bigg[ \bigg( \frac{1}{k!}\sum_{\tau \in S_{k} } \sgn(\tau) \tau\al \bigg) \otimes \beta \bigg] \\
				&= \frac{1}{(k+l)!} \sum_{\sig \in S_{k+l}} \sgn(\sig) \sig \bigg[   \frac{1}{k!} \sum_{\tau \in S_{k} } \sgn(\tau) (\tau\al)  \otimes \beta \bigg] \\
				&= \frac{1}{(k+l)!} \sum_{\sig \in S_{k+l}} \sgn(\sig) \sig \bigg[  \frac{1}{k!} \sum_{\tau \in S_{k} } \sgn(\tau) \tau (\al  \otimes \beta) \bigg] \\
				&=  \frac{1}{k! (k+l)!}\sum_{\sig \in S_{k+l} } \sum_{\tau \in S_{k} } \sgn(\sig \tau) \sig \tau (\al  \otimes \beta) \\
				&=  \frac{k!}{k!(k+l)!} \sum_{\mu \in S_{k+l} }  \sgn(\mu) \mu (\al  \otimes \beta)\\
				&=  \frac{1}{(k+l)!} \sum_{\mu \in S_{k+l} }  \sgn(\mu) \mu (\al  \otimes \beta)\\
				&= A(\al \otimes \beta)
			\end{align*} 
		\item Similar to (1).
		\end{enumerate}
	\end{proof}

	\begin{ex}
		The exterior product $\wedge: \Lam_k(V) \times \Lam_l(V) \rightarrow \Lam_{k+l}(V)$ is associative. 
	\end{ex}

	\begin{proof}
		Let $\al \in \Lam_k(V)$, $\beta \in \Lam_l(V)$ and $\gam \in \Lam_m(V)$. Then 
		\begin{align*}
			(\al \wedge \bet) \wedge \gam
			&= \bigg [ \frac{(k+l)!}{k! l!} A(\al \otimes \beta) \bigg] \wedge \gam \\ 
			&= \frac{(k+l+m)!}{(k+l)!m!} A \bigg( \bigg [ \frac{(k+l)!}{k! l!} A(\al \otimes \beta) \bigg] \otimes \gam \bigg)  \\ 
			&= \frac{(k+l+m)!}{(k+l)!m!}  \frac{(k+l)!}{k!l!}A(A(\al \otimes \bet) \otimes \gam) \\
			&= \frac{(k+l+m)!}{m!}  \frac{1}{k!l!} A((\al \otimes \beta) \otimes \gam) \\
			&= \frac{(k+l+m)!}{k!(l+m)!}  \frac{(l+m)!}{l!m!} A(\al \otimes (\bet \otimes \gam)) \\
			&= \frac{(k+l+m)!}{k!(l+m)!}  \frac{(l+m)!}{l!m!} A(\al \otimes A(\bet \otimes \gam)) \\
			&= \frac{(k+l+m)!}{k!(l+m)!} A(\al \otimes \frac{(l+m)!}{l!m!} A(\bet \otimes \gam)) \\
			&= \frac{(k+l+m)!}{k!(l+m)!} A(\al \otimes (\bet \wedge \gam)) \\
			&= \al \wedge (\bet \wedge \gam)) \\
		\end{align*}
	\end{proof}
	
	\begin{ex}
		Let $\al_i \in \Lam_{k_i}(V)$ for $i =1, \cdots, m$. Then $$\bigwedge_{i=1}^m \al_i = \frac{(\sum_{i=1}^m k_i)!}{\prod_{i=1}^m k_i!} A \bigg(\bigotimes_{i=1}^m \al_i \bigg)$$
	\end{ex}

	\begin{proof}
		To see that the statment is true in the case $m=3$, the proof of the previous exercise tells us that indeed $$\al_1 \wedge \al_2 \wedge \al_3 = \frac{(k_1 + k_2 + k_3)!}{k_1! k_2! k_3!}A(\al_1 \otimes \al_2 \otimes \al_3)$$
		Now, suppose that the statement is true for each $3 \leq m \leq m_0$. Then the proof of the previous exercise tells us the 
		\begin{align*}
			\bigwedge_{i=1}^{m_0 +1} \al_i
			&= \bigg( \bigwedge_{i=1}^{m_0 -1} \al_i \bigg) \wedge \al_{m_0} \wedge \al_{m_0+1} \\
			&= \frac{(\sum_{i=1}^{m_0-1} k_i + k_{m_0} + k_{m_0+1})! }{(\sum_{i=1}^{m_0-1} k_i)! k_{m_0}! k_{m_0+1}!} A\bigg( \bigg[ \bigwedge_{i=1}^{m_0-1} \al_i \bigg] \otimes \al_{m_0} \otimes \al_{m_0 +1}  \bigg) \\
			&= \frac{(\sum_{i=1}^{m_0-1} k_i + k_{m_0} + k_{m_0+1})! }{(\sum_{i=1}^{m_0-1} k_i)! k_{m_0}! k_{m_0+1}!} A\bigg( \bigg[ \frac{(\sum_{i=1}^{m_0-1} k_i)!}{\prod_{i=1}^{m_0-1} k_i!} A \bigg (\bigotimes_{i=1}^{m_0-1} \al_i \bigg) \bigg] \otimes \al_{m_0} \otimes \al_{m_0 +1}  \bigg) \\
			&= \frac{(\sum_{i=1}^{m_0+1} k_i)! }{ \prod_{i=1}^{m_0+1} k_i!}A \bigg( A\bigg [ \bigotimes_{i=1}^{m_0-1} \al_i \bigg] \otimes \al_{m_0} \otimes \al_{m_0 +1}  \bigg) \\
			&= \frac{(\sum_{i=1}^{m_0+1} k_i)! }{ \prod_{i=1}^{m_0+1} k_i!}A \bigg( \bigg [ \bigotimes_{i=1}^{m_0-1} \al_i \bigg] \otimes \al_{m_0} \otimes \al_{m_0 +1}  \bigg) \\
			&= \frac{(\sum_{i=1}^{m_0+1} k_i)! }{ \prod_{i=1}^{m_0+1} k_i!}A \bigg(  \bigotimes_{i=1}^{m_0+1} \al_i   \bigg) 
		\end{align*}
	\end{proof}
	
	\begin{ex}
		Define $\tau \in S_{k+l}$ by 
		\[\tau = 
		\begin{pmatrix}
			1 & 2 & \cdots &l &l+1 & l+2 & \cdots & l+k \\
			1+k & 2+ k & \cdots & l+k & 1 & 2 & \cdots & k 
		\end{pmatrix} 
		\]
		Then the inversion number of $\tau$ is $kl$.
		(Hint: inversion number)
	\end{ex}

	\begin{proof}
		\begin{align*}
			N(\tau) 
			&= \sum_{i = 1}^l k \\
			&= kl
		\end{align*}
		Since $\sgn (\tau) = (-1)^{N(\tau)}$ we know that  $\sgn(\tau) = (-1)^{kl}$.
	\end{proof}

	
	\begin{ex}
		Let $\al \in \Lam_k(V)$, $\beta \in \Lam_l(V)$. Then $$\al \wedge \bet = (-1)^{kl}\bet \wedge \al$$
	\end{ex}

	\begin{proof}
		Note that For $\sig \in S_{k+l}$ and $v_1, \cdots, v_{k +l} \in V$, we have that 
		\begin{align*}
			\sig \tau (\bet \otimes \al)(v_1, \cdots, v_l, v_{l+1}, \cdots v_{l+k}) 
			&= \bet \otimes \al(v_{\sig \tau(1)}, \cdots, v_{\sig \tau(l)}, v_{\sig \tau(l+1)}, \cdots v_{\sig \tau(l+k)}) \\
			&= \bet(v_{\sig \tau(1)}, \cdots, v_{\sig \tau(l)}) \al(v_{\sig \tau(l+1)}, \cdots v_{\sig \tau(l+k)}) \\
			&= \bet(v_{\sig (1+k)}, \cdots, v_{\sig (l+k)}) \al(v_{\sig (1)}, \cdots v_{\sig (k)})\\ 
			&= \al(v_{\sig (1)}, \cdots v_{\sig (k)}) \bet(v_{\sig (1+k)}, \cdots, v_{\sig (l+k)}) \\
			&= \al \otimes \bet (v_{\sig (1)}, \cdots v_{\sig (k)}, v_{\sig (1+k)}, \cdots, v_{\sig (l+k)}) \\
			&= \sig (\al \otimes \bet) (v_1, \cdots, v_k, v_{1+k}, \cdots v_{l+k})
		\end{align*}
		Thus $\sig \tau (\bet \otimes \al) = \sig (\al \otimes \bet)$.
		Define $\tau$ as in the previous exercise. Then 
		\begin{align*}
			\bet \wedge \al
			&= \frac{(k+l)!}{k!l!}A(\bet \otimes \al) \\
			&=  \frac{(k+l)!}{k!l!} \frac{1}{(k+l)!} \sum_{\sig \in S_{k+l}} \sgn(\sig) \sig (\bet \otimes \al) \\
			&= \frac{(k+l)!}{k!l!} \frac{1}{(k+l)!} \sum_{\sig \in S_{k+l}} \sgn(\sig \tau) \sig \tau (\bet \otimes \al) \\
			&= \sgn(\tau)\frac{(k+l)!}{k!l!} \frac{1}{(k+l)!} \sum_{\sig \in S_{k+l}} \sgn(\sig) \sig (\al \otimes \bet) \\
			&= \sgn(\tau)\frac{(k+l)!}{k!l!}  A(\al \otimes \bet) \\
			&= \sgn(\tau) \al \wedge \bet \\
			&= (-1)^{kl} \al \wedge \beta
		\end{align*}
	 
	\end{proof}

	\begin{ex}
		Let $\al_\in \Lam_k(V)$. If $k$ is odd, then $\al \wedge \al = 0$. 
	\end{ex}

	\begin{proof}
		Suppose that $k$ is odd. The previous exercise tells us that 
		\begin{align*}
			\al \wedge \al 
			&= (-1)^{k^2} \al \wedge \al \\
			&= -\al \wedge \al
		\end{align*}
		Thus $\al \wedge \al = 0$.
	\end{proof}
	
	\begin{ex}\textbf{(Fundamental Example)}
		Let $\al_1, \cdots, \al_m \in \Lam_1(V)$ and $v_1, \cdots, v_m \in V$. Then $$\bigg( \bigwedge_{i=1}^m \al_i \bigg)(v_1, \cdots, v_m) = \det (\al_i (v_j))$$
	\end{ex}

	\begin{proof}
		The previous exercises tell us that
		\begin{align*}
			\bigg( \bigwedge_{i=1}^m \al_i \bigg)(v_1, \cdots, v_m)
			&= m! A\bigg( \bigotimes_{i=1}^m \al_i \bigg) (v_1, \cdots, v_m) \\
			&= m! \bigg[ \frac{1}{m!} \sum_{\sig \in S_{m}} \sgn(\sig) \sig \bigg(\bigotimes_{i=1}^m \al_i \bigg) \bigg] (v_1, \cdots, v_m) \\
			&= \sum_{\sig \in S_{m}} \sgn(\sig)  \bigg(\bigotimes_{i=1}^m \al_i \bigg) (v_{\sig(1)}, \cdots, v_{\sig(m)}) \\
			&= \sum_{\sig \in S_{m}} \sgn(\sig)  \prod_{i=1}^m \al_i(v_{\sig(i)})   \\
			&= \det (\al_i (v_j))
		\end{align*}
	\end{proof}

	\begin{defn}
		Define $\MI_{k} = \{(i_1, i_2, \cdots, i_k) \in \N^k: i_1 < i_2 < \cdots < i_k \leq n \}$. Each element $I \in \MI_{k}$ is called a \textbf{multi-index}. Recall that $\# \MI_{k} = {n \choose k}$. 
	\end{defn}

	\begin{defn}
		Let $I = \{(i_1, i_2, \cdots, i_k) \in I_k$. \\ Define $e_I \in V^k$ by $$e_I = (e_{i_1}, \cdots, e_{i_k})$$ Define $\ep_I \in \Lam_k(V)$ by $$ \ep_I = \ep_{i_1} \wedge \cdots, \wedge \ep_{i_k} $$ 
	\end{defn}

	\begin{ex}
		Let $I = (i_1, \cdots, i_k)$ and $J = (j_1, \cdots, j_k) \in \MI_k$. Then $\ep_I(e_J) = \del_{I,J}$.
	\end{ex}

	\begin{proof}
		Put $A = \begin{pmatrix}
			\ep_{i_1}(e_{j_1}) & \cdots & \ep_{i_1}(e_{j_k}) \\
			& \vdots & \\
			\ep_{i_k}(e_{j_1}) & \cdots & \ep_{i_k}(e_{j_k}) 
		\end{pmatrix}$.
		A previous exercise tells us that $\ep_I(e_J) = \det A$.
		If $I = J$, then $A = I_{k\times k}$ and therefore $\ep_I(e_J) = 1$. Suppose that $I \neq J$. Put $l_0 = \min \{l: 1 \leq l \leq k, i_l \neq j_l\}$. If $i_{l_0} < j_{l_0}$, then all entries on the $l_0th$ row of $A$ are $0$. If $i_{l_0} > j_{l_0}$, then all entries on the $l_0th$ column of $A$ are $0$.
	\end{proof}

	\begin{ex}
		Let $\al , \bet \in \Lam_k(V)$. If for each $I \in \MI_k$, $\al(e_I) = \bet(e_I)$, then $\al = \bet$.
	\end{ex}

	\begin{proof}
		Suppose that for each $I \in \MI_k$, $\al(e_I) = \bet(e_I)$. Let $v_1, \cdots, v_k \in V$. For $i = 1, \cdots, k$, write $v_i = \sum_{j_i = 1}^n a_{i,j_i}e_{j_i}$. Then 
		\begin{align*}
			\al(v_1, \cdots, v_k) 
			&= \sum_{j_1, \cdots, j_k =1}^n \bigg( \prod_{i=1}^k a_{i, j_i} \bigg) \al(e_{j_1}, \cdots, e_{j_k}) \\
			&= \sum_{j_1 \neq \cdots \neq j_k}^n \bigg( \prod_{i=1}^k a_{i, j_i} \bigg) \al(e_{j_1}, \cdots, e_{j_k}) \\
			&= \sum_{J \in \MI_k} \bigg [ \sum_{\sig \in S_J} \sgn(\sig) \bigg( \prod_{i=1}^k a_{i, \sig(j_i)} \bigg) \bigg] \al(e_J) \\
			&= \sum_{J \in \MI_k} \bigg [ \sum_{\sig \in S_J} \sgn(\sig) \bigg( \prod_{i=1}^k a_{i, \sig(j_i)} \bigg) \bigg] \bet(e_J) \\
			&= \sum_{j_1, \cdots, j_k =1}^n \bigg( \prod_{i=1}^k a_{i, j_i} \bigg) \bet(e_{j_1}, \cdots, e_{j_k}) \\
			&= \bet(v_1, \cdots, v_k) 
		\end{align*}
	
	\end{proof}

	\begin{ex}
		The set $\{\ep_I: I \in \MI_k\}$ is a basis for $\Lam_k(V)$ and $\dim \Lam_k(V) = {n \choose k}$.
	\end{ex}

	\begin{proof}
		Let $(a_I)_{I \in \MI_k} \subset \R$. Let $\al = \sum_{I \in \MI_k}a_I \ep_I$. Suppose that $\al = 0$. Then for each $J \in \MI_k$, $\al(e_J) = a_J = 0$. Thus $\{e_I: I \in \MI_k\}$ is linearly independent. Let $\bet \in \Lam_k(V)$. For $I \in \MI_k$, put $b_I = \bet(e_I)$. define $\mu = \sum_{I \in \MI_k} b_I\ep_I$. Then for each $J \in \MI_k$, $\mu(e_J) = b_J = \bet(e_J)$. Hence $\mu = \bet$ and therefore $\bet \in \spn \{\ep_I:I \in \MI_k\}$.
	\end{proof}
	
	
	
	
	
	
	
	
	
	
	
	
	
	
	
	\newpage
	
	\subsection{$(r,s)$-Tensors}
	
	
	
	
	
	
	
	
	
	
	
	
	
	
	
	
	
	
	
	
	
	
	
	
	
	
	
	
	
	
	

	\newpage
	
	\section{Manifolds}
	
	\subsection{Smooth Manifolds}
	
	\begin{defn}
		Define the \textbf{upper half space} of $\R^n$, denoted $\H_n$, by $$\H_n = \{(x_1, x_2, \cdots, x_n) \in \R^n: x_n \geq 0\}$$ and define $$\partial\H_n = \{(x_1, x_2, \cdots, x_n) \in \R^n: x_n = 0\}$$ 
		$$(\H^n)^{\circ} = \{(x_1, x_2, \cdots, x_n) \in \R^n: x_n > 0\}$$
	\end{defn}
	
	\begin{defn}
		Let $M$ be a topological space and $n \geq 1$.
		\begin{enumerate}
			\item Let $U \subset M$, $V \subset \H^n$ open and $\phi:U \rightarrow V$. Then $(U, \phi)$ is said to be a \textbf{coordinate chart} on $M$ if $\phi$ is a homeomorphism. 
			\item Let $\MA = ((U_a,\phi_{a}))_{a \in A}$ be a collection of coordinate charts on $M$. Then $\MA$ is said to be an \textbf{atlas} on $M$ if  $\bigcup\limits_{a \in A} U_a = M$.
			\item The space $M$ is said to be \textbf{locally half Euclidean of dimension $n$} if there exists an atlas $\MA = ((U_a,\phi_{a}))_{a \in A}$ on $M$ such that for each $a \in A$, $\phi_a(U_a) \subset \H^n$.
			\item The space $M$ is said to be an \textbf{$n$-dimensional manifold} if $M$ is Hausdorff, second countable and locally half Euclidean of dimension $n$. 
		\end{enumerate}
	\end{defn}

	\begin{note}
		For the remainder of this section, we assume $M$ is an $n$-dimensional manifold.
	\end{note}

	\begin{defn}\
		\begin{enumerate}
			\item Define the 
			\textbf{boundary} of $M$, denoted $\partial M$, by $$\partial M = \{p \in M: \text{ there exists a chart } (U, \phi) \text{ on } M \text{ such that } p \in U \text{ and } \phi(p) \in \partial \H^n\}$$
			\item Define the 
			\textbf{interior} of $M$, denoted $M^{\circ}$, by $$M^{\circ} = M \setminus \partial M$$
		\end{enumerate}
	\end{defn}

	\begin{ex}
		Let $p \in M$. Then $p \in \partial M$ iff for each chart $(U, \phi)$ on $M$, $p \in U$ implies that $\phi(p) \in \partial \H^n$. (Hint: simply connected)
	\end{ex}

	\begin{proof}
		Supposet that $p \in \partial M$. Then there exists a coordinate chart $(V, \psi)$ on $M$ such that $\psi(p) \in \partial \H^n$. Let $(U, \phi)$ be a coordinate chart on $M$. Suppose that $p \in U$. Note that $\phi \circ \psi : \psi(V \cap U) \rightarrow \phi(V \cap U)$ is a homeomorphism. Choose open $n$-balls $B_{\phi}$, $B_{\psi} \subset \H^n$ such that $B_{\phi} \subset \phi(V \cap U)$, $B_{\psi} \subset \psi(V \cap U)$, $\phi(p) \in B_{\phi}$ and $\psi(p) \in B_{\psi}$. For the sake of contradiction, suppose that $\phi(p) \not \in \partial \H^n$. Put $ U' = B_{\phi} \setminus \{\phi(p)\}$ and $V' = B_{\psi} \setminus \{\psi(p)\}$. Define $\lam: V' \rightarrow U'$ by $\lam = \phi \circ \psi|_{B_{\psi}}$. Then $\lam$ is a homeomorphism. Note that $V'$ is simply connected and $U'$ is not. This is a contradiction. 
	\end{proof}

	\begin{ex}
		If $\partial M \neq \varnothing$, then 
		\begin{enumerate}
			\item $\partial M$ is an $n-1$-dimensional manifold \item $\partial (\partial M) = \varnothing$.
		\end{enumerate}
	\end{ex}

	\begin{proof}
		\begin{enumerate}
			\item Since subspaces of Hausdorff, second countable spaces are Hausdorff and second countable, we need only show that $\partial M$ is locally half euclidean of dimension $n-1$. Let $p \in \partial M$. Then there exists a coordinate chart $(U, \phi)$ on $M$ such that $p \in U$ and $\phi(p) \in \partial \H^n$. \\
			Put $U' = U \cap \partial M$. Note that $U'$ is open in $\partial M$ and $ \phi(U) \cap \partial \H^n$ is open in $\partial \H^n$. Define  $\phi': U' \rightarrow \phi(U) \cap \partial \H^n$ by $\phi' = \phi|_{U'}$. Then $\phi'$ is a homeomorphism. \\
			Since $\partial \H^n$ is homeomorphic to $\R^{n-1}$ which is homeomorphic to $(\H^{n-1})^{\circ}$ there exists $\psi: \partial \H^n \rightarrow (\H^{n-1})^{\circ}$ such that $\psi$ is a homeomorphism.\\ 
			Define $V' = \psi(\phi(U) \cap \partial \H^n)$ and $\psi': \phi(U) \cap \partial \H^n \rightarrow V' $ by and $\psi' = \psi|_{\phi(U) \cap \partial \H^n}$. Then $V'$ is open in $(\H^{n-1})^{\circ}$ and $\psi'$ is a homeomrophism. \\ 
			Define $\lam : U' \rightarrow V'$ by $\lam = \psi' \circ \phi'$. Then $\lam$ is a homeomorhism and $(U', \lam) $ is a cooridnate chart on $\partial M$. So $\partial M$ is locally Euclidean of dimension $n-1$.  
			\item Let $p \in \partial M$. Define $(U \cap \partial M, \lam \circ \psi)$ as in (1). Since $\lam \circ \psi(p) \in (\H^{n-1})^{\circ}$, we have that $p \in M^{\circ}$. Thus $ \partial M = (\partial M)^{\circ}$ and $\partial (\partial M) = \varnothing$.
		\end{enumerate}
	\end{proof}

	\begin{defn}\
		\begin{enumerate}
			\item Let $(U, \phi), (V, \psi)$ be coordinate charts on $M$. Then $(U, \phi)$ and $(V, \psi)$ are said to be \textbf{smoothly compatible} if $$\phi \circ \psi^{-1}: \psi(U \cap V) \rightarrow \phi (U \cap V) \text{ is a diffeomorphism}$$ 
			\item Let $\MA = ((U_a,\phi_{a}))_{a \in A}$ be an atlas on $M$. Then $\MA$ is said to be \textbf{smooth} if for each $a,b \in A$, $(U_a,\phi_{a})$ and $(U_b,\phi_{b})$ are smoothly compatible.
			\item Let $\MA$ be a smooth atlas on $M$. Then $\MA$ is said to be \textbf{maximal} if for each smooth atlas $\MB$ on $M$, $\MA \subset \MB$ implies that $\MA = \MB$. A maximal smooth atlas on $M$ is called a \textbf{smooth structure on $M$}.
			\item Let $\MA$ be a smooth structure on $M$. Then $(M, \MA)$ is said to be a \textbf{smooth $n$-dimensional manifold}. 
		\end{enumerate}
	\end{defn}

	\begin{ex}
		Let $\MB$ be a smooth atlas on $M$. Then there exists a unique smooth structure $\MA$ on $M$ such that $\MB \subset \MA$.
	\end{ex}

	\begin{proof}
		Define $\MA$ to be the set of all coordinate charts $(U, \phi)$ on $M$ such that for each coordinate chart $(V, \psi) \in \MB$,  $(U, \phi)$ and $(V, \psi) $ are smoothly compatible. \\
		Clearly $\MB \subset \MA$. \\
		Let $(U, \phi), (V, \psi) \in \MA$ and $p \in U \cap V$. Then there exists $(W, \chi) \in \MB$ such that $p \in W$. By assumption, $\phi \circ \chi^{-1} : \chi(U \cap W) \rightarrow \phi(U \cap W)$ and $ \chi \circ \psi^{-1} : \psi(W \cap V) \rightarrow \chi(W \cap V)$ are diffeomorphisms. Then $ (\phi \circ \chi^{-1}) \circ (\chi \circ \psi^{-1}) = \phi \circ \psi^{-1}: \psi(U \cap W \cap V) \rightarrow  \phi(U \cap W \cap V) $ is a diffeomorphism.  Since for each $q \in \psi(U \cap V)$, there exits an open neighborhood $N \subset \psi(U \cap V)$ of $q$ on which $\phi \circ \psi^{-1}$ are diffeomorphic, we have that $\phi \circ \psi^{-1}$ is a diffeomorphism on $\psi(U \cap V)$ and therefore $(U, \phi)$ and $ (V, \psi)$ are smoothly compatible. Hence $\MA$ is a smooth atlas.\\
		To see that $\MA$ is maximal, let $\MB'$ be a smooth atlas on $M$. Suppose that $\MA \subset \MB'$ and let $(U, \phi) \in \MB'$. By definition, for each chart $(V, \psi) \in \MB'$, $(U, \phi)$ and $(V, \psi)$ are smoothly compatible. Since $\MB \subset \MA \subset \MB'$, we have that $(U, \phi) \in \MA$. So $\MA = \MB'$ and $\MA$ is a maximal smooth atlas on $M$.
	\end{proof}

	\begin{ex}
		Let $\MA$ be a smooth atlas on $M$. Define $\lam: \partial \H^n \rightarrow \R^{n-1}$ by $\lam(x_1, \cdots, x_{n-1}, 0) = (x_1, \cdots, x_{n-1})$. Put $\MA|_{\partial M} = \{(U \cap \partial M, \lam \circ \phi_{U \cap \partial M}): (U, \phi) \in \MA\}$. Then 
		\begin{enumerate}
			\item $\MA|_{\partial M}$ is a smooth atlas on $\partial M$.
			\item if $ \MA $ is maximal, then $\MA|_{\partial M}$ is maximal.
		\end{enumerate}
	\end{ex}

	\begin{proof}
		
	\end{proof}
	
	\begin{note}
		For the rest of this section, we assume that $(M, \MA)$ is a smooth $n$-dimensional manifold and we denote the standard coordinate functions on $\R^n$ by $u_1, \cdots, u_n$. For a coordinate chart $(U, \phi)$ $\in \MA$ and $i \in \{1, \cdots, n\}$, we will typically denote the $i$th coordinate of $\phi$ by $x_i$, that is,  $x_i = u_i(\phi)$.
	\end{note}
	
	\newpage 
	
	
	\subsection{Smooth Maps}	
	
	\begin{defn}
		Let $f: M \rightarrow \R$. Then $f$ is said to be smooth if for each coordinate chart $(U, \phi) \in \MA$, $f \circ \phi^{-1}$ is smooth. The set of all smooth functions on $M$ is denoted $C^{\infty}(M)$. 
	\end{defn}

	\begin{ex}
		We have that $C^{\infty}(M)$ is a vector space.
	\end{ex}

	\begin{proof}
		Clear.
	\end{proof}

	\begin{defn}
		Let $(N, \MB)$ be a smooth manifold and $F: M \rightarrow N$. Then $F$ is said to be \textbf{smooth} if for each $(U, \phi) \in \MA$ and $(V, \psi) \in \MB$, $$\psi \circ F \circ \phi^{-1}: \phi(U \cap F^{-1}(V)) \rightarrow \psi(F(U) \cap V)$$ is smooth and $F$ is said to be a \textbf{diffeomorphism} if $F$ is a homeomorphism and $F,F^{-1}$ are smooth.
	\end{defn}
	
	\begin{ex}
		Let $(N, \MB)$ be a smooth manifold and $F: M \rightarrow N$ a diffeomorphism. Then for each $(U, \phi) \in \MA$, $(F(U), \phi \circ F^{-1}) \in \MB$.
	\end{ex}
	
	\begin{proof}
		Let $(V, \psi) \in \MB$. 
		\begin{enumerate}
		\item Since $\phi$ and $F^{-1}$ are homeomorphisms, $\phi \circ F^{-1}: F(U) \cap V \rightarrow \phi(U \cap F^{-1}(V))$ is a homeomorphism
		\item Since $F$ is a diffeomorphism, $$\phi \circ F^{-1} \circ \psi^{-1}: \psi(F(U) \cap V) \rightarrow \phi(U \cap F^{-1}(V))$$ and $$\psi \circ F \circ \phi^{-1}: \phi(F^{-1}(V) \cap U) \rightarrow \psi(V \cap F(U))$$ are smooth. 
		\end{enumerate}
		
		Therefore $(F(U), \phi \circ F^{-1})$ and $(V, \psi)$ are smoothly compatible. Since $\MB$ is maximal, $(F(U), \phi \circ F^{-1}) \in \MB$.
	\end{proof}

	\begin{ex}
	Let $(M, \MA)$ be smooth $m$-dimensional manifold, $(N, \MB)$ a smooth $n$-dimensional manifold and $F: M \rightarrow N$. If $F$ is a diffeomorphism, then $m = n$.
	\end{ex}

	\begin{proof}
	Suppose that $F$ is a diffeomorphism. Let $(U, \phi) \in \MA$. The previous exercise implies that $(F(U), \phi \circ F^{-1}) \in \MB$.  	
	\end{proof}








	\newpage
	\subsection{The Tangent Space}

	\begin{defn}
		Let $p \in M$. Define the relation $\sim_p$ on $C^{\infty}(M)$ by $f \sim_p g$ iff there exists an open $U \subset M$ such that $f|_U = g|_U$. Clearly $\sim_p$ is an equivalence relation on $C^{\infty}(M)$. We denote $C^{\infty}(M) / \sim_p$ by $C^{\infty}_p(M)$. For $f \in C^{\infty}(M)$, we define the \textbf{germ of $f$ at $p$} to be the equivalence class of $f$ under $\sim_p$. 
	\end{defn}
	
	\begin{ex}
		Let $p \in $We have that $C_p^{\infty}(M)$ is a vector space.
	\end{ex}
	
	\begin{proof}
		Clear.
	\end{proof}

	\begin{defn}
		Let $(U, \phi) \in \MA$ with $\phi = (x_1, \cdots, x_n)$, $p \in U$ and $f \in C_p^{\infty}(M)$. For $i \in \{1, \cdots, n\}$, define the partial derivative of $f$ with respect to $x_i$ at $p$, denoted $$\frac{\partial f}{\partial x_i} (p) \text{, } \eval{\pdv{x_i}}_{p} f \text{, } \partial_{x_i} f (p) \text{ or } \eval{\partial_{x_i}}_{p} f$$ by $$ \eval{\pdv{x_i}}_{p} f =  \eval{\pdv{u_i}}_{\phi(p)} f \circ \phi^{-1} $$
	\end{defn}

	\begin{ex}
		Let $(U, \phi) \in \MA$ with $\phi = (x_1, \cdots, x_n)$ and $p \in U$. Then for each $i,j \in \{1, \cdots, n\}$, we have that $${\pdv{x_i}{x_j}}(p) = \del_{i,j}$$
	\end{ex}

	\begin{proof}
		Let $i,j \in \{1, \cdots, n\}$. Then 
		\begin{align*}
			\eval{\pdv{x_j}}_{p} x_i 
			&=  \eval{\pdv{u_j}}_{\phi(p)} x_i \circ \phi^{-1} \\
			&= \eval{\pdv{u_j}}_{\phi(p)} u_i \circ \phi \circ \phi^{-1} \\
			&= \eval{\pdv{u_j}}_{\phi(p)} u_i  \\
			&= \del_{i,j}
		\end{align*}
	\end{proof}

	\begin{ex} \textbf{(Change of Coordinates):}
		Let $(U, \phi), (V, \psi) \in \MA$ with $\phi = (x_1, \cdots, x_n)$ and $\psi = (y_1, \cdots, y_n)$, $p \in U \cap V$ and $f \in C_p^{\infty}(M)$. Then for each $i \in \{1, \cdots, n\}$, we have $${\pdv{f}{y_i}}(p) = \sum_{j =1}^n {\pdv{f}{x_j}}(p)   {\pdv{x_j}{y_i}}(p) $$
	\end{ex}

	\begin{proof}
		Put $h = \phi \circ \psi^{-1}$ and write $h = (h_1, \cdots, h_n)$. Then $\phi = h \circ \psi$ and $\psi^{-1} = \phi^{-1} \circ h$. By definition and the chain rule, we have that 
		\begin{align*}
		\eval{\pdv{y_i}_{p}} f 
			&= \eval{\pdv{u_i}}_{\psi(p)} f \circ \psi^{-1} \\
			&= \eval{\pdv{u_i}}_{\psi(p)} f \circ \phi^{-1} \circ h \\
			&= \sum_{j=1}^n \bigg( \eval{\pdv{u_j}}_{h \circ \psi (p)} f \circ \phi^{-1} \bigg)  \bigg( \eval{\pdv{u_i}}_{\psi(p)} h_j \bigg) \\
			&= \sum_{j=1}^n \bigg( \eval{\pdv{u_j}}_{\phi (p)} f \circ \phi^{-1}  \bigg) \bigg( \eval{\pdv{u_i}}_{\psi(p)} x_j \circ \psi^{-1} \bigg) \\
			&= \sum_{j=1}^n \bigg( \eval{\pdv{x_j}}_{p} f \bigg)  \bigg(   \eval{\pdv{y_i}}_{p} x_j  \bigg)\\
		\end{align*}
	\end{proof}

	\begin{ex}\textbf{Taylor's Theorem:}\\
		Let $(U, \phi) \in \MA$ with $\phi = (x_1, \cdots, x_n)$, $p \in U$ and $f \in C^{\infty}_p(M)$. Then there exist $g_1, \cdots g_n \in C_p^{\infty}(M)$ such that $$f = f(p) + \sum_{i=1}^n(x_i - x_i(p)) g_i$$ and for each $i \in \{1, \cdots, n\}$, $$g_i(p) = \eval{\pdv{x_i}}_{p} f $$ 
	\end{ex}

	\begin{proof}
		Since we are interested in the germ of $f$ at $p$, we may assume that $\phi(U)$ is star-shaped with respect to $\phi(p)$. Let $q \in U$. From Taylor's theorem in section $1$, we know that there exist $\tilde{g_1}, \cdots, \tilde{g_n} \in C^{\infty}(\phi(U))$ such that for each $q \in U$, $$f \circ \phi^{-1} (\phi(q)) = f \circ \phi^{-1} (\phi(p)) + \sum_{i = 1}^n [u_i \circ \phi(q) - u_i \circ \phi(p)] \tilde{g_i} (\phi(q))$$ and for each $i \in \{1, \cdots, n\}$, $$\tilde{g_i}(\phi(p)) = \eval{\pdv{u_i}}_{\phi(p)} f \circ \phi^{-1}$$ 
		For each $i \in \{1, \cdots, n\}$, define $g_i = \tilde{g_i} \circ \phi$. Then for each $q \in U$, $$ f (q) = f (p) + \sum_{i = 1}^n [x_i(q) - x_i(p)] g_i(q)$$ and for each $i \in \{1, \cdots, n\}$, $$g_i(p) = \eval{\pdv{x_i}}_{p} f $$ 
	\end{proof}

	\begin{defn}
		Let $p \in M$ and $v: C^{\infty}_p(M) \rightarrow \R$. Then $v$ is said to be \textbf{Leibnizian} if for each $f,g \in  C^{\infty}_p(M)$, $$v(fg) = v(f)g(p) + f(p)v(g)$$ and $v$ is said to be a \textbf{derivation at $p$} if for each $f, g \in C^{\infty}_p(M)$ and $a \in \R$,
		\begin{enumerate}
			\item $v$ is linear 
			\item $v$ is Leibnizian
		\end{enumerate}
		We define the \textbf{tangent space of $M$ at $p$}, denoted $T_pM$, by $$T_pM = \{ v: C^{\infty}_p(M) \rightarrow \R: v \text{ is a derivation at }p\}$$
	\end{defn}

	\begin{ex}
		Let $f \in C^{\infty}_p(M)$ and $v \in T_pM$. If $f$ is constant, then $vf = 0$.
	\end{ex}

	\begin{proof}
		Suppose that $f \equiv 1$. Then $f^2 = f$ and $v(f^2) = 2v(f)$. So $v(f) = 2v(f)$ which implies that $v(f) = 0$. If $f \not \equiv 1$, then there exists $c \in \R$ such that $f \equiv c$. Since $v$ is linear, $v(f) = cv(1) = 0$.
	\end{proof}

	\begin{ex}
		Let $(U, \phi) \in \MA$ with $\phi = (x_1, \cdots, x_n)$ and $p \in U$. Then $$ \bigg \{\eval{\pdv{x_1}}_{p}, \cdots, \eval{\pdv{x_n}}_{p} \bigg \}$$ is a basis for $T_pM$ and $\dim T_pM = n$.
	\end{ex}

	\begin{proof}
		Clearly $\eval{\pdv{x_1}}_{p}, \cdots, \eval{\pdv{x_n}}_{p} \in T_pM$. Let $a_1, \cdots, a_n \in \R$. Suppose that $$v = \sum_{i=1}^n a_i \eval{\pdv{x_i}}_{p} = 0$$
		Then 
		\begin{align*}
			0
			&= v x_j \\
			&= \sum_{i=1}^n a_i \eval{\pdv{x_i}}_{p} x_j \\
			&= a_j
		\end{align*}
		Hence $\bigg \{\eval{\pdv{x_1}}_{p}, \cdots, \eval{\pdv{x_n}}_{p} \bigg \}$ is independent.\\
		Now, let $v \in T_pM$ and $f \in \C^{\infty}_p(M)$. By Taylor's theorem, there exist $g_1, \cdots g_n \in C_p^{\infty}(M)$ such that $$f = f(p) + \sum_{i=1}^n(x_i - x_i(p)) g_i$$ and for each $i \in \{1, \cdots, n\}$, $$g_i(p) = \eval{\pdv{x_i}}_{p} f $$ Then 
		\begin{align*}
			v(f)
			&= \sum_{i=1}^nv(x_i - x_i(p)) g_i(p) + \sum_{i=1}^n(x_i(p) - x_i(p)) v(g_i) \\
			&= \sum_{i=1}^nv(x_i)g_i(p) \\
			&= \sum_{i=1}^nv(x_i)\eval{\pdv{x_i}}_{p} f \\
			&= \bigg[ \sum_{i=1}^nv(x_i)\eval{\pdv{x_i}}_{p} \bigg] f
		\end{align*}
		So $$v = \sum_{i=1}^nv(x_i)\eval{\pdv{x_i}}_{p} $$ and $$v \in \spn \bigg \{\eval{\pdv{x_1}}_{p}, \cdots, \eval{\pdv{x_n}}_{p} \bigg \}$$
	\end{proof}



	\begin{defn}
		Let $(N, \MB)$ be a smooth manifold, $F: M \rightarrow N$ smooth and $p \in M$. We define the \textbf{differential of $F$ at $p$}, denoted $dF_p: T_pM \rightarrow T_{F(p)}N$, by $$\bigg[ dF_p(v) \bigg] (f) = v (f \circ F)$$  for $v \in T_pM$ and $f \in C^{\infty}_{F(p)}(N)$.
	\end{defn}
	
	
	
	\begin{ex}
	Let $(N, \MB)$ be a smooth manifold, $F: M \rightarrow N$ smooth and $p \in M$. Then $dF_p$ is well defined.
	\end{ex}
	
	\begin{proof}
	Let $v \in T_pM$, $f,g \in C_{F(p)}^{\infty}(N)$ and $c \in \R$. Then 
	\begin{enumerate}
	\item 
	\begin{align*}
		dF_p(v)(f+cg) 
		&= v((f+cg) \circ F) \\
		&= v(f \circ F + c g \circ F) \\
		&= v(f \circ F) + cv(g \circ F) \\
		&= dF_p(v)(f) + c dF_p(v)(g)
	\end{align*}
	So $dF_p(v)$ is linear.
	\item 
	\begin{align*}
	dF_p(v)(fg) 
	&= v (fg \circ F) \\
	&= v((f \circ F)* (g \circ F)) \\
	&= v(f \circ F)*(g \circ F)(p) +  (f \circ F)(p)* v(g \circ F) \\
	&= dF_p(v)(f) * g(F(p)) + f(F(p))*dF_p(v)(g) \\
	\end{align*}
	\end{enumerate}
	So $dF_p(v)$ is Leibnizian and hence $dF_p(v) \in T_{F(p)}N$
	\end{proof}

	\begin{ex}
		Let $(N, \MB)$ be a smooth manifold, $F: M \rightarrow N$ a diffeomorphism and $p \in M$. Then $dF_p$ is an isomorphism.
	\end{ex}
	
	\begin{proof}
		Since $F$ is a homeomorphism, $\dim N = n$. Choose $(U, \phi) \in \MA$ such that $p \in U$. A previous exercise tells us that $(F(U), \phi \circ F^{-1}) \in \MB$. Write $\phi = (x_1, \cdots, x_n)$ and $\phi \circ F^{-1} = (y_1, \cdots, y_n)$. Let $f \in C^{\infty}_{F(p)}(N)$ Then 
		\begin{align*}
			\eval{\pdv{y_i}}_{F(p)} f
			&= 	\eval{\pdv{u_i}}_{\phi \circ F^{-1} (F(p))} f \circ (\phi \circ F^{-1})^{-1} \\
			&= 	\eval{\pdv{u_i}}_{\phi(p)} f \circ F \circ \phi^{-1} \\
			&= 	\eval{\pdv{x_i}}_p f \circ F \\
		\end{align*}
		Therefore 
		\begin{align*}
			\bigg[ dF_p \bigg( \eval{\pdv{x_i}}_p \bigg) \bigg] (f)
			&= \eval{\pdv{x_i}}_p f \circ F \\
			&= \eval{\pdv{y_i}}_{F(p)} f 
		\end{align*}
	Hence $$dF_p \bigg( \eval{\pdv{x_i}}_p \bigg) = \eval{\pdv{y_i}}_{F(p)}$$ 
	Since $\bigg \{\eval{\pdv{x_1}}_{p}, \cdots, \eval{\pdv{x_n}}_{p} \bigg \}$ is a basis for $T_pM$ and $\bigg \{\eval{\pdv{y_1}}_{F(p)}, \cdots, \eval{\pdv{y_n}}_{F(p)} \bigg \}$ is a basis for $T_{F(p)}N$, $dF_p$ is an isomorphism.
	\end{proof}

	\begin{defn}
		Let $(N, \MB)$ be a smooth manifold, $F: M \rightarrow N$ a diffeomorphism. Define the \textbf{push forward of $F$}, denoted $$F_*:M \rightarrow \coprod_{p \in M} \iso(T_pM, T_{F(p)}N)$$ by $$p \mapsto dF_p$$
	\end{defn}
	
		\begin{defn}
		We define the \textbf{tangent bundle of $M$}, denoted $TM$, by $$TM = \coprod_{p \in M} T_pM$$ 
		and we define the \textbf{cotangent bundle of $M$}, denoted $T^*M$, by 
		$$T^*M = \coprod_{p \in M} T_p^*M$$
	\end{defn}
	
	\begin{defn}
		Let $X: M \rightarrow TM$. Then $X$ is said to be a \textbf{vector field on $M$} if for each $p \in M$, $X_p \in T_p M$. \\
		For $f \in \C^{\infty}(M)$ we define $Xf : M \rightarrow \R$ by $$(Xf)_p = X_p(f)$$ 
		Finally, $X$ is said to be \textbf{smooth} if for each $f \in \C^{\infty}(M)$, $Xf$ is smooth.\\
		We denote the set of smooth vector fields on $M$ by $\Gam(M)$.
	\end{defn}

	\begin{ex}
		Let $X \in \Gam(M)$ and $(U, \phi) \in \MA$ with $\phi = (x_1, \cdots, x_n)$. Then there exist $f_1, \cdots, f_n \in C^{\infty}(U)$ such that for each $p \in U$, $$X_p = \sum_{i=1}^n f_i(p) \eval{\pdv{x_i}}_{p}$$ 
	\end{ex}

	\begin{proof}
		Let $p \in M$. Then $X_p \in T_pM$ and $\bigg \{ \eval{\pdv{x_{1}}}_{p}, \cdots, \eval{\pdv{x_{n}}}_{p} \bigg \}$ is a basis of $T_pM$. So there exist $f_1(p), \cdots, f_n(p) \in \R$ such that $X_p = \sum\limits_{i=1}^n f_i(p) \eval{\pdv{x_i}}_{p}$. Let $j \in \{1, \cdots, n\}$. Since $X$ is smooth, the map 
		\begin{align*}
			p 
			& \mapsto X_p(x_j) \\
			&= \sum\limits_{i=1}^n f_i(p) {\pdv{x_j}{x_i}}(p) \\
			&= f_j(p) \\
		\end{align*} is smooth.
	\end{proof}
	
	\newpage
	
	\subsection{Submanifolds}
	
	

	
	



	
	
	

	
	\newpage	
	\subsection{Integration on Manifolds}
	
	\begin{defn}
		We define $$\Lam_k (TM) = \coprod_{p \in M} \Lam_k(T_p M)$$
	\end{defn}
	
	\begin{defn}
		Let $\om: M \rightarrow \Lam_k (TM)$. Then $\om$ is said to be a \textbf{$k$-form on $M$} if for each $p \in M$, $\om_p \in \Lam_k(T_pM)$.\\
		For each $X_1, \cdots, X_k \in \Gam(M)$, we define $\om(X_1, \cdots, X_k) : M \rightarrow \R$ by $$\om(X_1, \cdots, X_k)_p = \om_p({X_1}_p, \cdots, {X_k}_p)$$
		Finally, $\om$ is said to be \textbf{smooth} if for each $X_1, \cdots, X_k \in \Gam(M)$, $\om(X_1, \cdots, X_k)$ is smooth.\\
		The set of smooth $k$-forms on $M$ is denoted $\Om_k(M)$.\\
	\end{defn} 

	\begin{note}
		Observe that $\Om_0(M) = C^{\infty}(M)$.
	\end{note}

	\begin{defn}
		Define the \textbf{exterior product} $$\wedge: \Om_k(M) \times \Om_l(M) \rightarrow \Om_{k+l}(M) $$ by $$(\al \wedge \beta)_p = (\al)_p \wedge (\beta)_p$$
		Define the \textbf{permutation action of $S_k$ on $\Om_k(M)$} by $$(\sig \om)_p = \sig \om_p$$
	\end{defn}
	
	\begin{note}
		All of the results from multilinear algebra apply here.
	\end{note}

	\begin{note}
		For $f \in \Om_0(M)$ and $\al \in \Om_k(M)$, we have that $f \wedge \al = f \al$.
	\end{note}

	\begin{defn}
		We define the \textbf{exterior derivative} $d: \Om_k(M) \rightarrow \Om_{k+1}(M)$ inductively by 
		\begin{enumerate}
			\item $df(X) = Xf$ for $f \in \Om_0(M)$
			\item $d(\al \wedge \bet) = d\al \wedge \bet + (-1)^p \al \wedge d\bet$ for $al \in \Om_p(M)$ and $\bet \in \Om_q(M)$
			\item extending linearly
		\end{enumerate}
	\end{defn}

	\begin{ex}
		Let $(U, \phi)$ be a chart on $M$ with $\phi = (x_1, \cdots, x_n)$. Then on $U$, for each $i,j \in \{1, \cdots, n\}$, $$dx_i\bigg(\pdv{x_j} \bigg) \equiv \del_{i,j}$$
	\end{ex}

	\begin{proof}
		Let $p \in U$ and $i,j \in \{1, \cdots, n\}$. Then 
		\begin{align*}
			(dx_i)_p \bigg (\eval{\pdv{x_j}}_{p} \bigg ) 
			&= \eval{\pdv{x_j}}_{p} x_i \\
			&= \del_{i,j} \\
		\end{align*}
	\end{proof}

	\begin{note}
		The previous exercise tells us that for each $p \in U$, $\{(dx_1)_p, \cdots, (dx_n)_p \}$ is the dual basis to $\bigg \{ \eval{\pdv{x_1}}_{p}, \cdots, \eval{\pdv{x_n}}_{p} \bigg \}$.
	\end{note}

	\begin{ex}
		Let $f \in C^{\infty}(M)$ and $(U, \phi)$ be a chart on $M$ with $\phi = (x_1, \cdots, x_n)$. Then on $U$, $df = \sum_{i=1}^n \pdv{f}{x_i} dx_i$
	\end{ex}

	\begin{proof}
		Let $p \in U$. Since $\{dx_1, \cdots, dx_n\}$ is a basis for $\Lam(T_pM)$, for each there exist $a_1(p), \cdots, a_n(p) \in \R$ such that $(df)_p = \sum\limits_{i=1}^n a_i(p)(dx_i)_p$. Therefore, we have that 
		\begin{align*}
			(df)_p \bigg(\eval{\pdv{x_j}}_{p} \bigg) 
			&= \sum\limits_{i=1}^n a_i(p)(dx_i)_p \bigg(\eval{\pdv{x_j}}_{p} \bigg)  \\
			&=  a_j(p)
		\end{align*}
		By definition, we have that 
		\begin{align*}
			(df)_p\bigg(\eval{\pdv{x_j}}_{p} \bigg) 
			&= \eval{\pdv{x_j}}_{p} f \\ 
			&= {\pdv{f}{x_j}}(p)\\
		\end{align*}
		So $$(df)_p = \sum\limits_{i=1}^n {\pdv{f}{x_j}}(p)(dx_i)_p$$ and therefore on $U$, we have that $$df = \sum\limits_{i=1}^n {\pdv{f}{x_i}}dx_i$$
	\end{proof}

	\begin{defn}
		Let $(U, \phi)$ be a chart on $M$ with $\phi = (x_1, \cdots, x_n)$ and $I = (i_1, \cdots, i_k) \in \MI_k$. We define $$dx_I = dx_{i_1} \wedge \cdots \wedge dx_{i_k} \in \Om_k(M)$$ 
		and we define $$\pdv{x_I}= \bigg(\pdv{x_{i_1}}, \cdots, \pdv{x_{i_k}} \bigg)$$

	\end{defn}

	\begin{ex}
		Let $\om \in \Om_k(M)$ and $(U, \phi)$ be a chart on $M$ with $\phi = (x_1, \cdots, x_n)$. Then there exists $(f_I)_{I \in \MI_k} \subset C^{\infty}(U)$ such that for each $p \in U$, $$\om_p = \sum_{I \in \MI_k}f_I(p) (dx_I)_p$$
	\end{ex}

	\begin{proof}
		Let $p \in U$. For each $I \in \MI_k$, put $$f_I(p) = \om_p \bigg (  \eval{\pdv{x_I}}_{p} \bigg) \in \R$$ 
		Since $\{(dx_I)_p: I \in \MI_k\}$ is a basis for $\Lam_k(T_pM)$, we have that $\om_p = \sum\limits_{I \in \MI_k} f_I(p) (dx_I)_p$. Since $\om$ is smooth, we have that for each $J \in \MI_k$, 
		\begin{align*}
			\om\bigg (\pdv{x_J} \bigg ) 
			&= \sum\limits_{I \in \MI_k} f_I dx_I \bigg (\pdv{x_J} \bigg) \\
			&= f_J
		\end{align*} is smooth.
	\end{proof}

	\begin{ex}
		Let $\om \in \Om_k(M)$ and $(U, \phi)$ be a chart on $M$ with $\phi = (x_1, \cdots, x_n)$. If $\om = \sum\limits_{I \in \MI_k}f_I dx_I$, then $$d \om  = \sum_{I \in \MI_k} \sum_{i =1}^n \pdv{f_I}{x_i} dx_i \wedge dx_I$$.
	\end{ex}

	\begin{proof}
		First we note that
		\begin{align*}
			d(f_I dx_I) 
			&= df_I \wedge dx_I + (-1)^0 f d(dx_I) \\
			&= df_I \wedge dx_I \\
			&= \bigg( \sum\limits_{i=1}^n \pdv{f_I}{x_i}dx_i  \bigg) \wedge dx_I \\
			&= \sum\limits_{i=1}^n {\pdv{f_I}{x_i}}dx_i \wedge dx_I \\
		\end{align*}
		Then we extend linearly.
	\end{proof}
	
	
	\begin{defn}
		Let $(N, \MB)$ be a smooth manifold and $F: M \rightarrow N$ be a diffeomorphism. Define the \textbf{pullback of $F$}, denoted $F^*: \Om_k(N) \rightarrow \Om_k(M)$ by  $$(F^* \om)_p (v_1, \cdots, v_k) = \om_{F(p)} (dF_p(v_1), \cdots, dF_p(v_k))$$ for $\om \in \Om_k(N)$, $p \in M$ and $v_1, \cdots, v_k \in T_{p}M$
	\end{defn}

	
	
	
	
	
	
	
	
	
	
	
	\newpage.
	\newpage 
	\begin{defn}
		When working in $\R^n$, we introduce the formal objects $dx_1, dx_2, \cdots, dx_n$. Let $I = (i_1, i_2, \cdots, i_k)\in \MI_{k,n}$ and $\phi: \R^k \rightarrow \R^n$. Write $\phi = (\phi_1, \phi_2, \cdots, \phi_n)$. We formally define $dx_I = dx_{i_1}\wedge dx_{i_2} \wedge \cdots \wedge dx_{i_k}$ and $\phi_I = (\phi_{i_1}, \phi_{i_2}, \cdots, \phi_{i_k})$.   
	\end{defn}
	
	\begin{defn}
		Let $k \in \{0, 1, \cdots, n\}$. We define a $C^{\infty}(\R^n)$-module of dimension ${n \choose k}$, denoted $\Gamma^k(\R^n)$ to be 
		\[
		\Phi_k(\R^n) =
		\begin{cases}
			C^{\infty}(\R^n) & k = 0 \\
			\spn \{ dx_I: I \in \MI_{k,n} \} & k \geq 1
		\end{cases}
		\]
		For each $\om \in \Phi_k(\R^n)$ and $\chi \in \Gamma^l(\R^n)$,   we may form their \textbf{exterior product}, denoted by $\om \wedge \chi \in \Gamma^{k+l}(\R^n)$. Thus the exterior product is a map $\wedge : \Phi_k(\R^n) \times \Gamma^l(\R^n)\rightarrow \Gamma^{k+l}(\R^n)$. The exterior product is characterized by the following properties:
		\begin{enumerate}
			\item the exterior product is bilinear
			\item for each $\om \in \Phi_k(\R^n)$ and $\chi \in \Gamma^l(\R^n)$, $\om \wedge \chi = - \chi \wedge \om$ 
			\item for each $\om \in \Phi_k(\R^n)$, $\om \wedge \om = 0$
			\item for each $f \in C^{\infty}(\R^n)$ and $ \om \in \Phi_k(\R^n)$, $f \wedge \om = f \om$
		\end{enumerate}
		We call $\Phi_k(\R^n)$ the differential $k$-forms on $\R^n$. Let $\om$ be a $k$-form on $\R^n$. If $k \geq 1$, then for each $I \in \MI_{k,n}$, there exists $f_I \in C^{\infty}(\R^n)$ such that $\om = \sum\limits_{I \in \MI_{k,n}} f_I dx_I$
	\end{defn}
	
	
	\begin{note}
		The terms $dx_1, dx_2, \cdots, dx_n$ are are a sort of place holder for the coordinates of a point $x = (x_1, x_2, \cdots, x_n) \in \R^n$. When we work with functions $\phi: \R^k \rightarrow \R^n$, we will have different coordinates and to avoid confusion, we will write $\{du_1, du_2, \cdots, du_k\}$ when referencing the coordinates on $\R^k$ and $\{dx_1, dx_2, \cdots, dx_n\}$ when referencing the coordinates on $\R^n$. 
	\end{note}

	\begin{ex}
		Let $B_{n\times n} = (b_{i,j}) \in [C^{\infty}(M)]^{n \times n}$ be an $n\times n$ matrix. Then $$\bigwedge_{i=1}^n \bigg(\sum_{j=1}^n b_{i,j}dx_j\bigg) = (\det B) dx_1 \wedge dx_2 \wedge \cdots \wedge dx_n$$
	\end{ex}

	\begin{proof}
		Bilinearity of the exterior product implies that
		\begin{align*}
			\bigwedge_{i=1}^n \bigg(\sum_{j=1}^n b_{i,j}dx_j\bigg)
			 &=\bigg(\sum_{j=1}^n b_{1,j}dx_j\bigg) \wedge \bigg(\sum_{j=1}^n b_{2,j}dx_j\bigg) \wedge \cdots \wedge \bigg(\sum_{j=1}^n b_{n,j}dx_j\bigg) \\
			 &= \sum_{j_1, \cdots, j_n = 1}^n \bigg( \prod_{i=1}^n b_{i, j_i} \bigg) dx_{j_1}\wedge  dx_{j_2} \wedge \cdots \wedge  dx_{j_n} \\
			 &= \sum_{j_1 \neq \cdots \neq j_n} \bigg( \prod_{i=1}^n b_{i, j_i} \bigg) dx_{j_1}\wedge  dx_{j_2} \wedge \cdots \wedge  dx_{j_n} \\
			 &= \bigg[ \sum_{\sig \in S_n} \sgn(\sig) \bigg(\prod_{i=1}^n b_{i, \sig(i)} \bigg) \bigg] dx_{1}\wedge  dx_{2} \wedge \cdots \wedge  dx_{n} \\
			 &= (\det B) dx_{1}\wedge  dx_{2} \wedge \cdots \wedge  dx_{n}
		\end{align*} 
		
	\end{proof}

	\begin{defn}
		Let $f: \R^n \rightarrow \R$ be a $0$-form on $\R^n$. We define a $1$-form, denoted $df$, on $\R^n$ by $$df = \sum_{i = 1}^n \pdv{f}{x_i} dx_i$$
		Let $\om = \sum\limits_{I \in \MI_{k,n}} f_Idx_I$ be a $k$-form on $\R^n$. We can define a differential $k+1$-form, denoted $d \om$, on $\R^n$ by $$d\om = \sum\limits_{I \in \MI_{k,n}} df_I\wedge dx_I$$  
	\end{defn}

	\begin{ex}
		On $\R^3$, put 
		\begin{enumerate}
			\item $\om_0 = f_0$, 
			\item $\om_1 = f_1 dx_1 + f2 dx_2 + f_2 dx_3$, 
			\item $\om_2 = f_1dx_2\wedge dx_3 - f_2 dx_1 \wedge dx_3 + f_3 dx_1 \wedge dx_2$
		\end{enumerate} 
		Show that
		\begin{enumerate}
			\item $d\om_0 = \pdv{f_0}{x_1}dx_1 + \pdv{f_0}{x_2}dx_2 + \pdv{f_0}{x_3}dx_3$
			\item $d \om_1 = \bigg(\pdv{f_3}{x_2} - \pdv{f_2}{x_3} \bigg) dx_2 \wedge dx_3 + \bigg( \pdv{f_3}{x_1} - \pdv{f_1}{x_3}\bigg)dx_1 \wedge dx_3 + \bigg( \pdv{f_2}{x_1} - \pdv{f_1}{x_2} \bigg) dx_1 \wedge dx_2$
			\item $d \om_2 = \bigg( \pdv{f_1}{x_1} + \pdv{f_2}{x_2} + \pdv{f_3}{x_3} \bigg) dx_1 \wedge dx_2 \wedge dx_3$ 
		\end{enumerate}
	\end{ex}

	\begin{proof}
		Straightforward.
	\end{proof}

	\begin{ex}
		Let $I \in \MI_{k, n}$. Then there is a unique $I_* \in \MI_{n-k, n}$ such that $dx_I \wedge dx_{I_*} = dx_1 \wedge dx_2 \wedge \cdots \wedge dx_n$.
	\end{ex}
	
	\begin{defn}
		We define a linear map $*:\Phi_k(\R^n) \rightarrow \Gamma^{n-k}(\R^n)$ called the \textbf{Hodge $*$-operator} by $$* \sum\limits_{I \in \MI_{k,n}} f_I dx_I = \sum\limits_{I \in \MI_{k,n}} f_Idx_{I_*}$$
	\end{defn}

	\begin{defn}
		Let $\phi: \R^k \rightarrow \R^n$ be smooth. Write $\phi = (\phi_1, \phi_2, \cdots, \phi_n)$. We define $\phi^*:\Phi_k(\R^n) \rightarrow \Phi_k(\R^k)$ via the following properties: 
		\begin{enumerate}
			\item for each $0$-form $f$ on $\R^n$, $\phi^*f = f \circ \phi$
			\item  for $i = 1, \cdots , n$, $\phi^* dx_i = d\phi_i$ 
			\item for an $s$-form $\om$, and a $t$-form $\chi$ on $\R^n$,  $\phi^* (\om \wedge \chi) = (\phi^*\om) \wedge (\phi^*\chi)$
			\item for $l$-forms $\om, \chi$ on $\R^n$, $\phi^*(\om + \chi) = \phi^*\om + \phi^*\chi$ 
		\end{enumerate}
	\end{defn}

	\begin{ex}
			Let $M \subset \R^n$ be a $k$-dimensional smooth submanifold of $\R^n$, $\phi: U \rightarrow V$ a smooth parametrization of $M$, $\om = \sum_{I \in \MI_{k,n}} f_Idx_I$  an $k$-form on $\R^n$. Then $$\phi^* \om = \bigg( \sum_{I \in \MI_{k, n}} (f_I \circ \phi) (\det v\phi_I)\bigg)du_1 \wedge du_2 \wedge \cdots \wedge du_k$$ 
	\end{ex}

	\begin{proof}
		Using the definitions, we see that 
		\begin{align*}
			\phi^* \om 
			&= \phi^*  \sum_{I \in \MI_{k,n}} f_Idx_I \\
			&= \sum_{I \in \MI_{k,n}} (\phi^*f_I) \phi^*dx_I \\
			&= \sum_{I \in \MI_{k,n}} (f_I \circ  \phi)  d\phi_I
		\end{align*}
	
	A previous exercise tells us that for each $I \in \MI_{k,n}$,
	\begin{align*}
		d \phi_I 
		&= d\phi_{i_1} \wedge d\phi_{i_2} \wedge \cdots \wedge d \phi_{i_n} \\
		&= (\sum_{j = 1}^n \pdv{\phi_{i_1}}{u_j} du_j) \wedge (\sum_{j = 1}^n \pdv{\phi_{i_2}}{u_j} du_j) \wedge \cdots \wedge (\sum_{j = 1}^n \pdv{\phi_{i_k}}{u_j} du_j)   \\
		&= (\det v\phi_I)du_1 \wedge du_2 \wedge \cdots \wedge du_k
	\end{align*}
	Therefore 
	\begin{align*}
		\phi^* \om
		&= \sum_{I \in \MI_{k,n}} (f_I \circ  \phi)  d\phi_I \\
		&= \sum_{I \in \MI_{k,n}} (f_I \circ  \phi)  (\det v\phi_I)du_1 \wedge du_2 \wedge \cdots \wedge du_k \\
		&= \bigg(\sum_{I \in \MI_{k,n}} (f_I \circ  \phi)  (\det v\phi_I)\bigg)du_1 \wedge du_2 \wedge \cdots \wedge du_k
	\end{align*}
	\end{proof}
	
	\subsection{Integration of Differential Forms}
	
	\begin{defn}
		Let $U \subset \R^k$ be open and $\om = f dx_1 \wedge dx_2 \wedge \cdots \wedge dx_k$ a $k$-form on $\R^k$. Define $$\int_U \om = \int_U f dx$$
	\end{defn}
	
	\begin{defn}
		Let $M \subset \R^n$ be a $k$-dimensional oriented smooth submanifold of $\R^n$, $\om$ a $k$-form on $\R^n$ and $\phi: U \rightarrow V$ a local smooth, orientation-preserving parametrization of $M$. Define $$\int_V \om = \int_U \phi^*\om $$
	\end{defn} 

	\begin{ex}
		
	\end{ex}

	\begin{thm}{\textbf{(Stokes Theorem)}}
		Let $M \subset \R^n$ be a $k$-dimensional oriented smooth submanifold of $\R^n$ and $\om$ a $k-1$-form on $\R^n$. Then $$\int_{\partial M} \om = \int_M d \om$$
	\end{thm}

\end{document}