\chapter{Set Theory}

\section{Operations and Relations}

\begin{defn}\
	\begin{itemize}
		\item We define $[0] \defeq \varnothing$ and for $k \in \N$, we define $[k] \defeq \{1, \ldots, k\}$. 
		\item Let $S$ be a set and $k \in \N_0$. We define the \tbf{set of $k$-tupels with entries in $S$}, denoted $S^k$, by 
		$$S^k \defeq \{u: [k] \rightarrow S\}$$
		\item Let $S$ be a set. We define the \tbf{set of all tuples with entries in $S$}, denoted $S^*$, by 
		$$S^* \defeq \bigcup_{k \in \N_0} S^k$$
		\item Let $S$ be a set and $k \in \N_0$. We define the \tbf{set of $k$-ary operations on $S$}, denoted $\MF^k(S)$, by $\MF^k(S) \defeq S^{(S^k)}$. We define the \tbf{set of finitary operations on $S$}, denoted $\MF^*(S)$, by
		$$\MF^*(S) \defeq \bigcup_{k \in \N_0} \MF^k(S)$$
		\item Let $S$ be a set. We define the \tbf{operation arity map}, denoted $\ar: \MF^*(S) \rightarrow \N_0$, by 
		$$\ar f \defeq k, \quad f \in \MF^k(S)$$
		\item Let $S$ be a set, $\MF \subset \MF^*(S)$ and $k \in \N_0$. We define the \tbf{$k$-ary members of $\MF$}, denoted $\MF_k$, by 
		$$\MF_k \defeq \MF \cap \MF^k(S)$$
		\item Let $S$ be a set and $k \in \N_0$. We define the \tbf{set of $k$-ary relations on $S$}, denoted $\MR^k(S)$, by $\MR^k(S) \defeq \MP(S^k)$. We define the \tbf{set of finitary relations on $S$}, denoted $\MR^*(S)$, by
		$$\MR^*(S) \defeq \bigcup_{k \in \N_0} \MR^k(S)$$
		\item Let $S$ be a set. We define the \tbf{arity map}, denoted $\ar: \MR^*(S) \rightarrow \N_0$, by 
		$$\ar R \defeq k, \quad f \in \MR^k(S)$$
		\item Let $S$ be a set, $\MR \subset \MR^*(S)$ and $k \in \N_0$. We define the \tbf{$k$-ary members of $\MR$}, denoted $\MR_k$, by 
		$$\MR_k \defeq \MR \cap \MR^k(S)$$
	\end{itemize}
\end{defn}

\begin{defn}
	Let $S$ be a set, $k \geq 2$ and $f \in \MF^k(S)$. Then $f$ is said to be
	\begin{itemize}
		\item \tbf{associative} if for each $x_1, \ldots, x_k, x_{k+1}, \ldots, x_{k + (k-1)} \in S$, 
		\begin{align*}
			f(f(x_1, \ldots, x_k) x_{k+1}, \ldots, x_{k + (k-1)}) 
			& = f(x_1, f(x_2, \ldots, x_{k+1}), x_{k+2}, \ldots, x_{k + (k-1)}) \\
			& \vdots \\
			& = f(x_1, \ldots, x_{k-1}, f(x_k, \ldots, x_{k + (k-1)}) )
		\end{align*}
		\item \tbf{symmetric} if for each $x_1, \ldots, x_k \in S$, $\sig \in S_k$, $f(x_1, \ldots, x_k) = f(x_{\sig(1)}, \ldots, x_{\sig(k)})$.
		\item \tbf{idempotent} if for each $x \in S$,
		$f(x, \ldots, x) = x$
	\end{itemize}
\end{defn}

\begin{defn}
	Let $S$ be a set, $\MF \subset \MF^*(S)$ and $C \subset S$. Then $C$ is said to be  \tbf{$\MF$-closed} if for each $k \in \N_0$, $f \in \MF_k$ and $a \in C^k$, $f(a) \in C$.
\end{defn}

\begin{ex}
	Let $S$ be a set, $\MF \subset \MF^*(S)$ and $\MC \subset \MP(S)$. If for each $C \in \MC$, $C$ is  $\MF$-closed, then $\bigcap\limits_{C \in \MC} C$ is $\MF$-closed
	\tcr{need special case where $k=0$? maybe trivially true?}
\end{ex}

\begin{proof}
	Suppose that for each $C \in \MC$, $C$ is  $\MF$-closed. Let $k \in \N_0$, $f \in \MF_k$, $a_1, \ldots, a_k \in \bigcap\limits_{C \in \MC} C$ and $C_0 \in \MC$. Since $C_0 \in \MC$, we have that 
	\begin{align*}
		a_1, \ldots, a_k 
		& \in \bigcap_{C \in \MC} C \\
		& \subset C_0
	\end{align*}
	Since $C_0$ is $\MF$-closed, we have that $f(a_1, \ldots, a_k) \in C_0$. Since $C_0 \in \MC$ is arbitrary, we have that for each $C \in \MC$, $f(a_1, \ldots, a_k) \in C$. Hence $f(a_1, \ldots, a_k) \in \bigcap\limits_{C \in \MC} C$. Since $k \in \N_0$ and $a_1, \ldots, a_k \in \bigcap\limits_{C \in \MC} C$ are arbitrary, we have that $\bigcap\limits_{C \in \MC} C$ is $\MF$-closed.
\end{proof}

