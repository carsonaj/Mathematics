\documentclass[12pt]{amsart}
\usepackage[margin=1in]{geometry} 
\usepackage{amsmath,amsthm,amssymb,setspace, mathtools}
\usepackage{tikz-cd}

\usepackage{color}   %May be necessary if you want to color links
\usepackage{hyperref}
\hypersetup{
	colorlinks=true, %set true if you want colored links
	linktoc=all,     %set to all if you want both sections and subsections linked
	linkcolor=black,  %choose some color if you want links to stand out
	urlcolor=cyan
}


%
%
%
\newif\ifhideproofs
%\hideproofstrue %uncomment to hide proofs
%
%
%
%
\ifhideproofs
\usepackage{environ}
\NewEnviron{hide}{}
\let\proof\hide
\let\endproof\endhide
\fi

\theoremstyle{definition}
\newtheorem{definition}{Definition}[subsection]
\newtheorem{defn}[definition]{Definition}
\newtheorem{note}[definition]{Note}
\newtheorem{thm}[definition]{Theorem}
\newtheorem{lem}[definition]{Lemma}
\newtheorem{prop}[definition]{Proposition}
\newtheorem{cor}[definition]{Corollary}
\newtheorem{conj}[definition]{Conjecture}
\newtheorem{ex}[definition]{Exercise}


\newcommand{\al}{\alpha}
\newcommand{\gam}{\gamma}
\newcommand{\Gam}{\Gamma}
\newcommand{\be}{\beta} 
\newcommand{\ze}{\zeta} 
\newcommand{\del}{\delta} 
\newcommand{\Del}{\Delta}
\newcommand{\lam}{\lambda}  
\newcommand{\Lam}{\Lambda} 
\newcommand{\ep}{\epsilon}
\newcommand{\sig}{\sigma} 
\newcommand{\om}{\omega}
\newcommand{\kap}{\kappa} 
\newcommand{\Om}{\Omega}
\newcommand{\C}{\mathbb{C}}
\newcommand{\N}{\mathbb{N}}
\newcommand{\E}{\mathbb{E}}
\newcommand{\Z}{\mathbb{Z}}
\newcommand{\R}{\mathbb{R}}
\newcommand{\T}{\mathbb{T}}
\newcommand{\Q}{\mathbb{Q}}
\renewcommand{\P}{\mathbb{P}}
\newcommand{\MA}{\mathcal{A}}
\newcommand{\MC}{\mathcal{C}}
\newcommand{\MB}{\mathcal{B}}
\newcommand{\MF}{\mathcal{F}}
\newcommand{\MG}{\mathcal{G}}
\newcommand{\ML}{\mathcal{L}}
\newcommand{\MN}{\mathcal{N}}
\newcommand{\MS}{\mathcal{S}}
\newcommand{\MP}{\mathcal{P}}
\newcommand{\ME}{\mathcal{E}}
\newcommand{\MT}{\mathcal{T}}
\newcommand{\MM}{\mathcal{M}}
\newcommand{\MI}{\mathcal{I}}
\newcommand{\MU}{\mathcal{U}}
\newcommand{\MO}{\mathcal{O}}
\newcommand{\MQ}{\mathcal{Q}}

\newcommand{\dm}{\, d m}
\newcommand{\dmu}{\, d \mu}
\newcommand{\dnu}{\, d \nu}
\newcommand{\dlam}{\, d \lambda}


\newcommand{\io}{\text{ i.o.}}
\newcommand{\ev}{\text{ ev.}}
\renewcommand{\r}{\rangle}
\renewcommand{\l}{\langle}

\newcommand{\p}{\partial}

\newcommand{\RG}{[0,\infty]}
\newcommand{\Rg}{[0,\infty)}
\newcommand{\Ll}{L^1_{\text{loc}}(\R^n)}

\newcommand{\limfn}{\liminf \limits_{n \rightarrow \infty}}
\newcommand{\limpn}{\limsup \limits_{n \rightarrow \infty}}
\newcommand{\limn}{\lim \limits_{n \rightarrow \infty}}
\newcommand{\convt}[1]{\xrightarrow{\text{#1}}}
\newcommand{\conv}[1]{\xrightarrow{#1}} 
\newcommand{\seq}[2]{(#1_{#2})_{#2 \in \N}}

\newcommand{\loc}{\text{loc}}
\newcommand{\BV}{\text{BV}}
\newcommand{\NBV}{\text{NBV}}
\newcommand{\TV}{\text{TV}}

\DeclareMathOperator{\sgn}{sgn}
\DeclareMathOperator{\spn}{span}
\DeclareMathOperator{\supp}{supp}
\DeclareMathOperator{\Aut}{Aut}
\DeclareMathOperator{\Homeo}{Homeo}
\DeclareMathOperator{\Sym}{Sym}
\DeclareMathOperator{\diam}{diam}
\DeclareMathOperator{\iso}{Iso}
\DeclareMathOperator{\id}{id}
\DeclareMathOperator{\cl}{cl}
\DeclareMathOperator{\Int}{Int}
\DeclareMathOperator{\bal}{bal}
\DeclareMathOperator{\cnv}{conv}
\DeclareMathOperator{\epi}{epi}


\newcommand{\lex}[1]{\label{ex:#1}}
\newcommand{\ld}[1]{\label{defn:#1}}
\newcommand{\rex}[1]{Exercise \ref{ex:#1}}
\newcommand{\rd}[1]{Definition \ref{defn:#1}}



\begin{document}
	
	\title{Introduction to Measure and Integration}
	\author{Carson James}
	\maketitle
	
	\tableofcontents
	
	\newpage
	
	\section*{Preface}
	\begin{flushleft}
%		This aim of this book is to help students develope a basic grasp of the theory of integration. A typical student's first exposure to integration is in the context of the Darboux integral. This integral is applicable to a relatively small class of complex-valued functions of one or more real variables. Although this is integral is of critical importance, it is not sufficient for many purposes. Extending the Darboux integral to the Lebesgue integral, we can define a notion of integration for certain complex-valued functions on more general spaces, like for example topological spaces. To do so, we must first develop some measure theory, which is useful in its own right. Further extending the Lebesgue integral to the Bochner integral, we can define a notion of integration for certain vector valued functions. 
	\end{flushleft}

	\vspace{.2cm}

	\begin{flushleft}
%		The target audience of this book is composed of those who wish to deepen their understanding of integration beyond the Darboux integral. In practice, a basic understanding of the integral would benefit anyone who works with integration and limits in more exotic spaces. For example, students of statistics, physics and disciplines which utilize numerical solutions to differential equations would benefit greatly from a deeper understanding of the integral. 
	\end{flushleft}
	
	
	\section*{Notes}
	\begin{itemize}
	\item Replace the notation "$\text{Im} f$`` with $h$ where $f = g + ih$ so that $\text{Im}f$ can refer to \textbf{image of $f$}.
\end{itemize}		
	
	\newpage
	
	
	
	
	
	
	
	
	\section{The Darboux Integral}
	
	\subsection{Definition and Properties}
	\begin{defn} \ld{00000} 
		Let $a,b \in \R$. Suppose that $a<b$. Define $$B([a,b]) = \{f:[a,b] \rightarrow \R: f\text{ is bounded}\}$$
	\end{defn}
	
	\begin{defn} \ld{00000} 
		Let $a,b \in \R$. Suppose that $a<b$. Let $x_0, \cdots, x_n \in [a,b]$. Suppose that $a= x_0 < x_1 < \cdots < x_n = b$. Put $\MP = \{x_0, \cdots, x_n\}$. Then $\MP$ is said to be a \textbf{partion} of $[a,b]$. 
	\end{defn}
	
	\begin{defn} \ld{00000} 
		Let $f \in B([a,b])$ and $\MP = \{x_0, \cdots, x_n\}$ a partion of $[a,b]$. Suppose that $f$ is bounded. For $i = 1, \cdots, n$, put 
		$$M^f_i = \sup_{x \in [x_{i-1}, x_i]} f(x)$$ and 
		$$m^f_i = \inf_{x \in [x_{i-1}, x_i]} f(x)$$ 
		We define the \textbf{upper Darboux sum} of $f$ with respect to $\MP$, denoted $U_\MP f$, to be $$U_\MP f = \sum_{i=1}^n M^f_i (x_i - x_{i-1})$$ 
		and we define the \textbf{lower Darboux sum} of $f$ with respect to $\MP$, denoted $L_\MP f$, to be
		$$L_\MP f = \sum_{i=1}^n m^f_i (x_i - x_{i-1})$$ 
	\end{defn}

	\begin{ex} \lex{00000} 
		Let $f \in B([a,b])$ and $\MP$ a partition of $[a,b]$. Then $$\bigg[\inf_{x \in [a,b]} f(x) \bigg] (b-a) \leq L_\MP f \leq U_\MP f \leq \bigg[ \sup_{x \in [a,b]} f(x)  \bigg] (b-a)$$
	\end{ex}

	\begin{proof}
		Clear.
	\end{proof}

	\begin{ex} \lex{00000} 
			Let $f \in B([a,b])$ and $\MP$, $\MP'$ partitions of $[a,b]$. If $\MP \subset \MP'$, then 
			\begin{enumerate}
				\item $U_{\MP'} f \leq U_{\MP} f$
				\item $L_{\MP} f \leq L_{\MP'} f$
			\end{enumerate}
	\end{ex}

	\begin{proof} \
		\begin{enumerate}
			\item Assume that $\MP = \{x_0, \cdots, x_n\}$ and $\MP' = \MP \cup \{x'\}$. Then there exists $j \in \{1, \cdots, n\}$ such that $x_{j-1} < x' < x_j$. Define $$M'_1 = \sup_{x \in [x_{j-1}, x']} f(x), \hspace{.4cm} M'_2 = \sup_{x \in [x', x_{j}]} f(x)$$
			Since $[x_{j-1}, x'], [x', x_j] \subset [x_{j-1}, x_j]$, we have that $M'_1, M'_2 \leq  M^f_j$. Then 
			\begin{align*}
				U_{P'}f 
				&= \sum_{i =1}^{j-1} M^f_i(x_i - x_{i-1}) + M'_1(x' - x_{j-1}) + M'_2(x_j - x') + \sum_{i =j+1}^n M^f_i(x_i - x_{i-1})  \\
				&\leq   \sum_{i =1}^{n} M^f_i(x_i - x_{i-1}) \\
				&= U_P f 
			\end{align*}
			By induction, this is true for general partitions $P \subset \MP'$.
			\item Similar to $(1)$.
		\end{enumerate}
	\end{proof}

	\begin{ex} \lex{00000} 
		Let $f, g \in B([a,b])$ and $\MP = \{x_0, \cdots, x_n\}$ a partition of $[a,b]$. Then 
		\begin{enumerate}
			\item $U_\MP(f+g) \leq U_\MP f + U_\MP g$
			\item $L_\MP(f+g) \geq L_\MP f + L_\MP g$
		\end{enumerate}
	\end{ex}

	\begin{proof} \
		\begin{enumerate}
			\item For each $i \in \{1, \cdots, n\}$, $M^{f+g}_i \leq M^f_i + M^g_i$. So 
			\begin{align*}
				U_\MP (f+g) 
				&= \sum_{i=1}^n M^{f+g}_i (x_i - x_{i-1}) \\
				& \leq \sum_{i=1}^n (M^f_i + M^g_i)(x_i - x_{i-1}) \\
				&= \sum_{i=1}^n M^f_i(x_i - x_{i-1}) + \sum_{i=1}^n M^g_i(x_i - x_{i-1}) \\
				&= U_\MP f + U_\MP g
			\end{align*}
			\item Similar to $(1)$
		\end{enumerate}
	\end{proof}

	\begin{ex} \lex{00000} 
		Let $f \in B([a,b])$ and $\MP = \{x_0, \cdots, x_n\}$ a partition of $[a,b]$. Then 
		\begin{enumerate}
			\item $U_\MP (-f) = -L_P f$
			\item $L_\MP (-f) = - U_\MP f$
		\end{enumerate}
	\end{ex}

	\begin{proof}\
		\begin{enumerate}
			\item Since for $i \in \{1, \cdots, n\}$, $M^{-f}_i = -m^f_i$ we see that 
			\begin{align*}
				U_\MP(-f) 
				&= \sum_{i=1}^n M^{-f}_i(x_i - x_{i-1}) \\
				&= - \sum_{i=1}^n m^f_i(x_i - x_{i-1}) \\
				&= - L_\MP f
			\end{align*}
			\item Similar to $(1)$.
		\end{enumerate}
	\end{proof}

	\begin{ex} \lex{00000} 
		Let $f \in B([a,b])$, $c >0$ and $\MP = \{x_0, \cdots, x_n\}$ a partition of $[a,b]$. Then 
		\begin{enumerate}
			\item $U_\MP (cf) = c U_\MP f$ 
			\item $L_\MP (cf) = c L_\MP f $
		\end{enumerate}
	\end{ex}

	\begin{proof}\
		\begin{enumerate}
			\item Since for $i \in \{1, \cdots, n\}$, $M^{cf}_i = cM^f_i$, we see that  
			\begin{align*}
				U_\MP (cf) 
				&= \sum_{i=1}^n M^{cf}_i (x_i - x_{i-1}) \\
				&= c \sum_{i=1}^n M^f_i (x_i - x_{i-1}) \\
				&= c U_\MP f
			\end{align*}
			\item Similar to $(1)$
		\end{enumerate}
	\end{proof}
	
	\begin{defn} \ld{00000} 
		Let $f \in B([a,b])$. We define the \textbf{upper Darboux integral} of $f$, denoted $U f$, to be $$Uf = \inf \{U_\MP f: \MP \text{ is a partition of } [a,b]\}$$
		and we define the \textbf{lower Darboux integral} of $f$, denoted $L f$, to be $$Lf = \sup \{L_\MP f: \MP \text{ is a partition of } [a,b]\}$$ 
	\end{defn}

	\begin{ex} \lex{00000} 
		Let $f \in B([a,b])$. Then $$ \bigg[\inf_{x \in [a,b]} f(x) \bigg] (b-a) \leq Lf \leq Uf \leq \bigg[\sup_{x \in [a,b]} f(x) \bigg] (b-a)$$
	\end{ex}

	\begin{proof}
		Clearly $$\bigg[\inf_{x \in [a,b]} f(x) \bigg] (b-a) \leq Lf \hspace{.2cm } \text{ and } \hspace{.2cm } Uf \leq \bigg[\sup_{x \in [a,b]} f(x) \bigg] (b-a)$$ 
		Let $\ep >0$. Then there exist partitions $\MP_1$ and $\MP_2$ of $[a,b]$ such that $U_{\MP_1} f < U f + \ep/2$ and $L_{\MP_2} f > Lf - \ep/2 $. Define $\MP = \MP_1 \cup \MP_2$. Then 
		\begin{align*}
			Uf 
			& \geq U_{\MP_1}  - \ep/2 \\
			&> U_\MP f - \ep/2 \\
			&\geq L_\MP f -\ep/2 \\
			&\geq L_{\MP_2} f -\ep/2 \\
			&> L f - \ep
		\end{align*} 
		Since $\ep >0$ is arbitrary, we have that $Uf \geq Lf$.
	\end{proof}

	\begin{ex} \lex{00000} 
		Let $f, g \in B([a,b])$. Then 
		\begin{enumerate}
			\item $U(f+g) \leq U f + U g$
			\item $L(f+g) \geq L f + L g$
		\end{enumerate}
	\end{ex}

	\begin{proof}\
		\begin{enumerate}
			\item Let $\ep >0$. Then there exists a partitions $\MP_1$ of $[a,b]$ such that $U_{\MP_1} f < U f + \ep/2$ and $U_{\MP_2} g < U f + \ep/2$. Define $\MP = \MP_1 \cup \MP_2$. Then 
			\begin{align*}
				U_\MP(f +g) 
				&\leq U_\MP f + U_\MP g \\
				&\leq U_{\MP_1} f + U_{\MP_2} g \\
				&< U f + \ep/2 +  U g + \ep/2 \\
				&= Uf + Ug + \ep
			\end{align*}
			Since $\ep >0$ is arbitrary, $U_\MP(f +g)  \leq Uf + Ug$.
			\item Similar to $(1)$.
		\end{enumerate}
	\end{proof}

	\begin{ex} \lex{00000} 
		Let $f \in B([a,b])$. Then 
		\begin{enumerate}
			\item $U(-f) = -Lf$
			\item $L(-f) = -Uf$
		\end{enumerate}
	\end{ex}

	\begin{proof} \
		\begin{enumerate}
			\item Using a previous exercise, we have that
			\begin{align*}
				U(-f)
				&= \inf \{U_\MP(-f): \MP \text{ is a partition of } [a,b]\} \\
				&= \inf \{ -L_\MP f: \MP \text{ is a partition of } [a,b]\} \\
				&= - \sup \{L_\MP f: \MP \text{ is a partition of } [a,b]\} \\
				&= - Lf
			\end{align*}
			\item Similar to $(1)$
		\end{enumerate}
	\end{proof}

	\begin{ex} \lex{00000} 
		Let $f \in B([a,b])$ and $c \geq 0$. Then 
		\begin{enumerate}
			\item $U (cf) = c U f$ 
			\item $L (cf) = c L f $
		\end{enumerate}
	\end{ex}
	
	\begin{proof}\
		\begin{enumerate}
			\item Using a previous exercise, we have that
			\begin{align*}
				U(cf) 
				&= \inf \{U_\MP (cf): \MP \text{ is a partition of } [a,b]\} \\
				&= \inf \{c U_\MP f: \MP \text{ is a partition of } [a,b]\} \\
				&= c \inf \{U_\MP f: \MP \text{ is a partition of } [a,b]\} \\
				&= c Uf
			\end{align*}
			\item Similar to $(1)$
		\end{enumerate}
	\end{proof}

	\begin{defn} \ld{00000} 
		Let $f \in B([a,b])$. Then $f$ is said to be \textbf{Darboux integrable} if $Uf = Lf$. If $f$ is Darboux integrable, we define the \textbf{Darboux integral} of $f$, denoted by $$\int f \hspace{.2cm} \text{or} \hspace{.2cm} \int f(x) dx$$ to be $$\int f = Uf = Lf$$ The set of bounded, Darboux integrable functions is denoted by $D([a,b])$.
	\end{defn}

	\begin{ex} \lex{00000} 
		Let $f \in B([a,b])$. Then $f \in D([a,b])$ iff for each $\ep >0$, there exists a partition $\MP$ of $[a,b]$ such that $U_\MP f - L_\MP f < \ep$.
	\end{ex}

	\begin{proof}
		Suppose that $f \in D([a,b])$. Let $\ep >0$. Then there exist partions $\MP_1$, $\MP_2$ of $[a,b]$ such that $U_{\MP_1} f < Uf + \ep/2$ and $L_{\MP_2} f > Lf - \ep/2$. Define $\MP = \MP_1 \cup \MP_2$. Then $U_\MP f \leq U_{\MP_1} f$ and $L_P f \geq L_{\MP_2}f$. So  
		\begin{align*}
			U_\MP f - L_\MP f 
			&< Uf - L f + \ep \\
			&= \ep
		\end{align*}  
		Conversely, suppose that for each $\ep >0$, there exists a partition $\MP$ of $[a,b]$ such that $U_\MP f - L_\MP f < \ep$. For the sake of contradiction, suppose that $Uf - Lf > 0$. Choose $\ep = Uf - Lf$. Then there exists a partition $\MP$ of $[a,b]$ such that $U_\MP f - L_\MP f < \ep$. Since $Uf \leq U_\MP f$ and $Lf \geq L_\MP f$, we have that 
		\begin{align*}
			\ep 
			&> U_\MP f - L_\MP f \\
			&\geq Uf - Lf \\
			&= \ep
		\end{align*} 
		which is a contradiction. Hence $Uf = Lf$ and $f \in D([a,b])$.
	\end{proof}

	\begin{ex} \lex{00000} 
		Let $f, g \in D([a,b])$. Then $f+g \in D([a,b])$ and $$\int (f+g) = \int f + \int g$$
	\end{ex}
	
	\begin{proof}
		Clearly $f+g \in B([a,b])$. Using some previous results, we have that 
		\begin{align*}
			\int f + \int g 
			&= Lf + Lg \\
			&\leq L(f+g) \\
			&\leq U(f+g) \\
			&\leq Uf + Ug \\
			&= \int f + \int g
		\end{align*}
		So $U(f+g) = L(f+g) = \int f + \int g$. Therefore $f+g \in D([a,b])$ and $$\int (f+g) = \int f + \int g$$.
	\end{proof}

	\begin{ex} \lex{00000} 
		Let $f \in D([a,b])$ and $c \in \R$. Then $cf \in D([a,b])$ and $$\int (cf) = c \int f $$
	\end{ex}

	\begin{proof}
		Clearly $cf \in B([a,b])$. If $c \geq 0$, then 
		\begin{align*}
			L (cf) 
			&= c Lf \\
			&= c \int f \\
			&= c Uf \\
			&= U(cf)
		\end{align*} 
		So $$L(cf) = U(cf) = c \int f$$ If $c <0$, then 
		\begin{align*}
			L (cf) 
			&= L(-|c| f) \\
			&= -U(|c|f) \\
			&= - |c|Uf \\
			&= c \int f \\
			&= -|c| L f \\
			&= - L(|c|f) \\
			&= U(-|c|f) \\
			&= U (c f) \\
		\end{align*} 
		So $$L(cf) = U(cf) = c \int f$$ Therefore $cf \in D([a,b])$ and $$\int (cf) = c \int f$$
	\end{proof}

	\begin{cor}
		We have that $D([a,b])$ is a vector space and the map $I: D([a,b]) \rightarrow \R$ given by $If = \int f$ is linear.
	\end{cor}

	\begin{proof}
		Clear.
	\end{proof}

	\begin{ex} \lex{00000} 
		Let $f:[a,b] \rightarrow \R$. If $f$ is continuous, then $f \in D([a,b])$. 
	\end{ex}

	\begin{proof}
		Suppose that $f$ is continuous. Then $f$ is uniformly continuous. Let $\ep >0$. Uniform continuity implies that there exists $\del > 0$ such that for each $x, y \in [a,b]$, $|x-y| < \del$ implies that $|f(x) - f(Y)| < \ep/(b-a)$. Choose $n \in \N$ such that $(b-a)/n < \del$. For $i \in \{0, \cdots, n\}$, define $x_i = a + i(b-a)/n$. Put $\MP = \{x_0, \cdots, x_n\}$. Continuity implies that for each $i \in \{1, \cdots, n\}$, there exists $x_i^M, x_i^m \in [x_{i-1}, x_i]$ such that $f(x_i^M) = M_i^f$ and $f(x_i^m) = m_i^f$. Then 
		\begin{align*}
			U_\MP f - L_\MP f 
			&= \sum_{i=1}^n M^f_i(x_i - x_{i-1}) - \sum_{i=1}^n m^f_i(x_i - x_{i-1}) \\
			&= \sum_{i=1}^n (M^f_i - m^f_i)(x_i - x_{i-1}) \\
			&= \sum_{i=1}^n [f(x_i^M) - f(x_i^m)](x_i - x_{i-1}) \\
			& < \sum_{i=1}^n \frac{\ep}{b-a}(x_i - x_{i-1}) \\
			&= \ep 
		\end{align*}
		So for each $\ep >0$, there exists a partition $\MP$ of $[a,b]$ such that $U_\MP f - L_\MP f < \ep$. Hence $f \in D([a,b])$.
	\end{proof}

	\begin{ex} \lex{00000} 
		Let $f:[a,b] \rightarrow \R$. If $f$ is monotonic, then $f \in D([a,b])$.
	\end{ex}

	\begin{proof}
		Suppose that $f$ is increasing. Let $\ep >0$. Choose $n \in \N$ such that $(b-a)[f(b) - f(a)] /n < \ep $. For $i \in \{0, \cdots, n\}$, define $x_i = a + i(b-a)/n$. Put $\MP = \{x_0, \cdots, x_n\}$. Then 
		\begin{align*}
			U_\MP f - L_\MP f 
			&= \sum_{i=1}^n M^f_i(x_i - x_{i-1}) - \sum_{i=1}^n m^f_i(x_i - x_{i-1}) \\
			&= \sum_{i=1}^n (M^f_i - m^f_i)(x_i - x_{i-1}) \\
			&= \frac{b-a}{n} \sum_{i=1}^n [f(x_i) - f(x_{i-1})]  \\
			&= \frac{b-a}{n} [f(b) - f(a)] \\
			& < \ep
		\end{align*}
	So for each $\ep >0$, there exists a partition $\MP$ of $[a,b]$ such that $U_\MP f - L_\MP f < \ep$. Hence $f \in D([a,b])$. The case is similar if $f$ is decreasing. 
	\end{proof}

	\begin{ex} \lex{00000} 
		Define $\chi_{\Q}:[0,1] \rightarrow \R$ by $$\chi_{\Q}(x) = \begin{cases}
			1 & x \in \Q \\
			0 & x \not \in \Q
		\end{cases}$$
		Then $\chi_\Q \not \in D([a,b])$.
	\end{ex}

	\begin{proof}
		Let $\MP = \{x_0, \cdots, x_n\}$ be a partition of $[0,1]$. Then for each $i \in \{1, \cdots, n\}$, $M^{\chi_{\Q}}_i = 1$ and $m^{\chi_{\Q}}_i = 0$. So $U_\MP \chi_\Q = 1$ and $L_\MP \chi_\Q = 0$. Since $\MP$ is arbitrary, we have that $U \chi_\Q = 1$ and $L \chi_\Q = 0$
	\end{proof}

	
	
	
	
	
	
	
	
	
	
	
	
	
	
	\newpage
	
	\section{Measure Spaces}
	
	
	\subsection{Elementary Families and Algebras}
	
	\begin{defn} \ld{00000} 
		Let $X$ be a set and $\ME \subset \MP(X)$. Then $X$ is said to be an \textbf{elementary family on X} if 
		\begin{enumerate}
			\item $\varnothing \in \ME$
			\item for each $A, B \in \ME$, $A \cap B \in \ME$
			\item for each $A \in \ME$, there exist $(A_j)_{j=1}^n \subset \ME$ such that  $(A_j)_{j=1}^n $ is disjoint and $A^c = \bigcup\limits_{j=1}^n A_j$
		\end{enumerate}
	\end{defn}
	
	\begin{ex} \lex{00000} 
		Define $$\ME = \{(a,b]: a,b \in \overline{\R}\}$$ where we take $(a, \infty] = (a, \infty)$. Then $\ME$ is an elementary family on $\R$
	\end{ex}
	
	\begin{proof}\
		\begin{enumerate}
			\item $\varnothing = (0,0] \in \ME$
			\item Let $a_1, a_2, b_1, b_2 \in \overline{\R}$. Then 
			\[
			(a_1, b_1] \cap  (a_2, b_2] =
			\begin{cases}
				\varnothing  & b_1 \leq a_2 \\
				(a_2, b_1] & b_1 > a_2 
			\end{cases} 
			\]
			So $(a_1, b_1] \cap  (a_2, b_2] \in \ME$.
			\item Let $a, b \in \R$. Suppose that $a < b$. Then $(a,b]^c = (-\infty, a] \cup (b, \infty) \in \ME$. 
		\end{enumerate}
	\end{proof}
	
	\begin{defn} \ld{00000} 
		Let $X$ be a set and $\MA_0 \subset \MP(X)$. Then $\MA_0$ is said to be an \textbf{algebra} on $X$ if 
		\begin{enumerate}
			\item $\MA_0 \neq \varnothing$
			\item for each $A \in \MA_0$, $A^c \in \MA_0$
			\item for each $A,B \in \MA_0$, $A \cup B \in \MA_0$
		\end{enumerate}
	\end{defn}
	
	\begin{ex} \lex{00000} 
		Let $X$ be a set and $\ME$ an elementary family on $X$. Define $$\MA^{\ME}_0 = \bigg \{ \bigcup_{j=1}^n A_j: (A_j)_{j=1}^n \text{ is disjoint and } (A_j)_{j=1}^n \subset \ME \bigg \}$$ Then $\MA^{\ME}_0$ is an algebra on $X$.
	\end{ex}	
	
	\begin{proof}\
		\begin{enumerate}
			\item By definition, $\varnothing \in \ME \subset \MA^{\ME}_0$. So $\MA^{\ME}_0 \neq \varnothing$. 
			\item Let $A \in \MA^{\ME}_0$, there exists $(A_j)_{j=1}^n \subset \ME$ such that $(A_j)_{j=1}^n$ is disjoint and $A = \bigcup\limits_{j =1}^n A_j$. By definition of $\ME$, for each $j \in \{1, \ldots, n\}$, there exist $(B_{j,k})_{k=1}^{n_j} \subset \ME$ such that $(B_{j,k})_{k=1}^{n_j}$ is disjoint and $A_j^c = \bigcup\limits_{k=1}^{n_j}B_{j,k}$. Then 
			\begin{align*}
				A^c 
				&= \bigcap_{j=1}^n A_j^c \\
				&= \bigcap_{j=1}^n \bigg( \bigcup\limits_{k=1}^{n_j}B_{j,k} \bigg)\\
				&= \bigcup\\
			\end{align*}	 
			\item Let $A, B \in \MA^{\ME}_0$. Then there exist $(A_j)_{j=1}^n, (B_j)_{j=1}^m \subset \ME$ such that $A = \bigcup\limits_{j=1}^n A_j$ and  $B = \bigcup\limits_{j=1}^m B_j$. Then 
			\begin{align*}
				A \cup B 
				&=  \bigg( \bigcup\limits_{j=1}^n A_j \bigg) \cup \bigg( \bigcup\limits_{j=1}^m B_j \bigg) \\
			\end{align*}
		\end{enumerate}
		\textbf{FINISH!!!}
	\end{proof}

	
























	\newpage
	\subsection{Sigma Algebras}
		
	\begin{defn} \ld{00000} 
		Let $X$ be a set and $\MA \subset \MP(X)$. Then $\MA$ is said to be a $\sigma$\textbf{-algebra} on $X$ if 
		\begin{enumerate}
			\item $\MA \neq \varnothing$
			\item for each $A \in \MA$, $A^c \in \MA$
			\item for each $(A_n)_{n \in \N} \subset \MA$, $\bigcup\limits_{n \in \N}A_n \in \MA$
		\end{enumerate}
	\end{defn}
	
	\begin{ex} \lex{00000} 
		Let $X$ be a set and $\MA$ a $\sig$-algebra on $X$. Then 
		\begin{enumerate}
			\item $X, \varnothing \in \MA$
			\item for each $(A_n)_{n \in \N} \subset \MA$, $\bigcap\limits_{n \in \N} \in \MA$
			\item For each $A, B \in \MA$, $A \setminus B \in \MA$  
		\end{enumerate}
	\end{ex}
	
	\begin{proof}\
		\begin{enumerate}
			\item Since $\MA \neq \varnothing$, there exists $A \in \MA$. Then $A^c \in \MA$. Hence $X = A \cup A^c \in \MA$ and $\varnothing = X^c \in \MA$.
			\item Let $(A_n)_{n \in \N} \subset \MA$. Then $(A_n^c)_{n \in \N} \subset MA$. So $\bigcup\limits_{n \in \N}A_n^c \in \MA$. Therefore \begin{align*}
				\bigcap\limits_{n \in \N}A_n 
				&= \bigg(\bigcup\limits_{n \in \N}A_n^c\bigg)^c \in \MA
			\end{align*}
			\item Let $A,B \in \MA$. Then $A \setminus B = A \cap B^c \in \MA$. 
		\end{enumerate}
	\end{proof}
	
	\begin{ex} \lex{00000} 
		Let $X$ be a set and $(\MA_i)_{i \in I}$ a collection of $\sig$-algebras (resp. algebra) on $X$. Then $\bigcap\limits_{i \in I}\MA_i$ is a $\sig$-algebra (resp. algebra) on $X$.
	\end{ex}
	
	\begin{proof}\
		\begin{enumerate}
			\item For each $i \in I$, $X \in \MA_i$. Thus $X \in \bigcap\limits_{i \in I}\MA_i$ and $\bigcap\limits_{i \in I}\MA_i \neq \varnothing$.
			\item Let $A \in \bigcap\limits_{i \in I}\MA_i$. Then for each $i \in I$, $A \in \MA_i$. Hence for each $i \in I$, $A^c \in \MA_i$. Thus $A^c \in \bigcap\limits_{i \in I}\MA_i$. 
			\item Let $(A_n)_{n \in \N} \subset \bigcap\limits_{i \in I}\MA_i$. Then for each $i \in I$, $(A_n)_{n \in \N} \subset \MA_i$. Thus for each $i \in I$, $\bigcup\limits_{n \in \N}A_n \in \MA_i$. So $\bigcup\limits_{n \in \N}A_n \in \bigcap\limits_{i \in I}\MA_i$.
		\end{enumerate}
	\end{proof}
	
	\begin{defn} \ld{00000} 
		Let $X$ be a set and $\MC \subset \MP(X)$. Put $$\MS = \{\MA \subset \MP(X): \MA \text{ is a }\sig\text{-algebra on }X \text{ and } \MC \subset \ML\}$$ We define the \textbf{$\sig$-algebra generated by $\MC$} on $X$, $\sig(\MC)$, by $$\sig(\MC) = \bigcap_{\MA \in \MS} \MA $$
	\end{defn}
	
	\begin{note}
		Let $X$ be a set, $\MC \subset \MP(X)$ and $\MA$ a $\sig$-alg on $X$. By definition, if $\MC \subset \MA$, then $\sig(\MC) \subset \MA$.
	\end{note}
	
	\begin{note}
		Let $X$ be a set, $\MT$ an ordered set and $(\MA_t)_{t\in \MT}$ a collection of $\sig$-algebras on $X$. Suppose that for each $s,t \in \MT$, if $s \leq t$, then $\MA_s \subset \MA_t$. If there exists $t \in \MT$ such that $\MA_t = \bigcup\limits_{t \in \MT}\MA_t$, then $\bigcup\limits_{t \in \MT}\MA_t$ is a $\sig$-algebra on $X$. So if $\MT$ is finite or if $(\MA_t)_{t \in \MT}$ termintates, the union is $\sig$-algebra.
	\end{note}
	
	\begin{defn} \ld{00000} 
		Let $(X,\MT)$ be a topological space. We define the \textbf{Borel $\sig$-algebra} on $X$, $\MB(X)$, to be  
		$$\MB(X) = \sig(\MT)$$  
		Let $E \subset X$. Then $E$ is said to be \textbf{Borel} if $E \subset \MB(X)$.
	\end{defn}
	
	\begin{ex} \lex{00000} 
		The Borel $\sigma$-algebra on $\R$ with the standard topology is given by 
		\[
		\MB(\R) =
		\begin{cases}
			\sig(\{(a,b]:a,b \in \R \text{ and } a<b\}) \\
			\sig(\{[a,b]:a,b \in \R \text{ and } a<b\}) \\
			\sig(\{[a,b):a,b \in \R \text{ and } a<b\}) \\
			\sig(\{(a,b):a,b \in \R \text{ and } a<b\}) \\
		\end{cases}
		\]
	\end{ex}
	
	\begin{proof}
		Define 
		\begin{enumerate}
			\item $\MC_{lo} = \{(a,b]:a,b \in \R \text{ and } a<b\}$\\
			\item $\MC_{c} = \{[a,b]:a,b \in \R \text{ and } a<b\}$\\
			\item $\MC_{ro} = \{[a,b):a,b \in \R \text{ and } a<b\}$\\
			\item $\MC_{o} = \{(a,b):a,b \in \R \text{ and } a<b\}$\\
		\end{enumerate} 
		Recall that for each open set $A \subset \R$, there exist $(a_i)_{n \in \N}, (b_i)_{i \in \N} \subset \R$ such that for each $i \in \N$, $a_i < b_i$, for each $i,j \in \N$, if $i \neq j$, then $(a_i,b_i) \cap (a_j, b_j) = \varnothing$ and $A = \bigcup\limits_{i \in \N}(a_i, b_i)$. This implies that $\MB(\R) = \sig(\MC_o)$. \vspace{2mm}\\
		Now, let $a,b \in \R$. Suppose that $a<b$. Then 
		\begin{enumerate}
			\item $[a,b] = \bigcap\limits_{n \in \N}(a- \frac{1}{n}, b]$, so $\sig(\MC_{c}) \subset \sig(\MC_{lo})$\\
			\item $[a,b) = \bigcup\limits_{n \in \N} [a,b-\frac{1}{n}]$, so $\sig(\MC_{ro}) \subset \sig(\MC_{c})$ \\
			\item $(a,b) = \bigcup\limits_{n \in \N} [a+\frac{1}{n},b)$, so $\sig(\MC_{o}) \subset \sig(\MC_{ro})$\\
			\item $(a,b] = \bigcap\limits_{n \in \N} (a,b+\frac{1}{n})$, so $\sig(\MC_{lo}) \subset \sig(\MC_{o})$\\
		\end{enumerate}
		Hence $\MB(\R) = \sig(\MC_o) = \sig(\MC_{ro}) = \sig(\MC_{c}) = \sig(\MC_{lo}) = \sig(\MC_{o})$. 
	\end{proof}

	\begin{ex}
		Let $(X, \MT)$ be a topologcial space and $\ME \subset \MT$ a basis for $\MT$. If $\ME$ is countable, then $\MB(X) = \sig(\ME)$.  
	\end{ex}

	\begin{proof}
		Since $\ME \subset \MT$, 
		\begin{align*}
			\sig(\ME)
			& \subset \sig(\MT) \\
			& = \MB(X)
		\end{align*}
		Let $U \in \MT$. Since $\ME$ is a countable basis, there exists $\MC_U \subset \ME$ such that $\MC_U$ is countable and $U = \bigcup\limits_{C \in \MC_U}C$. Hence $U \in \sig(\ME)$. Since $U \in \MT$ is arbitary, $\MT \subset \sig(\ME)$. Thus 
		\begin{align*}
			\MB(X) 
			& = \sig(\MT) \\
			& \subset \sig(\ME)
		\end{align*} 
		Therefore $\MB(X) = \sig(\ME)$.
	\end{proof}
	
	\begin{ex} \lex{00000} 
		Let $X$ be a set. Define $\MA = \{A \in \MA: A \text{ is countable or }A^c  \text{ is countable}\}$. Then $\MA$ is a $\sig$-algebra on $X$.
	\end{ex}
	
	\begin{proof}\
		\begin{enumerate}
			\item Since $X^c = \varnothing$ is countable, $X \in \MA$.
			\item Let $A \in \MA$. Suppose that $A^c$ is  uncountable. Then by assumption, $A = (A^c)^c$ is countable. Hence $A^c \in \MA$.
			\item Let $(A_n)_{n \in \N} \subset \MA$. Then for each $n \in \N$, $A_n$ is countable or $A_n^c$ is countable. Suppose that $\bigcup\limits_{n \in \N}A_n$ is uncountable. Then there exists $N \in \N$ such that $A_N$ is uncountable. Hence $A_N^c$ is countable. Thus 
			\begin{align*}
				\bigg(\bigcup_{n \in \N}A_n \bigg)^c 
				&= \bigcap_{n \in \N}A_n^c \\
				& \subset A_N^c 
			\end{align*}
			So $\bigg(\bigcup\limits_{n \in \N}A_n \bigg)^c $ is countable and $\bigcup\limits_{n \in \N}A_n \in \MA$. 
		\end{enumerate}
	\end{proof}


	\begin{defn} \ld{00000} 
		Let $X$ be a set and $\MA$ be a $\sig$-algebra on $X$. Then $(X, \MA)$ is called a \textbf{measurable space}.
	\end{defn}






















	\newpage
	\subsection{Measurable Functions}
	
	\begin{defn} \ld{00000} 
		Let $(X,\MA)$ and $(Y, \MB)$ be measurable spaces and $f:X \rightarrow Y$. Then $f$ is said to be \textbf{$(\MA,\MB)$-measurable} if for each $B \in \MB$, $f^{-1}(B) \in \MA$. When $(Y, \MB) = (\R, \MB(\R))$ we say that $f$ is $\MA$-\textbf{measurbale}. If $(Y,\MB) = (\R, \MB(\R))$ and $(X,\MA) = (\R, \MB(\R))$ or $(\R, \ML)$, then we say that $f$ is \textbf{Borel measurable} or \textbf{Lebsgue measurbale} respectively.
	\end{defn}
	
	\begin{defn} \ld{00000} 
		Let $(X, \MA)$ and $(Y, \MB)$ be measurable spaces. Define 
		\begin{itemize}
			\item $L^{+}(X, \MA) = \{f:X \rightarrow \RG : f \text{ is measurable}\}$
			\item $L^0(X, \MA) = \{f:X \rightarrow \C : f \text{ is measurable}\}$ 
		\end{itemize}
	\end{defn}
	
	\begin{defn}
		Let $(X, \MA)$ and $(Y, \MB)$ be measurable spaces and $\phi: X \rightarrow Y$. Then $\phi$ is said to be a \textbf{isomorphism} if 
		\begin{enumerate}
			\item $\phi$ is a bijection
			\item $\phi$ is $(\MA, \MB)$-measurable and $\phi^{-1}$ is $(\MB, \MA)$-measurable
		\end{enumerate}
	\end{defn}

	\begin{defn}
		Let $(X, \MA)$ and $(Y, \MB)$ be measurable spaces. Then $(X, \MA)$ and $(Y, \MB)$ are said to be \textbf{isomorphic} if there exists $\phi:X \rightarrow Y$ such that $\phi$ is an isomorphism.
	\end{defn}
	
	\begin{defn}
		Let $(X, \MA)$ and $(Y,\MB)$ be measurable spaces and $f: X \rightarrow Y$. We define the 
		\begin{enumerate}
			\item \textbf{pushforward of $\MA$}, denoted $f_*\MA$, by 
			$$f_*\MA = \{B \subset Y: f^{-1}(B) \in \MA\}$$ 
			\item  \textbf{pullback of $\MB$}, denoted $f^*\MB$, by  
			$$f^*\MB = \{f^{-1}(B):  B \in \MB \}$$
		\end{enumerate}
	\end{defn}
	
	\begin{note}
		It is also common to write $\sig(f)$ or $f^{-1}(\MB)$ in place of $f^*\MB$.
	\end{note}	
	
	\begin{ex} \lex{00000} 
		Let $(X,\MA), (Y,\MB)$ be measurable spaces and $f: X \rightarrow Y$. Then 
		\begin{enumerate}
			\item $f_*\MA$ is a $\sig$-algebra on $Y$
			\item $f^*\MB$ is a $\sig$-algebra on $X$
		\end{enumerate}
	\end{ex}
	
	\begin{proof}\
		\begin{enumerate}
			\item 
			\begin{itemize}
				\item Since $f^{-1}(Y) = X \in \MA$, $Y \in f_*\MA$ and $f_*\MA \neq \varnothing$. 
				\item Let $B \in f_*\MA$. Then $f^{-1}(B) \in \MA$. Hence $$f^{-1}(B^c) = (f^{-1}(B))^c \in \MA$$ Thus $B^c \in f_*\MA$. 
				\item Now, let $(B_n)_{n \in \N} \subset f_*\MA$. Then for each $n \in \N$, $f^{-1}(B_n) \in \MA$. Thus $$f^{-1}\bigg(\bigcup_{n \in \N} B_n \bigg) = \bigcup_{n \in \N} f^{-1}(B_n) \in \MA$$ Hence $\bigcup\limits_{n \in \N} B_n \in f_*\MA$.
			\end{itemize}
			\item Similar to (1).
		\end{enumerate}
	\end{proof}
	
	\begin{ex} \lex{00000} 
		Let $(X,\MA)$ and $(Y, \MB)$ be measurable spaces. Suppose that there exists $\ME \subset Y$ such that $\sig(\ME) = \MB$. Let $f:X \rightarrow Y$. Then $f$ is $\MA$-$\MB$ measurable iff for each $B \in \ME$, $f^{-1}(B) \in \MA$.
	\end{ex}
	
	\begin{proof}
		By definition, if $f$ is $\MA$-$\MB$ measurable, then for each $B \in \ME$, $f^{-1}(B) \in \MA$. Conversely, suppose that for each $B \in \ME$, $f^{-1}(B) \in \MA$. The previous exercise tells us that $f_*\MA$ is a $\sig$-algebra on $Y$. Since $\ME \subset f_*\MA$, we have that $\MB = \sig(\ME) \subset f_*\MA$. So $f$ is $\MA$-$\MB$ measurable.
	\end{proof}
	
	\begin{ex} \lex{00000} 
		Let $X,Y$ be sets, $f:X \rightarrow Y$ and $\ME \subset \MP(Y)$. Then $\sig(f^{-1}(\ME)) = f^{-1}(\sig(\ME))$. 
	\end{ex}
	
	\begin{proof}
		Clealy $f^{-1}(\ME) \subset f^{-1}(\sig(\ME))$. Since $f^{-1}(\sig(\ME))$ is a $\sig$-algebra, we have that $\sig(f^{-1}(\ME)) \subset f^{-1}(\sig(\ME))$. Since $f^{-1}(\ME) \subset f^{-1}(\sig(\ME))$, the previous exercise tells us that $f$ is $f^{-1}(\sig(\ME))$-$\sig(\ME)$ measurable. Then $f^{-1}(\sig(\ME)) \subset f^{-1}(\sig(\ME))$. So $\sig(f^{-1}(\ME)) = f^{-1}(\sig(\ME))$.  \\
		\textbf{FINISH!!!}
	\end{proof}
	
	\begin{defn} \ld{}
		Let $X$ be a set, $(Y_{\al}, \MB_{\al})_{\al \in A}$ a collection of measurable spaces and $\MF \in \prod \limits_{\al \in A}Y_{\al}^X$ (i.e. $\MF = (f_{\al})_{\al \in A}$ where for each $\al \in A$, $f_{\al}:X \rightarrow Y_{\al}$). We define the \textbf{initial $\sig$-algebra generated by $\MF$} on $X$, denoted $\sig_X(\MF)$, by 
		\begin{align*}
			\sig_X(\MF) 
			&= \sig(\{f_{\al}^{-1}(B): B \in \MB_{\al} \text{ and } \al \in A \})
		\end{align*}	 
	\end{defn}
	
	\begin{note}
		If $\MF = \{f\}$, then $\sig_X(\MF) = f^*\MB$.
	\end{note}
	
	\begin{note}
		Essentially, $\sig_X(\MF)$ is the smallest $\sig$-algebra on $X$ such that for each $\al \in A$, $f_{\al}:X \rightarrow Y_{\al}$ is measurable. 
	\end{note}

	\begin{ex}
		Let $(Y_{\al}, \MB_{\al})_{\al \in A}$ be a a collection of measurable spaces, $X$ a set, $(Z, \MC)$ a measurable space, $\MF = (f_{\al})_{\al \in A} \in \prod \limits_{\al \in A}Y_{\al}^X$ and $g: Z \rightarrow X$. Then $g$ is $\MC$-$\tau_X(\MF)$ measurable iff for each $\al \in A$, $f_{\al} \circ g$ is $\MC$-$\MB_{\al}$ measurable:
		\[ \begin{tikzcd}
			Y_{\al}	
			& X  \arrow[l, "f_{\al}"'] \\
			& Z \arrow[ul, "g \circ f_{\al}"]  \arrow[u, "g"']
		\end{tikzcd}
		\]
	\end{ex}
	
	\begin{proof}
		If $g$ is $\MC$-$\tau_X(\MF)$ measurable, then clearly for each $\al \in A$, $ f_{\al} \circ g$ is $\MC$-$\MB_{\al}$ measurable. \\
		Conversely, suppose that for each $\al \in A$, $f_{\al} \circ g$ is $\MC$-$\MB_{\al}$ measurable. Let $\al \in A$ and $V \in \MB_{\al}$. Measurability implies that,
		\begin{align*}
			g^{-1}(f_{\al}^{-1}(V)) 
			& = (f_{\al} \circ g)^{-1}(V) \\
			& \in \MC
		\end{align*}
		Since $\al \in A$ and $V \in \MB_{\al}$ are arbitrary, we have that for each $\al \in A$ and $V \in \MB_{\al}$, $g^{-1}(f_{\al}^{-1}(V)) \in \MC$. Since $\tau_X(\MF) = \tau(\{f_{\al}^{-1}(V): \al \in A \text{ and } V \in \MB_{\al})$, a previous exercise implies that $g$ is $\MC$-$\tau_X(\MF)$  measurable.
	\end{proof}
	
	\begin{defn} \ld{}
		Let $(X_{\al}, \MA_{\al})_{\al \in A}$ be a a collection of measurable spaces, $Y$ a set and $\MF \in \prod \limits_{\al \in A}Y^{X^{\al}}$ (i.e. $\MF = (f_{\al})_{\al \in A}$ where for each $\al \in A$, $f_{\al}:X_{\al} \rightarrow Y$). We define the \textbf{final $\sig$-algebra generated by $\MF$} on $X$, denoted $\sig_Y(\MF)$, by 
		\begin{align*}
			\sig_Y(\MF) 
			&= \sig(\{V \subset Y: \text{ for each $\al \in A$, $f_{\al}^{-1}(V) \in \MA_{\al}$}\})
		\end{align*}	 
	\end{defn}
	
	\begin{note}
		If $\MF = \{f\}$, then $\sig_Y(\MF) = f_*\MA$.
	\end{note}
	
	\begin{note}
		Essentially, $\sig_X(\MF)$ is the largest $\sig$-algebra on $X$ such that for each $\al \in A$, $f_{\al}:X_{\al} \rightarrow Y$ is measurable. 
	\end{note}
	
	\begin{ex}
		Let $(X_{\al}, \MA_{\al})_{\al \in A}$ be a a collection of measurable spaces, $Y$ a set, $(Z, \MC)$ a measurable space, $\MF = (f_{\al})_{\al \in A} \in \prod \limits_{\al \in A}Y^{X_{\al}}$ and $g: Y \rightarrow Z$. Then $g$ is $\tau_Y(\MF)$-$\MC$ measurable iff for each $\al \in A$, $g \circ f_{\al}$ is $X_{\al}$-$\MC$ measurable, i.e. for each $\al \in A$, the following diagram commutes in the category of measurable spaces: 
		\[ \begin{tikzcd}
			X_{\al} \arrow[r, "f_{\al}"] \arrow[dr, "g \circ f_{\al}"'] 	
			& Y  \arrow[d, "g"] \\
			& Z 
		\end{tikzcd}
		\]
	\end{ex}
	
	\begin{proof}
		If $g$ is $\tau_Y(\MF)$-$\MC$ measurable, then clearly for each $\al \in A$, $g \circ f_{\al}$ is $X_{\al}$-$\MC$ measurable. \\
		Conversely, suppose that for each $\al \in A$, $g \circ f_{\al}$ is $X_{\al}$-$\MC$ measurable. Let $V \in \MC$. Measurability implies that for each $\al \in A$, $f_{\al}^{-1}(g^{-1}(V)) \in \MA_{\al}$. By definition, $g^{-1}(V) \in \tau_Y(\MF)$. So $g$ is $\tau_Y(\MF)$-$\MC$ measurable.
	\end{proof}
	
	\begin{ex} \lex{00000} 
		Let $(X_1,\MT_1), (X_2,\MT_2)$ be topological spaces and $f: X \rightarrow Y$. If $f$ is continuous, then $f$ is $\MB(X)$-$\MB(Y)$ measurable.
	\end{ex}
	
	\begin{proof}
		Recall that $\MB(Y) = \sig(\MT_2)$ and continuity tells us that for each $U \in \MT_2$, $f^{-1}(U) \in \MT_1 \subset \MB(X)$. 
	\end{proof}
	
	\begin{defn} \ld{00000} 
		Let $X$ be a set and $f:X \rightarrow \C$. Then $f$ is said to be \textbf{simple} if $f(X)$ is finite.
	\end{defn}
	
	\begin{defn} \ld{00000} 
		Let $(X,\MA)$ be a measurable space. We define $S^+(X,\MA) = \{f:X \rightarrow \Rg: f \text{ is simple, measurable}\}$ and $S(X,\MA) = \{f: X \rightarrow \C: f \text{ is simple, measurable}\}$
	\end{defn}
	
	\begin{thm}
		Let $(X, \MA)$ be a measurable space. Then 
		\begin{enumerate}
			\item If $f: X \rightarrow \RG$ is measurable, then there exists a sequence $(\phi_n)_{n \in \N} \subset S^+$ such that for each $n \in \N$, $\phi_n \leq \phi_{n+1} \leq f$ and $\phi_n \rightarrow f$ pointwise and $\phi_n \rightarrow f$ uniformly on any set on which $f$ is bounded.
			
			\item If $f: X \rightarrow \C$ is measurable, then there exists a sequence $(\phi_n)_{n \in \N} \subset S$ such that for each $n \in \N$, $|\phi_n| \leq |\phi_{n+1}| \leq |f|$ and $\phi_n \rightarrow f$ pointwise and $\phi_n \rightarrow f$ uniformly on any set on which $f$ is bounded.
		\end{enumerate}
	\end{thm}
	
	\begin{ex}
		Let $(X, \MA)$ and $(Y, \MB)$ be measurable spaces and $f: X \rightarrow Y$. If $f$ is $\MA$-$\MB$ measurable iff $f$ is $\MA$-$\MB \cap f(X)$ measurable.
	\end{ex}	
	
	\begin{proof}
		Suppose that $f$ is $\MA$-$\MB$ measurable. Let $E \in \MB \cap f(X)$. Then there exists $B \in \MB$ such that $E = B \cap f(X)$. Then 
		\begin{align*}
			f^{-1}(E)
			&= f^{-1}(B \cap f(X)) \\
			&= f^{-1}(B) \cap f^{-1}(f(X)) \\
			&= f^{-1}(B) \cap X \\
			&= f^{-1}(B) \\
			& \in \MA
		\end{align*}
		Conversely, suppose that $f$ is  $\MA$-$\MB \cap f(X)$ measurable. Let $B \in \MB$. Then as before, 
		\begin{align*}
			f^{-1}(B) 
			&= f^{-1}(B \cap f(X))  \\
			& \in \MA 
		\end{align*}
	\end{proof}
	
	\begin{ex}\textbf{Doob-Dynkin Lemma:} \\
		Let $(X_1, \MA_1)$, $(X_2, \MA_2)$ and $(X_3, \MA_3)$ be measurable spaces and $f: X_1 \rightarrow X_2$ and $g:X_1 \rightarrow X_3$. Suppose that $f$ is surjective and $\MA_1$-$\MA_2$ measurable and $g$ is $\MA_1$-$\MA_3$ measurable and for each $t \in X_3$, $\{t\} \in \MA_3$. Then $g$ is $f^*\MA_2$-$\MA_3$ measurable iff there exists a unique $\phi: X_2 \rightarrow X_3$ such that $\phi$ is $\MA_2$-$\MA_3$ measurable and $g = \phi \circ f$. \\
		\textbf{Hint:} For each $t \in X_3$, set $A_t = g^{-1}(\{t\}) \in f^* \MA_2$ and choose $B_t \in \MA_2$ such that $A_t = f^{-1}(B_t)$. Set $\phi(y) = t$ for $y \in B_t \cap f(X_1)$ and $t \in g(X_1)$.
	\end{ex}
	
	\begin{proof}
		Suppose that there exists a unique $\phi: X_2 \rightarrow X_3$ such that $\phi$ is $\MA_2$ - $\MA_3$ measurable and $g = \phi \circ f$. Since $f$ is $f^* \MA_2$ - $\MA_2$ measurable, we have that $g = \phi \circ f$ is $f^*\MA_2$-$\MA_3$ measurable.  \\
		Conversely, suppose that $g$ is $f^*\MA_2$-$\MA_3$ measurable. \\
		\begin{itemize}
			\item \textbf{(Existence)} \\
			For each $t \in X_3$, set $A_t = g^{-1}(\{t\}) \in f^* \MA_2$ and choose $B_t \in \MA_2$ such that $A_t = f^{-1}(B_t)$. \\
			Note that 
			\begin{itemize}
				\item for each $t \in g(X_1)$, there exists $x \in A_t$ such that $g(x) = t$. Hence $f(x) \in B_t$.\\
				\item for $t_1, t_2 \in g(X_1)$, $t_1 \neq t_2$ implies that
				\begin{align*}
					f^{-1}(B_{t_1} \cap B_{t_2}) 
					&= A_{t_1} \cap A_{t_2} \\
					&= g^{-1}(\{t_1\} \cap \{t_2\}) \\
					&= \varnothing
				\end{align*}	 
				and since $f$ is surjective, 
				\begin{align*}
					B_{t_1} \cap  B_{t_2} 
					& = f(f^{-1}(B_{t_1} \cap  B_{t_2} )) \\
					&= f(\varnothing) \\
					&= \varnothing
				\end{align*}
				\item we have that 
				\begin{align*}
					f^{-1} \bigg( \bigcup_{t \in g(X_1)} B_t\bigg) 
					&=  \bigcup_{t \in g(X_1)} A_t \\
					&= \bigcup_{t \in g(X_1)} g^{-1}(\{t\}) \\
					&= g^{-1}(g(X_1)) \\
					&= X_1
				\end{align*}
				Since $f$ is surjective, we have that 
				\begin{align*}
					X_2
					&= f(X_1) \\
					&= f \bigg( f^{-1} \bigg( \bigcup_{t \in g(X_1)} B_t\bigg)  \bigg) \\
					&= \bigcup_{t \in g(X_1)} B_t
				\end{align*}
			\end{itemize}
			Therefore, 
			\begin{itemize}
				\item for each $t \in g(X_1)$, $B_t \neq \varnothing$
				\item $(A_t)_{t \in g(X_1)}$ is a partion of $X_1$
				\item $(B_t)_{t \in g(X_1)}$ is a partition of $X_2$\\
			\end{itemize}		
			Define $\phi:X_2 \rightarrow X_3$ by $\phi(y) = t$ for $t \in g(X_1)$ and $y \in B_t $. Then the previous observations imply that $\phi$ is well defined and $\phi(X_2) = g(X_1)$. Since for each $t \in g(X_1)$ and $x \in A_t$, $f(x) \in B_t$ and $g(x) = t$, we have that $\phi \circ f (x) = t = g(x)$. So $\phi \circ f = g$. \\ \\
			To show that $\phi$ is measurable, let $C \in \MA_3$. Choose $B \in \MA_2$ such that $g^{-1}(C) = f^{-1}(B)$.
			Let $y \in \phi^{-1}(C) \subset X_2$. Set $t = \phi(y) \in C$ and choose $x \in X_1$ such that $y = f(x)$. Since 
			\begin{align*}
				g(x) 
				&= \phi \circ f (x) \\
				&= \phi(y) \\
				&= t \\
				&\in C
			\end{align*}		
			$x \in g^{-1}(C) = f^{-1}(B)$. Therefore, $y = f(x) \in B$. So $\phi^{-1}(C) \subset B$. \\
			Let $y \in B$. Choose $x \in X_1$ such that $f(x) = y$. Then $x \in f^{-1}(B) = g^{-1}(C)$. So 
			\begin{align*}
				\phi(y) 
				&= \phi \circ f (x) \\
				&= g(x) \\
				&\in C
			\end{align*}	 
			and $y \in \phi^{-1}(C)$. So $B \subset \phi^{-1}(C)$. 
			Hence $\phi^{-1}(C) = B \in \MA_2$ and $\phi$ is $\MA_2$ - $\MA_3$ measurable.\\
			\item \textbf{(Uniqueness)} \\
			Let $\psi: X_2 \rightarrow X_3$. Suppose that $\psi$ is $\MA_2$-$\MA_3$ measurable and $g = \psi \circ f$. Let $y \in X_2$. Then there exists $x \in X_1$ such that $y = f(x)$. Then 
			\begin{align*}
				\psi(y) 
				&= \psi \circ f(x) \\
				&= g(x) \\
				&= \phi \circ f(x) \\
				&= \phi(y)
			\end{align*}
			So $\psi = \phi$.
		\end{itemize}
		
	\end{proof}
	
	\begin{ex}
		Let $(X_1, \MA_1)$, $(X_2, \MA_2)$ and $(X_3, \MA_3)$ be measurable spaces and $f: X_1 \rightarrow X_2$ and $g:X_1 \rightarrow X_3$. Suppose that $f$ is $\MA_1$-$\MA_2$ measurable and $g$ is $\MA_1$-$\MA_3$ measurable and for each $t \in X_3$, $\{t\} \in \MA_3$. Then $g$ is $f^*\MA_2$-$\MA_3$ measurable iff there exists a unique $\phi: f(X_1) \rightarrow X_3$ such that $\phi$ is $\MA_2 \cap f(X_1)$ - $\MA_3$ measurable and $g = \phi \circ f$. \\
	\end{ex}
	
	\begin{proof}
		A previous exercise implies that $f: X_1 \rightarrow f(X_1)$ is $\MA_1$ - $\MA_2 \cap f(X_1)$ measurable. Now apply the previous exercise. 
	\end{proof}


	
	
	
	
	
	
	
	
	
	
	
	
	
	
	
	
	
	
	
	
	
	
	
	
	
	\newpage
	\subsection{Subspace Sigma Algebras}
	\begin{defn} \ld{00000} 
		Let $X$ be a set, $\MC \subset \MP(X)$ and $E \subset X$. We define $\MC \cap E \subset \MP(X)$ by $$\MC \cap E = \{S \cap E: S \in \MC\}$$ 
	\end{defn}
	
	\begin{ex} \lex{00000} 
		Let $X$ be a set, $\MA$ a $\sig$-algebra on $X$ and $E \subset X$. Then 
		$\MA \cap E$ is a $\sig$-algebra on $E$. 
	\end{ex}
	
	\begin{proof}\
		\begin{enumerate}
			\item Clearly $\varnothing, E \in \MA \cap E$.
			\item Let $B \in \MA \cap E$. Then there exists $A \in \MA$ such that $B = A \cap E$. Since $A^c \in \MA$, we have that 
			\begin{align*}
				E \setminus B 
				&= E \cap (A \cap E)^c \\
				&= E \cap (A^c \cup E^c) \\
				&= (E \cap A^c) \cup (E \cap E^c) \\
				&= A^c \cap E \\
				&\in \MA \cap E
			\end{align*}
			\item Let $(B_n)_{n \in \N} \subset \MA \cap E$. Then for each $n \in \N$, there exists $A_n \in \MA$ such that $B_n = A_n \cap A$. So $\bigcup\limits_{n \in \N}A_n \in \MA$. Hence 
			\begin{align*}
				\bigcup_{n \in \N}(B_n) 
				&= \bigcup_{n \in \N}(A_n \cap E) \\
				&= \bigg( \bigcup_{n \in \N}A_n \bigg) \cap E \\
				& \in \MA \cap E
			\end{align*}
		\end{enumerate}
	\end{proof}
	
	\begin{ex} \lex{00000} 
		Let $X$ be a set, $\MC \subset \MP(X)$ and $A \subset X$. Let $\sig_A(\MC \cap A)$ be the $\sig$-algebra on $A$ generated by $\MC \cap A$. Define $$\MG = \{S \subset X: S \cap A \in \sig_A(\MC \cap A)\}$$ 
		Then $\MG$ is a $\sig$-algebra on $X$. 
	\end{ex}
	
	\begin{proof}\
		\begin{enumerate}
			\item Clearly $\varnothing, X \in \MG$. 
			\item Let $S \in \MG$. Then $S \cap A \in \sig_A(\MC \cap A)$. Hence 
			\begin{align*}
				S^c \cap A 
				&= A \setminus (S \cap A)  \\
				& \in \sig_A(\MC \cap A)
			\end{align*}				
			So $S^c \in \MG$. 
			\item Let $(S_n)_{n \in \N} \subset \MG$. Then for each $n \in \N$, $S_n \cap A \in \sig_A(\MC \cap A)$. Thus $$\bigg( \bigcup_{n \in \N} S_n \bigg) \cap A = \bigcup\limits_{n \in \N}(S_n \cap A) \in \sig_A(\MC \cap A)$$ Thus $\bigcup\limits_{n \in \N} S_n \in \MG$.
		\end{enumerate}
	\end{proof}
	
	\begin{ex} \lex{00000} 
		Let $X$ be a set, $\MC \subset \MP(X)$ and $A \subset X$. Then $$\sig(\MC) \cap A = \sig_A(\MC \cap A)$$
	\end{ex}
	
	\begin{proof}
		Clearly $\MC \cap A \subset \sig(\MC) \cap A$. A previous exercise tells us that $\sig(\MC) \cap A$ is a $\sig$-algebra on $A$. Thus $\sig_A(\MC \cap A) \subset \sig(\MC) \cap A$. \vspace{3mm}\\ Conversely, from the previous exercise, we have that $\MG = \{S \subset X: S \cap A \in \sig_A(\MC \cap A)\}$ is a $\sig$-algebra on $X$. Clearly $\MC \subset \MG$. Then $\sig(\MC) \subset \MG$. The definition of $\MG$ implies that $\sig(\MC) \cap A \subset \sig_A(\MC \cap A)$. Hence $\sig(\MC) \cap A = \sig_A(\MC \cap A)$.
	\end{proof}
























	\newpage
	\subsection{Product Sigma Algebras}
	
	\begin{defn}
		Let $(X_{\al}, \MA_{\al})_{\al \in A}$ be a collection of measurable spaces. We define the \textbf{product $\sig$-algebra} on $\prod_{\al \in A}X_{\al}$, denoted by $\bigotimes\limits_{\al \in A} \MA_{\al}$, by $$\bigotimes\limits_{\al \in A} \MA_{\al} = \sig(\pi_{\al}: \al \in A)$$
	\end{defn}

	\begin{ex}
		Let $(X_{\al}, \MA_{\al})_{\al \in A}$ be a collection of measurable spaces and for each $\al \in A$, $\ME_{\al} \subset \MA_{\al}$. Suppose that for each $\al \in A$, $\MA_{\al} = \sig(\ME_{\al})$. Then 
		$$\bigotimes\limits_{\al \in A} \MA_{\al} = \sig( \pi_{\al}^{-1}(E_{\al}): \al \in A \text{ and } E_{\al} \in \ME_{\al})$$ 
		\textbf{Hint:} set $\MG = \{\pi_{\al}^{-1}(E_{\al}): \al \in A \text{ and } E_{\al} \in \ME_{\al}\}$ and for $\al \in A$, consider the pushforward $\sig$-algebra on $X_{\al}$, ${\pi_{\al}}_* \sig( \MG)$
	\end{ex}

 	\begin{proof}
 		Set 
 		\begin{itemize}
 			\item $\MF = \{\pi_{\al}^{-1}(V_{\al}): \al \in A \text{ and } V_{\al} \in \MA_{\al}\}$ 
 			\item $\MG = \{\pi_{\al}^{-1}(E_{\al}): \al \in A \text{ and } E_{\al} \in \ME_{\al}\}$
 		\end{itemize}
 		Clearly, $\MG \subset \MF$. By definition, $\bigotimes\limits_{\al \in A} \MA_{\al} = \sig(\MF)$. Therefore, 
 		\begin{align*}
 			\sig(\MG) 
 			& \subset \sig(\MF) \\
 			& = \bigotimes\limits_{\al \in A} \MA_{\al}
 		\end{align*}
 		Let $\al \in A$. By definition, for each $V \subset X_{\al}$, $V \in {\pi_{\al}}_* \sig(\MG)$ iff $\pi_{\al}^{-1}(V) \in \sig(\MG)$. Thus $\ME_{\al} \subset \pi_{\al}^* \sig(\MG)$ which implies that 
 		\begin{align*}
 			\MA_{\al} 
 			& = \sig(\ME_{\al}) \\
 			& \subset \pi_{\al}^* \sig(\MG)
 		\end{align*}
 		Since $\al \in A$ is arbitrary, $\MF \subset \sig(\MG)$. Hence 
 		\begin{align*}
 			\bigotimes\limits_{\al \in A} \MA_{\al} 
 			& = \sig(\MF) \\
 			& \subset \sig(\MG)
 		\end{align*}
 		Thus $\sig(\MG) = \bigotimes\limits_{\al \in A} \MA_{\al}$.
 	\end{proof}
 
 	\begin{ex}
 		Let $(X_{\al}, \MA_{\al})_{\al \in A}$ be a collection of measurable spaces. Define $$\MB = \bigg \{\prod_{\al \in A}B_{\al}: \text{ for each $\al \in A$, } B_{\al} \in \MA_{\al} \bigg\}$$
 		If $A$ is countable, then $\bigotimes\limits_{\al \in A} \MA_{\al} = \sig(\MB)$. 
 	\end{ex}
 	
 	\begin{proof}
 		Suppose that $A$ is countable. Set $\MC = \{\pi_{\al}^{-1}(B_{\al}): \al \in A, B_{\al} \in \MA_{\al}\}$. By definition, $\bigotimes\limits_{\al \in A} \MA_{\al} = \sig(\MC)$. Let $\al \in A$ and $B_{\al} \in \MA_{\al}$. For $\be \in A$, set 
 		\[
 		C_{\be} = 
 		\begin{cases}
 			B_{\be} & \be = \al \\
 			X_{\be} & \be \neq \al
 		\end{cases}
 		\]
 		Then 
 		\begin{align*}
 			\pi_{\al}^{-1}(B_{\al}) 
 			& = \prod_{\be \in A} C_{\be} \\
 			& \in \MB
 		\end{align*}
 		So $\MC \subset \MB$ and 
 		\begin{align*}
 			\bigotimes\limits_{\al \in A} \MA_{\al}
 			& = \sig(\MC) \\
 			& \subset \sig(\MB)
 		\end{align*}
 		For each $\al \in A$, let $B_{\al} \in \MA_{\al}$. Since $A$ is countable, we have that 
 		\begin{align*}
 			\prod_{a \in A} B_{\al} 
 			& = \bigcap_{a \in A} \pi_{\al}^{-1}(B_{\al}) \\
 			& \in \sig(\MC) 
 		\end{align*}
 		Thus $\MB \subset \sig(\MC)$ and 
 		\begin{align*}
 			\sig(\MB) 
 			& \subset \sig(\MC) \\
 			& = \bigotimes\limits_{\al \in A} \MA_{\al}
 		\end{align*}
 		Hence $\sig(\MB) = \bigotimes\limits_{\al \in A} \MA_{\al}$. 
 	\end{proof}
 
 	\begin{ex}
 		Let $(X_{\al}, \MA_{\al})_{\al \in A}$ be a collection of measurable spaces and for each $\al \in A$, $\ME_{\al} \subset \MA_{\al}$. Suppose that for each $\al \in A$, $X_{\al} \in \ME_{\al}$ and $\MA_{\al} = \sig(\ME_{\al})$. Set 
 		$$\MB = \bigg \{ \prod\limits_{\al \in A} E_{\al}:  \text{for each $\al \in A$, } E_{\al} \in \ME_{\al} \bigg \}$$ 
 		If $A$ is countable, then $\bigotimes\limits_{\al \in A} \MA_{\al} = \sig(\MB)$.
 	\end{ex}
 
 	\begin{proof}
 		Suppose that $A$ is countable. Set $\MC =  \bigg \{ (\pi_{\al}^{-1}(E_{\al}): \al \in A \text{ and }E_{\al} \in \ME_{\al} \bigg \}$. A previous exercise implies that $\sig(\MC) = \bigotimes\limits_{\al \in A} \MA_{\al}$. Let $\al \in A$ and $E_{\al} \in \ME_{\al}$. For $\be \in A$, set 
 		\[
 		C_{\be} = 
 		\begin{cases}
 			E_{\be} & \be = \al \\
 			X_{\be} & \be \neq \al
 		\end{cases}
 		\]
 		Then for each $\beta \in A$, $C_{\be} \in \ME_{\be}$ and
 		\begin{align*}
 			\pi_{\al}^{-1}(E_{\al}) 
 			& = \prod_{\be \in A} C_{\be} \\
 			& \in \MB
 		\end{align*}
 		So $\MC \subset \MB$ and 
 		\begin{align*}
 			\bigotimes\limits_{\al \in A} \MA_{\al}
 			& = \sig(\MC) \\
 			& \subset \sig(\MB)
 		\end{align*}
 		For each $\al \in A$, let $E_{\al} \in \ME_{\al}$. Since $A$ is countable, we have that 
 		\begin{align*}
 			\prod_{a \in A} E_{\al} 
 			& = \bigcap_{a \in A} \pi_{\al}^{-1}(E_{\al}) \\
 			& \in \sig(\MC) 
 		\end{align*}
 		Thus $\MB \subset \sig(\MC)$ and 
 		\begin{align*}
 			\sig(\MB)
 			& \subset \sig(\MC) \\ 
 			& \subset \bigotimes\limits_{\al \in A} \MA_{\al} 
 		\end{align*}
 		Hence $\sig(\MB) = \bigotimes\limits_{\al \in A} \MA_{\al}$.
 	\end{proof}
 
 	\begin{ex}
 		Let $(X_{\al}, \MT_{\al})_{\al \in A}$ be a collection of topological spaces. Then 
 		\begin{enumerate}
 			\item $$\bigotimes\limits_{\al \in A}  \MB(X_{\al}) \subset \MB \bigg( \prod_{\al \in A} X_{\al} \bigg)$$
 			\item if $A$ is countable and for each $\al \in A$, $X_{\al}$ is second-countable, then $$\bigotimes\limits_{\al \in A}  \MB(X_{\al}) = \MB \bigg( \prod_{\al \in A} X_{\al} \bigg)$$
 		\end{enumerate}
 	\end{ex}
 
 	\begin{proof} Set $X = \prod_{j=1}^n X_j$ and denote the product topology on $X$ by $\MT_X$. 
 		\begin{enumerate}
 			\item By definition, $\MB(X) = \sig(\MT_X)$ and for each $\al \in A$, $X_{\al} \in \MT_{\al}$ and $\MB(X_{\al}) = \sig(\MT_{\al})$. Set 
 			$$\ME = \{\pi^{-1}_{\al}(E_{\al}): \al \in A \text{ and } E_{\al} \in \MT_{\al} \}$$ 
 			A previous exercise implies that $\bigotimes\limits_{\al \in A} \MB(X_{\al}) = \sig(\ME)$. Since $\ME \subset \MT_X$, we have that
 			\begin{align*}
 				\bigotimes_{\al \in A} \MB(X_{\al}) 
 				& = \sig(\ME) \\
 				& \subset \sig(\MT_X) \\
 				& = \MB(X)
 			\end{align*}
 			\item Suppose that $A$ is countable and for each $\al \in A$, $X_{\al}$, is second-countable. Then for each $\al \in A$, there exists $\MB_{\al} \subset \MT_{\al}$ such that $\MB_{\al}$ is a countable basis for $\MT_{\al}$. Set 
 			\begin{align*}
 				\MB = 
 				& \bigg \{\prod_{\al \in A} U_{\al}: \text{there exists $J \subset A$ such that $ \# J < \infty$, } \\
 				& \text{ for each $\al \in J$, $U_{\al} \in \MB_{\al}$ and for each $\al \in J^c$, $U_{\al} = X_{\al}$ } \bigg \}
 			\end{align*} 
 			Since $A$ is countable, $\MB$ is a countable basis for $\MT_X$. An exercise in the section on $\sig$-algebras implies that $\MB(X) = \sig(\MB)$. The previous exercise implies that $\MB \subset \bigotimes\limits_{\al \in A} \MB(X_{\al})$. Hence
 			\begin{align*}
 				\MB(X) 
 				& = \sig(\MB) \\
 				& \subset \bigotimes_{\al \in A} \MB(X_{\al}) 
 			\end{align*}
 		\end{enumerate}
 	\end{proof}
	
	\begin{ex}
		Let $(X, \MA)$ be a measurable space, $(Y_{\al}, \MA_{\al})_{\al \in A}$ a collection of measurable spaces and $f: X \rightarrow \prod_{\al \in A}Y_{\al}$. Then $f$ is $(\MA, \bigotimes\limits_{\al \in A} \MA_{\al})$-measurable iff for each $\al \in A$, $\pi_{\al} \circ f$ is $(\MA, \MA_{\al})$-measurable.
	\end{ex}
	
	\begin{proof}
		Immediate by a previous exercise about the initial $\sig$-algebra.
	\end{proof}
	
	\begin{ex}
		Let $(X_{\al}, \MA_{\al})_{\al \in A}$ and $(Y_{\al}, \MB_{\al})_{\al \in A}$ be collections of measurable spaces and $(f_{\al})_{\al \in A} \in \prod\limits_{\al \in A} Y_{\al}^{X_{\al}}$, i.e. for each $\al \in A$, $f_{\al}:X_{\al} \rightarrow Y_{\al}$. Set $X = \prod\limits_{\al \in A} X_{\al}$ and $Y = \prod\limits_{\al \in A}Y_{\al}$. Define $f: X \rightarrow Y$ by $(f(x))_{\al} = f_{\al}(x_{\al})$. If for each $\al \in A$, $f_{\al}$ is $(\MA_{\al}, \MB_{\al})$-measurable, then $f$ is $(\bigotimes\limits_{\al \in A} \MA_{\al}, \bigotimes\limits_{\al \in A} \MB_{\al})$-measurable.
	\end{ex}
	
	\begin{proof} Denote the $\al$-th projection maps on $X$ and $Y$ by $\pi^X_{\al}$ and $\pi^Y_{\al}$ respectively. Let $\al \in A$ and $x \in X$. Then
		\begin{align*}
			\pi^Y_{\al} \circ f(x) 
			& = (f(x))_{\al} \\
			& = f_{\al}(x_{\al}) \\
			& = f_{\al} \circ \pi^X_{\al}(x) 
		\end{align*}
		Since $\al \in A$ and $x \in X$ are arbitrary, for each $\al \in A$, $\pi^Y_{\al} \circ f = f_{\al} \circ \pi^X_{\al}$. Suppose that for each $\al \in A$, $f_{\al}$ is $(\MA_{\al}, \MB_{\al})$-measurable. Let $\al \in A$. Then $f_{\al} \circ \pi^X_{\al}$ is $(\MA_{\al}, \MB_{\al})$-measurable. Hence $\pi^Y_{\al} \circ f$ is $(\MA_{\al}, \MB_{\al})$-measurable. Since $\al \in A$ is arbitrary, the previous exercise implies that $f$ is $(\bigotimes\limits_{\al \in A} \MA_{\al}, \bigotimes\limits_{\al \in A} \MB_{\al})$-measurable.
	\end{proof}
	
	\begin{ex}
		Let $X_1, X_2,Y_1,Y_2$ be topological spaces and $f_1:X_1 \rightarrow Y_1$, $f_2:X_2 \rightarrow Y_2$. If $f_1$ and $f_2$ are open, then $f_1 \times f_1$ is open.
	\end{ex}
	
	\begin{proof}
		Let $A_1 \subset X_1, A_2 \subset X_2$ be open. Then $f_1 \times f_2(A_1 \times A_2) = f_1(A_1) \times f_2(A_2)$ which is open in $Y_1 \times Y_2$. Since $\MB = \{A_1 \times A_2:  \text{$A_1 \subset X_1$ and $A_2 \subset X_2$ are open} \}$ is a basis for the product topology on $X_1 \times X_2$, an exercise in the section on continuous maps implies that $f_1 \times f_2$ is open.
	\end{proof}
	
	\begin{ex}
		Let $X$ and $Y$ be topological spaces and $U \subset X \times Y$ open. Then for each $(x_0,  y_0) \in U$, $U^{x_0}$ and $U^{y_0}$ are open.
	\end{ex}
	
	\begin{proof}
		Let $(x_0, y_0) \in U$. Define $\phi: X \rightarrow X \times Y$ by $\phi(x) = (x, y_0)$. Since $\pi_X \circ \phi = \id_X$ and $\pi_Y \circ \phi$ is constant, $\pi_X \circ \phi$ and $\pi_Y \circ \phi$ are continous. Therefore, $\phi$ is continuous. Then $U^{y_0}$ is open since $U$ is open and $\phi^{-1}(U) = U^{y_0}$. Similarly, $U_{x_0}$ is open.
	\end{proof}
	
	\begin{ex}
		Let $X$, $Y$ and $Z$ be topological spaces, $U \subset X \times Y$ open and $f: U \rightarrow Z$. Equip $U$ with the subspace topology. Suppose that $f$ is continuous. Let $(x_0, y_0) \in U$. Equip $U_{x_0}$ and $U^{y_0}$ with the subspace topology. Then $f_{x_0}:U_{x_0} \rightarrow Z$ and $f^{y_0}: U^{y_0} \rightarrow Z$ are continuous.
	\end{ex}
	
	\begin{proof}
		Let $(x_0, y_0) \in U$. Let $V \subset Z$. Suppose that $V$ is open. Continuity of $f$ implies that $f^{-1}(V)$ is open in $U$. Since $U$ is open in $X \times Y$, $f^{-1}(V)$ is open in $X \times Y$. A previous exercise in the section on product sets implies that $(f^{y_0})^{-1}(V) = (f^{-1}(V))^{y_0}$. The previous exercise implies that $(f^{-1}(V))^{y_0}$ is open in $X$. So $(f^{y_0})^{-1}(V)$ is open in $X$. Since $(f^{y_0})^{-1}(V) \subset U^{y_0}$, $(f^{y_0})^{-1}(V)$ is open in $U^{y^0}$. Thus $f^{y_0}: U^{y_0} \rightarrow Z$ is continuous. Similarly, $f_{x_0}: U_{x_0} \rightarrow Z$ is continuous.
	\end{proof}


	
	
	
	
	
	
	
	
	
	
	
	
	
	
	
	
	
	
	
	
	
	
	\newpage
	\subsection{Quotient Sigma Algebras}
	\begin{defn}
		Let $X,Y$ be sets, $\sim$ an equivalence relation on $X$ and $f: X \rightarrow Y$. Then $f$ is said to be \textbf{invariant under $\sim$} if for each $a,b \in X$, $\bar{a} = \bar{b}$ implies that $f(a) = f(b)$. 
	\end{defn}
	
	\begin{ex} \lex{34008}
		Let $(X, \MA)$, $(Y, \MB)$ be topological spaces, $\sim$ an eqivalence relation on $X$, $\pi:X \rightarrow X/\sim$ the projection map and $f:X \rightarrow Y$ measurable. If $f$ is invariant under $\sim$, then there exists a unique $\bar{f}:X / {\sim} \rightarrow Y$ such that 
		\begin{enumerate}
			\item $\bar{f} \circ \pi = f$
			\item $\bar{f}$ is $\MA$-$\pi_*\MA$ measurable 
		\end{enumerate}
	\end{ex}
	
	\begin{proof}
		Suppose that $f$ is invariant under $\sim$. Define $\bar{f}: X / {\sim} \rightarrow Y$ by $\bar{f}(\bar{x}) = f(x)$. By assumption, for each $a, b \in X$, $\bar{a} = \bar{b}$ implies that $f(a) = f(b)$. Thus $\bar{f}$ is well defined. By construction, $f = \bar{f} \circ \pi$. Let $V \in \MB$. Measurability of $f$ implies that $f^{-1}(V) \in \MA$. Since 
		\begin{align*}
			f^{-1}(V)
			&= \pi^{-1}(\bar{f}^{-1}(V)) \\
			& \in \MA
		\end{align*}
		by definition of $\pi_*\MA$, $\bar{f}^{-1}(V) \in \pi_*\MA$. So $\bar{f}$ is $\MA$-$\pi_*\MA$ measurable.
	\end{proof}
	
	
	
	
	
	
	
	
	
	
	
	
	
	
	
	
	
	
	
	
	
	
	\newpage
	\subsection{Dynkin's Lemma}
	\begin{defn}
		Let $X$ be a set and $\MP \subset \MP(X)$. Then $\MP$ is said to be a \textbf{$\pi$-system} on $X$ if for each $A,B \in \MP$, $A \cap B \in \MP$.
	\end{defn}
	
	\begin{defn}
		Let $X$ be a set and $\ML \subset \MP(X)$. Then $\ML$ is said to be a \textbf{$\lam$-system} on $X$ if 
		\begin{enumerate}
			\item $\ML \neq \varnothing$
			\item for each $A \in \ML$, $A^c \in \ML$
			\item for each $(A_n)_{n \in \N} \subset \ML$, if $(A_n)_{n \in \N}$ is disjoint, then $\bigcup\limits_{n \in \N}A_n \in \ML$
		\end{enumerate}
	\end{defn}
	
	\begin{ex}
		Let $X$ be a set and $\ML$ a $\lam$-system on $X$. Then 
		\begin{enumerate}
			\item $X, \varnothing \in \ML$
		\end{enumerate} 
	\end{ex}
	
	\begin{proof}
		Straightforward.
	\end{proof}
	
	\begin{defn}
		Let $X$ be a set and $\MC \subset \MP(X)$. Put $$\MS = \{\ML \subset \MP(X): \ML \text{ is a }\lam\text{-system on }\Om \text{ and } \MC \subset \ML\}$$ We define the \textbf{$\lam$-system on $X$ generated by $\MC$}, $\lam(\MC)$, to be $$\lam(\MC) = \bigcap_{\ML \in \MS}\ML$$
	\end{defn}
	
	\begin{ex}
		Let $X$ be a set and $\MA \subset \MP(X)$. If $\MA$ is a $\lam$-system and $\MA$ is a $\pi$-system, then $\MA$ is a $\sig$-algebra.
	\end{ex}
	
	\begin{proof}
		Suppose that $\MA$ is a $\lam$-system and $\MA$ is a $\pi$-system. Then we need only verify the third axiom in the definition of a $\sig$-algebra. Let $(A_n)_{n \in \N} \subset \MA$. Define $B_1 = A_1$ and for $n \geq 2$, define $B_n = A_n \cap \bigg( \bigcup\limits_{k=1}^{n-1}A_k \bigg)^c = A_n \cap \bigg( \bigcap\limits_{k=1}^{n-1}A_k^c \bigg) \in \MA$. Then $(B_n)_{n \in \N}$ is disjoint and therefore $\bigcup\limits_{n \in \N}A_n = \bigcup\limits_{n \in \N}B_n \in \MA$.
	\end{proof}
	
	\begin{thm} \textbf{Dynkin's Lemma:} \\
		Let $X$ be a set, $\MP$ be a $\pi$-system on $X$ and $\ML$ a $\lam$-system on $X$. Then
		\begin{enumerate}
			\item $\MP \subset \ML$ implies that $\sig(\MP) \subset \ML$ 
			\item $\sig(\MP) = \lam(\MP)$
		\end{enumerate} 
	\end{thm}
	
	\begin{ex}
		Define $\MP \subset \MB(\R)$ by $$\MP = \{(a,b]: a,b \in \R\} \cup \{\varnothing, X\}$$
		Then $\MP$ is a $\pi$-system on $X$.
	\end{ex}

	\begin{proof}
		Let $a_1, a_2, b_1, b_2 \in \R$. Then 
		\begin{align*}
			(a_1, b_1] \cap (a_2, b_2] 
			& = (a_2, b_1] \\
			& \in \MP
		\end{align*} 
	\end{proof}






























	\newpage
	\subsection{Borel Spaces}
	\begin{defn}
		Let $(X, \MA)$ be a measurable space. Then $(X, \MA)$ is said to be a \textbf{Borel space} if there exists $S \in \MB(\R)$ such that $(X, \MA)$ is isomorphic to $(S, \MB(\R) \cap S)$.
	\end{defn}
	
	
	
	
	
	
	
	
	
	
	
	
	
	
	
	
	
	
	
	
	
	
	
	
	
	
	
	
	\newpage
	\subsection{Limits of Sets}
	
	\begin{defn} \ld{00000} 
		Let $X$ be a set and $\MA \subset \MP(X)$. We define $$\inf \MA = \bigcap_{A \in \MA } A ,\hspace{.5cm} \sup \MA = \bigcup_{A \in \MA} A$$
		
	\end{defn}
	
	\begin{defn} \ld{00000} 
		Let $X$ be a set and $(A_n)_{n \in \N} \subset \MP(X)$ a sequence of subsets. We define
		$$\liminf_{n \rightarrow \infty} A_n = \sup_{n \in \N} \bigg( \inf_{k \geq n}A_k \bigg), \hspace{.5cm } \limsup_{n \rightarrow \infty} A_n = \inf_{n \in \N} \bigg(\sup_{k \geq n}A_k \bigg)$$
	\end{defn}
	
	\begin{note}\
		\begin{enumerate}
			\item $\liminf\limits_{n \rightarrow \infty} A_n$ is the set of elements that are in all $A_n$ except for finitely many. 
			
			\item $\limsup\limits_{n \rightarrow \infty} A_n$ is the set of elements that are in infinitely many $A_n$.
		\end{enumerate}
	\end{note}
	
	\begin{ex} \lex{00000} 
		Let $X$ be a set and $(A_n)_{n \in \N} \subset \MP(X)$ a sequence of subsets. Then 
		\begin{enumerate}
			\item $\liminf\limits_{n \rightarrow \infty}A_n = \bigg \{x \in X: \liminf\limits_{n \rightarrow \infty}\chi_{A_n}(x) = 1\bigg\}$
			\item $\limsup\limits_{n \rightarrow \infty}A_n = \bigg \{x \in X: \limsup\limits_{n \rightarrow \infty}\chi_{A_n}(x) = 1\bigg\}$
		\end{enumerate}
	\end{ex}
	
	\begin{proof}\
		\begin{enumerate}
			\item Let $x \in \liminf\limits_{n \rightarrow \infty}A_n$. Then there exists $n^* \in \N$ such that for each $k \in \N$, $k \geq n^*$ implies that $x \in A_k$. So for each $k \in \N$, $k \geq n^*$ implies that $\chi_{A_k}(x) = 1$. Then $\inf\limits_{k \geq n^*}\chi_{A_k}(x) = 1$ and thus $$1 = \sup\limits_{n \in \N} \bigg(\inf\limits_{k \geq n} \chi_{A_k}(x) \bigg) = \liminf_{n \rightarrow \infty}\chi_{A_n}(x)$$ \vspace{3mm} \\
			Conversely, if $1 = \liminf\limits_{n \rightarrow \infty}\chi_{A_n}(x)$, then choosing $\ep = \frac{1}{2}$, there exists $n \in \N$ such that for each $k \in \N$, $k \geq n$ implies that $\chi_{A_k}(x) > 1-\ep$. Hence for each $k \in \N$, $k \geq n$ implies that $\chi_{A_k}(x) = 1$. So for each for each $k \in \N$, $k \geq n$ implies that $x \in A_k$. So $x \in \liminf\limits_{n \rightarrow \infty} A_n$. 
			\item Similar to (1).
		\end{enumerate}
	\end{proof}
	
	\begin{ex} \lex{00000} 
		Let $A_k = [0, \frac{k}{k+1})$. Then 
		\begin{enumerate}
			\item $\inf\limits_{k \geq n}A_k = [0, \frac{n}{n+1})$ \\
			\item $\sup\limits_{k \geq n}A_k = [0,1)$ \\
			\item $\liminf\limits_{n \rightarrow \infty}A_n = [0,1)$ \\
			\item $\liminf\limits_{n \rightarrow \infty}A_n = [0,1)$
		\end{enumerate}
	\end{ex}
	
	\begin{proof}
		Straightforward.
	\end{proof}
	
	\begin{ex} \lex{00000} 
		Let $X$ be a set and $(A_n)_{n \in \N} \subset \MP(X)$ a sequence of subsets. Then $$\liminf_{n \rightarrow \infty} A_n \subset \limsup_{n \rightarrow \infty} A_n$$
	\end{ex}
	
	\begin{proof}
		Let $x \in \bigcup\limits_{n=1}^{\infty} \bigcap\limits_{k =n}^{\infty} A_k$. Then there exists $n^* \in \N$ such that for each $k \in \N$, if $k \geq n*$, then $x \in A_k$. Let $n \in \N$. Choose $k = \max\{n^*,n\} \geq n^*$. Then $x \in A_k$. Hence for each $n \in \N$, there exists $k \in \N$ such that $k \geq n$ and $x \in A_k$. So $x \in \bigcap\limits_{n=1}^{\infty} \bigcup\limits_{k=n}^{\infty} A_k$. Thus $\liminf\limits_{n \rightarrow \infty}A_n \subset \limsup\limits_{n \rightarrow \infty}A_n$.
	\end{proof}
	
	\begin{defn} \ld{00000} 
		Let $X$ be a set and $(A_n)_{n \in \N} \subset \MP(X)$ a sequence of subsets. If $$\liminf_{n \rightarrow \infty} A_n = \limsup_{n \rightarrow \infty} A_n$$ then we define $$\lim_{n \rightarrow \infty}A_n = \liminf_{n \rightarrow \infty} A_n = \limsup_{n \rightarrow \infty} A_n$$ 
	\end{defn}
	
	\begin{ex} \lex{00000} 
		Let $X$ be a set and $(A_n)_{n \in \N}, (B_n)_{n \in \N} \subset \MP(X)$ sequences of subsets. Suppose that for each $n \in \N$, $A_n \subset A_{n+1}$ and $B_{n+1} \subset B_n$. Then 
		\begin{enumerate}
			\item $\limn A_n = \sup\limits_{n \in \N}A_n = \bigcup\limits_{n=1}^{\infty}A_n$
			\item $\limn B_n = \inf\limits_{n \in \N}B_n = \bigcap\limits_{n=1}^{\infty}B_n$
		\end{enumerate}
	\end{ex}
	
	\begin{proof}\
		\begin{enumerate}
			\item Let $n \in \N$. Then 
			\begin{align*}
				\inf\limits_{k \geq n}A_k 
				&= \bigcap\limits_{k=n}^{\infty}A_k \\
				&= A_n
			\end{align*}
			Thus 
			\begin{align*}
				\liminf\limits_{n \rightarrow \infty}A_n 
				&= \bigcup\limits_{n=1}^{\infty} \inf\limits_{k \geq n}A_k \\
				&= \bigcup\limits_{n=1}^{\infty} A_n 
			\end{align*}  
			
			In addition,  
			\begin{align*}
				\sup_{n \geq k} A_k 
				&= \bigcup_{k=n}^{\infty}A_k \\
				&= \bigcup_{k=1}^{\infty}A_k 
			\end{align*}
			
			Therefore 
			\begin{align*}
				\limsup\limits_{n \rightarrow \infty}A_n 
				&= \bigcap\limits_{n=1}^{\infty} \inf\limits_{k \geq n}A_k \\
				&= \bigcup\limits_{n=1}^{\infty} \bigcup_{k=1}^{\infty}A_k  \\
				&= \bigcup_{n=1}^{\infty}A_n
			\end{align*}
			So $$\lim\limits_{n \rightarrow \infty} A_n = \sup_{n \in \N}A_n = \bigcup_{n =1}^{\infty}A_n$$
			
			\item Similar
		\end{enumerate}
	\end{proof}
	
	\begin{ex} \lex{00000} 
		Let $X$ be a set and $(A_n)_{n \in \N} \subset \MP(X)$ a sequence of subsets and $(A_{n_k})_{k \in \N}$ a subsequence of $(A_n)_{n \in \N}$. Then 
		\begin{enumerate}
			\item $\limsup\limits_{k \rightarrow \infty}A_{n_k} \subset \limsup\limits_{n \rightarrow \infty}(A_{n})$
			\item $\liminf \limits_{n \rightarrow \infty} A_{n} \subset \liminf\limits_{k \rightarrow \infty}(A_{n_k})$
		\end{enumerate}
	\end{ex}
	
	\begin{proof}\
		\begin{enumerate}
			\item The elements that are in $A_{n_k}$ for infinitely many $k$ are in $A_n$ for infinitely many $n$.
			\item Similar.
		\end{enumerate}
	\end{proof}
	
	\begin{ex} \lex{00000} 
		Let $X$ be a set and $(A_n)_{n \in \N} \subset \MP(X)$ a sequence of subsets, $(A_{n_k})_{k \in \N}$ a subsequence of $(A_n)_{n \in \N}$ and $A \subset X$. If $A_{n_k} \rightarrow A$, then $$\liminf\limits_{n \rightarrow \infty}A_n \subset A \subset \limsup\limits_{n \rightarrow \infty}A_n$$
	\end{ex}
	
	\begin{proof}
		The previous exercises tells us that 
		\begin{align*}
			\liminf\limits_{n \rightarrow \infty}A_n
			& \subset \liminf\limits_{k \rightarrow \infty}A_{n_k} \\
			&= A \\
			&= \limsup\limits_{k \rightarrow \infty}A_{n_k} \\
			& \subset \limsup\limits_{n \rightarrow \infty}A_n
		\end{align*}
	\end{proof}
	
	\begin{ex} \lex{00000} 
		Let $X$ be a set and $(A_n)_{n \in \N}, (B_n)_{n \in \N} \subset \MP(X)$ sequences of subsets. Suppose that for each $n \in \N$, $A_n \subset B_n$. Then 
		\begin{enumerate}
			\item $\limsup\limits_{n \rightarrow \infty}A_n \subset \limsup\limits_{n \rightarrow \infty}B_n$
			\item $\liminf\limits_{n \rightarrow \infty}A_n \subset \liminf\limits_{n \rightarrow \infty}B_n$
		\end{enumerate}
	\end{ex}
	
	\begin{proof}\
		\begin{enumerate}
			\item Let $x \in \limsup\limits_{n \rightarrow \infty}A_n$. Then for infinitely many $n \in \N$, $x \in A_n \subset B_n$. So for infinitely many $n \in \N$, $x \in B_n$. Hence $x \in \limsup\limits_{n \rightarrow \infty}B_n$. Therefore $\limsup\limits_{n \rightarrow \infty}A_n \subset \limsup\limits_{n \rightarrow \infty}B_n$.
			\item Similar.
		\end{enumerate}
	\end{proof}
	
	\begin{ex} \lex{00000} 
		Let $X$ be a set and $(A_n)_{n \in \N} \subset \MP(X)$ a sequence of subsets. Then 
		\begin{enumerate}
			\item $\limsup\limits_{n \rightarrow \infty}A_n = \bigg(\liminf\limits_{n \rightarrow \infty}A_n^c \bigg)^c$
			\item $\liminf\limits_{n \rightarrow \infty}A_n = \bigg(\limsup\limits_{n \rightarrow \infty}A_n^c \bigg)^c$
		\end{enumerate}
	\end{ex}
	
	\begin{proof}\
		\begin{enumerate}
			\item \begin{align*}
				\bigg( \liminf\limits_{n \rightarrow \infty}A_n^c \bigg)^c 
				&= \bigg( \bigcup\limits_{n=1}^{\infty} \bigcap\limits_{k=n}^{\infty}A_k^c \bigg)^c\\
				&= \ \bigcap\limits_{n=1}^{\infty}\bigcup\limits_{k=n}^{\infty}A_k  \\
				&=  \limsup\limits_{n \rightarrow \infty}A_n
			\end{align*}
			\item Similar.
		\end{enumerate}
	\end{proof}
	
	\begin{ex} \lex{00000} 
		For $n \in \N$, define $$A_n = \bigg\{ \frac{m}{n}: m \in \N \bigg\}$$ 
		Then
		\begin{enumerate}
			\item $\liminf\limits_{n \rightarrow \infty }A_n = \N$ 
			\item $\limsup\limits_{n \rightarrow \infty }A_n = \Q \cap (0,\infty )$
		\end{enumerate}
	\end{ex}
	
	\begin{proof}\
		\begin{enumerate}
			\item For each $x \in \N$ and $n \in \N$, $x = \frac{nx}{n} \in A_n$ Hence $\N \subset \liminf\limits_{n \rightarrow \infty }A_n$. Conversely, let $x \in \liminf\limits_{n \rightarrow \infty }A_n$. Then there exists $n \in \N$ such that for each $k \in \N$, if $k \geq n$, then $x \in A_k$. In particular, $x \in A_n$. Hence there exists $m_n \in \N$ such that $x = \frac{m_n}{n}$. Choose $s,t \in \N$ such that $x= \frac{s}{t}$ and $\gcd(s,t) = 1$. Choose a prime $p > n$. By assumption, $x \in A_p$. Then there exist $m_p \in \N$ such that $x = \frac{m_p}{p}$. Hence $\frac{s}{t} = \frac{m_p}{p}$ and $tm_p = sp$. Since $t | sp$ and $\gcd(s,t) = 1$, we see that $t | p$. If $t > 1$, then $p$ is not prime, which is a contradiction. So $t = 1$. Hence $x \in \N$. Thus $\liminf\limits_{n \rightarrow \infty }A_n \subset \N$. 
			\item Let $x \in \Q \cap (0, \infty)$. Then there exist $s,t \in \N$ such that $x = \frac{s}{t}$. Define the subsequence $(A_{n_k})_{k \in \N}$ by $A_{n_k} = A_{tk}$. Then for each $k \in \N$, $x = \frac{sk}{tk} \in A_{tk} = A_{n_k}$. Thus
			\begin{align*}
				x 
				&\in \inf_{k \in \N} A_{n_k} \\
				& \subset \liminf_{n \rightarrow \infty} A_{n_k} \\
				& \subset \limsup_{n \rightarrow \infty} A_{n_k} \\
				& \subset \limsup\limits_{n \rightarrow \infty } A_n
			\end{align*} 
			Conversely, clearly $\limsup\limits_{n \rightarrow \infty }A_n \subset \Q \cap (0, \infty)$ 
		\end{enumerate}
	\end{proof}
	
	\begin{ex} \lex{00000} 
		Let $X$ be a set and $(A_n)_{n \in \N}, (B_n)_{n \in \N} \subset \MP(X)$ sequences of subsets. Then $$\limsup\limits_{n \rightarrow \infty} A_n \cup B_n= \limsup\limits_{n \rightarrow \infty} A_n \cup \limsup\limits_{n \rightarrow \infty} B_n$$
	\end{ex}
	
	\begin{proof}
		Let $x \in \limsup\limits_{n \rightarrow \infty} A_n \cup B_n$. Suppose that $x \not \in \limsup\limits_{n \rightarrow \infty} A_n$. Then there exists $n^* \in \N$ such that for each $k \in \N$ if $ k \geq n^*$, then $x \not \in A_k$. Let $n \in \N$. Then there exists $k$ such that $k \geq \max\{n, n^*\}$ and $x \in A_{k} \cup B_k$. Since $k \geq n^*$, $x \not \in A_{k}$ Thus $x \in B_k$. So for each $n \in \N$, there exists $k \in \N$ such that $k \geq n$ and $x \in B_k$. Therefore $x \in \limsup\limits_{n \rightarrow \infty}  B_n$ and $$\limsup\limits_{n \rightarrow \infty} A_n \cup B_n \subset \limsup\limits_{n \rightarrow \infty} A_n \cup \limsup\limits_{n \rightarrow \infty} B_n$$ Conversely, a previous exercise tells us that $\limsup\limits_{n \rightarrow \infty} A_n \subset \limsup\limits_{n \rightarrow \infty} A_n \cup B_n$ and $\limsup\limits_{n \rightarrow \infty}  B_n \subset \limsup\limits_{n \rightarrow \infty} A_n \cup B_n$. Thus $$ \limsup\limits_{n \rightarrow \infty} A_n \cup \limsup\limits_{n \rightarrow \infty} B_n \subset \limsup\limits_{n \rightarrow \infty} A_n \cup B_n$$
	\end{proof}
	
	\begin{ex} \lex{00000} 
		Let $X$ be a set and $(A_n)_{n \in \N}, (B_n)_{n \in \N} \subset \MP(X)$ sequences of subsets. Then $$\liminf\limits_{n \rightarrow \infty} A_n \cap B_n= \liminf\limits_{n \rightarrow \infty} A_n \cap \liminf\limits_{n \rightarrow \infty} B_n$$
	\end{ex}
	
	\begin{proof}
		A previous exercise tells us that 
		\begin{align*}
			\liminf\limits_{n \rightarrow \infty} A_n \cap B_n
			&= \bigg( \limsup\limits_{n \rightarrow \infty} A_n^c \cup B_n^c \bigg)^c \\
			&= \bigg( \limsup\limits_{n \rightarrow \infty} A_n^c \cup \limsup\limits_{n \rightarrow \infty}B_n^c \bigg)^c \\
			&= \bigg( \limsup\limits_{n \rightarrow \infty} A_n^c \bigg)^c \cap \bigg( \limsup\limits_{n \rightarrow \infty}B_n^c \bigg)^c \\
			&= \liminf\limits_{n \rightarrow \infty} A_n \cap \liminf\limits_{n \rightarrow \infty}  B_n
		\end{align*}
	\end{proof}
	
	
	
	
	
	
	
	
	
	
	
	
	
	
	
	
	
	
	
	
	
	\newpage
	\subsection{Measures}
	
	\begin{defn} \ld{00000} 
		Let $(X, \MA)$ be a measurable space and $\mu:\MA \rightarrow \RG$. Then $\mu$ is said to be a \textbf{measure} on $(X, \MA)$ if 
		\begin{enumerate}
			\item there exists $A \in \MA$ such that $\mu(A)< \infty$
			\item for each $(A_n)_{n \in \N} \subset \MA$. If $(A_n)_{n \in \N}$ is disjoint, then $$\mu\bigg(\bigcup_{n \in \N}A_n \bigg) = \sum_{n \in \N}\mu(A_n)$$
		\end{enumerate}
	\end{defn}
	
	\begin{defn} \ld{00000} 
		Let $(X,\MA)$ be a measurable space and $\mu$ a measure on $(A, \MA)$. Then $(A, \MA, \mu)$ is called a \textbf{measure space}. 
	\end{defn}
	
	\begin{ex} \lex{00000} 
	Let $(X, \MA, \mu)$ be a measure space, $A$ and index set and $(E_{\al})_{\al \in A} \subset \MA$. Suppose that $\mu(X) < \infty$ and $(E_\al)_{\al \in A }$ is disjoint. Then $\{\al \in A: \mu(E_{\al}) >0\}$ is countable.\\
	\textbf{Hint:} set $A_n = \{\al \in A: \mu(E_{\al}) \geq 1/n \}$ 
	\end{ex}
	
	\begin{proof}
	For $n \in \N$, set $A_n = \{\al \in A: \mu(E_{\al}) \geq 1/n \}$ and define $ A_> = \{\al \in A: \mu(E_{\al}) >0\}$ . Then 
	$$A_> = \bigcup_{n \in \N}A_n$$
	For the sake of contradiction, suppose that 
	$A_>$ is uncountable. Then there exists $N \in \N$ such that $A_N$ is uncountable. So there exists a sequence $(\al_j)_{j \in \N} \subset A_N$. Then 
	\begin{align*}
	\infty 
	& > \mu(X) \\
	& \geq \mu \bigg( \bigcup_{j \in \N} E_{\al_{j}} \bigg) \\
	&= \sum_{j \in \N} \mu(E_{\al_j}) \\
	& \geq \sum_{j \in \N} \frac{1}{N} \\
	&= \infty
	\end{align*} 
	which is a contradiction. So $A_>$ is countable.
	\end{proof}
	
	\begin{ex} \lex{00000} 
		Let $(X, \MA, \mu)$ be a measure space. Then 
		\begin{enumerate}
			\item (monotonicity): for each $A,B \in \MA$, if $A \subset B$, then $\mu(A) \leq \mu(B)$.
			\item (subadditivity): for each $(A_n)_{n \in \N} \subset \MA$, $$\mu \bigg( \bigcup_{n \in \N} A_n \bigg) \leq \sum_{n \in \N}\mu(A_n)$$
			\item (continuity from below): for each $(A_n)_{n \in \N} \subset \MA$, if for each $n \in \N$, $A_n \subset A_{n+1}$, then $$\mu\bigg(\sup_{n \in \N} A_n\bigg) = \sup_{n \in \N}\mu(A_n)$$
			\item (continuity from above): for each $(A_n)_{n \in \N} \subset \MA$, if for each $n \in \N$, $ A_{n+1} \subset A_n$ and $\mu(A_1) < \infty$, then $$\mu\bigg(\inf_{n \in \N} A_n\bigg) = \inf_{n \in \N}\mu(A_n)$$
		\end{enumerate}
		
	\end{ex}
	
	\begin{proof}\
		\begin{enumerate}
			\item Let $A, B \in \MA$. Suppose that $A \subset B$. Then 
			\begin{align*}
				\mu(B) 
				&= \mu\bigg((B \cap A) \cup (B \cap A^c)\bigg)\\
				&= \mu(B \cap A) + \mu(B \cap A^c)\\
				&= \mu(A) + \mu(B \cap A^c)\\
				&\geq \mu(A)
			\end{align*}
			\item Let $(A_n)_{n \in \N} \subset \MA$. Define $B_1 = A_1$ and for $n \geq 2$, $B_n = A_n \setminus \bigg( \bigcup\limits_{k=1}^{n-1}A_k \bigg)$. Then $(B_n)_{n \in \N} \subset \MA$, $\bigcup\limits_{n \in \N}B_n = \bigcup\limits_{n \in \N}A_n $, $(B_n)_{n \in \N}$ disjoint and for each $n \in \N$, $B_n \subset A_n$. Thus 
			\begin{align*}
				\mu\bigg(\bigcup_{n \in \N}A_n \bigg)
				&= \mu\bigg(\bigcup_{n \in \N}B_n \bigg)\\
				&= \sum_{n \in \N}\mu(B_n) \\
				&\leq \sum_{n \in \N}\mu(A_n)
			\end{align*} 
			\item Let $(A_n)_{n \in \N} \subset \MA$. Suppose that for each $n \in \N$, $A_n \subset A_{n+1}$. Then for each $n \in \N$, $\mu(A_n) \leq \mu(A_{n+1})$ and $\lim\limits_{n \rightarrow \infty}\mu(A_n) = \sup\limits_{n \in \N} \mu(A_n)$. Recall that $\sup\limits_{n \in \N}A_n = \bigcup\limits_{n \in \N}A_n$. 
			Define $B_1 = A_1$ and for $n \geq 2$, $B_n = A_n \setminus A_{n-1}$. Then $(B_n)_{n \in \N} \subset \MA$, $(B_n)_{n \in \N}$ is disjoint, $\bigcup\limits_{n \in \N}A_n = \bigcup\limits_{n \in \N}B_n$ and for each $n \in \N$, $\bigcup\limits_{n=1}^{k}B_n = A_k$. Then 
			\begin{align*}
				\mu\bigg(\sup_{n \in \N}A_n \bigg)
				&= \mu\bigg(\bigcup_{n \in \N}A_n \bigg) \\
				&= \mu\bigg(\bigcup\limits_{n \in \N}B_n \bigg)\\
				&= \sum_{n \in \N} \mu(B_n) \\
				&= \lim_{k \rightarrow \infty} \sum_{n=1}^k \mu(B_n) \\
				&= \lim_{k \rightarrow \infty} \mu \bigg(\bigcup_{n=1}^k B_n \bigg) \\
				&= \lim_{k \rightarrow \infty} \mu(A_k) \\
				&= \sup_{n \in \N} \mu(A_n)
			\end{align*}
			\item Let $(A_n)_{n \in \N} \subset \MA$. Suppose that for each $n \in \N$, $ A_{n+1} \subset A_n$ and $\mu(A_1) < \infty$. Then for each $n \in \N$ $\mu(A_{n+1}) \leq \mu(A_n) \leq \mu(A_1) < \infty$ and the arithmetic that follows is well defined. Recall that $\inf\limits_{n \in \N}A_n = \bigcap\limits_{n \in \N}A_n$. For each $n\in \N$, define $B_n = A_1 \cap A_n$. Then for each $n \in \N$, $B_n \subset B_{n+1}$ and
			\begin{align*}
				\sup\limits_{n \in \N}B_n 
				&= \bigcup\limits_{n \in \N}B_n \\
				&= A_1 \setminus \ \bigcap\limits_{n \in \N}A_n \\
				&= A_1 \setminus \inf_{n \in \N}A_n
			\end{align*}  
			So $(3)$ implies that
			\begin{align*}
				\sup_{n \in \N}\mu(B_n) 
				&= \mu \bigg(\sup\limits_{n \in \N}B_n \bigg)\\
				&= \mu \bigg(A_1 \setminus \inf_{n \in \N}A_n \bigg)\\
				&= \mu(A_1) - \mu\bigg(\inf_{n \in \N}A_n \bigg)\\
			\end{align*}
			On the other hand, 
			\begin{align*}
				\sup_{n \in \N}\mu(B_n)
				&= \sup_{n \in \N}\mu(A_1 \setminus A_n)\\
				&= \sup_{n \in \N} \bigg[ \mu(A_1) - \mu(A_n) \bigg]\\
				&= \mu(A_1) - \inf_{n \in \N}\mu(A_n)
			\end{align*}
			Therefore $$\mu\bigg(\inf_{n \in \N}A_n \bigg) = \inf_{n \in \N}\mu(A_n)$$
		\end{enumerate}
	\end{proof}
	
	\begin{ex} \lex{00000} 
		Let $(X, \MA, \mu)$ be a measure space, $(A_n)_{n \in \N} \subset \MA$ and $A \in \MA$. Then 
		\begin{enumerate}
			\item $\mu \bigg(\liminf\limits_{n \rightarrow \infty} A_n \bigg) \leq \liminf\limits_{n \rightarrow \infty} \mu(A_n)$
			\item If $\mu\bigg(\sup\limits_{n \in \N}A_n \bigg) < \infty$, then $\limsup\limits_{n \rightarrow \infty}\mu(A_n) \leq \mu \bigg( \limsup\limits_{n \rightarrow \infty}A_n\bigg) $
		\end{enumerate} 
	\end{ex}
	
	\begin{proof}\
		\begin{enumerate}
			\item Since $\bigg(\inf_{k \geq n}A_k \bigg)_{n \in \N}$ is an increasing sequence and for each $n \in \N$ $\inf\limits_{k \geq n}A_k \subset A_n$, we have that
			\begin{align*}
				\mu\bigg(\liminf\limits_{n \rightarrow \infty} A_n \bigg) 
				& = \mu \bigg[\sup_{n \in \N} \bigg(\inf\limits_{k \geq n} A_k \bigg) \bigg] \\
				&= \sup_{n \in \N}  \mu\bigg( \inf_{k \geq n}A_k\bigg)\\
				& = \liminf_{n \rightarrow \infty} \mu\bigg( \inf_{k \geq n}A_k\bigg) \\
				& \leq \liminf_{n \rightarrow \infty}  \mu(A_n) \\
			\end{align*}
			\item Since $\mu\bigg(\sup\limits_{ \geq 1}A_k \bigg) < \infty$, $\bigg(\sup\limits_{k \geq n} \bigg)_{n \in \N}$ is a decreasing and for each $n \in \N$, $A_n \subset \sup_{k \geq n}A_n$, we have that 
			\begin{align*}
				\mu \bigg(\limsup_{n \rightarrow \infty} A_n 
				\bigg) 
				&= \mu \bigg[\inf_{n \in \N} \bigg(\sup_{k  \geq n} A_k \bigg) \bigg] \\
				&= \inf_{n \in \N}\mu\bigg( \sup_{k \geq n} A_k\bigg) \\
				& = \limsup_{n \rightarrow \infty} \mu \bigg( \sup_{k \geq n}A_k \bigg) \\
				& \geq \limsup_{n \rightarrow \infty} \mu ( A_n )\
			\end{align*} 
		\end{enumerate}
	\end{proof}
	
	\begin{ex} \lex{00000} 
		Let $(X, \MA, \mu)$ be a measure space, $(A_n)_{n \in \N} \subset \MA$ and $A \in \MA$. Suppose that $\mu\bigg(\sup\limits_{n \in \N}A_n\bigg) < \infty$. Then $A_n \rightarrow A$ implies that $\mu(A_n) \rightarrow \mu(A)$. 
	\end{ex} 
	
	\begin{proof}
		Suppose that $A_n \rightarrow A$. Then the previous exercise tells us that 
		\begin{align*}
			\mu(A)
			&= \mu\bigg(\liminf\limits_{n \rightarrow \infty}A_n \bigg)\\
			& \leq \liminf_{n \rightarrow \infty} \mu(A_n)\\
			& \leq \limsup_{n \rightarrow \infty}\mu (A_n) \\
			& \leq \mu( \limsup_{n \rightarrow \infty} A_n) \\
			&= \mu (A)
		\end{align*}
		
		Thus $\mu(A) = \limsup\limits_{n \rightarrow \infty}\mu(A_n) = \liminf\limits_{n \rightarrow \infty}\mu(A_n)$ and $\mu(A_n) \rightarrow \mu(A)$
	\end{proof}
	
	\begin{defn}
		Let $(X, \MA)$ be a measurable space and $\mu: \MA \rightarrow [0, \infty]$ a measure. Then $\mu$ is said to be 
		\begin{itemize}
			\item \textbf{finite} if $\mu(X) < \infty$
			\item \textbf{$\sig$-finite} if there exists $(E_j)_{j \in \N} \subset \N$ such that 
			\begin{enumerate}
				\item $X = \bigcup\limits_{j \in \N} E_j$
				\item for each $j \in \N$, $\mu(E_j) < \infty$
			\end{enumerate}
			\item \textbf{semifinite} if for each $F \in \MA$, $\mu(F) = \infty$ implies that there exists $E \in \MA$ such that $E \subset F$ and $\mu(E) < \infty$.
		\end{itemize}
	\end{defn}

	\begin{ex}
		Let $(X, \MA)$ be a measurable space and $\mu: \MA \rightarrow [0, \infty]$ a measure. 
		\begin{enumerate}
			\item If $\mu$ is finite, then $\mu$ is $\sig$-finite. 
			\item If $\mu$ is $\sig$-finite, then $\mu$ is semifinite.
		\end{enumerate}
	\end{ex}

	\begin{proof}\
		\begin{itemize}
			\item Suppose that $\mu$ is finite. Define $(E_j)_{j \in \N} \subset \MA$ by 
			\[E_j =
			\begin{cases}
				X & j=1 \\
				\varnothing & j > 1
			\end{cases}
			\]
			Then $X = \bigcup\limits_{j \in \N} E_j$ and for each $j \in \N$, $0 < \mu(E_j) < \infty$.
			\item Suppose that $\mu$ is $\sig$-finite. Then there exists $(E_j)_{j \in \N} \subset \MA$ such that $X = \bigcup\limits_{j \in \N} E_j$
			and for each $j \in \N$, $\mu(E_j) < \infty$. Let $F \in \MA$. Suppose that $\mu(F) = \infty$. Define $(A_n)_{n \in \N} \subset \MA$ by 
			$$A_n = \bigcup_{j=1}^n E_j$$
			Note that $X = \bigcup\limits_{n \in \N} A_n$ and for each $n \in \N$, $F \cap A_n \subset F \cap A_{n+1}$ and 
			\begin{align*}
				\mu \bigg (F \cap A_n \bigg)  
				& = \mu \bigg (F \cap \bigg[ \bigcup_{j=1}^n E_j \bigg] \bigg)  \\
				& \leq \mu \bigg (\bigcup_{j=1}^n E_j \bigg) \\
				& \leq \sum_{j=1}^n \mu(E_j) \\
				& < \infty
			\end{align*}
			For the sake of contradiction, suppose that for each $n \in \N$, $\mu  (F \cap A_n ) = 0$. Then
			\begin{align*}
				\infty 
				& = \mu(F) \\
				& = \mu(F \cap X) \\
				& = \mu\bigg(F \cap \bigg[ \bigcup_{n \in \N} A_n \bigg] \bigg) \\
				& = \mu\bigg(  \bigcup_{n \in \N} [F \cap A_n] \bigg) \\
				& = \sup_{n \in \N} \mu(F \cap A_n) \\
				& = 0
			\end{align*}
			which is a contradiction. So there exists $N \in \N$ such that $\mu  (F \cap A_N ) > 0$. Set $E = F \cap A_N$. Then $E \subset F$ and $0 < \mu(E) < \infty$. Hence $\mu$ is semifinite.
		\end{itemize}
	\end{proof}

	\begin{defn}
		Let $(X, \MA, \mu)$ be a measure space and $(f_\al)_{\al \in A} \subset L^0(X, \MA)$ a net. Suppose that for each $\al \in A$, $f_{\al}:X \rightarrow \R$. For each $\al, \be \in A$, define $M_{\al, \be}, N_{\al, \be} \in \MA$ by
		$$M_{\al, \be} = \{x \in X: f_{\al}(x) \leq f_{\be}(x) \}$$ 
		and 
		$$N_{\al, \be} = \{x \in X: f_{\al}(x) \geq f_{\be}(x) \}$$
		respectively. Define $M,N \subset X$ by $M = \bigcap\limits_{\substack{(\al, \be) \in A^2 \\ \al \leq \be}} M_{\al, \be}$ and $N = \bigcap\limits_{\substack{(\al, \be) \in A^2 \\ \al \leq \be}} N_{\al, \be}$ respectively. Then $(f_{\al})_{\al \in A}$ is said to be 
		\begin{itemize}
			\item \textbf{increasing $\mu$-a.e.} if $M^c$ is a $\mu$-null set
			\item \textbf{decreasing $\mu$-a.e.} if $N^c$ is a $\mu$-null set
			\item \textbf{monotonic $\mu$-a.e.} if $(f_n)_{n \in \N}$ is increasing $\mu$-a.e. or $(f_n)_{n \in \N}$ is decreasing $\mu$-a.e.
		\end{itemize} 
	\end{defn}

	\begin{ex}
		Let $(X, \MA, \mu)$ be a measure space and $(f_\al)_{\al \in A} \subset L^0(X, \MA)$ a net. Suppose that for each $\al \in A$, $f_{\al}:X \rightarrow \R$. If $A$ is countable, then 
		\begin{enumerate}
			\item $(f_\al)_{\al \in A}$ is increasing $\mu$-a.e. iff for each $\al, \be \in A$, $\al \leq \be$ implies that $f_{\al} \leq f_{\be}$ $\mu$-a.e.
			\item $(f_\al)_{\al \in A}$ is decreasing $\mu$-a.e. iff for each $\al, \be \in A$, $\al \leq \be$ implies that $f_{\al} \geq f_{\be}$ $\mu$-a.e.
		\end{enumerate}
	\end{ex}

	\begin{proof} Suppose that $A$ is countable. For each $\al, \be \in A$, define $M_{\al, \be},N_{\al, \be},M,N \in \MA$ as in the previous definition. Since $A$ is countable, $M,N \in \MA$. 
		\begin{enumerate}
			\item Suppose that $(f_{\al})_{n \in \N}$ is increasing $\mu$-a.e. By definition, $M^c$ is a $\mu$-null set. Since $M^c \in \MA$, $\mu(M^c) = 0$. Let $\al, \be \in A$. Suppose that $\al \leq \be$. Since $M \subset M_{\al, \be}$, $M_{\al, \be}^c \subset M^c$. Hence $\mu(M_{\al, \be}^c) = 0$. By definition, $f_{\al} \leq f_{\be}$ $\mu$-a.e.\\
			Conversely, suppose that for each $\al, \be \in A$, $\al \leq \be$ implies that $f_{\al} \leq f_{\be}$ $\mu$-a.e. Then for each $\al, \be \in A$, $\mu(M_{\al, \be}^c) = 0$. Since $A$ is countable, we have that
			\begin{align*}
				\mu(M^c) 
				& = \mu \bigg( \bigcup\limits_{\substack{(\al, \be) \in A^2 \\ \al \leq \be}} M_{\al, \be}^c \bigg) \\
				& \leq \sum_{\substack{(\al, \be) \in A^2 \\ \al \leq \be}} \mu(M_{\al, \be}^c) \\
				& = 0
			\end{align*}
			Thus $(f_n)_{n \in \N}$ is increasing $\mu$-a.e.
			\item Similar to $(1)$.
		\end{enumerate}
	\end{proof}







	
	
	
	
	
	
	
	
	
	
	
	
	
	
	
	
	\newpage
	\subsection{Outer Measures}
	
	\begin{defn} \ld{00000} 
		Let $X$ be a set and $\mu* : \MP(X) \rightarrow [0, \infty]$. Then $\mu^*$ is said to be an \textbf{outer measure on X} if 
		\begin{enumerate}
			\item $\mu^*(\varnothing) = 0$
			\item for each $A,B \subset X $, if $A \subset B$, then $\mu^*(A) \leq \mu^*(B)$.
			\item for each $(A_n)_{n \in \N} \subset \MP(X)$, $$\mu^*\big(\bigcup\limits_{n \in \N} A_n\big) \leq \sum\limits_{n \in \N}\mu^*(A_n) $$
		\end{enumerate}
	\end{defn}
	
		\begin{defn} \ld{00000} 
		Let $X$ be a set, $\mu^*$ an outer measure on $X$ and $A \subset X$. Then $A$ is said to be $\mu^*$-\textbf{outer measurable} if for each $E \subset X$, 
	\begin{equation*}
	\mu^*(E) = \mu^*(E \cap A) + \mu^*(E \cap A^c)
	\end{equation*}
	\end{defn}
	
	\begin{ex}
	Let $X$ be a set, $\mu^*$ an outer measure on $X$ and $A \subset X$. Then $A$ is $\mu^*$-outer measurable iff for each $E \subset X$, $\mu^*(E) < \infty$ implies that 
	\begin{equation*}
	\mu^*(E) \geq \mu^*(E \cap A) + \mu^*(E \cap A^c)
	\end{equation*}
	\end{ex}	
	
	\begin{proof}
	Suppose that $A$ is $\mu^*$-outer measurable. Let $E \subset X$, Suppose that $\mu^*(E) < \infty$. By definition $\mu^*(E) \geq \mu^*(E \cap A) + \mu^*(E \cap A^c)$. \\
	Conversely, suppose that for each $E \subset X$, $\mu^*(E) < \infty$ implies that $\mu^*(E) \geq \mu^*(E \cap A) + \mu^*(E \cap A^c)$. Let $E \subset X$. 
	\begin{itemize}
	\item If $\mu^*(E) < \infty$, then by assumption, 
	\begin{equation*}
	\mu^*(E) \geq \mu^*(E \cap A) + \mu^*(E \cap A^c)
	\end{equation*}		
	If $\mu^*(E) = \infty$, then trivially, 
	\begin{equation*}
	\mu^*(E) \geq \mu^*(E \cap A) + \mu^*(E \cap A^c)
	\end{equation*}	 
	So $\mu^*(E) \geq \mu^*(E \cap A) + \mu^*(E \cap A^c)$
	\item Since $E = (E \cap A) \cup (E \cap A^c)$, by definition, 
	\begin{equation*}
	\mu^*(E) \leq \mu^*(E \cap A) + \mu^*(E \cap A^c)
	\end{equation*}	
	\end{itemize}
	So $\mu^*(E) = \mu^*(E \cap A) + \mu^*(E \cap A^c)$ and $A$ is $\mu^*$-outer measurable.
	\end{proof}
	
	\begin{thm}\textbf{Construction of Outer Measures:} \\
		Let $X$ be a set and $\ME \subset \MP(X)$ and $\rho: \ME \rightarrow [0, \infty]$. Suppose that $\varnothing, X \in \ME$ and $\rho(\varnothing) = 0$. Define $\mu^*:\MP(X) \rightarrow [0, \infty]$ by $$\mu^*(A) = \inf \bigg \{\sum_{n \in \N}\rho(E_n): (E_n)_{n \in \N} \subset \ME \text{ and }A \subset \bigcup_{n \in \N}E_n \bigg \}$$ Then $\mu^*$ is an outer measure on $X$.
	\end{thm}
	
	\begin{note}
		In particular, for each $A \in \ME$, $\mu^*(A) = \rho(A)$.
	\end{note}
	
	\begin{defn} \ld{00000} 
		Let $X$ be a set and $\ME \subset \MP(X)$ and $\rho: \ME \rightarrow [0, \infty]$. Suppose that $\varnothing, X \in \ME$ and $\rho(\varnothing) = 0$. Let $\mu^*$ be the outer measure on $X$ defined as in the last theorem. Then $\mu^*$ is called the \textbf{outer measure on $X$ induced by $\rho$}.
	\end{defn}
	
	\begin{thm}
		Let $X$ be a set and $\mu^*$ an outer measure on $X$. Define $\MA = \{A \subset X: A \text{ is }\mu^*\text{-measurable}\}$. Then $\MA$ is a $\sig$-algebra on $X$ and $\mu^*|_{\MA}$ is a complete measure on $(X, \MA)$.
	\end{thm}
	
	\begin{defn} \ld{00000} 
		Let $X$ be a set, $\MA_0$ be an algebra on $X$ and $\mu_0:\MA_0 \rightarrow [0, \infty]$. Then $\mu_0$ is said to be a \textbf{premeasure on $(X,\MA_0)$} if 
		\begin{enumerate}
			\item there exists $A \in \MA_0$ such that $\mu_0(A)< \infty$
			\item for each $(A_n)_{n \in \N} \subset \MA_0$, if $(A_n)_{n \in \N}$ is disjoint and $\bigcup\limits_{n \in \N}A_n \in \MA_0$, then $$\mu_0 \bigg(\bigcup_{n\in \N}A_n \bigg) = \sum_{n \in \N}\mu_0(A_n)$$
		\end{enumerate}
	\end{defn}
	
	\begin{note}
		The same reasoning applied to measures shows that $\mu_0(\varnothing) = 0$.
	\end{note}
	
	\begin{thm}
		Let $X$ be a set, $\MA_0$ an algebra on $X$, $\mu_0$ a premeasure on $(X,\MA_0)$ and $\mu^*$ the outer measure on $X$ induced by $\mu_0$. Put $\MA = \sig(\MA_0)$. If $\mu_0$ is $\sig$-finite, then there exists a unique measure $\mu$ on $(X, \MA)$ such that $\mu|_{\MA_0} = \mu^*|_{\MA_0} = \mu_0$. 
	\end{thm}
	
	
	
	
	
	
	
	
	
	
	
	
	
	
	
	
	
	
	
	
	
	
	\newpage
	\subsection{Product Measures}
	
	\begin{defn} \ld{00000} 
		Let $(X,\MA, \mu), (Y,\MB, \nu)$ be $\sig$-finite measurable spaces. Put $\ME = \{A \times B: A \in \MA \text{ and } B \in \MB\}$. Then $\ME$ is an elementary family and thus $\MM_0 = \{\bigcup_{i =1}^n M_i: (M_i)_{i=1 }^n \subset \ME \text{ are disjoint}\}$ is an algebra on $X \times Y$. We define $\pi_0: \MM_0 \rightarrow \RG$ by $$\pi_0\bigg(\bigcup_{i=1}^n A_i \times B_i \bigg) = \sum_{i=1}^n\mu(A_i)\nu(B_i)$$ Then $\pi_0$ is a premeasure on $(X \times Y, M_0)$. Since $\MA \otimes \MB = \sig(\MM_0)$, we define the \textbf{product measure}, $\mu \times \nu$ on $(X \times Y, \MA \otimes \MB)$, to be the unique extension of $\pi_0$ to $\MA \otimes \MB$. The existence of which is guaranteed by a theorem in the previous section. In particular,
		
		\begin{align*}
			\mu \times \nu(E) 
			&= \inf \bigg\{\sum_{n \in \N}\pi_0(E_i): (E_i)_{i \in \N} \subset \MM_0 \text{ and } E \subset \bigcup_{i \in \N} E_i \bigg\}\\
			&= \inf \bigg\{\sum_{n \in \N}\mu(A_i)\nu(B_i): (A_i \times B_i)_{i \in \N} \subset \ME \text{ and } E \subset \bigcup_{i \in \N} A_i \times B_i \bigg \}
		\end{align*}
	\end{defn}

































	\newpage
	\subsection{Pushforward Measures}
	
	\begin{defn}
		Let $(X, \MA, \mu)$ be a measure space, $(Y, \MB)$ a measurable space and $f: X \rightarrow Y$. Suppose that $f$ is $(\MA, \MB)$ measurable. We define the \textbf{pushforward of $\mu$ by $f$ on $(Y, \MB)$}, denoted $f_*\mu: \MB \rightarrow [0, \infty]$, by $$f_*\mu(B) = \mu(f^{-1}(B))$$
	\end{defn}

	\begin{ex}
		Let $(X, \MA, \mu)$ be a measure space, $(Y, \MB)$ a measurable space and $f: X \rightarrow Y$. Suppose that $f$ is $(\MA, \MB)$ measurable. Then $f_*\mu: \MB \rightarrow \RG$ is a measure.
	\end{ex}

	\begin{proof}\
		\begin{enumerate}
			\item Since $f^{-1}(\varnothing) = \varnothing$, 
			\begin{align*}
				f_*\mu(\varnothing) 
				& = \mu(f^{-1}(\varnothing)) \\
				& = \mu(\varnothing) \\
				& = 0
			\end{align*}
			\item Let $(B_j)_{j \in \N} \subset \MB$. Suppose that $(B_j)_{j \in \N}$ is disjoint. Then $(f^{-1}(B_j))_{j \in \N}$ is disjoint. Hence 
			\begin{align*}
				f_*\mu \bigg( \bigcup_{j \in \N} B_j \bigg)
				& = \mu \bigg( \bigcup_{j \in \N} f^{-1}(B_j) \bigg) \\
				& = \sum_{j \in \N} \mu(f^{-1}(B_j)) \\
				& = \sum_{j \in \N} f_*\mu(B_j)
			\end{align*}
		\end{enumerate}
		Hence $f_* \mu$ is a measure.
	\end{proof}
	
	
	
	
	
	
	
	
	
	
	
	
	
	
	
	\newpage
	\section{The Lebesgue Integral}
	
	\subsection{Integration of Nonnegative Functions}
	
	\begin{thm}\textbf{Monotone Convergence Theorem:} \\
		Let $(f_n)_{n \in \N} \subset L^+$. Suppose that for each $n \in \N$, $f_n \leq f_{n+1}$. Then $$\sup_{n \in \N} \int f_n = \int \sup_{n \in \N} f_n$$.
	\end{thm}
	
	\begin{ex} \lex{00000} 
		Let $\mu_1, \mu_2$ be measures on $(X, \MA)$, $\lam \geq 0$ and $f \in L^+$. Then $$\int f d (\mu_1 + \lam \mu_2) = \int f d\mu_1 + \lam \int f d\mu_2$$.  
	\end{ex}
	
	\begin{proof}
		Suppose that $f$ is simple. Then there exist $(a_n)_{i=1}^n \subset \Rg$ and $(E_i)_{i=1}^n \subset \MA$ such that $f = \sum\limits_{i =1}^n a_i \chi_{E_i}$. Then 
		\begin{align*}
			\int f d(\mu_1 + \lam \mu_2) 
			&= \sum\limits_{i =1}^n a_i (\mu_1 + \lam \mu_2)(E_i)\\
			&= \sum\limits_{i =1}^n a_i (\mu_1(E_i) + \lam \mu_2(E_i))\\
			&= \sum\limits_{i =1}^n a_i \mu_1(E_i) + \lam \sum\limits_{i =1}^n a_i   \mu_2(E_i)\\
			&= \int f d\mu_1 + \lam \int f d\mu_2
		\end{align*}
		
		Now for a general $f$, choose $(\phi_n)_{n \in \N} \subset S^+$ such that $\phi_n \rightarrow f$ pointwise and for each $n \in \N$, $\phi_n \leq \phi_{n+1} \leq f$. Then monotone convergence tells us that 
		\begin{align*}
			\int f d(\mu_1 + \lam \mu_2) 
			&= \limn \int \phi_n d(\mu_1 + \lam \mu_2)\\
			&= \limn \int \phi_n \dmu_1 + \limn \lam \int \phi_n \dmu_2 \\
			&= \int f \dmu_1 +  \lam \int f \dmu_2
		\end{align*}
		
	\end{proof}
	
	
	\begin{ex} \lex{00000} 
		Let $\mu_1, \mu_2$ be measures on $(X,\MA)$. Suppose that $\mu_1 \leq \mu_2$. Then for each $f \in L^+$, $$\int f d\mu_1 \leq \int f d\mu_2$$
	\end{ex}
	
	\begin{proof}
		First suppose that $f$ is simple. Then there exist $(a_n)_{i=1}^n \subset \Rg$ and $(E_i)_{i=1}^n \subset \MA$ such that $f = \sum\limits_{i =1}^n a_i \chi_{E_i}$. Then 
		\begin{align*}
			\int f d\mu_1 
			&= \sum\limits_{i =1}^n a_i \mu_1(E_i)\\
			& \leq \sum\limits_{i =1}^n a_i \mu_2(E_i)\\
			&= \int f \dmu_2
		\end{align*} 
		
		for general $f$, 
		\begin{align*}
			\int f d\mu_1 
			&= \sup_{\substack{s \in S^+\\s \leq f}} \int s \dmu_1 \\
			& \leq \sup_{\substack{s \in S^+\\s \leq f}} \int s d\mu_2\\
			&= \int f d\mu_2
		\end{align*}
		
	\end{proof}
	
	\begin{thm} \textbf{Fatou's Lemma:}\\
		Let $(f_n)_{n \in \N} \subset L^+$. Then $$\int \limfn f_n \leq \limfn \int f_n.$$
	\end{thm}
	
	\begin{thm}
		Let $(f_n)_{n \in \N} \subset L^+$. Then $$\int \sum_{n \in \N} f_n= \sum_{n \in \N} \int f_n.$$
	\end{thm}
	
	\begin{ex} \lex{00000} 
		Let $f \in L^+$ and suppose that $\int f < \infty$. Put $N = \{x \in X: f(x) = \infty\}$ and $S = \{x \in X: f(x) > 0\}$. Then $\mu(N) = 0$ and $S$ is $\sig$-finite.
	\end{ex}
	
	\begin{proof}
		Suppose that $\mu(N) > 0$. Define $f_n = n \chi_{N} \in L^+$. Then for each $n \in \N$, $f_n \leq f_{n+1} \leq f$ on $N$. So 
		\begin{align*}
			\int f 
			&\geq \int_N f\\ 
			&= \lim\limits_{n \rightarrow \infty} \int_N f_n\\ 
			&= \lim\limits_{n \rightarrow \infty} n\mu(N)\\
			&= \infty \text{, a contradiction.}
		\end{align*}
		Hence $N$ is a null set. Now, put $S_n = \{x \in X: f(x)>1/n\}$. Then $S = \bigcup \limits_{n \in \N}S_n$. Suppose that there exists some $n \in \N$ such that $\mu(S_n) = \infty$. Then 
		\begin{align*}
			\int f 
			&\geq \int_{S_n} f \\
			&\geq \frac{1}{n}\mu(S_n) \\
			&= \infty \text{, a contradiction.}
		\end{align*}
		
		So for each $n \in \N$, $\mu(S_n) < \infty$ and $S$ is $\sig$-finite.
		
	\end{proof}
	
	\begin{ex} \lex{00000} 
		Let $f \in L^+$. Then $f =0$ a.e. iff for each $E \in \MA$, $\int_E f =0$.
	\end{ex}
	
	\begin{proof}
		$f = 0$ a.e. implies that for each $E \in \MA$, $\int_E f =0$ is clear. Conversely, suppose that for each $E \in \MA$, $\int_E f$ = 0. For $n \in \N$ put $N_n = \{x \in X: f(x) > 1/n\}$ and define $N = \{x \in X: f(x)>0\}$. So $N = \bigcup\limits_{n \in \N} N_n$. Let $n \in \N$. Then our assumption tells us that 
		\begin{align*}
			0 
			&= \int_{N_n} f \\
			& \geq \frac{1}{n}\mu(N_n)\\
			& \geq 0.
		\end{align*} 
		
		Hence for each $n \in \N$, $\mu(N_n) = 0$. Thus $\mu(N) = 0$ and $f =0$ a.e. as required.
		
	\end{proof}
	
	\begin{ex} \lex{00000} 
		
		Let $(f_n)_{n \in \N} \subset L^+$ and $f \in L^+$. Suppose that $f_n \xrightarrow{\text{p.w.}} f$, $\lim \limits_{n \rightarrow \infty} \int f_n = \int f$ and $\int f < \infty$. Then for each $E \in \MA$, $\lim \limits_{n \rightarrow \infty} \int_E f_n = \int_E f$. This result may fail to be true if $\int f = \infty$
		
	\end{ex}
	
	\begin{proof}
		
		Let $E \in \MA$. By Fatou's lemma, $\int_E f \leq \limfn \int_E f_n$. Note that since $\int f < \infty$, we have that $\int_{E^c} f \leq \int f < \infty$. Thus we may write
		\begin{align*}
			\int_E f 
			&= \int f - \int_{E^c} f\\
			&\geq \int f - \limfn \int_{E^c} f_n\\
			&= \int f - \limfn \bigg(\int f_n - \int_{E} f_n\bigg)\\
			&= \int f - \int f  + \limpn \int_{E} f_n\\
			&= \limpn \int_E f_n.
		\end{align*}
		
		Hence $$\limpn \int_E f_n \leq \int_E f \leq \limfn \int_E f_n$$ and therefore $$\limn \int_E f_n = \int_E f.$$ 
		
		If we drop the assumption that $\int f < \infty$, then the result would fail to be true for the functions $f = \infty \chi_{(0,1)}$ and $ f_n = \infty \chi_{(0,1)} + n \chi_{(1,1+1/n)}$. Here $f_n \xrightarrow{\text{p.w.}} f$, $\limn \int f_n = \int f = \infty$ and $\limn \int_{(1,\infty)} f_n = 1$ while $\int_{(1,\infty)} f = 0$.  
		
	\end{proof}
	
	\begin{ex} \lex{00000} 
		Let $f \in L^+$. Define $\lam: \MA \rightarrow \RG$ by $\lam(E) = \int_E f d\mu$ for $E \in \MA$
		Then $\lam$ is a measure on $(X, \MA)$ and for each $g \in L^+$, $\int g d\lam = \int g f d\mu$.
	\end{ex}
	
	\begin{proof}
		Clearly $\lam(\varnothing) = 0$. Let $(A_j)_{j \in \N} \subset \MA$ and suppose that for each $i, j \in \N$, if $i \neq j$, then $A_i \cap A_j = \varnothing$. For now, suppose that $f$ is simple. Then there exist $E_1, E_2, \cdots, E_n \in \MA$ and  $a_1, a_2, \cdots, a_n \in \Rg$ such that $f = \sum\limits_{i=1}^n a_i \chi_{E_i}$.  Then 
		\begin{align*}
			\lam\bigg(\bigcup_{j \in \N} A_j\bigg) 
			&= \int_{\bigcup_{j \in \N} A_j} f\\
			&= \sum_{i = 1} ^n a_i\mu\bigg(E_i \cap \bigg(\bigcup_{j \in \N} A_j\bigg)\bigg)\\
			&= \sum_{i = 1} ^n a_i\mu\bigg(\bigcup_{j \in \N} E_i \cap A_j\bigg)\\
			&= \sum_{i = 1} ^n a_i \sum_{j \in \N} \mu(E_i \cap A_j)\\
			&= \sum_{j \in \N} \sum_{i = 1} ^n a_i \mu(E_i \cap A_j)\\
			&= \sum_{j \in \N} \int_{A_j} f\\
			&= \sum_{j \in \N} \lam(A_j)
		\end{align*} 
		Hence $\lam$ is a measure on $(X, \MA)$. Now, for a general f, there exist $(\phi_n)_{n \in \N} \subset L^+$ such that for each $n \in \N$, $\phi_n$ is simple, $\phi_n \leq \phi_{n+1} \leq f$ and $\phi_n \xrightarrow{\text{p.w.}} f$. Put $A = \bigcup_{j \in \N}A_j$ and define the measures $\lam_n$ by $\lam_n(E) = \int_E \phi_n$. Note that we may define a monotonically increasing sequence of functions $g_n: \|\rightarrow \RG$ by $g_n(j) = \int_{A_j} \phi_n$. Using monotone convergence three times and a nice application of the counting measure on $\N$, we may write
		
		\begin{align*}
			\lam(A) 
			&= \int_A f\\
			&= \limn \int_A \phi_n\\
			&= \limn \sum_{j \in \N} \int_{A_j} \phi_n\\
			&= \sum_{j \in \N} \limn \int_{A_j} \phi_n \hspace{4mm} \text{(by the above)}\\
			&= \sum_{j \in \N} \int_{A_j} f\\
			&= \sum_{j \in \N} \lam(A_j).
		\end{align*} 
		
		Hence $\lam$ is a measure on $(X, \MA)$. Let $g \in L^+$. First assume that $g$ is simple. Then there exist $E_1, E_2, \cdots, E_n \in \MA$ and  $a_1, a_2, \cdots, a_n \in \Rg$ such that $g = \sum\limits_{i=1}^n a_i \chi_{E_i}$.
		In this case, we have that 
		\begin{align*}
			\int g d\lam 
			&= \sum_{i=1}^n a_i \lam(E_i)\\
			&= \sum_{i=1}^n a_i \int_{E_i} f d\mu\\
			&= \int \bigg(\sum_{i=1}^n a_i\chi_{E_i} \bigg) f d\mu\\
			&= \int gf d\mu.
		\end{align*}
		
		Now for a general $g \in L^+$, there exist $(\psi_n)_{n \in \N} \subset L^+$ such that for each $n \in \N$, $\psi_n$ is simple, $\psi_n \leq \psi_{n+1} \leq f$ and $\psi_n \xrightarrow{\text{p.w.}} g$. Monotone convergence then gives us 
		\begin{align*}
			\int g d\lam 
			&= \limn \int \psi_n d\lam\\
			&= \limn \int \psi_n f d\mu\\
			&= \int g f d\mu \text{ as required.}
		\end{align*}
	\end{proof}
	
	\begin{ex} \lex{00000} 
		Let $(f_n)_{n \in \N} \subset L^+$ and $f \in L^+$. Suppose that for each $n \in \N$, $f_n \geq f_{n+1}$, $f_n \xrightarrow{\text{p.w.}} f$ and $\int f_1 < \infty$. Then $\limn \int f_n = \int f$.
	\end{ex}
	
	\begin{proof}
		First we note that since $\int f_1 < \infty$, $f_1 < \infty$ a.e., for each $n \in \N$, $f_1 - f_n$ and $\int f_1 - \int f_n$ are well defined and $\int f_n \leq \int f_1 < \infty$. Also, for $n \in \N$, $f_1 -f_n \in L^+$. So we may write 
		\begin{align*}
			\int (f_1 - f_n) 
			&= \int (f_1 - f_n)  + \int f_n - \int f_n\\
			&= \int [(f_1 - f_n) + f_n] - \int f_n\\
			&= \int f_1 - \int f_n
		\end{align*}
		
		Put $g_n = f + (f_1 - f_n)$. Then $g_n \in L^+$, for each $n \in \N$, $g_n \leq g_{n+1}$ and $g_n \xrightarrow{\text{p.w.}} f_1$. Monotone convergence tells us that 
		\begin{align*}
			\int f_1 
			&= \limn \int g_n\\
			&= \limn \bigg[\int f + (f_1-f_n)\bigg]\\
			&= \limn \bigg[ \int f + \int (f_1-f_n)\bigg] \\
			&= \limn \bigg[ \int f + \int f_1- \int f_n\bigg] 
		\end{align*}
		
		Since $\limn \int f$ and $\limn \int f_1$ exist, $\limn \int f_n = \int f$ as required.  
		
	\end{proof}


	\begin{ex}
		Let $(X, \MA, \mu)$ be a measure space, $(Y, \MB)$ a measurable space and $f: X \rightarrow Y$. Suppose that $f$ is $(\MA, \MB)$-measurable. Then for each $g \in L^+(Y, \MB)$ and $B \in \MB$, $$\int_{f^{-1}(B)} g \circ f \dmu = \int_B g \, d f_*\mu$$
	\end{ex}

	\begin{proof}
		Let $g \in  L^+(X, \MA)$ and $B \in \MB$. Suppose that there exists $E \in \MB$ such that $g = \chi_E$. Then $g \circ f = \chi_{f^{-1}(E)}$ and
		\begin{align*}
			\int_{f^{-1}(B)} g \circ f \dmu
			& = \int_{f^{-1}(B)} \chi_{f^{-1}(E)} \dmu \\
			& = \mu(f^{-1}(E) \cap f^{-1}(B)) \\
			& = \mu( f^{-1}(E \cap B)) \\
			& = f_*\mu (E \cap B) \\
			& = \int_{B} \chi_{E} \, df_*\mu \\
			& = \int_{B} g \, df_*\mu 
 		\end{align*}
 		Suppose that $g$ is simple. Then there exist $(a_j)_{j=1}^n \subset \Rg$ and $(E_j)_{j=1}^n \subset \MB$ such that $g = \sum\limits_{j=1}^n a_j \chi_{E_j}$. 
 		Then 
 		\begin{align*}
 			g \circ f 
 			& = \bigg( \sum\limits_{j=1}^n a_j \chi_{E_j} \bigg) \circ f\\
 			& = \sum\limits_{j=1}^n a_j \chi_{E_j} \circ f
 		\end{align*}
 		and 
 		\begin{align*}
 			\int_{f^{-1}(B)} g \circ f \dmu
 			& = \int_{f^{-1}(B)} \sum\limits_{j=1}^n a_j \chi_{E_j} \circ f \dmu \\
 			& = \sum_{j=1}^n a_j \int_{f^{-1}(B)} \chi_{E_j} \circ f \dmu \\
 			& = \sum_{j=1}^n a_j \int_{B} \chi_{E_j} \, d f_*\mu \\
 			& = \int_{B} \sum_{j=1}^n a_j  \chi_{E_j} \, d f_*\mu \\
 			& = \int_{B} g \, d f_*\mu \\
 		\end{align*}
 		Suppose that $g \in L^+(Y, \MB)$. Then there exists $(\phi_n)_{n \in \N} \subset S^+(Y, \MB)$ such that $\phi_n \convt{p.w.} g$ and for each $n \in \N$, $\phi_n \leq \phi_{n+1}$. Then $\phi_n \circ f \convt{p.w.} g \circ f$ and for each $n \in \N$, $\phi_n \circ f \leq \phi_{n+1} \circ f$. The monotone convergence theorem implies that 
 		\begin{align*}
 			\int_{f^{-1}(B)} g \circ f \dmu
 			& = \limn \int_{f^{-1}(B)} \phi_n \circ f \dmu \\
 			& = \limn \int_{B} \phi_n \, d f_*\mu \\
 			& = \int_{B} g \, d f_*\mu \\
 		\end{align*}
	\end{proof}

	\begin{ex}
		Let $(X, \MA, \mu)$ be a measure space, $(Y, \MB)$ a measurable space and $f: X \rightarrow Y$. Suppose that $f$ is $(\MA, \MB)$-measurable. Let $g, h \in L^0(Y, \MB)$. If $g \circ f = h \circ f$ $\mu$-a.e., then $g = h$ $f_*\mu$-a.e.
	\end{ex}

	\begin{proof}
		Suppose that $g \circ f = h \circ f$ $\mu$-a.e. Then $|(g - h) \circ f| = 0$ $\mu$-a.e. The previous exercise implies that 
		\begin{align*}
			\int_Y |g - h|  \, d f_*\mu 
			& = \int_X |g - h| \circ f \dmu \\
			& = \int_X |(g - h) \circ f| \dmu \\
			& = 0
		\end{align*}
		Hence $|g - h| = 0$ $f_*\mu$-a.e. and $g = h$ $f_*\mu$-a.e.
	\end{proof}

	
	
	
	
	
	
	
	
	
	
	
	
	
	
	
	
	\newpage
	\subsection{Integration of Complex Valued Functions}
	
	\begin{defn} \ld{00000} 
		Let $f:X \rightarrow \C$ be measurable. Then $f$ is said to be \textbf{integrable} if $$\int |f| \dmu < \infty$$
	\end{defn}
	
	\begin{defn} \ld{00000} 
		Let $(X, \MA, \mu)$ be a measure space. Define $$L^1(X, \MA, \mu) = \bigg \{f:X \rightarrow \C : f \text{ is measurable and } \int |f| < \infty \bigg \}$$
	\end{defn}
	
	\begin{lem}
		Let $f:X \rightarrow \R$ be measurable. Then $f$ is integrable iff $f^+$ and $f^-$ are integrable. 
	\end{lem}
	
	\begin{proof}
		$f^+,f^- \leq |f| = f^+ + f^-$
	\end{proof}
	
	\begin{defn} \ld{00000} 
		Let $f:X \rightarrow \R$ be measurable. Then $f$ is said to be \textbf{extended integrable} if $$\int f^+ \dmu  < \infty \text{ or } \int f^- \dmu < \infty$$
	\end{defn}
	
	\begin{lem}
		Let $f:X \rightarrow \R$ be measurable. Then $f$ is integrable iff $Re(f)$ and $Im(f)$ are integrable.
	\end{lem}
	
	\begin{proof}
		$|Re(f)|, |Im(f)| \leq |f| \leq |Re(f)| + |Im(f)|$
	\end{proof}
	
	\begin{ex}\textbf{Dominated Convergence Theorem:}\\
	\lex{00000}
		Let $(f_n)_{n \in \N} \subset L^0$, $f \in L^0$ and $g \in L^1$. Suppose that $f_n \convt{a.e.} f$ and there exists $g \in L^1$ such that for each $n \in \N$, $|f_n| \leq g$. Then $f \in L^1$ and $$\int_X |f_n - f| \dmu \rightarrow 0$$ \\
		\textbf{Hint:} Fatou's lemma
	\end{ex}
	
	\begin{proof}
	Continuity imples that $|f| \leq g$ a.e. Since 
	\begin{align*}
	|f_n - f| 
	&\leq |f_n| + |f| \\
	&\leq 2g
	\end{align*}	 
	Fatou's lemma implies that 
	\begin{align*}
	\int 2g \dmu 
	&= \int \limfn (2g - |f_n - f|) \dmu \\
	&\leq \limfn \int 2g - |f_n - f| \dmu  \\
	&= \int 2g \dmu - \limpn \int |f_n -f| \dmu \\
	\end{align*}
	Hence $$ \limpn \int |f_n -f| \dmu \leq 0 $$ and thus $$\int |f_n -f| \dmu \rightarrow 0$$
	\end{proof}
	
	\begin{ex} \lex{00000} 
		Let $\mu_1, \mu_2$ be measures on $(X, \MA)$. Then
		\begin{enumerate}
			\item $L^1(\mu_1 + \mu_2) = L^1(\mu_1) \cap L^1(\mu_2)$
			
			\item for each $f \in L^1(\mu_1 + \mu_2)$, we have that $$\int f d(\mu_1 + \mu_2) = \int f \dmu_1 + \int f \dmu_2$$
		\end{enumerate}
		
	\end{ex}
	
	\begin{proof}
		\begin{enumerate}
			\item The firt part is clear since similar exercise from the section on nonnegative funtions tells us that $$\int |f| d(\mu_1 + \mu_2) = \int |f| \dmu_1 + \int |f| \dmu_2$$
			
			
			\item Suppose that $f$ is simple. Then there exist $(a_n)_{i=1}^n \subset \C$ and $(E_i)_{i=1}^n \subset \MA$ such that $f = \sum\limits_{i =1}^n a_i \chi_{E_i}$. Then 
			\begin{align*}
				\int f d(\mu_1 + \mu_2) 
				&= \sum\limits_{i =1}^n a_i (\mu_1 + \mu_2)(E_i)\\
				&= \sum\limits_{i =1}^n a_i (\mu_1(E_i) + \mu_2(E_i))\\
				&= \sum\limits_{i =1}^n a_i \mu_1(E_i) + a_i \mu_2(E_i)\\
				&= \int f \dmu_1 + \int f \dmu_2
			\end{align*}
			
			Now for general $f$, choose $(\phi_n)_{n \in \N} \subset S$ such that $\phi_n \rightarrow f$ pointwise and for each $n \in \N$, $|\phi_n| \leq |\phi_{n+1}| \leq |f|$. Then dominated convergence tells us that 
			\begin{align*}
				\int f d(\mu_1 + \mu_2) 
				&= \limn \int \phi_n d(\mu_1 + \mu_2)\\
				&= \limn \int \phi_n \dmu_1 + \limn \int \phi_n \dmu_2 \\
				&= \int f \dmu_1 + \int f \dmu_2
			\end{align*}
			
		\end{enumerate}
	\end{proof}
	
	\begin{thm}
		Let $(f_n)_{n \in \N} \subset L^1$. Suppose that $$\sum_{n \in \N} \int |f_n| < \infty.$$ Then after redefinition on a set of measure zero, $\sum_{n \in \N}f_n \in L^1$ and $$\int \sum_{n \in \N}f_n = \sum_{n \in \N} \int f_n$$
	\end{thm}
	
	\begin{thm}
		Let $f \in L^1$. Then for each $\ep > 0$, there exists $\phi \in L^1$ such that $\phi$ is simple and $\int |f - \phi| < \ep$. 
	\end{thm}
	
	\begin{ex} \lex{00000} \textbf{Generalized Fatou's Lemma:}
		Let $(X, \MA, \mu)$ be a measure space and $(f_n)_{n \in \N} \subset L^0(X, \MA)$. Suppose that for each $n \in \N$, $f:X \rightarrow \R$, there exists $g \in L^1$ such that $g \geq 0$ and for each $n \in \N$, $f_n \geq -g$. Then 
		$$ \int \limfn f_n \dmu \leq \limfn \int f_n \dmu $$ 
		What is the analogue of Fatou's lemma for measurable, real valued functions that are appropriately bounded above?  
	\end{ex}
	
	\begin{proof}
		First note that for each $n \in \N$, $\int f_n$ is well defined since $f_n^- \leq g \in L^1$. Since $g + f_n \geq 0$, we may use Fatou's lemma to write
		\begin{align*}
			\int g \dmu + \int \limfn f_n \dmu 
			&= \int \limfn (g+f_n)  \dmu \\
			& \leq \limfn \int (g + f_n) \dmu \\
			&= \int g \dmu + \limfn \int f_n \dmu 
		\end{align*}
		
		Since $\int g < \infty$, $\int \limfn f_n \leq \limfn \int f_n$ as required. The analogue is as follows: Let $(f_n)_{n \in \N}$ be a sequence of measurable real valued functions. Suppose that there exists $g \in L^1$ such that $g \geq 0$ and for each $n \in \N$, $f_n \leq g$. Then $\limpn \int f_n \leq \int \limpn f_n$. To show this, just use the result from above with the sequence $(g_n)_{n \in \N}$ given by $g_n = -f_n$.
		
	\end{proof}
	
	\begin{ex} \lex{00000} 
		Let $(f_n)_{n \in \N} \subset L^1(X, \MA, \mu)$ and $f:X \rightarrow \C$. Suppose that $f_n \xrightarrow{\text{u}} f$. Then 
		\begin{enumerate}
			\item if $\mu(X) < \infty$, then $f \in L^1(X, \MA, \mu)$ and $\limn \int f_n = \int f$
			\item if $\mu(X) = \infty$, then the conclusion of $(1)$ may fail (find an example on $\R$ with Lebesgue measure).
		\end{enumerate}
	\end{ex}
	
	\begin{proof}
		Choose $N \in \N$ such that for $n \geq N$ and $x \in X$, $|f(x) - f_n(x)| < 1$. Then $||f| - |f_N|| < 1$ and so $|f| < |f_N| +1$. Thus $\int |f| \leq \int |f_N| +\mu(X) < \infty$ and $f \in L^1$. Similarly for $n \geq N$, $|f_n| < |f|+ 1$. Dominated convergence then gives us that $\limn \int f_n = \int f$ as required. To see the necessity that $\mu(X) < \infty$, consider $f \equiv 0$ and $f_n = (1/n) \chi_{(0,n)}$. Then $f_n \xrightarrow{\text{u}} f$, but $1 = \limn \int f_n \neq \int f = 0$.  
	\end{proof}
	
	\begin{ex} \lex{00000} {Generalized Dominated Convergence}
		Let $f_n,g_n,f,g \in L^1$. Suppose that $f_n \xrightarrow{\text{a.e.}} f$, $g_n \xrightarrow{\text{a.e.}} g$ and for each $n \in \N$, $|f_n| \leq g_n$. If $$\int g_n \dmu \rightarrow \int g \dmu $$ then $$\int f_n \dmu \rightarrow \int f \dmu$$.
	\end{ex}
	
	
	\begin{proof}
		We simply use Fatou's lemma. Put $h_n = (g + g_n) - |f_n - f|$. Since for each $n \in \N$, $|f_n| \leq g_n$, we know that $|f| \leq g$. So $h_n \geq 0$ and $h_n \xrightarrow{\text{p.w.}} 2g$. Thus 
		\begin{align*}
			2\int g 
			&= \int \limfn h_n\\
			&\leq \limfn \bigg[ \bigg(\int g +\int g_n\bigg) - \int |f_n -f|\bigg]\\
			&= 2\int g + \limfn \bigg( - \int |f_n - f| \bigg)\\
			&= 2\int g - \limpn \int |f_n - f| 
		\end{align*}
		
		Hence $\limpn \int |f_n - f|  \leq 0$ which implies that $\int |f_n - f| \rightarrow 0$ and $\int f_n \rightarrow \int f$ as required. 
	\end{proof}
	
	\begin{ex} \lex{00000} 
		Let $(f_n)_{n \in \N} \subset L^1$ and $f \in L^1$. Suppose that $f_n \xrightarrow{\text{a.e.}} f$. Then $\int |f_n - f| \rightarrow 0$ iff $\int |f_n| \rightarrow \int |f|$.
	\end{ex}
	
	\begin{proof}
		Suppose that $\int |f_n - f| \rightarrow 0$. Since 
		\begin{align*}
			\bigg|\int |f_n| - \int |f|\bigg| 
			&= \bigg|\int (|f_n| - |f|)\bigg|\\
			&\leq \int ||f_n| - |f||\\
			&\leq \int |f_n - f|,
		\end{align*}
		we see that $\int |f_n| \rightarrow \int |f|$. Conversely, suppose that $\int |f_n| \rightarrow \int |f|$. Put $h_n = |f_n-f|$,  $g_n = |f_n| + |f|$, $h \equiv 0$ and $g = 2f$. Then $h_n \xrightarrow{\text{a.e.}} h$, $g_n \xrightarrow{\text{a.e.}} g$ and for each $n \in \N$, $h_n \leq g_n$. Our assumption implies that $\int g_n \rightarrow \int g$. Thus the last exercise tells us that $\int h_n \rightarrow \int h$ as required. 
		
	\end{proof}
	
	\begin{ex} \lex{00000} 
		Let $(r_n)_{n \in \N}$ be an enummeration of the rationals. Define $f: \R \rightarrow \Rg$ by 
		
		\[ f(x) = \begin{cases} 
			x^{-\frac{1}{2}} & x \in (0,1) \\
			0 & x \not\in (0,1)
		\end{cases}
		\]
		
		and define $g: X \rightarrow \RG$ by $$g(x) = \sum_{n \in \N}2^{-n}f(x -r_n).$$
		
		
		Then 
		\begin{enumerate}
			\item $g \in L^1$ (perhaps after redefinition on a null set) and particularly $g < \infty$ a.e. 
			\item $g^2 < \infty$ a.e., but $g^2$ is not integrable on any subinterval of $\R$
			\item Taking $g \in L^1$, $g$ is unbounded on each subinterval of $\R$ and discontinuous everywhere and remains so after redefinition on a null set
		\end{enumerate}
	\end{ex}
	
	\begin{proof} For convenience, define $f_n: \R \rightarrow \Rg$ by $f_n(x) = f(x-r_n)$ for $x \in \R$.
		To show $(1)$ we note that for each $n \in \N$, $f_n \in L^1$ and
		\begin{align*}
			\int |2^{-n} f_n| 
			&= 2^{-n}\int_0^1 x^{-1/2}dx\\ 
			&= 2^{n-1}
		\end{align*}
		
		Hence $$\sum_{n \in \N} \int |2^{-n} f_n| = 2 < \infty.$$
		
		Therefore after redefinition on a null set, $g \in L^1.$ In particular $\int |g| < \infty$ and so $|g|$ (and hence $g$) are finite almost everywhere. For $(2)$, since $g < \infty$ a.e., so too is $g^{2}$. Let $a,b \in \R$ and suppose that $a<b$. Choose $N \in \N$ such that $r_N \in (a,b)$. Since all the terms in the sum are nonnegative, $g^{2} \geq \sum_{n \in \N} 2^{-2n}f_n^2$ and so 
		
		\begin{align*}
			\int_{(a,b)} g^2 
			&\geq \int_{(a,b)} \sum_{n \in \N} 2^{-2n}f_n^2\\
			&= \sum_{n \in \N} 2^{-2n} \int_{(a,b)} f_n^2\\
			&\geq 2^{-2N} \int_{(a,b)} f_N^2\\
			&\geq 2^{-2N} \int_{r_N}^{b \wedge (r_N+1)} \frac{1}{x-r_N} dx\\
			&= \infty
		\end{align*}
		
		So $g^2$ is not integrable on any subinterval of $\R$. For $(3)$, note that redefining $g$ on a null set does not change the result of $(2)$. Suppose that there is a finite subinterval $I \subset \R$ such that $g$ is bounded on $I$. Hence there exists $M >0$ such that for each $x \in I$, $g(x)^2 \leq M$. Then 
		\begin{align*}
			\int_I g^2
			&\leq M^2 m(I)\\
			&< \infty
		\end{align*}
		
		which is a contradiction. So $g$ is not bounded on any subinterval of $\R$. Now, suppose that there exists $x_0 \in \R$ such that $g$ is continuous at $x_0$. Choose $\del > 0$ such that for each $x \in \R$, if $|x-x_0|< \del$, then $|g(x) - g(x_0)| < 1$. The reverse triangle inequality tells us that for each $x \in (x_0-\del, x_0 +\del)$, $|g(x)| < 1 + |g(x_0)|$. Hence $g$ is bounded on $(x_0-\del, x_0 +\del)$ which is a contradiction. So $g$ is discontinuous everywhere.
	\end{proof}
	
	\begin{ex} \lex{00000} 
		Let $f \in L^1$. 
		\begin{enumerate}
			\item If $f$ is bounded, then for each $\ep >0$, there exists $\del >0$ such that for each $E \in \MA$, if $\mu(E) < \del$, then $\int_E |f| < \ep $.
			\item The same conclusion holds for general $f \in L^1$.
		\end{enumerate} 
	\end{ex}
	
	\begin{proof}
		$(1)$ Since $f$ is bounded, there exists $M >0$ such that $|f| \leq M$. Let $\ep >0$. Choose $\del = \ep/2M$. Let $E \in \MA$. Suppose that $\mu(A) < \del$. Then 
		\begin{align*}
			\int_E|f| 
			& \leq M \mu(E)\\
			&= M\frac{\ep}{2M}\\
			&= \frac{\ep}{2}\\
			&< \ep
		\end{align*}
		
		$(2)$ Suppose that $f$ is unbounded. Let $\epsilon >0$. Then there exists $\phi \in L^1$ such that $\phi$ is simple and $\int|f-\phi| < \ep/2$. Since $\phi$ is bounded, there exists $\del >0 $ such that for each $E \in \MA$, if $\mu(E) < \del$, then $\int_E |\phi| < \ep/2$. Let $E \in \MA$. Suppose that $\mu(E) < \del$. Then 
		\begin{align*}
			\int_E|f|
			& \leq \int_E |f-\phi| + \int_E |\phi|\\
			& < \ep/2 + \ep/2\\
			& = \ep
		\end{align*}   
	\end{proof}
	
	\begin{ex} \lex{00000} 
		Let $f \in L^1(\R, \ML, m)$. Define $F: \R \rightarrow \R$ by $$F(x) = \int_{(-\infty,x]}f \dm$$
		Then $F$ is continuous.
	\end{ex}
	
	\begin{proof}
		Let $x_0 \in \R$ and $\epsilon >0$. Since $f \in L^1$, there exists $\del >0$ such that for $x \in \R$, if $|x-x_0| < \del$, then $$\int_{(x \wedge x_0,x \vee x_0]}|f| \dm < \ep.$$ Let $x \in \R$. Suppose that $|x-x_0|< \del$. Then 
		\begin{align*}
			|F(x)-F(x_0)|
			&= \bigg|\int_{(x \wedge x_0,x \vee x_0]}f \dm \bigg|\\
			& \leq \int_{(x \wedge x_0,x \vee x_0]}|f| \dm \\
			& < \ep
		\end{align*} 
		
		So $F$ is continuous.
		
	\end{proof}
	
	\begin{ex} \lex{00000} 
		Let $x \in X$ and denote by $\del_x$ the point mass measure at $x \in X$ on  measurable space $(X, \MP(X))$. Let $f:X \rightarrow \C$. Then $$\int f d \del_x = f(x)$$  
	\end{ex}
	
	\begin{proof}
		First assume that $f$ is simple. Then there exist $(a_j)_{j=1}^n \subset \C$ and $(E_j)_{j=1}^n \subset \MP(X)$ such that $(E_j)_{j=1}^n$ is disjoint and $f = \sum_{i = 1}^n a_i\chi_{E_i}$. Choose $j^* \in \{1, \ldots, n\}$ such that $x \in E_{j^*}$. Thus 
		\begin{align*}
		\int f d\del_x 
		&= \int \sum_{j=1}^n c_j \chi_{E_j} d \del_x \\
		&= \sum_{j=1}^n c_j \del_x(E_j) \\
		&= c_j \del_x(E_{j^*}) \\
		&= c_j \\
		&= f(x)
		\end{align*} 
		Now for $f \in L^+$, choose a sequence $(\phi_n)_{n \in \N} \subset S^+$ such that for each $n \in \N$, $\phi_n \leq \phi_{n+1}$ and $\phi_n \xrightarrow{\text{p.w}} f$. Then monotone convergence implies that 
		\begin{align*}
		\int f d \del_x 
		&= \int \limn \phi_n  \del_x \\
		&= \limn \int \phi_n \del_x \\
		&= \limn \phi_n(x) \\
		&= f(x) 
		\end{align*}
		Now just extend to complex valued functions.
		
	\end{proof}
	
	\begin{ex} \lex{00000} 
		Denote by $\#$ the counting measure on the measurable space $(X, \MP(X))$. Let $f:X \rightarrow \C$ and suppose that $f \in L^1$. Then $$\int f d\# = \sum_{x \in X}f(x).$$ In particular, if $f$ is integrable, then $\{x \in X: f(x) \neq 0\}$ is countable.
	\end{ex}
	
	\begin{proof} Please refer to the definition of the sum in the appendix. First suppose that $f(X) \subset \Rg$. For $n \in \N$, put $X_n = \{x \in X: f(x) > 1/n\}$ and define $X_+ = \{x \in X: f(x) > 0\}$, $X_0 = \{x \in X: f(x) = 0\}$ Then $X_+ = \bigcup\limits_{n \in \N}X_n$. Since $f \in L^1$, we have that for each $n \in \N$,
		\begin{align*}
			\infty 
			&> \int f d\#\\
			&\geq \int_{X_n} f d\# \\
			&\geq \frac{1}{n} \#(X_n).
		\end{align*}
		
		Thus for each $n \in \N$, $X_n$ is finite and $X_+$ is countable. Thus there exists $\{x_n\}_{n \in \N} \subset X$ such that $X_+ = \{x_n\}_{n \in \N}$. For $n \in \N$, define $E_n = \{x_1, x_2, \cdots, x_n\}$ and 
		\begin{align*}
			f_n 
			&= f \chi_{E_n}\\
			&= \sum_{i = 1}^n f(x_i)\chi_{\{x_i\}}
		\end{align*}
		
		Then $f_n \xrightarrow{\text{p.w.}} f\chi_{X_+} = f$ and for each $n \in \N, f_n \leq f_{n+1}$. So
		\begin{align*}
			\int f 
			&= \sup_{n \in \N} \int f_n\\
			&= \sup_{n \in \N} \sum_{i =1}^n f(x_i)\\
			&= \sum_{x \in X_+} f(x)\\
			&=\sum_{x \in X} f(x).
		\end{align*} 
		
		For $f:X \rightarrow \C$, our $L^1$ assumption and the result above tell us that $$\sum_{x \in X}|f(x)| < \infty.$$ Thus writing $f = g+ih$, we see that the same is true for $f^+,f^-,g^+,g^-$. Simply using the definitions of the sum and the integral, as well as the result from above, we have that $$\int fd\# = \sum_{x \in X}f(x).$$
	\end{proof}
	
	\begin{ex} \lex{00000} 
		Let $f,g:X \rightarrow \R$. Suppose that $f,g \in L^1$. Then $f \leq g$ a.e. iff for each $E \in \MA$, $$\int_E f \leq \int_E g$$  
	\end{ex}
	
	\begin{proof}
		Suppose $f \leq g$ a.e. Put $N = \{x\in X: f(x) > g(x)\} \subset N$. Then $\mu(N) = 0$ and $g-f \geq 0$ on $N^c$. So for each $E \in \MA$,
		\begin{align*}
			\int_E g \dmu - \int_E f \dmu 
			&= \int_E (g-f) \dmu\\
			&= \int_{E \cap N^c} (g-f) \dmu\\
			& \geq 0
		\end{align*} 
		Conversely, suppose that for each $E \in \MA$, 
		$$\int_E f\dmu \leq \int_E g \dmu$$ 
		Put $N_n = \{x \in X: f(x) - g(x) > 1/n\}$ and $N = \{x \in X: f(x) > g(x)\}$. Then $N = \bigcup\limits_{n \in \N}N_n$. Let $n \in \N$. Then our assumption tells us that 
		\begin{align*}
			0 
			&\geq \int_{N_n} f-g\\
			& \geq \frac{1}{n} \mu(N_n)\\
			& \geq 0.
		\end{align*} 
		So that $\mu(N_n) = 0$. Thus for each $n \in \N$, $\mu(N_n) = 0$ which implies $\mu(N) = 0$. Therefore $f \leq g$ a.e. as required. 
	\end{proof}
	
	\begin{ex} \lex{00000} 
	Let $(X, \MA, \mu)$ be a measure space and $f:X \times \R \rightarrow \C$. Suppose that for each $t \in \R$, $f(\cdot, t) \in L^1(\mu)$. Define $F: \R \rightarrow \C$ by 
	$$F(t) = \int_X f(x, t) \dmu(x)$$ 
	\begin{enumerate}
	\item Suppose that there exists $g \in L^1(\mu)$ such that for each $(x, t) \in X \times \R$, $|f(x,t)| \leq g(x)$. Let $t_0 \in \R$. If for each $x \in X$, $f(x, \cdot)$ is continuous at $t_0$, then $F$ is continuous at $t_0$. 
 	\item Suppose that $\p f / \p t$ exits and there exists $g \in L^1(\mu)$ such that for each $(x, t) \in X \times \R$, $|\p f / \p t(x,t) | \leq g(x)$. Then $F$ is differentiable and for each $t \in \R$, $$F'(t) = \int_X \frac{\p f}{\p t}(x, t) \dmu(x)$$
	\end{enumerate}
	\end{ex}
	
	\begin{proof}\
	\begin{enumerate}
	\item Suppose that for each $x \in X$, $f(x, \cdot)$ is continuous at $t_0$. Let $(t_n) \subset \R$. Suppose that $t_n \rightarrow t_0$. Then $f(\cdot , t_n) \convt{p.w.} f(\cdot, t_0)$. Since for each $n \in \N$, $|f(x,t_n)| \leq g(x)$, the dominated convergence theorem implies that $F(t_n) \rightarrow F(t_0)$.
	\item Let $t_0 \in \R$. Choose $(t_n)_{n \in \N}$ such that $ t_n \rightarrow t_0$ and for each $n \in \N$, $t_n < t_0$. For $n \in \N$, define $q_n:X \rightarrow \R$ by $$q_n(x) = \frac{f(x,t_n) - f(x, t_0)}{t_n - t_0}$$ So $q_n(\cdot) \convt{p.w.} \p f / \p t (\cdot, t_0)$. The mean value theorem implies that for each $x \in X$ and $n \in \N$, there exists $c_{n,x} \in (t_n,t_0)$ such that $q_n(x) = \p f / \p t (x, c_{n,x})$. Then for each $n \in \N$, $|q_n| \leq g$. The dominated convergence theorem then implies that $\p f / \p t (\cdot, t_0) \in L^1(\mu)$ and 
	\begin{align*}
	\int \frac{\p f }{\p t} (x, t_0) \dmu(x) 
	&=  \limn \int_X q_n \dmu  \\
	&= \limn \frac{F(t_n) - F(t_0)}{t_n - t_0} \\
	&= F'(t_0^-) 	
	\end{align*}
	So that $F$ is differentiable at $t_0$ from the left. Similarly, $F$ is differentiable at $t_0$ from the right. 
	\end{enumerate}
	\end{proof}

	\begin{ex}
		Let $(X, \MA, \mu)$ be a measure space, $(Y, \MB)$ a measurable space and $f: X \rightarrow Y$. Suppose that $f$ is $(\MA, \MB)$-measurable. Then for each $g \in L^0(Y, \MB)$ and $B \in \MB$, 
		\begin{enumerate}
			\item $g \circ f \in L^1(X, \MA)$ iff $g \in L^1(Y, \MB, f_* \mu)$
			\item if $g \circ f \in L^1(X, \MA, \mu)$, 
			$$\int_{f^{-1}(B)} g \circ f \dmu = \int_B g \, d f_*\mu$$
		\end{enumerate}
	\end{ex}

	\begin{proof} Let $g \in L^0(Y, \MB)$ and $B \in \MB$.
		\begin{enumerate}
			\item Suppose that $g \circ f \in L^1(X, \MA, \mu)$. Since $|g| \in L^+(X, \MA)$ and $|g \circ f| = |g| \circ f$, an exercise in the previous section implies that 
			\begin{align*}
				\int_B |g|  \, d f_*\mu  
				& = \int_{f^{-1}(B)} |g| \circ f \dmu \\
				& = \int_{f^{-1}(B)} |g \circ f| \dmu \\
				& < \infty 
			\end{align*}
			Hence $g \in L^1(Y, \MB, f_*\mu)$. \\
			Conversely, suppose that $g \in L^1(Y, \MB, f_*\mu)$. Since $|g \circ f| \in L^+(X, \MB)$, we have that 
			\begin{align*}
				\int_{f^{-1}(B)} |g \circ f| \dmu   
				& = \int_{f^{-1}(B)} |g| \circ f \dmu \\
				& = \int_B |g|  \, d f_*\mu \\
				& < \infty 
			\end{align*}
			Hence $g \circ f \in L^1(X, \MA, \mu)$. \\
			\item Suppose that $g \circ f \in L^1(X, \MA, \mu)$. Write $g = h_1^+ - h_1^- + i(h_2^+ - h_2^-)$. Since $h_1^+, h_1^-, h_2^+, h_2^- \in L^+(Y, \MB)$, an exercise in the previous section implies that 
			\begin{align*}
				\int_{f^{-1}(B)} g \circ f \dmu
				& = \int_{f^{-1}(B)} \bigg[ h_1^+ - h_1^- + i(h_2^+ - h_2^-) \bigg] \circ f \dmu \\
				& = \int_{f^{-1}(B)}  h_1^+ \circ f \dmu - \int_{f^{-1}(B)}  h_1^- \circ f \dmu \\
				& + i \int_{f^{-1}(B)} h_2^+ \circ f \dmu -i \int_{f^{-1}(B)}  h_2^- \circ f \dmu \\
				& = \int_B h_1^+ \, d f_*\mu - \int_B h_1^- \, d f_*\mu + i \int_B h_2^+ \, d f_*\mu - i \int_B h_2^- \, d f_*\mu \\
				& = \int_B  h_1^+ - h_1^- + i(h_2^+ - h_2^-)  \, d f_*\mu \\
				& = \int_B g \, d f_*\mu
			\end{align*}
		\end{enumerate}
	\end{proof}
	
	\begin{defn} \ld{00000} 
		Let $\MF \subset L^1$. Then $\MF$ is said to be \textbf{uniformly integrable} if for each $\ep >0$, there exists $K \in \N$ such that for each $k \in \N$, if $k \geq K$, then $\sup\limits_{f \in \MF} \int_{\{|f|>k\}}|f| < \ep$. (i.e. $\lim\limits_{k \rightarrow \infty} \sup\limits_{f \in \MF} \int_{\{|f| > k\}} |f| = 0$).
	\end{defn}
	
	\begin{ex} \lex{00000} 
		
		Suppose that $\mu$ is finite. Let $\MF \subset L^1$. Then $\MF$ is uniformly integrable iff 
		\begin{enumerate}
			\item there exists $M >0$ such that $\sup\limits_{f \in \MF}\int |f| \leq M$
			\item for each $\ep >0$, there exists $\del >0$ such that for each $E \in \MA$, if $\mu(E) < \del$, then $\sup\limits_{f \in \MF} \int_E |f| < \ep$.
		\end{enumerate}
	\end{ex}
	
	\begin{proof}
		($\implies$): (1) Suppose that $\MF$ is uniformly integrable. Then there exists $K \in \N$ such that for each $k \in \N$, if $k \geq K$, then $\sup\limits_{f \in \MF} \int_{\{|f|>k\}} |f| < 1$. Choose $M = \mu(X)K + 1$. Then for each $f \in \MF$, 
		\begin{align*}
			\int |f| 
			&= \int_{\{|f|>K\}} |f| + \int_{\{|f| \leq K|\}}|f|\\
			& \leq 1 + K\mu(X)\\
			&=M
		\end{align*}
		
		(2) Let $\ep >0$. Then choose $K \in \N$ such that $\sup\limits_{f \in \MF}\int_{\{|f|>K\}} |f| < \ep/2$ and choose $\del = \ep/2K$. Let $E \in \MA$. Suppose that $\mu(E) < \del$. Then for $f \in \MF$, 
		\begin{align*}
			\int_E |f| 
			&= \int_{E \cap \{|f| > K\}} |f| + \int_{E \cap \{|f| \leq K\}} |f|\\
			& \leq \ep/2 + K\del \\
			&=  \ep
		\end{align*}
		
		($\Leftarrow$): Choose $M >0$ as in (1). Suppose that there exists $\ep >0$ such that for each $K \in \N$, there exists $f \in \MF$ such that $\mu(\{|f| > K\}) \geq \ep$. Choose $K \in \N$ such that $K > M/\ep$. Then choose $f_K \in \MF$ such that $\mu(\{|f_K| > K\}) \geq \ep$. Then 
		\begin{align*}
			\int |f_K| 
			&\geq \int_{\{|f_K| > K\}} |f|\\
			& \geq K\mu(\{|f_K| > K\})\\
			& > \frac{M}{\ep} \cdot \ep\\
			&= M, \\
		\end{align*}  
		which is a contradiction. Hence for each $\ep >0$, there exists $K \in \N$ such that for each $f \in \MF$, $\mu(\{|f| > K\}) < \ep$. Since $\mu(\{|f| > k\})$ is a decreasing sequence in $k$, we have that $\lim\limits_{k \rightarrow \infty} \sup\limits_{f \in \MF} \mu(\{|f| > k\}) = 0$. Now, let $\ep > 0$. Choose $\del >0$ as in (2). Choose $K \in \N$ such that for each $k \in \N$, if $k \geq K$, then for each $f \in \MF$, $\mu(\{|f| > k\}) < \del$. Then for each $k \in \N$, if $k \geq K$, then for each $f \in \MF$, 
		$$\int_{\{|f| > k\}} |f| < \ep.$$ Thus $$\lim\limits_{k \rightarrow \infty} \sup\limits_{f \in \MF} \int_{\{|f|>k\}} |f| = 0$$ as required.
		
	\end{proof}
	
	\begin{defn}
	Let $(X, \MA, \mu)$ be a measure space. Define $\| \cdot \|_*: L^1(\mu) \rightarrow \Rg$ by  
	 $$\| f \|_* = \sup_{A \in \MA} \bigg | \int_A f \dmu \bigg |$$ 
	\end{defn}
	
	\begin{ex}
	Let $(X, \MA, \mu)$ be a measure space. Then $\| \cdot \|_*$ is a norm on $L^1(\mu)$ and there exists $C >0$ such that $C\|\cdot\|_1 \leq \|\cdot\|_* \leq \| \cdot \|_1$.
	\end{ex}
	
	
	
	
	
	
	
	
	
	
	\newpage
	\subsection{Integration on Product Spaces}
	
	\begin{defn} \ld{00000} 
		Let $X$, $Y$, and $Z$ be sets, $E \subset X \times Y$ and $f :X \times Y \rightarrow Z$. For each $x \in X$, define $E_x = \{y \in Y: (x,y) \in E\}$ and $f_x:Y \rightarrow Z$ by $f_x(y) = f(x,y)$. For each $y \in Y$, define $E^y = \{x \in X: (x,y) \in E\}$ and $f^y:X \rightarrow Z$ by $f^y(x) = f(x,y)$. 
	\end{defn}
	
	\begin{note}
		It is often helpful to observe that $(\chi_E)_x = \chi_{E_x}$ and $(\chi_E)^y = \chi_{E^y}$.
	\end{note}
	
	\begin{lem}
		Let $(X,\MA), (Y, \MB)$ be measurable spaces, $Z = \RG$ or $\C$ and $f:X\times Y \rightarrow Z$. 
		\begin{enumerate}
			\item For each $E \in \MA \otimes \MB$, $x \in X$, $y \in Y$, we have that $E_x \in \MB$ and $E^y \in \MA$
			\item If $f$ is  $\MA \otimes \MB$-measurable, then for each $x \in X$, $y \in Y$, we have that $f_x$ is $\MB$-measurable and $f^y$ is $\MA$-measurable.   
		\end{enumerate}
	\end{lem}
	
	\begin{thm}
		Let $(X,\MA, \mu), (Y, \MB, \nu)$ be $\sig$-finite measure spaces. Then for each $E \in \MA \otimes \MB$, the maps $\phi:X \rightarrow \RG$ and $\psi: Y \rightarrow \RG$ defined by $\phi(x) = \nu(E_x)$ and $\psi(y) = \mu(E^y)$ are $\MA$-measurable and $\MB$-measurable, respectively and $$\mu \times \nu(E) = \int_X \nu(E_x)\dmu(x) = \int_Y \mu(E^y)d\nu(y)$$ 
	\end{thm}
	
	\begin{thm}\textbf{Fubini, Tonelli:}
		Let $(X,\MA, \mu), (Y, \MB, \nu)$ be $\sig$-finite measure spaces. 
		
		\begin{enumerate}
			\item (Tonelli) For each $f \in L^+(X \times Y)$, the functions $g:X \rightarrow \RG$, $h:Y \rightarrow \RG$ defined by $g(x) = \int_Y f_x(y)d\nu(y)$ and $h(y) = \int_X f^y(x) \dmu(x)$ are $\MA$-measurable and $\MB$-measurable respectively and $$\int_{X \times Y}f \dmu \times \nu = \int_X g \dmu = \int_Y h d\nu$$
			
			\item (Fubini) For each $f \in L^1(X \times Y)$, $f_x \in L^1(\nu)$ for $\mu$-a.e. $x \in X$ and $f^y \in L^1(\mu)$ for $\nu$-a.e. $y \in Y$, respectively and  the functions (after redefinition of $f$ on a null set) $g:X \rightarrow \C$, $h:Y \rightarrow \C$ defined by $g(x) = \int_Y f_x(y)d\nu(y)$ and $h(y) = \int_X f^y(x) \dmu(x)$ are in $L^1(\mu)$ and $L^1(\nu)$ respectively. Furthermore 
			$$\int_{X \times Y}f \dmu \times \nu = \int_X g \dmu = \int_Y h d\nu$$
		\end{enumerate}
	\end{thm}
	
	\begin{note}
		We usually just write $\int \int f \dmu d\nu$ and $\int \int f d\nu \dmu$ instead of $\int h d\nu$ and $\int g \dmu$ respectively. We have a similar result for complete product measure spaces. See 
	\end{note}
	
	\begin{ex} \lex{00000} 
		Take $X=Y= [0,1]$, $\MA = \MB([0,1]), \MB = \MP([0,1])$ and $\mu,\nu$ to be Lebesgue measure and counting measure respectively. Define $D = \{(x,y) \in [0,1]^2: x=y\}$ Show that $$\int \chi_D \dmu \times \nu, \int \int \chi_D \dmu d \nu \text{ and } \int \int \chi_D d\nu \dmu$$ are all different. (Hint: for the first integral, use the definition of $\mu \times \nu$)
	\end{ex}
	
	\begin{proof}
		Let $x,y \in [0,1]$. Then $(\chi_D)_x = \chi_{D_x} = \chi_{x}$ and $(\chi_D)^y = \chi_{D^y} = \chi_{y}$. Thus
		
		\begin{align*}
			\int \int \chi_D \dmu d \nu
			&= \int \mu(\{y\}) d\nu\\
			&= \int 0 d\nu\\
			&= 0
		\end{align*}
		
		and
		
		\begin{align*}
			\int \int \chi_D \dmu d \nu
			&= \int \nu(\{x\}) \dmu\\
			&= \int 1 \dmu\\
			&= 1
		\end{align*}
		
		Now, Observe that $\int \chi_D \dmu \times \nu = \mu \times \nu(D)$. Recall from the section on product measures that $\mu \times \nu(D) = \inf \{\sum_{n \in \N}\mu(A_n)\nu(B_n): (A_n \times B_n)_{n \in \N} \subset \ME \text{ and } D \subset \bigcup_{n \in \N} A_n \times B_n \}$. Let $(A_n \times B_n)_{n \in \N} \subset \ME$. Suppose that $D \subset \bigcup_{n \in \N}A_n \times B_n$. Then for each $x \in [0,1]$, $(x,x) \in  \bigcup_{n \in \N} A_n \times B_n$. So for each $x \in [0,1]$, there exists $n \in \N$, such that $x \in A_n \cap B_n$. Thus $[0,1] \subset \bigcup_{n \in \N} A_n \cap B_n.$ Since $1  = \mu([0,1]) \leq \sum_{n \in \N}\mu(A_n \cap B_n)$, we know that there exists $n \in \N$ such that $0 < \mu(A_n \cap B_n)$. Thus $\mu(A_n)> 0$ and $\mu(B_n) > 0$. Since $\mu(B_n) > 0$, $B_n$ must be infinite and therefore $\nu(B_n) = \infty$. So $\sum_{n \in \N} \mu(A_n)\nu(B_n) = \infty$.
		
	\end{proof}
	
	\begin{ex} \lex{00000} 
		Let $(X, \MA, \mu)$ be a $\sig$-finite measure space and $f:X \rightarrow \Rg \in L^+$. Show that $G = \{(x,y) \in X \times \Rg: f(x) \geq y\} \in \MA \otimes \MB(\Rg)$ and $\mu \times m (G) = \int_X f \dmu$. The same is true if we replace "$\geq$" with "$>$". (Hint: to show that $G$ is measurable, split up $(x,y) \mapsto f(x) - y$) into the composition of measurable functions. 
	\end{ex}
	
	\begin{proof}
		Define $\phi: X \times \Rg \rightarrow \Rg^2$ and $\psi: \Rg^2 \rightarrow \Rg$ by $\phi(x,y) = (f(x),y)$ and $\psi(z,y) = z-y$. Then $G = \{(x,y) \in X \times \Rg: \psi \circ \phi(x,y) \geq 0\}$. Let $A, B \in \MB(\Rg)$. Then $\phi^{-1}(A \times B) = f^{-1}(A) \times B \in \MA \times \MB(\Rg)$. Since $\MB(\Rg^2) = \MB(\Rg) \otimes \MB(\Rg) = \sig(\{A \times B: A, B \in \MB(\Rg)\})$, we have that $\phi$ is $\MA \otimes \MB(\Rg)$-$\MB(\Rg^2)$ measurable. Since $\psi$ is continuous, we have that $\psi$ is $\MB(\Rg^2)$-$\MB(\Rg)$ measurable. This implies that $\psi \circ \phi$ is $\MA \otimes \MB(\Rg)$-$\MB(\Rg)$ measurable. Thus $G = \psi \circ \phi^{-1}(\Rg) \in \MA \otimes \MB(\Rg)$. Now for $x \in X$, $G_x = \{y \in \Rg: f(x) \geq y\} = [0, f(x)]$. Thus 
		
		\begin{align*}
			\mu \times m(G) 
			&= \int \chi_G \dmu \times m\\
			&= \int_X \int_{\Rg} \chi_{G_x} \dm \dmu(x)\\
			&= \int_X f(x) \dmu(x) 
		\end{align*}
		
		The same reasoning holds if we replace "$\geq$" with "$>$".
	\end{proof}
	
	\begin{ex} \lex{00000} 
		Let $(X, \MA, \mu), (Y, \MB, \nu)$ be $\sig$-finite measure spaces and $f:X \rightarrow \C$, $g:Y \rightarrow \C$. Define $h:X \times Y \rightarrow \C$ by $h(x,y) = f(x)g(y)$.
		
		\begin{enumerate}
			\item If $f$ is $\MA$-measurable and $g$ is $\MB$-measurable, then $h$ is $\MA \otimes \MB$-measurable.
			
			\item If $f \in L^1(\mu)$ and $g \in L^1(\nu)$, then $h \in L^1(\mu \times \nu)$ and $$\int_{X \times Y}h\dmu \times \nu = \int_X f \dmu \int_Y g d\nu$$
		\end{enumerate}
	\end{ex}
	
	\begin{proof}\
		\begin{enumerate}
			\item First suppose that $f$, $g$ are simple. Then there exist $(A_i)_{i=1}^n \subset \MA$, $(B_j)_{j=1}^m \subset \MB$ and $(a_i)_{i=1}^n, (b_i)_{j=1}^m \subset \C$ such that $f = \sum_{i=1}^n a_i \chi_{A_i}$ and $g = \sum_{j=1}^m b_j \chi_{B_j}$. Then $h = \sum_{i=1}^n \sum_{j=1}^m a_i b_j \chi_{A_i \times B_j}$. So $h$ is $\MA \otimes \MB$-measurable. For general $f,g$, there exist $(f_n)_{n \in \N} \subset S(X, \MA)$ and $(g_n)_{n \in \N} \subset S(Y, \MB)$ such that $f_n \rightarrow f$ pointwise, $g_n \rightarrow g$ pointwise and for each $n \in \N$, $|f_n| \leq |f_{n+1}| \leq |f|$ and $|g_n| \leq |g_{n+1}| \leq |g|$. For $n \in \N$, define $h_n \in S(X \times Y, \MA \otimes \MB)$ by $h_n = f_n g_n$. Then $h_n \rightarrow h$ pointwise and for each $n \in \N$, $|h_n| \leq |h_{n+1}| \leq |h|$. Thus $h$ is $\MA \otimes \MB$-measurable.
			
			\item First suppose $f$ and $g$ are simple as before. Then  
			\begin{align*}
				\int_{X \times Y} |h| \dmu \times \nu 
				& \leq \sum_{i=1}^n \sum_{j=1}^m |a_i b_j| \mu(A_i) \nu(B_j)\\ 
				&= \big(\sum_{i=1}^n |a_i| \mu(A_i) \big) \big( \sum_{j=1}^m |b_j| \nu(B_j) \big)\\
				&= \int_X |f| \dmu \int_Y |g| d \nu\\
				&< \infty
			\end{align*}
			
			So $h \in L^1(\mu \times \nu)$. Furthermore, 
			
			\begin{align*}
				\int_{X \times Y} h \dmu \times \nu 
				&= \sum_{i=1}^n \sum_{j=1}^m a_i b_j \mu(A_i) \nu(B_j)\\ 
				&= \big(\sum_{i=1}^n a_i \mu(A_i) \big) \big( \sum_{j=1}^m b_j \nu(B_j) \big)\\
				&= \int_X f \dmu \int_Y gd \nu
			\end{align*}
			
			For general $f \in L^1(\mu), g \in L^1(\nu)$, take $(h_n)_{n \in \N}$ as before. Monotone convergence and the result above say that 
			
			\begin{align*}
				\int_{X \times Y} |h| \dmu \times d\nu 
				&= \limn \int_{X \times Y} |h_n|\dmu \times \nu\\
				&=  \limn \bigg( \int_X |f_n| \dmu \int_Y |g_n| d\nu \bigg) \\
				&= \int_X |f| \dmu \int_Y |g| d\nu\\
				& < \infty
			\end{align*}
			
			So $h \in L^1(\mu \times \nu)$. Dominated convergence and the result above then tell us that 
			
			\begin{align*}
				\int_{X \times Y} h \dmu \times d\nu 
				&= \limn \int_{X \times Y} h_n \dmu \times d\nu \\
				&= \limn \bigg( \int_X f_n \dmu \int_Y g_n d\nu \bigg)\\
				&= \int_X f \dmu \int_Y g d\nu
			\end{align*}
			
		\end{enumerate}
	\end{proof}
	
	\begin{note}
		In the above exercise part (2), we can replace $L^1$ with $L^+$ and get the same result by the same method.
	\end{note}
	
	\begin{ex} \lex{00000} 
		Let $f:\R \rightarrow \Rg \in L^+$. Show that $$\int_{\R}f\dm = \int_{\Rg}m(\{x \in \R: f(x) \geq t\}) \dm(t)$$
	\end{ex}
	
	\begin{proof}
		Note that $$\int_{\Rg}m(\{x \in \R: f(x) \geq t\}) = \int_{\Rg} \bigg[\int_{\R} \chi_{\{x \in \R: f(x) \geq t\}}\dm \bigg]\dm(t)$$
		Comparing this with Tonelli's theorem, we can put $\chi_{\{x \in \R: f(x) \geq t\}} = (\chi_{E})^t = \chi_{E^t}$. Then $E = \{(x,t) \in \R \times \Rg: f(x) \geq t\}$ and $E_x = \{t \in \Rg: f(x) \geq t\} = [0,f(x)]$. Tonelli's theorem tells us that 
		\begin{align*}
			\int_{\Rg} \bigg[\int_{\R} \chi_{\{x \in \R: f(x) \geq t\}}(x) \dm(x) \bigg]\dm(t)
			&= \int_{\R} \bigg[ \int_{\Rg} \chi_{[0,f(x)]}(t) \dm(t) \bigg] \dm(x)\\
			&= \int_{\R} f(x) \dm(x)
		\end{align*} 
	\end{proof}
	
	
	
	
	
	
	
	
	
	
	
	
	
	
	\newpage
	\subsection{Modes of Convergence}
	
	\begin{defn} \ld{35001} 
		Let $(X, \MA, \mu)$ be a measure space, $(Y,d)$ a metric space, $(f_n)_{n \in \N} \subset L_Y^0(X, \MA, \mu)$ and $f \in L_Y^0(X, \MA, \mu)$. Then $(f_n)_{n \in \N}$ is said to \textbf{converge to $f$ in measure}, denoted $f_n \xrightarrow{\mu} f$, if for each $\ep > 0$, $$\mu(\{x \in X: d(f_n(x), f(x)) \geq \ep \}) \rightarrow 0 \hspace{.2cm }\text{ as } n \rightarrow \infty$$
	\end{defn}
	
	\begin{defn} \ld{35002} 
	Let $(f_n)_{n \in \N} \subset L^0$. Then $(f_n)_{n \in \N}$ is said to be \textbf{Cauchy in measure} if for each $\ep >0$, $$\mu(\{x \in X: |f_n(x) - f_m(x)| \geq \ep \}) \rightarrow 0 \hspace{.2cm }\text{ as } n,m \rightarrow \infty$$ 
	i.e. for each $\ep, \del >0$, there exists $N \in \N$ such that for each $n,m \in \N$, $n,m \geq N$ implies that $\mu(\{x \in X: |f_n(x) - f_m(x)| \geq \ep \}) < \del$.
	\end{defn}
	
	\begin{note}
		It is useful to observe that 
		$$\bigcup_{\ep >0}\limsup\limits_{n \rightarrow \infty} \{x \in X: |f_n(x) - f(x)| \geq \ep \} = \{x \in X: f_n(x) \not \rightarrow f(x) \}$$ 
		and 
		$$\bigcap_{\ep > 0} \liminf_{n \rightarrow \infty}\{x \in X: |f_n(x) - f(x)| < \ep \} = \{x \in X: f_n(x) \rightarrow f(x) \}$$ 
	\end{note}
	
	\begin{ex} \lex{35002.1} 
	Let $(X, \MA, \mu)$ be a measure space, $(f_n)_{n \in \N} \subset L^0$ and $f \in L^0$. If $f_n \conv{\mu} f$, then $(f_n)_{n \in \N}$ is Cauchy in measure.
	\end{ex}
	
	\begin{proof}
	Suppose that $f_n \conv{\mu} f$. For $\ep >0$ and $n,m \in \N$, set 
	$$A_{n, \ep} = \{x \in X: |f_n(x) - f(x)| \geq \ep \}$$ 
	and 
	$$B_{n,m, \ep} = \{x \in X: |f_n(x) - f_m(x)| \geq \ep \}$$
	Let $\ep >0$, $n, m \in \N$ and $x \in A_{n, \frac{\ep}{2}}^c \cap A_{m, \frac{\ep}{2}}^c$. Then 
	\begin{align*}
	|f_n(x) - f_m(x)| 
	& \leq  |f_n(x) - f(x)| + |f(x) - f_m(x)| \\
	& < \frac{\ep}{2} + \frac{\ep}{2} \\
	&= \ep  
	\end{align*}
	and $x \in B_{n,m, \ep}^c$. Therefore $A_{n, \frac{\ep}{2}}^c \cap A_{m, \frac{\ep}{2}}^c \subset B_{n,m, \ep}^c$. This implies that $B_{n,m, \ep} \subset A_{n, \frac{\ep}{2}} \cup A_{m, \frac{\ep}{2}}$. Let $\del >0$. Choose $N \in \N$ such that for each $n \in \N$, $n \geq N$ implies that $\mu(A_{n, \frac{\ep}{2}}) < \del/2$. Then for each $n,m \in \N$, $n, m \geq N$ implies that 
	\begin{align*}
	\mu(B_{n,m, \ep}) 
	&\leq \mu(A_{n, \frac{\ep}{2}}) + \mu(A_{m, \frac{\ep}{2}}) \\
	& < \frac{\del}{2} + \frac{\del}{2} \\
	&= \del
	\end{align*}
	So for each $\ep >0$, $$\mu(\{x \in X: |f_n(x) - f_m(x)| \geq \ep \}) \rightarrow 0 \hspace{.2cm }\text{ as } n,m \rightarrow \infty$$  
	and $(f_n)_{n \in \N}$ is Cauchy in measure.
	\end{proof}
	
	\begin{ex}
	Let $(f_n)_{n \in \N} \subset L^0$ and $f, g \in L^0$. Suppose that $f_n \conv{\mu} f$ and $f_n \conv{\mu} g$. Then $f = g$ a.e. 
	\end{ex}
	
	\begin{proof}
	Set $B = \{x \in X: |f(x) - g(x)|  \geq 0\}$ and for $n, k \in \N$, set 
	\begin{itemize}
	\item $B_k = \{x \in X: |f(x) - g(x)| \geq  \frac{1}{k}\}$
	\item $A_{f,n,k} = \{x \in X: |f_n(x) - f(x)| \geq  \frac{1}{k}\}$
	\item $A_{g,n,k} = \{x \in X: |f_n(x) - g(x)| \geq  \frac{1}{k}\}$
	\end{itemize} 
	As in the proof of \rex{35002.1}, for each $n, k \in \N$
	$$\mu(B_k) \leq \mu(A_{f,n,2k}) + \mu(A_{g, n, 2k})$$ 
	Let $\ep >0$. Convergence in measure implies that for each $k \in \N$, there exists $N_k \in \N$ such that for each $n \in \N$, $n \geq N$ implies that $\mu(A_{f,n,2k}), \mu(A_{g,n,2k}) < \ep 2^{-(1+k)}$.
	Then 
	\begin{align*}
	\mu(B)
	&= \mu \bigg( \bigcup_{k \in \N} B_k \bigg) \\
	& \leq \sum_{k \in \N} \mu(B_k) \\
	& \leq  \sum_{k \in \N}\mu(A_{f,N_k,2k}) + \sum_{k \in \N}\mu(A_{g, N_k, 2k}) \\
	&\leq  \sum_{k \in \N} \ep 2^{-(1+k)} + \sum_{k \in \N} \ep 2^{-(1+k)} \\
	&= \frac{\ep}{2} + \frac{\ep}{2} \\
	&= \ep
\end{align*}	 
	Since $\ep >0$ is arbitrary, $\mu(B) = 0$ and $f = g$ a.e.
	\end{proof}
	
	\begin{ex} \lex{35003} 
		Let $(f_n)_{n \in \N} \subset L^0$. Suppose that $(f_n)_{n \in \N}$ is Cauchy in measure. 
		\begin{enumerate}
		\item There exists a subsequence $(f_{n_j})_{j \in \N} \subset (f_n)_{n \in \N}$ such that for each $j \in \N$, $$\mu(\{x \in X: |f_{n_j}(x) - f_{n_{j+1}}(x)| \geq 2^{-j}\}) < 2^{-j}$$
		\item For $j,k \in \N$ set 
		$$E_j = \{x \in X: |f_{n_j}(x) - f_{n_{j+1}}(x)| \geq 2^{-j}\}$$
		and 
		$$F_k = \bigcup_{j \geq k}E_j$$
		Then $(F_k)_{k \in \N}$ is decreasing and for each $k \in \N$, $\mu(F_k) \leq 2^{1-k}$ and for each $i, j, k \in \N$, $i \geq j \geq k$ implies that for each $x \in F_k^c$, $$|f_{n_i}(x) - f_{n_j}(x)| \leq 2^{1-k}$$ 
		So for each $k \in \N$, $(f_{n_j})_{j \in \N}$  is uniformly Cauchy on $F_k^c$ and therefore $(f_{n_j})_{j \in \N}$  is pointwise Cauchy on $F_k^c$. \\
		\textbf{Hint:} get a telescoping sum via the triangle inequality 
		\item Set $$F = \bigcap_{k \in \N} F_k$$ 
		Then $\mu(F) = 0$ and there exists $f \in L^0$ such that $f_{n_j} \convt{a.e.} f$.  
		\item Finally, $f_{n_j} \conv{\mu} f$, $f_n \conv{\mu} f$ \\
		\textbf{Hint:} consider showing $\{x \in X: |f_{n_k}(x) - f(x)| \geq \ep\} \subset F_k$ and use something similar to the proof of \rex{35002.1}
		\end{enumerate}  
	\end{ex}
	
	\begin{proof}\
	\begin{enumerate}
	\item By definition, for each $j \in \N$, there exists $N_j \in \N$ such that for each $n,m \in \N$, $n,m \geq N_j$ implies that 
	$$\mu(\{x \in X: |f_n(x) - f_m(x)| \geq 2^{-j} \}) < 2^{-j}$$ 
	Setting $n_1 = N_1$ and for $j \geq 2$, setting $n_j = \max (n_{j-1}+1, N_j)$, we may obtain a subsequence $(f_{n_j})$ such that for each $j \in \N$, 
	$$\mu(\{x \in X: |f_{n_j}(x) - f_{n_{j+1}}(x)| \geq 2^{-j}\}) < 2^{-j}$$
	\item Clearly $(F_k)_{k \in \N}$ is decreasing. Let $k \in \N$. Part $(1)$ implies that 
	\begin{align*}
	\mu(F_k) 
	&\leq \sum_{j \geq k}2^{-j} \\
	&= 2^{1-k}\sum_{j \geq 1}2^{-j}  \\
	&= 2^{1-k}
	\end{align*}
	Let $i, j \in \N$. Suppose that $i \geq j \geq k$. Let $x \in F_k^c$. Then 
	\begin{align*}
	|f_{n_i}(x) - f_{n_j}(x)| 
	& \leq \sum_{l = j}^{i-1} |f_{n_{l+1}}(x) - f_{n_l}(x)| \\
	& < \sum_{l = j}^{i-1} 2^{-l} \\
	& <  \sum_{l \geq j} 2^{-l} \\
	&= 2^{1 - j} \\
	& \leq 2^{1 - k}	 
	\end{align*}
	Let $\ep >0$. Choose $k' \in \N$ such that $k' \geq k$ and $2^{1-k'} < \ep$. Let $i,j \in \N$. Suppose that $i,j \geq k'$. Let $x \in F_k^c \subset F_{k'}^c$. Then 
	\begin{align*}
	|f_{n_i}(x) - f_{n_j}(x)| 
	&< 2^{1-k'} \\
	&< \ep 
	\end{align*}
	So $(f_{n_j})_{j \in \N}$ is uniformly Cauchy on $F_k^c$
	\item Since $\mu(F_1) < \infty $, $(F_k)_{k \in \N}$ is decreasing and $F = \inf\limits_{k \in \N}F_k$, we have that
	\begin{align*}
	\mu(F) 
	&= \inf_{k \in \N} \mu(F_k) \\
	&\leq \inf_{k \in \N} 2^{1-k} \\
	&= 0
	\end{align*}
	Since for each $k \in \N$, $(f_{n_j})_{j \in \N}$ is pointwise Cauchy on $F^c_k$, $(f_{n_j})_{j \in \N}$ is pointwise Cauchy on $F^c$. Then $(f_{n_j}\chi_{F^c})_{j \in \N}$ is pointwise Cauchy. \\
	Define $f: X \rightarrow \C$ pointwise by $$f = \lim_{j \rightarrow \infty}f_{n_j}\chi_{F^c} $$
	Then $f \in L^0$ since $(f_{n_j}\chi_{F^c})_{j \in \N} \subset L^0$ and $f_{n_j}\chi_{F^c} \convt{p.w.} f$. Since $\mu(F) = 0$ and $\{x \in X: f_{n_j}(x) \not \rightarrow f(x)\} \subset F$, we have that $f_{n_j} \convt{a.e.} f$.\\
	\item For $n,m \in \N$ and $\ep >0$, set 
	$$A_{n, \ep} = \{x \in X: |f_n(x) - f(x) |\geq \ep \}$$ 
	and 
	$$B_{m, n, \ep} = \{x \in X: |f_m(x) - f_n(x)| \geq \ep\}$$ 
	Let $\ep, \del>0$. Choose $k \in \N$ such that $2^{2-k} < \ep$ and $\mu(F_k) < \del$. Let $x \in F_k^c$. Since $f_{n_j}(x) \rightarrow f(x)$, there exists $J \in \N$ such that $J \geq k$ and for each $j \in \N$, $j \geq J$ implies that $|f_{n_j}(x) - f(x)| < 2^{1-k}$. Let $l \in \N$. Suppose that $l \geq k$. Then part $(2)$ implies that
	\begin{align*}
	|f_{n_l}(x) - f(x)| 
	& \leq |f_{n_l}(x) - f_{n_J}(x)| + |f_{n_J}(x) - f(x)| \\
	&\leq 2^{1-k} + 2^{1-k}  \\
	&\leq 2^{2-k} \\
	&< \ep
	\end{align*}
	So $x \in A_{n_l, \ep}^c$. Hence $A_{n_l, \ep} \subset F_k$ and $\mu(A_{n_l, \ep}) < \del$. So $f_{n_j} \conv{\mu} f$. \\
	Let $\ep >0$, $\del > 0$. Since $(f_n)_{n \in \N}$ is Cauchy in measure, there exists $J_1 \in \N$ such that for each $m,n \in \N$ $m,n \geq J_1$ implies that $\mu(B_{m,n, \frac{\ep}{2}}) < \frac{\del}{2}$. Since $f_{n_j} \conv{\mu} f$, there exists $J_2$ such that for each $j \in \N$, $j \geq J_2$ implies that $\mu(A_{n_j, \frac{\ep}{2}}) < \frac{\del}{2}$. Set $J = \max(J_1, J_2)$. Let $j \in \N$. Suppose that $j \geq J$. Since $n_j \geq j$, the proof of \rex{35002.1} implies that, 
	\begin{align*}
	\mu(A_{j, \ep}) 
	&\leq \mu(B_{j, n_j, \frac{\ep}{2}}) + \mu(A_{n_j, \frac{\ep}{2}}) \\
	&< \frac{\del}{2} + \frac{\del}{2} \\
	&= \del
\end{align*}		
	So that $f_n \conv{\mu} f$.
	\end{enumerate}
	\end{proof}
	
	\begin{ex} \lex{35003.1} 
		Let $(f_n)_{n \in \N} \subset L^0$ and $f \in L^0$. 
		\begin{enumerate}
			\item If $(f_n)_{n \in \N}$ is Cauchy in measure, then there exists a $f_0 \in L^0$ and a subsequence $(f_{n_j})_{j \in \N} \subset (f_n)_{n \in \N}$ such that $f_n \conv{\mu} f_0$ and $f_{n_j} \convt{a.e.} f_0$.
			\item If $f_n \conv{\mu} f$, then there exists a subsequence $(f_{n_j})_{j \in \N} \subset (f_n)_{n \in \N}$ such that $f_n \convt{a.e.} f$.
		\end{enumerate}
	\end{ex}
	
	\begin{proof}\
	\begin{enumerate}
		\item Previous exercise.
		\item Suppose that $f_n \conv{\mu} f$. Then $(f_n)_{n \in \N}$ is Cauchy in measure. Part $(1)$ implies that there exists a $f_0 \in L^0$ and a subsequence $(f_{n_j})_{j \in \N} \subset (f_n)_{n \in \N}$ such that $f_n \conv{\mu} f_0$ and $f_{n_j} \convt{a.e.} f_0$. Since $f_n \conv{\mu} f$ and $f_n \conv{\mu} f_0$, $f = f_0$ a.e. Hence $f_{n_j} \convt{a.e.} f$.
	\end{enumerate}
	\end{proof}

	\begin{ex}
		Let $(X, \MA, \mu)$ be a measure space, $(f_n)_{n \subset \N} \subset L^0(X, \MA)$ and $f \in  L^0(X, \MA)$. Suppose that $f_n \conv{\mu} f$. 
		\begin{enumerate}
			\item If for each $n \in \N$, $f_n \leq f_{n+1}$ a.e., then $f_n \convt{a.e.} f$. 
			\item If for each $n \in \N$, $f_n \geq f_{n+1}$ a.e., then $f_n \convt{a.e.} f$. 
		\end{enumerate}
	\end{ex}

	\begin{proof}\
		\begin{enumerate}
			\item Suppose that for each $n \in \N$, $f_n \leq f_{n+1}$ a.e. Define $N_1 \in \MA$ by 
			$$N_1 = \bigcap_{n \in \N} \{x \in X: f_n(x) \leq f_{n+1}(x)\}$$ 
			By assumption, $\mu(N_1^c) = 0$. Since $f_n \conv{\mu} f$, there exists a subsequence $(f_{n_k})_{k \in \N} \subset (f_n)_{n \in \N}$ such that $f_{n_k} \convt{a.e.} f$. Hence there exists $N_2 \in \MA$ such that $\mu(N_2^c) = 0$ and $f_{n_k} \chi_{N_2} \convt{p.w.} f \chi_{N_2}$. Set $N = N_1 \cap N_2$. Then 
			\begin{align*}
				\mu(N^c)
				& = \mu(N_1^c \cup N_2^c) \\
				& \leq \mu(N_1^c) + \mu(N_2^c) \\
				& = 0
			\end{align*}
			By construction, $f \chi_N = \sup_{k \in \N} f_{n_k} \chi_N$ which implies that for each $n \in \N$, 
			\begin{align*}
				f_n \chi_N
				& \leq f_{n_n} \chi_N \\ 
				& \leq f \chi_N
			\end{align*}
			Let $x \in N$ and $\ep > 0$. Choose $K \in \N$ such that for each $k \in \N$, $k \geq K$ implies that $|f_{n_k}(x) - f(x)| < \ep$. Let $n \in \N$. Suppose that $n \geq n_K$. Then 
			\begin{align*}
				|f_{n}(x) - f(x)|
				& = f(x) - f_{n}(x) \\
				& \leq f(x) - f_{n_K}(x) \\
				& = |f_{n_K}(x) - f(x)| \\
				& < \ep 
			\end{align*}
			Hence $f_n(x) \rightarrow f(x)$. Since $x \in N$ is arbitrary, $f_n \chi_N \convt{p.w.} f \chi_N$. Since $\mu(N^c) = 0$, $f_n \convt{a.e.} f$.
			\item Similar to $(1)$. 
		\end{enumerate}
	\end{proof}
	
	
	\begin{defn} \ld{35004} 
		 Let $(f_n)_{n \in \N} \subset L^0$ and $f \in L^0$. Then $(f_n)_{n \in \N}$ is said to \textbf{converge to $f$ almost uniformly}, denoted $f_n \xrightarrow{\text{a.u.}} f$, if for each $\ep >0$, there exists $N \in \MA$ such that $\mu(N) < \ep$ and $f_n \convt{u} f$ on $N^c$. 
	\end{defn}	
	
	\begin{ex} \lex{35005} \textbf{Egoroff's Theorem:}
		Suppose that $\mu(X) < \infty$. Let $(f_n)_{n \in \N} \subset L^0$ and $f \in L^0$. Suppose that $f_n \convt{a.e.} f$. Then $f_n \convt{a.u.} f$.
	\end{ex}
	
	\begin{proof}
		For each $n, k \in \N$, define $E_{n, k} = \{x \in X: | f_n(x) - f(x) | \geq \frac{1}{k} \}$ and $F_{n,k} = \bigcup\limits_{m \geq n}E_{m,k}$. Then $F_{n,k}$ is decreasing in $n$ and $$\bigcap\limits_{n \in \N}F_{n,k} \subset \{x: f_n(x) \not \rightarrow f(x)\}$$ 
		Thus $\mu(\bigcap\limits_{n \in \N}F_{n,k}) = 0$. Since $\mu(X) < \infty$, $\inf\limits_{n \in \N}\mu(F_{n,k}) = 0$. Let $\ep >0$. We may choose a strictly increasing sequence $(n_k)_{k \in \N} \subset \N$ such that  $\mu(F_{n_k,k}) \leq \frac{\ep}{2^{k}}$. Put $N = \bigcup\limits_{k \in \N}F_{n_k,k}$. Then 
		\begin{align*}
			\mu(N) 
			&\leq \sum\limits_{k \in \N}\mu(F_{n_k,k}) \\
			& \leq \sum\limits_{k \in \N} \frac{\ep}{2^k}\\
			& = \ep
		\end{align*} 
		Let $\del > 0$. Choose $K \in \N$ such that $\frac{1}{K} < \del$. Then for each $m \geq n_K$ and $x \in N^c =\bigcap\limits_{k \in \N}\bigcap\limits_{m \geq n_k}E_{m,k}^c$, $|f_m(x)- f(x)| < \frac{1}{K} < \del$. So $f_n \convt{u} f$ on $N^c$. 
	\end{proof}
	
	\begin{ex} \lex{35006} 
		Let $(f_n)_{n \in \N} \subset L^1$ and $f \in L^1$. If $f_n \xrightarrow{L^1}f$, then $f_n \conv{\mu} f$.
	\end{ex}
	
	\begin{proof}
		Let $\ep >0$. for $n \in \N$, define $E_{e,n} = \{x \in X: |f(x) - f_n(x)|\geq \ep\}$. Then for $n \in \N$,
		\begin{align*}
			\int |f - f_n|
			& \geq \int_{E_{\ep,n}} |f- f_n|\\
			& \geq \ep \mu(E_{\ep,n}).
		\end{align*}
		
		So for each $n \in \N$, $\mu(E_{\ep, n}) \leq \ep^{-1}\int |f - f_n|$. Since $\int |f - f_n| \rightarrow 0$, we have that $\mu(E_{\ep,n}) \rightarrow 0$. Since $\ep >0$ is arbitrary, $f_n \conv{\mu} f$ as required. 
	\end{proof}
	
	\begin{ex} \lex{35007} 
		Let $(X, \MA, \mu)$ be a measure space. Suppose $\mu(X) < \infty$. Define $d:L^0 \times L^0 \rightarrow \Rg$ by $$d(f,g) = \int \frac{|f-g|}{1+|f-g|} \dmu $$
		Then $d$ is a metric on $L^0$ if we identify functions that are equal a.e. and convergence in this metric is equivalent to convergence in measure. Note that for each $f,g \in L^0$, $d(f,g) \leq \mu(X)$.
	\end{ex} 
	
	\begin{proof}
		Let $f,g \in L^0$. Clearly $d(f,g) = d(g,f)$. If $f = g$ a.e. then clearly $d(f,g) = 0$. Conversely, if $d(f,g) = 0$, then $\frac{|f-g|}{1 + |f-g|} = 0$ a.e and so $|f-g| = 0$ a.e. which implies $f =g$ a.e. It is not hard to show that $\phi: \Rg \rightarrow \Rg$ given by $\phi(x) = \frac{x}{1+x}$ satisfies $\phi(x+y) \leq \phi(x)+\phi(y)$. Thus satisfies the triangle inequality. Now, let $(f_n)_{n \in \N} \subset L^0$. Suppose that $f_n \not \conv{\mu} f$. Then there exists $\ep>0, \del>0$ and a subsequence $(f_{n_k})_{k \in \N}$ such that for each $k \in \N$, $\mu(E_{\ep,n_k}) = \mu(\{x \in X: |f_{n_k} - f| \geq \ep\}) \geq \del $. It is not hard to show that $\phi$ from earlier is increasing. Thus for each $k \in \N$, 
		\begin{align*}
			d(f_{n_k},f)
			&= \int \frac{|f_{n_k} -f|}{1+|f_{n_k} -f|}\\
			& \geq \int_{E_{\ep,n_k}} \frac{|f_{n_k} -f|}{1+|f_{n_k} -f|}\\
			& \geq \int_{E_{\ep, n_k}} \frac{\ep}{1+\ep}\\
			& \geq \frac{\ep\del}{1+\ep}
		\end{align*}
		
		So $f_{n_k} \not \conv{d} f$. Hence $f_{n_k} \conv{d} f$ implies that $f_{n_k} \conv{\mu} f$. Conversely, suppose that $f_{n_k} \conv{\mu} f$. Let $\ep >0.$ Then $\del = \frac{\ep}{1+\mu(X)} > 0$. Choose $N \in \N$ such that for each $n \in \N$, if $n \geq N$, then $\mu(E_{\del, n}) < \frac{\del}{1+\del}$. Let $n \in \N$. Suppose that $n \geq N$. Since $\phi$ is increasing and $\phi \leq 1$, we have that 
		\begin{align*}
			d(f_n,f)
			&= \int \frac{|f_n -f|}{1+|f_n -f|}\\
			&= \int_{E_{\del,n}} \frac{|f_n -f|}{1+|f_n -f|} + \int_{E_{\del,n}^c} \frac{|f_n -f|}{1+|f_n -f|}\\
			&\leq \mu(E_{\del,n}) + \mu(X)\frac{\del}{1+\del}\\
			& < \frac{\del}{1+\del}(1+\mu(X))\\
			& \leq \del(1+\mu(X))\\
			& = \ep
		\end{align*}
	\end{proof}
	
	\begin{ex} \lex{35008} 
		Let $(f_n)_{n \in \N} \subset L^0$ and $f \in L^0$. Suppose that for each $n \in \N$, $f_n \geq 0$ and $f_n \conv{\mu} f$. Then $f \geq 0$ a.e. and $$\int f \dmu \leq \limfn \int f_n \dmu $$
	\end{ex}
	
	\begin{proof}
		Since $f_n \conv{\mu} f$, there is a subsequence converging to $f$ a.e. So clearly $f \geq 0$ a.e. Now, choose a subsequence $(f_{n_k})_{k \in \N}$ of $(f_n)_{n \in \N}$ such that $\int f_{n_k} \conv{} \limfn \int f_n$. Since $f_n \conv{\mu} f$ so does $(f_{n_k})_{k \in N}$. Therefore there exists a subsequence $(f_{n_{k_j}})_{k \in \N}$ of $(f_{n_k})_{k \in \N}$ such that $f_{n_{k_j}} \convt{a.e.} f$. Thus $f \geq 0 $ a.e. and Fatou's lemma tells us that 
		\begin{align*}
			\int f 
			&\leq \liminf_{j \in \N} \int f_{n_{k_j}}\\
			&= \limfn \int f_n.
		\end{align*}
	\end{proof}
	
	\begin{ex} \lex{35009} 
		Let $(f_n)_{n \in \N} \subset L^0$ and $f \in L^0$. Suppose that there exists $g \in L^1$ such that for each $n \in \N$, $|f_n| \leq g$. Then $f_n \conv{\mu} f$ implies that $f \in L^1$ and $f_n \conv{L^1} f$. 
	\end{ex}
	
	\begin{proof}
		Clearly $(f_n)_{n \in \N} \subset L^1$. Since $f_n \conv{\mu} f$, there exists a subsequence $(f_{n_k})_{k \in \N} \subset (f_n)_{n \in \N}$ such that $f_{n_k} \convt{a.e.} f$. This implies that $|f| \leq g$ a.e. and so $f \in L^1$. For $n \in \N$, put $h_n = 2g - |f_n-f|$. Then for each $n \in \N$, $h_n \geq 0$ and $h_n \conv{\mu}2g$. By the previous exercise 
		\begin{align*}
			\int 2g 
			&\leq \limfn \int (2g - |f_n -f|)\\
			& = \int 2g - \limpn \int|f_n -f|.
		\end{align*}
		
		So $\limpn \int|f_n -f| \leq 0$ which implies that $\int|f_n -f| \rightarrow 0$ and $f_n \conv{L^1} f$ as required. 
	\end{proof}
	
	\begin{ex} \lex{35010} 
		Let $(f_n)_{n \in \N} \subset L^0$, $f \in L^0$ and $\phi :\C \rightarrow \C$. 
		\begin{enumerate}
			\item If $\phi$ is continuous, and $f_n \convt{a.e.} f$ then $\phi \circ f_n \convt{a.e.} \phi \circ f$.
			\item If $\phi$ is uniformly continuous and $f_n \rightarrow f$ uniformly, almost uniformly or in measure, then $\phi \circ f_n \rightarrow \phi \circ f$ uniformly, almost uniformly or in measure, respectively.
			\item Find a counter example to (2) if we drop the word "uniform".
		\end{enumerate} 
	\end{ex}
	
	\begin{proof}\
		\begin{enumerate}
			\item Clear
			\item Suppose that $\phi$ is unifomly continuous. 
			\begin{itemize}
				\item uniformly: \\
				Suppose that $f_n \convt{u} f$. Let $\ep > 0$. Choose $\del >0$ such that for each $z,w \in \C$, if $|z-w|<\del$, then $|\phi(z) - \phi(w)| < \ep$. Now choose $N \in \N$ such that for each $n \in \N$ if $n \geq n$ then for each $x \in X$, $|f_n(x)-f(x)| < \del$. Let $n \in \N$, suppose $n \geq N$, Let $x \in X$. Then $|\phi(f_n(x)) - \phi(f(x))| < \ep$. Thus $\phi \circ f_n \convt{u} \phi \circ f$. 
				\item almost uniformly: \\
				Suppose that $f_n \convt{a.u.} f$. Let $\ep > 0$. Choose $N \in \MA$ such $\mu(N) < \ep$ and $f_n \convt{u} f$ on $N^c$. Then from above, we know that $\phi \circ f_n \convt{u} \phi \circ f$ on $N^c$. Thus $\phi \circ f_n \convt{a.u.} \phi \circ f$.
				\item in measure:\\ Suppose that $f_n \conv{\mu} f$. Let $\ep > 0$. Choose $\del >0$ such that for each $z,w \in \C$, if $|z-w|<\del$, then $|\phi(z) - \phi(w)| < \ep$. Observe that for $x \in X$, if $|f_n(x) - f(x)| < \del$, then $|\phi(f_n(x)) - \phi(f(x))| < \ep$. Hence $E_{n,\ep} = \{x \in X: |\phi(f_n(x)) - \phi(f(x))| \geq \ep\} \subset F_{n,\del} = \{x \in X: |f_n(x) - f(x)| \geq \del\}$. By definition of convergence in measure, $\mu(F_{n,\del}) \rightarrow 0$. Thus $\mu(E_{n,\ep}) \rightarrow 0$. Hence $\phi \circ f_n \conv{\mu} \phi \circ f$.
			\end{itemize}
			\item
		\end{enumerate}
	\end{proof}
	
	\begin{ex} \lex{35011} 
		Let $(f_n)_{n \in \N} \subset L^0$ and $f \in L^0$. Suppose that $f_n \convt{a.u} f$. Then $f_n \conv{\mu}f$ and $f_n \convt{a.e.}f$. 
	\end{ex}
	
	\begin{proof}
		(measure) Let $\ep>0$, $\del >0$. Choose $M \in \MA$ such that $\mu(M) < \del$ and $f_n \convt{u} f$ on $M^c$. Choose $N \in \N$ such that for each $n \in \N$, if $n \geq N$, then for each $x \in M^c$, $|f_n(x) - f(x)| < \ep$. Let $n \in \N$. Suppose $n \geq N$. Then $E_{\ep,n} \subset M$ and $\mu(E_{\ep,n}) < \del$. Thus $\mu(E_{\ep,n}) \rightarrow 0$ and $f_n \conv{\mu} f$.
		
		(a.e.) For each $n \in \N$, Choose $N_n \in \MA$ such that $\mu(N_n) < 1/n$ and $f_n \convt{u} f$ on $N_n^c$. Observe that for $x \in X$, if $x \in \bigcup_{n \in \N}N_n^c$, then $f_n(x) \rightarrow f(x)$. Thus $N = \{x \in X: f_n(x) \not \rightarrow f(x)\} \subset \bigcap_{n \in \N} N_n$. Therefor $\mu(N) = 0$ and $f_n \convt{a.e.} f$.
	\end{proof}
	
	\begin{ex} \lex{35012} 
		Let $(f_n)_{n \in \N}, (g_n)_{n \in \N} \subset L^0$ and $f,g \in L^0$. Suppose that $f_n \conv{\mu} f$ and $g_n \conv{\mu}g$. Then 
		\begin{enumerate}
			\item $f_n + g_n \conv{\mu} f+g$
			\item if $\mu(X) < \infty$, then $f_n g_n \conv{\mu} fg$
		\end{enumerate}
	\end{ex}
	
	\begin{proof}
		
		\begin{enumerate}
			\item Let $\ep > 0$. For convenience, put $F_{n,\ep/2} = \{x \in X: |f_n(x) - f(x)| \geq \ep/2\}$, $G_{n, \ep/2} = \{x \in X: |g_n(x) - g(x)| \geq \ep/2\}$, and $(F+G)_{n,\ep} = \{x \in X: |f_n(x)+g_n(x) - (f(x) + g_n(x))| \geq \ep\}$ Observe that for $x \in X$, $|f_n(x) + g_n(x) - (f(x) + g(x))| \leq |f_n(x) - f(x)| + |g_n(x) - g(x)|$. Thus $(F+G)_{n,\ep} \subset F_{n,\ep/2} \cup G_{n, \ep/2}$. Since $\mu(F_{n,\ep/2} \cup G_{n, \ep/2}) \leq \mu(F_{n,\ep/2}) + \mu(G_{n, \ep/2}) \rightarrow 0$, we have that $\mu((F+G)_{n,\ep}) \rightarrow 0$. Hence $f_n + g_n \conv{\mu} f+g$.
			
			\item Suppose that $\mu(X) < \infty$. Let $(f_{n_k}g_{n_k})_{k \in \N}$ be a subsequence of $(f_ng_n)_{n \in \N}$. Choose a subsequence $(f_{n_{k_j}}g_{n_{k_j}})_{j \in \N}$ such that $f_{n_{k_j}} \convt{a.e} f$ and $g_{n_{k_j}} \convt{a.e} g$. Then $f_{n_{k_j}}g_{n_{k_j}} \convt{a.e.} fg$. Egoroff's theorem tells us that $f_{n_{k_j}}g_{n_{k_j}} \convt{a.u.} fg$, which implies that $f_{n_{k_j}}g_{n_{k_j}} \conv{\mu} fg$. Thus for each subsequence $(f_{n_k}g_{n_k})_{k \in \N}$ of $(f_ng_n)_{n \in \N}$, there exists a subsequence $(f_{n_{k_j}}g_{n_{k_j}})_{j \in \N}$ of $(f_{n_k}g_{n_k})_{k \in \N}$ such that $f_{n_{k_j}}g_{n_{k_j}} \conv{\mu} fg$. Using the fact that this is equivalent to convergence in a metric defined in an earlier exercise,
			we have that $f_ng_n \conv{\mu} fg$.
		\end{enumerate}
		
	\end{proof}
	
	\begin{ex} \lex{35013} 
		Let $(f_n)_{n \in \N}, \subset L^0$ and $f \in L^0$. Suppose that $\mu(X) < \infty$. Then $f_n \conv{\mu}f_n$ iff for each subsequence $(f_{n_k})_{k \in \N}$, there exists a subsequence $(f_{n_{k_j}})_{j \in \N}$ such that $f_{n_{k_j}} \convt{a.e.} f$.
	\end{ex}
	
	\begin{proof}
		Suppose that $f_n \conv{\mu} f$. Let $(f_{n_k})_{k \in \N}$ be a subsequence. Then $f_{n_k} \conv{\mu} f$. By a previous theorem, there exists a subsequence $(f_{n_{k_j}})_{j \in \N}$ such that $f_{n_{k_j}} \convt{a.e.} f$. Conversely, suppose that for each subsequence $(f_{n_k})_{k \in \N}$, there exists a subsequence $(f_{n_{k_j}})_{j \in \N}$ such that $f_{n_{k_j}} \convt{a.e.} f$. Let $\ep >0$. For $n \in \N$, define $E_{n} = \{x \in X: |f_n(x) - f(x) | \geq \ep\}$ and define $E = \{x \in X: f_n(x) \not \rightarrow f(x)\}$. Let $(f_{n_k})_{k \in \N}$ be a subsequence. Choose a subsequence $(f_{n_{k_j}})_{j \in \N}$ such that $f_{n_{k_j}} \convt{a.e.} f$. Since $\bigg \{x \in X: \limsup\limits_{j \rightarrow \infty} \chi_{E_{n_{k_j}}}(x) = 1\bigg \} = \limsup\limits_{j \rightarrow \infty} E_{n_{k_j}} \subset E$ and $\mu(E) = 0$, we have that $\limsup\limits_{j \rightarrow \infty} \chi_{E_{n_{k_j}}} = 0$ a.e. and $\chi_{E_{n_{k_j}}} \convt{a.e} 0$. Since $\mu(X) < \infty$, the dominated convergence theorem implies that 
		\begin{align*}
			\mu(E_{n_{k_j}}) 
			&= \int \chi_{E_{n_{k_j}}} \dmu  \rightarrow 0
		\end{align*} 
		So for each subsequence $(\mu(E_{n_k}))_{k \in \N}$, there exists a subsequence $(\mu(E_{n_{k_j}}))_{j \in \N}$ such that $\mu(E_{n_{k_j}}) \rightarrow 0$. Thus $\mu(E_n) \rightarrow 0$ and $f_n \conv{\mu} f$.
	\end{proof}
	
	\begin{ex} \lex{35014} 
		Let $(f_n)_{n \in \N}, \subset L^0$, $f \in L^0$ and $\phi: \C \rightarrow \C$. Suppose that $\mu(X) < \infty$. If $\phi$ is continuous and $f_n \conv{\mu} f$, then $\phi \circ f_n \conv{\mu} \phi \circ f$.
	\end{ex}
	
	\begin{proof}
		Suppose that $\phi$ is continuous and $f_n \conv{\mu} f$. Let $(\phi \circ f_{n_k})_{k \in \N}$ be a subsequence of $(\phi \circ f_{n})_{n \in \N}$. Then $(f_{n_k})_{k \in \N}$ is a subsequence of $(f_{n})_{n \in \N}$. Since $f_n \conv{\mu} f$, the previous exercise tells us that there exists a subsequence $(f_{n_{k_j}})_{j \in \N}$ such that $f_{n_{k_j}} \convt{a.e.} f$. A previous exercise implies that $\phi \circ f_{n_{k_j}}\convt{a.e.} \phi \circ f$. The previous exercise implies that $\phi \circ f_{n}\conv{\mu} \phi \circ f$.
	\end{proof}
	
	
	\begin{ex} \lex{35015} 
		Let $(f_n)_{n \in \N} L^0$ and $f \in L^0$. Suppose that for each $\ep >0$, $$\sum_{n \in \N}\mu(\{x \in X: |f_n(x)-f(x)| > \ep\}) < \infty$$
		Then $f_n \convt{a.e.} f$.
	\end{ex}
	
	\begin{proof}
		Let $\ep>0$. By assumption we know that
		\begin{align*}
			\int \bigg[ \sum_{n \in \N}\chi_{\{x \in X: |f_n(x)-f(x)| > \ep\}}\bigg] \dmu 
			&= \sum_{n \in \N}\int \chi_{\{x \in X: |f_n(x)-f(x)| > \ep\}}\dmu\\
			&=\sum_{n \in \N}\mu(\{x \in X: |f_n(x)-f(x)| > \ep\})\\
			& < \infty
		\end{align*}
		Thus we also know that $\sum_{n \in \N}\chi_{\{x \in X: |f_n(x)-f(x)| > \ep\}} < \infty$ a.e. Equivalently, we could say that for a.e. $x \in X$, $|\{n \in \N: f_n(x) - f(x) > \ep\}| < \infty$. For $k \in \N$, define $N_k = \{x \in X: \sum_{n \in \N}\chi_{\{x \in X: |f_n(x)-f(x)| > 1/k\}} = \infty\}$. Then for each $k \in \N$, $\mu(N_k) = 0$. Define $N = \bigcup_{k \in \N} N_k$. Then $\mu(N) = 0$. Let $x \in N^c$ and $\ep > 0$. Choose $k \in \N$ such that $1/k < \ep$. Then $\{n \in \N: f_n(x) - f(x) > \ep\} \subset \{n \in \N: f_n(x) - f(x) > 1/k\}$ which is finite because $x \in N_k^c$. Put $M = \max\{n \in \N: f_n(x) - f(x) > \ep\}$. Then for $m \geq M$, $|f_m(x) - f(x) \leq \ep|$. Thus $f_n(x) \rightarrow f(x)$. Hence $f_n \convt{a.e.} f$.
	\end{proof}
	
	
	
	
	
	
	
	
	
	
	
	
	
	\newpage
	\section{The Radon-Nikodym Derivative}
	
	\subsection{Signed Measures}
	
	\begin{defn} \ld{41001} 
		Let $(X, \MA)$ be a measurable space and $\nu : \MA \rightarrow [-\infty, \infty]$. Then $\nu$ is said to be a \textbf{signed measure} if 
		\begin{enumerate}
			\item for each $E \in \MA$, $\nu(E) < \infty$ or for each $E \in \MA$, $\nu(E) > -\infty$.
			\item $\nu(\varnothing) = 0$
			\item for each $(E_n)_{n \in \N} \subset \MA$ if $(E_n)_{n \in \N} \subset \MA$ is disjoint, then $\nu(\bigcup\limits_{n \in \N} E_n) = \sum\limits_{n \in \N} \nu(E_n)$ and if $|\sum\limits_{n \in \N} \nu(E_n)| < \infty$, then $\sum\limits_{n \in \N} \nu(E_n)$ converges absolutely.
		\end{enumerate}
	\end{defn}
	
	\begin{ex} \lex{41002} 
		Let $\nu: \MA \rightarrow \RG$ be a signed measure and $(E_n)_{n \in \N}$, $(F_n)_{n \in \N} \subset \MA$. If $(E_n)_{n \in \N}$ is increasing, then $\nu(\bigcup\limits_{n \in \N} E_n) = \limn \nu(E_n)$. If $(F_n)_{n \in \N}$ is decreasing and $|\nu(E_1)| < \infty$, then $\nu(\bigcap\limits_{n \in \N} F_n) = \limn \nu(F_n)$. 
	\end{ex}
	
	\begin{proof}
		Put $E'_1 = E_1$, $F'_1 = F_1$ and for $n \in \N$, $n \geq 2$, put $E'_n = E_n \setminus E_{n-1}$ and $F'_n = F_1 \setminus F_n$. Then $(E'_n)_{n \in \N} \subset \MA$ is disjoint. Thus 
		\begin{align*}
			\nu(\bigcup\limits_{n \in \N} E_n) 
			&= \nu(\bigcup\limits_{n \in \N} E'_n)\\
			&= \sum\limits_{n \in \N} \nu(E'_n)\\
			&= \limn \sum_{n=1}^n \nu(E'_n)\\
			&= \limn \nu(E_n)
		\end{align*}
		
		Since $(F'_n)_{n \in \N}$ is increasing, we now know that 
		\begin{align*}
			\nu(F_1) - \nu(\bigcap\limits_{n \in \N} F_n)
			&= \nu(F_1 \setminus \bigcap\limits_{n \in \N} F_n)\\
			&= \nu(\bigcup\limits_{n \in \N} F'_n) \\
			&= \limn \nu(F'_n) \\
			&= \limn \nu(F_1 \setminus F_n) \\ 
			&= \nu(F_1) - \limn \nu(F_n)
		\end{align*}
		
		Since $|\nu(F_1)| < \infty$, we see that $\nu(\bigcap\limits_{n \in \N} F_n) = \limn \nu(F_n)$.
	\end{proof}
	
	\begin{defn} \ld{41003} 
		Let $(X, \MA)$ be a measurable space and $\nu: \MA \rightarrow [-\infty, \infty]$ a signed measure and $E \in \MA$. Then $E$ is said to be $\nu$-\textbf{positive}, $\nu$-\textbf{negative} and $\nu$-\textbf{null} if for each $F \in \MA$, $F \subset E$ implies that $\nu(F) \geq 0$, $\nu(F) \leq 0$, $\nu(F) = 0$ respectively.
	\end{defn}
	
	\begin{ex} \lex{41004} 
		Let $E \subset \MA$. If $E$ is positive, negative or null, then for each $F \in \MA$, if $F \subset E$, then $F$ is positive, negative or null respectively.
	\end{ex}
	
	\begin{proof}
		Clear
	\end{proof}
	
	\begin{ex} \lex{41005} 
		Let $(E_n)_{n \in \N} \subset \MA$ be positive, negative or null. Then $\bigcup\limits_{n \in \N} E_n$ is positive, negative or null respectively. 
	\end{ex}
	
	\begin{proof}
		Suppose that $(E_n)_{n \in \N} \subset \MA$ is positive. Let $F \in \MA$. Suppose that $F \subset \bigcup\limits_{n \in \N} E_n$. Put $P_1 = E_1$ and for $n \in \N$, $n \geq 2$, put $P_n = E_n \setminus (\bigcup\limits_{j=1}^{n-1} E_j)$. So $\bigcup\limits_{n \in \N} P_n = \bigcup\limits_{n \in \N} E_n$ and $(P_n)_{n \in \N}$ is disjoint. Thus 
		\begin{align*}
			\nu(F) 
			&= \nu(F \cap \bigcup_{n \in \N} P_n)\\
			&= \nu(\bigcup_{n \in \N} (F \cap P_n))\\
			&= \sum_{n \in \N} \nu(F \cap P_n)\\
			& \geq 0 
		\end{align*}
		
		The process is the same if $(E_n)_{n \in \N}$ is negative and null.
	\end{proof}
	
	\begin{thm}{Hahn Decomposition:}
		Let $\nu$ be a signed measure on $(X, \MA)$. Then there exist $P,N \in \MA$ such that $P$ is positive, $N$ is negative, $X = N \cup P$ and $N \cap P = \varnothing$. Furthermore, these two sets are unique in the following sense: For any $P',N' \in \MA$, if $N,P$ satisfy the properties above, $P' \Delta P = N' \Delta N$ is $\nu$-null.
	\end{thm}
	
	\begin{defn} \ld{41006} 
		Let $\nu$ be a signed measure on $(X, \MA)$ and $P,N \in \MA$. Then $P$ and $N$ are said to form a \textbf{Hahn decomposition} of $X$ with respect to $\nu$ if $P,N$ satisfy the results in the above theorem.
	\end{defn}
	
	\begin{defn} \ld{41007} 
		Let $\mu, \nu$ be signed measures on $(X, \MA)$. Then $\mu$ and $\nu$ are said to be \textbf{mutually singular} if there exist $E, F \in \MA$ such that $X = E \cup F$, $E \cap F = \varnothing$ and $E$ is $\mu$-null and $F$ is $\nu$-null. We will denote this by $\mu \perp \nu$.
	\end{defn}
	
	\begin{thm}{Jordan Decomposition:}
		Let $\nu$ be a signed measure on $(X, \MA)$. Then there exist unique positive measures $\nu^+$ and $\nu^-$ on $(X, \MA)$ such that $\nu = \nu^+ - \nu^-$ and $\nu^+ \perp \nu^-$. 
	\end{thm}
	
	\begin{proof}
		Choose a Hahn decomposition $P,N$ of $X$ with respect to $\nu$. Define $\nu^+, \nu^-$ by $\nu^+(E)= \nu(E \cap P)$ and $\nu^-(E) = \nu(E \cap N)$.
	\end{proof}
	
	\begin{defn} \ld{41008} 
		Let $\nu$ be a signed measure on $(X, \MA)$. Then $\nu^+$ and $\nu^-$ from the last theorem are called the \textbf{positive} and \textbf{negative variations} of $\nu$ respectively. We define the \textbf{total variation of $\nu$}, denoted $|\nu|:\MA \rightarrow [0, \infty]$ by $$|\nu| = \nu^+ + \nu^-$$ 
	\end{defn}
	
	\begin{defn} \ld{41009} 
		Let $\nu$ be a signed measure on $(X,\MA)$. Then $\nu$ is said to be $\sig$-finite if $|\nu|$ is $\sig$-finite.
	\end{defn}
	
	\begin{ex} \lex{41010} 
		Let $\nu$ be a signed measure and $\lam, \mu$ positive measures on $(X,\MA)$. Suppose that $\nu = \lam - \mu$. Then $\lam \geq \nu^+$ and $\mu \geq \nu^-$.
	\end{ex}
	
	\begin{proof}
		Choose a Hahn decomposition $P,N$ of $X$ with respect to $\nu$. Let $E \in \MA$. Then 
		\begin{align*}
			\lam(E \cap P) - \mu(E \cap P) 
			&= \nu(E \cap P)\\
			&= \nu^+(E \cap P)
		\end{align*}
		So $\lam(E \cap P) \geq  \nu^+(E \cap P) $ and therefore 
		\begin{align*}
			\lam(E) 
			&= \lam(E \cap P) + \lam(E \cap N)\\
			& \geq \nu^+(E \cap P) + \lam (E \cap N)\\
			& \geq \nu^+(E \cap P)\\
			& = \nu^+(E)
		\end{align*} 
		Similarly $\mu(E \cap N) \geq \nu^-(E \cap N)$ and $\mu(E) \geq \nu^-(E)$.
	\end{proof}
	
	\begin{ex} \lex{41011} 
		Let $\nu_1, \nu_2$ be signed measures on $(X, \MA)$. Suppose that $\nu_1 + \nu_2$ is a signed measure. Then $|\nu_1 + \nu_2| \leq |\nu_1|+ |\nu_2|$. (Hint: use the last exercise)
	\end{ex}
	
	\begin{proof}
		Since 
		\begin{align*}
			\nu_1 + \nu_2 
			&= (\nu_1^+ - \nu_1^-) + (\nu_2^+ - \nu_2^-)\\
			&= (\nu_1^+ + \nu_2^+) - (\nu_1^- + \nu_2^-)
		\end{align*}
		the previous exercise tells us that $\lam = \nu_1^+ + \nu_2^+ \geq (\nu_1 + \nu_2)^+$ and $ \mu = \nu_1^- + \nu_2^- \geq (\nu_1 + \nu_2)^-$. Therefore 
		\begin{align*}
			|\nu_1 + \nu_2| 
			&= (\nu_1 + \nu_2)^+  + (\nu_1 + \nu_2)^-\\
			& \leq (\nu_1^+ + \nu_2^+) + (\nu_1^- + \nu_2^-)\\
			&= (\nu_1^+ + \nu_1^-) + (\nu_2^+ + \nu_2^-)\\
			&= |\nu_1| + |\nu_2|
		\end{align*}
	\end{proof}
	
	\begin{note}
		Recall that a previous exercise from the section on complex valued functions tells us that $L^1(|\nu|) = L^1(\nu^+) \cap L^1(\nu^-)$.
	\end{note}
	
	\begin{defn} \ld{41012} 
		Let $\nu$ be a signed measure on $(X, \MA)$. Then we define $L^1(\nu) = L^1(|\nu|)$. For $f \in L^1(\nu)$, we define $$\int f d \nu = \int f d \nu^+ - \int f d\nu^-$$
	\end{defn}
	
	\begin{ex} \lex{41013} 
		Let $\nu_1, \nu_2$ be signed measures on $(X, \MA)$. Suppose that $\nu_1 + \nu_2$ is a signed measure. Then 
		$L^1(\nu_1)\cap L^1(\nu_2) \subset L^1(\nu_1 + \nu_2)$
	\end{ex}
	
	\begin{proof}
		The previous exercise tells us that $|\nu_1 + \nu_2| \leq |\nu_1| + |\nu_2|$. Two previous exercises from the section on nonnegative functions tells us that 
		\begin{align*}
			\int |f|d |\nu_1 + \nu_2| 
			& \leq \int |f| d(|\nu_1|+|\nu_2|)\\
			&= \int |f|d |\nu_1| + \int |f| d|\nu_2|
		\end{align*}
	\end{proof}
	
	\begin{ex} \lex{41014} 
		Let $\nu, \mu$ be signed measures on $(X,\MA)$ and $E \in \MA$. Then 
		\begin{enumerate}
			\item $E$ is $\nu$-null iff $|\nu|(E) = 0$
			\item $\nu \perp \mu$ iff $|\nu| \perp \mu$ iff $\nu^+ \perp \mu$ and $\nu^- \perp \mu$.
		\end{enumerate}
	\end{ex}
	
	\begin{proof}
		\begin{enumerate}
			\item Suppose that $E$ is $\nu$-null. Choose a Hahn decomposition $P,N$ of $X$ with respect to $\nu$. Then $\nu^+(E) = \nu(E \cap P) = 0$ and $\nu^-(E) = \nu(E \cap N) = 0$. Therefore $|\nu|(E) = \nu^+(E) + \nu^-(E) = 0$. Conversely, suppose that $|\nu|(E) = 0$. Then $\nu^+(E) = \nu^-(E) = 0$. Let $F \in \MA$. Suppose that $F \subset E$. Then $\nu^+(F) = 0$ and $\nu^-(F) = 0$. Therefore $\nu(F) = \nu^+(F) - \nu^-(F) = 0$. So $E$ is $\nu$-null.
			
			\item Suppose that $\nu \perp \mu$. Then there exist $E,F \in \MA$ such that $E \cup F = X$, $E \cap F = \varnothing$, $E$ is $\mu$-null and $F$ is $\nu$-null. By (1), $F$ is $|\nu|$-null and thus $|\nu| \perp \mu$. If $|\nu| \perp \mu$, choose $E,F \in \MA$ as before. Since $F$ is $|\nu|$-null, we know that $\nu^+(F) + \nu^-(F) = |\nu|(F) = 0$. This implies that $F$ is $\nu^+$-null and $F$ is $\nu^-$-null. So $\nu^+ \perp \mu$ and $\nu^- \perp \mu$. Finally assume that $\nu^+ \perp \mu$ and $\nu^- \perp \mu$. \textbf{FINISH!!!!}
			
		\end{enumerate}
	\end{proof}
	
	\begin{ex} \lex{41015} 
		Let $\nu$ be a signed measure on $(X, \MA)$. Then 
		\begin{enumerate}
			\item for $f \in L^1(\nu)$, $|\int f d \nu| \leq \int |f| d |\nu|$
			\item if $\nu$ is finite, then for each $E \in \MA$, $$|\nu|(E) = \sup \bigg\{\bigg|\int_E f d \nu \bigg|: f  \text{ is measurable and } |f| \leq 1 \bigg \}$$
		\end{enumerate}
	\end{ex}
	
	\begin{proof}
		\begin{enumerate}
			\item Let $f \in L^1(\nu)$. Then 
			\begin{align*}
				\bigg|\int f d \nu \bigg| 
				&= \bigg|\int f d \nu^+ - \int f d \nu^-\bigg|\\
				& \leq \bigg|\int f d \nu^+\bigg| + \bigg|\int f d \nu^-\bigg|\\
				& \leq \int |f| d\nu^+ + \int |f| d\nu^-\\
				&= \int |f| d (\nu^+ + \nu^-)\\
				&= \int |f| d |\nu|
			\end{align*}
			
			\item Let $E \in \MA$. Let $f:X \rightarrow \R$ be measurable and suppose that $|f| \leq 1$. Since $\nu$ is finite, so is $|\nu|$ and thus $f \in L^1(\nu)$. Then (1) tells us that 
			\begin{align*}
				\bigg |\int_E f \dnu \bigg| 
				& \leq \int_E |f| \, d |\nu|\\
				& \leq |\nu|(E) 
			\end{align*}
			
			Now, choose a Hahn decomposition $P,N$ of $X$ with respect to $\nu$. Define $f = \chi_{P} - \chi_{N}$. Then $|f| \leq 1$, $f$ is measurable and 
			\begin{align*}
				\bigg|\int_E f d\nu\bigg|
				&= \bigg|\int_E f d \nu^+ - \int_E f d \nu^-\bigg|\\
				&= | \nu^+(E \cap P) + \nu^-(E \cap N)|\\
				&= \nu^+(E) + \nu^-(E)\\
				&= |\nu|(E).
			\end{align*}
			
		\end{enumerate}
	\end{proof}
	
	\begin{ex} \lex{41016} 
		Let $\mu$ be a positive measure on $(X, \MA)$ and $f \in L^0(X, \MA)$ extended $\mu$-integrable. Define $\nu$ on $(X, \MA)$ by $$\nu(E) = \int_E f \dmu$$ Then
		\begin{enumerate}
			\item $\nu$ is a signed measure
			\item for each $E\in \MA$, $$|\nu|(E) = \int_E|f|\dmu$$
		\end{enumerate} 
	\end{ex}
	
	\begin{proof}
		
		\begin{enumerate}
			
			\item Clearly $\nu(\varnothing) = 0$ and $\nu$ is finte by assumption. Let $(E_n)_{n \in \N} \subset \MA$. Suppose that $(E_n)_{n \in \N}$ is disjoint. Then 
			\begin{align*}
				\nu(\bigcup_{n \in \N} E_n)
				&= \int_{\bigcup_{n \in \N} E_n} f \dmu \\ 
				&= \int_{\bigcup_{n \in \N} E_n} f^+ \dmu - \int_{\bigcup_{n \in \N} E_n} f^- \dmu\\
				&= \sum_{n \in \N} \int_{ E_n} f^+ \dmu - \sum_{n \in \N} \int_{E_n} f^- \dmu\\
				&= \sum_{n \in \N} \bigg[ \int_{ E_n} f^+ \dmu - \int_{ E_n} f^- \dmu \bigg]\\
				&= \sum_{n \in \N} \int_{ E_n} f \dmu\\
				&= \sum_{n \in \N} \nu(E_n)
			\end{align*}
			
			If $|\nu(\bigcup_{n \in \N}E_n)| < \infty$, then $ \int_{\bigcup_{n \in \N} E_n} f^+ d  \mu < \infty$ and $ \int_{\bigcup_{n \in \N} E_n} f^- d  \mu < \infty$ because
			\begin{align*}
				|\nu(\bigcup_{n \in \N}E_n)|
				&=\bigg |\int_{\bigcup_{n \in \N} E_n} f \dmu\bigg| \\
				&= \bigg |\int_{\bigcup_{n \in \N} E_n} f^+ \dmu - \int_{\bigcup_{n \in \N} E_n} f^- \dmu \bigg|
			\end{align*} Therefore, we have that
			
			\begin{align*}
				\sum_{n \in \N} |\nu(E_n)|
				&=  \sum_{n \in \N} \bigg|\int_{E_n} f \dmu \bigg|\\
				&= \sum_{n \in \N} \bigg| \int_{E_n} f^+ \dmu -  \int_{E_n} f^- \dmu \bigg|\\
				& \leq \sum_{n \in \N}  \int_{E_n} f^+ \dmu  + \sum_{n \in \N}  \int_{E_n} f^- \dmu \\
				&= \int_{\bigcup_{n \in \N} E_n} f^+ \dmu + \int_{\bigcup_{n \in \N} E_n} f^- \dmu\\
				& < \infty
			\end{align*}
			So the sum $\sum_{n \in \N} \nu(E_n)$ converges absolutely and $\nu$ is a signed measure. 
			
			\item Put $P = \{x \in X: f(x) \geq 0\}$ and $N = \{x \in X: f(x) < 0\}$. Then $P,N$ form a Hahn decomposition of $X$ with respect to $\nu$. Thus for $E \in \MA$, 
			$$\nu^+(E) = \int_{E \cap P} f \dmu = \int_E f^+ \dmu$$ and $$\nu^-(E) = \int_{E \cap N} f \dmu = \int_E f^- \dmu$$
			 So for $E \in \MA$, $$|\nu|(E) = \int_E f^+ \dmu + \int_E f^- \dmu = \int_E |f| \dmu$$
		\end{enumerate}
	\end{proof}
	
	\begin{defn} \ld{00000} 
		Let $(X, \MA)$ be a measureable space, $\nu$ be a signed measure on $(X, \MA)$ and $\mu$ a measure on $(X,\MA)$. Then $\nu$ is said to be \textbf{absolutely continuous} with respect to $\mu$, denoted $\nu \ll \mu$, if for each $E \in \MA$, $\mu(E) = 0$ implies that $\nu(E) =0$. 
	\end{defn}
	
	\begin{note}
		If there exists an extended $\mu$-integrable $f \in L^0(X, \MA)$ such that for each $E \in \MA$, $\nu(E) = \int_E f \dmu$, then we write $d\nu = f \dmu$.
	\end{note}
	
	\begin{ex}
	Let $(X, \MA)$, $(Y, \MB)$ be measureable spaces, $f:X \rightarrow Y$ $\MA$-$\MB$ measurable, $\nu$ be a signed measure on $(X, \MA)$ and $\mu$ a measure on $(X,\MA)$. Suppose that $\nu \ll \mu$. Then $f_*\nu \ll f_*\mu$
	\end{ex}
	
	\begin{proof}
	Let $E \in \MB$. Suppose that $f_*\mu(E) = 0$. By definition, $\mu(f^{-1}(E)) = 0$. Since $\nu \ll \mu$, $\nu(f^{-1}(E)) = 0$. Hence $f_*\nu(E) = 0$ and $f_*\nu \ll f_*\mu$.
	\end{proof}
	
	\begin{thm}
		Let $(X, \MA)$ be a measureable space, $\nu$ be a $\sig$-finite signed measure on $(X, \MA)$ and $\mu$ a $\sig$-finite measure on $(X,\MA)$. Then there exist unique $\sig$-finite signed measures $\lam$, $\rho$ on $(X, \MA)$ such that $\lam \perp \mu$, $\rho \ll \mu$ and $\nu = \lam + \rho$, and there exists an extended $\mu$-integrable $f \in L^0(X, \MA)$ such that $d\rho = f \dmu$ and $f$ is unique $\mu$-a.e.  
	\end{thm}
	
	\begin{defn} \ld{00000} 
		The decomposition $\nu = \lam + \rho$ is referred to as the \textbf{Lebesgue decomposition of $\nu$ with respect to $\mu$}. In the case $\nu \ll \mu$, we have $\lam = 0$ and $\rho = \nu$ and we define the \textbf{Radon-Nikodym derivative of $\nu$ with respect to $\mu$}, denoted by $d\nu/d\mu$, to be $d\nu/d\mu = f$ where $d\nu = f\dmu$.   
	\end{defn}
	
	\begin{thm}
		Let $\nu$ be a $\sig$-finite signed measure on $(X, \MA)$ and $\mu$, $\lam$ $\sig$-finite measures on $(X,\MA)$. Suppose that $\nu \ll \mu$ and $\mu \ll \lam$. Then 
		\begin{enumerate}
			\item for each $g \in L^1(\nu)$, $g(d\nu/d\mu) \in  L^1(\mu)$ and $$\int g d\nu = \int g \frac{d\nu}{d\mu} \dmu$$
			\item $\nu \ll \lam$ and $$\frac{d \nu}{d\lam} = \frac{d \nu}{d\mu} \frac{d\mu}{d\lam} \hspace{4mm} \lam \text{-a.e.}$$
		\end{enumerate}
	\end{thm}
	
	\begin{ex} \lex{00000} 
		Let $(\nu_n)_{n \in \N}$ be a sequence of measures and $\mu$ a measure. 
		\begin{enumerate}
			\item If for each $n \in \N$, $\nu_n \ll \mu$, then $\sum_{n \in \N} \nu_n \ll \mu$. 
			\item If for each $n \in \N$, $\nu_n \perp \mu$, then $\sum_{n \in \N} \nu_n \perp \mu$.
		\end{enumerate} 
	\end{ex}
	
	\begin{proof}
		\begin{enumerate}
			\item Let $E \in \MA$. Suppose that $\mu(E) = 0$. Then for each $n \in \N$, $\nu_i(E) = 0$ and thus $\sum_{n \in \N} \nu_n(E) = 0$. Hence $\sum_{n \in \N} \nu_n \ll \mu$.
			\item For each $n \in \N$, there exist $N_i, M_i \in \MA$ such that $N_i \cap M_i = \varnothing$, $N_i \cup M_i = X$ and $\nu_i(M_i) = \mu(N_i) = 0$. Put $N = \bigcup_{n \in \N} N_i$ and $M = N^c$. Note that for each $n \in \N$, $M \subset N_i^c = M_i$. So $\mu(N) \leq \sum_{n \in \N} \mu(N_i) = 0$ and $(\sum_{n \in \N} \nu_i) (M) \leq \sum_{n \in \N} \nu_i(M_i) = 0$. Thus $\sum_{n \in \N} \nu_i \perp \mu$.
		\end{enumerate}
	\end{proof}
	
	
	\begin{ex} \lex{00000} 
		Choose $X = [0,1]$, $\MA = \MB_{[0,1]}$. Let $m$ be Lebesgue measure and $\mu$ the counting measure. 
		
		Then 
		\begin{enumerate}
			\item $m \ll \mu$ but for each $f \in L^+$, $dm \neq f \dmu$
			\item There is no Lebesgue decomposition of $\mu$ with respect to $m$.
		\end{enumerate}
	\end{ex}
	
	\begin{proof}
		\begin{enumerate}
			\item Let $E \in \MA$. If $\mu(E) = 0$, then $E = \varnothing$ and $m(E) = 0$. So $m \ll \mu$. Suppose for the sake of contradiction that there exists $f \in L^+$ such that $dm = f \dmu$. Then 
			\begin{align*}
				1
				&= m(X) \\
				&= \sum_{x \in X} f(x)
			\end{align*}
			
			Put $Z = \{x \in X: f(x) \neq 0 \}$. Then $Z$ is countable. So 
			\begin{align*}
				1
				&= m(X \setminus Z) \\
				&= \sum_{x \in X \setminus Z} f(x)\\
				&= 0
			\end{align*}
			
			This is a contradiction, so no such $f$ exists.
			
			\item Suppose for the sake of contradiction that there is a Lebesgue decomposition for $\mu$ with respect to $m$ given by $\mu = \lam + \rho$ where $\lam \perp m$ and $\rho \ll m$. We may assume $\lam$ and $\rho$ are positive. Then for each $x \in X$, $m(\{x\})=0$ which implies that $\rho(\{x\}) = 0$. Let $E \subset X$, if $E$ is countable, then $\lam(E) = \mu(E)$. If $E$ is uncountable, choose $F \subset E$ such that $F$ is countable. Then 
			\begin{align*}
				\lam(E) 
				& \geq \lam(F) \\
				& = \mu(F) \\
				&= \infty
			\end{align*}
			
			So $\lam = \mu$. This is a contradiction since $\mu \not \perp m$.
		\end{enumerate}
	\end{proof}
	
	\begin{ex}
		Let $(X, \MA)$ be a measurable space and $\mu, \nu$ $\sig$-finite measures on $(X, \MA)$. Suppose that $\nu \ll \mu$. Then $d\nu / \dmu > 0$ $\mu$-a.e. iff for each $E \in \MA$, $\mu(E) \neq 0$ implies that $\nu(E) > 0$. 
	\end{ex}

	\begin{proof}
		Since $\nu$ is a measure, there exists $f \in L^+(X, \MA)$ such that $f = d\nu/ d\mu$ $\mu$-a.e. Suppose that there exists $E \in \MA$ such that $\mu(E) > 0$ and $\nu(E) = 0$. Then 
		\begin{align*}
			\int_E f \dmu  
			& = \nu(E) \\
			& = 0 
		\end{align*}
		Hence 
		\begin{align*}
			\frac{d \nu}{d\mu}\chi_E 
			& = f \chi_E \\
			& = 0 \text{ $\mu$-a.e.}
		\end{align*} 
		Therefore $d\nu / \dmu \not > 0$ $\mu$-a.e.\\
		Conversely, suppose that $d\nu / \dmu \not > 0$ $\mu$-a.e. Then there exists $E \in \MA$ such that $\mu(E) > 0$ and $(d \nu / d\mu)\chi_E = 0$ $\mu$-a.e. Therefore 
		\begin{align*}
			\nu(E)
			& = \int_E \frac{d \nu}{d\mu}\chi_E  \dmu \\
			& = 0
		\end{align*}
	\end{proof}
	
	
	
	
	
	
	
	
	
	
	\newpage	
	\subsection{Complex Measures}
	
	\begin{defn} \ld{43001} 
		Let $(X, \MA)$ be a measurable space and $\nu:\MA \rightarrow \C$. Then $\nu$ is said to be a \textbf{complex measure} if 
		\begin{enumerate}
			\item $\nu (\varnothing) = 0$
			\item for each sequence $(E_n)_{n \in \N} \subset \MA$, if $(E_n)_{n \in \N}$ is disjoint, then $\nu(\bigcup_{n \in \N} E_n) = \sum_{n \in \N} \nu(E_n)$ and $\sum_{n \in \N} \nu(E_n)$ converges absolutely. 
		\end{enumerate}
	\end{defn}
	
	\begin{defn} \ld{43002} 
	Let $(X, \MA)$ be a measurable space. We define $$\MM(X, \MA) = \{\mu:\MA \rightarrow \C: \mu \text{ is a complex measure}\}$$
	When $X$ is a topological space, we write $\MM(X)$ in place of $\MM(X, \MB(X))$.
	\end{defn}

	\begin{ex}
		Let $(X, \MA)$ be a measurable space and $\mu, \nu \in \MM(X, \MA)$. Set $\ML_{\mu,\nu} = \{A \in \MA: \mu(A) = \nu(A)\}$. If $X \in \ML_{\mu, \nu}$, then $\ML_{\mu, \nu}$ is a $\lam$-system on $X$.
	\end{ex}
	
	\begin{proof} Suppose that  $X \in \ML_{\mu, \nu}$.
		\begin{enumerate}
			\item Since $X \in \ML_{\mu, \nu}$, $\ML_{\mu, \nu} \neq \varnothing$.
			\item Let $A \in \ML_{\mu, \nu}$. Then $\mu(A) = \nu(A)$. Thus 
			\begin{align*}
				\mu(A^c) 
				&= \mu(X)-\mu(A) \\
				&= \nu(X) -\nu(A) \\
				&= \nu(A^c)
			\end{align*}
			So $A^c \in \ML_{\mu, \nu}$. 
			\item Let $(A_n)_{n \in \N} \subset \ML_{\mu, \nu}$. So for each $n \in \N$, $\mu(A_n) = \nu(A_n)$.  Suppose that $(A_n)_{n \in \N}$ is disjoint. Then 
			\begin{align*}
				\mu\bigg(\bigcup_{n \in \N} A_n\bigg) 
				&= \sum_{n \in \N} \mu(A_n) \\
				&= \sum_{n \in \N} \nu(A_n) \\
				&= \nu\bigg(\bigcup_{n \in \N} A_n\bigg) 
			\end{align*}
			Hence $\bigcup_{n \in \N} A_n \in \ML_{\mu, \nu}$.
		\end{enumerate}
	\end{proof}
	
	\begin{ex}
		Let $(X, \MA)$ be a measurable space, $\mu, \nu \in \MM(X, \MA)$ and $\MP \subset \MA$ a $\pi$-system on $X$. Suppose that $X \in \MP$ and that for each $A \in \MP$, $\mu(A) = \nu(A)$. Then for each $A \in \sig(\MP)$, $\mu(A) = \nu(A)$.
	\end{ex}
	
	\begin{proof}
		The previous exercise implies that $\ML_{\mu, \nu}$ is a $\lam$-system on $X$. By assumtion, $\MP \subset \ML_{\mu, \nu}$. Dynkin's theorem implies that $\sig(\MP) \subset \ML_{\mu, \nu}$. So for each $A \in \sig(\MP)$, $\mu(A) = \nu(A)$.
	\end{proof}

	\begin{ex}
		Let $(X, \MT)$ be a topological space and $\mu, \nu \in \MM(X)$. If for each $A \in \MT$, $\mu(A) = \nu(A)$, then $\mu = \nu$.
	\end{ex}

	\begin{proof}
		Since $\MT \subset \MB(X)$ is a $\pi$-system on $X$ and $X \in \MT$, the previous exercise implies that for each $A \in \sig(\MT)$, $\mu(A) = \nu(A)$. Since $\sig(\MT) = \MB(X)$, $\mu = \nu$. 
	\end{proof}
	
	\begin{note}
		We use the same definitions for mutual orthogonality and absolute continuity when discussing complex measures instead of signed measures.
	\end{note}
	
	\begin{defn} \ld{43003} 
		Let $(X,\MA)$ be a measurable space and $\nu \in \MM(X, \MA)$ with $\nu = \nu_1 + i\nu_2$. We define $L^1(\nu) = L^1(\nu_1)\cap L^1(\nu_2)$. For $f \in L^1(\nu)$, we define $$\int f d\nu = \int fd\nu_1 + i \int f d \nu_2$$
	\end{defn}
	
	\begin{thm}\textbf{Lebesgue-Radon-Nikodym Theorem:}\\
		Let $(X,\MA)$ be a measurable space, $\nu \in \MM(X, \MA)$ and $\mu$ a $\sig$-finite measure on $(X, \MA)$. Then there exists unique $\lam$, $\rho \in \MM(X, \MA)$ such that $\lam \perp \mu$, $\rho \ll \mu$ and $\nu = \lam + \rho$, and there exists $f \in L^1(\mu)$ such that $d\rho = f \dmu$ and $f$ is unique $\mu$-a.e.  
	\end{thm}
	
	\begin{ex} \lex{43003.2}
		Let $\nu \in \MM(X, \MA)$ and $\mu$, $\lam$ $\sig$-finite measures on $(X,\MA)$. Suppose that $\nu \ll \mu$ and $\mu \ll \lam$. Then 
		\begin{enumerate}
			\item for each $g \in L^1(\nu)$, $g(d\nu/d\mu) \in  L^1(\mu)$ and $$\int g d\nu = \int g \frac{d\nu}{d\mu} \dmu$$
			\item $\nu \ll \lam$ and $$\frac{d \nu}{d\lam} = \frac{d \nu}{d\mu} \frac{d\mu}{d\lam} \hspace{4mm} \lam \text{-a.e.}$$
		\end{enumerate}
	\end{ex}
	
	\begin{defn} \ld{43004} 
	Let $(X,\MA)$ be a measurable space and $\nu \in \MM(X,\MA)$ with $ \nu = \nu_1 + i \nu_2$. Define $\mu = |\nu_1| + |\nu_2|$. Then $\nu \ll \mu$ and thus there exists $f \in L^1(\mu)$ such that $d\nu = f \dmu$. We define the \textbf{total variation of $\nu$}, denoted $|\nu|: \MA \rightarrow \Rg$, by $$|\nu|(E) = \int_E |f|\dmu$$ 
	\end{defn}
	
	\begin{ex} \lex{43004.5} 
	Let $(X,\MA)$ be a measurable space, $\nu \in \MM(X,\MA)$ and $\lam$ a $\sig$-finite measure on $(X, \MA)$. Suppose that $\nu \ll \lam$. Set $g = d\nu / d\lam$. Then for each $E \in \MA$, $$|\nu|(E) = \int_E |g| \dlam$$
	\end{ex}
	
	\begin{proof}
	Write $\nu = \nu_1 + i \nu_2$. Then $\nu_1, \nu_2 \ll \lam$. Set $f_1 = d\nu_1 / d\lam$ and $f_2 = d\nu_2 / d\lam$. Then \rex{41016} implies that $d|\nu_1| = |f_1| \dlam$ and $d|\nu_2| = |f_2|d\lam$. Set $\mu = |\nu_1| + |\nu_2|$ and $f = d\nu / \dmu$ as in \rd{43004}. Then by construction, 
	\begin{align*}
	d\mu
	&= d|\nu_1| + d|\nu_2| \\
	&= |f_1| \dlam + |f_2| \dlam \\
	&= (|f_1| + |f_2|) \dlam
	\end{align*}
	So that $\mu \ll \lam$ with $d\mu / d \lam = |f_1| + |f_2|$. Then \rex{43003.2} implies that $\nu \ll \lam$ with 
	\begin{align*}
	\frac{d\nu}{d\lam} 
	&= \frac{d\nu}{d\mu} \frac{d\mu}{d\lam}\\
	&= f(|f_1|+|f_2|) \\
	&= g
	\end{align*}
	and for each $E \in \MA$, 
	\begin{align*}
	|\nu|(E) 
	&= \int_E |f| d \, \mu \\
	&= \int_E |f| (|f_1| + |f_2|) \, d\lam \\
	&= \int_E|g| \, d\lam 
	\end{align*}
	\end{proof}
	
	\begin{ex} \lex{43005} 
		Let $\nu \in \MM(X, \MA)$ and $\mu$ a $\sig$-finite measures on $(X,\MA)$. If $\nu \ll \mu$, then $\{x \in X: d\nu / \dmu(x) = 0 \}$ is $\nu$-null.
	\end{ex}
	
	\begin{proof}
		Define $f = d\nu / \dmu$ and $E = \{x: f(x) = 0\}$. Let $A \in \MA$ and suppose that $A \subset E$. Then 
		\begin{align*}
			\nu(A) 
			&= \int_A f \dmu\\
			&= 0
		\end{align*} 
	\end{proof}
	
	\begin{ex} \lex{43006} 
		Let $(X, \MA)$ be a measurable space and $\nu \in \MM(X, \MA)$ with $\nu = \nu_1 + i\nu_2$. Then $|\nu_1|, |\nu_2| \leq |\nu| \leq |\nu_1| + |\nu_2|$.
		
	\end{ex}
	
	\begin{proof}
		Let $\mu$ and $f = f_1 + i f_2$ be as in the definition of $|\nu|$. Since for each $E \in \MA$, we have 
		\begin{align*}
			\nu(E) 
			&= \int_E f \dmu\\
			&= \int_E f_1 \dmu + i \int_E f_2 \dmu
		\end{align*}
		
		and $$\nu(E) = \nu_1(E) + i\nu_2(E)$$
		
		we know that $\nu_1 = f_1 \dmu$ and $\nu_2 = f_2 \dmu$. 
		
		A previous exercise tells us that $d|\nu_1| = |f_1|\dmu$ and $d |\nu_2| = |f_2|\dmu$. Since $|f_1|, |f_2| \leq |f| \leq |f_1|+|f_2|$, we have that 
		\begin{align*}
			|\nu_1|, |\nu_2| 
			&\leq |\nu| \\
			&\leq |\nu_1| + |\nu_2|\\
		\end{align*}
	\end{proof}
	
	\begin{ex} \lex{43007} 
		Let  $(X, \MA)$ be a measurable space, $\nu \in \MM(X, \MA)$ and $c \in \C$. Then $| c \nu | = | c | | \nu |$.
	\end{ex}
	
	\begin{proof}
		Define $\mu$ and $f$ as before so that $d \nu = f \dmu$. Then $d (c \nu) = c f \dmu$. Hence 
		\begin{align*}
			d | c \nu | 
			&= | cf | \dmu \\
			&= | c | | f | \dmu\\
			&= | c | d| \nu |
		\end{align*}
		So $| c \nu | = | c | |  \nu |$.
	\end{proof}

	\begin{ex} \lex{43008} 
	Define $\|\cdot \|: \MM(X, \MA) \rightarrow \Rg$ by $$\|\mu \|= | \mu | (X)$$ 
	Then $\|\cdot \|$ is a norm on $\MM(X, \MA)$. 
	\end{ex}
	
	\begin{proof}
		Let $\mu, \nu \in \MM(X, \MA)$ and $\al \in \C$. The previous exercises tell us that $| \mu + \nu | \leq | \mu | + | \nu |$ and $| \al \mu | = | \al | | \mu |$. So clearly $\|\mu + \nu \|\leq \|\mu \|+ \|\nu \|$ and $\|c \mu \|= | c | \|\mu \|$. If $\|\mu \|= 0$, then $X$ is $\mu$-null and $\mu$ is the zero measure.
	\end{proof}
	
	\begin{ex} \lex{43009} 
		Let $(X, \MA)$ be a measurable space and $\nu \in \MM(X, \MA)$. Then 
		
		\begin{enumerate}
			\item for each $E \in \MA$, $|\nu(E)| \leq |\nu|(E)$. 
			\item $\nu \ll |\nu|$ and $\big|d \nu /d |\nu|\big| = 1$ $|\nu|$-a.e.
			\item $L^1(\nu) = L^1(|\nu|)$ and for each $g \in L^1(\nu)$, $$ \bigg|\int g d\nu \bigg| \leq \int |g|d |\nu|$$
		\end{enumerate}
	\end{ex}
	
	\begin{proof}
		Let $\mu$, $f \in L^1(\mu)$ be as in the definition of $|\nu|$.
		\begin{enumerate}
			\item Let $E \in \MA$. Then 
			\begin{align*}
				|\nu(E)| 
				& = \bigg|\int_E f \dmu\bigg|\\
				& \leq \int_E |f| \dmu\\
				&= |\nu|(E)
			\end{align*}
			
			\item Let $E \in \MA$ and suppose that $|\nu|(E)=0$. The previous part implies $|\nu(E)|=0$ and $\nu \ll |\nu|$. Put $g = d \nu / d|\nu|$. Then 
			\begin{align*}
				f 
				&= \frac{d\nu}{d\mu}\\
				&= g|f| \hspace{2mm }\mu\text{-a.e.}
			\end{align*}
			
			Hence $|f| = |g||f|$ $\mu$-a.e. Since $|\nu| \ll \mu$, $|f| = |g||f|$ $|\nu|$-a.e.
			
			A previous exercise tells us that $|f| \neq 0$ $|\nu|$-a.e. Thus $|g|=1$ $|\nu|$-a.e.\\
			\item Write $\nu = \nu_1 + i\nu_2$ and $f = f_1 + if_2$. First we observe that
			\begin{align*}
				L^1(\nu)
				&= L^1(\nu_1) \cap L^1(\nu_2) \\
				&= L^1(|\nu_1|) \cap L^1(|\nu_2|)\\
				&= L^1(|\nu_1| + |\nu_2|)\\
				&= L^1(\mu)
			\end{align*}
			The previous exercise tells us that 
			\begin{align*}
				|\nu_1|, |\nu_2| 
				&\leq |\nu| \\
				&\leq |\nu_1|+ |\nu_2| \\
				&= \mu
			\end{align*}
			Let $g \in L^1(\mu)$. Then 
			\begin{align*}
				\int |g| d |\nu| 
				&\leq \int |g| \dmu \\
				&< \infty
			\end{align*}
			So $g \in L^1(|\nu|)$.
			Conversely, let $g \in L^1(|\nu|)$. Then 
			\begin{align*}
				\int |g| d|\nu_1|, \int |g| d |\nu_2| 
				& \leq \int |g|d |\nu|\\
				& < \infty
			\end{align*}
			So 
			\begin{align*}
				\int |g| \dmu
				& =\int |g| d|\nu_1| + \int |g| d |\nu_2| \\
				& < \infty
			\end{align*}
			and $g \in L^1(\mu)$. Hence $L^1(\nu) = L^1(|\nu|)$. 
			Now, let $g \in L^1(\nu) = L^1(|\nu|)$, then 
			\begin{align*}
				\bigg| \int g d\nu \bigg| 
				&= \bigg| \int g f \dmu \bigg| \\
				& \leq \int |g||f|\dmu\\
				& = \int |g| d |\nu|
			\end{align*}
			
		\end{enumerate}
	\end{proof}


	\begin{ex} \lex{43010} 
	Let $(X, \MA)$ be a measurable space, $\mu_1, \mu_2 \in \MM(X, \MA)$ and $\lam \in \C$. Then for each $f \in L^1(\mu_1 + \lam \mu_2)$, $$\int f d(\mu_1 + \lam \mu_2) = \int f \dmu_1 + \lam \int f \dmu_2$$
	\end{ex}
	
	\begin{proof}
	Clear by an exercise in section $3.2$.
	\end{proof}
	
	
	
	
	
	
	
	
	
	
	
	
	
	
	
	
	
	
	
	
	
	
	
	
	
	
	
	
	

	
	
	
	
	
	
	
	
	\newpage
	\subsection{Differentiation on $\R^n$}
	
	\begin{defn} \ld{00000} 
		Let $B \subset \R^n$. Then $B$ is said to be a \textbf{ball} if there exists $x \in \R^n$ and $r > 0$ such that $B = B(x, r)$. 
	\end{defn}
	
	\begin{defn} \ld{00000} 
		Let $f \in L^0(\R^n)$. Then $f$ is said to be \textbf{locally integrable} (with respect to Lebesgue measure) if $f$ is measurable and for each $K \subset \R$, $K$ is compact implies $\int_K |f| \dm < \infty$. We define $L^1_{\text{loc}}(\R^n) = \{f:\R^n \rightarrow \C: f \text{ is locally integrable}\}$
	\end{defn}
	
	\begin{defn} \ld{00000} 
		For $f \in \Ll$, $r>0$, $x \in \R^n$, we define the \textbf{average of $f$ over $B(x,r)$}, denoted by $Af(x,r)$, to be $$Af(x,r) = \frac{1}{m(B(x,r))}\int_{B(x,r)}f\dm$$
	\end{defn}
	
	\begin{ex} \lex{00000} 
		Let $f \in \Ll$. Define $$H^*f(x) = \sup\{\frac{1}{m(B)}\int_{B}|f|\dm: B \text{ is a ball and } x \in B\} \hspace{4mm} (x \in \R^n)$$
		
		Then $Hf \leq H^*f \leq 2^n Hf$. 
	\end{ex}
	
	\begin{proof}
		Let $x \in \R^n$. Then $$\bigg \{ \frac{1}{m(B(x,r))}\int_{B(x,r)}|f|\dm: r >0\bigg \} \subset \bigg\{ \frac{1}{m(B)}\int_{B}|f|\dm: B \text{ is a ball and } x \in B \bigg\} $$
		
		So $Hf(x) \leq H^*f(x)$. Let $B$ be a ball. Then there exists $y \in \R^n$, $R>0$ such that $B = B(y,R)$ Suppose that $x \in B$. Then $B \subset B(x,2R)$. Since $m(B(x,2R)) = 2^n m(B(y,R))$, we have that 
		\begin{align*}
			\frac{1}{m(B)}\int_{B}|f|\dm
			& \leq \frac{1}{m(B)} \int_{m(B(x,2R))}|f|\dm\\
			&= \frac{2^n}{m(B(x,2R))} \int_{m(B(x,2R))}|f|\dm
		\end{align*}
		
		Thus $H^*f(x) \leq 2^n Hf(x)$.
	\end{proof}
	
	\begin{lem}
		Let $f \in \Ll$, then $Af:\R^n \times (0, \infty)\rightarrow \R$ is continuous.
	\end{lem}
	
	\begin{defn} \ld{00000} 
		Let $f \in \Ll$. We define its \textbf{Hardy Littlewood maximal function}, denoted by $Hf$ to be $$Hf(x) = \sup_{r>0} A|f|(x,r) \hspace{4mm} x \in \R^n$$
	\end{defn}
	
	\begin{thm}
		There exists $C >0$ such that for each $f \in L^1(m)$ and $\al > 0$, $$m(\{x \in \R^n: Hf(x) > \al\}) \leq \frac{C}{a} \int |f|\dm$$
	\end{thm}
	
	\begin{ex} \lex{00000} 
		Let $f \in \L^1(\R^n)$. Suppose that $||f||_1>0$. Then there exist $C,R>0$ such that for each $x \in \R^n$, if $|x| > R$, then $Hf(x) \geq C|x|^{-n}$. Hence there exists $C' > 0$ such that for each $\al >0$, $m(\{x \in X: Hf(x)>\alpha\}) > C'/\al$ when $\al$ is small. 
	\end{ex}
	
	\begin{proof}
		Since $||f||_1 >0$, there exists $R>0$ such that $\int_{B(0,R)}|f|\dm >0$. Recall that there exists $K>0$ such that for each $x \in R^n$ and $r>0$, $m(B(x,r)) = Kr^n$. Choose $$C = \frac{1}{K2^n}\int_{B(0,R)}|f| \dm$$. Let $x \in \R^n$. Suppose that $|x|>R$. Then $B(0,R) \subset B(x,2|x|)$. Thus 
		\begin{align*}
			Hf(x) 
			&\geq \frac{1}{m(B(x,2|x|))}\int_{B(x,2|x|)}|f|\dm\\
			&= \frac{1}{K2^n|x|^n}\int_{B(x,2|x|)}|f|\dm \\
			&\geq \frac{1}{K2^n|x|^n}\int_{B(0,R)}|f|\dm \\
			&= \frac{C}{|x^n|}
		\end{align*}
		
		Let $a\ < \frac{C}{2R^n}$. Then $R^n < \frac{C}{2 \al}$. Choose $C' =\frac{KC}{2}$. Let $A = \{x \in \R^n: R < |x|< (\frac{C}{\al})^{\frac{1}{n}}\}$. For $x \in A$, 
		\begin{align*}
			Hf(x) 
			&\geq \frac{C}{|x|^n} \\
			& > \al
		\end{align*}
		
		Thus $A \subset m(\{x \in R^n: Hf(x)> \al\})$ and therefore 
		\begin{align*}
			m(\{x \in R^n: Hf(x)> \al\}) 
			&\geq m(A) \\
			&= m(B(0,(C/\al)^{1/n})) - m(B(0,R)) \\
			&= K\bigg [\frac{C}{\al} - R^n \bigg] \\
			&> K\bigg[\frac{C}{\al} - \frac{C}{2 \al}\bigg] \\
			&= \frac{KC}{2 \al}\\
			&= \frac{C'}{\al}
		\end{align*}
	\end{proof}
	
	\begin{thm}
		Let $f \in \Ll$, then for a.e. $x \in \R^n$, $$\lim_{r \rightarrow 0} Af(x,r) =f(x)$$ 
		Equivalently, for a.e. $x \in \R^n$, $$ \lim_{r \rightarrow 0} \bigg[ \frac{1}{m(B(x,r))}\int_{B(x,r)}[f(y)-f(x)]\dm(y)\bigg] =0$$
	\end{thm}
	
	\begin{note}
		We can a stronger result of the same flavor.
	\end{note}
	
	\begin{defn} \ld{00000} 
		Let $f \in \Ll$. We define the \textbf{Lebesgue set of $f$}, denoted by $L_f$, to be 
		\begin{align*}
			L_f 
			&= \{x \in \R^n: \lim_{r \rightarrow 0} A|f-f(x)|(x,r) =0 \}\\
			&= \bigg \{x \in \R^n: \lim_{r \rightarrow 0} \bigg[ \frac{1}{m(B(x,r))}\int_{B(x,r)}|f(y) - f(x)|\dm(y)\bigg] =0 \bigg \}
		\end{align*}
	\end{defn}
	
	\begin{ex} \lex{00000} 
		Let $f \in \Ll$ and $x \in \R^n$. If $f$ is continuous at $x$, then $x \in L_f$.
	\end{ex}
	
	\begin{proof}
		Suppose that $f$ is continuous at $x$. Let $\ep > 0$. By assumption, there exists $\del >0$ such that for each $y \in \R^n$, if $|x-y|< \del$, then $|f(x)-f(y)| < \ep$. Let $r >0$. Suppose that $r< \del$. Then for each $y \in \R^n$, $y \in B(x,r)$ implies that $|f(x) - f(y)| < \ep$ and thus 
		\begin{align*}
			\frac{1}{m(B(x,r))}\int_{B(x,r)}|f(y) - f(x)|\dm(y)
			& \leq \frac{1}{m(B(x,r))} \ep m(B(x,r))\\
			&=\ep
		\end{align*}
		Hence $$\lim_{r \rightarrow 0} \bigg[ \frac{1}{m(B(x,r))}\int_{B(x,r)}|f(y) - f(x)|\dm(y)\bigg] =0$$ 
		and $x \in L_f$.
	\end{proof}
	
	\begin{thm}
		Let $f \in \Ll$. Then $m((L_f)^c) = 0$
	\end{thm}
	
	\begin{defn} \ld{00000} 
		Let $x \in \R^n$ and $(E_r)_{r>0} \subset \MB(\R^n)$. Then $(E_r)_{r>0}$ is said to \textbf{shrink nicely to $x$} if 
		
		\begin{enumerate}
			\item for each $r>0$, $E_r \subset B(x,r)$
			\item there exists $\al >0$ such that for each $r>0$, $m(E_r)> \al m(B(x,r))$
		\end{enumerate} 
	\end{defn}
	
	\begin{thm}
		Let $f \in \Ll$ and $(E_r)_{r>0} \subset \MB(\R^n)$. Then for each $x \in L_f$, 
		
		$$\lim_{r \rightarrow 0} \bigg[ \frac{1}{m(E_r)}\int_{E_r}|f(y) - f(x)|\dm(y)\bigg] =0$$
		
		and 
		
		$$\lim_{r \rightarrow 0}  \frac{1}{m(E_r)}\int_{E_r}f\dm = f(x)$$
	\end{thm}
	
	\begin{defn} \ld{00000} 
		Let $\mu:\MB(\R^n) \rightarrow \RG$ be a Borel measure. Then $\mu$ is said to be \textbf{regular} if 
		\begin{enumerate}
			\item for each $K \subset \R^n$, if $K$ is compact, then $\mu(K)< \infty$
			\item for each $E \in \MB(\R^n)$, $\mu(E) = \inf \{\mu(U): U \text{ is open and }E \subset U\}$
		\end{enumerate}
		
		Let $\nu$ be a signed or complex Borel measure on $\R^n$. Then $\nu$ is said to be regular if $|\nu|$ is regular.
	\end{defn}
	
	\begin{thm}
		Let $\nu$ be a regular signed or complex measure on $\R^n$. Let $d\nu = d\lam + f \dm$ be the Lebesgue decomposition of $\nu$ with respect to $m$. Then for $m$-a.e. $x \in \R^n$ and $(E_r)_{r >0} \subset \MB(R^n)$, if $(E_r)_{r >0}$ shrinks nicely to $x$, then 
		
		$$\lim_{r \rightarrow 0} \frac{\nu(E_r)}{m(E_r)} = f(x)$$
	\end{thm}
	
	
	
	
	
	
	
	
	
	
	
	
	
	
	
	
	\newpage
	\subsection{Functions of Bounded Variation}
	
	\begin{defn} \ld{00000} 
		Let $F:\R \rightarrow \R$ be increasing. Define $F_+:\R \rightarrow \R$ and $F_-:\R \rightarrow \R$ by $$F_+(x) = \lim_{t \rightarrow x^+}F(t) = \inf \{F(t): t>x \}$$ and $$F_-(x) =  \lim_{t \rightarrow x^-}F(t) = \sup \{F(t): t < x \}$$ respectively.
	\end{defn}
	
	\begin{ex}
		Let $F:\R \rightarrow \R$ be increasing. Then 
		\begin{enumerate}
			\item 
			\begin{enumerate}
				\item $F \leq F_+$
				\item$F_+$ is increasing
			\end{enumerate}
			\item 
			\begin{enumerate}
				\item $F_- \leq F$
				\item $F_-$ is increasing
			\end{enumerate}
		\end{enumerate}
	\end{ex}

	\begin{proof}\
		\begin{enumerate}
			\item 
			\begin{enumerate}
				\item  Let $x \in \R$. Since $F$ is increasing, for each $t > x$, $F(x) \leq F(t)$. Hence 
				\begin{align*}
					F(x)
					& \leq \inf \{F(t): t>x \} \\
					& = F_+(x)
				\end{align*}
				Since $x \in \R$ is arbitrary, $F \leq F_+$.
				\item  Let $x,y \in \R$. Suppose that $x \leq y$. Then $\{F(t): t>y \} \subset \{F(t): t>x \}$. Thus 
				\begin{align*}
					F_+(x)
					& = \inf \{F(t): t>x \} \\
					& \leq \inf \{F(t): t>y \} \\
					& = F_+(y)
				\end{align*}
			\end{enumerate}
			\item Similar to $(1)$.
		\end{enumerate}
	\end{proof}

	\begin{ex}
		Let $F:\R \rightarrow \R$ be increasing and $x \in \R$. Then $F$ is discontinuous at $x$ iff $F_-(x) < F_+(x)$.
	\end{ex}

	\begin{proof}
		Since $F$ is continuous at $x$ iff $\lim_{t \rightarrow x^+}F(t) = F(x)$ and $\lim_{t \rightarrow x^-}F(t) = F(x)$, by definition, $F$ is continuous at $x$ iff $F_+(x) = F(x)$ and $F_-(x) = F(x)$. Then the previous exercise implies that $F$ is discontinuous at $x$ iff $F_+(x) > F(x)$ or $F_-(x) < F(x)$. Since $F_+(x) > F(x)$ implies that $F_-(x) < F_+(x)$ and $F_-(x) < F(x)$ implies that $F_-(x) < F_+(x)$, we have that $F$ is discontinuous at $x$ iff $F_-(x) < F_+(x)$.
 	\end{proof}
	
	\begin{ex} \lex{00000} 
		Let $F:\R \rightarrow \R$ be increasing. Then for each $x \in \R$ and $ \ep>0$, there exists $\del >0$ such that for each $y \in (x,x+\del)$, $0 \leq F_+(y) - F(y) \leq \ep$.
	\end{ex}
	
	\begin{proof}
		For the sake of contradiction, suppose not. Then there exists $x \in \R$ and $\ep >0$ such that for each $\del >0$, there exist $y \in (x,x+\del)$ such that $F_+(y) - F(y) > \ep$. Then there exists a sequence $(y_n)_{n \in \N} \subset \R$ such that for each $n \in \N$, $y_n \in (x, x+\frac{1}{n})$, $y_n > y_{n+1}$ and $F_+(y_n) - F(y_n) > \ep$. Choose $N \in \N$ such that $(N-1)\ep > F(y_1) - F(x)$. Note that for each $n \in \N$, $(y_n + y_{n+1})/ 2 < y_n$ which implies that
		\begin{align*}
			F_+(y_{n+1})
			& \leq F((y_n + y_{n+1})/ 2) \\
			& \leq F(y_n)
		\end{align*}
		Therefore
		\begin{align*}
			F(y_1) - F(x) 
			&= \sum_{j=1}^{N-1} \bigg[F(y_j)-F_+(y_{j+1}) + F_+(y_{j+1}) - F(y_{j+1}) \bigg] + F(y_N)- F(x)\\
			& = \sum_{j=1}^{N-1} \bigg[F(y_j)-F_+(y_{j+1}) \bigg] + \sum_{j=1}^{N-1} \bigg[ F_+(y_{j+1}) - F(y_{j+1}) \bigg] + F(y_N)- F(x) \\
			& \geq \sum_{j=1}^{N-1} \bigg[ F_+(y_{j+1}) - F(y_{j+1}) \bigg]  \\
			& \geq (N-1)\ep \\
			& > F(y_1) - F(x)
		\end{align*}
		This is a contradiction, so the claim holds.
	\end{proof}
	
	\begin{ex} \lex{00000} 
		Let $F:\R \rightarrow \R$ be increasing. Then $F_+$ is right continuous. 
	\end{ex}
	
	\begin{proof}
		Let $x \in \R$. Let $\ep >0$. By definition, there exists $\del_1>0$ such that for each $y \in (x,x+\del_1)$ $0 \leq F(y)-F_+(x) < \ep/2$. The previous exercise implies that there exists $\del_2 >0$ such that for each $y \in (x,x+\del_2)$, $0 \leq F_+(y)-F(y) < \ep/2$. Choose $\del = \min\{\del_1, \del_2\}$. Let $y \in (x, x+\del)$.
		\begin{align*}
			|F_+(x) - F_+(y)|
			& \leq |F_+(x) - F(y)| + |F(y)- F_+(y)| \\
			& = (F(y) - F_+(x)) + (F_+(y) - F(y)) \\
			& \leq \frac{\ep}{2} + \frac{\ep}{2}\\
			& = \ep
		\end{align*}
		
		So $\lim\limits_{t \rightarrow x^+} F_+(t) = F_+(x)$ and $F_+$ is right continuous.
	\end{proof}
	
	\begin{ex}
		Let $F:\R \rightarrow \R$ be increasing. Then 
		\begin{enumerate}
			\item $\{x \in \R: F \text{ is not continuous at }x\}$ is countable
			\item $F$ and $F_+$ are differentiable a.e. and $F' = F_+'$ a.e.
		\end{enumerate}
	\end{ex}

	\begin{proof}\
		\begin{enumerate}
			\item 
			\item 
		\end{enumerate}
	\end{proof}
	
	\begin{defn} \ld{00000} 
		Let $F:\R \rightarrow \C$. Define $T_F:\R \rightarrow \R$ by $$T_F(x) = \sup\bigg \{\sum_{i=1}^{n}|F(x_{i}) - F(x_{i-1})|: (x_i)_{i=0}^n \subset \R \text{ is increasing and } x_n=x  \bigg \} \hspace{4mm} (x \in \R)$$
		
		$T_F$ is called the \textbf{total variation function of $F$}.
	\end{defn}
	
	\begin{ex} \lex{00000} 
		Let $F:\R \rightarrow \C$. Then $T_F$ is increasing.
	\end{ex}
	
	\begin{proof}
		Let $x,y \in \R$. Suppose that $x<y_2$. \\Define  $A_x = \big \{\sum_{i=1}^{n}|F(x_{i}) - F(x_{i-1})|: (x_i)_{i=0}^n \subset \R \text{ is increasing and } x_n=x  \big \}$ and \\$A_y = \big \{\sum_{i=1}^{n}|F(x_{i}) - F(x_{i-1})|: (x_i)_{i=0}^n \subset \R \text{ is increasing and } x_n=y  \big \}$. Let $z \in A_x$. Then there exists $(x_i)_{i=0}^n \subset \R$ such that $(x_i)_{i=0}^n$ is increasing,\\ $x_n=x$ and $z = \sum_{i=1}^n |F(x_{i}) - F(x_{i-1})|$. Then
		
		\begin{align*}
			z 
			& \leq z+|F(y)-F(x)|\\
			&= \sum_{i=1}^n |F(x_{i}) - F(x_{i-1})| + |F(y)-F(x)|\\
			& \in A_y\\
		\end{align*} 
		So $z \leq \sup A_y = T_F(y) $ and thus $F_T(x)  = \sup A_x \leq T_F(y)$
	\end{proof}
	
	\begin{lem}
		Let $F:\R \rightarrow \R$. Then $T_F+F$ and $T_F-F$ are increasing.
	\end{lem}
	
	\begin{ex} \lex{00000} 
		For each $F:\R \rightarrow \C$, $T_{|F|} \leq T_F$.
	\end{ex}
	
	\begin{proof}
		Let $F:\R \rightarrow \C$, $x \in R$ and $(x_i)_{i=0}^n \subset \R$. Suppose that $(x_i)_{i=0}^n$ is increasing and $x_n=x$. Then by the reverse triangle inequality, $$ \sum_{i=1}^n\big||F(x_i)|-|F(x_{i-1})|\big|
		\leq \sum_{i=1}^n\big|F(x_i)-|F(x_{i-1})\big|$$
		Thus 
		\begin{align*}
			T_{|F|}(x) 
			&= \sup\bigg \{\sum_{i=1}^{n}\big||F(x_{i})| - |F(x_{i-1})|\big|: (x_i)_{i=0}^n \subset \R \text{ is increasing and } x_n=x  \bigg \} \\
			& \leq \sup\bigg \{\sum_{i=1}^{n}|F(x_{i}) - F(x_{i-1})|: (x_i)_{i=0}^n \subset \R \text{ is increasing and } x_n=x  \bigg \} \\
			&= T_F(x)
		\end{align*}
		Hence $T_{|F|} \leq T_F$
	\end{proof}
	
	\begin{defn} \ld{00000} 
		Let $F:\R \rightarrow \C$. Then $F$ is said to have \textbf{bounded variation} if $\lim \limits_{x \rightarrow \infty}T_F(x)<\infty$. The \textbf{total variation of $F$}, denoted by $\TV(F)$, is defined to be $\TV(F) = \lim\limits_{x\rightarrow \infty}T_F(x)$.
		We define $\BV = \{F:\R \rightarrow \C: \TV(F)<\infty \}$.
	\end{defn}
	
	\begin{defn} \ld{00000} 
		Let $F:[a,b] \rightarrow \C$. Define $G_F:\R \rightarrow \C$ by $G_F = F(a)\chi_{(-\infty,a)} + F\chi_{[a,b]}+F(b)\chi_{(b,\infty)}$. Then $F$ is said to have \textbf{bounded variation on $[a,b]$} if $G_F \in \BV$. The \textbf{total variation of $F$}, denoted $\TV(F)$, is defined to be $\TV(F) = \TV(G_F)$. We define $\BV(a,b) = \{F:[a,b] \rightarrow \C: \TV(F) < \infty\}$.
	\end{defn}
	
	\begin{note}
		Equivalently, $\TV(F) = \sup \big \{\sum_{i=1}^{n}|F(x_{i}) - F(x_{i-1})|: (x_i)_{i=0}^n \subset [a,b] \text{ is increasing, } x_0=a \text{, and } x_n=b\big \}$ and $F \in \BV(a,b)$ iff $\TV(F) < \infty$. In general, 
	\end{note}
	
	\begin{ex} \lex{00000} 
		Let $F \in \BV$. Then $F$ is bounded.
	\end{ex}
	
	\begin{proof}
		If $F$ is unbounded, then the supremum in the previous definition is clearly infinite.
	\end{proof}
	
	\begin{ex} \lex{00000} 
		Let $F:\R \rightarrow \R$. If $F$ is bounded and increasing, then $F \in \BV$.
	\end{ex}
	
	\begin{proof}
		Suppose that $F$ is bounded and increasing. Then $-\infty<\inf_{x \in \R}F(x) \leq \sup_{x \in \R}F(x)<\infty$. Let $x \in \R$ and $(x_i)_{i=0}^n \subset \R$. Suppose that $(x_i)_{i=0}^n$ is increasing and $x_n=x$. Then 
		\begin{align*}
			\sum_{i=1}^n|F(x_i)-F(x_{i-1})| 
			&= \sum_{i=1}^n F(x_i)-F(x_{i-1})\\
			&= F(x)-F(x_0)
		\end{align*}
		Thus 
		$$T_F(x) = F(x)-\inf_{x \in \R}F(x)$$ 
		This implies that 
		\begin{align*}
			\TV(F) 
			&= \sup_{x \in \R}F(x)-\inf_{x \in \R}F(x)\\
			&<\infty
		\end{align*}
		Hence $F \in \BV$.
	\end{proof}
	
	\begin{ex} \lex{00000} 
		Let $F:\R \rightarrow \C$. If $F$ is differentiable and $F'$ is bounded on $[a,b]$, then, $F \in \BV(a,b)$. 
	\end{ex}
	
	\begin{proof}
		Suppose that $F$ is differentiable and $F'$ is bounded on $[a,b]$. Then there exists $M>0$ such that for each $x \in [a,b]$, $|F(x)| \leq M$. Let $(x_i)_{i=1}^n \subset [a,b]$. Suppose that $(x_i)_{i=1}^n$ is strictly increasing, $x_0=a$ and $x_n=b$. By the mean value theorem, for each $i =1,2, \cdots, n$, there exists $c_i\in (x_{i-1}, x_i)$ such that $F(x_i)-F(x_{i-1})=F'(c_i)(x_i-x_{i-1})$. Then 
		\begin{align*}
			\sum_{i=1}^n|F(x_i)-F(x_{i-1})|
			&= \sum_{i=1}^n|F'(c_i)(x_i-x_{i-1})|\\
			&\leq  \sum_{i=1}^nM(x_i-x_{i-1})\\
			&=M(b-a)
		\end{align*}
		
		Hence $\TV(F) \leq M(b-a)$.
	\end{proof}
	
	\begin{ex} \lex{00000} 
		Define $F,G:\R \rightarrow \R$ by 
		\[ F(x) = \begin{cases}
			x^2 \sin(x^{-1}) & x \neq 0\\
			0 & x=0\\
		\end{cases}$$ and $$G(x)=
		\begin{cases}
			x^2 \sin(x^{-2}) & x \neq 0\\
			0 & x=0
		\end{cases}
		\]
		
		Then $F$ and $G$ are differentiable, $F \in \BV(-1,1)$ and $G \not \in \BV(-1,1)$.
	\end{ex}
	
	\begin{proof}
		On $\R \setminus \{0\}$, 
		\begin{align*}
			F'(x) 
			&= 2x \sin(x^{-1})- \sin(x^{-1})\\
			&= \sin(x^{-1})(2x-1)
		\end{align*} We see that $F$ is also differentiable at $x=0$ since 
		\begin{align*}
			F'(0) 
			&= \lim_{x \rightarrow 0} \frac{F(x)-F(0)}{x-0}\\
			&= \lim_{x \rightarrow 0} \frac{x^2 \sin(x^{-1})}{x}\\
			&= \lim_{x \rightarrow 0} x \sin(x^{-1})\\
			&=0
		\end{align*}
		
		Therefore for each $x \in [-1,1]$, $|F'(x)| \leq 3$. Which by a previous exercise implies that $F \in \BV(-1,1)$.
		
		On $\R \setminus \{0\}$, 
		\begin{align*}
			G'(x)
			&= 2x \sin(x^{-2})-\frac{2 \sin(x^{-2})}{x}\\
			&= \sin(x^{-2})(2x-\frac{2}{x})
		\end{align*}
		We see that $G$ is also differentiable at $x=0$ since 
		\begin{align*}
			G'(0) 
			&= \lim_{x \rightarrow 0} \frac{G(x)-G(0)}{x-0}\\
			&= \lim_{x \rightarrow 0} \frac{x^2 \sin(x^{-2})}{x}\\
			&= \lim_{x \rightarrow 0} x \sin(x^{-2})\\
			&=0
		\end{align*}
		
		For $n \in \N$, define $(x_i)_{i=0}^n \subset [-1,1]$ by $$x_i= \frac{-1}{\sqrt{\frac{\pi}{2}+i\pi}}$$
		
		Then for each $n \in \N$, $(x_i)_{i=1}^n$ is strictly increasing and for each $i=1,2,\cdots, n$ we have that 
		\begin{align*}
			|G(x_i)-G(x_{i-1})|
			&=\frac{1}{\frac{\pi}{2}+i\pi}+ \frac{1}{\frac{\pi}{2}+(i-1)\pi}\\
			&=\frac{2}{\pi}\bigg[\frac{(2i-1)+(2i+1)}{(2i+1)(2i-1)}\bigg]\\
			&=\frac{2}{\pi}\bigg[\frac{4i}{4i^2-1}\bigg]\\
			& > \frac{2}{i\pi}
		\end{align*}
		\newpage
		Hence for each $n \in \N$,
		\begin{align*}
			\TV(G,[-1,1]) 
			&\geq \sum_{i=1}^n|G(x_i)-G(x_{i-1})| \\
			& > \frac{2}{\pi}\sum_{i=1}^n \frac{1}{i}
		\end{align*}
		Therefore $G \not \in \BV([-1,1])$.
	\end{proof}
	
	\begin{ex} \lex{00000}  The following is stated for $\BV$, but is also true for $\BV(a,b)$.
		
		\begin{enumerate} 
			\item For each $F,G \in \BV$, $T_{F+G} \leq T_F + T_G$ and therefore $\BV$ is a vector space. 
			\item For each $F: \R \rightarrow \C$, $F \in \BV$ iff $Re(f) \in \BV$ and $Im(F) \in \BV$.
			\item For each $F:\R \rightarrow \R$, $F \in \BV$ iff there exist functions $F_1,F_2:\R \rightarrow \R$ such that $F_1,F_2$ are bounded, increasing and $F=F_1-F_2$
			\item For each $F \in \BV$ and $x \in \R$, $\lim_{t \rightarrow x^+}F(t)$ and $\lim_{t \rightarrow x^-}F(t)$ exist. 
			\item For each $F \in \BV$, $\{x \in R: F \text{ is not continuous at }x\}$ is countable.
			\item For each $F \in \BV$, $F$ and $F_+$ are differentiable a.e. and $F'=(F_+)'$ a.e.
			\item For each $F \in \BV, c \in \R$, $F-c \in \BV$
		\end{enumerate}
	\end{ex}
	
	\begin{proof}
		\begin{enumerate}
			\item Let $F, G \in \BV$, $x \in \R$ and $\ep>0$. Since $T_{F+G}(x) < \infty$, $T_{F+G}(x)-\ep< T_{F+G}(x)$. Thus there exists $(x_i)_{i=0}^n \subset \R$ such that $(x_i)_{i=0}^n$ is increasing, $x_n=x$ and $T_{F+G}(x) < \sum_{i=1}^n |(F+G)(x_i) - (F+G)(x_{i-1}))|+ \ep$. Thuerefore 
			
			\begin{align*}
				T_{F+G}(x)
				& < \sum_{i=1}^n |(F+G)(x_i) - (F+G)(x_{i-1}))|+ \ep\\
				& \leq \sum_{i=1}^n |F(x_i)-F(x_{i-1})|+ \sum_{i=1}^n|G(x_i)-G(x_{i-1})| + \ep\\
				& \leq T_F(x) + T_G(x) + \ep
			\end{align*}
			Since $\ep >0$ is arbitrary, $T_{F+G}(x) \leq T_F(x)+T_G(x)$. Therefore $\TV(F+G) \leq \TV(F)+\TV(G)<\infty$. Thus $F+G \in \BV$. It is straight forward to verify the other requirements needed to show that $\BV$ is a vector space.
			
			\item Let $F: \R \rightarrow \C$. Write $F=F_1+iF_2$ with $F_1, F_2:\R \rightarrow \R$. Suppose that $F \in \BV$. Note that for each $x_1,x_2 \in \R$ and $j =1,2$, $|F_j(x_1)-F_j(x_2)| \leq |F(x_1)-F(x_2)|$. Let $x\in \R$ and $(x_i)_{i=0}^n \subset \R$. Suppose that $(x_i)_{i=0}^n$ is increasing and $x_n=x$. Then for $j=1,2$ 
			$$\sum_{i=1}^n|F_j(x_i)-F_j(x_{i-1})| \leq \sum_{i=1}^n|F(x_i)-F(x_{i-1})|$$
			Thus for $j=1,2$ we have that $T_{F_j}(x) \leq T_F(x)$ which implies that $Re(f),Im(F) \in \BV$. Conversely, Suppose that $Re(f), Im(F) \in \BV$. Then $F=Re(f)+iIm(f) \in \BV$ by (1).
			\item Suppose that $F \in \BV$. Choose $F_1=\frac{1}{2}(T_F-F)$ and $F_2=\frac{1}{2}(T_F+F)$. Then $F_1,F_2$ are bounded, increasing and $F=F_1+F_2$. Conversely, if there exist $F_1,F_2:\R \rightarrow \R$ such that $F_1,F_2$ are bounded, increasing and $F=F_1-F_2$, then $F_1, F_2 \in \BV$. By (1) $F \in \BV$.
			\item This is clear by previous results and (3)
			\item This is clear by previous results and (3)
			\item This is clear by previous results and (3)
			\item Clearly constant functions have zero total variation. The rest is implied by (1).
		\end{enumerate}
	\end{proof}
	
	\begin{lem}
		Let $F \in BV$. Then $\lim_{x \rightarrow -\infty}T_F(x)=0$ and if $F$ is right continuous, then $T_F$ is right continuous.
	\end{lem}
	
	\begin{defn} \ld{00000} 
		Define $\NBV=\{F \in \BV: F \text{ is right continuous and }\lim_{x \rightarrow -\infty}F(x)=0\}$.
	\end{defn}
	
	\begin{thm}
		Let $M(\R)$ be the set of complex Borel measures on $\R$. For $F \in \NBV$, define $\mu_F \in M(\R)$ by $\mu_F((-\infty, x]) = F(x)$. Then $F \mapsto \mu_F$ defines a bijection $\NBV \rightarrow M(\R)$. In addition, $|\mu_F| = \mu_{T_F}$
	\end{thm}
	
	\begin{thm}
		Let $F \in \NBV$. Then $F' \in L^1(m)$, $\mu_F \perp m$ iff $F' =0$ a.e. and $\mu_F \ll m$ iff for each $x \in \R$, $$\int_{(-\infty, x]}F'\dm = F(x)$$
	\end{thm}
	
	\begin{defn} \ld{00000} 
		Let $F: \R \rightarrow \C$. Then $F$ is said to be \textbf{absolutely continuous} if for each $\ep>0$, there exists $\del>0$ such that for each disjoint $((a_i, b_i))_{i=1}^n \subset \MB(\R)$, $\sum_{i=1}^n b_i-a_i < \del$ implies that $\sum_{i=1}^n|F(b_i)-F(a_i)| < \ep$.
	\end{defn}
	
	\begin{defn} \ld{00000} 
		Let $F: [a,b] \rightarrow \C$. Then $F$ is said to be \textbf{absolutely continuous} if for each $\ep>0$, there exists $\del>0$ such that for each disjoint $((a_i, b_i))_{i=1}^n \subset \MB([a,b])$, $\sum_{i=1}^n b_i-a_i < \del$ implies that $\sum_{i=1}^n|F(b_i)-F(a_i)| < \ep$.
	\end{defn}
	
	\begin{ex} \lex{00000} 
		Let $F:[a,b] \rightarrow \C$. If $F$ is absolutely continuous, then $F \in \BV$.
	\end{ex}
	
	\begin{proof}
	Suppose that $F$ is absolutely continuous. Then for each $j \in \N$, there exists $\del >0$ such that for each disjoint $((a_i, b_i))_{i=1}^n \subset \MB([a,b])$, $\sum\limits_{i=1}^n b_i-a_i < \del$ implies that $\sum\limits_{i=1}^n|F(b_i)-F(a_i)| < 1$ \\
	Define Choose $n^* \in \N$ such that $(b-a)/n < \del$ and define $(x^*_j)_{j=0}^{n^*} \subset [a,b]$ by 
	$$x^*_j = a + \frac{b-a}{n}j$$ 
	Let $(x_j)_{j=1}^n \subset [a,b]$ be increasing. Consider the refinement 
	$$(x'_j)_{j=0}^{n'} = (x_j)_{j=0}^n \cup (x^*_j)_{j=0}^{n^*}$$ 
	For $j \in \{1, \ldots, n\}$, set $k_0 = 0$ and $k_j = \max \{k: x'_k \in [x^*_{j-1}, x^*_j] \}$. Then for each $k \in \{k_{j-1} + 1, \ldots, k_j\}$, $x'_k - x_{k-1}' < \del$. Then 
	\begin{align*}
	\sum_{j=1}^{n'} |F(x'_j) - F(x'_{j-1})| 
	&= \sum_{j=1}^n \sum_{k=k_{j-1}+1}^{k_j} |F(x'_k) - F(x'_{k-1})|  \\
	&< \sum_{j=1}^n 1  \\
	&= n
	\end{align*}
	So $\TV(F) \leq n < \infty$ and $F \in \BV$. 
	\end{proof}
	
	\begin{ex}
	There exists $F:\R \rightarrow \C$ such that $F$ is absolutely continuous and $F \not \in \BV$. 
	\end{ex}	
	
	\begin{proof}
	Define $F: \R \rightarrow \C$ by $F(x) = x$. 
	\end{proof}
	
	\begin{ex} \lex{00000} 
		Let $F: \R \rightarrow \C$. Suppose that there exists $f \in L^1(m)$ such that for each $x \in \R$,
		$$F(x) = \int_{(-\infty, x]}f\dm$$ 
		Then $F \in \NBV$.
	\end{ex}
	
	\begin{proof}
		Let $x \in \R$ and $(x_i)_{i=1}^n \subset \R$. Suppose that $(x_i)_{i=1}^n$ is increasing and $x_n=x$. Then 
		\begin{align*}
			\sum_{i=1}^n|F(x_i)-F(x_{i-1})| 
			&= \sum_{i=1}^n \bigg| \int_{(x_{i-1},x_i]}f\dm \bigg|\\
			& \leq \sum_{i=1}^n \int_{(x_{i-1},x_i]}|f|\dm \\
			& = \int_{(x_0,x]}|f|\dm\\
			& < \int|f|\dm
		\end{align*}
		
		Hence $T_F(x) \leq \int |f|\dm$. Since $x \in \R$ is arbitrary, $\TV(F) \leq \int |f|\dm$. Therefore $F \in \BV$. By the continuity from above and below for measures and the fact that $m({x})=0$ for each $x \in \R$, $F$ is continuous. By continuity from above for measures, $\lim\limits_{x \rightarrow -\infty} F(x) =0$. So $F \in \NBV$.
	\end{proof}
	
	\begin{lem}
		Let $F \in \NBV$. Then $F$ is absolutely continuous iff $\mu_F \ll m$.
	\end{lem}
	
	\begin{ex} \lex{00000} \textbf{The Fundamental Theorem of Calculus:}\\
		Let $F:[a,b] \rightarrow \C$. The following are equivalent:
		\begin{enumerate}
			\item $F$ is absolutely continuous on $[a,b]$.
			\item there exists $f \in L^1([a,b], m)$ such that for each $x \in [a,b]$, 
			$$F(x)-F(a)= \int_{(a,x]}f\dm$$
			\item $F$ is differentiable a.e. on $[a,b]$, $F' \in L^1([a,b], m)$ and for each $x \in [a,b]$, 
			$$F(x)-F(a)=\int_{(a,x]}F'\dm$$
		\end{enumerate}
	\end{ex}
	
	\begin{proof}
		$(1) \implies (3)$ \\
		Suppose that $F$ is absolutely continuous on $[a,b]$. Then $F \in \BV[a,b]$. Extend $F$ to $\R$ by setting $F(x) = F(a)$ for $x<a$ and $F(x)=F(b)$ for $x>b$. Then $G=F-F(a) \in \NBV$ and is absolutely continuus. The previous lemma implies that there exists $f \in L^1(m)$ such that $d \mu_G = f\dm$. A previous theorem implies that for a.e. $x \in [a,b]$
		\begin{align*}
			F'(x) 
			&= \lim_{r \rightarrow x} \frac{\mu_G((x,x+r])}{m((x,x+r])}\\
			&= f(x)
		\end{align*}  
		So $F$ is differentiable a.e. on $[a,b]$, $F' \in L^1([a,b], m)$ and by construction, for each $x \in [a,b]$, we have that
		\begin{align*}
			F(x)-F(a)
			&= \mu_G((a,x])\\
			&= \int_{(a,x]}f\dm\\
			&= \int_{(a,x]}F'\dm
		\end{align*}
		$(3) \implies (2)$\\
		Trivial.\\
		$(2) \implies (1)$\\
		Suppose that there exists $f \in L^1([a,b], m)$ such that for each $x \in [a,b]$, $F(x)-F(a)=\int_{(a,x]}f\dm$. Extend $F$ as before and obtain $G$ as before. Note that a previous exercise implies that $G \in \NBV$. Since $\mu_G \ll m$, the previous lemma implies that $G$ is absolutely continuous.
	\end{proof}
	
	\begin{ex} \lex{00000} 
		Let $F: \R \rightarrow \C$. If $F$ is absolutely continuous. Then $F$ is differentiable a.e.
	\end{ex}
	
	\begin{proof}
		Let $n \in \N$. Since $F$ is absolutely continuous on $\R$, $F$ is absolutely continuous on $[-n,n]$. The FTC implies that $F$ is differentiable a.e. on $[-n,n]$. Since $n \in \N$ is arbitrary, $F$ is differentiable a.e on $\R$.
	\end{proof}
	
	\begin{ex} \lex{00000} 
		Let $F: \R \rightarrow \C$. Then $F$ is Lipschitz continuous iff $F$ is absolutely continuous and $F'$ is bounded a.e.
	\end{ex}
	
	\begin{proof}
		Suppose that $F$ is Lipschitz continuous. Then there exists $M>0$ such that for each $x,y \in \R, |F(x)-F(y)| \leq M|x-y|$. Let $\ep >0$. Choose $\del = \frac{\ep}{M}$. Let $((a_i, b_i))_{i=1}^n \subset \MB(\R)$, Suppose that $\sum_{i=1}^n b_i-a_i < \del$. Then 
		\begin{align*}
			\sum_{i=1}^n|F(b_i)-F(a_i)| 
			&\leq \sum_{i=1}^n M(b_i - a_i)\\
			&< M\del\\
			&= \ep
		\end{align*}
		Hence $F$ is absolutely continuous. For each $x, y \in \R$, if $x \neq y$, then $\bigg|\frac{F(x)-F(y)}{x-y}\bigg| \leq M$. Hence for a.e. $x \in \R$, $|F'(x)| \leq M$. Conversely, suppose that $F$ is absolutely continuous and $F'$ is bounded a.e. Then there exits $M> 0$ such that for a.e. $x \in \R$, $|F'(x)| \leq M$. Let $x,y \in \R$. Suppose $x<y$. Then the FTC implies that 
		\begin{align*}
			|F(y)-F(x)|
			& = \bigg|\int_{(x,y]}F'\dm\bigg|\\
			&\leq \int_{(x,y]}|F'|\dm\\
			&=M|y-x|
		\end{align*}
		and $F$ is Lipschitz continuous.
	\end{proof}
	
	\begin{ex} \lex{00000} 
		Construct an increasing function $F: \R \rightarrow \R$ whose discontinuities is $\Q$.
	\end{ex}
	
	\begin{proof}
		Let $(q_n)_{n\in \N}$ be an ennumeration of $\Q$. Define $F:\R \rightarrow \R$ by $$F =\sum_{n\in \N} 2^{-n}\chi_{[q_n, \infty)}$$ 
		Equivalently, if we define $S_x=\{n \in \N: q_n \leq x\}$, then we may write $$F(x) = \sum_{n \in S_x}2^{-n}$$\\ Let $x, y \in \R$. Suppose that $x < y$. Then $S_x \subsetneq S_y$. So $F(x)<F(y)$ and therefore $F$ is strictly increasing.\\ 
		For each $x,y \in R$ with $x<y$, define $S_{x,y}= \{n \in \N:x< q_n \leq y\}$. Note that $\lim\limits_{y\rightarrow x^+}\min(S_{x,y}) = \infty$ and if $y \in \R \setminus \Q$, then $\lim\limits_{x\rightarrow y^-}\min(S_{x,y}) = \infty$.\\
		Now, let $x \in \R$ and $\ep>0$. Choose $N \in \N$ such that $\sum\limits_{n = N}^{\infty}2^{-n}<\ep$. Choose $\del>0$ such that $\min (S_{x,x+\del}) \geq N$.  Let $y \in [x,\infty)$. Suppose that $|x-y| < \del$. Then 
		\begin{align*}
			|F(x)-F(y)| 
			&= \sum_{n \in S_y}2^{-n} - \sum_{n \in S_x}2^{-n}\\
			& = \sum_{n \in S_{x,y}}2^{-n}\\
			& \leq \sum_{n=N}^{\infty}2^{-n}\\
			&<\ep
		\end{align*} 
		
		Hence $F$ is right continuous. Now let $x \in \R\setminus \Q$ and $\ep>0$. Choose $N \in \N$ as before and $\del>0$ such that $\min(S_{x-\del,x})\geq N$. Let $y \in (-\infty, x]$. Suppose that $|x-y|<\del$. Then 
		\begin{align*}
			|F(x)-F(y)| 
			&= \sum_{n \in S_x}2^{-n} - \sum_{n \in S_y}2^{-n}\\
			& = \sum_{n \in S_{y,x}}2^{-n}\\
			& \leq \sum_{n=N}^{\infty}2^{-n}\\
			&<\ep
		\end{align*}
		
		Hence $F$ is left continuous on $\R\setminus \Q$.\\
		Now, let $x \in \Q$. Then there exists $j \in \N$ such that $q_j=x$. Choose $\ep=2^{-j}$. Let $\del>0$. Choose $y=x-\frac{\del}{2}$. Then $|x-y|< \del$ and 
		\begin{align*}
			|F(x)-F(y)|
			&= \sum_{n \in S_{y,x}}2^{-n}\\
			&\geq 2^{-j}\\
			&= \ep
		\end{align*}
		
		Hence $F$ is discontinuous from the left at $x$. Since $x \in \Q$ is arbitrary, $F$ is discontinuous from the left on $\Q$.  
	\end{proof}
	
	\begin{ex} \lex{00000} 
		Let $(F_n)_{n\in \N} \in \NBV$ be a sequence of nonnegative, increasing functions. If for each $x \in \R$, $F(x)=\sum_{n \in \N}F_n(x)< \infty$, then for a.e. $x \in \R$, $F$ is differentiable at $x$ and $F'(x) = \sum_{n\in \N}F_n'(x)$. 
	\end{ex}
	
	\begin{proof}
		
		Define $\mu = \sum_{n \in \N}\mu_{F_n}$. Note that 
		\begin{align*}
			\mu((-\infty,x]) 
			&= \sum_{n \in \N}\mu_{F_n}((-\infty,x]) \\
			&= \sum_{n \in \N}F_n(x)\\
			&= F(x)
		\end{align*}
		Hence $F \in \NBV$ and $\mu=\mu_F$. For each $n \in \N$, there exist $\lam_n \in M(\R)$ and $f \in L^1(\R)$ such that $d\mu_{F_n} = d\lam_n + f_n\dm$ and $\lam \perp m$. Since for each $n \in \N$, $\lam_n, f_n$ are nonnegative, we have that $d\mu_F =  \sum_{n \in \N} d\lam_n + (\sum_{n \in \N}f_n)\dm$. By a previous theorem, for a.e. $x \in \R$, 
		\begin{align*}
			F'(x) 
			&= \lim_{r \rightarrow 0}\frac{\mu_F((x,x+r])}{m((x,x+r])} \\
			&= \sum_{n \in \N}f_n(x)\\
			&= \sum_{n \in \N}\lim_{r \rightarrow 0}\frac{\mu_{F_n}((x,x+r])}{m((x,x+r])} \\
			&= \sum_{n \in \N}F_n'(x)
		\end{align*}
	\end{proof}
	
	\begin{ex} \lex{00000} 
		Let $F:[0,1]\rightarrow [0, 1]$ be the Cantor function. Extend $F$ to $\R$ by setting $F(x) = 0$ for $x<0$ and $F(x)=1$ for $x>1$. Let $([a_n,b_n])_{n \in \N}$ be an ennumeration of the closed subintervals of $[0,1]$ with rational endpoints. For $n \in \N$, define $F_n:\R \rightarrow [0,1]$ by $F_n(x) = F(\frac{x-a_n}{b_n-a_n})$. Define $G:\R \rightarrow \R$ by $G = \sum_{n \in \N}2^{-n}F_n$. Then $G$ is continuous, strictly increasing on $[0,1]$ and $G'=0$ a.e.
	\end{ex}
	
	\begin{proof}
		Since $F$ is continuous on $\R$, we have that for each $n \in \N$, $F_n$ is continuous on $\R$. We observe that for each $x \in \R$ and $n \in \N$, $|2^{-n}F_n(x)| \leq 2^{-n}$. Thus the Weierstrass M-test implies that $G$ converges uniformly on $\R$ and is therefore continuous. Since $F$ is increasing, for each $n \in \N$, $F_n$ is increasing. Let $x, y \in \R$. Suppose that $x<y$. Choose $j \in \N$ such that $x<a_j<y<b_j$. Then 
		\begin{align*}
			G(x) 
			&= \sum_{n \in \N}2^{-n}F_n(x)\\
			&= \sum_{\substack{n \in \N\\ n \neq j}}2^{-n}F_n(x) + 0\\
			& < \sum_{\substack{n \in \N\\ n \neq j}}2^{-n}F_n(y) + 2^{-j}F_n(y)\\
			&=\sum_{n \in \N}2^{-n}F_n(y)\\
			&=G(y)
		\end{align*}
		So $G$ is strictly increasing.\\
		Now we observe that for each $n \in \N$, $F_n \in \NBV$. The previous exercise implies that $$G' = \sum 2^{-n}F_n'=0 \text{ a.e.}$$
	\end{proof}
	
	
	
	
	
	
	
	
	
	
	
	
	
	
	
	
	
	
	
	
	
	
	
	
	
	
	
	
	
	
	
	
	
	
	\newpage
	\subsection{Disintegration of Measures}
	
	\begin{ex} \lex{00000} 
		Let $(X, \MA, \mu)$ be a measure space and $\MB \subset \MA$ a sub $\sig$-algebra. Define $\mu_{\MB} = \mu|_{\MB}$. Let $f \in L^1(X, \MB, \mu_{\MB})$. Then 
		\begin{enumerate}
			\item $ L^1(X, \MB, \mu_{\MB}) \subset L^1(X, \MA, \mu)$ 
			\item for each $f \in L^1(X, \MB, \mu)$ and $B \in \MB$, $$\int_B f \dmu_{\MB} = \int_B f \dmu$$
		\end{enumerate}
	\end{ex}
	
	\begin{proof}
		Let $f \in L^1(X, \MB, \mu)$. Clearly $f$ is $\MA$-measurable. If $f$ is simple, then there exist $(b_i)_{i=1}^n \subset \Rg$ and $(B_i)_{i=1}^n \subset \MB$ such that $$f = \sum_{i=1}^n b_i \chi_{B_i}$$ such that for each $i \in \{1, \cdots, n\}$, 
		\begin{align*}
			\infty 
			&> \mu_{\MB}(B_i) \\
			&= \mu(B_i)
		\end{align*}
		So $f \in L^1(X, \MA, \mu)$ and 
		\begin{align*}
			\int_B f \dmu_{\MB} 
			&= \int_B \sum_{i=1}^n b_i \chi_{B_i} \dmu_{\MB} \\
			&= \sum_{i=1}^n b_i \mu_{\MB}(B_i \cap B)\\
			&= \sum_{i=1}^n b_i \mu(B_i \cap B)\\
			&= \int_B \sum_{i=1}^n b_i\chi_{B_i} \dmu \\
			&= \int_B f \dmu
		\end{align*}
		If $f \geq 0$, then there exist $(\phi_n)_{n \in \N} \subset S^+(X, \MB)$ such that for each $n \in \N$, $\phi_n \leq \phi_{n+1} \leq f$ and $\phi_n \convt{p.w.} f$. The monotone convergence theorem implies that for each $B \in \MB$,
		\begin{align*}
			\int_B f \dmu
			&= \limn \int_B \phi_n \dmu \\
			&= \limn \int_B \phi_n \dmu_{\MB} \\
			& = \int_B f \dmu_{\MB} \\
			& < \infty
		\end{align*}
		So $f \in L^1(X, \MA, \mu)$.
		Similarly, the statement also holds for general $f \in L^1(X, \MB, \mu_B)$ by writing $f = g+ih$ and applying the above to $g^+$, $g^-$, $h^+$ and $h^-$.
	\end{proof}
	
	\begin{note}
		Denote the $L^1$ norms on $L^1(X, \MA, \mu)$ and $L^1(X, \MB, \mu_{\MB})$ by $N$ and $N_{\MB}$ respectively. The previous exercise implies that $L^1(X, \MB, \mu_{\MB})$ is a subspace of $L^1(X, \MA, \mu)$ and $N|_{L^1(X, \MB, \mu_{\MB})} = N_{\MB}$.
	\end{note}
	
	\begin{ex}
		Let $(X, \MA, \mu)$ be a measure space, $\MB$ a sub $\sig$-algebra of $\MA$ and $f \in L^1(X, \MA, \mu)$. Define $\mu_{\MB}: \MB \rightarrow [0, \infty] $ and $\nu_f: \MB \rightarrow [0,\infty)$ by $\mu_{\MB} = \mu|_{\MB}$ and 
		$$\nu_f(B) = \int_B f \dmu $$ Then $\nu_f \ll \mu_{\MB}$. 
	\end{ex}	
	
	\begin{proof}
		Let $B \in \MB$. Suppose that $\mu_{\MB}(B) = 0$. By definition, $\mu(B) = 0$. So $\nu(B) = 0$ and $\nu \ll \mu_{\MB}$.
	\end{proof}
	
	\begin{note}
		Since $\nu_f \ll \mu_{\MB}$ and $\nu_f(X) < \infty$, if $\mu$ is $\sig$-finite, then $d \nu_{f} / d \mu_{\MB}$ exists and 
		\begin{align*}
			d \nu_{f} / d \mu_{\MB} 
			& \in L^1(X, \MB, \mu_{\MB}) \\
			& \subset L^1(X, \MA, \mu)
		\end{align*}
	\end{note}
	
	\begin{defn}
		Let $(X, \MA, \mu)$ be a $\sig$-finite measure space and $\MB$ a sub $\sig$-algebra of $\MA$. We define the \textbf{projection from $L^1(X, \MA, \mu)$ to $L^1(X, \MB, \mu_{\MB})$}, denoted $P_{\MB}:L^1(X, \MA, \mu) \rightarrow L^1(X, \MB, \mu_{\MB})$ by 
		$$P_{\MB}f = \frac{d \nu_f}{d \mu_{\MB}}$$ 
	\end{defn}
	
	\begin{ex}
		Let $(X, \MA, \mu)$ be a $\sig$-finite measure space and $\MB$ a sub $\sig$-algebra of $\MA$. Then 
		\begin{enumerate}
			\item $P_{\MB} \in L(L^1(X, \MA, \mu))$ and $\|P_{\MB}\| = 1$
			\item $P_{\MB}|_{L^1(X, \MB, \mu_{\MB})} = \id_{L^1(X, \MA, \mu_{\MB})}$ 
			\item $P_{\MB}$ is idempotent
		\end{enumerate}
	\end{ex}
	
	\begin{proof}\
		\begin{enumerate}
			\item Let $f, g \in L^1(X, \MA, \mu)$ and $\lam \in \C$. For each $B \in \MB$, we have that 
			\begin{align*}
				\nu_{f + \lam g} (B) 
				& = \int_{B} f + \lam g \dmu \\
				& = \int_{B} f  \dmu + \lam \int_{B} g \dmu \\
				& = \nu_{f}(B) + \lam \nu_{g}(B) \\
				& = (\nu_{f} + \lam \nu_{g})(B)
			\end{align*}
			Hence $\nu_{f + \lam g} + \nu_{f} + \lam \nu_{g}$. Thus 
			\begin{align*}
				P_{\MB}(f + \lam g)
				& = \frac{d \nu_{f + \lam g}}{d \mu_{\MB}} \\
				& = \frac{d \nu_f }{d \mu_{\MB}} + \lam \frac{d \nu_{g}}{d \mu_{\MB}} \\
				& = P_{\MB} f + \lam P_{\MB} g
			\end{align*}
			So $P_{\MB}$ is linear. Since $|P_{\MB}f| \in L^1(X, \MB, \mu_{\MB})$, a previous exercise implies that
			\begin{align*}
				\|P_{\MB}f\|_1
				& = \int |P_{\MB}f| \dmu \\
				& = \int |P_{\MB}f| \dmu_{\MB} \\
				& = |\nu_{f}|(X) \\
				& = \int |f| \dmu \\
				& = \|f\|_1
			\end{align*}
			Hence $\|P_{\MB} f \|_1 = \|f\|_1$ and $P_{\MB} \in L(L^1(X, \MA, \mu))$.
			\item Let $f \in L^1(X, \MB, \mu_{\MB})$. Then for each $B \in \MB$, 
			\begin{align*}
				\nu_f(B)
				& = \int_B f \dmu \\
				& = \int_B f \dmu_{\MB} 
			\end{align*}
			Uniqueness of the Radon-Nikodym derivative implies that $P_{\MB}f = f$. Since $f \in L^1(X, \MB, \mu_{\MB})$ is arbitrary, $P_{\MB}|_{L^1(X, \MB, \mu_{\MB})} = \id_{L^1(X, \MA, \mu_{\MB})}$.
			\item Let $f \in L^1(X, \MA, \mu)$. Since $P_{\MB}f \in L^1(X, \MB, \mu_{\MB})$ and $P_{\MB}|_{L^1(X, \MB, \mu_{\MB})} = \id_{L^1(X, \MA, \mu_{\MB})}$, we have that
			\begin{align*}
				P_{\MB}^2 f 
				& = P_{\MB}(P_{\MB} f) \\
				& = \id_{L^1(X, \MB, \mu_{\MB})} (P_{\MB} f) \\
				& = P_{\MB}f
			\end{align*}
			Since $f \in L^1(X, \MB, \mu_{\MB})$ is arbitrary, $P_{\MB}^2 = P_{\MB}$ and $P_{\MB}$ is idempotent. 
		\end{enumerate}
	\end{proof}

	\begin{ex}
		in{ex}
		Let $(X, \MA, \mu)$ be a $\sig$-finite measure space, $\MB$ a sub $\sig$-algebra of $\MA$, $f \in L^1(X, \MA, \mu)$ and $g \in L^1(X, \MB, \mu_{\MB})$. Then $g = P_{\MB}f$ iff for each $B \in \MB$, 
		$$\int_B g \dmu = \int_B f \dmu$$
	\end{ex}

	\begin{proof}
		Suppose that $g = P_{\MB}f$. Let $B \in \MB$. Then 
		\begin{align*}
			\int_B g \dmu 
			& = \int_B g \dmu_{\MB} \\
			& = \nu_f(B) \\
			& = \int_B f \dmu 
		\end{align*}
		Since $B \in \MB$ is arbitrary, for each $B \in \MB$, 
		$$\int_B g \dmu = \int_B f \dmu$$ 
		Conversely, suppose that for each $B \in \MB$, 
		$$\int_B g \dmu = \int_B f \dmu$$ 
		Then for each $B \in \MB$,
		\begin{align*}
			\int_B g \dmu_{\MB} 
			& = \int_B g \dmu \\
			& = \int_B f \dmu \\
			& = \nu_f(B) 
		\end{align*} 
		By definition, 
		\begin{align*}
			P_{\MB}f 
			& =  \frac{d \nu_f}{d \mu_{\MB}} \\
			& = g
		\end{align*} 
	\end{proof}
	
	\begin{ex}
		Let $(X, \MA, \mu)$ be a $\sig$-finite measure space, $\MB$ a sub $\sig$-algebra of $\MA$ and $(A_{j})_{j \in \N} \subset \MA$. Suppose that $(A_j)_{j \in \N}$ is disjoint and $\mu \bigg(\bigcup\limits_{j \in \N} A_j \bigg) < \infty$. Then 
		\begin{enumerate}
			\item $\chi_{\bigcup\limits_{j \in \N} A_j} \in L^1(X, \MA, \mu)$
			\item $P_{\MB} \chi_{\bigcup\limits_{j \in \N} A_j} = \sum\limits_{j \in \N} P_{\MB}\chi_{A_j}$
		\end{enumerate}
	\end{ex}
	
	\begin{proof}\
		\begin{enumerate}
			\item Since $(A_j)_{j \in \N}$ is disjoint, we have that 
			\begin{align*}
				\|\chi_{\bigcup\limits_{j \in \N} A_j}\|_1 
				& = \int \chi_{\bigcup\limits_{j \in \N} A_j} \dmu \\
				& = \mu\bigg(\bigcup\limits_{j \in \N} A_j \bigg) \\
				& < \infty
			\end{align*}
			So $\chi_{\bigcup\limits_{j \in \N} A_j} \in L^1(X, \MA, \mu)$.
			\item Since  $(A_j)_{j \in \N}$ is disjoint, we have that 
			$$\chi_{\bigcup\limits_{j \in \N} A_j} = \sum\limits_{j \in \N} \chi_{A_j}$$
			For each $n \in \N$, define $f_n = \sum\limits_{j = 1}^n \chi_{A_j}$. Set $f = \chi_{\bigcup\limits_{j \in \N} A_j}$. Then for each $n \in \N$, $f_n \leq f$ and $f_n \convt{p.w.} f$. Since $f \in L^1(X, \MA, \mu)$, the dominated convergence theorem implies that $f_n \conv{L^1(\mu)} f$. Since $P_{\MB} \in L(L^1(X, \MA, \mu))$, 
			\begin{align*}
				\sum_{j=1}^n P_{\MB} \chi_{A_j}
				& = P_{\MB} \sum_{j=1}^n \chi_{A_j} \\
				& = P_{B}f_n \\
				& \conv{L^1(\mu)} P_{\MB}f \\
				& = P_{\MB} \chi_{\bigcup\limits_{j \in \N} A_j} 
			\end{align*}  
			Hence  $P_{\MB} \chi_{\bigcup\limits_{j \in \N} A_j} = \sum\limits_{j \in \N} P_{\MB}\chi_{A_j}$.
		\end{enumerate}
	\end{proof}

	\begin{ex}
		Let $(X, \MA, \mu)$ be a $\sig$-finite measure space, $\MB$ a sub $\sig$-algebra of $\MA$ and $f \in L^1(X, \MA, \mu)$. If $f \geq 0$, then $P_{\MB}f \geq 0$ $\mu_{\MB}$-a.e.
	\end{ex}

	\begin{proof}
		Suppose that $f \geq 0$. Then $\nu_f: \MB \rightarrow [0, \infty)$ is a finite measure. Hence 
		\begin{align*}
			P_{\MB}f 
			& = \frac{d \nu_f}{d \mu_{\MB}} \\
			& \geq 0 \text{ $\mu_{\MB}$-a.e.}
		\end{align*}
	\end{proof}

	\begin{ex}
		Let $\mu$ be a finite measure on $(\R, \MB(\R))$ and $\MB$ a sub $\sig$-algebra of $\MB(\R)$. For each $z \in \R$, define $h_z \in L^1(X, \MA, \mu)$ by $h_z = \chi_{(-\infty, z]}$ and choose $f_z \in L_0(X, \MA)$ such that $f_z = P_{\MB}h_z$ $\mu$-a.e. Then there exists $M \in \MB$ such that $\mu_{\MB}(M^c) = 0$ and for each $x \in M$, $(f_q(x))_{q \in \Q}$ is increasing.
	\end{ex}
	
	\begin{proof}
		Let $q, r \in \Q$. Suppose that $q < r$. Then $\chi_{(-\infty, r]} - \chi_{(-\infty, q]} \geq 0$ and
		\begin{align*}
			f_r - f_q 
			& = P_{\MB} \chi_{(-\infty, r]} - P_{\MB} \chi_{(-\infty, q]} \\
			& = P_{\MB} \bigg[ \chi_{(-\infty, r]} - \chi_{(-\infty, q]} \bigg] \\\
			& \geq 0 \text{ $\mu_{\MB}$-a.e.}
		\end{align*}
		Hence $f_q \leq f_r$ $\mu_{\MB}$-a.e. An exercise in the section on measures implies that $(f_q)_{q \in \Q}$ is increasing $\mu_{\MB}$-a.e. and thus there exists $M \in \MB$ such that $\mu_{\MB}(M^c) = 0$ and for each $x \in M$, $(f_q(x))_{q \in \Q}$ is increasing.
	\end{proof}
	
	\begin{ex}
		Let $\mu$ be a finite measure on $(\R, \MB(\R))$ and $\MB$ a sub $\sig$-algebra of $\MB(\R)$. Define $(h_z)_{z \in \R} \subset L^1(\R, \MB(\R), \mu)$, $(f_z)_{z \in \R} \subset L^0(\R, \MB)$ and $M \in \MB$ as in the previous exercise. Choose $g \in \NBV(\R)$ such that $g: \R \rightarrow \R$, $g$ is increasing and $\sup\limits_{z \in \R} g(z) = 1$. Define $G: \R^2 \rightarrow \R$ by 
		\[
		G(z, x) = 
		\begin{cases}
			\inf\limits_{\substack{q \in \Q \\ q > z}}f_q(x) & x \in M \\
			g(z) & x \in M^c
		\end{cases}
		\] 
		Then for each $x \in \R$, $G(\cdot, x)$ is increasing and right continuous.
	\end{ex}

	\begin{proof}
		Let $x \in \R$. If $x \in M^c$, by defintion, $G(\cdot, x)$ is increasing and right continuous. Suppose that $x \in M$. Since $(f_q(x))_{q \in \Q}$ is increasing, slightly modifying the statement and proof of an exercise in the section on functions of bounded variation implies that $G(\cdot, x)$ is increasing and right continuous.
	\end{proof}

	\begin{ex}
		Let $\mu$ be a finite measure on $(\R, \MB(\R))$ and $\MB$ a sub $\sig$-algebra of $\MB(\R)$. Define $(h_z)_{z \in \R} \subset L^1(\R, \MB(\R), \mu)$, $(f_z)_{z \in \R} \subset L^0(\R, \MB)$, $M \in \MB$ and $G:\R^2 \rightarrow \R$ as in the previous exercise.
		\begin{enumerate}
			\item for each $z \in \R$, $G(z, \cdot) \in L^0(X, \MB)$ and $G(z, \cdot) = f_z$ $\mu_B$-a.e.
			\item $\sup\limits_{z \in \R} G(z, \cdot) = 1$ $\mu_{\MB}$-a.e.
			\item $\inf\limits_{z \in \R} G(z, \cdot) = 0$ $\mu_{\MB}$-a.e.
		\end{enumerate}
	\end{ex}
	
	\begin{proof}\
		\begin{enumerate}
			\item Let $z \in \R$. By definition,
			$$G(z, \cdot) = \inf\limits_{\substack{q \in \Q \\ q > z}}[ f_q \chi_M ] (\cdot) + g(z) \chi_{M^c}(\cdot)$$ 
			Since $(f_{q} \chi_{M})_{q \in \Q\cap (z, \infty)} \subset L^0(X, \MB)$ and is point-wise bounded below, $\inf\limits_{\substack{q \in \Q \\ q > z}} f_q \chi_M  \in L^0(X, \MB)$. Hence $G(z, \cdot) \in L^0(X, \MB)$. Choose $(q_n)_{n \in \N} \subset \Q$ such that for each $n \in \N$, $q_n \geq q_{n+1} > z$ and $q_n \rightarrow z$. Since for each $n \in \N$, $h_{q_n} - h_{z} = \chi_{(z, q_n]}$, $(z, q_{n+1}] \subset (z, q_n]$ and $\mu$ is finite, we have that 
			\begin{align*}
				\|h_{q_n} - h_z\|_1
				& = \|\chi_{(z, q_n]}\|_1 \\
				& = \mu((z, q_n]) \\
				& \rightarrow \mu(\varnothing) \\
				& = 0
			\end{align*}
			So that $h_{q_n} \conv{L^1(\mu)} h_z$. Therefore  
			\begin{align*}
				f_{q_n} 
				& = P_{\MB} h_{q_n} \\
				& \conv{L^1(\mu_{\MB})} P_{\MB} h_z \\
				& = f_z
			\end{align*}
			This implies that $f_{q_n} \conv{\mu_{\MB}} f_z$. Since $(f_{q_n})_{n \in \N}$ is decreasing $\mu_{\MB}$-a.e., an exercise in the section on modes of convergence implies that $f_{q_n} \convt{$\mu_{\MB}$-a.e.} f_z$. So there exists $N_1 \in \MB$ such that $\mu_{\MB}(N_1^c) = 0$ and $f_{q_n} \chi_{fi} \convt{p.w.} f_z \chi_{N_1}$. Set $E = M \cap N_1$. Then  
			\begin{align*}
				\mu_{\MB}(E^c)
				& = \mu_{\MB}(M^c \cup N_1^c) \\ 
				& \leq \mu_{\MB}(M^c) + \mu_{\MB}(N_1^c) \\
				& = 0
			\end{align*}
			and for each $x \in E$, $f_{q_n}(x) \rightarrow f_z(x)$ and $f_{q_n}(x) \rightarrow G(z, x)$. Hence $G(z, \cdot)\chi_{E}(\cdot) = f_z \chi_{E}( \cdot)$ which implies that $G(z, \cdot) = f_z$ $\mu_B$-a.e.\\
			\item Part $(1)$ implies that for each $n \in \N$, there exists $E_n \in \MB$ such that $E_n \subset M$, $\mu(E_n^c) = 0$ and $G(n, \cdot)\chi_{E_n}(\cdot) = f_n(\cdot) \chi_{E_n}(\cdot)$. Set $E = \bigcap\limits_{n \in \N} E_n$. Since for each $n \in \N$, $\chi_{\R} - h_n = \chi_{(n, \infty)}$, $(n+1, \infty) \subset (n, \infty)$ and $\mu$ is finite, we have that 
			\begin{align*}
				\|h_n - \chi_{\R}\|_1
				& = \mu((n, \infty)) \\
				& \rightarrow \mu(\varnothing) \\
				& = 0 
			\end{align*}
			So that $h_{n} \conv{L^1(\mu)} \chi_{\R}$. Therefore  
			\begin{align*}
				f_n 
				& = P_{\MB} h_n \\
				& \conv{L^1(\mu_{\MB})} P_{\MB} \chi_{\R} \\
				& = \chi_{\R}
			\end{align*}
			This implies that $f_n \conv{\mu_{\MB}} \chi_{\R}$. Since $(f_n)_{n \in \N}$ is increasing $\mu_{\MB}$-a.e., an exercise in the section on modes of convergence implies that $f_n \convt{$\mu_{\MB}$-a.e.} \chi_{\R}$. So there exists $N_2 \in \MB$ such that $\mu_{\MB}(N_2^c) = 0$ and $f_n \chi_{N_2} \convt{p.w.} \chi_{N_2}$. Set $M^+ = E \cap N_2$. Then  $M^+ \subset E \subset M$ and
			\begin{align*}
				\mu_{\MB}((M^+)^c)
				& = \mu_{\MB}(E^c \cup N_2^c) \\ 
				& \leq \mu_{\MB}(E^c) + \mu_{\MB}(N_2^c) \\
				& = \mu_{\MB} \bigg( \bigcup_{n \in \N} E_n^c \bigg) + \mu_{\MB}(N_2^c) \\
				& \leq \bigg[\sum_{n \in \N} \mu_{\MB}(E_n^c) \bigg] + \mu_{\MB}(N_2^c) \\
				& = 0
			\end{align*}
			Since $M^+ \subset M$, for each $x \in M^+$, $(f_n(x))_{n \in \N}$ is increasing. Hence for each $x \in M^+$,
			\begin{align*}
				\sup\limits_{z \in \R} G(z, x) 
				& = \sup\limits_{n \in \N} G(n, x) \\
				& = \sup\limits_{n \in \N} f_n(x) \\
				& = 1
			\end{align*}
			Thus $\sup\limits_{z \in \R} G(z, \cdot) = 1$ $\mu_{\MB}$-a.e.\\
		\item Part $(2)$ implies that for each $n \in \N$, there exists $E_n \in \MB$ such that $E_n \subset M$, $\mu(E_n^c) = 0$ and $G(n, \cdot)\chi_{E_n}(\cdot) = f_n(\cdot) \chi_{E_n}(\cdot)$. Set $E = \bigcap\limits_{n \in \N} E_n$. Since for each $n \in \N$, $h_{-n} = \chi_{(-\infty, -n)}$, $(-\infty, -(n + 1)) \subset (-\infty, -n)$ and $\mu$ is finite, we have that 
		\begin{align*}
			\|h_{-n}\|_1
			& = \mu((-\infty, -n)) \\
			& \rightarrow \mu(\varnothing) \\
			& = 0 
		\end{align*}
		So that $h_{-n} \conv{L^1(\mu)} 0$. Therefore  
		\begin{align*}
			f_{-n}
			& = P_{\MB} h_{-n} \\
			& \conv{L^1(\mu_{\MB})} P_{\MB} 0 \\
			& = 0
		\end{align*}
		This implies that $f_n \conv{\mu_{\MB}} 0$. Since $(f_{-n})_{n \in \N}$ is decreasing $\mu_{\MB}$-a.e., an exercise in the section on modes of convergence implies that $f_{-n} \convt{$\mu_{\MB}$-a.e.} 0$. So there exists $N_3 \in \MB$ such that $\mu_{\MB}(N_3^c) = 0$ and $f_{-n} \chi_{N_3} \convt{p.w.} 0$. Set $M^- = E \cap N_3$. Then $M^- \subset E \subset M$ and
		\begin{align*}
			\mu_{\MB}((M^-)^c)
			& = \mu_{\MB}(E^c \cup N_3^c) \\ 
			& \leq \mu_{\MB}(E^c) + \mu_{\MB}(N_3^c) \\
			& = \mu_{\MB} \bigg( \bigcup_{n \in \N} E_n^c \bigg) + \mu_{\MB}(N_3^c) \\
			& \leq \bigg[\sum_{n \in \N} \mu_{\MB}(E_n^c) \bigg] + \mu_{\MB}(N_3^c) \\
			& = 0
		\end{align*}
		Since $M^- \subset M$, for each $x \in M^-$, $(f_{-n}(x))_{n \in \N}$ is decreasing. Hence for each $x \in M^-$,
		\begin{align*}
			\inf\limits_{z \in \R} G(z, x) 
			& = \inf\limits_{n \in \N} G(-n, x) \\
			& = \inf\limits_{n \in \N} f_{-n}(x) \\
			& = 0
		\end{align*}
		Thus $\inf\limits_{z \in \R} G(z, \cdot) = 0$ $\mu_{\MB}$-a.e.
		\end{enumerate}
	\end{proof}

	\begin{ex}
		Let $\mu$ be a finite measure on $(\R, \MB(\R))$ and $\MB$ a sub $\sig$-algebra of $\MB(\R)$. Then there exists $F: \R^2 \rightarrow [0,1]$ such that
		\begin{enumerate}
			\item for each $z \in \R$, $F(z, \cdot) \in L^0(X, \MB)$ and $F(z, \cdot) = P_{\MB}\chi_{(-\infty, z]}$ $\mu_B$-a.e.
			\item for each $x \in \R$, $F(\cdot, x) \in \NBV(\R)$, $F(\cdot, x)$, increasing and $\sup\limits_{z \in \R} F(z, \cdot) = 1$.
		\end{enumerate}
	\end{ex}

	\begin{proof}
		Define $(h_z)_{z \in \R} \subset L^1(\R, \MB(\R), \mu)$, $(f_z)_{z \in \R} \subset L^0(\R, \MB)$ as in the previous exercises. Choose $g \in \NBV(\R)$ such that $g: \R \rightarrow \R$, $g$ is increasing and $\sup\limits_{z \in \R} g(z) = 1$. Define $M, M^+, M^- \in \MB$ and $G: \R^2 \rightarrow \R$ as in the previous exercises. Set $E = M \cap M^+ \cap M^-$. Define $F: \R^2 \rightarrow \R$ by 
		\[
		F(z, x) = G(z, x) \chi_{E}(x) + g(z) \chi_{E^c}(x)
		\]
		\begin{enumerate}
			\item Let $z \in \R$. Then $F(z, \cdot) = G(z, \cdot) \chi_{E}(\cdot) + g(z) \chi_{E^c}(\cdot)$. Since $G(z, \cdot) \in L^0(X, \MB)$, $F(z, \cdot) \in L^0(X, \MB)$. 
			Note that
			\begin{align*}
				\mu(E^c)
				& = \mu(M^c \cup (M^+)^c \cup (M^-)^c) \\  
				& \leq \mu(M^c) + \mu((M^+)^c) + \mu((M^-)^c) \\
				& = 0
			\end{align*}
			Since $E \subset M$, by definition of $G$ and $F$, we have that for each $x \in E$, $F(z, x) = G(z, x)$. Hence $\{x \in \R: F(z, x) \neq G(z, x)\} \subset E^c$. Thus 
			\begin{align*}
				F(z, \cdot) 
				& = G(z, \cdot) \\
				& = f_z \text{ $\mu_{\MB}$-a.e.}
			\end{align*}
			\item Let $x \in \R$. Suppose that $x \in E$. The previous exercise implies that $G(\cdot, x) \in \NBV(\R)$, $G(\cdot, x)$ is increasing and $\sup\limits_{z \in \R} G(z, x) = 1$. Since $F(\cdot, x) = G(\cdot, x)$, we have that $F(\cdot, x) \in \NBV(\R)$, $F(\cdot, x)$ is increasing and $\sup\limits_{z \in \R} F(z, x) = 1$. \\
			If $x \in E^c$, then $F(\cdot, x) = g$. By definition of $g$, $F(\cdot, x) \in \NBV(\R)$, $F(\cdot, x)$, increasing and $\sup\limits_{z \in \R} F(z, \cdot) = 1$.
		\end{enumerate}
	\end{proof}
	
	\begin{defn}
		Let $(X, \MA)$ and $(Y, \MB)$ be  measurable spaces and $\kap: X \times \MB \rightarrow [0,1]$. Then $\kap$ is said to be a \textbf{Markov kernel from $(X, \MA)$ to $(Y, \MB)$} if 
		\begin{enumerate}
			\item for each $x \in X$, $\kap(x, \cdot)$ is a probability measure on $(Y, \MB)$
			\item for each $B \in \MB$, $\kap(\cdot, B)$ is $\MA$-measurable
		\end{enumerate}
	\end{defn}

	\begin{ex}
		Let $(\R, \MB(\R), \mu)$ be a finite measure space and $\MB$ a sub $\sig$-algebra of $\MB(\R)$. Then there exists $\kap: \R \times \MB(\R) \rightarrow [0,1]$ such that 
		\begin{enumerate}
			\item $\kap$ is a markov kernel from $(\R, \MB)$ to $(\R, \MB(\R))$.
			\item For each $A \in \MB(\R)$, $\kap(\cdot, A) = P_{\MB}\chi_{A}$ $\mu_{\MB}$-a.e.
		\end{enumerate}
		\textbf{Hint:} 
		\begin{enumerate}
			\item Consider $F: \R^2 \rightarrow [0,1]$ defined in the previous exercise and $\mu_x((a,b]) = F(b,x) - F(a,x)$. 
			\item Consider Dynkin's lemma. 
		\end{enumerate}
	\end{ex}
	
	\begin{proof} Define $F:\R^2 \rightarrow [0,1]$ as in the previous exercise. For each $x \in \R$, define $\mu_x: \MB(\R) \rightarrow [0,1]$ to be the unique measure such that for each $a, b \in \R$, $a \leq b$ implies that $\mu_x((a,b]) = F(b, x) - F(a, x)$. Define $\kap: \R \times \MB(\R) \rightarrow [0,1]$ by $\kap(A, x) = \mu_x(A)$. \\
		\begin{enumerate}
			\item \
			\begin{enumerate}
				\item Let $x \in \R$. By definition, $\kap(x, \cdot) = \mu_x$ is a measure and 
				\begin{align*}
					\kap(x, \R) 
					& = \sup_{n \in \N} \mu_x((-\infty, n])\\
					& = \sup_{n \in \N} F(n) \\
					& = 1 
				\end{align*}
				\item Let $A \in \MB(\R)$. Recall that for each $x \in \R$,
				$$\mu_x(A) = \inf \bigg \{ \sum\limits_{j \in \N} F(b_j, x) - F(a_j, x): \text{ for each $j \in \N$, $a_j, b_j \in \R$ and } A \subset \bigcup\limits_{j \in \N} (a_j, b_j] \bigg \}$$ 
				Therefore, for each $x \in \R$ and $n \in \N$, there exist $(a^x_{n,j})_{j \in \N}$, $(b^x_{n,j})_{j \in \N} \subset \R$ such that $A \subset \bigcup\limits_{j \in \N} (a^x_{n,j}, b^x_{n,j}]$ and 
				$$\mu_x(A) \leq \sum\limits_{j \in \N} F(b^x_{n,j}, x) - F(a^x_{n,j}, x) < \mu_x(A) + \frac{1}{n}$$
				Define $(f_n)_{n \in \N} \subset L^0(X, \MB)$ by $$f_n(x) = \sum\limits_{j \in \N} F(b^x_{n,j}, x) - F(a^x_{n,j}, x)$$
				Then $f_n \convt{p.w.} \kap(\cdot, A)$ which implies that $\kap(\cdot, A) \in L^0(X, \MB)$. 
			\end{enumerate}
			Hence $\kap$ is a markov kernel from $(\R, \MB)$ to $(\R, \MB(\R))$.\\
			\item Let $B \in \MB$. Define $\nu_B, \lam_{B}:\MB(\R) \rightarrow \Rg$ by 
			$$\nu_B(A) = \int_B \kap(x, A) \dmu_{\MB}(x)$$ 
			and
			$$\lam_B(A) = \mu(A \cap B)$$ 
			Let $a,b \in \R$. Then 
			\begin{align*}
				\nu_B((a,b]) 
				& = \int_B \kap(x, (a,b]) \dmu_{\MB}(x) \\
				& = \int_{B} F(b, x) - F(a,x) \dmu_{\MB}(x) \\
				& = \int_{B} P_{\MB}\chi_{(-\infty, b]} - P_{\MB}\chi_{(-\infty, a]} \dmu_{\MB} \\
				& = \int_{B} P_{\MB}\chi_{(a, b]} \dmu_{\MB} \\
				& = \int_{B} \chi_{(a, b]} \dmu  \\
				& = \mu((a,b] \cap B) \\
				& = \lam_{B}((a,b])
			\end{align*}
			Define $\MP \subset \MB(\R)$ by $\MP = \{(a,b]: a,b \in \R\} \cup \{\varnothing, X\}$. A previous exercise in the sections on Dynkin's lemma implies that $\MP$ is a $\pi$-system. Since $\sig(\MP) = \MB(\R)$, an exercise in the section on complex measures implyand $\nu_{B} = \lam_{B}$. Let $A \in \MB(\R)$. Then 
			\begin{align*}
				\int_B \kap(x, A) \dmu_{\MB}(x) 
				& = \nu_B(A) \\
				& = \lam_{B}(A) \\
				& =  \mu(A \cap B) \\
				& = \int_B \chi_A \dmu \\
				& = \int_B P_{\MB} \chi_A \dmu \\
				& = \int_B P_{\MB} \chi_A \dmu_{\MB} \\
			\end{align*}
			Since $B \in \MB$ is arbitrary, $\kap(\cdot, A) = P_{\MB} \chi_A$ $\mu_{\MB}$-a.e. Since $A \in \MB(\R)$ is arbitrary, we have that for each $A \in \MB(\R)$, $\kap(\cdot, A) = P_{\MB} \chi_A$ $\mu_{\MB}$-a.e.
		\end{enumerate}
	\end{proof}
	
	\begin{ex} 
		Let $(X, \MA, \mu)$ be a measure space, $(Y, \MB)$ a measurable space, $f \in L^1(X, \MA, \mu)$ and $g: X \rightarrow Y$. Suppose that for each $y \in Y$, $\{y\} \in \MB$, $g$ is surjective and $g$ is $(\MA, \MB)$-measurable. Then there exists a $\phi \in L^0(Y, \MB)$ such that $\phi \circ g = \MP_{g^*\MB}f$ $\mu$-a.e. and $\phi$ is unique $g_*\mu$-a.e. \\
		\textbf{Hint:} Doob-Dynkin lemma
	\end{ex}	
	
	\begin{proof}\
		\begin{itemize}
			\item \textbf{Existence:} \\
			Since $P_{g^*\MB}f \in L^1(X, g^*\MB, \mu_{g^*\MB})$ and $\MB$, the Doob-Dynkin lemma implies that there exists a $\phi \in L^0(Y, \MB)$ such that $\phi \circ g = P_{g^*\MB}f$.
			\item \textbf{Uniqueness:} \\
			Suppose that there exists $\psi \in L^0(Y, \MB)$ such that $\psi \circ g = \MP_{g^*\MB}f$ $\mu$-a.e. Then $\phi \circ g = \psi \circ g$ $\mu$-a.e. An exercise in the section on integration of nonnegative functions implies that $\phi = \psi$ $g_*\mu$-a.e.
		\end{itemize} 
	\end{proof}
	
	\begin{ex}
		Let $(X, \MA, \mu)$ be a finite measure space, $(Y, \MB)$ a measurable space and $g: X \rightarrow Y$. Suppose that for each $y \in Y$, $\{y\} \in \MB$, $g$ is surjective and $g$ is $(\MA, \MB)$-measurable. Then there exists $\kap: Y \times \MA \rightarrow \Rg$ such that $\kap$ is a transition kernel from $(Y, \MB)$ to $(X, \MA)$.\\
		\textbf{Hint:} For $A \in \MA$, define $\phi_A \in L^0(Y, \MB )$ to be the $g_*\mu$-a.e.\ unique $\phi \in L^0(Y, \MB)$ such that $\phi \circ g = P_{g^*\MB} \chi_A$. Define $\kap': Y \times \MA \rightarrow \Rg$ by $\kap'(y, A) = \phi_A(y)$. For each $A \in \MA$, define $\kap(\cdot, A)$ by redefining $\kap'(\cdot, A)$ on a $g_*\mu$-null set.
	\end{ex}
	
	\begin{proof}\
		\begin{itemize}
			\item Since $\chi_{\varnothing} = 0$, $P_{g^*\MB}\chi_{\varnothing} = 0$ $\mu$-a.e. Therefore
			\begin{align*}
				0 \circ g 
				& =  0 \\
				& = P_{g^*\MB}\chi_{\varnothing} \text{ $\mu$-a.e.}
			\end{align*}
			Uniqueness of $\phi_{\varnothing}$ implies that $\phi_{\varnothing} = 0$ $g_*\mu$-a.e. Thus there exists $N_1 \in \MB$ such that $g_*\mu(N_1) = 0$ and for each $y \in N_1^c$, 
			\begin{align*}
				\kap'(y, \varnothing) 
				& = \phi_{\varnothing}(y) \\
				& = 0
			\end{align*}
			\item Let $(A_j)_{j \in \N} \subset \MA$. Suppose that $(A_j)_{j \in \N}$ is disjoint. Since $\mu$ is finite, $\mu \bigg( \bigcup\limits_{j \in \N}A_j \bigg) < \infty$. A previous exercise implies that  
			\begin{enumerate}
				\item $\chi_{\bigcup\limits_{j \in \N} A_j} \in L^1(X, \MA, \mu)$
				\item $P_{\MB} \chi_{\bigcup\limits_{j \in \N} A_j} = \sum\limits_{j \in \N} P_{\MB}\chi_{A_j}$
			\end{enumerate}
			Therefore
			\begin{align*}
				\phi_{\bigcup\limits_{j \in \N} A_j} \circ g
				& = P_{\MB} \chi_{\bigcup\limits_{j \in \N} A_j} \\
				& = \sum\limits_{j \in \N} P_{\MB}\chi_{A_j} \\
				& = \sum\limits_{j \in \N} \phi_{A_j} \circ g \text{ $\mu$-a.e.}
			\end{align*}
			Uniqueness of $\phi_{\bigcup\limits_{j \in \N} A_j}$ implies that $\phi_{\bigcup\limits_{j \in \N} A_j} = \sum\limits_{j \in \N} \phi_{A_j}$ $g_*\mu$-a.e. $\phi_{\varnothing}$ implies that $\phi_{\varnothing} = 0$ $g_*\mu$-a.e. Thus there exists $N_2 \in \MB$ such that $g_*\mu(N_2) = 0$ and for each $y \in N_2^c$,
			\begin{align*}
				\kap' \bigg(y, \bigcup\limits_{j \in \N} A_j \bigg)
				& = \phi_{\bigcup\limits_{j \in \N} A_j}(y) \\
				& = \sum\limits_{j \in \N} \phi_{A_j}(y) \\
				& = \sum\limits_{j \in \N} \mu_y(A_j) \\
				& =  \sum\limits_{j \in \N} \kap'(y, A_j)
			\end{align*}
		\end{itemize}
		Set $N = N_1 \cup N_2$. Then $g_*\mu(N) = 0$ and for each $y \in N^c$, $\kap'(y, \cdot): \MA \rightarrow \Rg$ is a measure on $(X, \MA)$. Choose $x \in X$. Define $\kap: Y \times \MA \rightarrow \Rg$ by $\kap (y, A) = \chi_{N}(y)\del_{x}(A) + \chi_{N^c}(y)\kap'(y, A)$. 
		\begin{enumerate}
			\item Let $A \in \MA$. Then 
			\begin{align*}
				\kap(\cdot, A) 
				& = \chi_N(\cdot) \del_x(A) + \chi_{N^c}(\cdot) \kap'(\cdot, A) \\ 
				& = \chi_N(\cdot) \del_x(A) + \chi_{N^c}(\cdot) \phi_A(\cdot) 
			\end{align*}
			Hence for each $A \in \MA$, $\kap(\cdot, A)$ is $\MB$-measurable.
			\item \
			Let $y \in Y$. Then 
			\[
			\kap(y, \cdot) = 
			\begin{cases}
				\del_x(\cdot) & y \in N \\
				\kap'(y, \cdot) & y \in N^c \\
			\end{cases}
			\]
			Hence for each $y \in Y$, $\kap(y, \cdot)$ is a measure on $(X, \MA)$. 
		\end{enumerate}
		Thus $\kap$ is a transition kernel from $(Y, \MB, g_*\mu)$ to $(X, \MA)$.
	\end{proof}
	
	\begin{defn}
		Let $(X, \MA, \mu)$ be a finite measure space, $(Y, \MB)$ a measurable space and $g: X \rightarrow Y$. Suppose that for each $y \in Y$, $\{y\} \in \MB$, $g$ is surjective and $g$ is $(\MA, \MB)$-measurable. For $A \in \MA$, define $\phi_A \in L^0(Y, \MB )$ to be the $g_*\mu$-a.e. unique $\phi \in L^0(Y, \MB)$ such that $\phi \circ g = P_{g^*\MB} \chi_A$. For $y \in Y$, we define the \textbf{conditional of $\mu$ on $y$}, denoted $\mu_y : \MA \rightarrow \Rg$, by $\mu_y(A) = \phi_A(y)$. 
	\end{defn}
	
	\begin{ex} \textbf{Disintegration of Measure:} \\
		Let $(X, \MA, \mu)$ be a finite measure space, $(Y, \MB)$ a measurable space and $g: X \rightarrow Y$. Suppose that for each $y \in Y$, $\{y\} \in \MB$, $g$ is surjective and $g$ is $(\MA, \MB)$-measurable. Then there exists a collection of measures $(\mu_y)_{y \in Y}$ such that  
		\begin{enumerate}
			\item for each $A \in \MA$, 
			$$\mu(A) = \int \mu_y(A) \, d g_*\mu(y) $$
			\item for each $f \in L^1(X, \MA, \mu)$, 
			$$\int f \dmu = \int \bigg[ \int f \, d\mu_y(x) \bigg] \, d g_*\mu(y)$$ 
		\end{enumerate}
	\end{ex}
	
	
	
	
	
	
	
	
	
	
	
	
	
	
	
	
	
	
	
	
	
	
	
	
	\newpage
	\section{$L^{p}$ Spaces}
	
	\subsection{Introduction}
	
	\begin{defn} \ld{00000} 
		Let $(X, \MA, \mu)$ be a measure space and $p \in (0, \infty]$. Define $  \| \cdot \|_p : L^0(X, \MA, \mu) \rightarrow [0, \infty]$ by $$\|f \|_p = \bigg(\int | f |^p \dmu \bigg)^{\frac{1}{p}} \hspace{1.5cm}( p < \infty)$$ 
		and 
		$$\|f \|_{\infty} = \inf \bigg \{\lam >0: \mu\big(\{x \in X: \lam < | f(x) |  \}\big) = 0 \bigg \} $$
		We define $$L^p(X, \MA, \mu) =  \{f \in L^0(X, \MA, \mu): \|f \|_p < \infty \}$$
	\end{defn}
	
	\begin{ex} \lex{00000} 
	Let $(X, \MA, \mu)$ be a measure space, $p \in (0, \infty]$ and $f,g \in L^p(X, \MA, \mu)$. If $|f| \leq |g|$ $\mu$-a.e., then $\|f\|_p \leq \|g\|_p$.
	\end{ex}
	
	\begin{proof}
	Suppose that $|f| \leq |g|$ $\mu$-a.e. Then $|f|^p \leq |g|^p$ $\mu$-a.e. This implies that $$\int |f|^p \dmu \leq  \int |g|^p \dmu$$ Hence $\|f\|_p \leq \|g\|_p$.
	\end{proof}
	
	\begin{thm}{\textbf{Hölder's Inequality:}}
		Let $(X, \MA, \mu)$ be a measure space, $p,q \in [1, \infty)$ and $f,g \in L^0$. Suppose that $\frac{1}{p} + \frac{1}{q} = 1$. Then $$\|fg\|_1 \leq \|f \|_p \|g \|_q$$
	\end{thm}
	
	\begin{ex} \lex{00000} \textbf{Minkowski Inequality:}
		Let $(X, \MA, \mu)$ be a measure space, $p \in [1, \infty)$ and $f,g \in L^p$. Then $f+g \in L^p$ and $$\|f+g\|_p  \leq \|f\|_p + \|g\|_p $$
	\end{ex}
	
	\begin{proof}
		Define $\phi:\R \rightarrow \Rg$ by $\phi(x) = | x |^p$. Then $\phi$ is convex because it is the composition of an increasing convex function with a convex function. By Jensen's inequality, we have that $$\phi\bigg(\frac{1}{2}[f+g] \bigg) \leq \frac{1}{2}[\phi(f)+\phi(g)]$$ 
		This implies that $$\frac{1}{2^p} | f+g|^p \leq \frac{1}{2}\bigg(| f|^p +| g |^p\bigg)$$ 
		Hence 
		\begin{align*}
			\int| f + g|^p \dmu 
			& \leq 2^{p-1}\int | f|^p +| g|^p \dmu \\
			& = 2^{p-1}\bigg(\int | f|^p \dmu + \int | g|^p \dmu \bigg) \\
			&= 2^{p-1}\bigg( \|f \|_p^p + \|g \|_p^p\bigg) \\
			& < \infty
		\end{align*}
		So $f+g \in L^p$. Now, it is not hard to see that $|f+g|^p \leq \big( |f| + |g| \big)|f+g|^{p-1}$. Let $q$ be the conjugate of $p$, so that $\frac{1}{p} + \frac{1}{q} = 1$. Then $q(p-1) = p$. We use Hölder's inequality to show that 
		\begin{align*}
			\|f+g \|_p ^p
			&= \int  |f+g|^p \dmu \\
			& \leq \int |f| |f+g|^{p-1} \dmu + \int |g| |f+g|^{p-1} \dmu \\
			& \leq \|f\|_p \bigg(\int |f+g|^{(p-1)q} \dmu\bigg)^{\frac{1}{q}} + \|g\|_p \bigg(\int |f+g|^{(p-1)q}\dmu\bigg)^{\frac{1}{q}} \\
			&= \|f\|_p \bigg(\int |f+g|^{p} \dmu\bigg)^{\frac{1}{q}} + \|g\|_p \bigg(\int |f+g|^{p}\dmu\bigg)^{\frac{1}{q}} \\ 
			&= (\|f\|_p + \|g \|_p) \bigg(\int |f+g|^{p} \dmu\bigg)^{\frac{1}{q}}\\
			&= (\|f \|_p + \|g \|_p) \|f+g \|_p^{p/q}
		\end{align*}
		Since $\|f+g \|_p < \infty$, we see that
		\begin{align*}
			\|f \|_p + \|g \|_p 
			& \geq \|f+g \|_p ^{p - p/q} \\
			&=  \|f+g \|_p ^{p(1 - 1/q)} \\
			&= \|f+g \|_p ^{p/p} \\
			&= \|f+g \|_p
		\end{align*}
	\end{proof}
	
	\begin{ex} \lex{00000} 
		Let $(X, \MA, \mu)$ be a measure space, $p,q \in (0, \infty]$. Suppose that $\mu(X) < \infty$ and $p < q$. Then $L^q \subset L^p$. In particular, if $\mu(X) = 1$, then for each $f \in L^q$, $\|f\|_p \leq \|f\|_q$.
	\end{ex}
	
	\begin{proof}
		Suppose that $q = \infty$. Let $f \in L^q$. Then
		\begin{align*}
			\|f \|_p 
			&= \bigg(\int | f |^p \dmu \bigg)^{\frac{1}{p}} \\
			& \leq \bigg(\int \| f \|_{\infty} ^p \dmu \bigg)^{\frac{1}{p}} \\
			&= \|f \|_{\infty} \mu(X)^{\frac{1}{p}}
		\end{align*} 
		If $q < \infty$, then $\frac{q}{p} > 1$ and the conjugate of $\frac{q}{p}$ is $\frac{1}{1- p/q}$. By Hölder's inequality, we have that 
		\begin{align*}
			\|f \|_p^p 
			&= \|f^p \|_1 \\
			&\leq \|f^p \|_{\frac{q}{p}} \|1 \|_{\frac{1}{1-p/q}} \\
			&= \bigg(\int |f|^{\frac{pq}{p}} \dmu \bigg)^{\frac{p}{q}} \mu(X)^{1-\frac{p}{q}} \\
			&= \bigg(\int |f|^{q} \dmu \bigg)^{\frac{p}{q}}\mu(X)^{1-\frac{p}{q}} \\
			&= \|f \|_q^p\mu(X)^{1-\frac{p}{q}}
		\end{align*}
		Hence 
		\begin{align*}
			\|f \|_p 
			&\leq \|f \|_q\mu(X)^{\frac{1}{p}-\frac{1}{q}} \\
			&< \infty
		\end{align*}
	\end{proof}
	
	\begin{ex} \lex{00000} 
	Let $(X, \MA, \mu)$ and $(Y, \MB, \nu)$ be $\sig$-finite measure spaces and $K\in L^0(X \times Y)$. Suppose that there exists $C > 0$ such that for $\mu$-a.e $x \in X$, $$\int_Y |K(x, y)| d \nu(y) < C$$ and for $\nu$-a.e $y \in Y$, $$\int_X |K(x, y)| \dmu(x) < C$$
	Let $f \in L^p(\nu)$.
	\begin{enumerate}
	\item Then for $\mu$-a.e. $x \in X$, $$\int_Y K(x,y) f(y) d \nu(y)$$ exists. \\
	\textbf{Hint:} Note that $|K(x, y) f(y)| = (|K(x,y)|^{1/q})(|K(x,y)|^{1/p}|f(y)|)$
	\item Define $Tf \in L^0(X)$ by $$Tf(x) = \int_Y K(x, y) f(y)d \nu(y)$$ Then $Tf \in L^p(\mu)$ and $\|Tf\|_p \leq C\|f\|_p$. 
	\end{enumerate}
	\end{ex}
	
	\begin{proof}Let $p,q \in (0,\infty)$ be conjugate.
	\begin{enumerate}
	\item Define $h \in L^0(X \times Y)$ by $h(x,y) = K(x, y) f(y)$. By assumption, there exists $N \in \MA$ such that $\mu(N) = 0$ and  $$\bigg \{x \in X: \int_Y |K(x, y)| d \nu(y) < C \bigg\} \subset N^c$$ 
	Let $x \in N^c$. Then Holder's inequality implies that
	\begin{align*}
	\int_Y |h(x,y)| d\nu(y) 
	&= \int_Y (|K(x,y)|^{1/q})(|K(x,y)|^{1/p}|f(y)|) d\nu(y) \\
	& \leq \bigg( \int_Y |K(x,y)| d \nu(y) \bigg)^{1/q} \bigg( \int_Y |K(x,y)||f(y)|^p d\nu(y) \bigg)^{1/p} \\
	& \leq C^{1/q} \bigg( \int_Y |K(x,y)||f(y)|^p d\nu(y) \bigg)^{1/p}
	\end{align*}
	Tonelli's theorem implies that the map $$x \mapsto \int_Y |h(x,y)| d\nu(y) $$ is measurable and that  
	\begin{align*}
	\int_X \bigg[ \int_Y |h(x,y)| d\nu(y) \bigg]^p \dmu(x) 
	&\leq  C^{p/q}  \int_X \bigg[ \int_Y |K(x,y)||f(y)|^p d\nu(y) \bigg] \dmu(x) \\
	&= C^{p/q}  \int_Y \bigg[ \int_X |K(x,y)||f(y)|^p \dmu(x) \bigg] d\nu(y) \\
	&= C^{p/q}  \int_Y \bigg[ \int_X |K(x,y)| \dmu(x) \bigg] |f(y)|^p d\nu(y) \\
	& \leq C^{1 + p/q} \int_Y |f(y)|^p  d\nu(y) \\
	& = C^{1 + p/q} \|f\|_p^p
	\end{align*}
	So for $\mu$-a.e. $x \in X$, $$\int_Y |h(x,y)| d\nu(y) < \infty$$ which implies that for $\mu$-a.e. $x \in X$, $h(x, \cdot) \in L^1(\nu)$. Therefore, for $\mu$-a.e. $x \in X$, $$\int_Y h(x,y) d\nu(y)$$ exists. The case is similar when $p \in \{1, \infty\}$.
	\item Let $x \in X$. Then $$|Tf(x)| \leq \int_Y |K(x,y)f(y)| d \nu(y)$$ which implies that $$|Tf(x)|^p \leq \bigg( \int_Y |K(x,y)f(y)| d \nu(y) \bigg)^p$$
	By part $(1)$, $$\int_X |Tf|^p \dmu \leq C^{1+p/q}\|f\|_p^p$$ 
	So $Tf \in L^p(\mu)$ and $\|Tf\|_p \leq C\|f\|_p$.
	The case is similar when $p \in \{1, \infty\}$.
	\end{enumerate}
	\end{proof}
	
	
	
	
	
	
	

	
	
	
	
	

	
	
	
	\newpage
	\section{Borel Measures}
	
	\subsection{Radon Measures}
	
	\begin{defn} \ld{00000} 
	Let $X$ be a topological space, $\mu: \MB(X) \rightarrow \RG$ a measure and $E \in \MB(X)$. Then $\mu$ is said to be 
	\begin{enumerate}
	\item \textbf{inner regular on $E$} if
	$$\mu(E) = \sup \{ \mu(K): K \subset E \text{ and $K$ is compact}\}$$
	\item \textbf{outer regular on $E$} if
	$$\mu(E) = \inf \{ \mu(U): E \subset U \text{ and $U$ is open}\}$$
	\item \textbf{regular on $E$} if $\mu$ is inner regular on $E$ and $\mu$ is outer regular on $E$
	\end{enumerate}
	\end{defn}
	
	\begin{defn} \ld{00000} 
	Let $X$ be a topological space and $\mu: \MB(X) \rightarrow \RG$ a measure. Then $\mu$ is said to be 
	\begin{enumerate}
	\item  \textbf{inner regular} if for each $E \in \MA$, $\mu$ is inner regular on $E$
	\item  \textbf{outer regular} if for each $E \in \MA$, $\mu$ is outer regular on $E$
	\item  \textbf{regular} if $\mu$ is inner regular and $\mu$ is outer regular
	\end{enumerate}
	\end{defn}
	
	\begin{defn} \ld{00000} 
	Let $X$ be a topological space, $\mu: \MB(X) \rightarrow \RG$ a measure and $E \in \MB(X)$. Then $\mu$ is said to be a \textbf{Radon measure} if for each $E \in \MB(X)$, 
	\begin{enumerate}
	\item $E$ is compact implies that $\mu(E) < \infty$
	\item $\mu$ is outer regular on $E$
	\item $E$ is open implies that $\mu$ is inner regular on $E$
	\end{enumerate}
	\end{defn}

	\begin{defn}
		Let $X$ be a topological space, $\mu \in \MM(X)$. Then $\mu$ is said to be \textbf{Radon} if $\|\mu\|$ is Radon.  
	\end{defn}
	
	\begin{ex} \lex{00000} 
	Let $X$ be a topological space and $\mu: \MB(X) \rightarrow \RG$ a Radon measure. Set $$\MN_{\mu} = \{U \subset X: U \text{ is open and } \mu(U) = 0\}$$ and $$N_{\mu} = \bigcup_{U \in \MN_{\mu}} U$$ 
	Then $N_{\mu}$ is open, $N_{\mu}^c$ is closed and $\mu(N_{\mu}) = 0$.\\
	\textbf{Hint:} use inner regularity and compactness
	\end{ex}
	
	\begin{proof}
	Since $N_{\mu}$ is the union of open sets, it is open and $N_{\mu}^c$ is closed since $N_{\mu}$ is open. Let $K \subset N_{\mu}$. Suppose that $K$ is compact. Since $\MN_{\mu}$ is an open cover for $K$, there exist $U_1, \ldots, U_n \in \MN_{\mu}$ such that $$K \subset \bigcup_{j=1}^n U_j$$ 
	This implies that 
	\begin{align*}
	\mu(K) 
	&\leq \mu\bigg( \bigcup_{j=1}^n U_j \bigg) \\
	& \leq \sum_{j=1}^n \mu(U_j) \\
	&= 0
	\end{align*}
	Inner regularity implies that 
	\begin{align*}
	\mu(N_{\mu}) 
	&= \sup \{ \mu(K): K \subset N_{\mu} \text{ and $K$ is compact}\} \\
	&= 0
	\end{align*}
	\end{proof}
	
	\begin{defn} \ld{00000} 
	Let $X$ be a topological space and $\mu: \MB(X) \rightarrow \RG$ a Radon measure. Define $\MN_{\mu}$ and $N_{\mu}$ as in the previous exercise. We define the \textbf{support of $\mu$}, denoted $\supp(\mu)$, by $$\supp(\mu) = N_{\mu}^c $$
	\end{defn}
	
	\begin{ex} \lex{00000} 
	Let $X$ be a topological space, $\mu: \MB(X) \rightarrow \RG$ a Radon measure and $E \in \MB(X)$. If $\mu(E) < \infty$, then for each $\ep >0$, 
	\begin{enumerate}
	\item there exists $U \in \MB(X)$ such that $U$ is open, $E \subset U$ and $\mu(U \setminus E) < \ep$ 
	\item there exists $C \in \MB(X)$ such that $C$ is compact, $C \subset U$ and $\mu(U) - \ep < \mu(C)$ 
	\item there exists $V \in \MB(X)$ such that $V$ is open, $U \setminus E \subset V$ and $\mu(V) < \ep$ 
	\end{enumerate}
	\end{ex}
	
	\begin{proof}
	Suppose that $\mu(E) < \infty$. Let $\ep >0$. 
	\begin{enumerate}
	\item Outer regularity om $E$ implies that there exists $U \in \MB(X)$ such that $U$ is open, $E \subset U$ and $\mu(U \setminus E) < \ep$. 
	\item Inner regularity on $U$ implies that there exists $C \in \MB(X)$ such that $C$ is compact, $C \subset U$ and $\mu(U) - \ep < \mu(C) $.
	\item Outer regularity on $U \setminus E$ implies that there exists $V \in \MB(X)$ such that $U$ and $V$ are open, $U \setminus E \subset V$ and $\mu(V) < \ep$.
	\end{enumerate}
	\end{proof}
	
	\begin{ex} \lex{00000} 
	Let $X$ be a topological space, $\mu: \MB(X) \rightarrow \RG$ a Radon measure and $E \in \MA$. Suppose that $\mu(E) < \infty$. Let $\ep >0$. Define $U$, $C$ and $V$ as in the previous exercise. Set $K = C \setminus V$. Then $K$ is compact, $K \subset E$ and $\mu(K) > \mu(E) - 2 \ep$ 
	\end{ex}
	
	\begin{proof}
	Since $C$ is closed and $V$ is open, $C \setminus V =  C \cap V^c$ is closed. Since $C$ is compact and $C \setminus V \subset C$, we have that $K = C \setminus V $ is compact.
	Set algebra implies that
	\begin{align*}
	K
	&= C \cap V^c \\
	& \subset U \cap V^c \\
	& \subset U \cap (U^c \cup E) \\
	&= (U \cap U^c) \cup (U \cap E) \\
	&=  U \cap E \\
	& \subset E
	\end{align*}
	The previous exercise implies that 
	\begin{align*}
	\mu(K)
	&= \mu(C \cap V^c) \\
	&= \mu(C) - \mu(C \cap V) \\
	&> \mu(U) - \ep - \mu(V) \\
	&> \mu(E) - 2 \ep
	\end{align*}
	\end{proof}		
	
	\begin{ex} \lex{00000} 
	Let $X$ be a topological space, $\mu: \MB(X) \rightarrow \RG$ a Radon measure and $E \in \MB(X)$. If $E$ is $\sig$-finite, then $\mu$ is inner regular on $E$. \\
	\textbf{Hint:} use the previous exercise
	\end{ex}
	
	\begin{proof}
	Suppose that $E$ is $\sig$-finite. \\
	If $\mu(E) < \infty$, the previous exercise implies that for each $\ep > 0 $, there exists $K \in \MB(X)$ such that $K$ is compact, $K \subset E$ and $\mu(K) > \mu(E) - \ep$. Hence $\mu$ is inner regular on $E$. \\
	If $\mu(E) = \infty$, then $\sig$-finiteness implies that there exists $(E_j)_{j \in \N} \subset \MB(X)$ such that $E = \bigcup\limits_{j \in N} E_j$, for each $j \in \N$, $\mu(E_j) < \infty$ and $\mu(E_j) \rightarrow \infty$. Let $N \in \N$. Choose $J \in \N$ such that $\mu(E_J) > N$. The above argument implies that there exists $K \in \MB(X)$ such that $K$ is compact, $K \subset E_J \subset E$ and $\mu(K) > N$. So 
	\begin{align*}
	\mu(E)
	&= \infty \\
	&= \sup_{\substack{ K \subset E \\ \text{ $K$ is compact}}} \mu(K)
	\end{align*}	 
and $\mu$ is inner regular on $E$.  
	\end{proof}		
	
	\begin{ex} \lex{00000} 
	Let $X$ be a topological space and $\mu: \MB(X) \rightarrow \RG$ a Radon measure. If $\mu$ is $\sig$-finite, then $\mu$ is regular.
	\end{ex}
	
	\begin{proof}
	Clear by previous exercise.
	\end{proof}
	
	\begin{ex} \lex{00000} 
	Let $X$ be a topological space and $\mu: \MB(X) \rightarrow \RG$ a Radon measure. If $X$ is $\sig$-compact, then $\mu$ is $\sig$-finite. The previous exercise implies that $\mu$ is regular. 
	\end{ex}
	
	\begin{proof}
	If $X$ is $\sig$-compact, then $\mu$ is $\sig$-finite. Hence $\mu$ is regular.
	\end{proof}
	
		
	
	\begin{ex} \lex{00000} 
	Let $X$ be a topological space and $\mu:\MB(X) \rightarrow \RG$ a Radon measure. Then for each $p \in [1, \infty]$, $C_c(X) \subset L^p(\mu)$.
	\end{ex}
	
	\begin{proof}
	Let $p \in [1, \infty]$ and $f \in C_c(X)$. Then $|f|^p \in C_c(X)$ and 
	\begin{align*}
	\|f\|_p 
	&= \int |f|^p \dmu \\
	& \leq \|f\|_{\infty}^p \mu(\supp (f)) \\
	& < \infty
\end{align*}	 
	\end{proof}
	
	
	
	
	
	
	
	
	
	
	
	
	\newpage	
	\subsection{Radon Measures on LCH Spaces}
	
	\begin{defn} \ld{00000} 
	Let $X$ be a topological space and $I: C_c(X) \rightarrow \C$ a linear functional. Then $I$ is said to be \textbf{positive} if for each $f \in C_c(X, \R)$, $f \geq 0 $ implies that $I(f) \geq 0$.
	\end{defn}
	
	\begin{ex} \lex{00000} 
	Let $X$ be a topological space, $I: C_c(X) \rightarrow \C$ a positive linear functional and $f,g \in C_c(X, \R)$. If $f \geq g$, then $I(f) \geq I (g)$.  
	\end{ex}	
	
	\begin{proof}
	Suppose that $f \geq g$. Then $f - g \geq 0$. So 
	\begin{align*}
	I(f) - I(g) 
	&= I(f -g) \\
	&\geq 0
	\end{align*}
	\end{proof}
	
	
	
	\begin{ex} \lex{00000} 
	Let $X$ be a LCH space, $I: C_c(X) \rightarrow \C$ a positive linear functional. Then for each $K \subset X$, $K $ is compact implies that there exists $C_K \geq 0$ such that for each $f \in C_c(X)$, if $\supp(f) \subset K$, then $I(f) \leq C_K \|f\|_{\infty}$.\\
	\textbf{Hint:} Urysohn's lemma 
	\end{ex}
	
	\begin{proof}
	Let $K \subset X$. Suppose that $K$ is compact. Then Urysohn's lemma implies that there exists $\phi \in C_c(X)$ such that $0 \leq \phi \leq 1$ and $\phi|_K = 1$. Then $I(\phi) \geq 0$. Choose $C_K = I(\phi)$. Let $f \in C_c(X)$. Suppose that $\supp(f) \subset K$. Then 
	\begin{align*}
	f,-f 
	&\leq |f| \\
	& \leq \|f\|_{\infty} \phi
	\end{align*}
	The previous exercise implies that $I(f), -I(f) \leq \|f\|_{\infty} I(\phi)$. So 
	\begin{align*}
	|I(f)| 
	&\leq \|f\|_{\infty} I(\phi) \\
	&\leq  C_K \|f\|_{\infty}  
	\end{align*}
	\end{proof}
	
	\begin{note}
	Let $X$ be a LCH space, $U \subset X$ open and $f \in C_c(X)$. We write $f \prec U$ to mean $0 \leq f \leq 1$ and $\supp(f) \subset U$. 
	\end{note}
	
	\begin{ex} \lex{00000} 
	Let $X$ be a LCH space, $I: C_c(C) \rightarrow \C$ a positive linear functional and $\mu:\MB(X) \rightarrow \RG$ a Radon measure. Suppose that for each $f \in C_c(X)$, $$I(f) = \int f \dmu$$
	Then 
	\begin{enumerate}
	\item for each $U \subset X$, $U$ is open implies that $$\mu(U) = \sup \{I(f): f \in C_c(X) \text{ and } f \prec U \}$$ 
	\item $\mu$ is the unique Radon measure such that for each $f \in C_c(X)$, $$I(f) = \int f d \, \mu$$
\end{enumerate}	 
	\end{ex}
	
	\begin{proof}\
	\begin{enumerate}
	\item Let $U \subset X$. Suppose that $U$ is open. For $f \in C_c(X)$, if $f \prec U$, then 
	\begin{align*}
	I(f) 
	&= \int f \dmu \\
	& \leq \mu(U) 
	\end{align*}
	Let $K \subset U$. Suppose that $K$ is compact. Then Urysohn's lemma implies that there exists $f \in C_c(X)$ such that $f \prec U$ and $f|_K = 1$. Then 
	\begin{align*}
	\mu(K) 
	&\leq \int f \dmu \\
	&= I(f)
	\end{align*}
	Inner regularity implies that 
	\begin{align*}
	\mu(U) 
	&= \sup \{\mu(K): K \subset X \text{ and $K$ is compact} \\
	& \leq \sup \{I(f): f \in C_c(X) \text{ and } f \prec U \} \\
	&\leq \mu(U)
	\end{align*}
	\item Let $\nu: \MB(X) \rightarrow \RG$ be a Radon measure. Suppose that for each $f \in C_c(X)$, $$I(f) = \int f d\nu$$
	Part $(1)$ implies that for each $U \subset X$, if $U$ is open, then 
	\begin{align*}
	\nu(U) 
	&= \sup \{I(f): f \in C_c(X) \text{ and } f \prec U \} \\
	&= \mu(U)
\end{align*}		
	Outer regularity implies that for each $E \in \MB(X)$, 
	\begin{align*}
	\nu(E) 
	&= \inf \{\nu(U): E \subset U \text{ and $U$ is open}\} \\
	&= \inf \{\mu(U): E \subset U \text{ and $U$ is open}\} \\
	&= \mu(E)
	\end{align*}
	So $\nu = \mu$ and $\mu$ is unique.
\end{enumerate}		  
	\end{proof}
	
	\begin{thm}\textbf{Representation Theorem 1:}\\
	Let $X$ be a LCH space and $I: C_c(C) \rightarrow \C$ a positive linear functional. Then there exists a unique Radon measure $\mu:\MB(X) \rightarrow \RG$ such that for each $f \in C_c(X)$, $$I(f) = \int f \dmu$$ 
	In addition, 
	\begin{enumerate}
	\item for each $U \subset X$, $U$ is open implies that $$\mu(U) = \sup \{I(f): f \in C_c(X) \text{ and } f \prec U \}$$
	\item for each $K \subset X$, $K$ is compact implies that $$\mu(U) = \inf \{I(f): f \in C_c(X) \text{ and } f  \geq \chi_K \}$$
\end{enumerate}	 
	\end{thm}
	
	\begin{note}
	Let $X$ be a topological space. Recall from section $(4.3)$ that we define $$\MM(X) = \{\mu:\MB(X) \rightarrow \C: \mu \text{ is a complex measure}\}$$ and that $\mu \mapsto |\mu|(X)$ is a norm on $\MM(X)$. 
	\end{note}		
	
	\begin{defn} \ld{00000} 
	Let $X$ be a topological space. For $\mu \in \MM(X)$, define $I_{\mu} :C_0(X) \rightarrow \C$ by $$I_{\mu} (f) = \int f \dmu$$.
	\end{defn}
	
	\begin{ex} \lex{00000} 
	Let $X$ be a topological space. For each $\mu \in \MM(X)$, $I_\mu \in C_0(X)^*$.
	\end{ex}
	
	\begin{proof}
	Let $\mu \in \MM(X)$ and $f \in C_0(X)$. An exercise in section $(4.3)$ implies that 
	\begin{align*}
	|I_{\mu}(f)| 
	&= \bigg| \int f \dmu \bigg| \\
	& \leq \int |f| d |\mu| \\
	& \leq \| \mu \| \|f\|_{\infty}
	\end{align*}
	So $I_{\mu}$ is bounded and $I_{\mu} \in C_0(X)^*$.
	\end{proof}
	
	\begin{thm}
	Let $I \in C_0(X, \R)^*$, then there exist positive linear functionals $I^+, I^- \in C_0(X, \R)^*$ such that $I = I^+ - I^-$
	\end{thm}
	
	\begin{ex} \lex{00000} 
	Let $X$ be a LCH space. Then the map $ \phi: \MM(X) \rightarrow C_0(X)^*$ given by $\phi(\mu) = I_{\mu}$ is a linear surjection.
	\end{ex}
	
	\begin{proof}
	An exercise in section $(4.3)$ implies that $\phi$ is linear. Let $I \in C_0(X)^*$. Then there exists positive linear functionals $I^{\pm}$, $J^{\pm} \in C_0(X)^*$ such that $I = I^+ - I^- + i(J^+ - J^-)$. The first representation theorem implies that there exist Radon measures $\mu^{\pm}$, $\nu^{\pm}$ such that $I^{\pm} = I_{\mu^{\pm}}$ and $J^{\pm} = I_{\mu^{\pm}}$. Set $\mu = \mu^+ - \mu^- +i(\nu^+ - \nu ^-)$. Then $I = \phi(\mu)$
	\end{proof}
	
	\begin{thm}\textbf{Representation Theorem 2:}\\
	Let $X$ be a LCH space. Then the map $\phi: \MM(X) \rightarrow C_0(X)^*$ given by $\phi(\mu) = I_{\mu}$ is an isometric linear isomorphism. 
	\end{thm}
	
	
	
	
	\begin{defn} \ld{00000} 
	Let $X$ be a LCH space, $(\mu_n)_{n \in \N} \subset \MM(X)$ and $\mu \in \MM(X)$. Then $\mu_n$ is said to  \textbf{converge to $\mu$ in weak-*}, denoted $\mu_n \conv{w^*}\mu$, if $I_{\mu_n} \conv{w^*} I_{\mu}$, i.e. for each $f \in C_0(X)$, $$\int f \dmu_n \rightarrow \int f \dmu$$
	\end{defn}
	
	\begin{ex} \lex{00000} 
	
	\end{ex}
	
	
	
	
	
	
	
	
	
	

	
	
	
	
	
	
	
	
	
	
	
	
	\newpage
	\subsection{Borel Measures on Metric Spaces}
	\begin{note}
		Let $X$ be a metric space and $A \subset X$. For $\ep >0$, we write $A_{\ep} = \{x \in X: d(x, A) < \ep\}$ and recall that $A_{\ep}$ is open.
	\end{note}

	\begin{ex}
		Let $X$ be a metric space, $\mu: \MB(X) \rightarrow \Rg$ be a finite measure and $E \in \MB(X)$. Then 
		$\mu(E) = \inf  \{\mu(U): E \subset U \text{ and $U$ is open} \}$ iff $\mu(E^c) = \sup  \{\mu(C): C \subset E^c \text{ and $C$ is closed} \}$ 
	\end{ex}

	\begin{proof}
		Suppose that $\mu(E) = \inf  \{\mu(U): E \subset U \text{ and $U$ is open} \}$. Let $\ep >0$. Then there exists $U \subset X$ such that $E \subset U$, $U$ is open and $\mu(U) < \mu(E) + \ep$. Choose $C = U^c$. Then $C \subset E^c$, $E^c$ is closed and 
		\begin{align*}
			\mu(E^c) - \ep 
			&= \mu(E^c \cap C) + \mu(E^c \cap C^c) - \ep \\
			&= \mu(C) + \mu(E^c \cap U) - \ep \\
			&= \mu(C) + [\mu(U) - \mu(E)] - \ep \\
			&< \mu(C) + \ep - \ep \\
			&= \mu(C)
		\end{align*}
		So for each $\ep > 0$, there exists $C \subset E^c$ such that $C$ is closed and $\mu(C) < \mu(E^c) - \ep$. is arbitrary, $\mu(E^c) = \sup\{\mu(C): C \subset E^c \text{ and $E$ is closed}\}$. \\
		The converse is similar. 
	\end{proof}

	\begin{ex}
		Let $X$ be a metric space and $\mu: \MB(X) \rightarrow \Rg$ be a finite measure. Then for each $C \subset X$, if $C$ is closed, then $\mu$ is outer regular on $C$. \\
		\textbf{Hint:} For $\ep >0$, consider $C_{\ep} = \{x \in X: d(x, C) < \ep\}$. 
	\end{ex}

	\begin{proof}
		Let $n \in \N$. Set $V_n = C_{1/n}$. Then $V_n$ is open and $C \subset V_n$. Since $C$ is closed, $C = \bigcap_{n \in \N} V_n$. Since for each $n \in \N$, $V_{n+1} \subset V_n$ and $\mu$ is finite, we have that $\mu(C) = \inf\limits_{n \in \N} \mu(V_n)$. So for each $\ep >0$, there exists $n \in \N$ such that $\mu(V_n) < \mu(C) + \ep$. Hence $\mu(C) = \inf \{\mu(U): C \subset U \text{ and $U$ is open} \}$ and $\mu$ is outer regular on $C$.    
	\end{proof}
	
	\begin{ex}
		Let $X$ be a metric space and $\mu: \MB(X) \rightarrow \Rg$ be a finite measure. Set 
		$$\MA = \bigg \{E \in \MB(X): \text{$\mu$ is outer regular on $E$ and $E^c$} \bigg\}$$  
		Then $\MA$ is a $\sig$-algebra on $X$.
	\end{ex}	

	\begin{proof}\ 
		\begin{enumerate}
			\item Clearly, $\varnothing \in \MA$.
			\item Let $E \in \MA$. Since $(E^c)^c = E$, by definition, $E^c \in \MA$. 
			\item Let $(E_n)_{n \in \N} \subset \MA$. Set $E = \bigcup\limits_{n \in \N} E_n$. Let $\ep >0$. 
			\begin{itemize}
				\item For each $n \in \N$, there exists $U_n \subset X$ such that $U_n$ is open, $E_n \subset U_n$ and $\mu(U_n) < \mu(E_n) + \ep 2^{-n - 1 }$. Set $U = \bigcup\limits_{n\in\N} U_n$. Then $U$ is open, $E \subset U$ and 
				\begin{align*} 
					U \setminus E
					&= \bigg( \bigcup_{n \in \N} U_n \bigg) \cap  E^c \\
					&= \bigg( \bigcup_{n \in \N} U_n \cap E^c \bigg)  \\
					&= \bigg( \bigcup_{n \in \N} U_n \cap \bigg[ \bigcap_{j \in \N} E_j^c \bigg] \bigg) \\
					&= \bigg( \bigcup_{n \in \N} \bigg [ \bigcap_{j \in \N} (U_n \cap E_j^c)\bigg ]  \bigg) \\
					& \subset \bigcup_{n \in \N} (U_n \cap E_n^c)  \\
					&= \bigcup_{n \in \N} (U_n \setminus E_n) 
				\end{align*}
				Therefore
				\begin{align*}
					\mu(U) - \mu(E)
					&= \mu(U \setminus E) \\
					& \leq \mu \bigg(\bigcup_{n \in \N} [U_n \setminus E_n] \bigg) \\
					& \leq \sum_{n \in \N} \mu(U_n \setminus E_n) \\
					&= \sum_{n \in \N} [\mu(U_n) - \mu(E_n)] \\
					& \leq \sum_{n \in \N} \ep 2^{-n-1} \\
					&= \frac{\ep}{2} \\
					& < \ep
				\end{align*}
				So for each $\ep >0$, there exists $U \subset X$ such that $U$ is open, $\bigcup_{n \in \N}E_n \subset U$ and $\mu(U) < \mu \bigg( \bigcup_{n \in \N} E_n \bigg) + \ep$. Therefore $$\mu \bigg( \bigcup_{n \in \N} E_n \bigg) = \inf \bigg \{ \mu(U): \bigcup_{n \in \N} E_n \subset U \text{ and $U$ is open} \bigg \}$$
				and $\mu$ is outer regular on $\bigcup\limits_{n \in \N} E_n$.
				\item A previous exercise implies that for each $n \in \N$, there exists $C_n \subset E_n$ such that $C_n$ is closed and $\mu(C_n) > \mu(E_n) - 2^{-n-1}\ep$. Since 
				$$\mu\bigg( \bigcup\limits_{n \in \N}C_n \bigg) = \sup\limits_{K \in \N} \mu\bigg(\bigcup\limits_{n=1}^K C_n \bigg)$$ 
				there exists $K \in \N$ such that $\mu\bigg(\bigcup\limits_{n=1}^K C_n \bigg) > \mu\bigg( \bigcup\limits_{n \in \N}C_n \bigg) - \ep/2 $. Set $C = \bigcup\limits_{n=1}^K C_n$.
				Then $C$ is closed, $C \subset E$ and similar to the previous part, we have that
				\begin{align*}
					\mu(E) - \mu(C)
					& < \mu(E) - \mu\bigg( \bigcup\limits_{n \in \N}C_n \bigg) + \frac{\ep}{2} \\
					&= \mu \bigg(E \setminus \bigcup\limits_{n \in \N}C_n   \bigg) +  \frac{\ep}{2} \\
					&= \mu \bigg( \bigcup\limits_{n \in \N} \bigg[ \bigcap_{j \in \N} (E_n \cap C_j^c)    \bigg] \bigg)  + \frac{\ep}{2} \\
					&\leq  \mu \bigg( \bigcup\limits_{n \in \N}  (E_n \cap C_n^c)  \bigg)  + \frac{\ep}{2} \\
					&\leq \bigg[ \sum_{n \in \N} \mu(E_n \cap C_n^c) \bigg] + \frac{\ep}{2}\\
					&=\bigg[ \sum_{n \in \N} \mu(E_n) - \mu(C_n) \bigg] +\frac{\ep}{2}\\
					&\leq \bigg[ \sum_{n \in \N} 2^{-n-1}\ep \bigg] + \frac{\ep}{2}\\
					&= \frac{\ep}{2} + \frac{\ep}{2} \\
					&= \ep 
				\end{align*}
				So for each $\ep >0$, there exists $C \subset X$ such that $C$ is closed, $C \subset \bigcup_{n \in \N}E_n$ and $\mu(C) > \mu \bigg( \bigcup_{n \in \N} E_n \bigg) - \ep$. Therefore 
				$$\mu \bigg( \bigcup_{n \in \N} E_n \bigg) = \sup \bigg \{ \mu(C): C \subset \bigcup_{n \in \N} E_n  \text{ and $C$ is closed} \bigg \}$$
				which implies that 
				$$\mu \bigg( \bigg[ \bigcup_{n \in \N} E_n \bigg]^c \bigg) = \inf \bigg \{ \mu(U): \bigg[ \bigcup_{n \in \N} E_n \bigg]^c \subset U  \text{ and $U$ is open} \bigg \}$$
				and $\mu$ is outer regular on $\bigg( \bigcup\limits_{n \in \N} E_n \bigg)^c$.
			\end{itemize}
		Hence $\bigcup\limits_{n \in \N} E_n \in \MA$.
		\end{enumerate}
	Therefore $\MA$ is a $\sig$-algebra on $X$.
	\end{proof}

	\begin{ex}
		Let $X$ be a metric space and $\mu: \MB(X) \rightarrow \Rg$ be a finite measure. Then $\mu$ is outer regular.
	\end{ex}

	\begin{proof}
		Set $\MT = \{U \subset X: X \text{ is open}\}$ and define $\MA$ as in the previous exercise. The previous exercises imply that $\MT \subset \MA$. Since $\MB(X) = \sig(\MT)$, we have that $\MB(X) \subset \MA$. Therefore $\MB(X) = \MA$ and $\mu$ is outer regular.  
	\end{proof}

	\begin{ex}
		Let $X$ be a metric space and $\mu: \MB(X) \rightarrow [0, \infty)$ a finite measure. If $\mu$ is inner regular on $X$, then $\mu$ is inner regular.
	\end{ex}

	\begin{proof}
		Suppose that is inner regular on $X$. Let $E \in \MB(X)$ and $\ep >0$. Then there exists $K_0 \subset X$ such that $K_0$ is compact and $\mu(K_0) > \mu(X) - \ep/2$. The previous exercise implies that there exists $C \subset E$ such that $C$ is closed and $\mu(C) > \mu(E) - \ep/2$. Set $K = K_0 \cap C$. Then $K \subset E$, $K$ is compact and 
		\begin{align*}
			\mu(E)
			& < \mu(C) + \frac{\ep}{2} \\
			& =[ \mu(C \cap K_0) + \mu(C \cap K_0^c)] + \frac{\ep}{2} \\
			& \leq \mu(C \cap K_0) + \mu(X \cap K_0^c) + \frac{\ep}{2} \\
			&= \mu(K) + [\mu(X) - \mu(K_0)] + \frac{\ep}{2} \\
			&< \mu(K) + \frac{\ep}{2} + \frac{\ep}{2} \\
			&= \mu(K) + \ep
		\end{align*}
		So for each $\ep >0$, there exists $K \subset E$ such that $K$ is compact and $\mu(K) > \mu(E) - \ep$. Hence $\mu(E) = \sup\{\mu(K): K \subset E \text{ and  $K$ is compact}\}$ and $\mu$ is inner regular on $E$. Since $E \in \MB(X)$ is arbitrary, $\mu$ is inner regular.
	\end{proof}

	\begin{ex}
		Let $X$ be a Polish space and $\mu: \MB(X) \rightarrow [0, \infty)$ a finite measure. Then $\mu$ is inner regular. \\
		\textbf{Hint:} If $(x_{n})_{n \in \N}$ is a countable dense of $X$, consider $K \subset X$ of the form 
		$$K = \bigcap\limits_{m \in \N} \bigcup\limits_{n = 1}^{n_m} \cl B(x_n, 1/m) $$
	\end{ex}

	\begin{proof}
		Let $\ep >0$. Since $X$ is separable, there exists a a countable dense subset $(x_{n})_{n \in \N}$ of $X$. Let $m \in \N$. Then $X = \bigcup\limits_{n \in \N} \cl B(x_n, 1/m)$. This implies that there exists $n_m \in \N$ such that 
		$$\mu \bigg(\bigcup\limits_{n = 1}^{n_m} \cl B(x_n, 1/m) \bigg) > \mu(X) - 2^{-m-1}\ep$$ 
		Set 
		$$K = \bigcap\limits_{m \in \N} \bigcup\limits_{n = 1}^{n_m} \cl B(x_n, 1/m) $$
		Then $K$ is closed. Let $\del >0$. Choose $m_{\del} \in \N$ such that $1/m_{\del} < \del$. Then 
		\begin{align*}
			K 
			&=  \bigcap\limits_{m \in \N} \bigcup\limits_{n = 1}^{n_m} \cl B(x_n, 1/m) \\
			& \subset \bigcup\limits_{n = 1}^{n_{m_{\del}}} \cl B(x_n, 1/{m_{\del}}) \\
			& \subset  \bigcup\limits_{n = 1}^{n_{m_{\del}}} B(x_n, \del) \\ 
		\end{align*}
		Hence $K$ is totally bounded. Since $X$ is complete, $K$ is compact. Finally, we have that 
		\begin{align*}
			\mu(X) - \mu(K)
			&= \mu(K^c) \\
			&= \mu \bigg( \bigcup\limits_{m \in \N} \bigg[ \bigcup\limits_{n = 1}^{n_m} \cl B(x_n, 1/m) \bigg]^c  \bigg) \\
			&\leq \sum_{m \in \N} \mu \bigg( \bigg[ \bigcup\limits_{n = 1}^{n_m} \cl B(x_n, 1/m) \bigg]^c \bigg) \\
			&= \sum_{m \in \N} \bigg[ \mu(X) - \mu \bigg(  \bigcup\limits_{n = 1}^{n_m} \cl B(x_n, 1/m)  \bigg) \bigg] \\
			& \leq \sum_{m \in \N} 2^{-m-1}\ep \\
			&= \frac{\ep}{2} \\
			&< \ep
		\end{align*} 
	So for each $\ep >0$, there exists $K \subset X$ such that $K$ is compact and $\mu(K) > \mu(X) - \ep$. Thus $$\mu(X) = \sup\{\mu(K): \text{$K \subset X$ and $K$ is compact}\}$$ and $\mu$ is inner regular on $X$. The previous exercise implies that $\mu$ is inner regular.   
	\end{proof}
	
	\begin{ex}
		Let $X$ be a Polish space and $\mu: \MB(X) \rightarrow [0, \infty)$ a finite measure. Then $\mu$ is regular and Radon.
	\end{ex}

	\begin{proof}
		Clear by preceeding exercises.
	\end{proof}
	
	
	\begin{defn}
		Let $X$ be a topological space. For $f \in C_b(X)$, define $\lam_f: \MM(X) \rightarrow \C$ by $$\lam_f(\mu) = \int f \dmu$$.
	\end{defn}
	
	\begin{ex}
		Let $X$ be a topological space. For each $f \in C_b(X)$, $\lam_f \in \MM(X)^*$. 
	\end{ex}
	
	\begin{proof}
		Let $f \in C_b(X)$ and $\mu \in \MM(X)$. Then 
		\begin{align*}
			|\lam_f(\mu)| 
			&= \bigg | \int f \dmu \bigg| \\
			&\leq \int |f| \, d |\mu| \\
			& \leq \|f\|_u \|\mu\|
		\end{align*}
		
		\rex{43010} implies that $\lam_f$ is linear. So $\lam_f \in \MM(X)^*$.
	\end{proof}
	
	\begin{defn}
		Let $X$ be a topological space. We define the \textbf{weak topology on $\MM(X)$} to be the weak topology generated by $\{\lam_f \in \MM(X)^*: f \in C_b(X) \}$. 
	\end{defn}
	
	\begin{defn}
		Let $X$ be a topological space and $(\mu_{n})_{n \in \N} \subset \MM(X)$ and $\mu \in \MM(X)$. Then $(\mu_{n})_{n \in \N}$ is said to \textbf{converge  weakly} to $\mu$, denoted $\mu_n \conv{w} \mu$, if $(\mu_{n})_{n \in \N}$ converges to $\mu$ in the weak topology, i.e. for each $f \in C_b(X)$, $$\int f \dmu_n \rightarrow \int f \dmu$$ 
	\end{defn}	
	
	\begin{ex}\textbf{Portmanteau Theorem:}
		Let $X$ be a topological space and $(\mu_{n})_{n \in \N} \subset \MM(X)$ and $\mu \in \MM(X)$. Suppose that for each $n \in \N$, $\mu_n(X) = \mu(X)$. Then the following are equivalent:
		\begin{enumerate}
			\item $\mu_n \conv{w} \mu$
			\item for each $A \in \MB(X)$, $A$ is open implies that $\mu(A) \leq \limfn\limits \mu_n(A) $
			\item for each $A \in \MB(X)$, $A$ is closed implies that $\mu(A) \geq \limpn \limits \mu_n(A) $
			\item for each $A \in \MB(X)$, $\mu( \p A) = 0$ implies that $\mu_n(A) \rightarrow \mu(A) $
		\end{enumerate}
	\end{ex}
	
	\begin{proof}\
		\begin{itemize}
			\item $(2) \iff (3)$: \\
			Suppose $(2)$. Let $A \in \MB(X)$. Suppose that $A$ is closed. Then $A^c$ is open. By assumption, $\mu(A^c) \leq \limfn \limits \mu_n(A^c)$. Hence 
			\begin{align*}
				\mu(A)
				&= \mu(X) - \mu(A^c) \\
				&\geq \mu(X) - \limfn \mu_n(A^c) \\
				&= \mu(X) + \limpn [- \mu_n(A^c)] \\
				&= \limpn \bigg[ \mu(X)  - \mu_n(A^c) \bigg] \\
				&= \limpn \bigg[ \mu_n(X)  - \mu_n(A^c) \bigg] \\
				&= \limpn \mu_n(A)
			\end{align*}
			So $(3)$ holds.
			Similarly, $(3)$ implies $(2)$.\\
			\item $(2) \iff (4)$: \\
			Suppose $(2)$. From above, $(3)$ holds. Let $A \in \MB(X)$. Then 
			\begin{align*}
				\mu(A^{\circ}) 
				& \leq \limfn \mu_n(A^{\circ}) \\
				& \leq  \limfn \mu_n(A) \\
				& \leq  \limpn \mu_n(A) \\
				& \leq  \limpn \mu_n(\overline{A}) \\
				& \leq  \mu(\overline{A}) \\
			\end{align*}
			Suppose that $\mu(\p A) = 0$. Then
			\begin{align*}
				\mu(A^{\circ}) 
				&\leq \mu(A) \\
				& \leq \mu(\overline{A}) \\
				&= \mu(A^{\circ}) + \mu(\p A) \\
				&= \mu(A^{\circ}) \\
			\end{align*}
			which implies that $\mu_n(A) \rightarrow \mu(A)$. Conversely, suppose $(4)$. 
			\item 
		\end{itemize}
	\end{proof}






















	\newpage
	\section{Haar Measure}
	
	\subsection{Introduction}
	
	\begin{note}
	This section assumes familiarity with topological groups. See section $8.1$ of \cite{analysis} for details. 
	\end{note}

	\begin{defn} \ld{00000} 
		Let $G$ be a group and $g \in G$. Define $l_g:G \rightarrow G$ and $r_g:G \rightarrow G$ by $l_g(x) = gx$ and $r_g(x) = xg^{-1}$. 
	\end{defn}

	\begin{defn} \ld{00000} 
		Let $G$ be a topological group, $y \in G$ and $f \in L^0$.  Define $L_y, R_y: L^0(G) \rightarrow L^0(G)$ by $L_y f = f \circ l_y^{-1}$ and $R_y f = f \circ r_y^{-1}$, that is, $L_yf(x) = f(y^{-1}x)$ and $R_yf(x) = f(xy)$.
	\end{defn}
	
	\begin{defn} \ld{00000} 
		Let $G$ be a topological group and $\mu$ a Radon measure on $G$. Then $\mu$ is said to be a \textbf{left Haar measure on $G$} if 
		\begin{enumerate}
			\item $\mu$ is nonzero  
			\item for each $U \in \MB(G)$ and $g \in G$, $\mu(gU) = \mu(U)$.  
		\end{enumerate}
		Similarly, $\mu$ is said to be a \textbf{right Haar measure on $G$} if 
		\begin{enumerate}
			\item $\mu$ is nonzero  
			\item for each $U \in \MB(G)$ and $g \in G$, $\mu(Ug) = \mu(U)$.  
		\end{enumerate}
	\end{defn}
	
	\begin{ex} \lex{00000} 
		Let $G$ be a topological group, $\mu$ a Radon measure on $G$. Then $\mu$ is a left Haar measure on $G$ iff $\iota_*\mu$ is a right Haar measure on $G$. 
	\end{ex}
	
	\begin{proof}
		Suppose that $\mu$ is a left Haar measure on $G$. Let $U \in \MB(G)$ and $g \in G$. Then 
		\begin{align*}
			\iota_*\mu(Ug)
			& = \mu(\iota^{-1}(Ug)) \\
			&= \mu (g^{-1}U^{-1}) \\
			&= \mu (U^{-1}) \\
			&= \mu(\iota^{-1}(U)) \\
			&= \iota_*\mu(U)
		\end{align*}
		So $\iota_*\mu$ is a right Haar measure on $G$. The converse is similar.
	\end{proof}

	\begin{ex} \lex{00000} 
		Let $G$ be a topological group, and $\mu$ a left Haar measure on $G$. Then for each $g \in G$, ${r_{g}}_*\mu$ is a left Haar measure on $G$.
	\end{ex}

	\begin{proof}
		Let $g \in G$ and $U \in \MB(G)$. Observe that ${r_{g}}_*\mu(U) = \mu(Ug)$. So for each $h \in G$, 
		\begin{align*}
			{r_{g}}_*\mu(hU) 
			& = \mu(hUg) \\
			& =  \mu(Ug) \\
			& = {r_{g}}_*\mu(U)
		\end{align*}
	\end{proof}
	
	\begin{ex} \lex{00000} 
		Let $G$ be a topological group, $\mu$ a left Haar measure on $G$ and $\nu$ a right Haar measure on $G$. Then for each $f \in L^1 \cup L^+$ and $y \in G$, 
		\begin{align}
			\int L_y f \dmu = \int f \dmu \\
			\int R_y f d\nu = \int f d\nu
		\end{align}
	\end{ex}
	
	\begin{proof}\
		\begin{enumerate}
			\item Let $y \in G$ and $E \in \MB(G)$. Put $f = \chi_E$. Then 
			\begin{align*}
				\int L_y f \dmu 
				& = \int L_y\chi_E \dmu \\
				& =  \int \chi_{yE} \dmu \\
				& = \mu(yE) \\
				& = \mu(E) \\
				& = \int \chi_E \dmu \\
				& = \int f \dmu
			\end{align*} 
			By linearity of $L_y$, for $f \in S^+$ we have that, $$\int L_y f \dmu = \int f \dmu$$ For $f \in L^+$, choose $\seq{\phi}{n} \subset S^+$ such that for each $n \in \N$ $\phi_n \leq \phi_{n+1} \leq f$ and $\phi_n \rightarrow f$. Then for each $n \in \N$ $L_y \phi_n \leq L_y \phi_{n+1} \leq L_y f$ and $L_y \phi \rightarrow L_y f$. So MCT implies that 
			\begin{align*}
				\int L_y f \dmu 
				& = \limn \int L_y \phi_n \dmu \\
				& = \limn \int \phi_n \dmu \\
				& = \int f \dmu \\
			\end{align*}
			Let $f \in L^1$. If $f$ is real valued, write $f = f^+ - f^-$. Then $L_y f = L_y f^+ - L_y f^-$ and 
			\begin{align*}
				\int L_yf \dmu 
				& = \int L_y f^+ \dmu - \int L_y f^- \dmu \\
				& = \int f^+ \dmu - \int f^- \dmu \\
				& = \int f \dmu
			\end{align*}
			If $f$ is complex valued, write $f = g + ih$ with $g, h \in L^1$ real valued. Then 
			\begin{align*}
				\int L_yf \dmu 
				& = \int L_y g \dmu + i \int L_y h \dmu \\
				& = \int g \dmu +i \int h \dmu \\
				& = \int f \dmu
			\end{align*}
			\item Similar
		\end{enumerate}
	\end{proof}
	
	\begin{ex} \lex{00000} 
		Let $G$ be a topological group and $\mu$ a left Haar measure on $G$. Then for each $U \subset G$, if $U$ is open and $U \neq \varnothing$, then $\mu(U) > 0$
	\end{ex}

	\begin{proof}
		Let $U \subset G$. Suppose that $U$ is open and $U \neq \varnothing$. Suppose that  $\mu(U) = 0$. Since $\mu$ is nonzero, inner regularity implies that there exists $K \subset G$ such that $K$ is compact and $\mu(K) > 0$. Then $ \{xU: x \in K\}$ is an open cover of $K$. Then there exist $x_1, \cdots, x_n \in K$ such that $K \subset \bigcap\limits_{k=1}^n x_kU$. Then 
		\begin{align}
			\mu(K) 
			& \leq \sum_{k =1}^n \mu(x_kU) \\
			& = \sum_{k =1}^n \mu(U) \\
			& = 0
		\end{align} 
		This is a contradiction. So $\mu(U) > 0$.
	\end{proof}

	\begin{ex} \lex{00000} 
		Let $G$ be a locally compact group and $\mu$ a left Haar measure on $G$. Then there exists $S \in \MB(G)$ such that $S$ is symmetric, $e \in S$ and $\mu(E) > 0$ 
	\end{ex}

	\begin{proof}
		Since $G$ is locally compact, there exists a compact neighborhood $K$ of $e$. Then $\mu(K) > 0$. Put $S = KK^{-1} \in \MB(G)$. Then $S$ is symmetric. Since $e \in K$, $K \subset S$ and $0 < \mu(K) \leq \mu(S)$.
	\end{proof}
	
	\begin{ex} \lex{00000} 
		Let $G$ be a locally compact group and $\mu$ a left Haar measure on $G$. Then 
		\begin{enumerate}
			\item  $\mu(\{e\}) > 0$ iff there exists $\lam >0$ such that $\mu = \lam \#$.
			\item $\mu$ is finite iff $G$ is compact
		\end{enumerate}
	\end{ex}

	\begin{proof}\
		\begin{enumerate}
			\item If there exists $\lam >0$ such that $\mu = \lam \#$, then $\mu(\{e\}) > 0$ Conversely, suppose that $\mu(\{e\}) > 0$. Define $\lam = \mu(\{e\}) > 0$. Let $B \in \MB(G)$. If $B$ is finite, then 
			\begin{align*}
				\mu(B) 
				& = \sum\limits_{x \in B}  \mu(\{x\}) \\
				& = \sum\limits_{x \in B} \mu(x \{e\}) \\
				& = \sum\limits_{x \in B} \mu( \{e\}) \\
				& = \sum\limits_{x \in B} \lam \\
				& = \lam \#(\{e\})
			\end{align*}
			If $B$ is infinite, then we may choose a countable subset and the same reasoning as above tells us that $$\mu(B) = \infty = \lam \#(B)$$
			\item If $G$ is compact, then $\mu$ is finite since $\mu$ is Radon. Conversely, suppose that $\mu$ is finite. Then \textbf{FINISH}
		\end{enumerate}
	\end{proof}
	
	\begin{thm}
		Let $G$ be a locally compact group. Then there exists a left Haar measure on $G$. 
	\end{thm}
	
	\begin{thm}
		Let $G$ be a locally compact group and $\mu_1, \mu_2$ left Haar measures on $G$. Then there exists $\lam > 0$ such that $\mu_1 = \lam \mu_2 $.
	\end{thm}
	
	\begin{defn} \ld{00000} 
		Let $G$ be a locally compact group and $\mu$ a left Haar measure on $G$. A previous exercise tells us that for each $g \in G$, ${r_g}_*\mu$ is a left Haar measure on $G$. The previous result tells us that for each $g \in G$ there exists $\lam_g >0$ such that ${r_g}_*\mu = \lam_g \mu$. Define $\Del: G \rightarrow (0, \infty)$ by $\Del(g) = \lam_g$. We call $\Del$ the \textbf{modular function of $G$}. 
	\end{defn}

	\begin{ex} \lex{00000} 
		Let $G$ be a locally compact group and $\mu$ a left Haar measure on $G$. Then 
		\begin{enumerate}
			\item $\Del $ is a homomorphism 
			\item for each $f \in L^1 \cup L^+$, $$\int R_{y^{-1}} f \dmu = \Del(y) \int f \dmu$$
		\end{enumerate}
	\end{ex}

	\begin{proof}\
		\begin{enumerate}
			\item Recall that for each $g \in G$, $\Del(g)\mu(U) = {r_g}_*\mu(U) = \mu(Ug)$. Let $g, h \in G$ and $U \in \MB(G)$. Then $\Del(gh)\mu(U) = \mu(Ugh) = \Del(h)\mu(Ug) = \Del(g) \Del(h)\mu(U)$. So $\Del(gh) = \Del(g) \Del(h)$.
			\item Let $y \in G$ and $U \in \MB(G)$. Put $f = \chi_U$ Then 
			\begin{align*}
				\int R_{y^{-1}} f \dmu 
				& = \int R_{y^{-1}} \chi_U \dmu \\
				& = \int \chi_{Uy} \dmu \\
				& = \mu(Uy) \\
				&= \mu(r_y^{-1}(U)) \\
				&= {r_y}_* \mu(U) \\
				& = \Del(y) \mu(U) \\
				& = \Del(y)  \int \chi_U \dmu \\
				&= \Del(y)  \int f \dmu
			\end{align*}
			By linearity of $R_{y^{-1}}$, for $f \in S^+$, $$\int R_{y^{-1}} f \dmu = \Del(y) \int f \dmu$$
			For $f \in L^+$, choose $\seq{\phi}{n} \subset S^+$ such that for each $n \in \N$ $\phi_n \leq \phi_{n+1} \leq f$ and $\phi_n \rightarrow f$. Then for each $n \in \N$ $R_{y^{-1}} \phi_n \leq R_{y^{-1}} \phi_{n+1} \leq R_{y^{-1}} f$ and $R_{y^{-1}} \phi \rightarrow R_{y^{-1}} f$. So the monotone convergence theorem implies that 
			\begin{align*}
				\int R_{y^{-1}} f \dmu 
				& = \limn \int R_{y^{-1}} \phi_n \dmu \\
				& = \limn \Del(y) \int \phi_n \dmu \\
				& = \Del(y) \int f \dmu \\
			\end{align*}
			Let $f \in L^1$. If $f$ is real valued, write $f = f^+ - f^-$. Then $R_{y^{-1}} f = R_{y^{-1}} f^+ - R_{y^{-1}} f^-$ and 
			\begin{align*}
				\int R_{y^{-1}} f \dmu 
				& = \int R_{y^{-1}} f^+ \dmu - \int R_{y^{-1}} f^- \dmu \\
				& = \Del(y) \int f^+ \dmu - \Del(y) \int f^- \dmu \\
				& = \Del(y) \int f \dmu
			\end{align*}
			If $f$ is complex valued, write $f = g + ih$ with $g, h \in L^1$ real valued. Then 
			\begin{align*}
				\int R_{y^{-1}} f \dmu 
				& = \int R_{y^{-1}} g \dmu + i \int R_{y^{-1}} h \dmu \\
				& = \Del(y) \int g \dmu +i \Del(y) \int h \dmu \\
				& = \Del(y) \int f \dmu
			\end{align*}
		\end{enumerate}
	\end{proof}

	\begin{defn} \ld{00000} 
		Let $G$ be a locally compact group. Then $G$ is said to be \textbf{unimodular} if $\ker \Del = G$.  
	\end{defn}
	
	\begin{ex} \lex{00000} 
		Let $G$ be a locally compact group. Then the following are quivalent: 
		\begin{enumerate}
			\item $G$ is unimodular 
			\item there exists a left  Haar measure $\mu$ on $G$ such that $\mu$ is a right Haar measure on $G$.
			\item for each nonzero Radon measure $\mu$ on $G$, $\mu$ is a left Haar measure on $G$ iff $\mu$ is a right Haar measure on $G$.
		\end{enumerate}
	\end{ex}
	
	\begin{proof}\
		\begin{itemize}
			\item $(1) \implies (2)$:\\ Since $G$ is a locally compact group, there exists a left Haar measure $\mu$ on $G$. Let $g \in G$ and $U \in \MB(G)$. Then $$\mu(Ug) = \Del(g) \mu(U) = \mu(U)$$ Since $G$ is unimodular, $\Del(g) = 1$. Then $\mu$ is a right Haar measure on $G$. 
			\item $(2) \implies (3)$:\\ By assumption, there exists a left  Haar measure $\mu'$ on $G$ such that $\mu'$ is a right Haar measure on $G$. Let $\mu$ be a nonzero Radon measure on $G$. If $\mu$ is a left Haar measure on $G$, then there exists $\lam >0$ such that $\mu = \lam \mu'$ and therefore $\mu$ is a right Haar measure. The same reasoning implies that if $\mu$ is a right Haar measure on $G$, then $\mu$ is a left Haar measure on $G$.
			\item $(3) \implies (1)$: \\ Since $G$ is locally compact, there exists a left $Haar$ measure $\mu$ on $G$. By assumption, $\mu$ is a right Haar measure on $G$. By inner regularity there exists $K \in \MB(G)$ such that $\mu(K) > 0$. Let $g \in G$. Then $$\Del(g) \mu(K) = \mu(Kg) = \mu(K)$$ So $\Del(g) = 1$.
		\end{itemize}
	\end{proof}

	\begin{note}
		If $G$ is a locally compact abelian group, then $G$ is unimodular.
	\end{note}
	
	\begin{ex} \lex{00000} 
		Let $G$ be a locally compact group and $\mu$ a left Haar measure on $G$. If $G$ is unimodular then $\iota_*\mu = \mu$.
	\end{ex}

	\begin{proof}
		Suppose that $G$ is unimodular. A previous exercise tells us that $\iota_*\mu$ is a right Haar measure on $G$. The unimodularity of $G$ implies that $\iota_*\mu$ a left Haar measure on $G$. Then there exists $\lam >0$ such that $\iota_*\mu = \lam \mu$. Since $G$ is locally compact, there exists $S \in \MB(G)$ such that $S$ is symmetric and $\mu(S) > 0$. Then 
		\begin{align*}
			\mu(S) 
			& = \mu(S^{-1}) \\
			& = \iota_*\mu(S) \\
			& = \lam \mu(S) 
		\end{align*}	
		So $\lam = 1$ and $\iota_*\mu = \mu$.

		it is also (Since $G$ is locally compact, there exists $S \in \MB(G)$ such that $S$ is symmetric and $\mu(S) > 0$. Then 
		$$\mu(S) = \mu(S^{-1}) = \iota_*\mu(S)$$ Since $\iota_*\mu$ is a right Haar measure on $G$ and $G$ is unimodular, $\iota_*\mu(S)$ is also a left Haar measure on $G$. Then there exists $\lam > 0$ such that $\mu(S) = \lam\iota_*\mu(S)$.
	\end{proof}




	
	



	
	
	
	
	
	
	
	\newpage
	\subsection{Fundamental Examples}		
	
	\begin{note}
		The Haar measure on  $(\R^n, +)$ is $m$.
	\end{note}
	
	\begin{ex} \lex{00000} 
		The Haar measure on $(\R^{\times}, \cdot)$  is $$d\mu(x) = \frac{1}{|x|} \dm(x)$$
	\end{ex}

	\begin{proof}
		Let $0 < a < b$ and $c >0$. Then
		\begin{align*}
			\mu(c(a, b))
			& = \mu((ca,cb)) \\
			& = \int_{(ca,cb)} \frac{1}{|x|} \dm(x)\\
			& = \int_{(ca,cb)} \frac{1}{x} \dm(x)\\
			& = \bigg[ \log|x| \bigg]_{ca}^{cb} \\
			& = \log(cb) - \log(ca) \\
			& = \log b - \log a \\
			& = \bigg[ \log|x| \bigg]_{a}^{b} \\ 
			& =  \int_{(a,b)} \frac{1}{x} \dm(x)\\
			& = \mu((a,b))
		\end{align*}
	Similarly, we have
	\begin{align*}
		\mu(-c(a, b))
		& = \mu((-cb,-ca)) \\
		& = \int_{(-cb,-ca)} \frac{1}{|x|} \dm(x)\\
		& = - \int_{(-cb,-ca)} \frac{1}{x} \dm(x)\\
		& = - \bigg[ \log|x| \bigg]_{-cb}^{-ca} \\
		& = \log(cb) - \log(ca) \\
		& = \log b - \log a \\
		& = \bigg[ \log|x| \bigg]_{a}^{b} \\ 
		& =  \int_{(a,b)} \frac{1}{x} \dm(x)\\
		& = \mu((a,b))
	\end{align*}
	\end{proof}

	\begin{ex} \lex{00000} 
		Define $f: [ 0,1) \rightarrow \T$ by $f(x) = e^{i2 \pi x}$. Let $m$ be Lebesgue measure on $[0,1)$, then the Haar measure on $\T$ is $f_*m$.
	\end{ex}

	\begin{proof}
		Note that $f$ is a bijection and the topology on $\T$ is generated by sets of the form $f((a, b))$ where $a,b \in [0,1)$ and $a< b$. Let $a,b \in [ 0,1 )$ and suppose that $a<b$. Put $A = f((a, b))$. Let $z \in \T$. Then there exists $\theta \in [0, 1)$ such that $z = f(\theta)$. If $1 \not \in zA$, then $f^{-1}(zA) = (\theta + a, \theta + b)$. If $1 \in zA$, then $f^{-1}(zA) = (\theta + a , 1) \cup [0,  \theta + b - 1)$. Suppose that $1 \not \in zA$. Then
		\begin{align*}
			& = f_*m(zA) 
			& = m(f^{-1}(zA)) \\
			& = m((\theta + a, \theta + b)) \\
			& = b - a \\
			& = m((a,b)) \\
			& = m(f^{-1}(A)) \\
			& = f_*m(A)
		\end{align*}
	Similarly if $1 \in zA$, $f_*m(zA) = f_*m(A)$.
	\end{proof} 

	\begin{ex} \lex{00000} 
		Let $p$ be a prime. Define $|\cdot |_p: \Q \rightarrow [0, \infty)]$ by 
		\[
		\begin{cases}
			|\frac{a}{b}p^n|_p = p^{-n}, \text{ if } \gcd(a,p) = \gcd(b,p) = 1 \\
			|0|_p = 0
		\end{cases}
		\]
		Then $|\cdot|_p$ is an absolute value on $\Q$. Define $\Q_p$ to be the completion of $\Q$ with respect to the metric induced by $|\cdot|_p$. Define $\Z_p = \{\al \in \Q_p: |\al|_p \leq 1 \}$. It is well known that $\Q_p$ is a locally compact field and $\Z_p$ is compact. Define $P = \{0, 1, \cdots, p-1\}$. It is known that the topology is generated by  $$\{x + p^n\Z_p: \text{ for } n \in \Z, x \in \Q_p\}$$ Another useful fact is that $$\Q_p = \{\sum_{j = -n}^{\infty} a_jp^j : a_j \in P, n \in \N_0\}$$ and $$\Z_p = \{\sum_{j = 0}^{\infty} a_jp^j : a_j \in P\}$$ 
		Let $\mu$ be the Haar measure on $\Q_p$. Then $\mu$ is completely determined by the value $\mu(\Z_p)$  
	\end{ex}

	\begin{proof}
		We observe that for $n \in \Z$, we may write $p^n \Z_p$ as the following disjoint union: $$p^n\Z_p = \bigcup\limits_{j \in P} jp^n + p^{n+1}\Z^p$$ Thus $\mu(p^n \Z^{p}) = p \mu(p^{n+1}\Z_p)$. If we set $\mu(\Z_p) = 1$, we obtain that $\mu(\Z_p) = p^n \mu(p^n\Z_p)$, which implies that $$\mu(p^n \Z_p) = \frac{1}{p^n}\mu(\Z_p)$$.  
	\end{proof}
	
	\begin{ex} \lex{00000} 
		Let $\nu$ be the Haar measure on $\Q_p$. Then the Haar measure on $\Q_p^{\times}$  is $\dmu = \frac{1}{|x|_p}d \nu$.
	\end{ex}

	\begin{proof}
		Let $x, y \in P^{\times}$ and $\al = xp^{n-1} + p^n\Z_p$. Then 
		\begin{align*}
			\al (yp^{k-1} + p^k \Z_p) 
			& = p^{(n-1)+ (k-1)}(xy + p^{n+k} \Z_p) 
		\end{align*}
	\end{proof}
	
	
	
	
	
	
	
	
	
	
	
	
	
	
	
	\subsection{Action on Measures}
	\begin{ex}
	Let $G$ be a locally compact group, $\mu$ a left Haar measure on $G$ and $\nu \in \MM(G)$. If $\nu \ll \mu$, then ${l_g}_*\nu \ll \mu$.
	\end{ex}
	
	\begin{proof}
	Suppose that $\nu \ll \mu$. Let $A \in \MB(G)$. Then
	\begin{align*}
	\mu(A) = 0 
	&\implies \mu(g^{-1}A) = 0 \\
	&\implies \nu(g^{-1}A) = 0 \\
	&\implies \nu(l_{g^{-1}}(A)) = 0 \\
	&\implies \nu(l_{g}^{-1}(A)) = 0 \\
	&\implies {l_g}_*\nu(A) = 0 \\
	\end{align*}
	So ${l_g}_*\nu \ll \mu$.
	\end{proof}

	\begin{defn}
	Let $G$ be a locally compact group and $\mu$ a left Haar measure on $G$. Define $\MM_{\mu} \subset \MM(G)$ by $$\MM_{\mu} = \{\nu \in \MM(G):\nu \ll \mu\}$$
	We define an action $\phi:G \times \MM_{\mu} \rightarrow \MM_{\mu}$ by $$g \cdot \nu = {l_g}_*\nu$$ 
	\end{defn}	
	
	\begin{ex}
	Let $G$ be a locally compact group, $\mu$ a $\sig$-finite left Haar measure on $G$, $\nu \in \MM_{\mu}$ and $g \in G$. Then $$ \frac{d (g \cdot \nu)}{d\mu} =  L_g \frac{d \nu}{d\mu}$$
	\end{ex}
	
	\begin{proof}
	Set $f = d \nu/ d \mu$. Let $A \in \MB(X)$. Then 
	\begin{align*}
	\int_A L_g f \dmu
	&= \int_A f \circ l_{g}^{-1} \,\dmu \\
	&= \int_A f \circ l_g^{-1} \, \dmu \\
	&= \int_{l_g^{-1}(A)} f \, d ({l_g^{-1}}_* \mu) \\
	&= \int_{l_g^{-1}(A)} f \, d ({l_{g^{-1}}}_* \mu) \\
	&= \int_{l_g^{-1}(A)} f \, d \mu \\
	&= \nu(l_g^{-1}(A)) \\
	&= {l_g}_*\nu(A) \\
	&= g \cdot \nu(A)
	\end{align*}
	Since $A$ is arbitrary, uniqueness implies that 
	$$ \frac{d (g \cdot \nu)}{d\mu} =  L_g \frac{d \nu}{d\mu}$$
	\end{proof}
	
	\begin{ex}
	Let $G$ be a locally compact group, $\mu$ a $\sig$-finite left Haar measure on $G$, $\nu \in \MM_{\mu}$ and $g \in G$. Then $\|g \cdot \nu\| = \|\nu\|$. 
	\end{ex}
	
	\begin{proof}
	\rex{43004.5} implies that
	\begin{align*}
	\|g \cdot \nu\| 
	&= \int \bigg | \frac{d (g \cdot \nu)}{d\mu} \bigg | \dmu \\
	&=  \int \bigg | L_g \frac{d \nu}{d\mu} \bigg | \dmu \\
	&= \int L_g \bigg | \frac{d \nu}{d\mu} \bigg | \dmu \\
	&= \int \bigg | \frac{d \nu}{d\mu} \bigg | \dmu \\
	&= \|\nu\|
	\end{align*}
	\end{proof}

	
	
	
	
	
	
	
	
	
	
	
	
	
	
	
	
	
	
	
	
	\newpage
	\subsection{Measures Invariant under Group Actions}
	\begin{defn} \ld{00000} 
		Let $G$ be a group, $X$ a set, $\phi: G \times X \rightarrow X$ a group action and $g \in G$. Define $l_g:X \rightarrow G$ by $l_g(x) = g \cdot x$. 
	\end{defn}
	
	\begin{defn} \ld{00000} 
		Let $G$ be a topological group, $X$ a set, $\phi: G \times X \rightarrow X$ a group action and $g \in G$. Define $L_g: L^0(G) \rightarrow L^0(G)$ by 
		$$L_g f = f \circ l_g^{-1}$$ 
		i.e. $L_g f(x) = f(g^{-1} \cdot x)$
	\end{defn}

	\begin{defn}
		Let $G$ be a group, $(X, \MA, \mu)$ a measure space, $\phi: G \times X \rightarrow X$ a group action and $\zeta: G \rightarrow (0, \infty)$. Then $\mu$ is said to be \textbf{relatively $\phi$-invariant with multiplier $\zeta$} if for each $g \in G$ and $U \in \MA$ $\mu(g^{-1} \cdot U) = \zeta(g) \mu(U)$. If for each $g \in G$, $\zeta(g) = e$, then $\mu$ is said to be \textbf{$\phi$-invariant}.
	\end{defn}

	\begin{ex}
		Let $G$ be a locally compact group and $\mu: \MB(G) \rightarrow \RG$ a left Haar measure. Define the actions $\phi, \psi : G \times G \rightarrow G$ by $\phi(g, x) = gx$ and $\psi(g, x) = xg^{-1}$. Then $\mu$ is $\phi$-invariant and $\mu$ is relatively $\psi$-invariant with multiplier $\Del$. 
	\end{ex}
	
	\begin{proof}
		Clear.
	\end{proof}

	\begin{ex}
		Let $G$ be a group, $(X, \MA, \mu)$ a semifinite measure space, $\phi: G \times X \rightarrow X$ a group action and $\zeta: G \rightarrow (0, \infty)$. Suppose that $\mu \neq 0$. If $\mu$ is relatively $\phi$-invariant with multiplier $\zeta$, then 
		\begin{enumerate}
			\item $\zeta$ is a homomorphism
			\item for each $g \in G$, $f \in L^1(\mu) \cup L^+$, $$\int L_g f \dmu = \zeta(g) \int f \dmu$$ 
		\end{enumerate} 
	\end{ex}

	\begin{proof}\
		\begin{enumerate}
			\item Let $g,h \in G$. Choose $U \in \MA$ such that $\mu(U) \in (0, \infty)$. Then 
			\begin{align*}
				\zeta(gh) \mu(U)
				&= \mu(gh \cdot U) \\
				&= \mu(g \cdot (h \cdot U)) \\
				&= \zeta(g) \mu(h \cdot U) \\
				&= \zeta(g) \zeta(h) \mu(U)
			\end{align*}
			Then $\zeta(gh) = \zeta(g) \zeta(h)$. Since $g, h \in G$ are arbitary, $\zeta$ is a homomorphism.
			\item Let $g \in G$ and $U \in \MA$. Set $f = \chi_U$. Then 
			\begin{align*}
				\int L_g f \dmu
				&= \int \chi_{gU} \dmu \\
				&= \mu(gU) \\
				&= \zeta(g) \mu(U) \\
				&= \zeta(g) \int f \dmu \\
			\end{align*} 
			Linearity of $L_g$ implies that for each $f \in S^+$, 
			$$\int L_g f \dmu = \zeta(g) \int f \dmu \\$$
			Let $f \in L^+$. Then there exists a sequence $(f_n)_{n \in \N} \subset S^+$ such that $f_n \convt{p.w.} f$ and for each $N \in \N$, $f_n \leq f_{n+1}$. Hence $L_g f_n \convt{p.w.} L_gf$ and for each $N \in \N$, $L_g f_n \leq L_g f_{n+1}$. The monotone convergence theorem then implies that 
			\begin{align*}
				\int L_g f \dmu
				&= \limn \int L_g f_n \dmu \\
				&= \limn \zeta(g) \int f_n \dmu \\ 
				&= \zeta(g) \limn \int f_n \dmu \\ 
				&= \zeta(g) \int f \dmu \\
			\end{align*} 
		Let $f \in L^1(\mu)$. If $f:X \rightarrow \R$, then $f = f^+ - f^-$ and 
		\begin{align*}
			\int L_g f \dmu
			&= \int L_g(f^+ - f^-) \dmu \\
			&= \int L_g f^+ \dmu - \int L_g f^- \dmu \\
			&= \zeta(g) \int f^+ \dmu - \zeta(g) \int f^- \dmu \\
			&= \zeta(g) \int f^+  -  f^- \dmu \\
			&= \zeta(g) \int f \dmu 
		\end{align*} 
		If $f:X \rightarrow \C$, then there exist $a,b:X \rightarrow \R$ such that $f = a + ib$. Then 
		\begin{align*}
			\int L_g f \dmu
			&= \int L_g(a + i b) \dmu \\
			&= \int L_g a \dmu +i \int L_g b \dmu \\
			&= \zeta(g) \int a \dmu + i \zeta(g) \int b \dmu \\
			&= \zeta(g) \int a + ib \dmu \\
			&= \zeta(g) \int f \dmu 
		\end{align*} 
		\end{enumerate}
	\end{proof}

	\begin{defn}
		Let $X$ be a set, $G$ a group, $\phi: G \times X \rightarrow X$ a group action, $f :X \rightarrow \C$ and $x \in X$. We define $f^x: G \rightarrow \C$ by $$f^x(g) = f(g^{-1} \cdot x)$$
	\end{defn}

	\begin{ex}
		Let $X$ be a LCH space, $G$ a locally compact group $\phi: G \times X \rightarrow X$ a proper group action and $f \in C_c(X)$. Then for each $x \in X$, $f^x \in C_c(G)$.  
	\end{ex}

	\begin{proof}
		
	\end{proof}

	\begin{ex}
		Let $X$ be a LCH space, $G$ a locally compact group with left Haar measure $\mu$, $\phi: G \times X \rightarrow X$ a group action and $f \in C_c(X)$. Define $f^* :X \rightarrow \C$ by $$f^*(x) = \int f(g^{-1} \cdot x) \dmu(g)$$  
	\end{ex}


	
	
	
	
	
	
	
	
	
	
	
	
	
	
	
	
	
	
	
	
	
	
	
	
	
	
	
	
	
	\newpage
	\section{Hausdorff Measure}
	
	\subsection{Introduction}
	
	\begin{defn}
	Let $X$ be a metric space and $\mu^*: \MP(X) \rightarrow [0, \infty]$ an outer measure on $X$. Then $\mu^*$ is said to be a \textbf{metric outer measure on $X$} if for each $A, B \subset X$, $d(A,B) > 0$ implies that 
	\begin{equation*}
	\mu^*(A \cup B) = \mu^*(A) + \mu^*(B)
	\end{equation*}
	\end{defn}	
	
	\begin{ex}
	Let $X$ be a metric space and $\mu^*: \MP(X) \rightarrow [0, \infty]$ a metric outer measure on $X$.
	Then for each $A \in \MB(X)$, $A$ is $\mu^*$-outer measurable. 
	\end{ex}
	
	\begin{proof}
	
	\end{proof}
	
	
	\begin{defn}
	Let $X$ be a metric space, $E \subset X$ and $\del >0$. Define $\MA_{E, \del} \subset \MP(X)^{\N}$ by 
	\begin{equation*}
	 \MA_{E, \del} = \inf \bigg \{(A_j)_{j \in \N} \subset \MP(X): E \subset \bigcup\limits_{j \in \N}A_j \text{ and for each $j \in \N$, } \diam(A_j) < \del \bigg \}
	\end{equation*}
	\end{defn}
	
	\begin{ex}
	Let $X$ be a metric space, $E \subset X$ and $\del_1, \del_2 >0$. If $\del_1 \leq \del_2$, then $\MA_{E, \del_1} \subset \MA_{E, \del_2}$.
	\end{ex}
	
	\begin{proof}
	Clear.
	\end{proof}
	
	\begin{defn}
	Let $X$ be a metric space, $d \geq 0$ and $\del >0$. Define $H_{d, \del}: \MP(X) \rightarrow [0, \infty]$ by 
	\begin{equation*}
	H_{d, \del}(E) = \inf \bigg \{\sum_{j \in \N} \diam(A_j)^d: (A_j)_{j \in \N} \in \MA_{E, \del} \bigg \}
	\end{equation*}
	\end{defn}
	
	\begin{ex}
	Let $X$ be a metric space, $d \geq 0$ and $\del_1, \del_2 >0$. If $\del_1 \leq \del_2$, then $H_{d, \del_2} \leq H_{d, \del_1}$.
	\end{ex}
	
	\begin{proof}
	Clear.
	\end{proof}
	
	\begin{defn}
	Let $X$ be a metric space and $d \geq 0$. We define the \textbf{$d$-dimensional Hausdorff outer measure}, denoted $H_{d}: \MP(X) \rightarrow [0, \infty]$, by 
	\begin{align*}
	H_{d}(E) 
	&= \sup_{\del > 0} H_{d, \del}(E) \\
	&= \lim_{\del \rightarrow 0^+} H_{d, \del}(E)
	\end{align*}
	\end{defn}
	
	\begin{ex}
	Let $X$ be a metric space and $d \geq 0$. Then $H_d: \MP(X) \rightarrow [0, \infty]$ is an outer measure on $X$.
	\end{ex}
	
	\begin{proof}
	
	\end{proof}
	
	\begin{ex}
	Let $X$ be a metric space and $d \geq 0$. Then $H_d: \MP(X) \rightarrow [0, \infty]$ is a metric outer measure on $X$.
	\end{ex}
	
	\begin{proof}
	
	\end{proof}
	
	
	
	
	
	
	
	
	
	
	
	
	
	
	
	
	
	
	\newpage
	\subsection{Hausdorff Measure on Smooth Manifolds}
	
	
	
	
	
	
	
	
	
	
	
	
	
	
	
	
	
	
	
	
		
	\newpage
	\subsection{Induced Measures on Isometric Orbit Spaces}
	
	\begin{note}
	This section assumes familiarity with induced metrics on orbit spaces of metric spaces under isometric group actions. See section $9.1$ of \cite{analysis} for details. 
	\end{note}
	
	\begin{note}
	
	\end{note}
	
	\begin{defn}
	Let $(X, d)$ be a metric space, $G$ a group, and $\phi: G \times X \rightarrow X$ an isometric group action. Suppose that $(X/G, \bar{d})$ is a metric space. Let $\mu: \MB(X) \rightarrow [0, \infty]$ be a measure on $X$. We define $\bar{\mu}: \MB(X/G) \rightarrow [0, \infty]$ by $\bar{\mu} = \pi_* \mu$. 
	\end{defn}
	
	\begin{note}
	If $\mu \ll H_p^X$, where $X$ has Hausdorff dimension $p$, I want to be able to define $\bar{\mu}$ in terms of $H_q^{X/G}$ where $X/G$ has Hausdorff dimension $q$. I was unable to do this. It might be possible with some manifold theory, for instance $O(2)$ acting on $\R^2$.
	\end{note}
	
	\begin{defn}
	Let $(X, d)$ be a metric space, $G$ a group, and $\phi: G \times X \rightarrow X$ an isometric group action. Suppose that $(X/G, \bar{d})$ is a metric space. Let $\mu: \MB(X) \rightarrow [0, \infty]$ be a measure on $X$. Then $\mu$ is said to be $G$-invariant if for each $g \in G$, $U \in \MB(X)$, 
	\begin{equation*}
	\mu(g \cdot U) = \mu(U)
	\end{equation*}
	\end{defn}
	
	\begin{ex}
	Let $X$ be a metric space, $G$ a group, and $\phi: G \times X \rightarrow X$ an isometric group action. Then for each $p \geq 0$, $H_p$ is $G$-invariant. 
	\end{ex}	
	
	\begin{proof}
	Clear.
	\end{proof}
	
	\begin{ex}
	Let $X$ be a metric space, $G$ a group, and $\phi: G \times X \rightarrow X$ an isometric group action. Let $\mu: \MB(X) \rightarrow [0, \infty]$ be a measure on $X$. Suppose that $\mu \ll H_p$. Then $\mu$ is $G$-invariant iff $d\mu /d H_p$ is $G$-invariant.
	\end{ex}	
	
	\begin{proof}
	Suppose that $\mu$ is $G$-invariant. Let $g \in G$ and $U \in \MB(X)$. Then 
	\begin{align*}
	\int_U L_g \frac{d\mu}{d H_p}(x) \, d  H_p (x)
	&= \int_U \frac{d\mu}{d H_p} \circ l_{g}^{-1}(x) \, d  H_p(x) \\
	&= \int_{l_{g}^{-1}( U) } \frac{d\mu}{d H_p}(x) \, d (l_{g}^{-1})_*H_p(x) \\
	&= \int_{g^{-1} \cdot U } \frac{d\mu}{d H_p}(x) \, d H_p(x) \\
	&= \mu(g^{-1} \cdot U) \\
	&= \mu (U)
	\end{align*}
	So that \begin{equation*}
	L_g \frac{d\mu}{d H_p} = \frac{d\mu}{d H_p}
	\end{equation*}
	The Converse is similar.
	\end{proof}
	
	\begin{ex}
	Let $(X, d)$ be a metric space, $G$ a group, and $\phi: G \times X \rightarrow X$ an isometric group action. Suppose that $\bar{d}$ is a metric. Let $\mu: \MB(X) \rightarrow [0, \infty]$ be a measure on $X$. Suppose that $\mu$ is $G$-invariant, $\mu \ll H_p^X$ and $d\mu / dH_p^X$ is continuous. Then $\bar{\mu} \ll \bar{H}_p^X$, $d\bar{\mu}/d \bar{H}_p^X$ is $G$-invariant, $d\bar{\mu}/d \bar{H}_p^X$ is continuous and 
	\begin{equation*}
	\frac{d \bar{\mu}}{d \bar{H}_p^X} = \overline{\frac{d \mu}{d H_p^X}}
	\end{equation*}
	\end{ex}
	
	\begin{proof}
	A previous exercise implies that $\bar{\mu} \ll \bar{H}_p^X$. Set $f = d \mu /d H_p^X$. Since $\mu$ is $G$-invariant, $f$ is $G$-invariant. Since $f$ is continuous, an exercise in section $9.2$ of \cite{analysis} implies that $\bar{f}$ is continuous and $f = \bar{f} \circ \pi$. Let $E \in \MB(X/G)$. Then 
	\begin{align*}
	\int_E \bar{f} d \bar{H}_p^X 
	&= \int_{\pi^{-1}(E)} \bar{f} \circ \pi dH_p^X \\
	&= \int_{\pi^{-1}(E)} f dH_p^X \\
	&= \mu(\pi^{-1}(E)) \\
	&= \bar{\mu}(E) \\
\end{align*}	 
	Therefore, by definition, we have that
	\begin{equation*}
	\frac{d \bar{\mu}}{d \bar{H}_p^X} = \bar{f} = \overline{\frac{d \mu}{d H_p^X}}
	\end{equation*}
	\end{proof}
	
	
	
	
	
	
	
	
	
	
	
	
	
	
	
	
	
	
	
	
	
	
	
	
	
	
	\newpage
	\section{Measure and Integration on Banach Spaces}
	
	\subsection{Borel Measures on Banach Spaces}
	
	\begin{defn}
		Let $X$ be a normed vector space. We define the \textbf{cylindrical $\sig$-algebra on $X$}, denoted $\ME(X)$, by $$\ME(X) = \sig_X(X^*)$$
	\end{defn}

	\begin{ex}
		Let $(X, \MA)$ be a measurable space, $Y$ a normed vector space and $f: X \rightarrow Y$. Then $f$ is ($\MA$-$\ME(Y)$) measurable iff for each $\phi \in X^*$, $\phi \circ f$ is ($\MA$-$\MB(\C)$) measurable. 
	\end{ex}

	\begin{proof}
		Immediate by exercise about initial $\sig$-algebra.
	\end{proof}
	
	\begin{ex}
		Let $X$ be a normed vector space. Then $\ME(X) \subset \MB(X)$.
	\end{ex}
	
	\begin{proof}
		Let $\phi \in X^*$. Since $\phi$ is continuous, $\phi$ is $\MB(X)$-measurable. Hence for each $E \in \MB_{\C}$, $\phi^{-1}(E) \in \MB(X)$. Thus $\{ \phi^{-1}(E) : E \in \MB(\C) \text{ and } \phi \in X^* \} \subset \MB(X)$.  This implies that 
		\begin{align*}
			\ME(X) 
			& = \sig_X(X^*) \\
			& = \sig(\{ \phi^{-1}(E):E \in \MB(\C) \text{ and } \phi \in X^* \}) \\
			& \subset \MB(X) 
		\end{align*} 
	\end{proof}
	
	\begin{ex} \textbf{Mourier's Theorem:} \\
		Let $X$ be a normed vector space. If $X$ is separable, then $\ME(X) = \MB(X)$. \\
		\textbf{Hint:} Let $(x_n)_{n \in \N} \subset X$ be a dense subset. An exercise in the section on duality implies that there exist $(\phi_n)_{n \in \N} \subset X^*$ such that for each $n \in \N$, $\|\phi_n\| = 1$ and $\phi_n(x_n) = \|x_n\|$ and for each $x \in X$, $\|x\| = \sup\limits_{n \in \N} |\phi_n(x)|$. Then $ \cl B(0, 1) \in \ME(X)$.  
	\end{ex}
	
	\begin{proof}
		Suppose that $X$ is separable. Then there exists $(x_n)_{n \in \N} \subset X$ such that $(x_n)_{n \in \N}$ is dense in $X$. An exercise from the section on duality in \cite{analysis} implies that there exists $(\phi_n)_{n \in \N} \subset X^*$ such that for each $n \in \N$, $\|\phi_n\| = 1$ and $\phi_n(x_n) = \|x_n\|$. A previous exercise implies that for each $x \in X$, $\|x\| = \sup\limits_{n \in \N} |\phi_n(x)|$. Let $x \in X$ and $r > 0$. Then $r^{-1}\|x - y \| = \sup\limits_{n \in \N} |r^{-1}\phi_n(x - y)|$ and 
		\begin{align*}
			\cl B(x,r)
			& = \{y \in X: \|x -y \| \leq r\} \\
			& = \{y \in X: r^{-1}\|x - y \| \leq 1\} \\
			& = \bigcap_{n \in \N} \{y \in X: |r^{-1}\phi_n(x - y)| \leq 1\} \\
			& = \bigcap_{n \in \N} \{y \in X: |\phi_n(x - y)| \leq r\} \\ 
			& = \bigcap_{n \in \N} \{y \in X: |\phi_n(x) - \phi_n(y)| \leq r\} \\
			& = \bigcap_{n \in \N} \phi_n^{-1}(\cl B_{\C}(\phi_n (x), r)) \\
			& \in \ME(X)
		\end{align*}
		Let $A \subset X$. Suppose that $A$ is open. Since $X$ is separable, there exist $(a_n)_{n \in \N} \subset A$ and $(r_n)_{n \in \N} \subset (0, \infty)$ such that 
		\begin{align*}
			A 
			& = \bigcup_{n \in \N} \cl B(a_n, r_n) \\
			& \in \ME(X)
		\end{align*}
		Therefore, $\MB(X) \subset \ME(X)$. \\
		The previous exercise implies that $\ME(X) \subset \MB(X)$. So $\ME(X) = \MB(X)$.
	\end{proof}
	
	\begin{ex}
		Let $X$ be a separable normed vector space and $\mu, \nu \in \MM(X)$. Then $\mu = \nu$ iff for each $\phi \in X^*$, $\phi_*\mu = \phi_*\nu$.
	\end{ex}
	
	\begin{proof}
		If $\mu = \nu$, then clearly for each $\phi \in X^*$, $\phi_*\mu = \phi_*\nu$. \\
		Conversely, suppose that for each $\phi \in X^*$, $\phi_*\mu = \phi_*\nu$. Let $E \in \MB(\C)$ and $\phi \in X^*$. Then 
		\begin{align*}
			\mu(\phi^{-1}(E)) 
			& = \phi_*\mu(E) \\
			& = \phi_*\nu(E) \\
			&= \nu(\phi^{-1}(E))
		\end{align*}
		Set $\MP = \{\phi^{-1}(E): \phi \in X^* \text{ and } E \in \MB(\C)\}$. Then $\MP$ is a $\pi$-system. Since 
		\begin{align*}
			\sig(\MP) 
			& = \ME(X) \\
			& = \MB(X)
		\end{align*}
		An exercise from the section on complex measures that uses Dynkin's lemma implies that $\mu = \nu$.
	\end{proof}
	
	\begin{defn}
		Let $X$ be a real normed vector space and $\mu \in \MM(X)$. We define the \textbf{Fourier transform of $\mu$}, denoted $\hat{\mu}: X^* \rightarrow \C$, by
		$$\hat{\mu}(\phi) = \int_X e^{-i\phi(x)} \dmu(x)$$ 
	\end{defn}
	
	\begin{ex}
		Let $X$ be a real normed vector space and $\mu \in \MM(X)$. Then $\hat{\mu} : X^* \rightarrow \C$ is bounded.
	\end{ex}
	
	\begin{proof}
		Let $\phi \in X^*$. 
		\begin{align*}
			|\hat{\mu}(\phi)|
			& = \bigg | \int_X e^{-i \phi(x)} \dmu(x) \bigg| \\
			& \leq \int_X |e^{-i \phi(x)}| \, d|\mu|(x) \\
			& = |\mu|(X) \\
		\end{align*}
		So $\hat{\mu}$ is bounded.
	\end{proof}
	
	\begin{ex}
		Let $X$ be a real normed vector space and $\mu \in \MM(X)$. Then $\hat{\mu} \in C_b(X^*)$.
	\end{ex}
	
	\begin{proof}
		Let $(\phi_{n})_{n \in \N} \subset X^*$ and $\phi \in X^*$. Suppose that $\phi_n \rightarrow \phi$. Then $e^{-i \phi_n} \convt{p.w.} e^{-i\phi}$ and for each $n \in N$, 
		\begin{align*}
			|e^{-i \phi_n}| 
			& = 1 \\
			& \in L^1(|\mu|)
		\end{align*}
		The dominated convergence theorem implies that
		\begin{align*}
			|\hat{\mu}(\phi_n) - \hat{\mu}(\phi)| 
			& = \bigg| \int_X e^{-i \phi_n(x)} \dmu(x) - \int_X e^{-i \phi(x)} \dmu(x)\bigg| \\
			& =  \bigg| \int_X e^{-i \phi_n(x)} - e^{-i \phi(x)} \dmu(x) \bigg| \\
			& \leq \int_X |e^{-i \phi_n(x)} - e^{-i \phi(x)}| \, d|\mu|(x) \\
			& \rightarrow 0
		\end{align*}
		So $\hat{\mu}: X^* \rightarrow \C$ is continuous (in the norm topology). Hence $\hat{\mu} \in C_b(X^*)$.
	\end{proof}
	
	\begin{defn}
		Let $X$ be a real normed vector space. We define $\MF: \MM(X) \rightarrow C_b(X^*)$ by $$\MF(\mu) = \hat{\mu}$$
	\end{defn}
	
	\begin{ex}
		Let $X$ be a real normed vector space. Then $\MF: \MM(X) \rightarrow C_b(X^*)$ is linear.
	\end{ex}
	
	\begin{proof}
		Let $\mu, \nu \in \MM(X)$ and $\phi \in X^*$. Then 
		\begin{align*}
			\MF[\mu + \nu](\phi) 
			& = \int_X e^{-i \phi(x)} \, d[\mu + \nu](x) \\
			& = \int_X e^{-i \phi(x)} \dmu(x) + \int_X e^{-i \phi(x)} \dnu(x) \\
			& = \MF[\mu](\phi) + \MF[\nu](\phi) 
		\end{align*}
		Since $\phi \in X^*$ is arbitrary, $\MF(\mu + \nu) = \MF(\mu) + \MF(\nu)$ and $\MF$ is linear.
	\end{proof}
	
	\begin{ex}
		Let $X$ be a real normed vector space. If $X$ is separable, then $\MF$ is injective.  
	\end{ex}
	
	\begin{proof}
		Suppose that $X$ is separable. Let $\mu \in \MM(X)$. Suppose that $\mu \in \ker \MF$. Then $\hat{\mu} =0$ and for each $\phi \in X^*$, 
		\begin{align*}
			0 
			& = \hat{\mu}(\phi) \\
			& = \int_X e^{-i \phi(x)} \dmu(x) \\
			& = \int_{\R} e^{-ix} \, d[\phi_*\mu](x)
		\end{align*}
	\end{proof}
	
	\begin{ex}
		Let $X$ be a real normed vector space. Then $\MF \in L(\MM(X), C_b(X^*))$ and $\|\MF\| \leq 1$.
	\end{ex}
	
	\begin{proof}
		For $\mu \in \MM(X)$ and $\phi \in X^*$, we have that 
		\begin{align*}
			|\MF[\mu](\phi)|
			& =  \bigg| \int_X e^{-i \phi(x)} \dmu(x) \bigg| \\
			& \leq \int_X |e^{-i \phi(x)}| \, d|\mu|(x) \\
			& = |\mu|(X) \\
			& = \|\mu\|
		\end{align*}
		Hence 
		\begin{align*}
			\|\MF(\mu)\| 
			& = \sup_{\phi \in X^*} |\MF[\mu](\phi)| \\
			& \leq \|\mu\|
		\end{align*}
		which implies that $\MF \in L(\MM(X), C_b(X^*))$ and $\|\MF\| \leq 1$.
	\end{proof}
	
	
	
	
	
	
	
	
	
	
	
	
	
	
	
	
	
	
	
	
	
	
	
	
	
	\newpage
	\subsection{The Bochner Integral}
	
	\begin{defn}
		Let $(X, \MA)$ be a measurable space, $Y$ a Banach space and $f:X \rightarrow Y$. Then $f$ is said to be \textbf{strongly measurable} if 
		\begin{enumerate}
			\item $f$ is ($\MA$-$\MB(Y)$) measurable
			\item $f(X)$ is separable
		\end{enumerate}
		We define $L^0_Y(X, \MA) = \{f:X \rightarrow Y: f \text{ is strongly measurable}\}$
	\end{defn}

	\begin{ex}
		Let $(X, \MA)$ be a measurable space, $Y$ a Banach space and $f:X \rightarrow Y$. Then $f$ is strongly measurable iff 
		\begin{enumerate}
			\item $f$ is ($\MA$-$\ME(Y)$) measurable
			\item $f(X)$ is separable
		\end{enumerate}
	\end{ex}

	\begin{proof}
		
	\end{proof}


	
	\begin{ex} \lex{00000} 
	Let $(X, \MA, \mu)$ be a measure space and $Y$ a Banach space. Then $L_Y^0(X, \MA)$ is a vector space.
	\end{ex}
	
	\begin{proof}
	Let $f, g \in L_Y^0(X, \MA)$ and $\lam \in \C$. By definition, $f$ and $g$ are measurable. Since $f+\lam g $ is a composition of measurable maps, $f + \lam g$ is measurable. Therefore $f + \lam g \in L_Y^0(X, \MA)$. Clearly constant maps are measurable and hence $0 \in L_Y^0(X, \MA)$. So $L_Y^0(X, \MA)$ is a vector space. 
	\end{proof}

	\begin{defn} \ld{00000} 
		Let $(X, \MA)$ be a measurable space, $Y$ a Banach space and $\phi: X \rightarrow Y$. Then $\phi$ is said to be \textbf{simple} if 
		\begin{enumerate}
			\item $\phi$ is $(\MA, \MB(X))$-measurable
			\item $\phi(X)$ is finite
		\end{enumerate}
		If $\phi$ is simple then the \textbf{standard representation of $\phi$} is defined to be the sum $$\phi = \sum\limits_{j=1}^n \chi_{E_j}y_j$$ where $(y_j)_{j=1}^n = \phi(X)$ and for each $j \in \{1, \cdots, n\}$, $E_j = \phi^{-1}(y_j)$. We define $$S_Y(X, \MA) = \{f \in L_Y^0(X, \MA): f \text{ is simple}\}$$
	\end{defn}
	
	\begin{note}
		If $\phi = \sum\limits_{j=1}^n \chi_{E_j}y_j$ is in the standard representation, then $(E_j)_{j=1}^n$ are disjoint and $\bigcup\limits_{j=1}^n E_j = X$.
	\end{note}

	\begin{ex}
		Let $(X, \MA)$ be a measurable space, $Y$ a Banach space. Then 
		\begin{enumerate}
			\item $S_Y $ is a subspace of $L^0_Y(X, \MA)$
			\item Let $\phi, \psi \in S_Y$. Suppose that the standard representation of $\phi$ is 
			$$\phi = \sum\limits_{j=1}^n\chi_{A_j}a_j$$ 
			and the standard representation of is $\psi$ is 
			$$\psi = \sum\limits_{j=k}^m\chi_{B_k}b_k$$ 
			Set
			$$L = \{(j,k) \in \N^2: j \leq n, k \leq m, \text{ and } A_j \cap B_k \neq \varnothing\}$$  
			Then the standard representation of $\phi + \psi$ is 
			$$\phi + \psi = \sum\limits_{(j,k) \in L} \chi_{A_j \cap B_k}(a_j + b_k)$$
		\end{enumerate} 
	\end{ex}

	\begin{proof}
		Let $\phi, \psi \in S_Y$ and $\lam \in \C$. Then write Write $\phi = \sum\limits_{j=1}^n\chi_{A_j}a_j$ and $\psi = \sum\limits_{j=k}^m\chi_{B_k}b_k$ in the standard representation. Put $$L = \{(j,k) \in \N^2: j \leq n, k \leq m, \text{ and } A_j \cap B_k \neq \varnothing\}$$ 
		Then the standard representation of $\phi + \lam \psi$ is given by  $\phi + \lam \psi = \sum\limits_{(j,k) \in L} \chi_{A_j \cap B_k}(a_j + \lam b_k)$.
		
	\end{proof}
	
	\begin{defn} \ld{00000} 
		Let $(X, \MA, \mu)$ be a measure space, $Y$ a Banach space and $p \in [1, \infty]$. Define $  \| \cdot \|_p : L_Y^0(X, \MA, \mu) \rightarrow [0, \infty]$ by $$\|f \|_p = \bigg(\int  \|f\|^p d\mu \bigg)^{\frac{1}{p}} \hspace{1.5cm}( p < \infty)$$ 
		and 
		$$\|f \|_{\infty} = \inf \bigg \{\lam >0: \mu\big(\{x \in X: \lam < \|f(x)\|  \}\big) = 0 \bigg \} $$
		We define $$L_Y^p(X, \MA, \mu) =  \{f \in L_Y^0(X, \MA, \mu): \|f \|_p < \infty \}$$
	\end{defn}	
	
	\begin{ex} \lex{00000} 
	Let $(X, \MA, \mu)$ be a measure space, $Y$ a Banach space and $p \in [1, \infty]$. Then $L_Y^p(X, \MA, \mu)$ is a subspace of $L_Y^0(X, \MA, \mu)$. 
	\end{ex}
	
	\begin{proof}Let $f, g \in L^p_Y(X, \MA, \mu)$ and $\lam \in \C$. Then $\|f\|_p, \|g\|_p < \infty$.
	\begin{enumerate}
	\item Clearly $\|\lam f\|_p = |\lam|\|f\|_p < \infty$.
	So $\lam f \in L^p_Y$.
	\item Let $\|\cdot \|'_p: L^0(X, \MA, \mu) \rightarrow \RG$ denote the usual $L^p$ norm. Since $\|f + g\| \leq \|f\| + \|g\|$, we have that 
	\begin{align*}
	\|f+g\|_p 
	&= \| \|f+g\| \|'_p \\
	& \leq \|\|f\| + \|g\| \|'_p \\
	& \leq  \|\|f\| \|'_p + \|\|g\| \|'_p \\
	&= \|f \|_p + \|g\|_p \\
	& < \infty
\end{align*}	
So $f+g \in L^p_Y$.
	\end{enumerate}
	Hence $L^p_Y$ is a subspace.
	\end{proof}
	
	\begin{ex} \lex{00000} 
	Let $(X, \MA, \mu)$ be a measure space, $Y$ a Banach space and $p \in [1, \infty]$. Then 
	\begin{enumerate}
	\item $\|\cdot\|_p$ is a seminorm on $L^p_Y(X, \MA, \mu)$
	\item if we identify functions that are equal $\mu$-a.e., then $\|\cdot\|_p$ is a norm on $L^p_Y(X, \MA, \mu)$
	\end{enumerate}
	\end{ex}
	
	\begin{proof} 
	Let $f, g \in L^p_YX, \MA, \mu)$ and $\lam \in \C$. 
	\begin{enumerate}
	\item The previous exercise implies that, $\|\lam f\|_p = |\lam|\|f\|_p$ and $\|f+g\|_p \leq \|f\|_p + \|g \|_p$. So $\|\cdot\|_p$ is a seminorm on $L_Y^p$.
	\item If $f = 0$ $\mu$-a.e., then $\|f\| = 0$  $\mu$-a.e. Hence
	\begin{align*}
	\|f\|_p 
	&= \|\|f\|\|'_p \\
	&= 0
\end{align*}
	So if we identify functions that are equal $\mu$-a.e., $\|\cdot\|_p$ becomes a norm on $L^p_Y$. 	  
	\end{enumerate}
	\end{proof}
	
	\begin{note}
	So for $(f_n)_{n \in \N} \subset L^p_Y$ and $f \in L^p_Y$, $$f_n \conv{L^p_Y} f \text{ iff } \int \|f_n - f\|^p \rightarrow 0$$ 
	\end{note}
	
	\begin{defn} \ld{00000} 
	Let $(X, \MA, \mu)$ be a measure space, $Y$ a Banach space and $\phi: X \rightarrow Y$. Then $\phi$ is said to be \textbf{simple} if $\phi$ is measurable, $\phi(X)$ is finite and for each $y \in \phi(X) \setminus \{0\}$, $\mu(\phi^{-1}(y)) < \infty$. If $\phi$ is simple then the \textbf{standard representation of $\phi$} is defined to be the sum $$\phi = \sum\limits_{j=1}^n \chi_{E_j}y_j$$ where $(y_j)_{j=1}^n = \phi(X)$ and for each $j \in \{1, \cdots, n\}$, $E_j = \phi^{-1}(y_j)$. We define $$S_Y(X, \MA, \mu) = \{f \in L_Y^0(X, \MA): f \text{ is simple}\}$$
	\end{defn}
	
	\begin{note}
	If $\phi = \sum\limits_{j=1}^n \chi_{E_j}y_j$ is in the standard representation, then $(E_j)_{j=1}^n$ are disjoint and $\bigcup\limits_{j=1}^n E_j = X$.
	\end{note}
	
	\begin{ex}
	Let $(X, \MA, \mu)$ be a measure space and $Y$ a Banach space. Then $S_Y \subset L^1_Y$. 
	\end{ex}
	
	\begin{proof}
	Let $\phi \in S_Y$. Write $\phi = \sum\limits_{j=1}^n \chi_{E_j}y_j$ in the standard representation. Then $\|\phi\| = \sum\limits_{j=1}^n \|y_j\|\chi_{E_j}$. By definition, for each $j \in \{1, \cdots, n\}$, $y_j \neq 0$ implies that $\mu(E_j) < \infty$. Then 
	\begin{align*}
	\int \|\phi\| d\mu 
	&= \sum\limits_{j=1}^n \|y_j\| \mu(E_j) \\
	&< \infty
	\end{align*}
	So $\phi \in L^1_Y$.
	\end{proof}
	
	\begin{ex} \lex{00000} 
	Let $(X, \MA, \mu)$ be a measure space and $Y$ a Banach space. Then $S_Y(X, \MA, \mu)$ is a subspace of $L_Y^0(X, \MA)$
	\end{ex}
	
	\begin{proof}
	Clear.
	\end{proof}
	
	\begin{note}
	For the remainder of this section, we will use the shorthand notation $L^0_Y, L^p_Y$ and $S_Y$ unless the context underlying measure space $(X, \MA, \mu)$ is unclear.
	\end{note}
	
	\begin{defn} \ld{00000} 
	Let $(X, \MA, \mu)$ be a measure space and $Y$ a Banach space. Let $\phi \in S_Y$. Write $\phi = \sum\limits_{j=1}^n\chi_{E_j}y_j$ in the standard representation. With the convention that $\infty \cdot 0_Y = 0_Y$, we define $$\int \phi d\mu = \sum\limits_{j=1}^n \mu(E_j)y_j $$ For $A \in \MA$, define  $$\int_A \phi d \mu = \int \chi_A \phi d\mu \\$$
	\end{defn}
	
	\begin{ex} \lex{00000} 
	Let $(X, \MA, \mu)$ be a measure space, $Y$ a Banach space, $\phi \in S_Y$ and $A \in \MA$. Write $\phi = \sum\limits_{j=1}^n\chi_{E_j}y_j$ in the standard representation. Then $$\int_A \phi d\mu = \sum_{j=1}^n \mu(A \cap E_j)y_j$$
	\end{ex}
	
	\begin{proof}
	Note that $\chi_A \phi = \sum\limits_{j=1}^n\chi_{A \cap E_j}y_j$.
	\end{proof}		
	
	\begin{ex} \lex{00000} 
	Let $(X, \MA, \mu)$ be a measure space, $Y$ a Banach space, $\phi, \psi \in S_Y$ and $\lam \in \C$. Then $$\int \phi + \lam \psi d\mu = \int \phi d \mu + \lam \int \psi d\mu $$
	\end{ex}
	
	\begin{proof}
	If $\lam =0$, then the result clearly holds. Suppose that $\lam \neq 0$.	Write $\phi = \sum\limits_{j=1}^n\chi_{A_j}a_j$ and $\psi = \sum\limits_{j=k}^m\chi_{B_k}b_k$ in the standard representation. Put $$L = \{(j,k) \in \N^2: j \leq n, k \leq m, \text{ and } A_j \cap B_k \neq \varnothing\}$$ Then the standard representation of $\phi + \lam \psi$ is given by  $\phi + \lam \psi = \sum\limits_{(j,k) \in L} \chi_{A_j \cap B_k}(a_j + \lam b_k)$.
	So 
	\begin{align*}
	\int \phi + \lam \psi d \mu 
	&= \int \sum_{(j,k) \in L} \chi_{A_j \cap B_k}(a_j + \lam b_k) d \mu \\
	&= \sum_{(j,k) \in L} \mu(A_j \cap B_k)(a_j + \lam b_k) \\
	&= \sum_{j = 1}^n \sum_{k=1}^m \mu(A_j \cap B_k)(a_j + \lam b_k)  \\ 
	&= \sum_{j = 1}^n \sum_{k=1}^m \mu(A_j \cap B_k)a_j + \lam \sum_{j = 1}^n \sum_{k=1}^m \mu(A_j \cap B_k) b_k \\
	&= \sum_{j = 1}^n \mu(A_j)a_j +  \lam\sum_{k=1}^m \mu(B_k) b_k \\
	&= \int \phi d \mu + \lam \int \psi d\mu 
	\end{align*}
	\end{proof}

	\begin{ex} \lex{00000} 
	Let $(X, \MA, \mu)$ be a measure space, $Y$ a Banach space, $\phi \in S_Y$. Then $$\bigg \| \int \phi d\mu   \bigg \|  \leq \int \|\phi \| d \mu$$
	\end{ex}
	
	\begin{proof}
	Write $\phi = \sum\limits_{j=1}^n\chi_{E_j}y_j$ in the standard representation. Note that $\|\phi \| = \sum\limits_{j=1}^n\chi_{E_j} \|y_j\|$. Then 
	\begin{align*}
	\bigg \|  \int \phi d\mu  \bigg \| 
	&=  \bigg \|  \int \sum\limits_{j=1}^n\chi_{E_j}y_j  d\mu  \bigg \| \\
	&= \bigg \| \sum\limits_{j=1}^n \mu(E_j)y_j \bigg \| \\
	& \leq \sum\limits_{j=1}^n \mu(E_j) \|y_j \| \\
	&= \int \sum\limits_{j=1}^n \|y_j \| \chi_{E_j} d \mu \\
	&= \int \|\phi \| d\mu
	\end{align*}
	\end{proof}
	
	\begin{ex} \lex{00000} 
	Let $(X, \MA, \mu)$ be a measure space, $Y$ a Banach space, $f \in L^1_Y$ and $(\phi_n)_{n \in \N} \subset S_Y$. If $\phi_n \conv{L^1_Y} f$, then $$\limn \int \phi_n d\mu $$ exists.
	\end{ex}
	
	\begin{proof}
	Suppose that $\phi \conv{L^1_Y} f$. Then by definition, $$\int \|\phi_n -f\| d\mu \rightarrow 0$$ Let $m,n \in \N$. Then 
	\begin{align*}
	\bigg \|\int \phi_m d\mu - \int \phi_n d\mu \bigg \| 
	&= \bigg \| \int \phi_m  - \phi_n d\mu  \bigg \| \\
	& \leq  \int \|\phi_m  - \phi_n\| d \mu \\
	& \leq \int \|\phi_m  - f \| d \mu + \int \|f  - \phi_n\| d \mu
	\end{align*}
	Hence $( \int \phi_n  d \mu)_{n \in \N} \subset Y$ is Cauchy and $\lim\limits_{n \rightarrow \infty} \int \phi_n d\mu$ exists.
\end{proof}		

	\begin{ex} \lex{00000} 
	Let $(X, \MA, \mu)$ be a measure space, $Y$ a Banach space, $f \in L^1_Y$ and $(\phi_n)_{n \in \N}, (\psi_n)_{n \in \N} \subset S_Y$. If $\phi_n \conv{L^1_Y} f$ and $\psi_n \conv{L^1_Y} f$, then $$\lim\limits_{n \rightarrow \infty} \int \phi_n d\mu = \lim\limits_{n \rightarrow \infty} \int \psi_n d\mu$$
	\end{ex}
	
	\begin{proof}
	Suppose that $\phi_n \conv{L^1_Y} f$ and $\psi_n \conv{L^1_Y} f$. Let $\ep >0$. By defintion, there exist $N_1 \in \N$ such that for each $n \in \N$, $ n \geq N_1$ implies that $\int \|\phi_n - f\| d\mu < \frac{\ep}{6}$ and $\int \|\psi_n - f\| d\mu < \frac{\ep}{6}$. Similarly to the previous exercise we have that for each $n \in \N$, $n \geq N_1$ implies that
	\begin{align*}
	\bigg \| \int \phi_n d\mu  - \int \psi_n d \mu \bigg \|
	&= \bigg \| \int \phi_n - \psi_n d\mu \bigg \| \\
	& \leq \int \| \phi_n - \psi_n \| d\mu \\
	& \leq  \int \| \phi_n -f \| d \mu + \int \| f - \psi_n \| d\mu 	\\
	& <  \frac{\ep}{6} + < \frac{\ep}{6} \\
	&= \frac{\ep}{3}
	\end{align*}	 
Put $I_\phi = \lim\limits_{n \rightarrow \infty} \int \phi_n d\mu$ and $I_{\psi} = \lim\limits_{n \rightarrow \infty} \int \psi_n d\mu$. Then there exists $N_2 \in \N$ such that for each $n \in \N$, if $ n \geq N_2$, then $$\bigg \| \int \phi_n d\mu - I_\phi \bigg \| < \frac{\ep}{3}$$ and $$\bigg \| \int \psi_n d\mu - I_\psi \bigg \| < \frac{\ep}{3}$$ 
	Choose $N = \max(N_1, N_2)$. Then for each $n \in \N$, $n \geq N$ implies that
	\begin{align*}
	\|I_\phi - I_\psi\| 
	& \leq \bigg \|I_\phi - \int \phi_n d\mu \bigg \| + \bigg \| \int \phi_n d\mu - \int \psi_n d\mu \bigg \| +  \bigg \| \int \psi_n d\mu - I_\psi \bigg \| \\
	& = < \frac{\ep}{3} + < \frac{\ep}{3} + < \frac{\ep}{3} \\
	&= \ep
	\end{align*}
	Since $\ep >0$ is arbitrary, $I_\phi = I_\psi$.
	\end{proof}
	
	\begin{ex} \lex{00000} 
	Let $Y$ be a Banach space and $(y_n)_{n \in \N} \subset Y$ a countable dense subset. For $\ep >0$ and $n \in \N$, define $B^{\ep}_n \in \MB(Y)$ by 
	$$B^{\ep}_n = \{y \in Y: \|y - y_n\| < \ep \| y_n \| \}$$ 
	Then for each $\ep \geq 0$, 
	\begin{enumerate}
		\item $$Y \setminus \{ 0 \} \subset \bigcup\limits_{n \in \N}B^{\ep}_n$$
		\item if $\ep \leq 1$, $$Y \setminus \{ 0 \} = \bigcup\limits_{n \in \N}B^{\ep}_n$$
	\end{enumerate}
	\end{ex}	
	
	\begin{proof}
	Let $\ep \geq 0$. 
	\begin{enumerate}
		\item For the sake of contradiction, suppose that $Y\setminus \{ 0 \} \not \subset \bigcup\limits_{n \in \N}B^{\ep}_n$. Then there exists $y \in Y$ such that $y \neq 0$ and for each $n \in \N$, $\|y- y_n\| \geq \ep \|y_n\|$. Since $(y_n)_{n \in \N}$ is dense in $Y$, there exists a subsequence $(y_{n_j})_{j \in \N} \subset (y_n)_{n \in \N}$ such that for each $j \in \N$, $\|y_{n_j} - y\| < 1/j$. Then for each $j \in \N$,
		\begin{align*}
			\|y_{n_j}\| 
			& \leq  \ep^{-1}\|y- y_{n_j}\| \\
			& < \ep^{-1}1/j
		\end{align*}
		So that $y_{n_j} \rightarrow y$ and $y_{n_j} \rightarrow 0$. Since $y \neq 0$, this is a contradiction and thus $$Y \setminus \{ 0 \} \subset \bigcup\limits_{n \in \N}B^{\ep}_n$$ 
		\item Suppose that $\ep \leq 1$. For the sake of contradiction, suppose that $0 \in \bigcup\limits_{n \in \N}B^{\ep}_n $. Then there exists $n \in \N$ such that $0 \in B^{\ep}_n$.  By definition, 
		\begin{align*}
			\|y_n\| 
			& = \|0 - y_n\| \\
			& < \ep \|y_n\| \\
			& \leq \|y_n\|
		\end{align*}
		Which is a contradiction. So $0 \not \in  \bigcup\limits_{n \in \N}B^{\ep}_n$. Hence $\{0\} \subset \bigg( \bigcup\limits_{n \in \N}B^{\ep}_n \bigg)^c$ and $ \bigcup\limits_{n \in \N}B^{\ep}_n \subset \{0\}^c$. Hence $ \bigcup\limits_{n \in \N}B^{\ep}_n \subset Y \setminus \{0\}$ and $Y \setminus \{ 0 \} = \bigcup\limits_{n \in \N}B^{\ep}_n$.
	\end{enumerate}
	\end{proof}

	\begin{ex}
	Let $(X, \MA)$ be a measurable space, $Y$ a separable Banach space and $f \in L^0_Y(X, \MA)$.  Let $(y_n)_{n \in \N} \subset Y$ be a countable dense subset. For $j \in \N$, define $(A_n^j)_{n \in \N} \subset \MB(Y)$ and $(E_n^j)_{n \in \N} \subset \MA$ by  
	\begin{itemize}
			\item $A_1^j = B^{1/j}_1$ 
			\item $A_n^j = B^{1/j}_n  \setminus \bigg( \bigcup \limits_{k=1}^{n-1} B^{1/j}_k \bigg)$ 
			\item $E_n^j = f^{-1}(A^j_n)$ 
	\end{itemize}
	Let $j \in \N$. Then
	\begin{enumerate}
		\item $(A_n^j)_{n \in \N}$ is disjoint and $$\bigcup_{n \in \N}A_n^j = Y \setminus \{0\}$$
		\item $(E_n^j)_{n \in \N}$ is disjoint and $$\bigcup_{n \in \N}E_n^j = X \setminus f^{-1}(\{0\})$$
		\item if $j \geq 2$, then for each $n \in \N$ and $x \in E_n^j$, $$\|y_n\| < \frac{j}{j-1} \|f(x)\|$$
	\end{enumerate}
	\textbf{Hint:} reverse triangle inequality
	\end{ex}

	\begin{proof}\
		\begin{enumerate}
			\item Clear by previous exercice
			\item Clear
			\item Suppose that $j \geq 2$. Let $n \in \N$ and $x \in E_n^j$. Then $f(x) \in A_n^j \subset B_n^{1/j}$. Hence 
			\begin{align*}
				\|y_n\| - \|f(x)\|
				& \leq \bigg| \|y_n\| - \|f(x)\| \bigg| \\
				& \leq \|y_n - f(x)\| \\
				& < \frac{1}{j} \|y_n\|
			\end{align*}
			Thus $(1 - 1/j) \|y_n\| < \|f(x)\|$. Since $j-1 > 0$, we have that $$\|y_n\| < \frac{j}{j-1} \|f(x)\|$$
		\end{enumerate}
	\end{proof}
	
	\begin{ex} \lex{00000} 
	Let $(X, \MA, \mu)$ be a measure space, $Y$ a separable Banach space and $f \in L^1_Y(X, \MA, \mu)$. Let $(y_n)_{n \in \N} \subset Y$ be a countable dense subset. For $j \in \N$, define $(E_n^j)_{n \in \N} \subset \MA$ as in the previous exercise and $(\psi_j)_{j \in \N} \subset L^0_Y(X, \MA)$ by 
	$$\psi_j = \sum\limits_{n \in \N}\chi_{E_n^j}y_n$$  
	Then for each $j \in \N$, $j \geq 2$ implies that 
	\begin{enumerate}
		\item $\psi_j \in L^1(X, \MA, \mu)$ 
		\item $\|\psi_j - f\| < \frac{1}{j - 1} \|f\|_1$
	\end{enumerate}
	\end{ex}
	
	\begin{proof}	
	Let $j \in \N$. 
	Suppose that $j \geq 2$. Then 
	\begin{enumerate}
		\item \begin{align*}
			\|\psi_j\|_1 
			& = \int \|\psi_j\| \dmu \\
			& = \int \sum_{n \in \N} \|y_n\| \chi_{E_n^j} \dmu \\
			& =  \sum_{n \in \N} \int_{E_n^j} \|y_n\| \dmu \\
			& \leq \frac{j}{j-1} \sum_{n \in \N} \int_{E_n^j} \|f\| \dmu \\
			& = \frac{j}{j-1} \int_{\bigcup\limits_{n \in \N} E_n^j} \|f\| \dmu \\
			& = \frac{j}{j-1} \int \|f\| \dmu \\
			& = \frac{j}{j-1} \|f\|_1
		\end{align*}
		So $\psi_j \in L^1_Y(X, \MA, \mu)$. 
		\item Similarly, we have that
		\begin{align*}
			\|\psi_j - f\|_1 
			& = \int \|\psi_j - f\| \dmu \\
			& = \int_{f^{-1}(\{0\})} \|\psi_j - f\| \dmu +  \sum_{n \in \N} \int_{E_n^j} \|\psi_j - f\| \dmu \\
			& = \sum_{n \in \N} \int_{E_n^j} \|y_n - f\| \dmu \\
			& \leq  \sum_{n \in \N} \int_{E_n^j}  \frac{1}{j - 1} \|y_n\| \dmu \\
			& \leq  \sum_{n \in \N} \int_{E_n^j}  \frac{1}{j - 1} \|f\| \dmu \\
			& =  \frac{1}{j - 1} \int \|f\| \dmu \\
			& =  \frac{1}{j - 1} \|f\|_1
		\end{align*}
		So $\|\psi_j - f\| < \frac{1}{j - 1} \|f\|_1$.
	\end{enumerate}
	\end{proof}	

	\begin{ex}
		such that $\phi_n \convt{a.e.} f$ and $\phi_n \conv{L_Y^1} f$.\\
		\textbf{Hint:} Choose a countable dense subset $(y_n)_{n \in \N} \subset f(X)$ and define 
	\end{ex}	
	
	\begin{defn} \ld{00000} \textbf{Bochner Integral:}\\
	Let $(X, \MA, \mu)$ be a measure space, $Y$ a separable Banach space and $f:X \rightarrow Y$. Then $f$ is said to be \textbf{Bochner} integrable if $f \in L^1_Y$. If $f$ is Bochner integrable, then there exists $(\phi_n)_{n \in \N} \subset S_Y$ such that $\phi_n \convt{a.e.} f$ and $\phi_n \conv{L_Y^1} f$ and the \textbf{Bochner integral of $f$} with respect to $\mu$, denoted $$\int f d\mu$$ is defined to be $$\int f d\mu = \limn \int \phi_n d\mu$$ 
	\end{defn}
	
	\begin{ex} \lex{00000} 
	Let $(X, \MA, \mu)$ be a measure space, $Y$ a separable Banach space, $f,g \in L^1_Y$ and $\lam \in \C$. Then $$\int f+\lam g d\mu = \int f d \mu + \lam \int g d\mu$$
	\end{ex}
	
	\begin{proof}
	Choose $(\phi_n)_{n \in \N} \subset S_Y$ such that $\phi_n \conv{L^1_Y} f$ and $(\psi_n)_{n \in \N} \subset S_Y$ such that $\psi_n \conv{L^1_Y} g$. Since addition and  scalar multiplication are continuous, $\phi_n + \lam \psi_n \conv{L^1_Y} f+\lam g$. By definition, we have that $$\int \phi_n + \lam \psi_n  d\mu \rightarrow \int f+\lam g d\mu $$ $$\int \phi_n d\mu \rightarrow \int f d\mu$$ and $$ \int \psi_n d\mu \rightarrow \int g d\mu$$ 
	Hence 
	\begin{align*}
	\int f+ \lam g d\mu 
	&= \limn \int \phi_n + \lam \psi_n  d\mu \\
	&= \limn \int \phi_n d\mu + \lam \limn \int \psi_n  d\mu \\
	&= \int f d\mu + \lam \int g d\mu
	\end{align*}
	\end{proof}
	
	\begin{ex} \lex{00000} 
	Let $(X, \MA, \mu)$ be a measure space and $Y$ a separable Banach space. Define $I: L^1_Y \rightarrow Y$ by $$If = \int f d\mu$$
	Then $I \in L(L^1_Y, Y)$ and $\|I\| \leq 1$.
	\end{ex}
	
	\begin{proof}
	Let $f \in L^1_Y$. Choose $(\phi_n)_{n \in \N} \subset S_Y$ such that $\phi_n \conv{L^1_Y} f$. Then 
	\begin{align*}
	\bigg | \int \| \phi_n \| d \mu - \int \| f\| d\mu \bigg |
	&= \bigg | \int \| \phi_n \| - \| f\| d\mu \bigg | \\
	& \leq \int |\| \phi_n \| - \| f\| | d\mu \\
	& \leq  \int \| \phi_n - f\| d\mu \\
	& \rightarrow 0
	\end{align*}
	So  $$ \int \| \phi_n \| d \mu \rightarrow  \int \| f \| d \mu$$
	By continuity of $\|\cdot\|:Y \rightarrow \Rg$,
	\begin{align*}
	\|I f\|
	& = \bigg \| \int f d\mu \bigg \| \\
	&= \bigg \| \limn \int \phi_n d\mu \bigg \| \\
	&= \limn  \bigg \| \int \phi_n d\mu \bigg \| \\
	& \leq \limn \int \| \phi_n \| d \mu \\
	&= \int \| f \| d\mu  \\
	&= \|f\|_1
	\end{align*}
	\end{proof}

	
	\begin{ex} \lex{00000} 
	Let $Y$ be a separable Banach space and $f:[a,b] \rightarrow Y$ continuous. Then $f$ is Banach-integrable.
	\end{ex}
	
	\begin{proof}
	Continuity implies that $f \in L_Y^{\infty}$ and 
	\begin{align*}
	\int \| f \| dm 
	&\leq \| f\|_{\infty}(b-a) \\
	&< \infty
\end{align*}
	so that $f \in L^1_Y$ and $f$ is Bochner integrable.
\end{proof}			
	
	\begin{ex} \lex{00000} \textbf{Dominated Convergence Theorem:}\\  
		Let $(X, \MA, \mu)$ be a measure space, $Y$ a separable Banach space, $(f_n)_{n \in \N} \subset L^1_Y$ and $f \in L^0_Y$. Suppose that $f_n \convt{a.e.} f$ and there exists $g \in L^1$ such that for each $n \in \N$, $\|f_n\| \leq g$. Then $f \in L^1_Y$ and $f_n \conv{L^1} f$. 
	\end{ex}
	
	\begin{proof}
	Since $f_n \convt{a.e.} f$, $\|f\| \leq g$ a.e. and $f \in L^1_Y$. Also, 
	\begin{align*}
	\|f_n - f\| 
	&\leq \|f_n\| + \|f\| \\
	& \leq 2g \text{ a.e.}
	\end{align*}
	Hence $2g - \|f_n - f\| \geq 0$ a.e.
	Fatou's lemma implies that 
	\begin{align*}
	\int 2g \dmu 
	&= \int \limfn(2g - \|f_n - f\|) \dmu \\
	&\leq \limfn \bigg[ \int 2g - \|f_n - f\| \dmu \bigg] \\
	&= \int 2g \dmu - \limpn \int \|f_n - f\| \dmu
\end{align*}	 
	Hence $$0 \leq \limpn \int \|f_n - f\| \dmu \leq 0$$  and $f_n \conv{L^1_Y} f$.
	\end{proof}		
	
	\begin{ex} \lex{00000} 
	Let $(X, \MA, \mu)$ be a measure space, $Y,Z$ separable Banach spaces and $f \in L^1_Y$ and $T \in L(Y,Z)$. Then $T \circ f \in L^1_Z$ and $$\int T \circ f d \mu = T\bigg( \int f d\mu \bigg)$$
	\end{ex}
	
	\begin{note}
	The statement remains true if $T$ is continuous and conjugate-linear. 
	\end{note}
	
	\begin{proof}
	Suppose that $f \in S_Y$. Write $f = \sum\limits_{j=1}^n \chi_{E_j}y_j $ in the standard representation. Then $T \circ f = \sum\limits_{j=1}^n \chi_{E_j}T(y_j)$  and 
	\begin{align*}
	\int T \circ f d \mu 
	&= \sum\limits_{j=1}^n \mu(E_j)T(y_j) \\
	&= T \bigg(\sum\limits_{j=1}^n \mu(E_j)y_j \bigg) \\
	&= T \bigg( \int f d\mu \bigg)
	\end{align*}
	
	For $f \in L^1_Y$, choose $(\phi_n)_{n \in \N} \subset S_Y$ such that $\phi_n \convt{a.e.} f$ and $\phi_n \conv{L^1_Y} f$. Then 
	\begin{align*}
	\|T \circ \phi_n - T \circ f\| 
	&= \|T \circ (\phi_n - f)\| \\
	& \leq \|T\| \|\phi_n - f\|
	\end{align*}
	So $T \circ \phi_n \convt{a.e.} T \circ f$ and $T \circ \phi_n \conv{L^1_Z} T \circ f $. Thus 
	\begin{align*}
	\int T \circ f d\mu 
	&= \limn \int T \circ \phi_n d\mu \\
	&= \limn T \bigg( \int \phi_n d\mu \bigg) \\
	&= T \bigg( \limn \int \phi_n d\mu \bigg) \\
	&= T \bigg( \int f d\mu  \bigg)
	\end{align*}
	\end{proof}
	
	\begin{note}
	Recall that for a function $f:X \times Y \rightarrow Z$, $x \in X$ and $y \in Y$, the functions $f_x:Y \rightarrow Z$ and $f^y:X \rightarrow Z$ are defined by $f_x(y) = f(x,y)$ and $f^y(x) = f(x,y)$.
	\end{note}
	
	\begin{ex} \lex{00000} 
	Let $(X, \MA, \mu)$ be a measure space,  $Y$ a Banach space, $A \subset Y$ open and $f:X \times A \rightarrow Z$. Suppose that for each $y \in A$, $f^y \in L^1(\mu)$. Define $F: Y \rightarrow \C$ by 
	$$F(y) = \int_X f^y \dmu $$ 
	\begin{enumerate}
	\item Suppose that there exists $g \in L^1(\mu)$ such that for each $(x, y) \in X \times A$, $\|f(x,y)\| \leq g(x)$. Let $y_0 \in A$. If for each $x \in X$, $f_x$ is continuous at $y_0$, then $F$ is continuous at $y_0$. 
 	\item Suppose that for each $x \in X$, $f_x:A \rightarrow Z$ is Gateaux differentiable and there exists $g \in L^1(\mu)$ such that for each $(x, y) \in X \times A$,$h \in Y$,  $| df_x(y)(h) | \leq g(x)$. Then $F$ is Gateaux differentiable and for each $y \in A$, $h \in Y$, $$dF(y)(h) = \int_X df_x(y)(h) \dmu(x)$$
	\end{enumerate}
	\end{ex}
	
	\begin{proof}\
	\begin{enumerate}
	\item Suppose that for each $x \in X$, $f_x$ is continuous at $y_0$. Let $(y_n) \subset A$. Suppose that $y_n \rightarrow y_0$. Continuity implies that $f^{y_n} \convt{p.w.} f^{y_0}$. Since for each $n \in \N$, $|f^{y_n}| \leq g$, the dominated convergence theorem implies that $F(y_n) \rightarrow F(y_0)$.
	\item Let $y_0 \in \R$. Choose $(y_n)_{n \in \N}$ such that $ y_n \rightarrow y_0$ and for each $n \in \N$, $y_n \neq y_0$. For $n \in \N$, define $q_n:X \rightarrow \R$ by 
	$$q_n(x) = \frac{f(x,t_n) - f(x, t_0)}{t_n - t_0}$$ So $h_n(\cdot) \convt{p.w.} \p f / \p t (\cdot, t_0)$. The mean value theorem implies that for each $x \in X$ and $n \in \N$, there exists $c_{n,x} \in (t_n,t_0)$ such that $h_n(x) = \p f / \p t (x, c_{n,x})$. Then for each $n \in \N$, $|h_n| \leq g$. The dominated convergence theorem then implies that $\p f / \p t (\cdot, t_0) \in L^1(\mu)$ and 
	\begin{align*}
	\int \frac{\p f }{\p t} (x, t_0) d\mu(x) 
	&=  \limn \int_X h_n d\mu  \\
	&= \limn \frac{F(t_n) - F(t_0)}{t_n - t_0} \\
	&= F'(t_0^-) 	
	\end{align*}
	So that $F$ is differentiable at $t_0$ from the left. Similarly, $F$ is differentiable at $t_0$ from the right. 
	\end{enumerate}
	\textbf{FINISH!!!}
	\end{proof}





















	




















	
	
	
	
	
	
	
	
	\newpage
	\section{Banach Space Valued Measures}
	

























	\newpage
	\section{TODO}
	\begin{itemize}
		\item Add background for banach space valued measures like riesz representation theorem and radon-nikodym derivatives to be be able to talk about condition expectation of banach space valued random variables
		\item Discuss disintegration of measures independently of probability by discussing the projection of $L^1(X, \MA)$ onto $L^1(X, \MB)$ for $\MB \subset \MA$ and the Doob-Dynkin Lemma. Use this to define the disitegration measure. Also do this for disintegration of vector measures.
		\item Talk about homology when conditioning measures on a value in relation to the entropy of that distribution (maybe make a new set of notes about entropy and put it there)
		\item Consider the category $\MC$ of measurable spaces with measurable singletons. Fix an object $(X, \MA) \in \MC$. Consider the coslice category of $\MC$ under $(X, \MA)$. Introduce an equivalence relation on objects in the coslice category by $f:X \rightarrow (Y, \MF) \sim g: X \rightarrow (Z, \MG)$ iff $f^*\MF = g^*\MG$. Introduce a partial order on the quotient by $f:X \rightarrow (Y, \MF) \leq g: X \rightarrow (Z, \MG)$ iff $f^*\MF \subset g^*\MG$.  Describe the Doob-Dynkin Lemma in this context, i.e. that $f \leq g$ implies that there is exactly one morphism from $g$ to $f$ in the coslice category.
	\end{itemize}






	
	
	
	
	
	
	
	
	
	
	
	
	
	
	
	
	
	\newpage
	\subsection{Applications to Hilbert Spaces}	
	
	\begin{ex} \lex{00000} 
	Let $(X, \MA, \mu)$ be a measure space, $H$ a separable Hilbert space, $f \in L^1_H$ and $a \in H$. Then $$\int \l f(x), a \r d\mu(x) = \bigg \l \int f(x) d \mu(x) , a\bigg \r$$ 
	\end{ex}	
	
	\begin{proof}
	Define $T \in L^*(H, \C)$ by $T(x) = \l x, a \r$ and apply a previous exercise.
	\end{proof}
	
	
	
	
	
	
	
	
	
	
	
	
	
	
	
	
	
	
	
	
	
	
	
	
	
	
	
	

	
	

	\newpage
	\section{Appendix}
	
	\subsection{Summation}
	
	\begin{defn} \ld{00000} 
		Let $f:X \rightarrow \Rg$, Then we define $$\sum_{x \in X} f(x) := \sup_{\substack{F \subset X \\ F \text{ finite}}} \sum_{x \in F} f(x)$$ This definition coincides with the usual notion of summation when $X$ is countable. For $f:X \rightarrow \C$, we can write $f = g +ih$ where $g,h:X \rightarrow \R$. If $$\sum_{x \in X}|f(x)| < \infty,$$ then the same is true for $g^+,g^-,h^+,h^-$. In this case, we may define $$\sum_{x \in X} f(x)$$ in the obvious way.
	\end{defn} 
	
	The following note justifies the notation $\sum_{x \in X}f(x)$ where $f:X \rightarrow \C$.
	
	\begin{note}
		Let $f:X \rightarrow \C$ and $\al:X \rightarrow X$ a bijection. If $\sum_{x \in X}|f(x)|< \infty$, then $\sum_{x \in X}f( \al (x)) = \sum_{x \in X}f(x) $.
	\end{note}
	
	\begin{thebibliography}{4}
	
\bibitem{algebra} \href{https://github.com/carsonaj/Mathematics/blob/master/Introduction\%20to\%20Algebra/Introduction\%20to\%20Algebra.pdf}{Introduction to Algebra}

\bibitem{analysis}  \href{https://github.com/carsonaj/Mathematics/blob/master/Introduction\%20to\%20Analysis/Introduction\%20to\%20Analysis.pdf}{Introduction to Analysis}	

\bibitem{foranal}  \href{https://github.com/carsonaj/Mathematics/blob/master/Introduction\%20to\%20Fourier\%20Analysis/Introduction\%20to\%20Fourier\%20Analysis.pdf}{Introduction to Fourier Analysis}

\bibitem{measure}  \href{https://github.com/carsonaj/Mathematics/blob/master/Introduction\%20to\%20Measure\%20and\%20Integration/Introduction\%20to\%20Measure\%20and\%20Integration.pdf}{Introduction to Measure and Integration}

\end{thebibliography}
	
\end{document}


